%%%%%%%%%%%%%%%%%%%%%%%%%%%%%%%%%%%%%%%%%%%%%%%%%%%%%%%%%%%%%%%%%
%%%%
%%%% The (original) Douay Rheims Bible 
%%%%
%%%% Old Testament
%%%% Three Kings
%%%% fourth age
%%%%
%%%%%%%%%%%%%%%%%%%%%%%%%%%%%%%%%%%%%%%%%%%%%%%%%%%%%%%%%%%%%%%%%




\startcomponent fourth-age


\project douay-rheims


%%% 0721
%%% o-0648
\startArgument[
  title={\Sc{The Continvance of the Chvrch and Religion in the Fovrth
  Age}: From the parting of Iſrael out of Ægypt, to the fundation of the
  Temple. The space of 480.~yeares.},
  marking={The Fourth Age}
  ]

%%% !!! Most, but not all, XRefs are in parentheses. Regularize this?
VVe haue
\Fix{ſenne}{ſeen}{likely typo, fixed in other}
already in the three firſt ages, or diſtinct times of the
world, the
\Fix{biginning,}{beginning,}{likely typo, fixed in other}
increaſe, and continuance of the Church and
Religion of God, without interruption.
\MNote{Articles of faith, other pointes of religion, & ſtate of the
Church more expreſſed in this fourth age then before.}
Now in this fourth age, in which God gaue his people a written Law, it
is yet more euident, that the ſame faith and religion, not only
continued but alſo was more expreſſed, and explicated; and the Church
had more varietie of Sacrifices, Sacraments, and other holie Rites, &
Obſeruances: & the two ſtates Eccleſiaſtical and Temporal more
diſtinguished, and ech of them, eſpecially the Prieſtlie and Leuitical
Hierarchie, more diſpoſed in ſubordination: the ciuil gouernment alſo
vnder Dukes, Iudges, and Kinges, more diſtributed among ſuperiour and
inferiour officers then before.

For
\MNote{Beleefe in one God.}
firſt the principal point and ground of al religion, \Emph{the beleefe
in one God}, and his proper diuine worship, is aboue al moſt ſtrictly
commanded, often repeated, diligently obſerued by the good, and ſeuerely
punished in
%%% 0722
tranſgreſsours. To which end and purpoſe, after that God had ſingularly
ſelected three more renowmed Patriarches, \Emph{Abraham, Iſaac, and
Iacob}, preſeruing them by his ſpecial grace from idolatrie, and from
wicked wayes of moſt peoples and nations, bleſſed their ſeede, not in the
whole progenie of the two former, but in Iacob onlie, whom he otherwiſe
named \Emph{Iſrael}, multiplying his children excedingly, yea moſt of al
(which was moſt maruelous) in hotte perſecution: then bringing them
forth of the fornace of Ægypt, in his ſtrong hand, as is recorded in the
former age,
\MNote{Diuine lawes.}
at laſt his Diuine Maieſtie deliuered to them his perfect and eternal
Law, conteyned in two tables, diſtributed into tenne preceptes,
\MNote{Moral.}
teaching them their proper duties firſt towards himſelfe their God and
Lord, then towards ech other. Adding moreouer for the practiſe and
execution therof, other particular precepts of two ſortes, to witte,
\MNote{Ceremonial.}
\Emph{Ceremonial} preſcribing certaine determinate maners and rites, in
obſeruing the commandements of the firſt table pertaining to God: and
\MNote{Iudicial.}
\Emph{Iudicial} lawes directing in particular how to fulfil the
commandements of the ſecond table, concerning our duties towards our
neighbours. So we ſee the whole law is nothing els, but to
\CNote{\XRef{Mat.~22.}}
\Emph{loue God aboue al, and our neighboures as our ſelues}. The
maner of performing al, is \Emph{to beleue and hope in one onlie Lord
God, honour and ſerue him alone}, who made al
%%% o-0649
of nothing, conſerueth al, wil iudge al, and render to al men as they
deſerue, and therfore fully to confirme this point, he beginneth his
commandements with expreſse prohibition of al falſe and imaginarie
goddes, ſaying:
\XRef{(Exod.~20. v.~3.)}
\MNote{Onely God to be ſerued with diuine honour.}
\Emph{Thou shalt not haue ſtrange goddes}, & after threates to the
tranſgreſſours, and recital of the other nine commandementes, he
concludeth
\XRef{(v.~23.)}
with repetition of the firſt, ſaying: \Emph{You shal not make goddes of
ſiluer, nor goddes of gold shal you make to you.} The ſame is repeted
and explaned
\XRef{(Deut.~5.)}
And in the next chapter Moyſes exhorting the people ſaith: \Emph{Heare
Iſrael, the Lord our God is one Lord.} And God himſelfe ſpeaking againe
ſayth:
\XRef{(Exod.~23.)}
\Emph{See ye that I am onlie, and there is no other God beſides me.} The
royal prophet Dauid
\XRef{(2.~Reg.~22.}
and
\XRef{Pſalm.~17.)}
\Emph{who is God but our God?} and in ſundrie other places the ſame
doctrine of one God is grounded, confirmed, and eſtablished.

The
\MNote{The B.~Trinitie.}
Myſterie of \Emph{the B.~Trinitie}, or of three Diuine Perſons, is no
leſse true and certaine, then that there is but one God, though not ſo
manifeſt to reaſon, nor ſo expreſsly taught in the old Teſtament, yet
beleued then alſo, and often inſinuated, where God is expreſsed by names
of the plural number: as \HH{Elohim, Elim, Elohe, Saddai, Adonai,
Iſebaoth}: which import pluralitie of Perſons in God, who is but one
nature and ſubſtance. \Emph{Diſtinction} alſo \Emph{of Perſons} in God
is deduced
\XRef{(Exod.~33.)}
God ſaying: \Emph{I wil cal in the name of the Lord.} That is (as
\CNote{\Cite{q.~154. in Exod.}}
S.~Auguſtin and other fathers expound it) the ſecond Perſon by his grace
maketh his ſeruants to cal vpon God. More diſtinctly
%%% 0723
\XRef{(Pſalm.~2.)}
\Emph{The Lord ſaid to me: Thou art my Sonne, I this day haue begotten
thee.}
\XRef{(Pſalm.~109.)}
\Emph{The Lord ſaid to my Lord}: that is, \Emph{God the Father to God
the Sonne}: who according to his diuinitie is the Lord of Dauid,
\CNote{\XRef{Mat.~22.}}
according to his humanitie the ſonne of Dauid. The ſame king Dauid
maketh mention alſo of the third Perſon, \Emph{the Holie Ghoſt}, praying
\XRef{(Pſalm.~50.)}
\Emph{Thy holie Spirit take not from me.} In the forme of bleſsing the
people
\XRef{(Num.~6.)}
al three Perſons ſeme to be vnderſtood in the name of our Lord thriſe
repeted; \Emph{our Lord} (the Father) \Emph{bleſſe thee and keepe
thee. Our Lord} (the Sonne) \Emph{ſhew his face to thee, and haue mercie
vpon thee. Our Lord} (the Holie Ghoſt) \Emph{turne his countenance vnto
thee, and geue thee peace.}

Of
\MNote{Incarnation of Chriſt.}
the \Emph{Incarnation} of the Sonne of God, we haue in this age manie
prophecies and figures. Moyſes euidently
\XRef{(Deut.~18.)}
forsheweth that after other prophets \Emph{Chriſt the Sonne of God}
should come in flesh, and redeme mankind, as S.~Peter teacheth.
\XRef{(Act.~3.)}
Likewiſe in his Canticle, and Bleſsing of the tribes
\XRef{(Deut.~32.}
&
\XRef{33.)}
he ſpeaketh more expreſly of \Emph{Chriſt} and his Church, then of the
Iewes and
\Fix{thier}{their}{likely typo, fixed in other}
Synagogue. The ſtarre prophecied by Balaam
\XRef{(Num.~24.)}
forshewed both to Iewes and Gentiles, that Chriſt should ſubdue al
nations. Ioſue both in name and office was a manifeſt figure of
\Sc{Iesvs} Chriſt. Alſo the Iudges, and Kinges, ſome in one thing, ſome
in an other, moſt eſpecially king Dauid and king Salomon, were figures
of our Lord and Sauiour \Sc{Iesvs} Chriſt. The braſen ſerpent
\XRef{(Num.~21.)}
ſignified \Emph{Chriſt} to be crucified, as him ſelfe expoundeth it.
\XRef{(Ioan.~3.)}
Briefly the whole Law was a pedagogue, or conductor to bring men
%%% o-0650
to Chriſt
\XRef{(Galat.~3.)}
and by him to know God and them ſelues: to wit, God
omnipotent, al perfect, Creator of al, our Father, Redemer, and
Sanctifier: and man his chief earthlie creature; though of himſelfe
weake and impotent, yea through ſinne miſerable, yet in nature of free
condition, indued with \Emph{vnderſtanding}, to conceiue, and diſcourſe;
and with \Emph{freewil}, to chooſe or refuſe what liketh or diſpleaſeth
him.

For
\MNote{Freewil in Angels and men.}
God appointing al creatures their offices, ingraſſed in al other thinges
inuariable inclination to performe the ſame, ſo that they could neither
by vertue nor ſinne make their ſtate better nor worſe then it was
created, but ordaining Angels and men to a higher end of eternal
felicitie, left their wils free to agree vnto, or to reſiſt his
precepts, and counſels. VVherupon Angels cooperating with Gods grace
were confirmed in glorie, and ſome reuolting were eternally damned. Man
alſo offending fel into damnable ſtate, but through penance may be
ſaued, if he cooperate with new grace of our Redemer, which is in his
choiſe to doe, or omitte. As when God gaue his people meate in the
deſert
\XRef{(Exod.~16.)}
he ſo inſtructed them, how to receiue it and vſe it, without force or
compulſion, that he might \Emph{proue them} (as himſelf ſpeaketh)
%%% 0724
\Emph{whether they would walke in his law or no}. And after making
couenant with them
\XRef{(Exod.~19.}
\XRef{Deut.~26.)}
required and accepted their voluntarie conſent: entring into formal
contract or bargaine betwen him ſelf and them: he promiſing on the one
partie \Emph{to make them his peculiar people, a prieſtlie kingdome, and
a holie nation}: they on the other partie promiſing loyaltie, obedience,
and obſeruation of his commandements, ſaying: \Emph{Al thinges that our
Lord hath ſpoken we wil doe.} For which cauſe Gods promiſes are
conditional
\XRef{(Deut.~7.)}
\Emph{if thou kepe his iudgements, God wil keepe his couenant to thee.}
Againe moſt plainly
\XRef{(Deut.~11.)}
\Emph{Behold I ſette before your ſight this day benediction and
malediction}, and
\XRef{(Deut.~30.)}
\Emph{I cal for witneſſes this day heauen and earth, that I haue
propoſed to you life and death, bleſſing and curſing. Chooſe therfore
life that thou mayeſt liue.} In al which it is certaine that Gods
promiſe being firme, mans wil is variable, and ſo the euent not
neceſſarie: which made Caleb hoping of victorie to ſay:
\XRef{(Ioſue.~14.)}
\Emph{If perhaps our Lord be with me.}
\MNote{Obiection of Gods fornovvledge anſwered.}
Neither doth Gods foreknowledge make the euent neceſſarie, for he ſeeth
the effect in the cauſe, as it is voluntarie or caſual: yea God knoweth
al before, and ſome times fortelleth thinges, vvhich conditionally
vvould happen, and in deed, (the condition fayling) come not to paſſe,
as
\XRef{(1.~Reg.~23.)}
God anſwered, \Emph{that the men of Ceila would betray Dauid} (meaning
if he ſtaied there) vvhich they did not; for he parted from thence.

Yet
\MNote{Grace neceſſarie.}
is not man able by this his freedome, nor otherwiſe of himſelf, to do,
nor ſo much as to thinke anie good thing but through Gods mere mercie,
and
\CNote{\XRef{2.~Cor.~3.}}
\Emph{grace, geuen him without} his \Emph{deſeruing}, ſufficient to al,
and effectual to thoſe that accept it. God alſo geueth particular grace
for ſpecial functions; as
\XRef{(Leuit.~8.)}
to Prieſtes
\XRef{(Num.~11.)}
to ſeuentie ancients, and
\XRef{(1.~Reg.~10.)}
to king Saul.

%%% o-0651
By
\MNote{Gods cõmandmentes poſſible to be kept.}
vvhich diuine aſsistance the \Emph{commandements of God are poſſible},
as himſelfe auoucheth, ſaying:
\XRef{(Deut.~30.)}
\Emph{This commandment that I command thee this day is not aboue thee.}
Againe: \Emph{I haue ſette before thee life and good, death and euil,
that thou mayeſt loue God, walke in his wayes, and keepe his
commandementes.}

Workes
%%% !!! Two separate MNotes in original?
\MNote{Good workes meritorious.}
done by grace and freevvil are good and commendable, Moyſes ſo
teſtifying:
\XRef{(Deut.~14.)}
\Emph{This is your wiſdome and vnderſtanding before peoples.} Yea are
\Emph{meritorious}: and revvardes are promiſed for the ſame
\XRef{(Leuit.~16.)}
and contrariwiſe punishments threatned to the tranſgreſſours. And Booz
knowing revvard to be due for vvel doing, prayed God to render to Ruth
\XRef{(ch.~2.)}
a ful reward for her wel deſeruing. The royal prophet affirmeth
\XRef{(Pſalm.~18.)}
that \Emph{in keeping Gods preceptes is much reward}, and
\XRef{(Pſal.~118.)}
profeſseth that he inclined his hart to keepe them \Emph{for reward}.

Amongſt
\MNote{Diuers ſortes of Sacrifices.}
other ſeruices of God, and meanes of mens ſaluation, \Emph{external
%%% 0725
Sacrifice} is of the greateſt. And therfore the maner of offering al
ſortes is at large preſcribed in the Law, eſpecially in the
\XRef{ſeuen firſt Chapters of \Emph{Leuiticus}.}
\MNote{Holocauſt.}
The firſt and principal was \Emph{Holocauſt}, wherin al the oblation was
burned and conſumed in the honour of God our Soueraigne Lord.
\MNote{For ſinne.}
The ſecond was \Emph{Sacrifice for ſinne}, according to the diuerſitie
of offices, and perſones, wherof part was burned, the other part
remained to the prieſtes, except it were for the ſinnes of prieſtes, or
of the whole multitude
\XRef{(Leuit.~4.)}
for then the prieſtes had no portion, but al was offered to God.
\MNote{Pacifique.}
The third was \Emph{pacifique ſacrifice}, either \Emph{of thankſgeuing}
for benefites receiued, \Emph{or to obtaine Gods fauour} in al occurrent
neceſsities, and good deſires. And of both theſe ſortes one part was
conſumed in Gods honour, an other part was the prieſtes, the third was
theirs that gaue the oblation. In confirmation of theſe ſacrifices God
at firſt miraculouſly ſent fire to burne them
\XRef{(Leuit.~9.)}
wherof he had geuen commandment before
\XRef{(Leuit.~6.)}
that
\MNote{Fire ſent from God ſignifieth charitie.}
it should be conſerued, and neuer extinguished, to teach vs eſpecially
of the new Teſtament, that haue the real Sacrifice, and verie bodie of
the former shadowes and figures, to nourish and keepe the fire of
charitie, not procured by our owne power, but geuen by God, that it
neuer ceaſe, nor be extinguished in our hartes.

Likewiſe
\MNote{Sacraments.}
in the ſame law of Moyſes, beſides \Emph{Circumciſion} inſtituted before
\XRef{(Gen.~17.)}
and here confirmed and continued
\XRef{(Leuit.~12.}
\XRef{Ioſue.~5.)}
al \Emph{hoſtes} and \Emph{ſacrifices for ſinne}
\XRef{(Leuit.~4. 5. 6. and~7.)}
\Emph{conſecration of Prieſtes},
\XRef{(Leuit.~8.)}
and the \Emph{ſacrifices adioyned} therunto, alſo diuers other
\Emph{waſhinges} and \Emph{purifications} of legal vncleannes
\XRef{(Leuit.~14. 15. 16. and~17.)}
\CNote{\Cite{Alanus de Sacra. c.~9.}}
were al \Emph{Sacraments}; ſignifying either firſt iuſtification and
remiſsion of ſinne, or increaſe of grace, and puritie; of which ſort it
is alſo probable that the \Emph{Paſchal lambe}, and \Emph{Loaues of
propoſition} were ſacramentes.
\XRef{(Exod.~12.}
\XRef{25.)}
VVhich multitude S.~Auguſtin comparing with ours of the new Teſtament,
ſayth:
\CNote{li. de vera Religion c.~17.}
\MNote{Manie more in the old Teſtament then in the new.}
\Emph{The people bound with feare in the old law, was burdened with
manie ſacraments. For this was profitable to ſuch men} (ſaith
he) \Emph{to make them deſire the grace, foretold by the prophetes, which
%%% o-0652
being come from the wiſdome of God becoming Man, by whom we are called
into freedom, a few moſt wholſome Sacraments are inſtituted, which hold
the ſocietie of chriſtian people vnder one God of a free multitude.}
\MNote{Chriſts Sacraments more excellent.}
But as Chriſtes Sacraments are fewer in number, ſo they are more
excellent in vertue.
\MNote{Moſt of Chriſts Sacraments prefigured in the old law but not al.}
And to moſt of theſe new, the former do anſwere as
figures and shadowes. So to our \Emph{Baptiſme} anſwereth Circumciſion,
as S.~Paul teacheth
\XRef{(Coloſſ.~1.)}
\Emph{that Chriſtians are circumciſed in the circumciſion of Chriſt,
buried with him in Baptiſme}. To our holie \Emph{Euchariſt}, as it is a
Sacrament, did anſwere the Paſchal lambe, & Loaues of propoſition, as
alſo Manna, and bloud of the Teſtament. It was prophecied
\XRef{Pſal.~18.}
\Emph{Adore his
%%% 0726
foote ſtoole}: as
\CNote{\Cite{S.~Aug. in hunc Pſal.}
\Cite{ſer. de verbis Domini.}
\Cite{li.~17. ciuit. c.~20.}
\Cite{li.~1. cont. aduerſ. leg. c.~18.}
\Cite{S.~Cyril. li.~3. in Ioan.}
\Cite{S.~Leo ſer.~8. de paſſione.}}
holie Fathers expound it. And as the ſame \Emph{Euchariſt} is
a \Emph{Sacrifice}, it was prefigured by al the old Sacrifices of the
law of nature, and of Moyſes: as S.~Auguſtin, and S.~Leo do proue; and
prophecied
\XRef{(Pſal.~19.)}
\Emph{Be he mindful of al thy ſacrifice, &c.} To the ſacrament of
\Emph{holie Orders} anſwered conſecration of Prieſts. Al the ablutions,
purifications, cleanſinges, and oblations for ſinne, which in great part
were both Sacramentes and Sacrifices, anſwered to our Sacrament of
\Emph{Penance}, which was alſo prefigured by the ſecond tables of the
decalogue.
\XRef{(Exod.~34.)}
More plainly forshewed by example of particular confeſsion of ſinnes
and ſatisfaction
\XRef{(Num.~5.}
\XRef{14.}
and
\XRef{29.)}
Contrition alſo was no leſse required, as appeareth by the example of
king Dauid.
\XRef{2.~Reg.~24.}
Mariage in the old Teſtament, though not a ſacrament yet ſignified the
Sacrament of \Emph{Mariage} among Chriſtians. But the Sacrament of
\Emph{Confirmation} had not anie ſo anſwerable a figure, in the old law,
which brought not to perfection. Neither \Emph{Extreme vnction}, becauſe
the law gaue not immediate entrance into the kingdom of heauen, which
defectes were ſignified by the high prieſtes entring only once in the
yeare into \L{Sancta Sanctorum}.
\XRef{Leuit.~16.}

Likewiſe
\MNote{Some like impediments in vſe of holie Rites.}
touching practiſe of holie Rites; diuers vncleannes hindering
participation of ſacrifices, and conuerſation with other men.
\XRef{(Leuit.~14.)}
Degrees of conſanguinitie and affinitie, hindering mariage
\XRef{(Leuit.~18.)}
and ſundrie Irregularities excluding from the office of Prieſts
\XRef{(Leuit.~21.)}
were \Emph{figuratiue reſemblances of ſinnes and cenſures}, and of
\Emph{impediments to holie Orders}, and \Emph{to Mariage}, in the new
Teſtament.

To
%%% !!! Two separate MNotes in original?
\MNote{Tabernacle. Propitiatorie with appertinances.}
the peculiar ſeruice of God perteyned alſo \Emph{the Tabernacle}, with
the \Emph{Propitiatorie, Arke, Cherubims, Table} for loaues of
propoſition, \Emph{Candleſticke, Lampes, Altares} for Holocauſtes, &
Incenſe, \Emph{Veſtments for Prieſtes, a braſen lauer}, and other
veſsels deſcribed
\XRef{Exod.~25. et~ſeq.}
Al which were kept and carried by the Leuites, reſting or marching in
the middes of the campe.
\XRef{Num.~2.~3.}
And when the Land of Chanaan was conquered, the ſame were fixed in Silo,
\XRef{Ioſue.~18.}
whither the people reſorted at certaine ſette times, and vpon ſundrie
occaſions. From thence long after they tooke the Arke, and often vpon
diuers occaſions remouing it, made Oratories, or Chappels, whereſoeuer
it reſted, deuotion increaſing, & religious eſtimation of it in al
Iſrael.
\XRef{1.~Reg.~4.}
\XRef{7.}
\XRef{10.}
Yea the infidel Philiſthims in Azotus ſeing and feeling the vertue
therof, ouerthrovving their god Dagon, and them ſelues ſore plagued
found it beſt for them to ſend the Arke home to the
%%% o-0653
Iſraelites, not vvithout coſtlie and pretious oblations.
\XRef{(1.~Reg.~5. &~6.)}
King Dauid moſt ſpecially honoring it.
\XRef{(2.~Reg.~6.)}
VVho further conſidering that himſelf dvvelt in a houſe of cedar, and the
Arke of God remained in the tabernacle couered vvith skinnes, intended
to build a more excellent houſe for God.
\XRef{2.~Reg.~7.}
But his godlie purpoſe vvas differed by Gods appointment and
%%% 0727
his ſonne king Salomon builded the famous Temple in Hieruſalem.
\XRef{3.~Reg.~6.}

VVhich
\MNote{The Tabernacle, and afterwardes the Temple, the onlie place for
Sacrifice.}
ſucceding in place of the Tabernacle, ech of them (one after the other)
was the only ordinarie place of Sacrifice. The law commanding
\XRef{(Leuit.~17.)}
\Emph{If anie man of the houſe of Iſrael, kil an oxe, or a sheepe, or a
goate} (to wit, for Sacrifice, as
\CNote{\Cite{Queſt.~56. in Leuit.}}
S.~Auguſtin, and other fathers expound it) \Emph{and offer it not at the
dore of the tabernacle}, (afterwards at the dore of the Temple) \Emph{he
shal be guiltie of bloud, as if he had shed bloud}, and \Emph{ſo shal he
perish out of the middes of his people.}
\MNote{Yet God ſome times diſpenſed therein.}
Neuertheles vpon occaſions, and by ſpecial reuelation ſacrifice was
lawfully offered in other places. For ſo in the time of the tabernacle,
\Emph{Samuel} the prophet, \Emph{offered Sacrifice in Maſphath.}
\XRef{1.~Reg.~7.}
And the prophet \Emph{Elias offered Sacrifice without the Temple}, vvhen
he conuinced the falſe prophetes of Baal,
\XRef{3.~Reg.~18.}
whoſe fact (as
\CNote{\Cite{ibidem.}}
S.~Auguſtin noteth) the miracle ſufficiently shewed to be donne by Gods
diſpenſation.

And
\MNote{Feaſtes of the old law.}
as peculiar places were dedicated, ſo alſo ſpecial times were
ſanctified, and diuers feaſtes, and feſtiuities partly ordained before
(as the Sabbath
\XRef{Gen.~2.}
and Paſch
\XRef{Exod.~12.)}
were confirmed by the Law
\XRef{(Exod.~20.}
\XRef{23.)}
and others likevviſe inſtituted
\XRef{(Exod.~23.}
\XRef{Leuit.~23.}
\XRef{Num.~28.}
\XRef{29.}
and
\XRef{Deut.~16.)}
with proper ſacrifices for euerie ſort. Firſt and moſt general was the
\Emph{dailie ſacrifice} of a lambe euerie day twiſe, at morning and
euening
\XRef{(Exod.~29.)}
which was not properly a feaſt, but a ſacred perpetual office in the
tabernacle, and after in the temple.
\MNote{Eight ſortes of feaſtes, beſides the dailie ſacrifice.}
Al the reſt were feſtiual dayes, in which it was not lawful to do
ſeruile worke. The firſt of theſe was
\Emph{the Sabbath}, that is the ſeuenth and laſt day of euerie weke,
which is our ſaturday: Kept ſtil ſolemnly by the Iewes, euen at this
time, in al places vvhere they dvvel; but not by Chriſtians, becauſe the
old Lavv is abrogated; and vve kepe the next day, which is \Emph{Sunday,
holie, by inſtitution and tradition of the Church}. The ſecond,
\Emph{Neomenia}, or new moone, in which day they alwaies beganne the
moneth; and twelue ſuch monethes made a yeare, by the courſe of the
moone; for by the courſe of the ſunne, the yeare conteineth eleuen dayes
more, which in three yeares make aboue a moneth. And ſo euerie third
yeare, and ſometimes the ſecond (for it happened ſeuen times in nintene
yeares) had thirtene monethes: and was called
\CNote{\Cite{S.~Beda de emboliſmo. to.~1.}}
\L{Annus embolismalis}, being increaſed by meanes of thoſe eleuen
dayes. The third feaſt was \Emph{Paſch}, or Phaſe, firſt inſtituted at
the parting of the children of Iſrael out of Ægypt, in the ful moone of
the firſt moneth in the ſpring, in which the Paſchal lambe was eaten, as
is preſcribed.
\XRef{Exod.~12.}
The fourth feaſt was \Emph{Pentecoſt}, or firſt fruites, the fiftith day
after Paſch, when Moyſes receiued the Lavv in mount Synai. The fifth,
\Emph{the feaſt of Trumpets}, the firſt day of the ſeuenth moneth, in
gratful memorie that a ramme ſticking by the hornes, vvas offered in
ſacrifice by Abraham in place of Iſaac. The ſixth vvas \Emph{the feaſt
of Expiation}, the tenth
%%% 0728
%%% o-0654
day of the ſeuenth moneth;
\MNote{Preſcribed faſt from euen to euen.}
vvherein \Emph{ſolemne faſt} vvas alſo preſcribed from euening of the
ninth day to euening of the tenth, for remiſsion of ſinnes in general,
beſides particular ſacrifices and ſatisfaction for euerie ſinne, wherof
anie man found himſelf guiltie. The ſeuenth vvas \Emph{the feaſt of
Tabernacles}, ſeuen dayes together, beginning the fiftenth of the
ſeuenth moneth, in memorie of Gods ſpecial protection, vvhen they
remained in tabernacles, fourtie yeares in the deſert. The eight feaſt
vvas of \Emph{Aſſemblie and Collection}, the next day after the forſaid
ſeuen, in commemoration of vnion in the people, and peaceable poſſeſsion
in the promiſed land. In this day general collection vvas made 
for neceſsarie expences in the publique ſeruice of God.

Moreouer
\MNote{Seuenth yeare of reſt: and Iubiley yeare.}
\Emph{the ſeuenth yeare} vvas a Sabbath of reſt
\XRef{(Leuit.~25.)}
in vvhich no land vvas plowed, no vines pruned, nor thoſe fruites
gathered that ſprong vvithout mans induſtrie of the earth. Againe the
fiftith yeare vvas peculiarly made holie, and called \Emph{the Iubiley},
or ioyful yeare. In it al bondmen vvere ſette free; al inheritances
amongſt the Iſraelites, being for the time, ſold or otherwiſe alienated,
returned to the former ovvners.

Beſides
\MNote{Other ceremonial obſeruances.}
Sacrifices, Sacramentes, holie places, holie times, and manie other
ſacred things belonging therto; there were yet more \Emph{ceremonial
Obſeruances} commanded by Moyſes law, as vvel perteyning to the ſeruice
of God in that time, as ſignifying chriſtian life and maners.
\MNote{Cleane and vncleane.}
So certaine beaſtes, birdes, and fishes were reputed vncleane
\XRef{(Leuit.~11.)}
and Gods people forbid to eate them;
\MNote{No bloud to be eaten, nor fatte.}
as alſo that they should not eate anie bloud at al, nor fatte.
\XRef{Leuit.~3.}
The reaſon of al which vvas not, as though anie creature were il in
nature, but partly to auoide idolatrie, partly to exerciſe them in
obedience, and temperance; & partly for that the ſame thinges ſignified
vices and corruptions, from which Chriſtians eſpecially ought to
refrain.
\MNote{Not diuers ſeede in one field.}
Likewiſe
\XRef{Leuit.~19.}
they were commanded not to ſovv their fieldes vvith tvvo ſortes of
ſeede;
\MNote{No cloth of diuers matter.}
nor to vveare garmentes wouen of tvvo ſortes of ſtuffe, that they might
be more diſtinguished from infidels by external ſignes, and not only by
circumciſion, but eſpecially to teach chriſtians to practiſe ſimple
innocencie, & to auoid duble & deceptful dealing.

Al
\MNote{Strict commandment to kepe al the Law.}
vvhich, and other \Emph{preceptes} as wel moral, as ceremonial and
iudicial, vvere moſt \Emph{ſtrictly cõmanded};
\MNote{The obſeruers bleſſed and rewarded.}
the obſeruers bleſsed & \Emph{rewarded}, &
\MNote{Tranſgreſſours curſed and puniſhed.}
tranſgreſsours ſeuerly \Emph{threatned} vvith great curſes
\XRef{(Leuit.~20.}
\XRef{26.}
\XRef{Deut.~4.}
\XRef{27.}
\XRef{28.)}
and diuers actually \Emph{puniſhed},
\XRef{Exod.~32.}
three thouſand ſlaine for committing idolatrie. Manie ſwallovved vp in
the earth,
\XRef{(Num.~16.)}
deſcending quicke into hel, & manie more burned vvith fire from heauen,
for making and fauoring Schiſme. Yea by one meanes & other, al that
vvere aboue twentie yeare of age, coming forth of Ægypt, except tvvo
onlie (Ioſue & Caleb) died in the deſert, for the general murmur of the
people.
\XRef{Num.~11.}
\XRef{14.}
\XRef{25.}
&
\XRef{26.}
Al Iſrael beaten in battle til one malefactor Achan was diſcouered &
punished.
\XRef{Ioſ.~7.}
Al the tribes were punished for ſuffering publique idolatrie in Dan: and
Beniamin
%%% 0729
almoſt extirpate, for not punishing certaine malefactours.
\XRef{Iudic.~20.}
And the vvhole people vvere often inuaded & ſore afflicted for their
ſinnes; as appeareth in the booke of Iudges. In particular alſo diuers
were aduanced & proſpered for their virtues,
%%% o-0655
as Ioſue, Caleb, Phinees,
Samuel, Dauid, and others. Contrariwiſe Nadab and Abiu prieſts were
miraculouſly burnt for offering ſtrange fire.
\XRef{Leuit.~10.}
One ſtoned to death for gathering ſtickes on the ſabbath day.
\XRef{Num.~15.}
King Saul depoſed, for preſuming to offer ſacrifice, & not deſtroying
Infidels
\XRef{(1.~Reg.~13.}
\XRef{15.)}
& Oza,
\XRef{2.~Reg.~6.}
ſodenly ſlaine for touching the Arke of God, the Lavv forbidding vnder
paine of death,
\XRef{Num.~1. v.~5.}
&
\XRef{18. v.~7.}
that none should approch to holie office being not therto orderly
called. 

Of
\MNote{VVorkes of ſupererogation.}
workes alſo of \Emph{Supererogation} (called counſailes not preceptes) vve
haue examples in
\MNote{Vowes.}
vovves, voluntarily made of thinges not commanded; the law preſcribing
vvhat vovves might be made, & by vvhom.
\XRef{Nu.~30.}
And
\XRef{Num.~6.}
\Emph{a particular rule} was propoſed to ſuch as of their ovvne accord,
vvould embrace it, & a diſtinct name geuen them, to be called
\MNote{Nazarites.}
\Emph{Nazarites}, that is, \Emph{Seperate or Sanctified}. In which ſtate
they vvere to remain either for a time, limited by themſelues or their
parents, or perpetually, if they ſo promiſed.
\XRef{Iudic.~13.}
\XRef{1.~Reg.~1.}
For ſo farre as their promiſe extended, they were ſtrictly obliged to
performe.
\XRef{Deut.~23.}
\Emph{When thou haſt vowed a vow to our Lord thy God, thou ſhalt not
ſlacke to pay it: becauſe our Lord thy God wil require it: and if thou
delay, it ſhal be reputed to thee for ſinne. If thou wilt not promiſe,
thou shalt be without} (this) \Emph{ſinne. Pay thy vowes vnto the
Higheſt.}
\XRef{Pſal.~75.}
\Emph{Vow ye, and render} (your vowes) \Emph{to our Lord your God.}
\XRef{Pſal.~49.}
The
\MNote{Rechabites.}
\Emph{Rechabites} aftervvardes had a like \Emph{rule} to the Nazarites;
& the ſame perpetual
\XRef{(Hierem.~35.)}
\Emph{neuer to drinke wine, not to build nor dwel in houſes, but in
tabernacles, nor ſow corne, nor plant vineyardes.} VVhich rule though
inſtituted by a man, yet the obſeruation therof was much commended &
rewarded by God.
\XRef{v.~19.}
Such \Emph{diſtinct ſtate of religious perſons, with other ſtates of the
church} of Chriſt, were alſo prefigured
\XRef{(Leuit.~11.)}
\MNote{Three ſortes of Chriſtians prefigured.}
by the cleane fishes, of three diſtinct vvaters, as ſome holie Fathers
do myſtically expound that place. To vvitte, the cleane fishes of the
ſea are the multitude of
\MNote{Laitie.}
\Emph{layperſons}, which are dravven out of the ſea of this vvorld, and
happily found good fishes in our Lords nette.
\XRef{Math.~13.}
The cleane fishes of the riuers, are the good and fruitful
\MNote{Clergie.}
\Emph{Clergie men}, that vvatter the vvhole earth, by teaching Chriſtian
doctrin, and miniſtring holie Sacramentes, vvith other Rites, and
Gouerning the whole Church. And the cleane fishes of ſtanding pooles,
are the
\MNote{Mounkes.}
\Emph{Monaſtical perſons}, liuing perpetually in Cloyſters, vvhere good
ſoules are alwayes readie for our Lordes table, as
\CNote{\Cite{S.~Bern. Ser.~1. de S.~Andrea.}}
S.~Bernard teacheth.
\MNote{Holie ſcripture expounded myſtically.}
Much more the more ancient fathers,
\CNote{\Cite{S.~Beda to.~4.}}
S.~Beda,
\CNote{\Cite{S.~Greg. in li.~1. Reg. et in Iob.}}
S.~Gregorie,
\CNote{\Cite{S.~Aug. cont. Fauſt.}}
S.~Auguſtin,
and others explicate innumerable places of holie Scripture myſtically;
relying therin
%%% 0730
vpon example of the new Teſtament ſo expounding the old. Namely S.~Paul
teaching (as before is noted) that the whole law was a pedagogue guiding
men to Chriſt, and affirming that al thinges happened to the people of
the old Teſtament in figure of the new.

Leauing therfore to proſecute the ſame further, which would require a
verie great worke, it may here ſuffice to geue according to the literal
ſenſe, a briefe view of certaine other pointes of Religion, practiſed in
this fourth age.

%%% o-0656
VVhere
\MNote{Inuocation of Patriarches.}
it is clere, that as Iacob the Patriarch had fortold
\XRef{(Gen.~48.)}
that \Emph{Abrahams, Iſaacs}, and his owne \Emph{name} should be
\Emph{inuocated}, ſo Moyſes prayed God for his promiſe made to them, and
for their ſake, to pardon the people, ſaying:
\XRef{Exod.~32.}
\Emph{Remember ô Lord Abraham, Iſaac, & Iſrael. And our Lord was
pacified, from doing the euil which he had ſpoken againſt his people.}
His diuine prouidence ſo diſpoſing, that he could be hindered, by ſuch
prayers, from that which he threatned.
\CNote{\Cite{S.~Hiero. Ep.~12. ad Gauden.}}
\MNote{Obiections anſwered by holie Scriptures.}
And wheras Moyſes did not directly inuocate the holie Patriarches, as
Chriſtians now cal vpon glorified Sainctes, to pray for them, the cauſe
of difference is, for that now Sainctes ſeing God, know in him,
whatſoeuer perteyneth to their glorie, which ſtate none before Chriſt
attained vnto.
\XRef{Num.~35. v.~25.}
\XRef{Deut.~4. v.~12.}
Againe Proteſtantes obiect, that for ſo much as God knoweth al our
neceſsities, deſires, diſpoſitions, and whatſoeuer is in man, it is
needles (ſay they) ſuperfluous & in vaine, that Sainctes should commend
our cauſes. To this we anſwer, that not only glorious Sainctes, but alſo
mortal men by Gods ordinãce (by which nothing is done vainely) do ſuch
offices, as mediators betwen God and other men, for ſo Moyſes \Emph{told
the wordes of the people to our Lord}
\XRef{(Exod.~19.)}
notwithſtanding \Emph{Gods omniſcience}, or knowledge of al
thinges. Alſo
\CNote{\XRef{Iob.~42.}}
God expreſly commanded Iobs freinds to goe to Iob, promiſing to heare
his prayer for them.
\MNote{How Sainctes know mens prayers.}
As for Sainctes hearing or knowing our prayers made to them, though
onlie God of himſelfe, and by his owne power, ſeeth mens ſecrete
cogitations, and therfore is properly called \Emph{the ſearcher of
hartes}
\XRef{(1.~Reg.~16.)}
yet God communicateth this power to prophetes, to ſee the ſecrete
thoughtes of others; ſo Samuel knew the cogitations of Saul.
\XRef{(1.~Reg.~9. v.~20.)}
And Ahias ſaw by reuelation the coming of Ieroboams wife to him in Silo.
\XRef{(3.~Reg.~14.)}
Much more God reuealeth our preſent ſtate, and actes to \Emph{glorified
ſoules}; vvho are \Emph{as Angels in heauen}
\XRef{(Math.~22.)}
\Emph{and being ſecure of their owne glorie, are careful} (ſayeth
\CNote{\Cite{lib. de mortalitate.}}
S.~Cyprian) \Emph{of our Saluation}.
\MNote{Titles geuen to men in office, and to Sainctes.}
Neither is it derogation to God that Saints are honoured, and titles
aſcribed to them, of interceſsors, mediators, and the like; for ſuch
titles are geuen to them not as to God, but by vvay of participation
only. So \Emph{Iudges} are called \Emph{goddes} and \Emph{ſauiours}
\XRef{(Exod.~21.}
\XRef{Iudic.~3.)}
and \Emph{Prieſtes} called \Emph{goddes}.
\XRef{(Exod.~21.)}
Praiſe geuen \Emph{to God and Gedeon}.
\XRef{Iudic.~7.}
\MNote{Angels adored.}
\Emph{Protection} and \Emph{adoration of Angels} is very frequent.
\XRef{Exod.~23.}
\XRef{31.}
%%% 0731
\XRef{Num.~22.}
\XRef{Ioſue.~5.}
\XRef{Iudic.~2.}
\XRef{6.}
\XRef{13.}
\Emph{The names of the twelue ſonnes of Iſrael} were grauen in the two
chiefe ornaments of the high prieſt, in the \Emph{Ephod and Rationale}.
\XRef{(Exod.~28.)}
\Emph{Manna} was not only reſerued as a memorie of Gods ſingular
benefite, but alſo honorably repoſed as a
\MNote{Reliques.}
\Emph{Relique in a golden veſſel}, and kept in the Arke of God.
\XRef{(Exod.~16.}
\XRef{Heb.~9.)}
\Emph{Ioſephs bones} reſerued and remoued.
\XRef{(Ioſue.~24.)}
\MNote{Images.}
\Emph{Images} of holie \Emph{Cherubims} were made and ſette vp together
with the Arke, and Propitiatorie in the chiefe place of the Tabernacle,
called \L{Sancta Sanctorum}.
\XRef{(Exod.~25.)}
An \Emph{image alſo of a ſerpent} was made in braſse for the health of
thoſe that were ſtriken by ſerpentes.
\XRef{(Num.~21.)}
\Emph{Images} alſo \Emph{of lions and oxen} were made, and ſette vnder
the foote of the lauer (called a ſea) in the Temple.
\XRef{(3.~Reg.~7.)}
The \Emph{honour}
%%% o-0657
done to anie holie thing, namely \Emph{to the Arke}
\XRef{(2.~Reg.~6.)}
redounded to Gods more honour, and al this ſo farre from idolatrie, that
quite contrarie, in preſence of the Arke the idol Dagon fel to the
ground, and broke in peeces.
\XRef{1.~Reg.~5.}

\Emph{Exequies}
\MNote{Exequies for the dead.}
for the dead with \Emph{weeping and faſting} were then practiſed in the
Church, as appeareth by the peoples mourning for Aaron thirtie dayes.
\XRef{Num.~20.}
Alſo for Moyſes.
\XRef{(Deut.~34.)}
By the Gabaonites faſting \Emph{ſeuen dayes for Saul and his ſonnes
lately ſlaine}.
\XRef{1.~Reg.~31.}
Likewiſe king Dauid with al his court \Emph{mourning weping and faſting}
for them.
\XRef{2.~Reg.~1.}
Al which were to no purpoſe, if ſoules departed could not be releiued by
ſuch meanes. It moreouer appeareth that the ſame royal prophet beleued
diuers places to be in hel, when he ſaid:
\XRef{(Pſal.~85.)}
\Emph{Thou haſt deliuered my ſoule from the lower hel}, ſignifying
plainly that there is \Emph{a lower and a higher hel}: which higher the
Church calleth
\MNote{Purgatorie.}
\Emph{Purgatorie}, where ſoules ſuffer that paine in ſatisfaction for
their ſinnes, which remaineth not ſatisfied before death, & is due after
the guilt of ſinne is remitted, the law preſcribing that beſides
reſtitution of damage, ſacrifice should alſo be offered.
\XRef{(Leuit.~5.}
\XRef{6.}
\XRef{16.)}
And Dauid was punished by the death of his child
\XRef{2.~Reg.~12.}
& by the plague ſent amongſt his people
\XRef{2.~Reg.~24.}
after his ſinnes were remitted. He feared alſo punishment in the other
world, yea two ſortes and therfore prayed to be deliuered from both,
ſaying:
\XRef{(Pſal.~6.)}
\Emph{Lord rebuke me not in thy furie, nor chaſtice me in thy wrath.}
That is (ſaith
\CNote{\Cite{To.~2. in ſept. Pſal. pænitent.}}
S.~Gregorie) \Emph{Strike me not with the reprobate, nor afflict me with
thoſe, that are purged by the punishing flames.} And moſt expreſly
ſignifieth alſo
a higher place called hel, ſaying
\XRef{(Pſal.~15.)}
in the perſon of Chriſt to his Father:
\MNote{\L{Limbus patrũ.}}
\Emph{Thou shalt not leaue my ſoule in hel.} From vvhence Chriſt
deliuered the holie Patriarches, Prophetes, and other perfect ſoules,
reſting vvithout ſenſible paine, & brought them into heauen,
\MNote{No entrance into heauen before Chriſt.}
vvither
before him none could enter. VVhich vvas alſo ſignified by the cities of
refuge, whence none might depart to their proper
%%% 0732
countrie, \Emph{til the death of the high prieſt}
\XRef{(Num.~35.)}
& by \Emph{Moyſes} dying in the deſert, and \Emph{not entring into the
promiſed land} ouer Iordan.
\XRef{Deut.~4.}
\XRef{31.}
&
\XRef{34.}

Preſuppoſing
\MNote{Reſurrection.}
the \Emph{general Reſurrection} of al men (as a truth knovven by former
traditions) king Dauid shevveth the difference of the vvicked, and
godlie in that time, ſaying:
\XRef{(Pſal.~1.)}
\Emph{The impious ſhal not riſe againe in iudgement: nor ſinners in the
councel of the iuſt.} That is, the vvicked shal not riſe to ioy &
glorie, as the iuſt & godlie shal doe.

\Emph{Of
\MNote{Iudgement.}
general iudgement} is more plainly prophecied,
\XRef{1.~Reg.~2.}
\Emph{That our Lord ſhal iudge the endes of the earth}, not that Dauid,
nor Salomon, but Chriſt should raigne in his militant Church, euen
\Emph{to the endes of the earth}, and in fine iudge the vvhole
vvorld. The ſame is confirmed
\XRef{Pſal.~49.}
\Emph{God wil come manifeſtly our God, and he wil not kepe ſilence. Fire
ſhal burne forth in his ſight.}
\XRef{Pſal.~95.}
\Emph{He ſhal iudge the round world in equitie, and the peoples in his
truth.}
\XRef{Pſal.~96.}
\Emph{Fire ſhal goe before him, and ſhal inflame his enemies round
about.} Againe the ſame royal prophete
\XRef{(Pſalm.~48.)}
deſcribeth the
\MNote{Eternal paine of the damned and glorie of the bleſſed.}
future and eternal ſtate of the \Emph{damned} ſaying: as
%%% o-0658
\Emph{ſheepe} (creatures vnable to helpe themſelues) \Emph{they are put in
hel, death ſhal feede vpon them}. Of the bleſſed he addeth: \Emph{And
the iuſt shal rule ouer them in the morning}, that is, in the
reſurrection, and
\XRef{Pſal.~149.}
\Emph{The Sainctes shal reioyſe in glorie, they shal be ioyful in their
beddes} (in eternal reſt.) \Emph{The exaltations} (prayſes) \Emph{of God
in their throate, and two edged ſwordes in their handes: to doe reuenge
in the nations, punishments among the peoples. To bind their kinges in
fetters, and their nobles in yron manicles. That they may doe in them
the iudgement that is written: This glorie is to al his Sainctes.} And
much greater glorie belongeth to \Emph{Sainctes}: for this is but
accidental, vttered according to vulgar capacitie.
\CNote{\XRef{1.~Cor.~2.}}
The eſsential and perfect glorie, which no eye hath ſeene, nor eare hath
heard, nor hart can conceiue, conſiſteth in ſeeing God. Among accidental
glorious giftes,
\MNote{Foure dowries of glorified bodies prefigured.}
the foure dowries of glorified bodies are eſpecially prefigured:
\CNote{\XRef{1.~Cor.~15.}}
\Emph{Impaſſibilitie} by \Emph{the wood Setim}, wherof the Arke was
made.
\XRef{(Exod.~25.)}
\Emph{Agilitie} and \Emph{Penetrabilitie} in ſome ſorte by Dauids
quicknes againſt Goliath, and his conueying of him ſelf into Sauls campe
and forth againe.
\XRef{(1.~Reg.~17.}
and
\XRef{26.}
But a more plaine figure of \Emph{Claritie} was in Moyſes face
\XRef{(Exod.~34.)}
\CNote{\Cite{Catheciſ. Rom. p.~1. c.~12. q.~9.}}
which by his conuerſation with God, became more glorious then mortal
eyes were able to behold, gliſtering and shining as moſt ſplendent light
through chriſtal, deſcribed as if his skinne had benne a clere
\Emph{horne}, appearing and ſpreading beames like the ſunne, proceding
from the beautie of his ſoule, ſo that none of al the people could looke
directly vpon him, except he couered his face.

%%% 0733
Thus
\MNote{The Church more knowen to other nations then before.}
much concerning particular pointes of faith and religion. And it is no
leſse euident, that the vniuerſal \Emph{Church} and Citie of God ſtil
continued: yea was \Emph{more viſible}, and conſpicuous to the whole
world then before. Firſt by Gods maruelous protection therof in the
deſert, and famous victories and conqueſtes of the land of Chanaan. And
by the excellent lawes geuen to this people; which al nations admired,
and none had the like.
\XRef{Deut.~4.}
\MNote{The Eccleſiaſtical and temporal ſtates more diſtinguiſhed.}
For in this fourth age, beſides other lavves and preceptes, the
ſpiritual and temporal States were more diſtinguished, and the
\Emph{Eccleſiaſtical Hierarchie} eſpecially diſpoſed in ſubordination of
one ſupreme head, with inferiour gouerners, ech in their place and
office, for edification of the whole bodie. For Moyſes being chief ruler
and conducter of the Iſraelites out of Ægypt, recieued and deliuered to
them the written Law.
\XRef{(Exod.~20.)}
And for obſeruation and conſeruation therof by Gods expreſse appointment
\XRef{(Leuit.~8.)}
conſecrated \Emph{Aaron the ordinarie High prieſt}, himſelf remayning
ſtil extraordinarie Superiour, alſo aboue Aaron.
\MNote{Succeſſion of High Prieſtes.}
And after Aaron he conſecrated in like maner his ſonne \Emph{Eleazar}
high prieſt, and ſucceſſour to his father.
\XRef{(Num.~20.)}
To whom ſucceded others in this order
\XRef{(1.~Paralip.~6.)}
\Emph{Phinees, Abiſuë, Bocci, Ozi, Zacharias}, (otherwiſe
\XRef{1.~Reg.~1.}
called Heli) \Emph{Meraioth, Amarias}, (otherwiſe Achimelec, whom Saul
ſlew,
\XRef{1.~Reg.~22.)}
\Emph{Achitob} (othervviſe Abiathar, vvho vvas depoſed,
\XRef{3.~Reg.~2.)}
and \Emph{Sadoc}, in vvhoſe time the Temple vvas founded.

%%% o-0659
To
\MNote{Diſtinction of offices in Prieſtes & Leuites.}
theſe vvere adioyned other \Emph{Prieſtes}, alſo conſecrated in a
præſcript forme.
\XRef{(Leuit.~8.)}
and \Emph{Leuites} ordayned to aſsiſt in lower and diſtinct offices.
\XRef{(Num.~3. &~4.)}
In the firſt degree \Emph{the Caathites}, whoſe office was to carrie the
Sanctuarie, and veſſel therof vvrapped vp by the prieſtes, but vvere forbid
in paine of death, to touch them, or to ſee them. In the ſecond degree
\Emph{the Gerſonites}; vvho carried the cortines and couers of the
Tabernacle, and veſſel of the Altar. In the third degree \Emph{the
Merarites}; vvho carried the bordes, barres, and pillers, vvith their
feete, pinnes, cordes, and other implementes of the tabernacle;
\Emph{euerie one according to their office and burdens}.
\XRef{Num.~4. v.~vlt.}

But
\MNote{Succeſſion of temporal princes interrupted.}
\Emph{in the temporal ſtate} and gouernment \Emph{Ioſue} of the tribe of
Ephraim \Emph{ſucceeded to Moyſes}.
\XRef{(Num.~27.}
\XRef{Deut.~3.}
&
\XRef{34.)}
And after Ioſue were diuers \Emph{interruptions of ſucceſſion}, with
gouerners of diuers tribes, and \Emph{change of gouernment}, from
\MNote{Dukes.}
\Emph{Dukes} to \Emph{Iudges}, and from Iudges to \Emph{Kinges}. For
after Ioſues death the people being ſore afflicted by inuaſions of
Infidels, God raiſed certaine ſpecial men, with title of
\MNote{Iudges.}
\Emph{Iudges} to deliuer and ſaue them. Firſt \Emph{Othoniel} of the tribe of Iuda;
then \Emph{Aod} of Beniamin; after him \Emph{Samgar} (the Scripture not
ſignifying of what tribe) then \Emph{Barach} with \Emph{Debora} of
Ephraim; \Emph{Gedeon} of Manaſses; \Emph{Abimelech}, his baſe ſonne,
\Emph{an vſurper}; \Emph{Thola} of Iſſachar; \Emph{Iair, and Iephte} of
Manaſses;
%%% 0734
\Emph{Abeſan} of Iuda; \Emph{Aialon} of Zabulon; \Emph{Abdon} of
Ephraim; \Emph{Sampſon} of Dan; and \Emph{Heli}, who was alſo high
prieſt of Aarons ſtocke, otherwiſe called Zaraias
\XRef{(1.~Paralip.~6.)}
and \Emph{Samuel} alſo of the tribe of Leui a \Emph{Prophet}. In his
time the people demanding and vrging to haue a
\MNote{Kinges.}
\Emph{King, Saul} of the tribe of \Emph{Beniamin} was annointed.
\XRef{1.~Reg.~10.}
But for tranſgreſsing Gods commandments, eſpecially for exerciſing
ſpiritual function without warrant
\XRef{(1.~Reg.~13.)}
and not deſtroying idolaters
\XRef{(1.~Reg.~15.)}
was depoſed, and \Emph{Dauid} of the tribe of \Emph{Iuda} was annointed
King; who after manie great trubles, poſseſsed the whole kingdome, and
died in peace, leauing his ſonne \Emph{Salomon} inueſted and annointed
king in his throne.

The
\MNote{Manie ſinnes & difficulties in the Church.}
Church being thus eſtablished in diſtinct ſtates and orders, albeit
there were manie imperfections in al ſortes of perſons, and great ſinnes
committed, yet God ſo punished offenders, and chaſtiſed the whole
people, that he ſtil conſerued, the greateſt, or chiefe part, in true
faith and religion. For whiles they were in the deſert, they
\MNote{Murmure.}
\Emph{murmured} very often againſt God, and his Miniſters their Superiours.
\XRef{(Exod.~17.}
\XRef{Num.~11.}
\XRef{14.}
\XRef{20.}
\XRef{21.)}
\MNote{Idolatrie.}
Manie \Emph{fel to idolatrie}.
\XRef{(Exod.~32.)}
\Emph{Aaron not free from cooperating} in the peoples ſinne.
\Emph{Nadab} and \Emph{Abiu} Aarons ſonnes, and conſecrated prieſtes,
\Emph{offered ſtrange fire}.
\XRef{(Leuit.~10.)}
\MNote{Schiſme.}
\Emph{Core, Dathan}, and \Emph{Abiron}, with their complices
\Emph{made a great ſchiſme}.
\XRef{(Num.~16.)}
\MNote{Carnal fornication cauſe of Idolatrie.}
Manie committed \Emph{carnal fornication} with Infidels; and were therby
drawen to \Emph{ſpiritual}.
\XRef{(Num.~25.)}
Of which and other like ſinnes the Pſalmiſt ſpeaketh
\XRef{(Pſal.~94.)}
exhorting his people \Emph{not to harden their hartes, as in the deſert
their fathers had tempted God. Fourtie yeares was I offended} (ſayth
God)
\Emph{with that generation, and ſayd: They alwayes erre in hart.} And
therfore he ſware in his
%%% o-0660
wrath: that the ſame generation should not enter into the promiſed land
of Chanaan: but their children entred and poſseſſed it.
\XRef{Num.~14.}
\XRef{Ioſue.~3.}

Againe
\MNote{The Church afflicted for ſinnes, yet was ſtil conſerued.}
the people falling to idolatrie and other ſinnes, were afflicted and
ſore preſſed by forraine enemies, but repenting were deliuered and ſaued
by certain capitaines called \Emph{Iudges} and
\CNote{\XRef{Iudic.~3.}}
\Emph{Sauiours}: as appeareth in the booke of Iudges. They had alſo
tribulations by ſome of their owne nation, for among the Iudges one
(called Abimelec) was a \Emph{tyrannical vſurper}.
\XRef{(Iudic.~9.)}
\Emph{Saul} their firſt King falling from God vniuſtly \Emph{perſecuted
Dauid}.
\XRef{(1.~Reg.~18. &c.)}
Ambitious \Emph{Abſolom} rebelled againſt the King his father,
\XRef{(2.~Reg.~15.)}
and \Emph{Seba} of the tribe of Beniamin raiſed an other rebellion.
\XRef{(2.~Reg.~20.)}
Likewiſe \Emph{Adonias}, aſsiſted by \Emph{Abiathar} the high prieſt,
and by \Emph{Ioab} general of the armie, pretended to reigne his father
Dauid yet liuing, to preuent Salomon of the kingdom.
\XRef{(3.~Reg.~1.)}
So God both shewed his iuſtice, in ſuffering ſuch afflictions to happen,
for punishment of ſinne: and his mercie, in ſauing his Church from
ruine.

Moreouer
\MNote{Ordinarie meanes of conſeruing the Church.}
for preſeruatiõ of the Church, there were diuers diuine Ordinances
%%% 0735
prouided by the law.
\MNote{No participation with Infidels.}
For firſt al were ſtrictly commanded, not to cõmunicate with Infidels in
their idolatrie
\XRef{(Ex.~23.)}
nor with Schiſmatikes in their ſchiſme
\XRef{(Nu.~16.)}
but \Emph{to deſtroy al Idolaters}
\XRef{(Num.~33.)}
\MNote{No noueltie to be admitted.}
and \Emph{shunne al nouelties} in religion, as a ſure marke of
idolatrie, or falſe doctrine.
\XRef{(Deut.~13.)}
\CNote{\Cite{S.~Chriſ. orat.~1. aduerſ. Iudeos.}}
Further to conſerue vnitie there was but
\MNote{But one Tabernacle.}
\Emph{one Tabernacle}, and
\MNote{One Altar for ſacrifice.}
\Emph{one Altar for Sacrifice}, in the whole people of Iſrael. VVherupon
when the two tribes and halfe, on the other ſide Iordan, had made a
ſeueral altar, al the tribes that dwelt in Chanaan, ſuſpecting it was
for ſacrifice, ſent preſently to admonish them, and prepared to make
warre againſt them, except they deſtroyed their new altar, but
being aduertiſed that it was only \Emph{an altar of monument}, and not
for ſacrifice, were therwith ſatisfied.
\XRef{(Ioſue.~22.)}
Afterwards the tribe of \Emph{Dan, ſetting vp idolatrie}, and the other
tribes \Emph{not correcting} it, they were al punished. VVhich happened
by occaſion of an other enormous ſinne, committed and not corrected in
the tribe of Beniamin. For the other eleuen tribes making warre againſt
them for this iuſt cauſe, yea \Emph{by Gods direction}, and warrant, yet
\Emph{had the worſe}, ſuſteyning great ſlaughter of men in two
conflictes, and in the third Beniamin was almoſt deſtroyed.
\XRef{Iudic.~20.}

Finally
\MNote{One ſupreme Iudge of controuerſies.}
for \Emph{deciſion of al controuerſies} and ending of ſtrife, the
\Emph{High Prieſt} was expreſly \Emph{ordayned} ſupreme Iudge.
\XRef{(Deut.~17.)}
\MNote{Al bound to obey him.}
And al were commanded in paine of death to ſubmitte their opinions, and
obey his ſentence:
\MNote{His ſentence infallible.}
with promiſe of Gods aſsiſtance, wherby his \Emph{definitions} were
\Emph{certaine and infallible}. For in conſultation of doubtes, and
difficult caſes, God inſpired him with \Emph{doctrine of veritie}.
\XRef{(Exod.~28.}
\XRef{29.}
\XRef{Leuit.~8.}
\XRef{Num.~3.}
\XRef{7.}
\XRef{9.}
\XRef{1.~Reg.~23.}
\XRef{30.}
VVhich iudgement Seate Chriſt admonished the Iewes to repayre vnto and
folow
\XRef{(Math.~23.)}
though the Iudges themſelues did not the thinges which they taught. In
ſo much that Caiphas, through this aſsiſtance of Gods ſpirite, being
otherwiſe a wicked man, yet pronounced the truth, \Emph{That one muſt
die for the people.} VVhich therfore S.~Iohn the Euangeliſt aſcribeth to
his Chayre and office, \Emph{becauſe he was High prieſt that yeare}.
\XRef{Ioan.~11.}

%%% o-0661
Seing
\MNote{The Church of Chriſt preſerued from erring in Religion.}
then Gods prouidence and continual aſsiſtance was ſo clere, and aſſured
in the Church of the old Teſtament, much more is the
%%% !!! Where do these really go?
\CNote{\XRef{Math.~16.}
\XRef{28.}
\XRef{Luc.~22.}
\XRef{Ioan.~14.}
\XRef{16.}
\XRef{Eph.~4.}
\XRef{1.~Tim.~3.}}
\Emph{Church of Chriſt builded vpon a ſure rocke, aſſured of his
perpetual aſſiſtance, and always preſerued from erring in Faith, or in
general practiſe of Religion.} And that by Gods like aſſured ordinance
of \Emph{one ſupreme head and Iudge, S.~Peter, & his Succeſſour}: for
vvhom our Sauiour prayed, that \Emph{his faith should not
faile}. Further commanding him, that \Emph{he ſhould confirme his
brethren}. Al vvhich vve ſee is performed in the Succeſsours of
S.~Peter, vvheras the ſucceſſours of the other Apoſtles, are al failed
long ſince. The ſame moſt aſsured ſtabilitie of the Church of Chriſt, is
further confirmed by the whole Lavv and Prophetes. Namely,
\XRef{Deut.~12.}
and
\XRef{33.}
vvhere Moyſes fortelleth more povver and grace in \Emph{the Church, to
be collected in the
%%% 0736
Gentiles of al natiõs}, then euer vvas in that of the Iſraelites or
Iewes. Likewiſe,
\XRef{1.~Reg.~2.}
The ſame vvas both prefigured and prophecied by holie Anna:
%%% !!! XRef ?
\Emph{The hungrie} (thoſe that deſire Gods grace and glorie) \Emph{are
filled: vntil the barren woman} (the Church of the Gentiles) \Emph{bare
verie manie: & ſhe that had manie children was weakned.} Shewing that
the Church of the Iewes had manie, vntil the plenitude of Gentiles much
more abounded. Wherfore the Pſalmiſt inuiteth al nations to praiſe God,
ſaying:
\XRef{Pſal.~116.}
\Emph{Praiſe our Lord al ye Gentiles: praiſe him al ye peoples.}
Alſo
\XRef{2.~Reg.~7.}
God promiſed Dauid, ſaying:
\MNote{Not anie temporal but Chriſts kingdom is in al nations and
perpetual.}
\Emph{Thy Kingdome for euer before thy face, and thy throne ſhal be
firme continually.}
\CNote{\Cite{S.~Aug. li.~17. c.~8. de ciuit.}}
Which was not verified in Dauids temporal kingdome. For it was quickly
diuided, after Salomons death, and a ſmal part left to his ſonne
Roboam. And after the captiuitie in Babilon, his ſeede had onlie title
and right without poſſeſsion of royal throne. Againe
\XRef{2.~Reg.~22.}
\CNote{\Cite{S.~Epiph. hæreſ.~29.}}
The ſame royal prophet in his Canticle of thankeſgeuing, and laſt
prophetical wordes
\XRef{(chap.~23.)}
much preferreth the ſpiritual kingdome of Chriſt, before the earthlie
kingdome of the Iewes. But moſt ſpecially and plainly in the Pſalmes.
\XRef{Pſal.~2.}
\Emph{Why did the Gentiles rage, & peoples meditate vaine thinges?}
Signifying that the furie of al aduerſaries rageth in vaine, againſt
Chriſt and his Church.
\MNote{The Church of Chriſt vniuerſal.}
\Emph{For, I am appointed, by him} (ſayth Chriſt of his
Father) \Emph{king ouer Sion, his holie hil.
\CNote{\XRef{Act.~4.}}
I wil geue thee} (ſayth God to his Sonne) \Emph{the Gentiles for thine
inheritance, and thy poſſeſſion the endes of the earth.}
\XRef{Pſal.~17.}
\Emph{A people which I knew not, hath ſerued me.}
\XRef{Pſal.~44.}
\Emph{The Queene} (the Church) \Emph{ſtood on thy right hand in golden
rayment, compaſſed with varietie}; of vertues, and diuers ſortes of
holie profeſsions.
\XRef{Pſal.~47.}
\Emph{Mount Sion is ſounded with the exultation of the whole earth. For
euer and euer he} (Chriſt) \Emph{ſhal rule vs euermore.}
\XRef{Pſal.~86.}
\Emph{Glorious thinges are ſayd of thee, ô citie of God.} But omitting
innumerable other ſuch textes, the
\XRef{88.~Pſalme}
conteyneth a large prophecie of Chriſt and his Church, where
\CNote{\Cite{in hunc. Pſalm.}}
S.~Auguſtin geueth vs this brief admonition. \L{Chriſtiani eſtis,
Chriſtum agnoſcite.} \Emph{You are Chriſtians, agnize Chriſt.}
\Emph{I wil put} (ſayth God) \Emph{his hand in the ſea}, Chriſts
dominion in the Gentile, and \Emph{his right hand in the riuers}; al
ſortes shal ſerue him. \Emph{He ſhal be high aboue the kinges of the
earth.} Of the Church he addeth: \Emph{I wil put his ſeede for euer and
euer, and his throne as the dayes of heauen.} Neither
%%% o-0662
do ſinnes fruſtrate this promiſe of God, therfore it foloweth: \Emph{But if
his children ſhal forſake my law: and wil not walke in my iudgements. If
they ſhal profane my iuſtices, and not keepe my commandements};
\MNote{The Iewes wil not ſee Chriſt:
\XRef{2.~Cor.~3.}
And Heretikes wil not ſee the Church: which yet is alwayes viſible.
\Cite{S.~Aug. in Pſal.~30. conc.}
\Cite{2.~Collat. Carthag. et cont. Donatiſt.}}
VVhat then, wil Chriſt for al this abandon his Church, as he did the old
Synagogue, of which God ſayth:
\XRef{Deut.~32.}
\Emph{They haue prouoked me in that which was no God: and I wil prouoke
them, in that which is no people?} Not ſo. How then? \Emph{I wil
%%% 0737
viſite}, ſayth our Lord, \Emph{their iniquities with a rodde, and their
ſinnes with ſtripes. But my mercie I wil not take away from him.} This
is a ſtrong Firmament (ſayth
\CNote{\Cite{Ibidem.}}
S.~Auguſtin) \Emph{God promiſeth}, yea \Emph{ſweareth}, and vvil
\Emph{not lie to Dauid, that his ſeede ſhal continew for euer. His
throne as the Sunne in Gods ſight, and the Moone perfected for euer.} So
this great Doctor sheweth by holie Scriptures againſt the Donatiſtes,
and in them againſt Proteſtantes, that the militant Church of Chriſt
hath benne ſtil, and shal be viſible, during this tranſitorie world.


\stopArgument


\stopcomponent


%%% Local Variables:
%%% mode: TeX
%%% eval: (long-s-mode)
%%% eval: (set-input-method "TeX")
%%% fill-column: 72
%%% eval: (auto-fill-mode)
%%% coding: utf-8-unix
%%% End:
