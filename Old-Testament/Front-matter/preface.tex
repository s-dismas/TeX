%%%%%%%%%%%%%%%%%%%%%%%%%%%%%%%%%%%%%%%%%%%%%%%%%%%%%%%%%%%%%%%%%
%%%%
%%%% The (original) Douay Rheims Bible 
%%%%
%%%% Old Testament
%%%% Front matter
%%%% Preface
%%%%
%%%%%%%%%%%%%%%%%%%%%%%%%%%%%%%%%%%%%%%%%%%%%%%%%%%%%%%%%%%%%%%%%




\startcomponent preface


\project douay-rheims


%%% 0003
\startPreface[
  title={\Sc{to the right vvelbeloved english reader grace and glorie in
      Iesvs Christ everlasting}},
  marking={Preface}
  ]

At laſt through Gods goodnes (moſt dearly beloued) we ſend you here
the greater part of the Old Teſtament: as long ſince you receiued the
New; faithfully tranſlated into Engliſh.  The reſidue is in hãd to be
finiſhed: and your deſire therof ſhal not now (God proſpering our
intention) be long fruſtrate.
\MNote{The cauſe of delay in ſetting forth this Engliſh Bible.}
As for the impediments, which hitherto haue hindered this vvorke, they
al proceded (as manie do know) of one general cauſe, our poore eſtate in
baniſhment.  VVherin expecting better meanes, greatter difficulties
rather enſued.  Neuertheles you wil hereby the more perceiue our feruent
good wil, euer to ſerue you, in that we haue brought forth this Tome, in
theſe hardeſt times, of aboue fourtie yeares,
\SNote{1568}
ſince this College was
moſt happely begune.  VVherfore we nothing doubt, but you our deareſt,
for whom we haue dedicated our liues, wil both pardon the long delay,
which we could not wel preuent, and accept now this fruict of our
laboures, with like good affection, as we acknowlege them due, and offer
the ſame vnto you.

If anie demand,
\MNote{VVhy & how it is allowed to haue holie Scriptures in vulgar
  tongues.}
why it is now allowed to haue the holie Scriptures in vulgar tongues,
which generally is not permitted, but in the three ſacred only: for
further declaration of this, & other like pointes we remite you to the
Preface, before the New Teſtament.  Only here, as by an Epitome,
%%% 0004  
we ſhal repete the ſumme of al, that is there more largely diſcuſſed.
To this firſt queſtion therfore we anſwer, that both iuſt reaſon, &
higheſt authoritie of the Church, iudge it not abſolutly neceſſarie, nor
always conuenient, that holie Scriptures ſhould be in vulgar tongues.
\MNote{Scriptures being hard are not to be read of al.}
For being as they are, hard to be vnderſtood, euen by the lerned, reaſon
doth dictate to reaſonable men, that they were not written, nor ordayned
to be read indifferently of al men.  Experiẽce alſo teacheth, that
through ignorance, ioyned often with pride and preſumption,
\MNote{Manie take harme by reading holie Scriptures.}
manie reading Scriptures haue erred groſly, by miſunderſtanding Gods
word.  VVhich though it be moſt pure in it ſelf,
\CNote{\Cite{lib. de Prescrip.}}
\Emph{yet the ſenſe
  being adulterated is as perilous} (ſaith Tertullian) \Emph{as the
  ſtile corrupted}.  S.~Ambroſe obſerueth: that
\CNote{\Cite{lib.~2. ad Gratian. c.~1.}}
\Emph{vvhere the text is
  true, the Arrians interpretation hath errors}.  S.~Auguſtin alſo
teacheth, that
\CNote{\Cite{Tract.~18. in Ioan.}}
\Emph{hereſies and peruerſe doctrines entangling ſoules,
  and throvving them dovvne headlong into the depth, do not othervviſe
  ſpring vp, but vvhen good (or true) Scriptures are not vvel (and
  truly) vnderstood, and vvhen that vvhich in them is not vvel
  vnderſtood, is alſo rashly & boldly auouched}.  For the ſame cauſe,
\CNote{\Cite{Epist.~103. c.~6.}}
S.~Ierom vtterly diſallowed, that al ſortes of men & wemen, old &
yong, preſumed to read & talke of the Scriptures: wheras \Emph{no
  articene, no tradſman dare preſume to teach anie facultie, vvhich he
  hath not firſt lerned}.  Seing therfore that dangers, & hurtes happen
in manie, the careful chief Paſtores in Gods Church,
\MNote{Reading of Scriptures moderated.}
haue alwaies moderated the reading of holie
Scriptures, according to perſons, times, and other circumſtances;
prohibiting ſome, and permitting ſome, to haue and read them, in their
mother tongue.
\MNote{Scriptures tranſlated into diuers tongues.}
So S.~Cryſoſtom trãnſlated the Pſalmes & ſome other partes of holie
Scriptures for the Armenians, when he was there in baniſhment.  The
\CNote{\Cite{Bibl. Sanct. lib.~4.}}
Slauonians and Gothes ſay they haue the Bible in their languages.  It
was tranſlated into Italian by an Archbyſhop of Genua.  Into French in
the time of king Charles the fiſt: eſpecially
becauſe the waldenſian heretikes had corruptly tranſlated
%%% 0005
it, to maintaine their errors.  VVe had ſome partes in Engliſh
tranſlated by Venerable Bede: as
\CNote{\Cite{Lib.~1. Hist. c.~47.}}
Malmesburie witneſſeth.  And Thomas
Arundel Archbishop of Canturburie in a Councel holden at Oxford,
\CNote{\Cite{Linvvod lib.~1.}}
ſtraictly ordayned, that no heretical tranſlation ſet forth by
wicliffe, and his complices, nor anie other vulgar Edition ſhould be
ſuffered, til it were approued by the Ordinarie of the Dioceſe:
alleaging S.~Ieroms iudgement of the difficultie & danger in
tranſlating holie Scriptures out of one tongue into an other.  And
therfore it muſt nedes be much more dangerous, when ignorant people
read alſo corrupted tranſlations.  Now ſince Luther, and his folowers
haue pretended, that the
\MNote{A calumnious ſuggeſtion of Lutheranes.}
Catholique Romane faith and doctrine, ſhould be contrarie
to Gods written word, & that the Scriptures were not ſuffered in
vulgar languages, leſt the people ſhould ſee the truth, & vvithal
theſe new maiſters corruptly turning the Scriptures into diuers
tongues, as might beſt ſerue their owne opinions: againſt this falſe
ſuggeſtion, and practiſe, Catholique Paſtores haue, for one eſpecial
remedie, ſet forth true and ſincere Tranſlations in moſt languages of
the Latin Church.  But ſo, that people muſt read them with licence of
their ſpiritual ſuperior, as in former times they were in like ſort
limited.  Such alſo of the Laitie, yea & of the meaner lerned Clergie,
as were permitted to read holie Scriptures, did not preſume to
interprete hard places, nor high Myſteries, much leſſe to diſpute and
contend, but leauing the diſcuſſion therof to the more lerned,
ſearched rather, and noted the
\MNote{VVhat part of Scriptures be moſt conuenient for vulgar readers.}
godlie and imitable examples of
good life, and ſo lerned more humilitie, obedience, hatred of ſinne,
feare of God, zele of Religion, and other vertues.  And thus holie
Scriptures may be rightly vſed in anie tongue,
\CNote{\XRef{2.~Tim.~3.}}
\Emph{to teach, to
argue, to correct, to inſtruct in iuſtice, that the man of God may be
perfect, and} (as S.~Paul addeth) \Emph{inſtructed to euerie good
vvorke}, when men laboure rather to be
\CNote{\XRef{Iac.~1.}}
\Emph{doers of Gods wil &
vvord, then readers or hearers only, deceiuing themſelues}.

%%% 0006
But here an other queſtion may be propoſed:
\MNote{VVhy we tranſlate the old Latin text.}
VVhy we tranſlate the Latin text, rather then the
Hebrew, or Greke, which Proteſtantes preferre, as the fountaine tongues,
wherin holie Scriptures were firſt written?  To this we anſwer, that if
in dede thoſe firſt pure Editions were now extant, or if ſuch as be
extant,
\MNote{More pure then the Hebrew or Greke now extant.}
were
more pure then the Latin, we would alſo preferre ſuch fountaines before
the riuers, in whatſoeuer they ſhould be found to diſagree.  But the
ancient beſt lerned
\CNote{\Cite{Tertulliã li.~5. cont. Marcion}
\Cite{S.~Ambrose li.~3. de Spirit. San. c.~11.}
\Cite{S.~Ierom. li.~1. con. Iouiniã.}}
Fathers, & Doctors of the Church, do much complaine,
and teſtifie to vs, that both the Hebrew and Greke Editions are fouly
corrupted by Iewes, and Heretikes, ſince the Latin was truly tranſlated
out of them, whiles they were more pure.  And that the ſame Latin hath
bene farre better conſerued from corruptions.  So that the old Vulgate
Latin Edition hath bene preferred, and vſed for moſt authentical aboue a
thouſand and three hundered yeares.  For by this verie terme
\CNote{\Cite{in 49.~Iſaia.}}
S.~Ierom
calleth that Verſion \Emph{the vulgate or common}, which he conferred
with the Hebrew of the old Teſtament, and with the Greke of the New;
which he alſo purged from faultes committed by writers, rather amending
then tranſlating it.  Though in regard of this amending, S.~Gregorie
calleth it
\CNote{\Cite{li.~20. c.~24. mora.}}
\Emph{the nevv verſiõ of S.~Ierom}: who neuertheles in an
other place calleth the ſelf ſame,
\CNote{\Cite{Epiſt. dedicat. ad Leander.}}
\Emph{the old Latin Edition}, iudging it moſt worthy to be folowed.
S.~Auguſtin calleth it the
\CNote{\Cite{li.~2. Doct. Chriſt. c.~14.}}
\Emph{Italian}.
\CNote{\Cite{lib.~6. Etym. c.~5.}
&
\Cite{li.~1. de Diuin offic. c.~12.}}
S.~Iſidorus witneſſeth
\MNote{Receiued by al Churches.}
that \Emph{S.~Ieroms verſion} was receiued and \Emph{approued
by al Chriſtian Churches}.
Sophronius alſo a moſt lerned man, ſeing
S.~Ieroms Edition ſo much eſtemed, not only of the Latines, but alſo of
the Grecians,
\MNote{Turned into Greke.}
turned the Pſalter &
Prophetes, out of the ſame Latin into Greke.  Of latter times what ſhal
we nede to recite other moſt lerned men?  S.~Bede S.~Anſelme,
S.~Bernard, S.~Thomas, S.~Bonauenture, & the reſt?  VVho al vniformly
allege this only text as authentical.
\MNote{Al others growne out of vſe.}
In ſo much that al other Latin Editions, which
\CNote{\Cite{Prefat. in Ioſue.}}
S.~Ierom ſaith were in his time almoſt innumerable, are as it were
%%% 0007
fallen out of al Diuines handes, and growne out of credite and vſe.  If
moreouer we conſider
\MNote{S.~Ierom excelled al other Doctors in tranſlating &
expounding holie Scriptures.} 
S.~Ieroms lerning, pietie, diligence, and ſinceritie, together with the
commodities he had of beſt copies, in al languages then extant, and of
other lerned men, with whom he conferred: and if we ſo cõpare the ſame
with the beſt meanes that hath bene ſince, ſurely no man of indifferent
iudgement, wil match anie other Edition with S.~Ieroms: but eaſely
acknowlege with the whole Church Gods particular prouidẽce in this great
Doctor, as wel for expounding, as moſt eſpecialy for the true text and
Edition of Holie Scriptures.  Neither do we flee vnto this old Latin
text, for more aduantage.  For beſides that
\MNote{His Edition free from partialitie.}
it is free from partialitie, as being moſt ancient of
al Latin copies, and long before the particular Controuerſies of theſe
dayes beganne; the Hebrew alſo & the Greke when they are truly
tranſlated, yea and Eraſmus his Latin, in ſundrie places, proue more
plainly the Catholique Romaine doctrine, then this which we relie vpon.
So that
%%% !!! Does this make sense here
\CNote{\XRef{luc.~22. v.~20.}
\Cite{Prefat. Noui. Teſtam. Anno. 1556.}
\XRef{Luc.~1. v.~1.}}
\MNote{Preferred before al other Editions by Beza.}
Beza &
his folowers take alſo exception againſt the Greke, when Catholiques
allege it againſt them.  Yea the ſame Beza preferreth the old Latin
Verſion before al others, & freely teſtifieth, that the old Interpreter
tranſlated religiouſly.  VVhat then do our countriemen, that refuſe this
Latin, but depriue themſelues of the beſt, and
\MNote{None yet in England allowed for ſufficient.}
yet al this while, haue ſet forth none,
that is allowed by al Proteſtantes, for good or ſufficient.

How wel this is donne the lerned may iudge, when by mature conference,
they ſhal haue made trial therof.  And if anie thing be miſtaken, we wil
\SNote{\XRef{Preface before the nevv Teſtam.}}
(as ſtil we promiſe) gladly correct it.  Thoſe that trãſlated it about
thirtie yeares ſince, were wel knowen to the world, to haue bene
excellent in the tongues, ſincere men, and great
Diuines.
\MNote{VVhat is done in this Edition}
Only one thing we
haue donne touching the text, wherof we are eſpecially to geue notice.
That whereas heretofore in the beſt Latin Editions, there remained manie
places differing
%%% 0008
in wordes, ſome alſo in ſenſe, as in long proceſſe of time, the
writers erred in their copies;
\MNote{Diuers readinges reſolued vpon, & none leift in the margent.}
now lately by the care & diligence
of the Church, thoſe diuers readings were maturely, and iuditiouſly
examined, and conferred with ſundrie the beſt written and printed
bookes, & ſo reſolued vpon, that al which before were leift in the
margent, are either reſtored into the text, or els omitted; ſo that
now none ſuch remaine in the margent.  For which cauſe, we haue againe
conferred this Engliſh tranſlation, and conformed it to the moſt
perfect Latin Edition.  VVhere yet by the way we muſt geue the vulgar
reader to vnderſtand,
\MNote{They touched not preſent controuerſies.}
that very few or none of the former varieties, touched
Controuerſies of this time.  So that this Recognition is no way
ſuſpicious of partialtie, but is merely donne for the more ſecure
conſeruation of the true text; and more eaſe, and ſatisfaction of
ſuch, as otherwiſe ſhould haue remained more doubtful.

\MNote{VVhy ſome vvordes are not tranſlated into vulgar Engliſh.}
Now for the ſtrictnes obſerued in tranſlating ſome wordes, or rather
the not tranſlating of ſome, which is in more danger to be diſliked,
we doubt not but the diſcrete lerned reader, deeply weighing and
conſidering the importance of ſacred wordes, and how eaſely the
tranſlatour may miſſe the ſenſe of the Holie Ghoſt, wil hold that
which is here donne for reaſonable and neceſſarie.
\MNote{Some Hebrew wordes not tranſlated into Latin, nor Greke.}
VVe haue alſo the
example of the Latin, and Greke, where ſome wordes are not tranſlated,
but left in Hebrew, as they were firſt ſpoken & written; which ſeeing
they could not, or were not conuenient to be tranſlated into Latin or
Greke, how much leſſe could they, or was it reaſon to turne them into
Engliſh?
\CNote{\Cite{li.~2. Doct. Chriſt. cap.~11.}}
S.~Auguſtin alſo yeldeth a reaſon, exemplifying in the
wordes \Emph{Amen} and \Emph{Alleluia,
\MNote{More authoritie in ſacred tongues.}
for the more ſacred authoritie therof} which
doubtles is the cauſe why ſome \Emph{names of ſolemne Feaſtes,
Sacrifices}, & other holie thinges are \Emph{reſerued in ſacred
tongues}, Hebrew, Greke, or Latin.  Againe for neceſſitie, Engliſh
not hauing a name, or ſufficient terme, we either
%%% 0009
kepe the word, as we
find it, or only turne it to our Engliſh termination,
\MNote{Some vvordes can not be turned into Engliſh.}
becauſe it would otherwiſe
require manie wordes in Engliſh, to ſignifie one word of an other
tongue.  In which caſes, we commonly put the explication in the
margent.  Briefly our Apologie is eaſie againſt Engliſh Proteſtantes;
\MNote{Proteſtantes leaue ſome vvordes vntranſlated.}
becauſe they
alſo reſerue ſome wordes in the original tongues, not tranſlated into
Engliſh: as \Emph{Sabath, Ephod, Pentecoſt, Proſelyte}, and ſome
others.  The ſenſe wherof is in dede as ſoone lerned, as if they were
turned ſo nere as is poſſible into Engliſh.  And why then may we not
ſay \Emph{Prepuce, Phaſe} or \Emph{Paſch, Azimes, Breades of
Propoſition, Holocauſt}, and the like? rather then as Proteſtantes
tranſlate them: \Emph{Foreſkinne, Paſſouer, The feaſt of ſvvete
breades, Shevv breades, Burnt offerings}: &c.  By which termes, whether
they be truly tranſlated into Engliſh or no, we wil paſſe ouer.  Sure
it is an Engliſh man is ſtil to ſeke, what they meane, as if they
remained in Hebrew, or Greke.  It more importeth, that nothing be
wittingly and falſly tranſlated, for aduantage of doctrine in matter
of faith.  VVherein as we dare boldly auouch the ſinceritie of this
Tranſlation, and that nothing is here either vntruly, or obſcurely
donne of purpoſe, in fauour of Catholique Romane Religion: ſo we can
not but complaine, and
\MNote{Corruptions in Proteſtantes Tranſlations of holie Scriptures.}
chalenge Engliſh Proteſtantes, for
corrupting the text, cõtrarie to the Hebrew, & Greke, which they
profeſſe to tranſlate, for the more ſhew, and mainteyning of their
peculiar opinions againſt Catholiques.  As is proued in
the \Emph{Diſcouerie of manifold corruptiõs}.  For example we ſhal put
the reader in memorie of one or two.  Gen.~4.~v.~7. whereas (God
ſpeaking to Cain) the Hebrew wordes in Grammatical conſtruction may
be tranſlated either thus: \Emph{Vnto thee alſo perteyneth the
luſt} \Sc{therof}, \Emph{& thou ſhalt haue dominion
ouer} \Sc{it}: or thus;
\MNote{Of purpoſe againſt Catholique doctrine}
\Emph{Alſo vnto thee} \Sc{his} \Emph{deſire ſhal be
ſubiect, & thou ſhalt rule ouer} \Sc{him}: though the coherẽce of
the text requireth the former, & in the Bibles printed 1552. and. 1577.
Proteſtantes did ſo tranſlate it: yet in
%%% 0010
the yeare 1579. and 1603. they tranſlate it the other way, rather
ſaying,
\MNote{Againſt free wil.}
that Abel was ſubiect to Cain,
and that Cain by Gods ordinance, had dominion ouer his brother Abel,
then that concupiſcence or luſt of ſinne is ſubiect to mans wil, or
that man hath powre of free wil, to reſiſt (by Gods grace) tentation
of ſinne.  But as we heare in a new Edition (which we haue not yet
ſene) they trãſlate it almoſt as in the firſt.
\MNote{Againſt Melchiſedechs ſacrifice.}
In like ſorte Gen.~14.~v.~18. The Hebrew
particle \Sc{Vav}, which S.~Ierom, and al Antiquitie
tranſlated \Sc{Enim} (\Sc{For}) Proteſtants wil by no meanes
admitte it, becauſe (beſides other argumentes) we proue therby
Melchiſedechs Sacrifice.  And yet themſelues tranſlate the ſame, as
S.~Ierom doth, \Emph{Gen}.~20.~\Emph{v}.~3. ſaying:
\MNote{And againſt holie Images.}
\Sc{For} \Emph{ſhe is a mans vvife}.  &c.
Againe \Emph{Gen}.~31.~\Emph{v}.~19. the Engliſh Bibles. 1552. and
1577. tranſlate \Emph{Theraphim}, \Sc{Images}. VVhich the Edition
of 1603. correcting, tranſlateth \Sc{Idoles}.  And the marginal
Annotation wel proueth, that it ought to be ſo tranſlated.

VVith this then we wil conclude moſt deare (we ſpeake to you al, that
vnderſtand our tongue,
\MNote{This Edition dedicated to al that vnderſtand Engliſh.}
whether you be of contrarie opinions in faith, or
of mundane feare participate with an other Congregation; or profeſſe
with vs the ſame Catholique Religion) to you al we preſent this worke:
dayly beſeching God Almightie, the Diuine VViſedom, Eternal Goodnes, to
create, illuminate, and repleniſh your ſpirites, with his Grace, that
you may attaine eternal Glorie.  Euerie one in his meaſure, in thoſe
manie Manſions, prepared and promiſed by our Sauiour in his Fathers
houſe.  Not only to thoſe which firſt receiued, & folowed his Diuine
doctrine, but to al that ſhould afterwardes beleue in him, & kepe the
ſame preceptes.  For there is one God, one alſo Mediatour of God and
men: Man Chriſt Ieſus.  VVho gaue himſelf a Redemption for al.  VVherby
appeareth his wil, that al ſhould be ſaued.  VVhy then are not al ſaued?
The Apoſtle addeth:
\MNote{Chriſt redemed al, but al are not ſaued.}
that they muſt firſt come to the knowlege of the truth.
\MNote{True faith firſt neceſſarie.}
Becauſe without faith it is impoſſible to pleaſe
%%% 0011
God.  This groundworke therfore of our creation in Chriſt by true faith,
S.~Paul labored moſt ſeriouſly by word and writing, to eſtabliſh in the
hartes of al men.  In this he confirmed the Romanes by his Epiſtle,
cõmending their faith, as already receiued, and renowmed in the whole
world.  He preached the ſame faith to manie Nations.  Amongſt others to
the lerned Athenians.  VVhere it ſemed to ſome, as abſurde, as ſtrange;
in ſo much that they ſcornfully called him
\CNote{\XRef{Act.~17. v.~18.}}
\Emph{a vvord-ſovver}, and Preacher of new gods.  But
\CNote{\Cite{Ser.~42. de Sanct.}}
S.~Auguſtin alloweth the terme for good,
which was reprochfully ſpoken of the ignorant.
\MNote{The twelue Apoſtles were firſt Reapers, before they were Sowers.
S.~Paul at firſt a Sower, or Seminarie Apoſtle}
And ſo diſtinguiſhing
betwen \Emph{Reapers}, and \Emph{Sovvers} in Gods Church, he teacheth,
that wheras the other Apoſtles reaped in the Iewes, that which their
Patriarches and Prophetes had ſowne; S.~Paul ſowed the ſeede of
Chriſtian Religion in the Gentiles.  And ſo in reſpect of the
Iſraelites, to whom they were firſt ſent, calleth the other
Apoſtles \Emph{Meſſores, Reapers}, and S.~Paul, being ſpecially ſent to
the Gentiles, \Emph{Seminatorem a Sovver, or Seminarie Apoſtle}.  VVhich
two ſortes of Gods workmen are ſtil in the Church,
with
\MNote{Paſtoral cures and Apoſtolical miſſions.}
diſtinct
offices of Paſtoral cures, and Apoſtolical miſſions, the one for
perpetual gouernment of Catholique countries: the other for conuerſion
of ſuch, as either haue not receiued Chriſtian Religion, or are
relapſed.  As at this time in our country, for the diuers ſortes of
pretended religions, theſe diuers ſpiritual workes are neceſſary, to
teach and feede al Britan people.  Becauſe ſome in error of opinions
preach an other Goſpel,
\MNote{New doctrine is falſly called the Goſpel.}
wheras in veritie there is no other Goſpel. They preach in dede
new doctrines, which can not ſaue.
\CNote{\Cite{S.~Aug. de vtilis. cred. c.~1.}}
Others folow them beleuing falſhood. But
\CNote{\XRef{Mat.~15.}}
\Emph{vvhen the blinde lead the blinde} (not the one only,
but) \Emph{both fal into the ditch}.
\MNote{The ſeduced, and externally conformable are puniſhed with the
authors of iniquitie.}
Others conforme themſelues, in external ſhew, fearing them that can
puniſh, and kil the bodie.  But
\CNote{\XRef{Pſalm.~124.}}
\Emph{our Lord vvil bring ſuch as
decline into} (vniuſt) \Emph{obligations, vvith them that vvorke
iniquitie}.  The Reliques and final flock of Catholiques in our country,
haue
%%% 0012
great ſadnes, and ſorow of hart; not ſo much for our owne affliction,
for that is comfortable, but for you our brethren, and kinſemen in fleſh
and bloud.  VViſhing with our owne temporal damage whatſoeuer, your
ſaluation.
\CNote{\XRef{2.~Cor.~6.}}
\MNote{Grace in the new Teſtamẽt more abundant then in the old.}
Now is the acceptable time, now are the dayes of ſaluation,
the time of Grace by Chriſt, whoſe dayes manie Kinges & Prophetes
deſired to ſee: they ſaw them
\CNote{\XRef{Luc.~10.}}
(\Emph{in ſpirite}) and reioyced.  But we
are made partakers of Chriſt, and his Myſteries; ſo that our ſelues
neglect not his heauenly riches: if we receiue & kepe the beginning of
his ſubſtance, firme vnto the end; that is, the true Catholique faith;
building theron good workes by his grace; without which we can not
thinke a good thought, by which we can do al thinges neceſſarie to
ſaluation.  But if we hold not faſt this ground, al the building
fayleth.  Or if
\CNote{\XRef{Tit.~1.}}
confeſſing to know God in wordes,
\MNote{Both wicked workes, and omiſſion of good workes are damnable.}
we denie him in deedes, committing workes of darknes; or
\CNote{\XRef{Mat.~25.}}
omitting workes of mercie, when
we may doe them to our diſtreſſed neighbors; brifly
\CNote{\XRef{1.~Cor.~13.}}
if we haue not
charitie, the forme and perfection of al vertues, al is loſt, and
nothing worth.  But if we builde vpon firme grounde, gold, ſiluer, and
precious ſtones, ſuch building ſhal abide, and make our vocation ſure by
good workes as
\CNote{\XRef{1.~Pet.~1.}}
S.~Peter ſpeaketh.  Theſe (ſaith S.~Paul) are the heyres of God,
coheyres of Chriſt. 
\CNote{\XRef{Apoc.~7.}}
\MNote{Innumerable ſaued by Chriſt.}
Neither is the number of Chriſts bleſſed children coũted, as of
the Iewes, an hundred fourtie foure thouſand, of euerie tribe of Iſrael
twelue thouſand ſigned; but a moſt great multitude of Catholique
Chriſtians, which no man can number, of al nations, and tribes, and
peoples, and tongues, ſtanding before the throne of the lambe, clothed
in white robes, and palmes (\Emph{of triumph}) in their handes: hauing
ouercome tentations in the vertuous race of good life.
\MNote{They are more happie that ſuffer perſecution for the truth.}
Much more thoſe
which alſo indure perſecution for the truthes ſake, ſhal receiue moſt
copious great rewardes in heauen.  For albeit the paſſions of this time
(\Emph{in themſelues}) are not
\SNote{VVorthie, or comparable in dignitie.}
condigne, to the glorie to come, that ſhal be reueled
%%% 0013
in vs: yet our tribulation, which preſently is momentanie, and light,
worketh
\CNote{\XRef{2.~Cor.~4.}}
(\Emph{through grace}) aboue meaſure excedingly an eternal
weight of glorie.  VVhat ſhal we therfore meditate of the eſpecial
prerogatiue of Engliſh Catholiques at this time?
\MNote{Engliſh Catholiques moſt happie in this age.}
For to you it is geuen for Chriſt,
not only that you beleue in him, but alſo that you ſuffer for him.  A
litle now, if you muſt be made penſiue in diuers tentations, that the
\CNote{\XRef{1.~Pet.~1.}}
probation of your faith, much more precious then gold, which is proued
by the fire, may be found vnto praiſe, and glorie, and honour, in the
reuelation of Ieſus Chriſt.  Manie of you haue ſuſteyned the ſpoile of
your goodes with ioy, knowing that you haue a better and a permanent
ſubſtance.  Others haue benne depriued of your children, fathers,
mothers, brothers, ſiſters, and nereſt frendes, in readie reſolution
alſo, ſome with ſentence of death, to loſe your owne liues.  Others haue
had trial of reproches, mockeries, and ſtripes.  Others of bandes,
priſons, and baniſhmentes.  The innumerable renowmed late Engliſh
Martyres, & Confeſſors, whoſe happie ſoules for confeſſing true faith
before men,
\MNote{The due praiſe of Martyres, and other glorious Sainctes excedeth
mortal tongues.}
are now moſt glorious in heauen, we
paſſe here with ſilence; becauſe their due praiſe, requiring longer
diſcourſe, yea rather Angels, then Engliſh tongues, farre ſurpaſſeth the
reach of our conceiptes.  And ſo we leaue it to your deuout meditation.
They now ſecure for themſelues, and ſolicitous for vs their deareſt
clients, inceſſantly (we are wel aſſured) intercede before Chriſts
Diuine Maieſtie, for our happie conſummation, with the conuerſion of our
whole countrie.  To you therfore (deareſt frendes mortal) we direct this
ſpeach: admoniſhing ourſelues & you, in the Apoſtles wordes, that for ſo
much
\MNote{Patience neceſſarie to the end of mans life.}
as we haue
not yet reſiſted tentations to (laſt) bloud (and death itſelf) patience
is ſtil neceſſarie for vs, that doing the wil of God, we may receiue the
promiſe.  So we repine not in tribulation, but euer loue them that hate
vs, pittying their caſe, and reioycing in our owne.  For
%%% 0014
neither can we ſee during this life,
\MNote{Perſecution profitable.}
how much good they do vs; nor know how manie of them ſhal
be (as we hartely deſire they al may be) ſaued: our Lord and Sauiour
hauing paide the ſame price by his death, for them and for vs.  Loue
al therfore, pray for al.  Do not loſe your confidence, which hath a
great remuneration.  For yet a litle, and a very litle while, he that
is to come, wil come, and he wil not ſlacke.  Now
\CNote{\XRef{Rom.~10.}}
the iuſt liueth by faith, beleeuing with hart to iuſtice,
\MNote{Confeſſion of faith before men neceſſarie to ſaluation.}
and confeſſing with mouth to ſaluation.
\CNote{\XRef{Heb.~10.}}
But he that withdraweth himſelf ſhal not pleaſe Chriſts
ſoule.  Attend to your ſaluation, deareſt countriemen.  You that are
farre of, draw nere, put on Chriſt.  And you that are within Chriſts
fold, kepe your ſtanding, perſeuere in him to the end.  His grace dwel
and remaine in you, that glorious crownes may be geuen you.  {\sc
Amen}

From the Engliſh College in Doway, the Octaues of \Sc{al
Sainctes}. 1609.

\Emph{The God of patience and comfort geue you to be of one mind, one
tovvards another in} \Sc{Iesvs} \Emph{Chriſt; that of one mind, vvith one
mouth you may glorifie God}.

\stopPreface


\stopcomponent


%%% Local Variables:
%%% mode: TeX
%%% eval: (long-s-mode)
%%% eval: (set-input-method "TeX")
%%% fill-column: 72
%%% eval: (auto-fill-mode)
%%% eval: (setq line-spacing 3)
%%% coding: utf-8-unix
%%% End:
