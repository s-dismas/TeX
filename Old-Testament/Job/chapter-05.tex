%%%%%%%%%%%%%%%%%%%%%%%%%%%%%%%%%%%%%%%%%%%%%%%%%%%%%%%%%%%%%%%%%
%%%%
%%%% The (original) Douay Rheims Bible 
%%%%
%%%% Old Testament
%%%% Job
%%%% Chapter 05
%%%%
%%%%%%%%%%%%%%%%%%%%%%%%%%%%%%%%%%%%%%%%%%%%%%%%%%%%%%%%%%%%%%%%%




\startcomponent chapter-05


\project douay-rheims


%%% 1086
%%% o-0981
\startChapter[
  title={Chapter 5}
  ]

\Summary{Eliphaz proſecuteth his diſcourſe to conuince Iob of great
  ſinnes, becauſe he is ſo vehemently afflicted, 17.~exhorteth him
  therfore to acknowlege his ſinnes, ſo al thinges shal ſuccede
  proſperouſly.}

Cal therfore
\SNote{This diſputer hauing pretended an imaginarie viſion from God
againſt Iob, now he ſuppoſeth that neither God nor Angel nor other holie
perſon wil patronage his cauſe, nor iudge of his caſe as he doth, but
that al wil condemne him of impatience, follie, enuie, and other
ſinnes.}
if there be that wil anſwer thee, &
\LNote{Turne to ſome of the Sainctes.}{Eliphaz
\MNote{Inuocation of Sainctes, eſpecially Angels in Iobs time.}
prouoking Iob to produce ſome of his opinion, or to ſeeke the helpe and
patronage of ſome Sainct in his cauſe, plainly sheweth the common faith
and practiſe of inuocating Sainctes in that time. Els it had benne a
friuolous ſpeach, which is not to be imputed to a ſenſible wiſe man as
he was. For it appeareth by the drift of his reaſoning, that he
ſuppoſed ſome of Gods ſpecial ſeruantes would maintaine a good cauſe,
but that Iobs cauſe was ſuch as neither God, nor holie Angel, nor good
man would defend, and therfore boldly prouoked him to this trial,
preſuming that he ſhould finde no ſuch patron. Neither did he wil Iob in
theſe wordes to cal vpon God only, for he could not erre ſo groſly, as
to cal God \Emph{ſome of the Sainctes} but muſt meane ſome other holie
perſon. And it is clere by the Septuaginta Interpreters, that Eliphaz
willed Iob to inuocate the Angels, ſaying: \Emph{Inuocate if anie vvil
anſvver thee, or if thou canſt behold anie of the holie Angels.}
S.~Gregorie
\CNote{\Cite{li.~5. c.~30.}}
expoundeth it to the ſame ſenſe, that Sainctes were to be inuocated in
a good cauſe, but, that Eliphaz here diſpicing, and deriding holie Iob, ſayd to
him: \Emph{Thou canſt not find Sainctes thy helpers in affliction, vvhom
thou vvouldeſt not haue thy felovves in proſperitie.}}
turne to ſome of the ſainctes. \V Anger in deede killeth the
%%% 1087
fooliſh, and enuie fleaeth the litle one. \V I haue ſeene a foole with
firme roote, and I curſed his beautie by and by. \V His children ſhal be
made far from ſaluation, and ſhal be deſtroyed in the gate, and there
ſhal be none to deliuer. \V Whoſe harueſt the hungrie ſhal eate, & the
armed ſhal take him by violence, and the thirſtie ſhal drinke his
riches. \V Nothing in the earth is done without a cauſe, and out of the
ground ſorrow shal not riſe. \V
\SNote{This prouerb importeth that a man muſt not thincke to paſſe his
life without trauel, but muſt \Emph{get his bread vvith ſvveat of his
brovves}, or ſuffer other calamities.}
Man is borne to labour, and the bird to flight. \V For the which thing I
wil beſech our Lord, and toward God I wil ſet my ſpeach. \V Who doeth
great and vnſearchable and meruelous things without number. \V Who
geueth raine vpon the face of the earth, and watereth al thinges with
waters. \V Who ſetteth the humble on high, and them that are in
heauineſſe he conforteth with health. \V Who diſſipateth the cogitations
of the malignant, that their handes can not accomplish that which they
began. \V Who apprehendeth the wiſe in their ſubteltie, and diſſipateth
the counſel of the wicked. \V By day they ſhal incurre darkeneſſe, and
as it were in the night, ſo shal they grope at noone daies. \V Moreouer
he shal ſaue the needy from the ſword of their mouth, and the poore from
the hand of the violent. \V And to the needie there shal be hope, but
iniquitie shal draw together her mouth. \V Bleſſed is the man that is
corrected of God: refuſe not therfore the chaſtiſing of our Lord. \V
Becauſe he woundeth, and cureth: ſtriketh, and his hands shal heale. \V
In
\SNote{Gods goodnes deliuereth his ſeruantes the ſpace of this laborious
life,}
ſix tribulations he shal deliuer thee, and in the
\SNote{and moſt eſpecially in the houre of death.
\Cite{S.~Greg. li.~6. c.~18.}}
ſeuenth euil shal not touch thee. \V In famine he shal deliuer thee from
death; and in battel, from the hand of the ſword. \V From the ſcourge of
the tongue thou shalt be hid; & thou shalt not feare calamitie when it
cometh. \V In waſte and famine thou shalt laugh; and the beaſtes of the
earth thou shalt not feare. \V But with the ſtones of the landes thy
couenant, and the beaſtes of the earth shal be peaceable to thee. \V And
thou shalt know that thy tabernacle hath peace, and viſiting thy
beautie, thou shalt not ſinne. \V Thou shalt know alſo that thy ſeed
shal be manifold, and thy progenie as the graſſe of the earth. \V Thou
shalt enter into the graue in abundance, as a heape of wheate is caryed
in his time. \V Behold, this is euen ſo, as we haue ſearched out: which
thou hauing heard reuolue in thy mind.


\stopChapter


\stopcomponent


%%% Local Variables:
%%% mode: TeX
%%% eval: (long-s-mode)
%%% eval: (set-input-method "TeX")
%%% fill-column: 72
%%% eval: (auto-fill-mode)
%%% coding: utf-8-unix
%%% End:
