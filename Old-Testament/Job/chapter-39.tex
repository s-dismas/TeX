%%%%%%%%%%%%%%%%%%%%%%%%%%%%%%%%%%%%%%%%%%%%%%%%%%%%%%%%%%%%%%%%%
%%%%
%%%% The (original) Douay Rheims Bible 
%%%%
%%%% Old Testament
%%%% Job
%%%% Chapter 39
%%%%
%%%%%%%%%%%%%%%%%%%%%%%%%%%%%%%%%%%%%%%%%%%%%%%%%%%%%%%%%%%%%%%%%




\startcomponent chapter-39


\project douay-rheims


%%% 1123
%%% o-1016
\startChapter[
  title={Chapter 39}
  ]

%%% !!! Here we are missing a couple of pages. So I follow other for a
%%% bit.
\Summary{God Almightie proſecuteth his diſcourſe, shewing his admirable
  power and prouidence in liuing creatures. 34.~Wherupon Iob
  acknowledgeth his owne ouerſight in ſome light words.}

Haſt
\SNote{By Gods meruelous prouidence appearing in the natural inſtinct of
other creatures, man may conſider that the ſame is greater towards him. And
therfore God here propoſeth the examples of}
thou knowen the time when the
\SNote{Wild goats,}
wild goats bring forth yong among the rocks, or haſt thou obſerued the
\SNote{Hynds,}
hynds when they fawne? \V Haſt thou numbred the months of their
conceiuing, and knowen the time of their bearing? \V They bow downe
themſelues to bring forth yong, and they caſt them, and make
roarings. \V Their yong are ſeparated, and goe to feed: they goe forth,
and returne not to them. \V Who hath diſmiſt the
\SNote{Wilde aſſes,}
wilde aſſe free, and who hath looſed his bonds? \V To whom I haue giuen
a houſe in the wildernes, and his tabernacles in the land of
ſaltneſſe. \V He contemneth the multitude of the citie, the crie of the
exactour he heareth not. \V He looketh about the mountaines of his
paſture, and ſeeketh out al green places. \V Wil the
\SNote{Vnicornes,}
Rhinoceros ſerue thee, and wil he tarie at thy ſtal? \V Shalt thou tie
the Rhinoceros with thy
%%% o-1017
coller to plough, or wil he breake the cloddes of the valleys after
thee? \V Shalt thou haue confidence in his great ſtrength, and leaue thy
labours vnto him? \V Wilt thou credit him that he wil render thee the
ſeed, and gather together thy barne floore? \V The wing of the
\SNote{Oſtriches,}
Oſtrich is like to the wings of the
\SNote{Faulcons, or Ierfaulcons, & other hauks.}
Herodius, and of the hawke. \V When ſhe leaueth her egges on the earth,
thou perhaps wilt heate them in the duſt. \V She forgetteth that foot
may tread vpon them, or beaſt of the field breake them. \V She is
hardned toward her yong, as though they were not hers, ſhe hath laboured
in vaine, no feare compelling her. \V For God hath depriued her of
wiſedome, neither hath he giuen her vnderſtanding. \V When time ſhal be,
ſhe ſetteth vp the wings on high: ſhe skorneth the horſe and his
rider. \V Shalt thou giue ſtrength to the
\SNote{Horſes are of ſingular great courage.}
horſe, or put neying about his necke? \V Shalt thou raiſe him vp as
Locuſts? the glorie of his noſthrels is terrour. \V He diggeth the earth
with his houſe, he prawnſeth boldly, he goeth forward to meet the armed
men. \V He contemneth feare, neither yealdeth he to the ſword. \V Vpon
him ſhal the quiuer ſound, the ſpeare ſhal gliſter and the ſhilde. \V
Feruent and foming he ſuppeth the earth, neither doth he make account
when the noyſe of the trumpet ſoundeth. \V When he
%%% 1124
%%% Here we pick back up with this, but the verse numbers differ!
ſhal heare the trumpet he ſayth: Vah, he ſmelleth battel far of, the
exhortation of the captaines, and the crie of the armie. \V Doth the
\SNote{Haukes wherof Ariſtotle ſaith there be ten kindes: Plinie
ſixtẽne.}
hawke waxe fethered by thy wiſedom, ſpreding her winges to the South? \V
Shal the
\SNote{Eagles, of moſt ſtrong ſight.}
eagle mount at thy commandment, and put her neſt in high places? \V She
abideth in rockes, and tarieth among cragged flintes, and ſtonie hilles
where is no acceſſe. \V Thence ſhe beholdeth the praye, and her eies ſee
a far of. \V Her yong ones ſhal licke bloud: & wherſoeuer the carcaſſe
shal be, she is preſent by and by. \V And our Lord added, and ſpake to
Iob: \V He that contendeth with God is he quieted ſo eaſily? Verely he
that reproueth God, ought to anſwer him. \V But Iob anſwering our Lord,
ſayd: \V I that haue ſpoken
\SNote{If we diſcuſſe al Iobs wordes (\Emph{ſaith S.~Gregorie}) we ſhal
find nothing wickedly ſpoken, but only ſmale ſpeece of pride in ſpeaking
too much of his owne afflictiõ and too litle of Gods goodnes towards
him.
\Cite{li.~23. c.~1.}}
lightly what can I anſwer? I wil put my hand vpon my mouth. \V One
thing I haue ſpoken, which I would I had not ſayd: and an other, to the
which thinges I wil adde no more.


\stopChapter


\stopcomponent


%%% Local Variables:
%%% mode: TeX
%%% eval: (long-s-mode)
%%% eval: (set-input-method "TeX")
%%% fill-column: 72
%%% eval: (auto-fill-mode)
%%% coding: utf-8-unix
%%% End:
