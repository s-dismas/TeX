%%%%%%%%%%%%%%%%%%%%%%%%%%%%%%%%%%%%%%%%%%%%%%%%%%%%%%%%%%%%%%%%%
%%%%
%%%% The (original) Douay Rheims Bible 
%%%%
%%%% Old Testament
%%%% Job
%%%% Argument
%%%%
%%%%%%%%%%%%%%%%%%%%%%%%%%%%%%%%%%%%%%%%%%%%%%%%%%%%%%%%%%%%%%%%%




\startcomponent argument


\project douay-rheims


%%% 1079
%%% o-0973
\startArgument[
  title={\Sc{The Argvment of the Booke of Iob.}},
  marking={Argument of the Booke of Iob.}
  ]

Holie
\MNote{Iob of the race of Eſau.}
\Emph{Iob} otherwiſe called \Emph{Iobab}
\XRef{(Gen.~16.)}
as
\CNote{\Cite{li.~18. c.~47. ciuit.}}
S.~Auguſtine,
\CNote{\Cite{conc.~2. de Lazar}}
S.~Chriſoſtom,
\CNote{\Cite{Rom.~9.}}
S.~Ambroſe,
\CNote{\Cite{Præfat. in Iob.}}
S.~Gregorie, and other fathers teach, \Emph{the ſonne of Rahuel, the
ſonne of Eſau}, was king (or abſolute prince) of the land of Hus.
\MNote{VVhy God ſuffered him to be ſo afflicted.}
Who being perfect in religion, ſincere in life, rich in wealth, and
bleſſed with children, for an admirable example of patience, and to shew
that a mortal man through Gods grace, may reſiſt al the diuels
tentations, by Gods permiſsion,
\MNote{The contentes according to the hiſtorie.}
ſudainly loſt al his goodes and children, was ſtriken with horrible
ſores in al his bodie, reuiled by his wife, and in ſteed of comforth
which his ſpecial freindes pretented towards him, was iniuriouſly
charged by them, with impatience, arrogancie, blaſphemie, and other
crimes, for which they falſly ſuppoſed he was ſo afflicted, affirming,
and by diuers ſophiſtical argumentes, grounded as they pretended vpon
Gods iuſtice, wiſdome, powre, mercie, and goodnes, would proue that God
ſuffereth none but wicked men to be ſo afflicted. But Iob conſtantly
defendeth his owne iuſt innocencie, and that worldlie calamities, and
proſperitie happen indifferently
%%% 1080
to good and bad in this life, and that the true reward of the iuſt, and
punishment of the wicked, is to be expected in the other world. At laſt
God, with due reprehenſion of Iob for ſome imperfections, sharply
rebuketh the errors, and inſolencie of his aduerſe freindes; geueth
ſentence of Iobs ſide; pardoneth them at his interceſſion; and reſtoreth
al thinges to him duble, to that he had before.

Beſides
\MNote{Iob an eſpecial figure of Chriſt.}
the literal ſenſe \Emph{Iob} in al his actions, ſufferinges, and whole
life, was \Emph{a ſpecial figure of Chriſt, shewing} (ſayth
\CNote{\Cite{Præfat.}}
S.~Gregorie) \Emph{by thoſe thinges which he did and ſuſteyned, what our
Redemer should do and ſuffer}. Yea more particularly then moſt part of
the Patriarches, which S.~Ierome
\Cite{(epiſt. ad Paulin.)}
alſo admireth and teſtifieth, ſaying: \Emph{What myſteries of Chriſt
doth not this booke comprehend? Euerie word is ful of ſenſe.}
\MNote{Moral documentes in this booke.}
Moreouer this hiſtorie is replenished with \Emph{moral documents}, how
to embrace vertue, and eſchew vice: propoſing the life of a right godlie
man, neither inſolent in proſperitie, nor deſparing in aduerſitie,
alwayes reſolute in Gods ſeruice, as wel in his proſperous kingdom as in
the miſerable dunghil.
\MNote{True logike & Philoſophie.}
Here alſo we haue the true maner of arguing, according to the rules of
Logike, with detection of ſophiſtrie, \Emph{Iob prouing and diſprouing
aſſertions by propoſition, aſſumption, and concluſion}, as S.~Ierom
obſerueth, with profound knovvlege of natural thinges
%%% o-0974
and cauſes, as appeareth in very manie places.
\MNote{Hard and eaſie thinges to be vnderſtood are both profitable.}
Al which varietie and abundance of matter, compriſed in ſmal rowme, make
manie thinges hard and obſcure, yet are the ſame ſo tempered with other
thinges plaine and eaſie, that here is verified S.~Auguſtins obſeruation
\Cite{(li.~2. c.~6. doct. Chriſt)}
\Emph{certaine places of holie Scriptures ſerue as delectable meate to
them that hunger and thirſt diuine knowlege, and the obſcure take away
tediouſnes from them, that loath vſual plaine doctrin.}

It
\MNote{VVritten by Iob himſelf, moſt part in verſe.}
is moſt probable that Iob himſelf, inſpired by the Holie Ghoſt, by whoſe
grace \Emph{he excelled al in right ſimplicitie}
\XRef{(c.~1.)}
writte his owne hiſtorie; the moſt part in verſe, only the two firſt
chapters and the laſt in proſe, in the Arabian tongue, which Moyſes
tranſlated into Hebrew, for the conſolation of the Iſraelites afflicted
in Ægypt.

And
\MNote{Diuided into three partes.}
it may be diuided into three general partes. Firſt the change of
Iobs ſtate from proſperitie into affliction, with his lamentation for the
ſame, are recorded in the three firſt chapters. In foure and thirtie
chapters folowing are ſundrie diſputations, conflictes, and diſcourſes
betwen him and his freindes, touching the cauſe of his ſo vehement
affliction. In the fiue laſt chapters God diſcuſseth the quarel, geueth
ſentence for Iob againſt his aduerſaries, pardoneth them, and rewardeth
him.


\stopArgument


\stopcomponent


%%% Local Variables:
%%% mode: TeX
%%% eval: (long-s-mode)
%%% eval: (set-input-method "TeX")
%%% fill-column: 72
%%% eval: (auto-fill-mode)
%%% coding: utf-8-unix
%%% End:
