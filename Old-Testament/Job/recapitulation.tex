%%%%%%%%%%%%%%%%%%%%%%%%%%%%%%%%%%%%%%%%%%%%%%%%%%%%%%%%%%%%%%%%%
%%%%
%%%% The (original) Douay Rheims Bible 
%%%%
%%%% Old Testament
%%%% Job
%%%% Recapitulation
%%%%
%%%%%%%%%%%%%%%%%%%%%%%%%%%%%%%%%%%%%%%%%%%%%%%%%%%%%%%%%%%%%%%%%




\startcomponent recapitulation


\project douay-rheims


%%% 1128
%%% o-1022
\startArgument[
  title={\Sc{A Brief Recapitvlation} of this Strange and Sacred Hiſtorie.},
  marking={A Brief Recapitvlation.}
  ]

For
\MNote{VVhy we haue made few annotations in this booke.}
auoiding prolixitie (this volume growing great) we haue for moſt part
contracted our Annotations into the margen, making very few others in
this booke, which otherwiſe offered much more occaſion, as wel for
explicating hard places, as of other doctrinal and moral
inſtructions. Neither in dede can ordinarie Annotations wel ſuffice for
vnderſtanding of this, and other hard bookes. But rather large
\Fix{Comentaries}{Commentaries}{possible typo, fixed in other}
are required, ſuch as \Emph{S.~Gregorie}, and other ancient Fathers: as
alſo \Emph{F.~Iohn de Pineda}, and others haue lately written. Wherto we
therfore remitte the lerned readers.
\MNote{The argument of this hiſtorie.}
And for the benefite of others of our nation, we shal here briefly
recapitulate the ſumme, and principal pointes of this holie and
admirable hiſtorie, conſiſting in a ſingular holie mans conflictes, and
glorious victorie, againſt inuiſible and viſible aduerſaries, both in
proſperous and aduerſe fortune.

Firſt
\MNote{Iob in proſperitie was tempted inuiſibly more then ordinary men
of lower ſtate, or leſſe perfection.}
this holie man \Emph{Iob in al abundance of wealth}, and riches,
\Emph{bleſſed with} manie \Emph{children}
\XRef{(ch.~1.)}
ſitting in a princelie throne, and \Emph{royal dignitie}
\XRef{(ch.~29.)}
\Emph{in the land of Hus}, was not only \Emph{aſſaulted} with common
tentations of the enuious enimie,
\CNote{\XRef{2.~Tim.~3.}}
as al are that liue piouſly in God, but ſo \Emph{much the more}, as he
was more godlie, ſincerer and perfecter then \Emph{other men}, yet he
\Emph{neuer ſet his hart vpon worldlie} or temporal \Emph{thinges}, but
with al due feare ſo \Emph{ſerued God, and parted from euil}, that
\Emph{Satan} himſelf (the calumniator of mankind) could not charge him
with anie ſinne at al. Though he would not for al that confeſſe him to
be iuſt, or perfect: but for further trial of him, demanded and obtained
licence of God to touch al his poſſeſions,
\MNote{Much more by loſſe of al his goodes and children in one day.}
and ſo \Emph{bereued him of al his goodes, & children} in one day. And
when he
%%% 1129
\Emph{perſeuering} conſtant \Emph{in vertue, thanking God for al, not
ſinning in his lippes, neither ſpeaking anie fooliſh thing againſt God},
the diuel getting more ample permiſſion to touch his bones and flesh
\XRef{(chap.~2.)}
\MNote{Moſt of al by bodilie affliction.}
\Emph{ſudenly ſtroock him with moſt grieuous botch} (or boyle)
\Emph{from the ſole of the foote to the toppe of the head: who ſitting
on a dunghil, and ſcraping the corruption of his ſores with a ſhel} in
extreme paine,
\MNote{And reuiling of his wife.}
\Emph{his} owne \Emph{wife}, by the diuels ſuggeſtion, \Emph{reuiled
him} for his ſinceritie, \Emph{and prouoked him to blaſpheme God}: but
he ſeuerely reprehended her follie, ſtil keeping neceſſarie patience.

Then came \Emph{three} ſpecial \Emph{freindes}, noble wiſemen (or litle
kinges) to viſite, and comforte him, who \Emph{in ſeuen dayes not
ſpeaking one word of
%%% o-1023
conſolation}, nor entering into anie diſcourſe with him,
\MNote{Holie Iob lamented his affliction, and the general miſeries of
man.}
at laſt \Emph{Iob} himſelf
\XRef{(chap.~3.)}
broke this long ſilence (but not his patience) \Emph{lamentably
bewayling the extremitie of his paines}, imputing al to the miſeries of
mans eſtate, corrupted by ſinne, \Emph{diſcourſed of certaine penal
euiles}, or
\Fix{malades}{maladies}{obvious typo, fixed in other}
enſuing therupon, \Emph{wishing} for his owne part (if it had ſo bene
Gods pleaſure, for he ſincerly feared God) \Emph{that either he had not
bene borne, or bene shortly taken out of
\Fix{his}{this}{obvious typo, fixed in other}
world, curſing ſinne and the proper effectes therof} remaining in
man, \Emph{wishing} alſo \Emph{to haue wanted the ordinarie benefites
of} education in his infancie, and al his former proſperitie, ſo that he
might haue eſcaped the calamities, wherwith he was now afflicted.
\MNote{VVhere Iob expected comforth in tribulation, the diuel procured
him more affliction.}
Al which he vttering in way of contemning al worldlie thinges, and
ſuppoſing \Emph{his freindes} there preſent, would haue ſo vnderſtood
him, and had compaſſion with him: they contrariwiſe (by art of the
diuel, God ſo permitting) \Emph{fel into indignation}, & in ſtead of
comforting their moſt afflicted freind, \Emph{sharply reprehended him},
rashly iudged his conſcience, and falſly \Emph{condemned him}, not only
\Emph{of impatient ſpeach}, as offenſiue to God, and his Angels, and to
al good men; but alſo of other enormious ſinnes: as \Emph{pride,
tyrannie, preſumption, hypochriſie, and blaſphemie}, becauſe heretofore
he ſemed to the world as iuſt and holie, and now (as they imagined) in
his deſerued punishment, charged God with iniuſtice.
\MNote{Iob ſore afflicted in bodie had nine ſeueral
\Fix{confflictes}{conflictes}{likely typo, same in both}
about the cauſe therof before it was decided.}
Wherupon grew diuers long diſputes betwen Iob and his three freindes; a
fourth alſo intruding himſelfe, when the others ceaſed. So that
\Emph{Iob indured nine conflictes}, and \Emph{in the tenth God iudged
him the victour}, which shal yet better appeare, if we repete the ſumme
of their argumentes, & his anſwers, with Gods deciſion of the
controuerſie.

In
\MNote{The firſt conflict.}
the firſt conflict \Emph{Eliphaz} the chiefeſt of Iobs freindes (in the
\XRef{4.~and 5.~chapter})
\Emph{accuſed Iob of great impatience, and inſolencie} againſt God, alſo
both \Emph{him and his familie of tyrannie, like to a cruel lion, and
lions whelpes}, alleaging for proofe the proſperitie of good men,
punishment of the wicked, and a particular viſion. Adiured him therfore
to
%%% 1130
acknowlege and repent the ſame.
\MNote{The maine point of the controuerſie.}
But \Emph{Iob} (in
\XRef{other two chapters})
\Emph{auouched that in deede his afflictions were greater then his
ſinnes deſerued}, relying vpon his innocencie knowen to his owne
conſcience. Deſcribed alſo the manifold calamities of mans life, deſired
to die, and ſo to end his worldlie miſeries.

Then
\MNote{The ſecond conflict.}
\Emph{Baldad} the ſecond oppoſite freind (in the
\XRef{8.~chap.})
\Emph{pretending to free Gods iudgement} from al shew or reſemblance of
iniuſtice, \Emph{charged Iob and his children with former wickednes},
and him as iniurious to God in his ſpeaches, of which if he would
repent, he should be healed, and proſper as before:
\MNote{The ground of theſe mens error.}
\Emph{Arguing} in general, \Emph{that God neuer afflicteth the innocent,
nor aſſiſteth the malignant}. Inſinuating therby, that Iob was an
hypocrite. Wherto Iob anſwered
\XRef{(chap.~9.~&.~10.)}
that in dede \Emph{no man may compare, nor iuſtifie himſelfe before
God}. Neuertheles it ſtandeth wil with Gods iuſtice, powre, & wiſdome,
that innocentes be ſometimes exerciſed with tribulations, more then
their offences deſerue.

Thirdly
\MNote{The third conflict.}
\Emph{Sophar} (the third diſputer) \Emph{aſſaulted Iob}
\XRef{(ch.~11.)}
\Emph{imputing his ſpeach, and defence of himſelf to loquacitie, and
audacious temeritie},
%%% o-1024
in that he deſired to know the cauſes of Gods prouidence, in ſo
grieuouſly afflicting him. Of which faultes holie \Emph{Iob purged
himſelf} (in the 
\XRef{three next chapters})
ſtil maintayning his innocencie, according to his owne conſcience,
better knowen to himſelf then to them, deſiring God to inſtruct him, if
he had anie vnknowen ſinnes. Diſcourſed alſo much more profoundly of
Gods powre, wiſdome, iuſtice, and prouidence, as wel in general, as
towards himſelf in particular: and profeſſed his faith, and great
confidence of the Reſurrection.

Againe
\MNote{The fourth conflict.}
\Emph{Eliphaz}
\XRef{(ch.~15.)}
\Emph{more bitterly} then before, \Emph{condemned Iob of preſumption,
and blaſphemie}, diſcourſed of mans corruptnes and
\Fix{prones}{pronenes}{obvious typo, fixed in other}
to ſinne, deſcribing the maners of hypochrites, and other impious men,
with their miſerable endes, and argued \Emph{Iob} for ſuch a one. VVho
(in the 
\XRef{next two chapters})
\Emph{expoſtulated with theſe his freinds}, that they coming with pretence
to comforth him, did ſo violently afflict him, by charging him with
falſe and heynous crimes, his owne conſcience better knowing, and
teſtifying his former life, and ſtate of his ſoule, then that their
imaginations could alter his iudgement. And ſo \Emph{with contempt of
this world}, & deſire of death and reſt, \Emph{appealed to Gods
iudgement againſt his three freindes}, touching the matter in
controuerſie. In the meane time comforted himſelf with meditation of the
next world.

\Emph{Baldad}
\MNote{The fifth conflict.}
likewiſe \Emph{replied}
\XRef{(ch.~18.)}
\Emph{with hote contention}, accuſing Iob of inſolent impatience,
inculcating the greuous punishmentes both of him, and others for their
impietie. In anſwer wherto \Emph{he lamented againe the want of expected
comforth}, eſpecially by ſuch freindes. Stil comforted himſelf with
aſſured faith of the Reſurrection.

%%% 1131
\Emph{Sophar}
\MNote{The ſixth conflict.}
alſo
\XRef{(ch.~10.)}
\Emph{attempted againe to cõuince Iob of impietie, and hypochriſie}, by
the miſerable, and ſpeedie fal of wicked men after proſperitie: for ſo
he imagined Iob to be fallen into irrecouerable miſerie. But \Emph{Iob
ſhewed} the contrarie, \Emph{that ſome wicked men proſper long}, yea al
their life, and the ſame long, and then \Emph{in a moment goe downe to
hel}, and ſo the arguement of preſent affliction proued not their opinion
againſt him.

\Emph{Eliphaz}
\MNote{The ſeuenth conflict.}
\Emph{diſputed} the third time
\XRef{(ch.~22.)}
\Emph{contending that the cauſes of affliction, are not to be attributed
to Gods ſecrete prouidence}, but to aſſured ſinnes of the wicked. Vpon
whom only he ſuppoſed, that afflictions fal: inferring that Iob was
guiltie of enormious crimes, & groſſe errors. Vrged him therfore to
returne to God, that he might be reſtored to former proſperitie.
\Emph{Iob againe appealed to Gods ſentence}, not in his terrour, nor
rigour of his iuſtice, but againſt his aduerſaries in this quarel,
\Emph{deſcribing Gods powre, and wiſdome, by which he permitteth the
innocent to be afflicted, & the wicked to proſper}: no man knowing how
ſoone, or how late, al shal receiue as they deſerue.

Moreouer
\MNote{The eight conflict.}
\Emph{Baldad diſputed} the third time, very briefly
\XRef{(ch.~25.)}
\Emph{endeuouring to terrifie Iob from further anſwering}, and
eſpecially \Emph{from appealing to Gods iudgement}. But \Emph{Iob} very
largely (in 
\XRef{ſix enſuing chapters)}
\Emph{diſcourſed diuinely of Gods ſouereigne Maieſtie, Powre, Wiſdom,
exact Iuſtice, and infinite Mercie}. Alſo of wicked mens deſtruction;
\Emph{of his owne former
%%% o-1025
proſperitie, and preſent calamitie}, together \Emph{with his good
workes}, and innocencie, which he ſtil auouched in reſpect of great
iniquities.

After
\MNote{The ninth conflict.}
that Iob and his three freindes ceaſſed, nothing being agreed vpon in
the point of controuerſie, the diuel yet ceaſed not, but ſturred vp a
yongman, called \Emph{Eliu, proud and arrogant}, but not vnlerned, who
\Emph{abruptly condemned them al}; to witte, Iob of pertinacie, the
others of inſufficiencie. And therfore tooke vpon him to conuince Iob,
though the others could not.
\MNote{Neweſt Sectaries hold themſelues the wiſeſt.}
Very like to late-riſing \Emph{Proteſtantes, or Puritaines bragging that
by new argumentes}, and proofes neuer heard of, \Emph{they wil ouerthrow
the Papiſtes}, or Catholique Romaine Church, and doctrin, which al
former enimies, \Emph{Iewes, Pagaines, Turkes, and Heretikes, nor Hel
gates, could not ouercome}.
\MNote{Eſpecially theſe of our dayes, that relie ech one vpon his owne
priuate ſpirite.}
This \Emph{yong Eliu} therfore, \Emph{with his Priuate ſpirite}, wiſer
in his owne conceipt then al that went before him, \Emph{aſſaulted}
conſtant \Emph{Iob}
\XRef{(ch.~32. and fiue more enſuing)}
with manie wordes, and bragges, often chalenging & prouoking, but not extorting anie
anſwer, from ſo graue a man to his friuolous and idle argumentes,
largely diſcourſing of thinges either not denied, or ſo manifeſt falſe,
that euerie meane ſeruant of God, could eaſely conuince them, and neuer
approching to the maine controuerſie, only railed againſt holie Iob,
\Emph{charging him more} furiouſly then anie had donne before,
%%% 1132
\Emph{with impietie, impatience, ignorance, pride, blaſphemie}, and
obſtinacie, vices farre from Iobs ſanctitie, dilating alſo of Gods
iuſtice, mercie, wiſdome, powre, and prouidence; and that no man ought
to contend, nor expoſtulate with God, that afflictions muſt be borne
patiently, and that God is iuſt, and maruelous in his workes, wherof no
wiſeman euer doubted; and ſo Iob conuinced him with ſilence.

But
\MNote{In the tenth place God decided the controuerſie.}
\Emph{God himſelf} for deciſion of al
\XRef{(from ch.~38. to the end of the Booke.)}
firſt by way of examining \Emph{inſtructed Iob} more particularly,
reciting manie maruelous workes of nature, shewing therby his Diuine
Maieſtie, Powre, and Wiſdome, \Emph{exerciſing Iob in more patience},
and withal \Emph{perfecting him in humilitie}. So that with al reuerent
feare and ſubiection, he offered and ſubmitted him ſelfe to Gods onlie
good pleaſure. Then finally \Emph{God gaue ſentence that Iob had
defended the truth, & his three freindes had erred}.
\MNote{Penitentes pardoned.}
VVhom after Sacrifice, and Iobs prayer to them, he pardoned;
\MNote{Iob rewarded.}
reſtored Iob to health, and to duble proſperitie, of al he had loſt
before, geuing him alſo long life, and a happie end.

In
\MNote{The literal ſenſe of this hiſtorie.}
this hiſtorie beſides the literal ſenſe, shewing that Iob was iuſt and
ſincere, and not for his ſinnes (as his freindes falſly ſuppoſed) but
for his more merite was moſt extremly afflicted, and afterwards reſtored
to health and wealth:
\MNote{Allegorical.}
we haue alſo here in the Allegorical ſenſe, \Emph{an eſpecial figure of
Chriſt}. Who as he was abſolutly moſt innocent, & moſt perfect: ſo was
he without cõpariſon moſt afflicted of al mankind. 
\MNote{Anagogical.}
Likewiſe Iobs reſtauration to better ſtate then before, ſignified in the
Anagogical ſenſe, \Emph{the Reſurrection}, and reſtauratiõ of better, &
moſt glorious qualities in the bleſſed, \Emph{with fulnes of daies, in
eternal glorie}.
\MNote{Moral.}
Finally in the Moral ſenſe (which S.~Gregorie moſt
%%% o-1026
eſpecially proſecuteth) al Chriſtians haue here \Emph{a moſt notable
example of} al vertues, namely \Emph{of patience}, wherin Iob proceded
by degrees to great perfection.
\MNote{Holie Iob proceded by degrees to perfect patience.}
For he was firſt tried by the loſſe of al his goodes & children, and was
proued to be very patient. He was then moſt greuouſly tormented in
bodie, & being left without comforth, albeit he lamentably bewailed ſo
great extremitie, wishing ſuch dayes had bene preuented, yet he neither
ſpoke againſt God, nor good men, nor his owne ſoule, & according to
truth auouched & defended his owne innocencie. And at laſt by Gods
inſpiration, and ſwete conſolation, he reprehended himſelf, of former
imperfections vttered in ſome wordes, and with ful reſignation to Gods
wil, ſuſteyned al his loſſes and paynes, not only with contentment, but
alſo with ioy.


\stopArgument


\stopcomponent


%%% Local Variables:
%%% mode: TeX
%%% eval: (long-s-mode)
%%% eval: (set-input-method "TeX")
%%% fill-column: 72
%%% eval: (auto-fill-mode)
%%% coding: utf-8-unix
%%% End:
