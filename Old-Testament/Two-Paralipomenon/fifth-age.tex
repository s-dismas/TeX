%%%%%%%%%%%%%%%%%%%%%%%%%%%%%%%%%%%%%%%%%%%%%%%%%%%%%%%%%%%%%%%%%
%%%%
%%%% The (original) Douay Rheims Bible 
%%%%
%%%% Old Testament
%%%% Two Paralipomenon
%%%% fifth age
%%%%
%%%%%%%%%%%%%%%%%%%%%%%%%%%%%%%%%%%%%%%%%%%%%%%%%%%%%%%%%%%%%%%%%




\startcomponent fifth-age


\project douay-rheims


%%% 0954
%%% o-0856
\startArgument[
  title={\Sc{The Continvance of the Chvrch and Religion in the Fifth
  Age}: From the fundation of the Temple, to the captiuitie in
  Babylon. The space of 430.~yeares.},
  marking={The Fifth Age}
  ]

Albeit there were greater Schiſmes, Hereſies, and more reuoltes from Gods
law and ſeruice in this fifth age, then in the former:
\MNote{The Church ſtil viſible, and the ſame faith as before.}
Yet the true Church and Religion continued ſtil, and were no leſſe
conſpicuous then before. VVhich being clere and euident, touching manie
principal Articles, we wil here only remitte the reader to ſome ſpecial
places, for confirmation therof: neither wil we be prolixe, in declaring
other pointes denied, or called into controuerſie at this time, by the
impugners of Catholique Religion.

\Emph{Beleefe
\MNote{One God.}
in one God} appeareth plainly in building, adorning, & dedicating the
Temple with ſo great ſolemnitie of the Prieſtes, Leuites, and al the
Tribes, and particularly by king Salomons prayer.
\XRef{3.~Reg.~7. &~8.}
\XRef{2.~Paral.~2.~&c.}
Alſo
\XRef{Prouerb.~8.}
\XRef{Eccle.~12.}
\XRef{Iſaie.~41.}
\XRef{44.}
\XRef{45.}
The Myſterie of the
\MNote{Three Perſons.}
\Emph{B.~Trinitie},
\XRef{Prouer.~12.}
\XRef{Iſaiæ.~6.}
\XRef{48.}
\XRef{49.}
\XRef{Oſe.~11.}
\XRef{Ioel.~2.}
\MNote{Chriſt.}
\Emph{Of Chriſt our Redemer.}
\XRef{Iſaie.~7.~8.~9.}
\XRef{28.}
\XRef{53.}
\XRef{Ierem.~23.}
\XRef{30.}
\XRef{33.}
\XRef{Ezech.~17.}
\XRef{34.}
\XRef{37.}
\XRef{Dan.~7.}
\XRef{9.}
\XRef{Oſee.~6.}
\XRef{11.}
\XRef{14.}
\XRef{Ioel.~2.}
\XRef{Sophon.~2.}
\XRef{Aggai.~2.}
\XRef{Zachar.~2.~&c.}
\Emph{Sacrifices, Sacramentes,} & other Rites the ſame as before.
\MNote{Sacrifices, Sacramentes to be changed by Chriſt.}
But more frequent Prophecies, that they \Emph{ſhould be changed into
better}, and perfecter \Emph{by Chriſt}.
\XRef{Prou.~9.}
\XRef{Iſai.~12.}
\XRef{52.}
\XRef{55.}
\XRef{61.}
In the meane time for more ſignification of the \Emph{ſingular vertue of
Chriſts Sacramentes}, the effect of \Emph{penitential workes} is often
recorded.
\MNote{Fruict of penance.}
For example, wicked \Emph{Achab by hairecloth, faſting, and} other
\Emph{humiliation eſcaped} part of his deſerued \Emph{puniſhment}.
\XRef{3.~Reg.~21.}
\Emph{Manaſſes recouered Gods fauoure}, and \Emph{his temporal
kingdom}.
\XRef{2.~Par.~33.}
VVho yet was punished in his poſteritie.
\XRef{4.~Reg.~23.}
And the \Emph{Niniuites} by ſuch penance \Emph{auoided deſtruction}.
\XRef{Ione.~3.}
Yea nothing is more frequent in the Prophetes then preaching of
penance.
\XRef{Iſa.~1.~2.~3.}
\XRef{30.}
\XRef{Iere.~3.}
\XRef{18.~&c.}
and others, al aſcribing the cauſe of plagues, and afflictions to the
want of repentance. And \Emph{falſe Prophetes} condemned of errour and
falſe doctrine, \Emph{for promiſing the people peace}, and ſecuritie
\Emph{in their ſinnes}.
\XRef{Ierem.~14.}
\XRef{Lamen.~2.}
Beſides 
\MNote{Abſtinence.}
\Emph{abſtinence} from diuers ſortes of meates, counted vncleane
\XRef{(Iſaiæ.~66.)}
and
\MNote{Faſtes.}
\Emph{ordinarie faſtes}, according to the law, \Emph{other faſtes} were
\Emph{appointed} ſometimes, vpon occaſions requiring, not only to
ſubdue, and mortifie the flesh, but alſo \Emph{to obtaine mercie} at
Gods handes in ſpecial diſtreſſes.
\XRef{2.~Par.~20.}
\XRef{Ioel.~1.~2.}
\XRef{Ione.~3.}
\Emph{Elias faſting fourtie
%%% o-0857
dayes},
\XRef{3.~Reg.~19.}
prefigured \Emph{Chriſts faſt}.
\MNote{Lent.}
VVhich the Church imitateth \Emph{in Lent of fourtie daies}, according
to the humane habilitie, for the faſtes of Chriſt, Elias, and Moyſes
were miraculous.

To
\MNote{Feaſtes.}
the Feaſtes inſtituted before, was added the \Emph{Dedication of the
%%% 0955
Temple}.
\XRef{3.~Reg.~7.}
\XRef{2.~Par.~3.}
\Emph{Which was built in Mount Moria},
\XRef{2.~Par.~3.}
\MNote{Place of the Temple deſigned long before.}
the ſpecial place \Emph{deſigned long before} for this purpoſe, when
Abraham was directed thither by God, & was there readie to ſacrifice his
ſonne Iſaac,
\XRef{Gen.~22.}
where Dauid alſo offered ſacrifice.
\XRef{2.~Reg.~24.}
\XRef{1.~Par.~21.}

This being the \Emph{onlie ordinarie place for Sacrifice}, there were
for other vſes of daylie prayer reading, preaching, and hearing the word
of God other
\MNote{Synagogues.}
\Emph{Synagogues built} (as it were Parish churches) in great number: in
Ieruſalem it ſelf foure hundred and foure ſcore, and manie more in the
whole kingdom, as the Hebrew Traditions teſtifie.
\MNote{Sanctuarie.}
Of al which places, eſpecially \Emph{of the Temple}, there was
\Emph{venerable reſpect} had. For which cauſe when Ioiada the High
Prieſt gaue order to kil Athalia, he ſuffered it not to be donne in the
Temple, but commanded firſt to draw her forth.
\XRef{4.~Reg.~11.}
\XRef{2.~Par.~23.}
And as peculiar places, ſo
\MNote{Sette forme of prayers.}
\Emph{ſpecial Pſalmes}, and \Emph{Hymnes} were \Emph{appointed} for
diuers purpoſes and occaſions.
\XRef{2.~Par.~20.}

The
\MNote{Miniſterie of Angeles.}
\Emph{miniſterie of Angeles} was very vſual in this time. One was
\Emph{ſent to} comforte and direct \Emph{Elias} the prophet in his
afflictions.
\XRef{3.~Reg.~19.}
\XRef{4.~Reg.~1.}
\Emph{An Angel ſtroke the Aſſyrians} whole campe,
\XRef{4.~Reg.~19.}
\XRef{2.~Par.~32.}
Alſo the \Emph{Interceſſion of Angels is} ſo euident,
\XRef{Tobiæ~12.}
\Emph{Raphael offering Tobias prayer to God}, that Proteſtants haue no
other refuge to auoide this point of faith, but by denying the Booke to
be Canonical Scripture.

\Emph{Honour
\MNote{Honour and Interceſſion of Sainctes.}
of} other \Emph{Sainctes, and} their \Emph{Interceſſion is}
proued \L{a Maiori}. For ſo much as honour was religiouſly exhibited to
ſpiritual power and excellencie, in men yet liuing in this world. So
\Emph{a Noble man adored Elias} the Prophet, being farre greater then he
in ciuil, and worldlie reſpectes.
\XRef{3.~Reg.~18.}
\Emph{Eliſeus} alſo was \Emph{adored by his diſciples}, not for anie
worldlie authoritie or eminence, but \Emph{for his ſpiritual power and
ſuperioritie} amongſt them.
\XRef{4.~Reg.~2.}
Likewiſe al \Emph{Prophetes}, and \Emph{Prieſtes} were \Emph{religiouſly
honored for their holie and ſpiritual functions}.
\XRef{3.~Reg.~13.}
Much more Sainctes are rightly honored being immortal, and in eternal
glorie. It appeareth alſo that \Emph{Elias, ſeuen} yeares \Emph{after}
that \Emph{he was tranſlated} from humane conuerſation (when Eliſeus was
chiefe Prophet
\XRef{4.~Reg.~3.}
which was in or before the eightenth yeare of Ioſaphat, who reigned fiue
and twentie,
\XRef{3.~Reg.~22.)}
\Emph{had care of Ioram}, and his kingdom, \Emph{admoniſhing him by
letters} of Gods wrath, againſt him and his people for their ſinnes.
\XRef{2.~Par.~21.}
And the Scripture ſaieth often, that \Emph{God ſpared and protected
Ieruſalem, and the kingdom of Iuda for Dauids ſake}.
\XRef{3.~Reg.~11.}
\XRef{15.}
\XRef{4.~Reg.~8.}
\XRef{19.}
\XRef{20.}
\XRef{2.~Par.~6.}
\XRef{21.}
\XRef{Iſa.~37.}
\MNote{Reliques.}
We haue alſo example of \Emph{Sainctes Reliques} in the \Emph{cloke of
Elias},
\XRef{4.~Reg.~2.}
in \Emph{Eliſeus bones},
\XRef{4.~Reg.~13.}
and in \Emph{an other Prophetes bodie buried} in Bethel. VVhich Ioſias
would not ſuffer to be touched.
\XRef{4.~Reg.~23.}
\MNote{Images.}
\Emph{Images} were conſerued \Emph{in the Temple},
\XRef{3.~Reg.~7.}
as before in the
%%% o-0858
Tabernacle: when idolatrie was moſt deſtroyed.
%%% 0956
\XRef{3.~Reg.~15.}
\XRef{4.~Reg.~23.}
Yea an abuſe riſing of the \Emph{braſen ſerpẽt}, for which Ezechias
\Emph{deſtroyed it}
\XRef{4.~Reg.~18.}
yet he touched not the images of Cherubins in the Temple. VVhich none
but Infideles ſought to deſtroy. And \Emph{Oſee} the Prophet
\XRef{(ch.~3.)}
\Emph{bewayleth the want of Theraphim} or \Emph{Images}, amongſt other
ſacred thinges, Sacrifice, Altar, and Ephod. VVherby the ancient Rabbins proue very
wel, that \Emph{Images of Angels} (and the ſame of other Sainctes) are
\Emph{not contrarie to the Decalogue}, but the images of Idoles.
\MNote{Good workes meritorious.}
\Emph{Good workes} were \Emph{rewarded}, and \Emph{bad puniſhed},
\XRef{3.~Reg.~9.}
and the whole hiſtorie of this age teſtifieth the ſame. VVhere by the
way may be obſerued, that ſome \Emph{iuſt men fel from their iuſtice},
as Salomon
\XRef{1.~Par.~28.}
\XRef{3.~Reg.~11.}
\XRef{Ioas.}
\XRef{4.~Reg.~12.}
\XRef{2.~Par.~24.}
\XRef{Ozias.}
\XRef{2.~Paral.~26.}
\Emph{Others from wickednes returned to pietie}, as Manaſſes
\XRef{4.~Reg.~23.}
\XRef{2.~Par.~33.}
the multitude of the people very often much folowing the diſpoſition of
their kinges.
\MNote{Euangelical counſelles prefigured.}
\Emph{Special State of life} not commanded by the law,
was \Emph{voluntarily profeſſed}, and obſerued by ſome Prophetes, and
their diſciples, called \Emph{the children of Prophetes}: Keping
\Emph{particular Rules}, and wearing \Emph{diſtinct habite}.
\XRef{4.~Reg.~1.}
\XRef{2.}
\XRef{4.}
The orders of \Emph{Nazarites}, and \Emph{Rechabites} inſtituted before,
\Emph{continued} ſtil.
\XRef{Amos.~2.}
\XRef{Ierem.~35.}
\Fix{ſtil}{Al}{obvious typo, fixed in other}
which were very examplar \Emph{figures of Religious State, and Orders in
the new Teſtament},
\MNote{Chaſtitie of clergie men, & religious orders.}
and perpetual chaſtitie of clergie men embraced by ſuch, as folow
\Emph{Euangelical counſailes} propoſed, and not commanded by our
Sauiour. To which
\CNote{\XRef{Mat.~19.}
\XRef{1.~Cor.~7.}
\XRef{Act.~5.}
\XRef{1.~Tim.~5.}}
S.~Paul likewiſe exhorteth, though there be no precept therof to anie,
before they bind themſelues.

\Emph{Exequies
\MNote{Solemne Exequies for the dead.}
for the dead} were continually kept, as the ſacred hiſtorie witneſſeth,
recording where and with what
\Fix{ſolennitie}{ſolemnitie}{obvious typo, fixed in other}
the kinges were buried, which
would be ouerlong, & nedeles to recite: the like is alſo writen of ſome
Prophetes.
\XRef{3.~Reg.~13.}
\XRef{4.~Reg.~23.}
Holie \Emph{Tobias} by example, and fatherlie admonition \Emph{exhorted}
his ſonne, \Emph{to do workes of mercie}, not only to the liuing, but
alſo \Emph{to the dead. Put thy bread, and thy wine vpon the ſepulcure
of the iuſt},
\XRef{c.~4.}
\XRef{Iſaias.~ch.~57.}
as the Iewes both vnderſtood and practiſed,
\CNote{\XRef{Gen.~5.}}
\Emph{prayed, that peace be geuen to the iuſt, in his couch}, or reſting
place after his death.
\MNote{Reſurrection.}
\Emph{Of the general Reſurrection}, Elias tranſlation is a figure, who
yet liuing sheweth, that God can and wil reſtore al men to life againe
in their bodies, after death, as he conſerueth him, and Enoch in their
mortal bodies without corruption. Ezechiel alſo \Emph{prophecieth of the
Reſurrection} of the dead, applying it myſtically to ſpiritual
reſurrection, and reſtauration of Iſrael to former ſtate.
\XRef{ch.~37.}
\MNote{Iudgement.}
Of the \Emph{laſt Iudgement},
\MNote{Eternal glorie or paine.}
and \Emph{eternal glorie to} the good, and euerlaſting \Emph{paine} to
the wicked, Salomon agreably to the doctrin of other Prophetes,
diſcourſeth in his booke of Eccleſiaſtes, namely
\XRef{ch.~3.}
\XRef{11.}
and in
\XRef{the laſt}
concludeth that. \Emph{Let vs al together heare the end of ſpeaking:
Feare God, and obſerue his commandmentes:
%%% 0957
for this is euerie man} (or, to this end man is created) \Emph{and God wil
bring into Iudgement al thinges, that are done, for euerie errour} (or
obſcure thing) \Emph{whether it be good or euil.}

Neither
\MNote{Church without interruption.}
were theſe and other \Emph{pointes of Faith and Religion} interrupted,
%%% o-0859
but \Emph{ſtil beleued and profeſſed} in the \Emph{Church alwaies
viſible and incontaminate}, notwithſtanding \Emph{ſome boughes} and
branches became
%%% !!! vnſtruitful in other
\Fix{vnſtructful,}{vnfruictful,}{likely typo, same in both}
and rotten: others \Emph{brake of and were ſeparated} from this
vine. For when \Emph{Salomon falling} to luxurie, multiplying manie
wiues and concubines, was by them ſeduced and brought \Emph{to ſpiritual
fornication}, and idolatrie, \Emph{making altars, & offering ſacrifices
to Idoles, the Prieſtes, Prophetes, and people generally perſeuered in
Gods law} & ſeruice.
\XRef{3.~Reg.~11.}
\MNote{Ieraboams wicked policie.}
After whoſe death \Emph{Ieroboam} his ſeruant, of the tribe of Ephraim,
\Emph{poſſeſſing Tenne Tribes} (called the kingdom of Iſrael) to
maintaine his new ſtate, fearing that if the people reſorted to
Ieruſalem, for religions ſake, they would depart from him, and returne
to the right heyres of Dauid and Salomon, \Emph{made an egregious
Schiſme; ſetting vp two golden calues in Bethel, and Dan},
\XRef{3.~Reg.~12.}
\Emph{made temples, altares, and prieſtes} to ſerue them, al oppoſite to
Gods ordinance. But not only the other \Emph{Two Tribes}, called the
kingdom of Iuda, but alſo \Emph{the greateſt part of Iſrael}, eſpecially
\Emph{Prieſtes, Leuites}, and \Emph{deuouteſt people, repayred ſtil to
Ieruſalem, not yelding to} that \Emph{ſchiſme} and idolatrie.
\XRef{2.~Par.~11.}
Moreouer God raiſed vp and \Emph{ſent} ſpecial \Emph{Prophetes}, to
confirme the weake and recal the ſeduced.

For
\MNote{Prophets inſpired by God to reſiſt Schiſme and Hereſie.}
Ieroboam had no ſowner ſette vp his new altar in Bethel, and begunne to
offer incenſe vpon it, but
\CNote{\XRef{4.~Reg.~23.}}
\Emph{a Prophet came out of Iuda, in the word of our Lord}: and cried
againſt that altar, foretelling that wheras for that preſent, they burnt
frankincenſe vpon it, the time should come, when the falſe prieſtes
should be burned there, confirming by preſent \Emph{miracles} that which
he auerred in wordes, \Emph{the kings hand ſuddanly withering}, &
\Emph{reſtored} againe \Emph{by the prophets prayer}, and the
new \Emph{altar cleuing} in ſunder, that the ashes fel out.
\XRef{3.~Reg.~13.}
\MNote{The often change of Kinges, and euil ſucceſſe in the kingdõ of
Iſrael.}
Further an other Prophet called \Emph{Ahias foreſhewed} the deſtruction
and vtter \Emph{extirpation of Ieroboams familie}, for his enormious
wickednes, and namely, (which is moſt often inculcate) \Emph{for making
Iſrael to ſinne}, by deuiſing and ſetting abroch a new religion,
\XRef{3.~Reg.~14.}
which ruine happened very shortly.
\MNote{The firſt familie reigned but 24.~yeares.}
For himſelf reigning twentie two yeares
\XRef{(3.~Reg.~14.)}
one of his ſonnes died preſently \Emph{according to the Prophets word}.
\XRef{v.~18.}
An other called \Emph{Nadab} ſucceding to his father, reigned only two
yeares, and vvas \Emph{ſlaine} together \Emph{with their whole race} and
kindred, by \Emph{Baaſa} of the tribe of Iſsachar.
\XRef{3.~Reg.~15.}
\MNote{The ſecond newe familie 26.}
Likewiſe Baaſa folowing the bad ſteppes of Ieroboam was forewarned by
\Emph{Iehu a Prophet}, that his houſe should alſo be deſtroyed. And
accordingly when he had reigned foure and
%%% 0958
twentie yeares, his ſonne \Emph{Ela} reigning but two yeares, was
\Emph{ſlaine} by his ſeruant \Emph{Zambri}, and al his kinred deſtroyed.
\MNote{The third but 7.~daies.}
VVhich Zambri reigned but ſeuen dayes. For being forthwith beſieged by
\Emph{Amri}, of the tribe of Beniamin, he deſperatly burned him ſelf
together with the kinges palace.
\MNote{The fourth, 48.~yeares.}
Neither did Amri then poſſeſſe the kingdome with peace. For he being
choſen king by the armie only, whereof he was general, an other part of
the people choſe & folowed \Emph{Thebni. Wherof} aroſe \Emph{ciuil
warre} betwen the Antikinges, continuing three yeares: til Thebni died,
and ſo Amri reigned alone, but wickedly as his
%%% o-0860
predeceſsors, twelue yeares in al. Then ſucceeded his ſonne \Emph{Achab}
moſt wicked. \Emph{Who maried Iezabel} a Sydonian, & by her was
perſwaded to worshippe Baal.
\XRef{3.~Reg.~16.}
To him notwithſtanding God ſent manie admonitions by ſundrie Prophetes,
and beſtovved great benefites vpon him, wherupon he did ſome notorious
penitential workes; but not perſeuering in anie good thing, returned to
his wickednes.
\XRef{3.~Reg.~20.}
And finally \Emph{beleuing falſe prophetes}, and perſecuting Michaes for
prophecying the truth, was ſlaine in battel when he thought him ſelf
moſt ſecure,
\XRef{3.~Reg.~22.}
hauing reigned twentie two yeares.
\XRef{3.~Reg.~16.}
His ſonne \Emph{Ochozias} reigning but two yeares fel through a window,
and died of the hurt.
\XRef{4.~Reg.~1.}
His other ſonne \Emph{Ioram}, after twelue yeares was \Emph{ſlaine}
by \Emph{Iehu} of an other familie: who then diſpatched \Emph{Iezabel},
and leauing her in the ſtreete, the dogges did eate her carcaſſe. He
alſo cauſed ſeuentie ſonnes of Ioram to be ſlaine, and vtterly deſtroyed
al Achabs houſe.
\XRef{4.~Reg.~10.}
\MNote{The fifth 103.}
For which ſeruice he was eſtablished in the kingdome, for foure
generations,
\XRef{v.~30.}
So himſelfe reigning twentie eight yeares,
\XRef{3.~Reg.~10.}
after him reigned ſucceſsiuely his ſonne \Emph{Ioachaz} ſeuenetene
yeares, his ſonne \Emph{Ioas}, ſixtene yeares,
\XRef{4.~Reg.~13.}
his ſonne \Emph{Ieroboam} one and fourtie yeares. Laſtlie his ſonne
\Emph{Zacharias}, vvhom his ſeruant \Emph{Sellum} of an other race,
killed when he had reigned but ſix monethes.
\XRef{4.~Reg.~15.}
\MNote{The ſixth, one moneth.}
And after one moneth Sellum vvas ſlaine by \Emph{Manahen} of an other
progenie.
\MNote{The ſeuenth, 12.~yeares.}
VVho reigned tenne yeares. Then his ſonne \Emph{Phaceia} reigning two
yeares, was ſlaine by \Emph{Phacee} of an other generation.
\MNote{The eight 20.~yeares.}
He reigning twentie yeares, \Emph{manie} of his people
were \Emph{carried captiue into Aſſiria}, and himſelfe was ſlaine by
\Emph{Oſee} of an other kindred.
\XRef{4.~Reg.~15.}
\MNote{The ninth nine yeares.}
Finallie \Emph{the Aſſirians} taking Samaria by three yeares ſiege, in
the ninth yeare of Oſee
\MNote{Then ouerthrowen and the kingdom neuer reſtored.}
\Emph{poſſeſſed the kingdome of Iſrael, and led al the principal perſons
captiues} into Aſsiria:  about two hundred fourtie two yeares after that
Ieroboam firſt reigned ouer the Ten Tribes. Thus there were in al
\Emph{ninetene kinges. Beſides Thebni}, who onlie reigned in part
againſt an other. Of which the firſt Ieroboam, and Iehu were aduanced by
Gods ordinance, for punishment of others. Amri was choſen by the armie,
the reſt of the people choſing Thebni. Six inuaded by mere force,
killing their predeceſsors. The reſt ſucceeded, by ſuch titles as their
fathers had. And though ſome were better ſome worſe then others, al were
wicked, and at laſt ouerthrowen.

%%% 0959
Contrariwiſe
\MNote{The kingdom of Iuda for Dauids ſake conſerued in his ſede.}
\Emph{in the kingdome of Iuda} ſtanding after the ſeperation of tenne
tribes about foure hundred yeares, though ſome kinges were wicked, yet
ſome were good; and in them al \Emph{God preſerued Dauids ſeede, by the
line of Salomon}, in this direct ſucceſsion: \Emph{Roboam, Abias, Aſa,
Ioſaphat, Ioram, Ochozias, Ioas}, (in whoſe infancie, his grandmother
\Emph{Athalie vſurped} the kingdome ſix yeares) \Emph{Amaſias, Ozias,
Ioathan, Achaz, Ezechias, Manaſſes, Amon, Ioſias, Ioachaz} (hitherto the
ſonne euer ſucceeding his father) then \Emph{Ioakim} (brother of
Ioachaz) Ioachin otherwiſe called \Emph{Iechonias}, ſonne of
Ioachaz. And finallie his vncle \Emph{Sedecias}; who was carried captiue
into Babilon. But \Emph{Iechonias by Gods ſpecial prouidence},
was \Emph{fauored and exalted} by a new king of Babilon. Whither he was
led captiue
%%% o-0861
before. \Emph{In whoſe line Dauids offspring continued} though not with
title of kinges, \Emph{yet in eminent ſtate}, and eſtimation. As reſteth
to be noted in the ſixth age of the world.

The
\MNote{Succeſſion of Prieſtes continued.}
\Emph{progenie} alſo of \Emph{Aaron continued} in their office and
function of Prieſthood, \Emph{with ſucceſſion of High Prieſtes}; as
before from Aaron to Sadoc, partly in the line of Eleazar, partlie of
Ithamar, both Aarons ſonnes; ſo from Sadoc, by the like ſucceſsion of
both families. For \Emph{of Eleazar} is recorded this \Emph{Genealogie}
\XRef{1.~Paral.~6.}
\Emph{Sadoc, Achimaas, Azarias, Iohanan, Azarias, Amarias, Achitob,
Sadoc, Sellum, Helcias, Azarias, Zaraias}, and \Emph{Ioſedech}. VVho
vvas High Prieſt in the captiuitie,
\XRef{(v.~15.)}
being caried into Babilon in the firſt tranſmigration vvith king
Iechonias, before the general captiuitie of al, as it ſeemeth
\XRef{4.~Reg.~24.}
his father \Emph{Zaraias} yet liuing, vvho vvas \Emph{ſlaine} nine
yeares after by Nabuchadonoſor,
\XRef{4.~Reg.~25.}
And amongſt theſe there vvere ſome \Emph{High Prieſts of Ithamars
line}. To witte, \Emph{Ioram, Ioiada}
\XRef{(4.~Reg.~11.}
\XRef{2.~Par.~23.)}
\Emph{Ioathan, Vrias},
\XRef{(4.~Reg.~16.)}
\Emph{and} ſome others; or els ſome of the aboue mentioned, had other
names, recited by Ioſephus,
\Cite{lib.~10. cap.~11. Antiq.}
and Nicephorus
\Cite{lib.~2. cap.~4. Hiſt. Eccleſ.}

Moreouer
\MNote{Extraordinary miſſion of Prophetes.}
beſides this ordinarie ſucceſsion of Prieſtes, there vvas an
\Emph{extraordinarie miſſion of Prophetes}: to ſupply more fullie the
office of preaching the truth, and admonishing offenders. And theſe God
\Emph{inſpired} and ſent, moſt eſpeciallie \Emph{when and where errors
ſprong, and ſinnes moſt abounded}: geuing them \Emph{extraordinarie
grace} and moſt excellent vertues, to conterpoiſe the enormities of
vvicked men. Such vvere in the times of Achab and Iezabel, in the
kingdome of Iſrael, beſides manie others, the \Emph{two famous great
Prophets Elias, & Eliſeus}.
\MNote{Great effectes of their preaching and miracles.}
VVhoſe admirable liues and holie conuerſation vvere a mirrour to the
vvorld, and great terrour to the vvicked. VVhoſe vvorkes and miracles
meruelouſlie confirmed the vvel diſpoſed, encouraged the weake,
conuerted manie tranſgreſſors, confounded falſe Prophets, iuſtified
their ovvne preaching, and much glorified God.
\MNote{Elias his miracles.}
\Emph{Elias 1.~Shutte the
%%% 0960
heauen, that it rayned not} in three yeares. \Emph{2.~Was fedde by
rauens. 3.~Multiplied} a poore vvidovves \Emph{meale & oile. 4.~Raiſed
her} dead \Emph{ſonne} to life.
\XRef{(3.~Reg.~17.)}
\Emph{5.~Brought fire from heauen, to burne his ſacrifice}: thereby
confounding foure hundred and fifty falſe prophets of Baal. 6.~By
prayer \Emph{procured rayne}.
\XRef{(3.~Reg.~18.)}
\Emph{7.~Faſted} vvithout eating or drincking \Emph{fourtie daies} and
nightes together.
\XRef{(3.~Reg.~19.)}
\Emph{8.~Procured fire} from heauen, \Emph{which deuoured two} inſolent
\Emph{captaines, and their hundred men}.
\XRef{(4.~Reg.~1.)}
\Emph{9.~Diuided} the riuer of \Emph{Iordan} vvith his cloke, that
himſelfe and Eliſeus paſſed ouer the drie chanel. \Emph{10.~VVas
aſſumpted in a firie chariote} into ſome place, vvhere he yet
liueth. And parting a vvay \Emph{obtained of God}, the like \Emph{duble
ſpirit} (of prophecie and miracles) to Eliſeus.
\MNote{Eliſeus his miracles.}
In like manner \Emph{Eliſeus 1.~Diuided Iordan againe by Elias cloke},
and ſo returned to his diſciples. \Emph{2.~Amended} the bitternes of
certaine \Emph{waters, by caſting in ſalte. 3.~Boies} being \Emph{curſed
by him}, for deriding him, \Emph{were} forthvvith \Emph{torne by
beares}.
\XRef{(4.~Reg.~2.)}
4.~He \Emph{procured water without rayne}, for three kinges in the
campe.
\XRef{(4.~Reg.~3.)}
\Emph{5.~Multiplied} a poore
%%% o-0862
vvidovves \Emph{oile. 6.~By his prayers a barren woman became
frutefull.} 7.~He \Emph{raiſed} her ſonne \Emph{from death. 8.~Made} the
\Emph{bitter broth of} his diſciples \Emph{ſweete. 9.~Fedde manie with
few loaues}.
\XRef{(4.~Reg.~4.)}
\Emph{10.~Cured Naaman of leproſie. 11.~Stroke Giezi with the ſame}.
\XRef{(4.~Reg.~5.)}
\Emph{12.~Made yron to ſwimme. 13.~Knewe the ſecret counſels} of the
Syrian \Emph{king. 14.~Made one ſee horſemen, and firie chariotes, which
to others were inuiſible. 15.~Made the Syrianes blinde}, that vvere ſent
to apprehend him, and ſo ledde them into Samaria. \Emph{16.~Forſhewed
vnexpected plentie of corne} the next day, vvith the \Emph{death of a}
great \Emph{man, that would not beleue it}.
\XRef{(4.~Reg.~7.)}
17.~And after his death \Emph{an other mans dead bodie, touching his bones
reuiued}.
\XRef{4.~Reg.~13.}

Other Prophets vvrought alſo miracles, but theſe for example may ſuffice
to shevve, that
\MNote{Religion not wholly deſtroyed in the kingdom of Iſrael.}
\Emph{God preſerued religion alſo in the kingdome of Iſrael}. VVhich
himſelfe further teſtified, euen in moſt deſolate times, vvhen Elias
lamented that he vvas leift alone.
\XRef{(3.~Reg.~19.)}
For God anſvvered, \Emph{that ſeuen thouſand} (meaning therby a great
multitude) \Emph{had not bowed their knees to Baal}, not ſo much as in
outvvard shevve conformed themſelues to infidelitie, or idolatrie.
\Emph{Iehu} in his time, deſtroyed all the \Emph{worſhippers of Baal}.
\XRef{(4.~Reg.~10.)}
But \Emph{none} at anie time \Emph{could wholy deſtroy true
Iſraelites}. For \Emph{God would not ſuffer it}.
\XRef{4.~Reg.~14. v.~27.}

Yea
\MNote{Hereſies in the kingdom of Iſrael.}
not vvithſtanding \Emph{diuers notorious hereſies} vvere preached, &
folovved \Emph{in that kingdome of the Tenne tribes}, yet al did not
fall, nor embrace them.
\MNote{Ieroboamites.}
\Emph{Ieroboam} not onlie made and ſet vp golden calues, but alſo
\Emph{taught, that} they vvere gods, ſaying: \Emph{Behold} thy goddes, O
Iſrael, which \Emph{brought thee out of the land of Ægypt}.
\XRef{3.~Reg.~12.}
\Emph{making temples,
%%% 0961
altars}, and \Emph{imaginarie prieſtes, which were not of the children
of Leui}. Alſo \Emph{a feaſt the fiftenth day of the moneth, after the
ſimilitude of the ſolemnitie, that was celebrated in Iuda}. Al which the
holie Scripture ſaith: \Emph{He ſcourged of his owne hart}. The very
propertie of Archeretickes.
\MNote{Manie conſtant in true religion.}
But the \Emph{true Prieſtes, Leuites, and} manie \Emph{others, that had
geuen their hart to ſeke our Lord, went into Ieruſalem, to immolate
theire victimes before our Lord the God of their fathers}.
\XRef{2.~Par.~11.}
Yea \Emph{Naaman} a ſtranger of Syria, and a Neophite in religion,
\Emph{taught} by his example, \Emph{that none may yeld conformitie, nor
otherwiſe communicate with Infideles, then Gods Prieſts, or Prophetes
approue for lawful}.
\XRef{4.~Reg.~5.}

Vnto
\MNote{Iezabelites.}
this hereſie of Ieroboam \Emph{Achab, by Iezabels perſwaſion, added the
worſhipping of Baal}, as God,
\XRef{3.~Reg.~16.}
making both temple and altar to him in Samaria. Ieroboams prieſts
ſeruing fitly this purpoſe. Though al the former heretikes no more
agreed to this new hereſie, then Lutherans now admitte of
Caluiniſme. For \Emph{Iehu a Ieroboamite deſtroyed al Iezabelits} that
he could by a ſtratageme gette together.
\XRef{4.~Reg.~10. v.~28.~29.}
Much leſſe did al Iſrael ſerue Baal.

Againe
\MNote{Samaritanites diuided into manie Sectes.}
after that Salmanazar king of Aſsyria had taken Samaria, and placed there
a new people,
\XRef{4.~Reg.~17.}
they learning the \Emph{rites of the} Iſraelits
%%% o-0863
\Emph{religion, mixed} their \Emph{Paganiſme} there with, and \Emph{made
a new hereſie}, or \Emph{rather manie new hereſies}. For
being \Emph{diuers nations} they \Emph{had} in ſeueral conuenticles,
their \Emph{particular goddes}, and \Emph{ſo manie diuers
Sects}. The \Emph{Babylonians, Cutheites, Emathites, Heueites},
and \Emph{Sapharuaimites}. 
\XRef{4.~Reg.~17.}
But as the Prieſtes, which taught them rites of true religion, allowed
not of this mixture, ſo doubtles ſome people harkened to their
admonitions, and kept religion ſimply and ſincerely.
\MNote{Tobias neuer yelded to Schiſme.}
And at this very time of the Tenne tribes captiuitie, \Emph{holie
Tobias} who was carried captiue with the reſt, \Emph{neither before nor
after the captiuitie, leift the law of God. But went to Ieruſalem} (when
others ſerued Ieroboams golden calues) to the Temple of our Lord, and
there adored the Lord God of Iſrael. And in captiuitie beſtowed himſelfe
in workes of mercie, towardes the liuing and dead of his nation.
\XRef{Tob.~1.}

As
\MNote{The kingdom of Iuda more free from hereſie.}
for \Emph{the kingdom of Iuda}, it was \Emph{more free from
hereſies}. For very few or none of thoſe kinges that fell to other
groſse enormities, yea to manifeſt idolatrie, became heretikes, as is
probablie collected by that Iſaias the Prophet being ſent to
\MNote{King Achaz.}
Achaz, admonished him, conuerſed and dealt with him, as with one that
beleued wholly and ſolely true religion: aſſuring him that God would
protect Ieruſalem, bidding him not to \Emph{feare the two ſmoking
firebrandes, in the wrath of Raſin king of Syria, and of Phacee king of
Iſrael}.
\XRef{Iſa.~7.}
Further bidding him aske a ſigne of God, he anſwered,
%%% 0962
though frovvardlie, yet not as an infidel: \Emph{I wil not aske: and I
wil not tempt our Lord.}
\MNote{Vrias high prieſt.}
Yea though \Emph{Vrias} the \Emph{High Prieſt} by commandment of the
ſame king
\XRef{(4.~Reg.~16.)}
\Emph{made a new altar} in place of Gods Altar, \Emph{yet}
he \Emph{erred not in faith}, nor in doctrine, as teaching in Moyſes
chayre, \Emph{but in fact} onlie, and of frailtie for feare of the king,
as the king offended in his external act, to flatter the king of
Syria. And in this caſe God ſent Iſaias to admonish the king, which
Vrias neglected, or durſt not do.
\MNote{King Ioram and others committing idolatrie in fact, manie others
ſtil profeſſed true Religion.}
Likewiſe \Emph{Ioram}
\XRef{(4.~Reg.~8.}
\XRef{2.~Par.~21.)}
\Emph{Ochozias}
\XRef{(2.~Par.~22.)}
\Emph{Ioas} in the latter part of his life,
\XRef{2.~Par.~24.}
\Emph{Manaſſes} in the former part of his reigne
\XRef{(4.~Reg.~2.}
\XRef{2.~Par.~33.)}
and ſome other kinges of Iuda committing idolatrie, and making others to
fall with them, either were \Emph{not wholie peruerted}, or at leaſt
\Emph{drew not al with them}. For not onlie \Emph{Prophets, in whoſe
hand} (or miniſterie) \Emph{God ſpake}, and reproued theſe ſinnes, but
manie others kept their zele of true religion, as appeared in their
promptnes to ſerue God, when by good kinges Aſa, Ioſaphat, Ezechias, Ioſias
and others, they were exhorted, or admitted ſo to do.
\XRef{4.~Reg.~18.}
\XRef{23.}
\XRef{2.~Par.~15.}
\XRef{17.}
\XRef{29.}
\XRef{30.}
\XRef{31.}
\XRef{33.}
\XRef{34.}
&c.

Finally wheras diuers \Emph{good princes diſpoſed, thinges belonging to
Diuine ſeruice} in the temple, \Emph{correcting faultes}, and
\Emph{puniſhing offenders} in that behalfe,
\XRef{(3.~Reg.~15.}
\XRef{4.~Reg.~18.}
\XRef{23.}
they \Emph{did the ſame without preiudice of the High Prieſtes
ſupremacie} in ſpirituall cauſes, and
\MNote{Authoritie depending vpon diuine ordinance, is not changed by
factes or practiſe.}
their godlie actes \Emph{make nothing for the Engliſh Paradox of
Laiheadſhippe}. For ſuperior authoritie, and ordinarie povvre is not
proued by factes good or euil, but rather by Gods ordinance and
inſtitution. For as the factes of vſurpers make no lawfull preſcription;
ſo neither the factes of good men, do
%%% o-0864
change Gods general ordinance and law: But are done either by waie of
execution, or ſometimes by diſpenſation. Often alſo by commiſsion and
ſpecial inſpiration of God.
\CNote{\XRef{Mat.~12.}}
As king Dauid by diſpenſation did eate the holie bread, which was
ordained for Prieſts onlie.
\XRef{1.~Reg.~21.}
\MNote{Good kinges defended and promoted religion not as chiefe in
ſpiritual cauſes, but by way of execution, diſpenſation, or cõmiſſion.}
He diſpoſed of Prieſtes and Leuites offices about the Arke of God
\XRef{1.~Par.~15.}
\XRef{19.}
by way of execution according to the law. And of the like offices in the
Temple (when it should be built)
\XRef{1.~Par.~23. 24. 25. 26.}
by diuine inſpiration. And Salomon by commiſsion from God depoſed
Abiathar the High Prieſt, from his office and put Sadoc in his place.
\XRef{3.~Reg.~2.}
VVherefore albeit good kinges did excellentlie well in calling together
the Prieſtes, and diſpoſing them in their offices, for execution of Gods
ſeruice, yea in commanding what they should do
\XRef{4.~Reg.~18.}
\XRef{19.}
\XRef{22.}
and in punishing Prieſtes
\XRef{(4.~reg.~23.)}
yet they did ſuch thinges as \Emph{Gods Commiſſioners}, not as ordinarie
Superiors \Emph{in ſpiritual cauſes}, and ſtill the ordinarie
ſubordination made by the law,
\XRef{Deut.~17.}
\XRef{Num.~27.}
ſtood firme and inuiolable, the \Emph{High Prieſt ſupreme Iudge} of all
doubtes in faith, cauſes, and quarels in religion, when other
ſubordinate inferior Iudges varied in their iudgementes. Of which offices
Malachias the Prophet
\XRef{(cap.~2.)}
%%% 0963
admonished Prieſtes in his time,
\MNote{Prieſtes by their negligẽce do ſinne but loſe not their
authoritie.}
that wheras they were negligent, not performing their dutie, their ſinne
was the greater, for that their authoritie ſtil remained, and the
perpetual Rule of the lavv, that \Emph{the lippes of the Prieſt ſhal
kepe knowlege, and they} (other men generally) \Emph{ſhal require the
law of his mouth, becauſe he is the Angel of the Lord of hoſtes}.
\CNote{\XRef{Deut.~17. v.~18.}}
And al Princes & others \Emph{were to receiue the law at the prieſtes
hãd} of the Leuitical Tribe.
\MNote{The Church of the old Teſtament conſerued in truth.}
This vvas the vvarant of ſtabilitie in truth of the Synagogue in the old
Teſtament.
\MNote{Much more the Church of Chriſt.}
Much more \Emph{the Church} and Spouſe \Emph{of Chriſt}, vvhoſe
excellencie and ſingular priuileges \Emph{Salomon deſcribeth in his
canticle of canticles}, hath ſuch vvarant. \Emph{Of this ſpouſe al the
Prophets write}, & that more plainlie then of Chriſt himſelfe,
\Emph{forſeing more aduerſaries bending their forces againſt her}, as
\CNote{\Cite{in Pſal.~30. conc.~2.}}
S.~Auguſtine obſerueth, \Emph{then againſt Chriſt} her head. And the
\CNote{\Cite{li.~3. c.~32. de doctrin Chriſt.}}
ſame holie father in manie places teacheth, that she \Emph{neither
periſheth, nor loſeth her beutie}, for the mixture of euil members, in
reſpect of whom she \Emph{is blacke, but fayre} in reſpect of the good,
\XRef{Cantic.~1.}
Notwithſtanding therfore ſinners remaining within the Church,
ſchiſmatikes and heretickes breaking from the Church, ſtil she remaineth
the
\CNote{\XRef{1.~Tim.~3.}}
\Emph{pillar and firmament of truth,
\CNote{4.~Reg.~19.}
the virgin daughter of Sion}.


\stopArgument


\stopcomponent


%%% Local Variables:
%%% mode: TeX
%%% eval: (long-s-mode)
%%% eval: (set-input-method "TeX")
%%% fill-column: 72
%%% eval: (auto-fill-mode)
%%% coding: utf-8-unix
%%% End:
