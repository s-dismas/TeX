%%%%%%%%%%%%%%%%%%%%%%%%%%%%%%%%%%%%%%%%%%%%%%%%%%%%%%%%%%%%%%%%%
%%%%
%%%% The (original) Douay Rheims Bible 
%%%%
%%%% Old Testament
%%%% Two Paralipomenon
%%%% Argument
%%%%
%%%%%%%%%%%%%%%%%%%%%%%%%%%%%%%%%%%%%%%%%%%%%%%%%%%%%%%%%%%%%%%%%




\startcomponent argument


\project douay-rheims


%%% 0889
%%% o-0798
\startArgument[
  title={\Sc{The Argvment of the Second Booke of Paralipomenon.}},
  marking={Argument of Two Paralipomenon.}
  ]

As
\MNote{The connexion of this booke with the former.}
the former booke sheweth,
\CNote{\XRef{1.~Par.~1.}}
how after manie generations from the beginning of the world, God
ſelecting one ſpecial nation for his peculiar people, and the ſame being
afterwardes made a kingdome, the Scepter therof, both by Gods and the
peoples election,
\CNote{\XRef{11.}}
came to Dauid, and
\CNote{\XRef{28.}}
his ſonne Salomon: Soe this booke declareth that firſt \Emph{Salomon
reigned peaceably ouer the whole kingdom}. In the nine firſt chapters.
\MNote{The contentes diuided into two partes.}
Then, in the other twentie ſeuen chapters, relateth how the ſame kingdom
was diuided, tenne tribes being taken away (the hiſtorie wherof is but
here briefly touched) and two only, with the title of the \Emph{kingdom
of Iuda}, were \Emph{poſſeſſed, by ſucceſſion of ninetenne kinges}, al
of Dauids and Salomons iſſue, in royal eſtate til the captiuitie in
Babylon.


\stopArgument


\stopcomponent


%%% Local Variables:
%%% mode: TeX
%%% eval: (long-s-mode)
%%% eval: (set-input-method "TeX")
%%% fill-column: 72
%%% eval: (auto-fill-mode)
%%% coding: utf-8-unix
%%% End:
