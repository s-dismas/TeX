%%%%%%%%%%%%%%%%%%%%%%%%%%%%%%%%%%%%%%%%%%%%%%%%%%%%%%%%%%%%%%%%%
%%%%
%%%% The (original) Douay Rheims Bible 
%%%%
%%%% Old Testament
%%%% Psalmes
%%%% Psalme 03
%%%%
%%%%%%%%%%%%%%%%%%%%%%%%%%%%%%%%%%%%%%%%%%%%%%%%%%%%%%%%%%%%%%%%%




\startcomponent psalme-03


\project douay-rheims


%%% 1153
%%% o-1045
\startChapter[
  title={Psalme 3}
  ]

\PSummary{King
\MNote{Dauid perſecuted by his ſonne. The 8.~key.}
Dauid recounteth his danger, when his ſonne Abſalom conſpired againſt
him: 4.~and thanketh God for his deliuerie, 9.~acknowledging al helpe to
be from God. Miſtically, Chriſts perſecution, Death, Burial, and
Reſurrection.}

\PTitle{The
\LNote{Pſalme of Dauid.}{Al
Interpreters agreably teach, that king Dauid made not the titles, which
are before the Pſalmes. Neuertheles they are authentical, as endited by
the Holie Ghoſt.
\MNote{Titles of the Pſalmes added by Eſdras and the Septuagint.}
And it is moſt probable Eſdras added thoſe titles which
are in the Hebrew: and the Seuentie interpreters writte the other, in
their Greke Edition. Both which S.~Ierom tranſlated into Latin.

In
\MNote{Fiue thinges to be noted in the titles.}
theſe titles fiue thinges may be noted. Firſt, the former two hauing no
title at al, the general name of Pſalme, common to al, is particularly
appropriated to ſome, and other names to others. VVhich in al are
twelue. To witte: Pſalme, Inſcription, Prayer, Canticle, Pſalme of
Canticle, Canticle of Pſalme, Hymne, Teſtimonie, Vnderſtanding, Praiſe
of Canticle, Alleluia, & Gradual Canticle. Secondly, in the titles of
ſome Pſalmes are the names of certaine perſons, which by S.~Auguſtins
iudgement, cited in the
\XRef{Proemial Annotations,}
and others, proueth not the ſame perſons to be the authores of thoſe
Pſalmes, but ſignifieth ſome other thing. Thirdly, in ſome titles the
time is ſignified, when the Pſalme was made, or ſong. Fourtly, the
matter conteyned in the Pſalme, or vpon what occaſion it was made, is
expreſſed in ſome titles. Fiftly diuers other termes are often vſed in
the titles of ſundrie Pſalmes, as 
\CNote{\XRef{Pſal.~4.}
\XRef{6.}
\XRef{8.}
\XRef{15.}
\XRef{16.}
&c.}
\Emph{To the end, For the Octaue, For preſſes}, and the like, al which
we shal briefly explicate, where they firſt occurre.

Firſt
\MNote{VVhy this is called the Pſalme of Dauid.}
therfore this third Pſalme is called \Emph{the Pſalme of Dauid}, not
becauſe he is author therof, for he is alſo author of the former, where
his name is not expreſſed, as is euident by the teſtimonie of al the
Apoſtles,
\XRef{Act.~4. v.~25.}
but becauſe it treateth particularly and literally of him.}
Pſalme of Dauid,
\LNote{VVhen he fled from the face of Abſalom.}{Here
\MNote{The time and occaſion of making this Pſalme.}
the time is ſignified, when this Pſalme was made, to wite, immediatly
after the ouerthrow of his rebellious ſonne Abſalom, mentioned
\XRef{2.~Reg.~18.}
before his returne to Ieruſalem. For albeit of humaine, natural, and
fatherlie affections, he greatly lamented the death of his ſonne, yet he
rendered thankes and praiſes to God, as reaſon and dutie bond him.}
when he fled from the face of Abſalom his ſonne.
\XRef{(2.~Reg.~15.)}}

\VV Lord
\CNote{\XRef{Ioan.~2.}}
\SNote{O God let me know how greuiouſly I haue ſinned,}
why are they
\SNote{that al Iſrael
\XRef{(1.~Reg.~15. v.~13.)}
with al their hart foloweth Abſalom. So againſt Chriſt, the Prieſtes,
the People, & Gentiles al conſpired.}
multiplied that truble me? manie riſe vp againſt me.

\V Many ſay to
\SNote{my life.}
my ſoule: There is
\SNote{he can not eſcape.}
no ſaluation for him in his God.

\V But
\SNote{But I auouch that God alwaies defendeth me,}
thou Lord art my protectour, my
\SNote{geuing me victorie,}
glorie, & exalting
\SNote{& cõfirming my kingdom.}
my head.

\V With my voice I haue cried to our Lord: and he hath heard me from his
\SNote{Heauen.}
holie hil.

\V
\LNote{I haue ſlept, and haue benne at reſt, and haue riſen vp.}{King
\MNote{King Dauid prefigured Chriſt.}
Dauid by his fleeing in perſecution, and by his reſting, and deliuerie
from his perſecuters, prefigured Chriſts Death, Burial, &
Reſurrection. As appeareth,
\XRef{Ioan.~2. v.~22.}
VVhere the Euangeliſt ſaith: that after Chriſts Reſurrection, \Emph{his
diſciples beleued the ſcripture}, to witte, this and other like
prophecies. For otherwiſe the old Teſtament doth not ſo expreſly declare
ſuch Myſteries, as the Goſpel doth: but one thing in the proper, and
grammatical ſignification of the wordes, and an other thing, in ſhadowes
and figures, and both literal. 
\MNote{The ſame Scripture hath diuers literal ſenſes.}
VVhereupon S.~Gregory teacheth
\Cite{(li.~20. c.~1. Moral.)}
that holie Scripture (amongſt other incomparable excellences) ſurpaſſeth
al other doctrines, in the verie maner of ſpeaking: becauſe by one and
the ſame ſpeach, it reporteth a thing donne, and proclameth a Myſterie:
ſo relating thinges paſt, that with the verie ſame wordes, it
foreſheweth thinges to come.}
I haue
\SNote{I lay downe,}
ſlept, and haue
\SNote{and reſted in expectation of thy helpe.}
bene at reſt; and haue
\SNote{And am deliuered. Chriſt dyed, was buried, & roſe againe.}
riſen vp, becauſe our Lord hath taken me.

\V I wil not feare thouſandes of people compaſſing me:
\SNote{I know thou wilt help me, and ſo I beſech thee to do.}
ariſe Lord, ſaue me my God.

\V Becauſe thou haſt ſtroken al that are my aduerſaries without cauſe:
thou haſt broken the
\SNote{The ſtrẽgth and furie.}
teeth of ſinners.

\V Saluation
\SNote{Health and ſaftie cõmeth from God.}
is our Lordes: and thy
\SNote{Abundance of grace promiſed to Gods ſeruantes.}
bleſſing vpon thy people.


\stopChapter


\stopcomponent


%%% Local Variables:
%%% mode: TeX
%%% eval: (long-s-mode)
%%% eval: (set-input-method "TeX")
%%% fill-column: 72
%%% eval: (auto-fill-mode)
%%% coding: utf-8-unix
%%% End:
