%%%%%%%%%%%%%%%%%%%%%%%%%%%%%%%%%%%%%%%%%%%%%%%%%%%%%%%%%%%%%%%%%
%%%%
%%%% The (original) Douay Rheims Bible 
%%%%
%%%% Old Testament
%%%% Psalmes
%%%% Psalme 90
%%%%
%%%%%%%%%%%%%%%%%%%%%%%%%%%%%%%%%%%%%%%%%%%%%%%%%%%%%%%%%%%%%%%%%




\startcomponent psalme-90


\project douay-rheims


%%% 1304
%%% o-1194
\startChapter[
  title={Psalme 90}
  ]

\PSummary{Whoſoeuer
\MNote{Gods prouidence.

The 3.~key.}
faithfully and firmly truſteth in Gods prouidence, is ſecure from al
dangers of ſecrete, ſutle, and open enimies. 7.~His aduerſaries shal
come to ruine. 11.~Angels shal defend him: 13.~no kind of ſerpent, nor
beaſt shal hurt him. 14.~God himſelf aſſureth him of his protection, and
of eternal ſaluation.}

\PTitle{Prayſe
%%% !!! SNote belongs before 'Prayſe'
\SNote{Praiſe of Gods prouidence, with thankes,}
of a
\SNote{which Dauid ſongue with voice.}
Canticle to Dauid.}

%%% !!! But this is verse 1, not the title?
\NV He
\SNote{He that firmely relieth and reſteth vpon Gods prouidence, is
aſſuredly protected by him.}
that dwelleth in the helpe of the Higheſt, shal abide in the protection
of the God of heauen.

\V He shal ſay to our Lord: Thou art my protectour, and my refuge: my
God I wil hope in him.

\V Becauſe he hath deliuered me from
\SNote{Al ſecret and ſutle machinations:}
the ſnare of the hunters, and from
\SNote{and from al crueltie of tyrants.}
the sharpe word.

\V With his shoulders shal he ouershadowe thee: and vnder his winges
thou shalt hope.

\V With shilde shal his truth compaſſe thee:
\LNote{Thou shalt not be afraid.}{S.~Auguſtin
\MNote{Foure ſortes of perſecution for the Catholique faith.}
%%% !!! Cite ? upon this place ?
here obſerueth foure maners of tempting the faithful to fal from true
Religion.
\MNote{1.}
Sometimes with tentations that is but light and obſcure, which the
Prophet here calleth feare in the night: when ignorant men are tempted
by ſuggeſtion, or apprehenſion of temporal afflictions, not knowing that
they fal into damnation, by fleing from worldlie, or bodily calamities.
\MNote{2.}
Sometimes the tentation threatneth preſent death to them that are wil
inſtructed in the truth, and knovv that they muſt confeſſe it euen to
death, which the Prophet calleth as arrovv flying in the day: vvhen the
faithful clerly ſeeth vvhat danger hangeth ouer him, to vvit, preſent
death if he ſtand conſtant, and damnation if he denie his faith.
\MNote{3.}
Sometimes the tentation is more vehement, but yet obſcure, which he
calleth, buſines vvalking in darknes: vvhen by ſutle endeuoures, framing
arguments in excuſe of ſinne, men are perſvvaded that they may lavvfully
take ſome oath, or do ſome other thing, vvhich in dede is not lavvful:
and ſo by earneſt, and ſutle perſvvaſions they ignorantly decline from
Catholique Religion, or committe other greuous ſinnes.
\MNote{4.}
But the greateſt and manifeſt tentation is called inuaſion & midday
diuel: when perſecuters ſeing neither more eaſie perſvvaſions can
deceiue Gods ſeruants, nor preſent death force them to denie the truth,
they then aſſault them more vehemently, and more dangerouſly vvith long,
and continual afflictions, not remitting their crueltie til the
afflicted either yeld to their vvil, or dye in long torments. And by
theſe tvvo latter kindes of perſecution manie are ouerthrovvne, vvhich
vvere conſtant in the former. For vvhiles tyrants propoſed dangers to
ſimple people, and deceiued ſome, yet threatning preſent death to
others, that were better inſtructed, and confirmed in Religion,
innumerable perſeuered, & gloriouſly died in confeſſion of Chriſtian
Catholique faith. But by ſutle arguing of hard pointes of chriſtian
doctrin of practiſe; and by long torments manie haue bene ſeduced,
blindly falling into errors, and manie wittingly haue denied the truth,
which they clerly beleued in their hartes, to auoide this midday diuel,
the extremitie of long manifeſt, and greuous afflictions.
\MNote{God leaueth none but thoſe that firſt leaue him.}
Neuertheles in al theſe tentations God protecteth them that firmly truſt
in him. Thoſe (ſaith this holie Father) haue failed, which preſumed of
themſelues, which dwelt not in the helpe of the Higheſt, and in
protection of the God in heauen: which ſaid not to our Lord: Thou art my
Protector, and my refuge, which truſted not vnder the ſhadow of his
winges, but relied, or attributed much to their owne ſtreingth.}
thou shalt not be afrayed
\SNote{Terrors obſcurly ſuggeſted by euil men or ſpirites, with
erronious conceipte that men are not bond in time of temporal dangers,
to confeſſe the truth.}
of the feare in the night.

\V Of
\SNote{Open perſecution threatning preſent death, except men denie the
truth which they know.}
the arrow flying in the day,
\SNote{Circumuentions of craftie enimies by ſutle arguing, and drawing
men into error, and ſo to decline from Catholique Religion.}
of buſines walking in darkenes:
\SNote{Long torments, euen to death, except Gods ſeruants wil relent, and
denie the truth, which they aſſuredly beleue, and know in their
conſcience, that they are bond to profeſſe it.}
of inuaſion, and the midday diuel.

%%% 1305
%%% o-1195
\V A
\SNote{On thy left ſide, in aduerſitie manie fal from God,}
thouſand shal fal on thy ſyde, &
\SNote{& on thy right ſide, in proſperitie manie more forgete, and
forſake God.}
ten thouſand: on thy righthand: but to thee it shal not approch.

\V But thou shalt conſider with thine eies: and shalt ſee the
retribution of ſinners.

\V Becauſe
\SNote{In ſincerely
\Fix{ſayng}{ſaying}{obvious typo, fixed in other}
thou art my hope: thou makeſt God thy refuge.}
thou ô Lord art my hope: thou haſt made the Higheſt thy refuge.

\V There shal no euil come to thee: and ſcourge shal not approch to thy
tabernacle.

\V Becauſe he hath geuen
\SNote{Angels haue protection of men by Gods ordinance.}
his Angels charge of thee:
\SNote{The diuel corruptly alleageth this ſcripture
\XRef{(Mat.~4.)}
omitting the latter part of this verſe: which ſheweth when Angels
protect iuſt men, to witte, when they walke in a right path, obſeruing
ordinarie courſe in their actions, not in geuing themſelues headlong
into needles danger, as the ſame diuel propoſed to our Sauiour, to caſt
himſelf downe from the pinnacle of the temple. Such falling is not the
way of the iuſt, but of Lucifer, that fel from heauen. So S.~Bernard
noteth.
\Cite{Ser.~15. in hunc Pſal.}}
that they keepe thee in al thy waies.

\V In their handes they shal beare thee: leſt perhaps thou knocke thy
foote againſt a ſtone.

\V Vpon the aſpe, and the baſiliſcus thou shalt walke: & thou shalt
tread vpon the lion, and the dragon.

\V
\SNote{God ſpeaketh the reſt that foloweth in this Pſalme.}
Becauſe he hath hoped in me, I wil deliuer him: I wil protect him,
becauſe he hath knowne my name.

\V He shal crie to me, and I wil heare him: with him I am in
tribulation: I wil deliuer him, and
\SNote{In eternal ſaluation.}
wil glorifie him.

\V With length of daies I wil replenish him: and I wil shew him my
ſaluation.



\stopChapter


\stopcomponent


%%% Local Variables:
%%% mode: TeX
%%% eval: (long-s-mode)
%%% eval: (set-input-method "TeX")
%%% fill-column: 72
%%% eval: (auto-fill-mode)
%%% coding: utf-8-unix
%%% End:
