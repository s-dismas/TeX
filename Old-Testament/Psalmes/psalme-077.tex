%%%%%%%%%%%%%%%%%%%%%%%%%%%%%%%%%%%%%%%%%%%%%%%%%%%%%%%%%%%%%%%%%
%%%%
%%%% The (original) Douay Rheims Bible 
%%%%
%%%% Old Testament
%%%% Psalmes
%%%% Psalme 077
%%%%
%%%%%%%%%%%%%%%%%%%%%%%%%%%%%%%%%%%%%%%%%%%%%%%%%%%%%%%%%%%%%%%%%




\startcomponent psalme-077


\project douay-rheims


%%% 1277
%%% o-1167
\startChapter[
  title={Psalme 77}
  ]

\PSummary{The
\MNote{Gods great Benefites beſtovved vpon the Ievves, and their
  ingratitude.

The 4.~key.}
royal prophet exhorting the people to attend, 5.~reciteth manie great
benefites of God towards their forefathers (whoſe ingratitude, often
rebellion, and chatiſement he ſtil noteth) 9.~as in their entrance into
the land of Chanaan: 12.~alſo before the ſame in Ægypt, and in the
deſert. 42.~How God plagued the Ægyptians: 52.~protected, and conducted
his people into the promiſed land, 56.~where likewiſe they often
offended, were punished: 65.~yet were ſtil conſerued: 69.~and the tribe
of Iuda exalted in king Dauid.}

\PTitle{Vnderſtanding
\SNote{Commended to Aſaph a chiefe muſitian, that the people might
vnderſtand and conſider Gods benefites.}
to Aſaph.}

\NV My people attend ye to
\SNote{Neither the lavv, nor the people vvas Dauids, but preſenting
Gods perſon, he ſpeaketh in his name or authoritie, vvith vvhoſe
inſpiration he vvas repleniſhed.
\Cite{S.~Greg. Prepat. in Iob. c.~2.}}
my law: incline your eare vnto the wordes of my mouth.

%%% o-1168
\V
\CNote{\XRef{Mat.~13,~5.~35.}}
I wil open my mouth in
\SNote{Albeit the prophet reciteth hiſtorically thinges donne, yet the
ſame vvere parables, ſimilitudes, and figures of other thinges:}
parables: I wil ſpeake
\SNote{yea of ſecret hidden Myſteries, obſcurely ſignified in the old
Teſtament, and reueled in the nevv.}
propoſitions from the beginning.

\V How great thinges haue we heard and
\SNote{Which partly we know by written holie Scriptures:}
haue knowen them, and
\SNote{partly by Traditions.}
our fathers haue told vs.

%%% 1278
\V They were not hid from their children, in an other generation.

Telling the prayſes of our Lord, and his powers, and his meruelous
workes which he hath done.

\V And he
\SNote{God of his mercie without precedent merite, raiſed vp a peculiar
people of Abraham, Iſaac, and Iacob,}
raiſed vp a teſtimonie in Iacob: and
\SNote{and gaue them a particular law, firſt of
\Fix{Circunciſion}{Circumciſion,}{likely typo, fixed in other}
& more largely by Moyſes.}
made a law in Iſrael.

How great thinges he commanded our fathers,
\SNote{So Abraham inſtructed his children and his houſe after him,
\XRef{Gen.~18.}}
to make the ſame knowne to their children: \V that
\SNote{in like ſorte others taught their children.}
an other generation may know.

The children that shal be borne, and shal riſe vp, and shal tel their
children.

\V That they may
\SNote{For three cauſes God gaue his law, that his people may haue
confidence in him, he ſhewing his care to inſtruct and gouerne them;}
put their hope in God, and may
\SNote{that they remember his benefites;}
not forget the workes of God: and may
\SNote{and kepe his commandmentes.}
ſeeke after his commandmentes.

\V That they become not as their fathers:
\SNote{The Iewiſh nation very often, and in great numbers murmured,
rebelled, and committed other great ſinnes, and therfore Dauid exhorted
the people of his time, not to do the like. And this exhortation
perteyneth more eſpecially to Chriſtianes, as S.~Paul teacheth.
\XRef{1.~Cor.~10.}}
a peruerſe generation and exaſperating.

A generation, that hath not directed their hart, their ſpirit hath not
bene faithful towards God.

\V
\CNote{\XRef{1.~Par.~7. v.~21.}}
The
\SNote{They firſt (truſting in their owne ſtrength) without Gods
commandment
\XRef{(Num.~14.)}
went forth to batle and were ouerthrowne.
\XRef{1.~Par.~7. v.~21.}}
children of Ephrem bending, and shooting with bow: were turned in the
day of battel.

\V They kept not the teſtament of God: and in his law they would not
walke.

\V And they forgate his benefites, and his meruelous workes, which he
shewed them.

\V Before their fathers he did meruelous thinges in the land of Ægypt,
in the filde
\SNote{Tanis the principal citie in Ægypt nere the riuer Nilus, where
Moyſes wrought his great miracles.}
of Tanis.

\V
\CNote{\XRef{Exo.~14.}}
He diuided the ſea & brought them through: and he made the waters to
ſtand as in a bottle.

%%% 1279
%%% o-1169
\V And he
\SNote{This cloude ſhadowed them from the heate of the ſunne in the
day, and the fire ſhined in the night, al the time that they were in the
deſert.}
conducted them in a cloude by day: and al the night by light of fire.

\V He ſtroke the rocke in
\SNote{In mount Horeb: and there was continual water in al the campe,
which occupied nere foure miles in length and breadth.}
the deſert: and gaue them water to drinke as in a great depth.

\V And he brought forth water out of the rocke: and made waters runne
downe as riuers.

\V And they added as yet to ſinne vnto him: they prouoked the Higheſt to
wrath in the place
\SNote{Which naturally wanted water: but by miracle had abundance.}
without water.

\V And they tempted God in their hartes: ſo that they aſked
\SNote{Not content with Manna, they demanded to haue flesh.}
meats for their liues.

\V And they ſpake euil of God: they ſaide:
\SNote{Stil incredulous, not beleuing Gods omnipotencie, they thought
that, albeit he had geuen them manna, and water, yet he could not geue
them fleſh.}
Can God prepare a table in the deſert?

\V Becauſe he ſtroke the rocke and waters ranne, & torrentes flowed:

Can he alſo giue
\SNote{By bread in general is vnderſtood al competent meate vſual for a
table.}
bread, or prepare a table for his people?

\V Therfore our Lord heard, and
\SNote{For this incredulitie, murmuring, and other ſinnes God kept the
children of Iſrael fourtie
\Fix{yares}{years}{obvious typo, fixed in other}
in the deſert, til al that were of age, when they came from Ægypt, were
dead, except only Ioſue and Caleb.}
made delay: and
\SNote{In the meane time amongſt other puniſhments, manie murmurers were
burnt to death with ſtrange fire.
\XRef{Num.~11.}}
fire was kindled in Iacob, and wrath aſcended vpon Iſrael.

\V Becauſe they beleued not in God, nor hoped in his ſaluation.

\V And he commanded the cloudes from aboue, and opened the gates of
heauen.

\V And he rayned them Manna to eate, and bread of heauen he gaue to
them.

\V Bread
\SNote{Manna made by Angels.}
of Angels did man eate: he ſent them victuals in abundance.

\V He
\SNote{God ſo changed the wind, that it brought abundance of quailes and
other birdes, into their campe.
\XRef{Exo.~16.}
\XRef{Nu.~11.}}
tranſported the Southwinde from heauen: and in his powre he brought in
the Southweſt winde.

%%% 1280
\V And he rayned vpon them flesh as duſt: and as the ſand of the ſea
fethered fowles.

\V And they fel in the middes of their campe: about their tabernacles.

\V And they did eate and were filled excedingly, and their deſire he
brought to them:

\V They were not defrauded of their deſire.

As
%%% !!! SNote goes before 'As'
\SNote{Immediatly after a moneth (for ſo long they had abundance of
theſe birdes
\XRef{ibid. v.~20.)}
they were ſtriken with a plague, and manie died, for their
concupiſcence.}
yet their meats were in their mouth:

%%% o-1170
\V And the wrath of God aſcended vpon them.

And he killed their fat ones, and
\SNote{The moſt freſh ſtrong men died, and ſo were hindered from
poſſeſſing the promiſed land of Chanaan.}
the choſen of Iſrael he hindered.

\V In al theſe thinges they ſinned as yet: and they beleued not in his
meruelous workes.

\V And their daies failed in vanitie: and their years
\SNote{In fourtie yeares aboue ſix hundred thouſand died.}
in haſt.

\V When he ſlew them, they ſought him: and they returned, and
\SNote{They offered morning ſacrifice.}
early they came to him.

\V And they remembred that God is their helper: and the high God is
their redemer.

\V And they loued him
\SNote{But were not ſincere in their hartes.}
with their mouth, and with their tongue they did lie to him.

\V But their hart was not right with him: neither were they counted
faithful in his teſtament.

\V
\SNote{Howſoeuer multitudes of people committe great ſinnes, and are
ſeuerely puniſhed, yet Gods mercie preſerueth ſome by his effectual
grace, and neuer ſuffereth the whole Church to faile, nor to be
deſtroyed.}
But he is merciful, and wil be propicious to their ſinnes: and he wil
not deſtroy them.

And he abunded to turne away his wrath: and he kindled not al his wrath.

\V And he remembred that they are flesh:
\SNote{Mans life is like the winde, that ſtil paſſeth, and the ſame
returneth not. As Ariſtotel teacheth.

Here the Hebrewes note the middes of the Pſalter, in 1263.~verſes, and ſo
manie in the reſt.}
ſpirit going, and not returning.

\V
\LNote{How often haue they exaſperated?}{Moyſes
\MNote{The people of Iſrael often murmured in the deſert.}
\XRef{(Deu.~9. v.~7.)}
repeting what had paſſed in the deſert, chargeth the people that they
had ſtil prouoked our Lord to wrath, from the day, that they came out of
Egypt, and alwayes contended aganſt him. And our Lord himſelf
expoſtulating their ingratitude, & often murmuring faith
\XRef{(Num.~14. v.~22.)}
in the beginning of the ſecond yeare, that they had then tempted him tenne
times; either by this certaine number ſignifying an vncertaine, or els
\MNote{Tenne times more notoriouſly.}
chiefly tenne times: for ſo often we find recorded that they tempted
him, and murmured within that ſmal time more notoriouſly.
\MNote{1.}
Firſt, nere vnto the redde ſea
\XRef{(Exod.~14. v.~11.)}
where ſeing the Egyptians purſuing them, they murmured againſt Moyſes,
for bringing them out of Egypt, ſaying: It had benne much better to haue
ſerued the Egyptians, then to die in the wildernes.
\MNote{2.}
Secondly, for want of ſwete water.
\XRef{Exod.~15. v.~24.}
\MNote{3.}
Thirdly, for lack of meate,
\XRef{Exod.~16. v.~3.}
\MNote{4.}
Fourtly, keeping Manna for the next day, contrarie to Gods commandment.
\XRef{ibid. v.~20.}
\MNote{5.}
Fiftly, going on the Sabbath day, alſo contrarie to Gods commandment, to
gather Manna.
\XRef{ibid. v.~27.}
\MNote{6.}
Sixtly, for want of water in Raphidim.
\XRef{Exod.~17. v.~2.}
\XRef{Num.~26. v.~2.}
\MNote{7.}
Seuently in Horeb, adoring a calfe & the image therof.
\XRef{Exod.~32.}
\MNote{8.}
Eightly, repyning for their trauels in the wildernes.
\XRef{Nu.~11. v.~1.}
\MNote{9.}
Nintly, loathing Manna, and burning with deſire to eate fleſh.
\XRef{ibid. v.~4.~5.~6.}
\MNote{10.}
Tently, deſparing to poſſeſſe the promiſed land of Chanaan, after that
the diſcouerers had reported the difficulties, with the force of the
people, and of the cities againſt which they muſt fight.
\XRef{Nu.~14. v.~1.}
Al which and the reſt, ſaith S.~Paul, happened to them in figure of vs:
and are written for our correption (or admonition) that we murmur not as
they did.
\XRef{1.~Cor.~10.}}
How
\SNote{The people of Iſrael murmured ſo often in the deſert, that it was
not eaſie to tel how often. See
\XRef{the Annotation.}}
often haue they exaſperated him in the deſert: prouoked him to wrath in
the place without water.

%%% 1281
\V And
\SNote{For eſtſoones repenting they offended God againe and againe.}
they returned, and tempted God: and the holie one of Iſrael they
exaſperated.

\V They did not remember his hand: in the day that he redemed them from
the hand of the afflicter.

\V As he put
\SNote{The firſt ſigne was in turning a rodde into a ſerpent, which was
a miracle, but no plague, the other ſignes were alſo plagues to the
Ægyptians.}
his ſignes in Ægypt, and his wonders in the filde of Tanis.

\V And he turned
\SNote{The firſt plague.}
their riuers into bloude, & their
\SNote{Pooles, lakes, and al ſortes of water, yea showers, or raine
water; which ſeldome happeneth in Ægypt.}
showers that they might not drinke.

%%% o-1171
\V He ſent vpon them
\SNote{The fourth plague, in order as they are recited in Exodus.}
a
\TNote{\L{cœnomyiam}}
ſwarme of flies, and it eate them: and
\SNote{The ſecond plague.}
the frogge, and it deſtroyed them.

\V And he gaue their fruites to
\SNote{This was a leſſe plague, not mentioned in with the greater.}
the blaſt, and their labors to
\SNote{The eight plague.}
the locuſte.

\V And he killed their vineyeardes with
\SNote{The ſeuenth plague.}
haile: and their mulberie trees with
\SNote{This alſo is omitted in Exodus.}
horefroſt.

\V And he deliuered
\SNote{Not only al trees, and plantes, but alſo beaſtes were ſubiect to
the haile,}
their beaſt to haile: and their poſſeſſion
\SNote{and to firie lightnings.}
to fire.

\V He ſent vpon them
\SNote{In theſe general termes, of wrath, indignation, and tribulation,
the Prophet comprehendeth al the other plagues, to witte, the third of
feinies, the fifth of peſtilence, the ſixt of boyles in men and beaſtes,
the ninth of darknes three dayes together.}
the wrath of his indignation: indignation, & wrath, and tribulation:
immiſſions
\SNote{Al which God ſent by the miniſterie of diuels, euil angels.}
by euil angels.

\V He made a way to the path of his wrath, he ſpared not their liues
from death: and their cattel he shut vp in death.

\V And
\SNote{The tenth and greateſt plague,
\XRef{Exo.~11. v.~5.}
&
\XRef{c.~12. v.~29.}}
he ſtroke al the firſtborne in the land of Ægypt: the firſt fruites of
al their labors in the tabernacles
\SNote{Egyptians alſo deſcended from Cham, by his ſonne Meſraim.
\XRef{Gen.~10. v.~6.}}
of Cham.

\V And he
\SNote{After that Ægypt was thus plagued, God brought Iſrael out of
their ſeruitude, as a shepheard leddeth his sheepe, and defendeth them.}
tooke away his people as sheepe: and led them as a flock in the deſert.

%%% 1282
\V And he brought them forth in hope, and they feared not: and the ſea
couered their enemies.

\V And he brought them into
\SNote{Iudea a hillie countrie.}
the mount of his
\SNote{Into that countrie which God choſe, and endewed with manie
bleſſinges.}
ſanctification, the mount, which his right hand purchaſed.

And he caſt
\SNote{As is written in
\XRef{Ioſue.}}
out the gentiles from their face: and by lot he diuided the land of
them in a corde of diſtribution:

\V And he made the tribes of Iſrael to dwel in their tabernacles.

\V And
\SNote{After the conqueſt and quiet poſſeſſion, the Iſraelites often fel
into groſſe ſinnes, eſpecially in the time of Iudges.}
they tempted, and exaſperated God the higheſt, and they kept not his
teſtimonies.

\V And they turned away themſelues, & kept not the couenant: euen as
their fathers, they were turned as a
\SNote{A croked bow deceiueth the archer, ſo this people failed to ſerue
God, and deceiued them ſelues.}
crooked bow.

\V They incenſed him to wrath in their
\SNote{In their altares erected in hilles to Idoles.}
hilles: and in their
\TNote{\L{Sculptilibus}}
grauens
%%% o-1172
they prouoked him
\SNote{By grauen
\Fix{imagies}{images}{obvious typo, fixed in other}
of Idoles, they prouoked God to indignation.}
to emulation.

\V God heard, and contemned: and he brought Iſrael to nothing
\SNote{Not abſolutely to nothing, but punished them exceedingly, til
they repented, and then ſpared and deliuered them from tribulation, as
appeareth in the
\XRef{booke of Iudges.}}
excedingly.

\V And he reiected the tabernacle
\SNote{The Arke of God ſometime kept in Silo,
\XRef{Ioſue.~18.}
in the tribe of Ephraim, was taken by the Philiſtims.
\XRef{1.~Reg.~4.}
and neuer returned thither agane.}
of Silo, his tabernacle,
\SNote{But wherſoeuer the Arke was, there God more eſpecially heard
their petitions, and gaue anſwers.}
where he dwelt among men.

\V And he deliuered
\SNote{For their ſinnes God ſuffered the Arke to be taken.}
their force into captiuitie: and their beautie into the hands of the
enemie.

\V And he
\SNote{And the Iſraelites to be ſore afflicted by their enimies.}
shut vp his people in the ſword: and he diſpiſed his inheritance.

\V
\SNote{The zele, and iuſt wrath of God ſuffered theſe calamities to
happen.}
Fyre deuoured their young men: and their virgins were not lamented.

\V Their
\SNote{Ophni and Phinees the ſonnes of Heli ſlaine and Heli himſelfe
hearing that the Arke was taken fel from his ſtoole and broke his neck.
\XRef{1.~Reg.~4.}}
Prieſtes fel by the ſworde: and their widowes were not wept for.

%%% 1283
\V And
\SNote{Neuertheles God plagued the infidels, and conſerued his Church.
\XRef{1.~Reg.~5.}}
our Lord was raiſed vp as one that ſleepeth: as a mightie man hauing
ſurfited of wine.

\V And he ſtroke his enimies on the hinder partes: an euerlaſting
reproch he gaue to them.

\V And he reiected the tabernacle
\SNote{As before
\XRef{v.~60.}}
of Ioſeph: and the tribe of Ephraim he choſe not.

\V But he choſe the tribe
\SNote{After a time the Arke was brought into the tribe of Iuda.}
of Iuda, mount Sion which he loued.

\V And he built
\SNote{The Church was firme, and euer preſerued in the old teſtament til
Chriſt, and from Chriſts time to the end of the world.}
his ſanctuarie as of vnicornes in the land, which he hath founded for
euer.

\V And
\SNote{Gods particular grace in chooſing, and exalting Dauid, was a
ſpecial benefite to the Iſraelites.}
he choſe Dauid his ſeruant, and tooke him from the flockes of sheepe:
from after the ewes with yong he tooke him.

\V To
\SNote{To rule and gouerne the people of Iſrael.}
feede Iacob his ſeruant, and Iſrael his inheritance.

\V And he fedde them in the innocencie of his hart: and in the
vnderſtandings of
\SNote{Prudently vſing his powre and authoritie.}
his hands he conducted them.


\stopChapter


\stopcomponent


%%% Local Variables:
%%% mode: TeX
%%% eval: (long-s-mode)
%%% eval: (set-input-method "TeX")
%%% fill-column: 72
%%% eval: (auto-fill-mode)
%%% coding: utf-8-unix
%%% End:
