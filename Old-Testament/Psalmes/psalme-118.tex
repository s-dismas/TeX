%%%%%%%%%%%%%%%%%%%%%%%%%%%%%%%%%%%%%%%%%%%%%%%%%%%%%%%%%%%%%%%%%
%%%%
%%%% The (original) Douay Rheims Bible 
%%%%
%%%% Old Testament
%%%% Psalmes
%%%% Psalme 118
%%%%
%%%%%%%%%%%%%%%%%%%%%%%%%%%%%%%%%%%%%%%%%%%%%%%%%%%%%%%%%%%%%%%%%




\startcomponent psalme-118


\project douay-rheims


%%% 1349
%%% o-1239
\startChapter[
  title={Psalme 118}
  ]

\PSummary{A
\MNote{Perfect iuſtice is in keping Gods law.

The 7.~key.}
perpetual recommendation of the ſingular excellencie, abſolute
neceſſitie, and eternal heauenlie profite of Gods law: with frequent
aſpirations to perfection, hatred of ſinne, loue of vertue, and feruent
deſire to reſt in God.}

%%% !!! Bad SNote placement.
\SNote{This title vvas added by the Septuagint, to admoniſh vs that this
Pſalme conteyneth that ſingular maner of praiſing God, ſignified by the
two Hebrevv vvordes Alleluja, as before
\XRef{Pſal.~104.}}
\PTitle{Alleluja.}

%%% 1352
%%% o-1242
\LetterNote{Aleph.}{Doctrine.}

\NV Bleſſed
\SNote{VVhereas al, without exception, deſire to be happie and bleſſed;}
are
\SNote{they are in dede happie (according to the perfect happines of
this life) that are immaculate:}
the immaculate in the way: which
\SNote{and they are immaculate, that walke in the law of God. VVhere the
holie Pſalmiſt preſuppoſeth, that ſome can and do kepe the law of God,
and ſo are immaculate, and bleſſed in the vvay of this life.}
walke in the law of our Lord.

\V Bleſſed are they,
\SNote{Thoſe that are immaculate, are againe bleſſed, by ſearching Gods
teſtimonies, that is, his lavv, teſtifying that the good ſhal be
revvarded, and the vvicked puniſhed, but ſearching theſe teſtimonies,
vvhiles one is contaminate vvith ſinnes againſt Gods lavv, maketh not
bleſſed:}
that ſearch his teſtimonies: that
\SNote{neither doth euerie ſuperficial careles ſearch bring this
bleſſing, but ſearching vvith true affection of the hart.}
ſeeke after him with al their hart.

\V For
\SNote{Contrariewiſe they that vvorke iniquitie are not bleſſed;}
they that worke iniquitie, haue not walked in his
\SNote{becauſe they haue not vvalked in the vvayes of God, to witte, not
kept his commandments and lavv, vvhich are the vvay to happines.}
waies.

\V Thou haſt
\SNote{For mans ovvne good, that he may come to true happines, God hath
moſt ſeriouſly commanded vs to kepe his commandments, that is, to
obſerue his Lavv commanded by moſt ſufferaine diuine authoritie.}
very much commanded thy commandmentes to be kept.

%%% 1353
\V Would God my waies
\SNote{Therfore the faithful ſeruant of God, knovving his ovvne
inſufficiencie, deſireth that God by his grace vvil direct and
ſtreingthen him,}
might be directed, to keepe thy
\SNote{to kepe his lavv, called Iuſtifications, becauſe therby man is
made iuſt.}
iuſtifications.

\V Then shal I
\SNote{They ſhal be ſafe from eternal confuſion, when they ſhal kepe not
only part, but al thy commandments, becauſe breach of anie bringeth
confuſion.}
not be confounded, when I shal looke throughly in al thy commandmentes.

%%% o-1243
\V
\SNote{So ſhal I praiſe thee, and render thankes,}
I wil confeſſe to thee
\SNote{with ſincere not fayned affection,}
in direction of hart: in that I haue lerned the
\SNote{for this great benefite, that I haue lerned, that thy law is
according to moſt iuſt iudgement.}
iudgements of thy iuſtice.

\V
\SNote{I haue therfore a firme purpoſe, & do faithfully promiſe to kepe
thy law, which maketh the keper therof iuſt.}
I wil keepe thy iuſtifications:
\SNote{Albeit thou ſuffer me ſometimes to be in tribulation, or in
tentation, yet forſake me not wholy. The Pſalmiſt knew wel (ſaith
\CNote{li.~20. c.~21. Mar.}
S.~Gregorie) that he might be profitably leift a while, who prayed, that
he ſhould not be wholy forſaken.}
forſake me not wholy.

\LetterNote{Beth.}{Houſe.}

\V Wherein
\SNote{In this ſecond Octonarie, as alſo in al the reſt, the Holie Ghoſt
by the prophets penne teacheth the meanes how to come to perfection &
happines. Here by way of interrogation, as it were demanding how a
youngman, that is euerie man prone to worldlie pleaſure, & ſlow in Gods
ſeruice, ſhal beginne to correct his courſe?}
doth a yongman correct his way?
\SNote{VVherto the ſame Holie Ghoſt anſwereth, that he muſt kepe Gods
law, called here his wordes. For al the wordes which God vttereth, are
lawes to his ſeruants.}
in keeping thy
\TNote{\L{sermones}}
wordes.

\V
\SNote{The Pſalmiſt now ſpeaketh in the perſon of perfect iuſt men, or
of the whole Church in general. VVhoſe common ſpirite ſeeketh God
intyrely.}
With my whole hart I haue ſought after thee:
\SNote{And conſidering that this perfect good wil is the gift of God,
prayeth that he wil conſerue the ſame, and not ſuffer it to be altered,
or to erre from his commandments.}
repel me not from thy commandmentes.

\V
\SNote{An other ſincere profeſſion of a reſolute good purpoſe not to
ſinne.}
In my hart I haue hid thy
\TNote{\L{eloquia}}
wordes: that I may not ſinne to thee.

\V
\SNote{A gratful aſpiration praiſing God.}
Bleſſed art thou ô Lord:
\SNote{Againe the iuſt prayeth to be more and more inſtructed in
iuſtifications: that which S.~Iohn exhorteth vnto: He that is iuſt, let
him yet be iuſtified.
\XRef{Apoc.~22.}}
teach me thy iuſtifications.

\V In my lippes, I haue pronounced al the
\SNote{Gods law is alſo called his Iudgements, becauſe ſitting in
iudgement he geueth ſentence according to his Law.}
iudgementes of thy mouth.

%%% 1354
\V I am
\SNote{As the iuſt profeſſeth by mouth, ſo he delighteth in hart.}
delighted in the way of thy teſtimonies, as in al riches.

\V I
\SNote{Practiſeth in worke:}
wil be exerciſed in thy commandmentes: and I wil conſider thy waies.

\V I
\SNote{and diligently meditateth Gods law.}
wil meditate in thy iuſtification: I wil not forget thy
\TNote{\L{ſermones.}}
wordes.

\LetterNote{Gimel.}{Fulnes.}

\V Render
\SNote{O Lord liberally geue me that which I here craue,}
to thy ſeruant,
\SNote{quicken me with ſpiritual life, thy grace,}
quicken me:
\SNote{ſo I ſhal kepe thy law, which otherwiſe I can not.}
and I shal keepe thy
\TNote{\L{ſermones.}}
wordes.

\V
\SNote{Illuminate myn vnderſtanding, by thy grace,}
Reuele mine eies:
\SNote{that I may be able to ſee the meruelous great and iuſt reaſons of
thy law, inſtructing al, threatning the peruerſe, encoreging the wel
diſpoſed, puniſhing the wicked, rewarding the good, doing right to al.}
and I shal conſider the meruelous thinges of thy law.

%%% o-1244
\V
\SNote{I that haue but a ſmal time in this world,}
I am a ſeiourner in the land,
\SNote{deſire to be inſtructed in thy law, what is therein commanded.}
hide not thy commandmentes from me.

\V My ſoule hath coueted to deſire thy iuſtifications, at al time.

\V Thou haſt
\SNote{I conſider that thou ô God, doſt ſharply reproue the prowd
contemners of thy commandments:}
rebuked the prowde:
\SNote{laying curſes vpon them for declining from thyn obedience.}
curſed are they that decline from thy commandmentes.

\V Take from me reproch, and contempt: becauſe I haue ſought after thy
teſtimonies.

\V For
\SNote{Though perſecutors were very potent,}
princes ſate, and they ſpake againſt me: but
\SNote{yet the faithful ſeruant of God perſeuered in his ſeruice.}
thy ſeruant was exerciſed in thy iuſtifications.

\V For both
\SNote{In time of perſecution and tentation we muſt thincke and
meditate, that Gods law teſtifieth eternal revvard, or puniſhment,}
thy teſtimonies are my meditation: and
\SNote{and in our deliberation or conſultation, we muſt conſider that
keping Gods law maketh iuſt: and conſequently meriteth reward.}
thy iuſtifications my counſel.

\LetterNote{Daleth.}{Of Tables.}

\V My ſoule
\SNote{This alſo is vttered in the perſon of the iuſt, who is often
brought to great diſtreſſe: as it were, euen nere to death,}
hath cleaued to the pauement:
\SNote{in which caſe he confidently prayeth to be reliued, according to
Gods word, law, and promiſe.}
quicken me according to thy word.

%%% 1355
\V I haue vttered my wayes, and thou haſt heard me: teach me thy
iuſtifications.

\V Inſtruct me the way of thy iuſtifications: and I shal be exerciſed in
thy meruelous workes.

\V My ſoule
\SNote{Being is ſo great anxietie that my minde is almoſt diſtracted, or
ouercome,}
hath ſlumbered for tediouſnes:
\SNote{I cal to thee ô God, that thou wilt conſerue me, that I ſtil kepe
thy law, vttered by thy vvordes.}
confirme me in thy wordes.

\V
\SNote{Protect me that I fal not to iniquitie.}
Remoue from me the way of iniquitie: and according to thy law,
\SNote{And of thy mercie conſerue me in ſtate of grace.}
haue mercie on me.

\V I haue choſen the way of truth: I haue not forgotten thy iudgements.

\V I haue cleaued to thy teſtimonies ô Lord:
\SNote{Suffer me not to be confounded.}
do not confound me.

%%% o-1245
\V
\SNote{Man is able, and doth runne in the right vvay of Gods
commandments,}
I ranne the way of thy commandments:
\SNote{yet not of himſelfe, but vvhen God repleniſheth his hart vvith
grace.}
when thou didſt dilate my hart.

\LetterNote{He.}{This Thing.}

\V
\SNote{Impreſſe ô God thy lavv in myn affection, make me to loue it, and
to deſire to be iuſtified,}
Set me a law ô Lord the way of thy iuſtifications:
%%% !!! SNote not marked in either
\SNote{ſo ſhal I hartely and alvvayes ſeeke it.}
and I wil ſeeke after it alwayes.

\V
\SNote{After thou haſt geuen me a deſire to kepe thy lavv, geue me alſo
vnderſtanding,}
Geue me vnderſtanding,
\SNote{then ſhal I fruictfully ſearch it. For this is the right order
(as before in the firſt and ſecond verſes) firſt to loue Gods lavve, to
be iuſtified, and to become immaculate; and then to ſearch to knovv the
lavve, and ſo it is more eaſily lerned.}
and I wil ſearch thy law: and I wil keepe it with my whole hart.

\V
\SNote{Gods grace firſt dravveth and leadeth,}
Conduct me into the path of thy commandments:
\SNote{then freevvil inflamed vvith deſire effectually concurreth.}
becauſe I would it.

\V
\SNote{Stil the Prophet inculcateth the neceſſitie of Gods grace, as
vvel to make vs deſire that is good,}
Incline my hart into thy teſtimonies: and
\SNote{as to flee from euil.}
not into auarice.

\V
%%% !!! SNote marked only in other
\SNote{It is neceſſarie alſo to pray that God vvil take avvay occaſions,
vvhich might moue to ſinne:}
Turne away mine eies that they ſee not vanitie:
\SNote{and ſtil to grant his helping grace in progreſſe of vertue.}
in thy way quicken me.

%%% 1356
\V
\SNote{Againe the iuſt prayeth for confirmation in grace, to be
eſtablished in the feare of God.}
Eſtablish thy
\TNote{\L{eloquiũ}}
word to thy ſeruant, in thy feare.

\V
\SNote{To be deliuered alſo from al the effectes of former ſinnes,}
Take away my reproch, which I haue feared:
\SNote{for ſinne is therfore reprochful and odious, becauſe it is
contrarie to Gods lavv, and true iudgements, vvhich are moſt pleaſant.}
becauſe thy iudgements are pleaſant.

\V Behold I haue coueted thy commandments: in thine equitie
\SNote{Being thus affected vvith deſire to kepe the commandments, the
ſoule prayeth to be ſtil quickned, more and more vvith good ſpirite, and
ſo to perſeuere to the end.}
quicken me.

\LetterNote{Vau.}{And.}

\V And
\SNote{Againe conſidering that vvithout Gods grace preuenting, man can
not do anie good thing, the prophet renevveth his prayer, requeſting
Gods mercie,}
let thy mercie come vpon me ô Lord:
\SNote{and his helpe freely promiſed to al that aske it.}
thy ſaluation according to thy
\TNote{\L{eloquiũ}}
word.

\V And
\SNote{VVhervvith being aſſiſted and ſtreingthned, he that before vvas
vveake vvil boldly anſvver al calumniators, that reprochfully ſay: God
wil not helpe him:}
I shal anſwer a word to them that vpbrayde me:
\SNote{that in dede he hath not in vaine truſted in Gods promiſed
helpe.}
becauſe I haue hoped in thy
\TNote{\L{ſermonibus.}}
wordes.

%%% o-1246
\V And
\SNote{He alſo prayeth, though he be ſometimes fearful, that God vvil
not ſuffer him vvholly to omitte manifeſt profeſſion of faith and true
religion,}
take not away out of my mouth the word of truth vtterly:
\SNote{ſeing by thy former grace I haue already repoſed my truſt in thy
promiſes, made to them that are reſolued to kepe thy lavv.}
becauſe I haue much hoped in thy iudgementes.

\V And
\SNote{For I do firmly purpoſe euer and alwayes to kepe thy law.}
I wil keepe thy law alwayes: for euer, and for euer and euer.

\V And
\SNote{In this I haue had great ioy and comforte of mind:}
I walked in largeneſſe:
\SNote{becauſe I did in dede ſeeke after thy commandements, vvhich is
ſpecially vttered (as alſo the three next verſes) in the perſon of
thoſe, that are in trial of perſecution for their faith:}
becauſe I haue ſought after thy commandments.

\V And
\SNote{vvho boldly in time of perſecution, euen before perſecuting
Kinges and Emperors, profeſſe Chriſts true Religion. Veryfied in
innumerable glorious Martyrs, yea alſo of the fraile ſexe, in
S.~Catharin, S.~Cecilie, S.~Lucie, S.~Margaret, S.~VVenefrede,
S.~Vrſula, and her felovves, and manie more, moſt conſtantly anſvvering
al vvordes of reproch obiected, as if it vvere a baſe or contemtible
thing to be Chriſtians, to be Catholiques, to be Papiſtes. No, al theſe
and the like, are honorable and glorious titles; importing the true
ſeruice of Chriſt; in vnitie of the Catholique Church; and ſpiritual
participation vvith the viſible head therof, Chriſts Vicar in earth.}
I ſpake of thy teſtimonies in the ſight of kinges: and was not
confounded.

%%% 1357
\V And
\SNote{Such confeſſors as yet mortal, reioyce in that they haue
meditated in Gods commandments, vvhich they haue feruently loued.}
I meditated in thy commandments, which I loued.

\V And
\SNote{Alſo ſhevved the ſame in external vvorke, not diſſembling by
ſilence, by vvord, nor fact;}
I haue lifted vp my handes to thy commandments, which I loued: and
\SNote{euerie way exerciſing Gods lavv, vvhich maketh the obſeruers
iuſt.}
I was exerciſed in thy iuſtifications.

\LetterNote{Zain.}{This.}

\V Be
\SNote{That vvhich God hath decreed, and promiſed, being in itſelfe moſt
certaine and aſſured, yet includeth the meanes, vvherby it ſhal be put
in execution: and therfore the iuſt, his elect, do pray for the
performance of his vvil.}
mindeful of thy word to thy ſeruant, wherein thou haſt geuen me hope.

\V This hath comforted me in my humiliation: becauſe
\SNote{Expectation of thy promiſe hath geuen me corege.}
thy
\TNote{\L{eloquium}}
word hath quickened me.

\V The
\SNote{Provvde contemners of Gods lavv, haue euerie vvay moleſted me, by
detracting, deriding, calumniating, and violently perſecuting me.}
prowd did vniuſtly excedingly:
\SNote{Al vvhich I haue borne patiently, and not declined from thy
lavv.}
but I declined not from thy law.

%%% o-1247
\V I
\SNote{I remembred and conſidered thy iuſt puniſhments inflicted vpon
the impious,}
haue bene mindful of thy iudgements from
\SNote{euen from the beginning of the vvorld (both vpon the diuels, and
vvicked men) and that thou vvilt exerciſe the like hereafter,}
euerlaſting ô Lord:
\SNote{vvhich conſideration of thy iuſtice comforted me.}
and was comforted.

\V
\SNote{Otherwiſe if I had not ſene thy iuſtice, my zele againſt
contemners of thy law, would haue killed me.}
Faynting poſſeſſed me, becauſe of ſinners forſaking thy law.

\V
\SNote{In this place of my peregrination from heauen, I am comforted by
remembring, celebrating, and ſinging thy iuſt commandments and lavves,
vvhich make thy ſeruants iuſt.}
Thy iuſtifications were ſongue by me, in the place of my peregrination.

\V I haue bene mindful in
\SNote{In perſecution, and in al tribulation, I kept thy law becauſe I
would not dishonour thy name.}
the night of thy name ô Lord: and haue kept thy law.

\V This was done to me:
\SNote{And my tribulation eſpecially fel vpon me, becauſe I ſought to be
iuſtified by keping thy law.}
becauſe I ſought after thy iuſtifications.

%%% 1358
\LetterNote{Heth.}{Life.}

\V My
\SNote{The Prophet procedeth ſpeaking in the perſon of the iuſt tending
to perfection, and ſaying: This is my happie choiſe, that I deſire no
other enheritance, nor poſſeſſion, but to kepe Gods lavv.}
portion ô Lord, I ſayd to keepe thy law.

\V
\SNote{And ſeeing this excedeth my proper ſtreingth, I prayed God of his
mercie to make me able to kepe it.}
I beſought thy face, with al my hart: haue mercie on me according to thy
\TNote{\L{eloquiũ}}
word.

\V I
\SNote{Pondering my former actions, I turned my pathes to obſerue more
perfectly the Law, which God hath teſtified to be the right way.}
thought vpon my wayes: and conuerted my feete vnto thy teſtimonies.

\V I
\SNote{With prompnes of mind, and without heſitation I reſolued to kepe
the commandments.}
am prepared, and am not trubled: to keepe thy commandments.

\V The
\SNote{The wicked laide cordes, nettes, or ſnares to intrappe, and
hinder me,}
cordes of ſinners haue wrapped me round about: and
\SNote{but I kept thy law fresh in memorie.}
I haue not forgotten thy law.

\V At
\SNote{That this is not vnderſtood only myſtically in time of
affliction, but alſo literally and prophetically, that ſome ſpecial
ſeruantes of God ſhould obſerue a godlie profeſſion of praying at
midnight, the vvord (I roſe) maketh it probable.
\CNote{\XRef{Act.~16.}}
S.~Paul & Silas, either of a holie cuſtome, or at leaſt vpon ſpecial
occaſion (and ſuch occaſions vvere to them, and others frequent) prayed,
and praiſed God at midnight. And novv in the Church of Chriſt ſome
religious men pray, and praiſe God continually at midnight, beſides
other houres, mentioned more diſtinctly
\XRef{v.~164.}}
midnight I roſe to confeſſe to thee, for the iudgements of thy
iuſtification.

%%% o-1248
\V
\SNote{A great benefite, and a ſingular conſolation, that al true liuing
members of Chriſt, are partakers of al the prayers, good workes, and
merites, of the whole Church militant and triumphant. VVhich in our
Crede is called, The Communion of Sainctes.}
I am partaker of al that feare thee: and that keepe thy commandments.

\V The
\SNote{So great is the mercie of God, extended, communicated, and
multiplied in the whole earth:}
earth ô Lord is ful of thy mercie:
\SNote{inſtruct me, and direct me therfore ô God, that I may lerne and
obſerue thy law, and ſo be iuſtified, and made participant of ſo great
mercie.}
teach me thy iuſtifications.

\LetterNote{Teth.}{Good.}

\V Thou haſt
\SNote{Dealt very bountifully}
done bountie with thy ſeruant ô Lord:
\SNote{as thou didſt promiſe.}
according to thy word.

%%% 1359
\V
\SNote{He that hath bountifully receiued grace at Gods hand, prayeth for
more grace, that he may be beneficial to others in releuing the needie;}
Teach me goodneſſe,
\SNote{in inſtructing the ignorant,}
and diſcipline, and
\SNote{in perſwading to kepe the law of God:}
knowledge:
\SNote{becauſe he hath lerned and beleueth the commandments, by which he
is bond to loue, and haue care of his neighbour.}
becauſe I haue beleued thy commandments.

\V Before I
\SNote{Before I was afflicted, I often fel into ſinne:}
was humbled I offended:
\SNote{but vexation gaue me vnderſtanding, therfore now I kepe thy law.}
therfore haue I kept thy
\TNote{\L{eloquiũ}}
word.

\V Thou art good: and in thy goodneſſe teach me thy iuſtifications.

\V The iniquitie of
\SNote{Contemners of thy law haue endeuoured to intangle me,}
the prowd is multiplied vpon me:
\SNote{but I perſiſt in keping thy commandments. As before
\XRef{v.~51.}
\XRef{61.}}
but I in al my hart wil ſearch thy commandments.

\V Their hart is
\SNote{Though the wicked combine themſelues together againſt me,}
crudded together as milke:
\SNote{yet I conſider, that it is neceſſarie to perſeuer in thy law.}
but I haue meditated thy law.

\V It is good for me that thou haſt humbled me: that I may learne thy
iuſtifications.

\V The law of thy mouth is
\SNote{A clere compariſon, that it is better to kepe Gods law, which
bringeth life euerlaſting, then to haue al the riches & kingdoms of this
world.}
good vnto me, aboue thouſands of gold, and ſiluer.

\LetterNote{Iod.}{Beginning.}

\V Thy handes haue made me, and formed me:
\SNote{God being our Creator, we may with confidence pray him to
illuminate our mindes, that we may lerne what is his pleaſure, and ſo
endeuour to fulfil it.}
geue me vnderſtanding, and I wil learne thy commandmentes.

\V They that feare thee
\SNote{Others that loue God wil be gladde to ſee me alſo ſerue him.}
shal ſee me, & shal reioyce: becauſe I haue much hoped in thy wordes.

%%% o-1249
\V
\SNote{The iuſt being afflicted, and not ſeing the particular cauſe
therof, yet knoweth and confeſſeth, that God doth it for moſt iuſt
cauſe.}
I know ô Lord that thy iudgements are equitie: and in thy truth thou
haſt humbled me.

\V
%%% !!! SNote should follow above verse.
\SNote{And therfore with patience prayeth for comforth, as foloweth:}
Let thy mercie be done to comfort me, according to thy
\TNote{\L{eloquiũ}}
word vnto thy ſeruant.

\V Let thy commiſerations come to me, and I
\SNote{Who am almoſt dead in tribulation.}
shal liue: becauſe thy law is my meditation.

%%% 1360
\V Let the prowde
\SNote{The iuſt alſo prayeth that the wicked may be ashamed, and
conuerted, for ſo the hebrew word here ſignifieth, though it is alſo
lawful to deſire the iuſt punishment of obſtinate ſinners.}
be confounded, becauſe they haue done vniuſtly toward me: but I wil be
exerciſed in thy commandments.

\V Let them
\SNote{He prayeth againe for conuerſion of the wicked, and to haue peace
with them.}
be conuerted to me that feare thee: and that know thy teſtimonies.

\V Let my hart be made immaculate in thy iuſtifications, that I be not
confounded.

\LetterNote{Caph.}{Hand, or Palme of the Hand.}

\V My ſoule hath fainted for
\SNote{Manie iuſt of the old teſtament moſt feruently deſired the coming
of Chriſt our Sauiour, as our Lord himſelfe teſtifieth,
\XRef{Mat.~13. v.~17.}
And now the iuſt deſire his coming in glorie.
\XRef{2.~Tim.~4. v.~8.}}
thy ſaluation: and I haue much hoped in thy word.

\V Myne eies haue fayled for thy
\TNote{\L{eloquiũ}}
word, ſaying:
\SNote{Delayed hope afflicteth.}
When wilt thou comfort me?

\V Becauſe I am made
\SNote{As a leather bottel made of a beaſts skinne, congeled with the
froſt, and after parched in ſmoke, ſo is the bodie of the iuſt mortified
by diuers ſortes of afflictions, made a new bottel fitte to receiue new
wine, that is, perfect doctrin of Chriſtian life, as of faſting, and
other auſteritie, wherof our Sauiour ſpeaketh,
\XRef{Mat.~9. v.~17.}}
as a bottel in the hoare froſt: I haue not forgotten thy iuſtifications.

\V How manie are
\SNote{Such is mans infirmitie, yea of the iuſt, that he apprehendeth
tribulations to be very long, and therfore deſireth conſummation; and
that without ſinne, ſo he ſtil ſubmitte his wil to Gods wil.}
the daies of thy ſeruant: when wilt thou doe iudgement on them that
perſecute me?

\V The vniuſt haue told me
\SNote{Friuolous idle tailes, which are not according to Gods law.}
fables: but not as thy law.

\V Al thy commandmentes are truth: they haue vniuſtly perſecuted me,
helpe me.

\V They haue wel nere
\SNote{I was in great danger, but am not ouerthrowne.}
made an end of me in the earth: but I haue not forſaken thy
commandments.

\V According to
\SNote{And by thy merciful grace ſhal perſiſt.}
thy mercie quicken me: and I shal keepe the teſtimonies of thy mouth.

%%% o-1250
\LetterNote{Lamed.}{Diſcipline.}

\V For euer Lord
\SNote{The praiſe of Gods workes; which are firme and permanent in the
order, wherin he ſet them.}
thy word is permanent in heauen.

\V Thy truth in generation and generation: thou haſt founded the earth,
and it is permanent.

%%% 1361
\V By thy ordinance the day contineweth: becauſe
\SNote{Al thinges of this world, man excepted, do Gods wil.}
al thinges ſerue thee.

\V
\SNote{Man, except he meditate Gods law, and therby be holden vp, is in
danger,}
But that thy law is my meditation: I had then
\SNote{perhaps in euerie tentation to periſh eternally. For he can neuer
riſe out of mortal ſinne, by his owne powre, and al ſhould periſh if
Gods mercie did not ſpare ſome, and geue them new effectual grace to
repent.}
perhaps peridhed in my humiliation.

\V I wil not forget thy iuſtifications for euer: becauſe in them thou
haſt quickned me.

\V I am thine, ſaue me: becauſe
\SNote{Alwayes vnderſtood, that Gods grace preuented, els no man can
ſeeke to obſerue the commandments.}
I haue ſought out thy iuſtifications.

\V Sinners haue expected me to deſtroy me: I vnderſtood thy teſtimonies.

\V
\SNote{Al worldlie thinges haue their conſummation and end:}
Of al conſummation I haue ſene the end:
\SNote{Gods commandment continueth euer. For we are perpetually bond, to
loue and ſerue God: to loue our neighboures, yea and enimies. The reward
alſo for keping Gods commandments, & puniſhment for breaking them, are
eternal vvithout end.}
thy commandment is exceding large.

\LetterNote{Mem.}{Of Them.}

\V
\SNote{It is meruel to a perfect iuſt man, that he hath ſo much loued, and
obſerued God lavv. By acknovvleging vvherof, he yeldeth praiſe and
thankes to God, vvhoſe gift it is.}
How haue I loued thy law ô Lord! al the day it is my meditation.

\V
\SNote{The fruictes of obſeruing Gods law are manie and great. Amongſt
others, it maketh the obſeruer, wiſer then his enimies.}
Aboue mine enemies thou haſt made me wiſe by thy commandment: becauſe it
is to me for euer.

\V Aboue
\SNote{It maketh the obſeruer wiſer, then his temporal maiſters, that
taught him, to wit, then thoſe that teach wel, and do not performe the
ſame.}
al that taught me haue I vnderſtood: becauſe thy teſtimonies are my
meditation.

\V Aboue
\SNote{Yonger in yeares that kepe Gods commandments, are vviſer then the
more ancient that kepe them not.}
ancientes haue I vnderſtood: becauſe I haue ſought thy commandments.

\V I haue ſtaied my feete from al euil way: that I may keepe thy wordes.

\V I haue not declined from thy iudgements: becauſe thou haſt ſet me a
law.

%%% 1362
\V
\SNote{An other fruict is the ſwetnes, which the iuſt feeleth in his
owne ſoule.}
How ſweete are thy
\TNote{\L{eloquia}}
wordes to my iawes, more then honie to my mouth!

\V By thy commandments I haue vnderſtood: therfore haue I
\SNote{It brideth alſo iuſt hatred to ſinne.}
hated al the way of iniquitie.

%%% o-1251
\LetterNote{Nun.}{Euerlaſting.}

\V
\SNote{The word or law of God declared by Prophets, Paſtors, or other
Preachers is the ordinarie meanes for others to lerne, how to direct
their wayes, and actions.}
Thy word is a lampe to my feete, and a light to my pathes.

\V I
\SNote{Such profeſſion Gods people made in the old law, in Circumciſion,
or at other times: Chriſtians make it in Baptiſme.}
ſware, and haue determined to keepe the iudgements of thy iuſtice.

\V I am
\SNote{Al that wil liue godly in Chriſt \Sc{Iesvs}, shal ſuffer
perſecution.
\XRef{2.~Tim.~3. v.~12.}}
humbled excedingly ô Lord: quicken me according to thy word.

\V The
\SNote{Beſides the commandments, the iuſt alſo offer voluntarie workes
of ſupererogation, acceptable to God.}
voluntaries of my mouth make acceptable ô Lord: and teach me thy
iudgementes.

\V My
\SNote{By this Hebrevv prouerb is ſignified, that a iuſt mans temporal
life is in continual danger, as the thing that is in ones hand, is
readie to be laide by, put out of his hand, or to be by and by diſpoſed
of, or may forthvvith fal from his hand.}
ſoule is in my handes alwaies: and I haue not forgotten thy law.

\V Sinners laid a ſnare for me: and I haue not erred from thy
commandments.

\V For inheritance I haue purchaſed thy teſtimonies for euer: becauſe
they are the ioy of my hart.

\V I haue inclined my hart to doe thy iuſtifications for euer,
\SNote{A moſt euident place, that the keping of Gods commandments merite
reward, and are rightly obſerued in reſpect of reward.}
for reward.

\LetterNote{Samech.}{Helpe.}

\V I haue
\SNote{Not anie mortal man is to be hated, in his perſon, but his
iniquitie, by vvhich he is an enimie to Gods lavv,}
hated the vniuſt: and I haue
\SNote{vvhich euerie iuſt man loueth.}
loued thy law.

\V Thou art my helper, and protectour: and vpon thy word I haue much
hoped.

%%% 1363
\V
\SNote{Whoſoeuer wil ſeriouſly and ſecurely ſearch the law of God, muſt
auoide the conuerſation of euil men.}
Depart from me ye malignant: and I wil ſearch the commandmentes of my
God.

\V
\SNote{A general and very fitte prayer, vvhen vve addict ourſelues by a
firme reſolution to ſerue God, beſeching him to receiue vs into his
protection:}
Receiue me according to thy
\TNote{\L{eloquium}}
word,
\SNote{vvherby ſpiritual life is conſerued:}
and I shal liue: and
\SNote{vvhich we pray vvith great confidence, becauſe he hath promiſed
to heare thoſe that ſeeke, aske, and knocke at the dore of his grace.}
confound me not of myn expectation.

\V
\SNote{VVe muſt pray alſo that he ſuffer vs not to be confounded, or
made fruſtrate of the revvard vvhich vve expect; for hope confoundeth
not, if charitie be povvred in our hartes, by the Holie Ghoſt, vvhich is
geuen vs.
\XRef{Rom.~5. v.~5.}}
Helpe me, and I shal be ſaued: and I wil meditate in thy iuſtifications
alwayes.

%%% o-1252
\V Thou haſt
\SNote{God reiecteth ſinners:}
deſpiſed al that reuolt from thy iudgementes:
\SNote{ſo long as they thincke peruerſly: that is, vntil by his grace,
ſome returne to a better mind, vvhich of themſelues they can neuer do.}
becauſe their cogitation is vniuſt.

\V Al the ſinners of the earth I
\SNote{As God accounteth of ſinners, ſo the iuſt alſo eſtemeth them,
conforming his iudgement to Gods.}
haue reputed preuaricatours: therfore haue I loued thy teſtimonies.

\V
\SNote{Seruil feare is profitable as this place maketh euident: though
perfect charitie aftervvards ſucceding, expelleth that feare, and moueth
to do vvel for the loue of God, not for feare of puniſhment.
\XRef{1.~Ioan.~4.}}
Pearſe my flesh with thy feare: for I am afrayd of thy iudgementes.

\LetterNote{Ain.}{A Fountaine, or an Eye.}

\V I
\SNote{The iuſt againe in feruent zele, not arrogantly, but confidently
profeſſing his innocencie,}
haue done iudgement and iuſtice:
\SNote{prayeth to be defended from calumniators.}
deliuer me not to them that calumniate me.

\V Receiue thy ſeruant
\SNote{Grant therfore ô God the good, and lawful requeſt which I demand.}
vnto good: let not the prowde calumniate me.

\V Mine eies haue fayled
\SNote{By long expecting to be deliuered, and ſaued from tribulation.}
after thy ſaluation: and for the
\TNote{\L{eloquium}}
word of thy iuſtice.

\V Doe with thy ſeruant according to thy mercie: and
\Fix{teacher}{teach}{obvious typo, fixed in other}
me thy iuſtifications.

\V I am thy ſeruant: geue me vnderſtanding, that I may know thy
teſtimonies.

%%% 1364
\V
\SNote{It is time, and hiegh time, ſaith feruent zele of the iuſt man,
that God deliuer the innocent:}
It is time to doe ô Lord:
\SNote{vvhen the vvicked haue not only perſecuted the good, but haue
alſo contemptuouſly made houoke of Gods lavv, and true religion.}
they haue diſſipated thy law.

\V
\SNote{For this zele of Gods lavv ſo deſpiſed, and
\Fix{diſſiputed,}{diſſipated,}{likely typo, fixed in other}
the iuſt more and more loueth, that vvhich the vvicked ſo deadly hate.}
Therfore haue I loued thy commandementes, about gold and topazius:

\V
\SNote{Euen by the mortal hate of the vvicked I ſavv, that Gods lavv is
moſt excellent, and therfore addicted myſelfe ſo much the more to loue
it,}
Therfore was I directed to al thy commandements:
\SNote{and to hate al vvicked vvayes.}
al wicked way I haue hated.

\LetterNote{Phe.}{Mouth.}

\V Thy
\SNote{Gods meruelous povvre and vviſdom, teſtified by his vvorkes and
commandments,}
teſtimonies are meruelous:
\SNote{vvorthely inuite iuſt ſoules, to meditate and contemplate the
ſame.}
therfore hath my ſoule ſearched them.

%%% o-1253
\V The
\SNote{Firſt entrance into knowlege of holie Scripture, illuminateth the
vnderſtanding of the humble, wherby they procede to know more.}
declaration of thy
\TNote{\L{Sermonum}}
wordes doth illuminate: and geueth vnderſtanding to litle ones.

\V I
\SNote{By this Metaphor, of gaping, or vvide opening the mouth, and
dravving breath, the Prophet deſcribeth the great deſire of the iuſt, to
knovv and kepe Gods commandments.}
opened my mouth, and drew breath: becauſe I deſired thy commandments.

\V Looke vpon me, and haue mercie on me, according to
\SNote{According to thy accuſtomed equitie, in shewing mercie to them
that loue thy name.}
the iudgement of them that loue thy name.

\V Direct my ſteppes according to thy
\TNote{\L{eloquiũ}}
word: and let not anie iniuſtice haue domination ouer me.

\V Redeme me from the calumnies of men: that I may kepe thy
commandmentes.

\V Illuminate
\SNote{Let thy diuine Maieſtie looke vpon me with fauorable
countenance.}
thy face vpon thy ſeruant: and teach me thy iuſtifications.

\V
\SNote{True repentance conſiſteth not only in purpoſe to auoide ſinne
hereafter, vvhich in dede is firſt required, but alſo in ſorovv and
lamentation for ſinnes paſt.}
Mine eies haue gushed forth iſſues of waters: becauſe they haue not kept
thy law.

\LetterNote{Sade.}{Iuſtice.}

\V Thou art iuſt ô Lord: and
\SNote{God being eſſentially iuſt of himſelfe, maketh men iuſt according
to right iudgement, by geuing them grace of mercie, vvhervvith they
cooperating, are iuſt by iuſtice in dede inherent in their ſoules, not
by imputation only: for it vvere not right iudgement to impute, or
account man iuſt, vvho is not ſo in dede.}
thy iudgement is right.

%%% 1365
\V Thou haſt commanded
\SNote{The ſame is more confirmed, by theſe three ſynonyma, Iuſtice,
Teſtimonies, Veritie, ſignifying the law of God, moſt earneſtly
commanded.}
iuſtice thy teſtimonies: and thy veritie excedingly.

\V My zele hath made me to pine away: becauſe mine enimies haue
forgotten thy wordes.

\V Thy
\TNote{\L{eloquia}}
word is
\SNote{Gods lavv is as pure as anie thing purged by fire.}
fired excedingly: and thy ſeruant hath loued it.

\V I am
\SNote{A iuſt man is often iudged ignorant, immature, vnexperienced, by
the vvorldlie vviſe:}
a yongman, and contemned: I
\SNote{but in dede is vviſe, in that he forgetteth not to kepe the lavv,
vvhich maketh him iuſt.}
haue not forgotten thy iuſtifications.

\V Thy iuſtice, is iuſtice for euer: and thy law is veritie.

\V Tribulation, and diſtreſſe haue found me: thy commandments are my
meditation.

\V Thy teſtimonies are equitie for euer
\SNote{Hauing profeſſed the neceſſitie of perfect iuſtice, he concludeth
this Octonarie, praying to be illuminated in his vnderſtanding, that ſo
he may attaine iuſtice, and liue therby.}
geue me vnderſtanding, and I shal liue.

%%% o-1254
\LetterNote{Coph.}{Vocation.}

\V I
\SNote{Moſt ſerious and feruent inuocation of God for his grace, is
neceſſarie, to the fulfilling of his lavv.}
haue cried in my whole hart, heare me ô Lord: I wil ſeeke after thy
iuſtifications.

\V I haue cried to thee, ſaue me: that I may keepe thy commandmentes.

\V I haue preuented in
\SNote{I haue preuented the mature, and ordinarie time of the night, and
haue prayed}
maturitie, and
\SNote{very attentiuely.}
haue cried: becauſe I hoped much in thy wordes.

\V Mine eies
\SNote{Againe in the morning I haue preuented the accuſtomed time of
prayer.}
haue preuented early vnto thee: that I might meditate thy
\TNote{\L{eloquia}}
wordes.

\V Heare my voice according to thy mercie ô Lord: and according to
\SNote{According to thy accuſtomed maner of shewing mercie, ſhew it me,
that therby I may liue.}
thy iudgement quicken me.

\V They that perſecute me haue approched to iniquitie: but from thy law
they are made far of.

\V Thou art
\SNote{God is alwayes readie to heare al that ſincerly inuocate  him.}
nigh ô Lord: and al thy wayes are truth.

\V
\SNote{Gods law is the ſame in ſubſtance from the beginning of the
world, and wil be foreuer.}
From the beginning I knewe of thy teſtimonies: that thou haſt founded
them for euer.

%%% 1366
\LetterNote{Res.}{Head.}

\V See
\SNote{An other prayer of the iuſt in affliction.}
my humiliation, and deliuer me: becauſe I haue not forgotten thy law.

\V Iudge my iudgement, & redeme me: for thy
\TNote{\L{eloquiũ}}
word
\SNote{Conſerue me in thy grace.}
quicken thou me.

\V Saluation is far from ſinners: becauſe they haue not ſought after thy
iuſtifications.

\V Thy mercies are manie ô Lord:
\SNote{As before
\XRef{v.~149.}}
according to thy iudgement quicken me.

\V There are manie that perſecute me, and afflict me: I haue not
declined from thy teſtimonies.

\V I ſaw the preuaricatours, and I pyned away: becauſe they kept not thy
\TNote{\L{eloquia}}
wordes.

\V
\SNote{Sincere profeſſion of innocencie is no arrogancy.}
See that I haue loued thy commandmentes ô Lord: in thy mercie quicken
me.

\V The
\SNote{Gods eſſential veritie is the beginning from vvhence, as from the
fountaine al other truthes are deriued:}
beginning of thy wordes is truth:
\SNote{and al commandments proceding from this firſt truth, are for euer
immutable.}
al the iudgementes of thy iuſtice are for euer.

%%% o-1255
\LetterNote{Sin.}{Tooth.}

\V Princes haue perſecuted me
\SNote{Potent wicked men perſecute the godlie without cauſe, that is,
vvithout anie iuſt reaſon mouing them; & vvithout the effect intended by
them, vvhich is to drawe Gods ſeruants from truth, and equitie;}
without cauſe: and my
\SNote{vvhoſe hart being poſſeſſed vvith the true feare of God, they
perſiſt in keping Gods commandments.}
hart hath bene afrayd of thy wordes.

\V I
\SNote{Yea they alſo reioyce in keping the commandments, vvith ſuch
difficultie, as thoſe doe, that gaining the victorie ouer their enimies,
carie avvay great and rich ſpoyles.}
wil reioyce at thy
\TNote{\L{eloquia}}
wordes: as he that findeth manie ſpoyles.

\V I haue hated iniquitie, and abhorred it: but thy law I haue loued.

\V
%%% !!! Originally a SNote
\LNote{Seuentimes in the day.}{Euerie day the iuſt praiſe God often,
ſignified by the number of ſeuen.

From
\MNote{Inſtitution of Canonical Houres by the Church.}
hence alſo the Church of Chriſt tooke example to inſtitute the ſeuen
Canonical Houres, vvhich is the ordinarie Eccleſiaſtical Office;
conſiſting, as S.~Iſidorus, and manie other Fathers teſtifie, of Hymnes,
Pſalmes, Canticles, Antiphones, Leſſons, Reſponſories, & other Prayers &
Praiſes, diſtributed into diſtinct times, beginning in the night,
vvherof that part is called the Nocturne (one or three according to the
diuerſitie of the Office) and perteineth to one or more of the foure
Vigiles, into vvhich ſouldiars diuide the vvhole night. VVherto alſo the
Laudes are added. Then Prime, in the morning. Aftervvards, the Third
houre, Sixt, Ninth; and in the euening, Euenſongue, and Compline.

Againſt
\MNote{This religious inſtitution reprehended by Drowſie Heretikes.}
vvhich moſt ancient and religious Conſtitution, eſpecially
againſt the part called Vigiles, or Nocturnes, certaine Heretikes
repined, and calumniated the Churches cuſtome, as ſuperfluous and
vnfruictful to ſpiritual worke, violating of Gods ordinance, who made
the night for reſt, and the day for laboure. For which cauſe they were
called Nyctazontes, Somnicoloſi, Drowſy heretikes. As the ſame
S.~Iſodorus teſtifieth
\Cite{li.~1. c.~22. de Offic. Eccleſ.}
S.~Ierom
\Cite{Epiſt. ad Riparium,}
noteth the ſame hereſie in
\MNote{Vigilantius.}
Vigilantius, calling him Dormitantius, becauſe he reprehended holie
Vigiles, as if it were better to ſleepe, then wake in time of Diuine
ſeruice.
\MNote{VViclifiſts.}
VViclif alſo raiſed vp the ſame hereſie, as witneſſeth Thomas
VValdenſis.
\Cite{To.~3. Tit.~3. c.~22.}
\MNote{Lutherans.}
Laſtly Luther and al his broode. But the holie obſeruation of Canonical
Houres is proued, by manie ancient Fathers to be altogether agreable to
the holie Scriptures, both of the old and new Teſtament.
\MNote{Approued by S.~Beda.}
So S.~Beda
\Cite{in 18.~Luc.}
&
\Cite{li.~4. c.~7. Hiſt. Angl.}
\MNote{S.~Gregorie.}
S.~Gregorie the Great
\Cite{li.~3. Dialogi. c.~14.}
\MNote{S.~Auguſtin.}
S.~Auguſtin
\Cite{(Ser.~55. de temp.)}
exhorting the people to riſe early to the Vigiles (or Nocturnes) and
in aniewiſe to come to the Third houre, Sixt, and Ninth. Let none (ſaith
he) withdravv himſelfe from the holie vvorke, but vvhom either ſicknes,
or publique vtilitie, or perhaps ſome great neceſſitie holdeth backe.
\MNote{S.~Ierom.}
S.~Ierom
\Cite{Epiſt.~22. ad Euſtoch}
&
\Cite{in Epitaph. Paule c.~10.}
maketh expreſſe mention of the Third houre, Sixt, Ninth, Morning, and
Euening; alſo of
\Fix{Midnighſt,}{Midnight,}{obvious typo, fixed in other}
adding that no Religious is ignorant that ſometimes they muſt riſe to
Diuine Seruice, tvviſe, yea thriſe in the night.
\MNote{S.~Baſil.}
S.~Baſil, in
\Cite{Regulua ſuſius diſput. ad Interrog.~37.}
&
\Cite{de Inſtit. Monachorium}
firſt ſheweth this ordinance to be agreable to the holie Scriptures, and
namely to this place of the Pſalmiſt.
\MNote{S.~Cyprian}
S.~Cyprian
\Cite{in fine expoſit. Orat. Domini}
affirmeth that beſides the three houres in vvhich Daniel and his
felovves prayed, the Church of Chriſt hath added more. And (as manie
ſuppoſe)
\MNote{S.~Clement.}
S.~Clement
\Cite{li.~8. Conſtitut. Apoſtol. c.~40.}
ſhevveth the ſette Houres of prayers, and the reaſons therof: Make your
prayers Early in the morning, at the Third houre; Sixt, Ninth, Euening,
and at the Time of cocke crovving.
\MNote{VVhy publike prayer is conſtituted at theſe houres.}
Early geuing thankes becauſe our Lord hath illuminated vs, the night
being paſſed, & the day coming in; the Third, becauſe that houre our
Lord receiued Pilats ſentence; the Sixt houre, becauſe then he was
Crucified; the Ninth, becauſe al thinges were moued, when our Lord was
crucified, 
abhorring the audacitie of the wicked, & not bearing the
ignominie of our Lord; at Euening, geuing thankes, for that God hath
geuen vs the night for reſt of dayes labours; at the Cocke crovving,
becauſe at that time the coming of the day is denounced, to exerciſe the
vvorkes of
\Fix{lighſt,}{light,}{obvious typo, fixed in other}
thus S.~Clement. Touching the diſtinct and ſette times of publique
prayer, the continual practiſe by tradition teacheth, that Mattines
vvith Laudes vver ſaid in the night, about the firſt Cock
crovving. Prime early in the morning. The other partes in the day time:
At euening Euenſongue, and laſt of al Compline. And touching the place:
\MNote{Not lavvful to goe to Church, nor to pray vvith Infidels.}
If for the infidels (ſaith the ſame holie Father) there be not acceſſe
to the Church, the Biſhop muſt geather the Aſſemblie at home, that the
godlie may not enter into the Church of the vvicked: for the place doth
not ſanctifie man, but man the place. VVherfore if the vvicked occupie
the place, that place is to be ſhunned, becauſe it is prophaned by them:
for as Prieſtes do ſanctifie holie thinges, ſo the vvicked do
contaminate them. If neither at home, nor in the Church Aſſemblies can
be celebrated, let euerie one by himſelfe ſing, read, pray, or tvvo or
three be geathered together.
\CNote{\XRef{Mat.~18.}
\XRef{2.~Cor.~6.}}
For vvhere tvvo or three are geathered in my name (ſaith Chriſt) there
am I, in the middes of them.
\MNote{Not vvith Heretikes.}
Let not the godlie pray with an heretike, no not at home. For
vvhat ſocietie is there of light vvith darknes?}
Seuentimes in the day I haue ſayd prayſe to thee, for the iudgements of
thy iuſtice.

%%% 1368
%%% o-1256
\V There is
\SNote{Amongſt other benefites, it is a ſpecial commoditie, that thoſe
which perfectly loue Gods law, haue alwayes peace in their owne
conſcience:}
much peace to them that loue thy law: &
\SNote{and are neuer ſcandalized, that is, do not fal nor committe
ſinne, by anie occaſion whatſoeuer geuen them by others. For it is a
general aſſured doctrine, that the perfect are not ſcandalized, becauſe
they are conſtant in vertue, and not moued by anie example, perſwaſion,
prouocation, or other meanes to offend God: but only the weake and
vnperfect are moued, and drawne to ſinne by occaſions geuen them, who
otherwiſe would not haue ſinned. Neither is he excuſed that falleth by
ſuch occaſions, becauſe he ought to be conſtant.}
there is no ſcandal to them.

\V I expected thy ſaluation ô Lord: and haue loued thy commandmentes.

\V My ſoule hath kept thy teſtimonies: and
\SNote{Not of ſeruile feare, but of true charitie & filial loue.}
hath loued them
\Fix{excedindgly.}{excedingly.}{obvious typo, fixed in other}

\V I haue kept thy commandmentes, and thy teſtimonies:
%%% !!! Snote not marked in either
\SNote{Becauſe whatſoeuer I do is in thy ſight, whom I wil in no caſe
offend.}
becauſe al my waies are in thy ſight.

\LetterNote{Tau.}{Signe.}

\V
\SNote{In this laſt Octonarie, and concluſion of this Pſalme, the
faithful ſeruant of God prayeth,}
Let my petition approch in thy ſight ô Lord: according to thy word
giue me
\SNote{not for humane knowlege, or other temporal thinges, but to
vnderſtand Gods law.}
vnderſtanding.

\V Let my requeſt enter in thy ſight:
\SNote{Thou that haſt promiſed to heare al that inuocate thee,}
according to thy word
\SNote{\Fix{vouchſaffe}{vouchſafe}{likely typo, fixed in other}
to deliuer me in time of tribulations and tentations.}
deliuer me.

\V My lippes shal vtter an hymne, when thou shalt teach me thy
iuſtifications.

\V My tongue shal pronounce thy
\TNote{\L{eloquiũ}}
word:
\SNote{God geuing grace, it behoueth his ſeruants thankefully to ſerue,
and praiſe him.}
becauſe al thy commandmentes are equitie.

%%% o-1257
\V
\SNote{Though man be indued with grace, yet he nedeth more grace, that
by Gods hand and powre, not by his owne, he may reſiſt tentations.}
Let thy hand be to ſaue me: becauſe I haue choſen thy commandmentes.

\V I haue coueted thy ſaluation ô Lord: and thy law is my meditation.

\V
\SNote{By this aſſiſtance of grace, the ſoule continueth in ſpiritual
life, and praiſeth God.}
My ſoule shal liue, and shal prayſe thee: and thy iudgementes shal helpe
me.

\V
\SNote{Al mankind, and vniuerſally euerie one hath bene as a loſt
ſheepe,}
I haue ſtrayed, as a sheepe, that is loſt:
\SNote{and Chriſt came into this world to ſeeke and ſaue al:}
ſeeke thy ſeruant,
\SNote{but effectually findeth and ſaueth thoſe only, that forgette not
to kepe his commandments.}
becauſe I haue not forgotten thy commandmentes.


\stopChapter


\stopcomponent


%%% Local Variables:
%%% mode: TeX
%%% eval: (long-s-mode)
%%% eval: (set-input-method "TeX")
%%% fill-column: 72
%%% eval: (auto-fill-mode)
%%% coding: utf-8-unix
%%% End:
