%%%%%%%%%%%%%%%%%%%%%%%%%%%%%%%%%%%%%%%%%%%%%%%%%%%%%%%%%%%%%%%%%
%%%%
%%%% The (original) Douay Rheims Bible 
%%%%
%%%% Old Testament
%%%% Psalmes
%%%% Psalme 118 note
%%%%
%%%%%%%%%%%%%%%%%%%%%%%%%%%%%%%%%%%%%%%%%%%%%%%%%%%%%%%%%%%%%%%%%


\startcomponent psalme-118-note


\project douay-rheims


%%% 1349
%%% o-1239
\startChapter[
  title={\Sc{General Annotations vpon this CXVIII Pſalme.}}
  ]

As
\MNote{The obſcuritie of this profound Pſalme appeareth not to the
vulgar reader.}
this Pſalme is the longeſt in the whole Pſalter, ſo it ſemeth to the
ancient Fathers moſt profound in ſenſe. And ſo much the harder to be
vnderſtood, becauſe alſo the very hardnes therof lieth hidden, which in
diuers other Pſalmes, and partes of holie Scripture, eaſily appeareth to
the reader. But here the wordes being clere, and the ſenſe alſo plaine
and eaſie in ſome pointes of doctrine, yet the more diligence is
imployed, the more difficultie is found in ſearching the whole ſenſe and
\Fix{mearning}{meaning}{obvious typo, fixed in other}
of euerie word and ſentence, with the maner obſerued in compoſing it,
and the frequent repetition of the ſame or like wordes.
\MNote{S.~Auguſtin differred the explication of this Pſalme.}
Al which maturely conſidered cauſed that great Clerke, and light of the
Church S.~Auguſtin, to omitte this Pſalme, when he explicated al the
reſt.
\MNote{Omitted to diſcuſſe one difficultie.}
And when at laſt he added alſo this, he wittingly omitted one ſpecial
difficultie, which he doubted not, to be conteyned in the maner of
compoſing it, not only by order of the Hebrew Alphabet, as diuers more
Pſalmes, and ſome other partes of holie Scripture, but more artificially
then anie other, the firſt eight verſes al beginning with the firſt
letter Aleph; the next eight, with the ſecond letter Beth; and ſo to the
laſt of the two and twentie letters. Of which omiſſion he yeldeth this
only reaſon, becauſe he found
%%% o-1240
nothing (as he humbly affirmeth) that might properly perteyne
therunto. Confeſſing alſo expreſly that whenſoeuer he applied his
cogitations to expound the text itſelf, it alwayes exceded his
habilitie.
\MNote{At laſt made 32.~ſermons in explication therof.}
But finally to ſatisfie the often and earneſt requeſt of his bretheren
and freindes, truſting (as alvvayes) in Gods ſpecial helpe, he largely
expoundeth it, in thirtie two diſtinct Sermons.

%%% 1350
S.~Ambroſe
\MNote{S.~Ambroſe writte 22.~ſermons vpon this Pſalme.}
alſo moued with like pietie, made two and twentie Sermons in expoſition
of this Pſalme. Affirming in his Prologue, that amongſt other Pſalmes,
eſpecially this ſheweth
\MNote{King Dauid a great maſter of moral doctrin.}
how great a maſter king Dauid was of moral good life. For al moral
doctrine, being of his owne nature ſwete, yet moſt delighteth the eares,
and gently toucheth the minde, being vttered, as here it is, with
pleaſantnes of verſe, and ſwetenes of ſongue. Againe whereas this Royal
Prophet in manie places of this booke, powreth out ſentences of moral
pſalmes or ſongues, as bright ſtarres, that ſhine and gliſter to al the
world, here moſt excellently he produceth a more ſingular mirrhor, as
the ſunne, of ful light, burning with meridian heate.
\MNote{VVhy this Pſalme was compoſed in order of the Alphabet.}
And for the profite of al, the better to draw our attentions, to lerne
that we may, though we can not attaine to al that we vvould, he diſpoſed
this Pſalme through al the Alphabeth: that as children beginning vvith
the firſt letters, make entrance to further knovvlege: ſo by the ſame
beginninges vve ſhould lay the firſt foundation, and therupon procede in
our ſpiritual building, tovvards perfection in good life, the true
ſeruice of God.
\MNote{VVhy eight verſes are begunne with euerie letter.}
VVhich is yet further inſinuated (as the ſame Doctor teacheth) by the
eight verſes continually beginning vvith the ſame letter, and ſo other
eight in order through the vvhole Alphabet, to ſignifie that after ſeuen
dayes trauel in this temporal life, vve may come to that vnitie, vvhich
vve expect in the eight day of reſurrection, vvhen vve hope to riſe
reuiued in our Lord \Sc{Iesvs}, in nevvnes of eternal life.

Likevviſe S.~Baſil in the Argument of this Pſalme admoniſheth, that
vvheras holie Dauid, according to diuers ſtates, vvhich he paſſed,
vvritte diuers Pſalmes: as vvhen he fled from his enimies, vvhen he
lamented his diſtreſſes, mourned in penſiuenes, enioyed peace and
comforte, ranne a right courſe of vertue, fel from God by ſinne, &
againe returning obſerued Gods lavves:
\MNote{S.~Baſils iudgement that this Pſalme conteyneth the argument of
manie Pſalmes.}
in this one Pſalme he comprehendeth al his prayers made to God at
ſundrie times, & here propoſeth the ſame, as a certaine profitable moral
doctrine, to al ſortes and ſtates of men. Neither doth he pretermite
doctrinal pointes of faith, but interpoſeth them alſo with moral
documents, in ſuch ſorte, that this one Pſalme may ſuffice to teach the
vvel diſpoſed, hovv to attaine to perfection in vertue, to ſturre vp the
ſlouthful vnto diligent care of their ſoules, to recreate the deſolate
vvith ſpiritual conſolations, & briefly it adminiſtereth al kinde of
medicine, to the diuers paſſions of mortal men.

For
\MNote{Other expoſitors of this Pſalme.}
the like iudgements of other Fathers vve remitte the lerned reader, to
S.~Hilarie, Theodoret, Proſper, Arnobius, Caſſiodorus, Beda, Enthymius,
and others, but can not wel omitte a brief inſtruction of S.~Ierom. VVho
in his
\Cite{Epiſtle to Paula Vrbica:}
not only ſheweth the interpretation of the two and twentie letters, but
alſo explicateth their ſenſe in this place, by connecting them into
certaine ſhorte ſentences,
%%% !!! MNote should go with table below
\MNote{S.~Ieroms interpretation, and explication of the Hebrew
Alphabet.}
%% !!! MNote is with 'the fourth connexion' in both. Why?
\SNote{Moſt of theſe letters haue alſo other ſignifications. And are
diuerſly explicated by S.~Ambroſe, S.~Beda, and others. VVherby we may
lerne (though we vnderſtand no more) that holie 
Scriptures are ful of myſteries (as S.~Ierom calleth this) and hard to
be vnderſtod.}
in this maner.

\placetable[force,none]{}{%
\starttabulate[|c|c|c|c|]
\NC Aleph \NC Beth \NC Gimel \NC Daleth \NC \NR
\NC \L{Doctrina} \NC \L{Domus} \NC \L{Plenitudo} \NC \L{Tabularum} \NC \NR
\NC Doctrine \NC Of the houſe \NC Fulneſſe \NC Of tables \NC \NR
\stoptabulate
}

VVhich is the firſt connexion, ſignifying that the doctrine of the
houſe, that is, the Church of God, is found in the fulnes of diuine
bookes.

%%% !!! Look up how to center text
\hfill The ſecond connexion is: \hfill
\placetable[force,none]{}{%
\starttabulate[|c|c|c|c|]
\NC He \NC Vau \NC Zain \NC Heth \NC \NR
\NC \L{Iſta} \NC \L{Et}  NC \L{Hæc} \NC \L{Vita} \NC \NR
\NC This thing \NC And \NC This \NC Life \NC \NR
\stoptabulate
}

For what other life can there be without knowledge of Scriptures? wherby
alſo Chriſt is knowen, who is the life of them that beleue in him.

%%% 1351
%%% o-1241
%%% !!! Look up how to center text
\hfill The
%%% !!! ???
\CNote{\L{Idem Prœm. lamen.}}
third connexion is: \hfill
\placetable[force,none]{}{%
\starttabulate[|c|c|]
\NC Teth \NC Iod \NC \NR
\NC \L{Bonum} \NC \L{Principium} \NC \NR
\NC Good \NC Beginning \NC \NR
\stoptabulate
}

Albeit we now could know al thinges which are written,
\CNote{\XRef{1.~Cor.~13.}}
yet we know but in part, and in part we prophecie: for we ſee now by a
glaſſe, in a dark ſort, but when we ſhal be worthie to be with Chriſt,
and ſhal be like to Angels, then doctrine of bookes ſhal ceaſe, and then
we ſhal ſee face to face: the
\TNote{God in himſelf.}
Good Beginning, euen as he is.

%%% !!! Look up how to center text
\hfill The fourth connexion is: \hfill
\placetable[force,none]{}{%
\starttabulate[|c|c|]
\NC Caph \NC Lamed \NC \NR
\NC \L{Manus} \NC \L{Diſciplinæ, ſiue cordis} \NC \NR
\NC The hande \NC Of diſcipline, or of hart \NC \NR
\stoptabulate
}

The handes are vnderſtood in worke, hart and diſcipline are vnderſtood
in ſenſe or meaning, becauſe we can not rightly doe anie thing, vnles
vve firſt knovv vvhat thinges are to be donne.

%%% !!! Look up how to center text
\hfill The fift connexion is: \hfill
\placetable[force,none]{}{%
\starttabulate[|c|c|c|]
\NC Mem \NC Nun \NC Samech \NC \NR
\NC \L{Ex ipſis} \NC \L{Semptiernum} \NC \L{Adiutorum} \NC \NR
\NC Of them \NC Euerlaſting \NC Helpe \NC \NR
\stoptabulate
}

This needeth not explication, for it is manifeſt as the light, that from
Scriptures are eternal helpes.

%%% !!! Look up how to center text
\hfill The ſixt connexion is: \hfill
\placetable[force,none]{}{%
\starttabulate[|c|c|c|]
\NC Ain \NC Phe \NC Sade \NC \NR
\NC \L{Fons, ſiue Oculus} \NC \L{Oris} \NC \L{Iuſtitiæ} \NC \NR
\NC Fountaine, or eye \NC Of the mouth \NC Of iuſtice \NC \NR
\stoptabulate
}

According to that vvhich vve haue expounded in the fourth connexion:
that dedes and intention muſt concurre.

%%% !!! Look up how to center text
\hfill The ſeuenth connexion vvhich is laſt, in vvhich number of ſeuen
is alſo myſtical vnderſtanding: \hfill
\placetable[force,none]{}{%
\starttabulate[|c|c|c|c|]
\NC Coph \NC Res \NC Shin \NC Tau \NC \NR
\NC \L{Vocatio} \NC \L{Capitis} \NC \L{Dentium} \NC \L{Signa} \NC \NR
\NV Vocation \NC Of the head \NC Of teeth \NC Signes \NC \NR
\stoptabulate
}

Diſtinct voice is produced by the teeth, & in theſe ſignes vve come to
the Head of al, vvhich is Chriſt, by vvhom vve haue acceſſe to the
euerlaſting kingdom.

Or thus (not tranſpoſing the vvordes) By vocation of Chriſt the Head,
throught diſtinct voice of ſignes (for vvordes are ſignes ſhevving the
mind) vve are conducted to the eternal kingdom, the happines vvhich al
men deſire.

VVhat I pray thee (ſaith this holie Doctor) is more ſacred then this
myſterie, vvhat more pleaſant then this delight? VVhat meate, & vvhat
honey are ſvveeter, then to knovv Gods vviſdom; to enter into his
ſecreete cloſſet; to behold the ſenſe of our Creator; and to teach the
vvordes of thy Lord God, ful of ſpiritual vviſdom, vvhich are derided
by the vviſe of this vvorld.

VVe
\MNote{Gods lavv eſpecially commended in this Pſalme.}
muſt alſo aduertiſe the reader of the like diſcourſes of ancient Fathers
(ouer long to be here recited) concerning the manifold hiegh praiſes of
Gods Lavv conteyned in this Pſalme, vvith frequent repetitiõ of certaine
Synonyma vvordes ſignifying the ſame thing, in al fourtene, to vvitte:
\MNote{14.~Synonyma ſignifying the lavv of God.}
The Lavv of God, his VVayes, Teſtimonies, Commandments, Precepts,
Statutes, Iuſtifications, Iudgements, Iuſtice, Equitie, Veritie,
VVordes, Speaches, & Sermons: of vvhich there is commonly one in euerie
verſe, and ſomtimes tvvo or three in the ſame verſe. But our Engliſh
tongue hardly ſufficing rightly to diſtinguiſh the three
%%% 1352
laſt, which in latin are \L{Verba, Eloquia, Sermones}, we tranſlate
\Sc{VVordes} only, adding in the margen, \L{Eloquia}, and \L{Sermones},
when they occurre.

Leauing therfore larger commentaries to others, we ſhal proſecute our
wonted maner of briefe gloſſes. Only here premoniſhing the diligent
readers, eſpecially Clergimen (our ſelues and our brethren) who euerie
day ſing or read this whole
%%% o-1242
Pſalme in the Canonical houres, to obſerue
two particular pointes of Chriſtian doctrine, euidently proued by manie
places of this Pſalme.
\MNote{Gods grace neceſſarie in euerie good vvorke.}
The one againſt the Pelagians hereſie, denying the neceſſitie of Gods
ſpecial grace in meritorious workes. For the Pſalmiſt often here
inculcateth mans inſufficiencie, that of himſelfe, and by natural
forces, he can not kepe the commandments of God, but needeth alwayes the
particular grace of God, as vvel to beleue in him, to repent for ſinnes,
and to beginne good vvorkes; as to procede, and perſeuere in good ſtate
to the end.
\MNote{It enableth freevvil to merite.}
The other againſt the hereſie of our time, denying merite by grace &
freewil. For here it is alſo manifeſt, that Gods grace maketh man able, to kepe
his commandments, and by keeping them to become iuſt in this life, and
ſo to merite eternal glorie. Sundrie other principal Articles of
Chriſtian Catholique Religion are likevviſe compriſed in this one
Pſalme: but eſpecially Moral doctrin.


\stopChapter


\stopcomponent


%%% Local Variables:
%%% mode: TeX
%%% eval: (long-s-mode)
%%% eval: (set-input-method "TeX")
%%% fill-column: 72
%%% eval: (auto-fill-mode)
%%% coding: utf-8-unix
%%% End:
