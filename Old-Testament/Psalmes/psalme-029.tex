%%%%%%%%%%%%%%%%%%%%%%%%%%%%%%%%%%%%%%%%%%%%%%%%%%%%%%%%%%%%%%%%%
%%%%
%%%% The (original) Douay Rheims Bible 
%%%%
%%%% Old Testament
%%%% Psalmes
%%%% Psalme 29
%%%%
%%%%%%%%%%%%%%%%%%%%%%%%%%%%%%%%%%%%%%%%%%%%%%%%%%%%%%%%%%%%%%%%%




\startcomponent psalme-29


\project douay-rheims


%%% 1195
%%% o-1087
\startChapter[
  title={Psalme 29}
  ]

\PSummary{King
\MNote{Dauid rendereth thankes for his eſtabliſhment in his kingdome.

The 8.~key.}
Dauid by voice and inſtrument rendereth thankes to God for his
peacable ſtate in the kingdom, 5.~inuiteth others to reioyce in Gods
benefites, teaching by his owne example that God ſometimes geueth more
\Fix{conforth}{comfort}{likely typo, fixed in other}
ſometimes sheweth his wrath, but al for our good.}

\PTitle{A
\SNote{The general name of Pſalme common to this whole booke conteyning
in al 150. is more particularelie appropriated to ſome, which more
ſpecially were playde vpon muſical inſtruments as on the Pſalter, Harpe,
&c. Others are called Canticles, which were moſt vſuallie ſongue with
humaine voices. So this, called a \Emph{Pſalme of Canticle}, ſignifieth
that voyces begane the muſike and inſtruments were adioyned. As
contrariwiſe others are called \Emph{Canticles of Pſalmes}, where
inſtruments begane and voices folowed.}
Pſalme of Canticle,
\SNote{After manie great tribulations, King Dauid proſpering built an
excellent houſe or palace.
\XRef{(2.~Reg.~5. v.~11.}
\XRef{Paralip.~14. v.~1.)}
And at his firſt dwelling therein, made this Pſalme, beginning himſelfe
to ſing the ſame with voice, other muſitians ioyned with him in the
praiſes of God and thankeſgeuing for his benefites.}
in the dedication of Dauids houſe.}

%%% o-1088
\VV I wil
\SNote{Though God in himſelfe is moſt high, and neither nedeth, nor can
be exalted by men, yet the royal prophet knew it vvas his dutie to ſing
thankes and praiſes to him,}
exalt thee ô Lord,
\SNote{for his deliuerie from manie trubles, and dangers.}
becauſe thou haſt receiued me: neither haſt
\SNote{Not ſuffering his enemies to be delighted in his ruine.}
delighted myne enemies ouer me.

\V Ô Lord my God I haue cried to thee, and thou haſt
\SNote{Conſerued my bodie in health amõgſt innumerable dangers.}
healed me.

\V Lord thou haſt
\SNote{Preſerued my ſoule from greater dãgers of ſinnes, and ſo from
hel.}
brought forth my ſoule out of hel: thou haſt ſaued me from them that goe
downe into the lake.

\V Sing to our Lord
\SNote{Ye that are iuſt and holie praiſe God for it, from vvhom it
cometh, and not from your ſelues:}
ye his ſainctes: and
\SNote{confeſſe his mere goodnes vvithout your deſertes.}
confeſſe to the memorie of his holines.

\V Becauſe
\SNote{VVhen he is angrie,}
wrath is in his indignation:
\SNote{yet he meaneth vvel vnto vs.}
and life in his wil.

%%% 1196
At
\SNote{The ſtate of a iuſt mans life is often changed from ſorovv to
comforte, and from conforte to ſorovv.}
euening shal weeping abide: and in the morning gladneſſe.

\V And I ſaid in my abundance:
\SNote{Though vve ſuppoſe our ſelues firmly eſtablished:}
I wil not be moued for euer.

\V Ô Lord
\SNote{yet God of his good vvil tovvardes vs ſometimes
geueth \Emph{ſtrength}, and corege,}
in thy wil, thou haſt geuen ſtrength to my beautie. Thou haſt
\SNote{ſometimes ſuffereth vs to our ovvne vveakenes,}
turned away thy face from me, and I became trubled.

\V To thee ô Lord
\SNote{therfore we muſt ſtil \Emph{crie and pray} for Gods helpe,}
I wil crie: and I wil pray to my God.

\V
\SNote{in manner here expreſſed or the like.}
What profite is in my bloud, whiles I deſcend into corruption?

Shal duſt confeſſe to thee, or declare thy truth?

\V Our Lord hath heard, and had mercie on me: our Lord is become my
helper.

\V Thou haſt turned my mourning into ioy vnto me: thou haſt cut my
ſackcloth, and haſt compaſſed me with gladnes.

\V That
\SNote{Finally in this my good ſtate}
my glorie may ſing to thee: and I be not compunct: Lord my God for euer
\SNote{I ſhal alvvayes confeſſe and praiſe thee.}
wil I confeſſe to thee.



\stopChapter


\stopcomponent


%%% Local Variables:
%%% mode: TeX
%%% eval: (long-s-mode)
%%% eval: (set-input-method "TeX")
%%% fill-column: 72
%%% eval: (auto-fill-mode)
%%% coding: utf-8-unix
%%% End:
