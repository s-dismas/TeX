%%%%%%%%%%%%%%%%%%%%%%%%%%%%%%%%%%%%%%%%%%%%%%%%%%%%%%%%%%%%%%%%%
%%%%
%%%% The (original) Douay Rheims Bible 
%%%%
%%%% Old Testament
%%%% Psalmes
%%%% Psalme 13
%%%%
%%%%%%%%%%%%%%%%%%%%%%%%%%%%%%%%%%%%%%%%%%%%%%%%%%%%%%%%%%%%%%%%%




\startcomponent psalme-13


\project douay-rheims


%%% 1167
%%% o-1059
\startChapter[
  title={Psalme 13}
  ]

\PSummary{After
\MNote{Of Chriſts Incarnation. The 5.~key.}
general groſſe ignorance and impiety in the world,
  7.~Chriſt shal be incarnate, the Redemer of mankind.}

\PTitle{Vnto the end, the Pſalme of Dauid.}

\VV The
\CNote{\XRef{Pſal.~52.}}
\SNote{Wicked men drowned in ſinne are at laſt ſo beſotted in their
  vnderſtanding, that they thinke in their hart (though they dare not
  vtter it) that there is \Emph{no God}: that is, none that hath diuine
  prouidence in gouerning the world, nor that wil iudge al in the end.}
foole hath ſaid in his hart: There is no God. They are corrupt, and are
become
\SNote{Defiled with al ſortes of ſinne,}
abominable in their ſtudies: there is
\SNote{not only the moſt wicked, but alſo al mankind were vnable without
a Redemer to do good.}
not that doth good
\LNote{No not one.}{S.~Paul
\MNote{VVithout Chriſts grace no man is nor can be iuſt.}
by this place, and the like
\XRef{(Iſaie.~59. v.~7.)}
confirmeth his doctrin
\XRef{(Rom.~3.)}
that both the Iewes and the Gentils (meaning al mankind) were in that
ſtate, that none, no not one without the grace of Chriſt, were iuſt, nor
could be iuſtified, nor ſaued by the law of Nature, nor of
Moyſes. VVhich proueth the neceſſitie of faith. But neither that only
faith iuſtifieth, nor that the iuſteſt are ſtil wicked, as Caluin and
Beza falſly expound theſe Scriptures. For the Prophets and S.~Paul
ſpeake in theſe places of men before they be iuſtified, teaching that al
mankind was once in ſinne, and none could be iuſtified but by
Chriſt. Neuertheleſſe they teach alſo that men being iuſtified muſt, and
may \Emph{ſerue iuſtice vnto ſanctification}. And that their workes are
not then vnprofitable.
\MNote{The law ſhewed the inſufficiencie of mans wil.}
\Emph{For being made free from ſinne} (ſaith the ſame Apoſtle to the
Romanes
\XRef{c.~6.})
\Emph{and become ſeruants to God, you haue your fruict, vnto
ſanctification, and the end is life euerlaſting.} VVhich point of
doctrin, how man is iuſtified, S.~Auguſtin excellently, & briefly
explicateth
\Cite{(li.~1. de Spiritu et lit. c.~9.)}
in theſe wordes:
\MNote{Grace cureth the wil.}
The iuſt are \Emph{iuſtified freely by (Chriſt) his grace}, they are not
therfore purified by the lavv: they are not iuſtified by their proper
wil, \Emph{but iuſtified freely by (Chriſt) his grace}.
\MNote{The wil being cured cooperateth with grace.}
Not that it is done without our wil, but by the law our wil is ſhewed
weake, that grace might cure the wil, and the wil being cured might
fulfil the law, not being vnder the law, nor needing the law.

VVherto we may here adde (and ſo ſaue labour of repeting this in
other places) an other document of the ſame Doctor, in the ſame booke
\Cite{(de ſpirit & lit. c.~27.)}
\MNote{Venial ſinnes exclude not from heauen.}
that the iuſt do not liue without ſome ſinnes, and yet remaine in ſtate
of ſaluation:
\MNote{Good workes done in mortal ſinnes auail not to ſaluation.}
the wicked do ſometimes certaine good workes, & ſtil remaine in ſtate of
damnation. For euen as (ſaith he) venial ſinnes without which this life
is not ledde, do not exclude the iuſt from eternal life: ſo certaine
good workes, without which the life of the very worſt is hardly found,
profite nothing the vniuſt man to eternal ſaluation, but in euerlaſting
damnation, ſome ſhal haue more and ſome leſſe torment.}
no not one.

%%% o-1060
\V
\CNote{\XRef{Rom.~3.}}
Our Lord hath looked forth from heauen vpon the children of men, to ſee
if there be that vnderſtandeth, and ſeeketh after God.

%%% 1168
\V Al haue declined, they are become
\SNote{Without faith in Chriſt none had meritorious workes.}
vnprofitable together: there is not that doth good, no not one.

Their
%%% !!! Some other kind of note maybe? Placement?
\TNote{Theſe three verſes being not in the Hebrew, nor Greke, yet are in
the
\Cite{Engliſh an.~1577.}
and are three diſtinct verſes in other pſalmes
\XRef{5.}
\XRef{9.}
&
\XRef{35.}}
throte is an open ſepulchre: with their tongues they did deceitfully,
the poyſon of aſpes vnder their lippes.

Whoſe mouth is ful of curſing and bitterneſſe: their feete ſwift to
sheed bloud.

Deſtruction
%%% !!! These two notes really should be before 'deſtruction'.
\CNote{Iſa.~59. v.~8.}
\SNote{They are wholly occupied in vexing others.}
and infelicitie in their waies, and the way of peace they haue not
knowen: there is no feare of God before their eies.

\V
\SNote{The Prophet ſpeaketh this in the perſon of God.}
Shal not al they know that worke iniquitie, that deuoure my people
\SNote{With greedines to hurt the good.}
as foode of bread?

\V They haue not inuocated our
\Fix{Lard,}{Lord,}{obviious typo, fixed in other}
\SNote{Not beleeuing in God, they feared Idols, that is, diuels:}
there haue they trembled for feare,
\SNote{who in dede can not hurt Gods ſeruants.}
where no feare was.

\V Becauſe our Lord is in
\SNote{Though innumerable be very wicked, yet ſome are iuſt.}
the iuſt generation, you haue
\SNote{Mocked and derided thoſe that truſt in God.}
confounded the counſel of the poore man: becauſe our Lord is his hope.

\V
\SNote{The Prophet wisheth, and withal prophecieth that Chriſt our
Sauiour wil come, who is promiſed to Iſrael.}
Who wil geue from Sion the ſaluation of Iſrael? when our Lord shal haue
\SNote{Redemed man from the captiuitie of the diuel,}
turned away the captiuitie of his people,
\SNote{thoſe that ſupplant vice,}
Iacob shal reioyce, and
\SNote{and contemplate God.}
Iſrael shal be glad.


\stopChapter


\stopcomponent


%%% Local Variables:
%%% mode: TeX
%%% eval: (long-s-mode)
%%% eval: (set-input-method "TeX")
%%% fill-column: 72
%%% eval: (auto-fill-mode)
%%% coding: utf-8-unix
%%% End:
