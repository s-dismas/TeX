%%%%%%%%%%%%%%%%%%%%%%%%%%%%%%%%%%%%%%%%%%%%%%%%%%%%%%%%%%%%%%%%%
%%%%
%%%% The (original) Douay Rheims Bible 
%%%%
%%%% Old Testament
%%%% Psalmes
%%%% Psalme 146
%%%%
%%%%%%%%%%%%%%%%%%%%%%%%%%%%%%%%%%%%%%%%%%%%%%%%%%%%%%%%%%%%%%%%%




\startcomponent psalme-146


\project douay-rheims


%%% 1396
%%% o-1287
\startChapter[
  title={Psalme 146}
  ]

\PSummary{God
\MNote{Gods excellencie in creating and gouerning the vvorld.

The 2.~key.}
is alſo to be praiſed by his peculiar people, for particular benefites,
4.~& for his omnipotent powre, wiſdom, goodnes, in creating, and
gouerning this whole world, 11.~and moſt ſpecial benignitie towards
thoſe that truſt in him.}

\PTitle{Alleluia.}

\NV Prayſe ye our Lord becauſe Pſalme
\SNote{It is good to ſing Pſalmes of praiſe to God.}
is good: to our God let there be pleaſant, and comelie praiſe.

\V Our Lord building vp Ieruſalem:
\SNote{A prophecie of the reſtaoration of Ieruſalem after the
captiuitie.}
wil gather together the diſperſions of Iſrael.

\V Who
\SNote{Remitteth ſinnes to the penitent.}
healeth the contrite of hart: and bindeth vp their ſores.

\V Who
\SNote{Beſides experience of euerie one, that ſhal behold the firmament
in a clere night, the holie Scripture
\XRef{(Gen.~15. v.~5.)}
ſheweth, that the ſtarres are innumerable to man. For albeit Ptolomæy
and other Aſtronomers numbereth certaine more notorious ſtarres, which
ſerue eſpecially for ſome knowlege in the ſcience of Aſtronomie,
numbering 349. ſuch in the Zodiach; 316. in the South part therof; and
360. on the North part, which are in al 1025. Yet al acknowlege that no
man can come nere to anie probable coniecture of the whole number, nor
is able to attaine anie perfect knovvlege of their natural influences,
and ſpecial proprieties. And therfore the Pſalmiſt propoſeth here the
admirable, and vnſearchable knovvlege of God: who both moſt exactly
knovveth the number,}
numbereth the multitude of ſtarres: and
\SNote{and ſo perfectly their nature, that his diuine Omniſcience geueth
to euerie ſtarre a proper name, according to their ſingular differences
and proprieties.}
geueth names to them al.

%%% 1397
\V Great is our Lord, and great is his ſtrength: and of his wiſdom there
is no
\SNote{Thinges ſubiect to Gods knovvlege and vviſdom are innumerable.}
number.

\V Our Lord receiuing the meeke: & humbling ſinners euen to the ground.

\V Sing ye to our Lord in confeſſion: ſing ye to our God on harpe.

\V Who
\SNote{Al theſe and the like benefites do ſhevv Gods
\Fix{imcomparable}{incomparable}{possible typo, fixed in other}
greatnes, vviſdom, and goodnes.}
couereth the heauen with cloudes: and prepareth rayne for the earth.

Who bringeth forth graſſe in the mountaines: and herbe for the ſeruice
of men.

\V Who geueth to beaſtes their foode: and to
\SNote{Both ſacred and prophane auctors teſtifie, that rauens ſeing
their yong ones, either vvithout fethers, or to haue vvhitiſh, vnlike to
theirs, as ſuſpecting that they are not their ovvne birdes, but of ſome
other kinde, leaue them deſtitute of meate; therfore God the auctor of
nature, and conſeruer of al kindes of creatures, by his ſpecial
prouidence, feedeth them: either by a certaine dew, hanging neere them
in the ayre, as
%%% !!! Cite?
Iſidorus ſuppoſeth; or by litle beaſtes, or flees, ſent by Gods
prouidence, vvhich they catching into their mouthes, are nouriſhed and
brought vp, as S.~Chryſoſtom teacheth,
\Cite{ſer. in Heliam;}
or by vvhat other meanes ſoeuer, al agree that yong rauens are neglected
by their parents, and are fedde meruelouſly by Gods ordinance; by vvhich
example the Pſalmiſt ſhevveth, that much more God hath care of men:
eſpecially of ſuch men (ſaith
\CNote{\Cite{Ho. in hunc Pſalm.}}
S.~Chryſoſtom) as honour him vvith hymnes and praiſes, vvhom alſo he
hath called to be his peculiar people, and his ovvne portion or
inheritance.}
the young rauens that cal vpon him.

%%% o-1288
\V He shal not haue pleaſure in the ſtrength of an horſe: nor in the
legges of a man shal he be wel pleaſed.

\V Our Lord is wel pleaſed toward them that feare him: and in them, that
hope vpon his mercie.


\stopChapter


\stopcomponent


%%% Local Variables:
%%% mode: TeX
%%% eval: (long-s-mode)
%%% eval: (set-input-method "TeX")
%%% fill-column: 72
%%% eval: (auto-fill-mode)
%%% coding: utf-8-unix
%%% End:
