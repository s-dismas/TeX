%%%%%%%%%%%%%%%%%%%%%%%%%%%%%%%%%%%%%%%%%%%%%%%%%%%%%%%%%%%%%%%%%
%%%%
%%%% The (original) Douay Rheims Bible 
%%%%
%%%% Old Testament
%%%% Psalmes
%%%% Psalme 009
%%%%
%%%%%%%%%%%%%%%%%%%%%%%%%%%%%%%%%%%%%%%%%%%%%%%%%%%%%%%%%%%%%%%%%




\startcomponent psalme-009


\project douay-rheims


%%% 1162
%%% o-1054
\startChapter[
  title={Psalme 9}
  ]

\PSummary{The
\MNote{Gods prouidence in protecting the good and permitting euil. The
  3.~key.}
Church prayeth God for her protection, 4.~in repelling the
  enemies force, 8.~in punishing the wicked, and rewarding the iuſt.}

\PTitle{Vnto the end, for the
\SNote{Chriſts coming in humilitie, and Chriſtians afflictions, are
hidden from the world, in Gods prouidence.}
ſecrets of the ſonne, the Pſalme of Dauid.}

\VV I wil
\SNote{Geue thanks,}
confeſſe to thee Ô Lord with al my hart: I wil tel al thy meruelous
thinges.

\V I wil be
\SNote{in mind,}
glad and
\SNote{and bodie.}
reioyce in thee: I wil ſing to thy name Ô moſt High.

\V In
\SNote{God repelleth the enemie, when man is not able to reſiſt.}
turning mine enemie backward: they shal be weakened, and perish before
thy face.

\V Becauſe thou haſt done
\SNote{A iuſt man doth his endeuour, not of him ſelfe, but by Gods grace
ouercometh the enemie.}
my iudgement and my cauſe: thou haſt ſitte vpon the throne which iudgeſt
iuſtice.

\V Thou haſt rebuked the
\SNote{Al ſinners called gentils, becauſe they were generally accounted
wicked.}
Gentiles, and the impious hath perished: their
\SNote{The vaine glorious fame of ſinners partly decayeth in this world
but moſt eſpecially in the world to come.}
name thou haſt deſtroyed for euer, and for euer and euer.

\V The ſwordes of the enemie haue fayled vnto the end: and their cities
thou haſt deſtroyed.

\V Their memorie hath perished with a ſound: and our Lord abideth for
euer.

%%% o-1055
He hath prepared his throne in
\SNote{Iudicial ſeates of men are often corrupted but Gods neuer.}
iudgement: \V & he wil iudge the whole world in equitie, he wil iudge
the people in iuſtice.

\V And our Lord is made a refuge for the poore: an helper
\SNote{God doth not preſently deliuer the good from affliction: but when
it is to their ſpiritual profitte.}
in opportunities, in tribulation.

%%% 1163
\V And let them hope in thee that know thy name: becauſe thou haſt not
forſaken them that ſeeke thee Ô Lord.

\V Sing to our Lord, which dwelleth in Sion: declare his
\SNote{His precepts which men ought chiefly to ſtudie.}
ſtudies among the Gentiles:

\V Becauſe he
\SNote{God reuengeth the blood of Martyrs.}
requiring bloud remembred them: he hath not forgotten the crie of the
poore.

\V Haue mercie on me Ô Lord: See my humiliation
\SNote{Procured by mine enemies.}
by my enemies.

\V Which exalteſt me from the gates of death, that I may declare al thy
prayſes in
\SNote{In the publique view of the Church.}
the gates of the daughter of Sion.

\V I wil reioyce in thy ſaluation: the Gentiles are
\SNote{The wicked are intangled in the ſnares which they lay for
others.}
faſtened in the deſtruction, which they made. In this ſnare, which they
hid, is their foote taken.

\V Our Lord shal be knowen doing iudgements: the ſinner is taken in the
workes of his owne handes.

\V
\SNote{In zele of iuſtice not in deſire of reuenge.}
Let ſinners be turned into hel, al nations that forget God.

\V Becauſe to the end there shal not be obliuion of the poore man: the
patience of the poore, shal not perish in the end.

\V Ariſe Lord, let not man be ſtrengthned: let the
\SNote{By Gentiles is often vnderſtood al great ſinners. For the Iewes
deſpiſed Gentiles: as the Romans did al Barbarous nations.}
Gentiles be iudged in thy ſight.

\V Appoint Lord
\SNote{Suffer a tyrant to rule ouer them that thereby they may lerne
what it is to vſe others vniuſtly. It ſemeth to S.~Auguſtine a
prophecie, that 
\CNote{\XRef{2.~Theſ.~2.}}
ſuch as receiue not Chriſt, shal beleue Antichriſt.}
a lawgeuer ouer them: that the Gentiles may know that they be men.

%%% !!! Better formatting needed here.
\bigskip

The
\LNote{The 10.~Pſalme.}{After
\MNote{Some diuide this Pſalme into two.}
the 21.~verſe the late Hebrew Doctors diuide this Pſalme, beginning
there the tenth, without anie new title: but only this word
\MNote{\HH{Sela} a note of change, or of reſt in muſike, or rather of
attention.}
\HH{Sela}: VVhich the Septuagint, Theodotion, and Symmachus
tranſlate \L{Diapſalma}, that is, change of meeter, or muſike, alſo
pauſe or reſt in ſinging.
\CNote{\Cite{Epiſt, ad Marcel.}}
Aquila whom S.~Ierom rather approueth, tranſlateth \L{ſemper} euer.
\CNote{\Cite{Anno. 1577.}
\Cite{1552.}
\Cite{1603.}}
Some Engliſh Bibles omitte it, others leaue it in the text, not
tranſlating it into Engliſh. It ſemeth to moſt Interpreters to be added
as a note to ſturre vp attention. And it occureth often, not only in the
end of Pſalmes, but alſo in other places. For it is thriſe in the third
Pſalme. And therfore maketh no argument, that this Pſalme should be
diuided. And thoſe which diuide this into two, ioyne two in the
147.~Pſalme.
\MNote{Al the Pſalmes are iuſt~150.}
So that al agree in the number of 150.~Pſalmes in the whole Pſalter.}
10.~Pſalme, according to
\SNote{The latter Hebrew Doctors.}
the Hebrevves.

\bigskip

\V
\SNote{In great perſecution it ſemeth to the weake, that God differreth
his aſſiſtance very long.}
Why Lord haſt thou departed far of, deſpiſeſt in opportunities, in
tribulation?

\V Whiles the impious is proude, the poore is
\SNote{Extremely vexed & tormented.}
ſet on fyre:
\SNote{The Prophet anſwereth to the complaint of the iuſt, that in deede
the wicked are caught in their owne ſnares.}
they are caught in the counſels which they deuiſe.

%%% o-1056
\V Becauſe the ſinner is prayſed in the deſires of his ſoule: and the
vniuſt man is bleſſed.

%%% 1164
\V The ſinner hath exaſperated our Lord, according to the multitude of
his wrath he shal
\SNote{Not ſeeke to recouer Gods fauour.}
not ſeeke.

\V There is no God in his ſight: his waies are defiled at al time. Thy
iudgementes are taken away from his face: he shal
\SNote{The wicked doth dominier for a time, and thinketh he shal do ſo
ſtil.}
rule ouer al his enemies.

\V For he hath ſayd in his hart: I wil not be moued from generation vnto
generation,
\SNote{And neuer fal into any aduerſitie but ſtil remaine without
miſerie or anie euil.}
without euil.

\V
\CNote{\XRef{Rom.~3.}}
Whoſe mouth is ful of curſing, and bitterneſſe, and guile: vnder his
tongue labour and ſorrow.

\V He ſitteth in waite with the rich in ſecrete places, to kil the
innocent.

\V His eyes looke vpon the poore: he lyeth in wayte in ſecret, as a lyon
in his denne.

\V He lyeth in wayte to take the poore man violently: violently to take
the poore man whiles he draweth him. In his ſnare he wil humble him
ſelfe, and shal fal when he shal haue dominion ouer the poore.

\V For he hath ſayed in his hart: God hath forgotten, he hath turned
away his face not to ſee for euer.

\V
\SNote{The prayer of the iuſt in tribulation.}
Ariſe Lord God, let thy hand be 
\Fix{axalted:}{exalted:}{likely typo, fixed in other}
forget not the poore.

\V Wherfore hath the impious prouoked God? for he hath ſaid in his hart:
He wil not enquire.

\V Thou ſeeſt, that thou conſidereſt labour and ſorrow: that thou mayeſt
deliuer them into thy handes.

To thee is the poore left: to the orphane thou wilt be an helper.

\V Breake the arme of the ſinner and malignant: his ſinne shal be
ſought, and shal not be found.

\V Our Lord shal reigne for euer, and for euer and euer: ye
\SNote{Ye vileſt men.}
Gentiles shal perish from his land.

\V Our Lord hath heard the deſire of the poore: thy eare hath heard the
\SNote{The iuſt ought alwayes to be readie prepared in hart, to ſuffer
patiently al that shal happen vnto them.}
preperation of their hart.

\V To iudge
\SNote{As the firſt workes of Chriſt in al humility and patience were
ſtrange, and hidden to the world
\XRef{(v.~1.)}
ſo his laſt iudgement shal be in maieſty and manifeſt to al in exalting
the bleſſed and ſuppreſſing the wicked.}
for the pupil and the humble, that man adde no more to magnifie him
ſelfe vpon the earth.


\stopChapter


\stopcomponent


%%% Local Variables:
%%% mode: TeX
%%% eval: (long-s-mode)
%%% eval: (set-input-method "TeX")
%%% fill-column: 72
%%% eval: (auto-fill-mode)
%%% coding: utf-8-unix
%%% End:
