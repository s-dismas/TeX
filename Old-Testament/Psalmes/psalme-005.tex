%%%%%%%%%%%%%%%%%%%%%%%%%%%%%%%%%%%%%%%%%%%%%%%%%%%%%%%%%%%%%%%%%
%%%%
%%%% The (original) Douay Rheims Bible 
%%%%
%%%% Old Testament
%%%% Psalmes
%%%% Psalme 005
%%%%
%%%%%%%%%%%%%%%%%%%%%%%%%%%%%%%%%%%%%%%%%%%%%%%%%%%%%%%%%%%%%%%%%




\startcomponent psalme-005


\project douay-rheims


%%% 1157
%%% o-1049
\startChapter[
  title={Psalme 5}
  ]

\PSummary{Iuſt
\MNote{The general iudgement. The 9.~key.}
men in affliction appeale to God, the reuenger of iniuries, 5.~knowing
and profeſſing that God hateth iniquity, 9.~therfore remitte their cauſe
to him, 11.~recite certaine enormous vices of the wicked, 13.~and expect
Gods final iudgement of the good and bad.}

\PTitle{Vnto the end, for
\SNote{The faithful iuſt ſoule that ouercõmeth her enimies by vertue.}
her that obtaineth the
\SNote{Eternal glorie.}
inheritance. The Pſalme of Dauid.}

\VV Receive Ô Lord
\SNote{The praier of the whole Church, or of anie faithful (euer
beloued) ſoule.}
my wordes with thine eares, vnderſtand my crie.

\V Attend to the voice of my prayer, my king and my God.

\V Becauſe I wil pray to thee: Lord in
\SNote{Gods helpe is preſently granted of his part, though it be
ſometimes differed for the more good of his ſeruantes.}
the morning thou wilt heare my voice.

\V In
\SNote{Before al other affayres we muſt pray to God.
\Cite{S.~Ciprian. in fine orat. Dominicæ.}}
the morning I wil ſtand by thee and wil ſee: becauſe thou art
\LNote{Not a God that vvilt iniquitie.}{Seing
\MNote{God is not author nor cauſe of ſinne.}
God \Emph{vvil not iniquitie}, as theſe wordes teſtifie in plaine
termes, it foloweth neceſſarily, that he is not author, nor cauſe of
anie ſinne. For God doth nothing contrarie to his owne wil. But he
hateth iniquitie, and in reſpect therof \Emph{hateth al that vvorke
iniquities}, as the authours of iniquity, though he loueth them as his
creatures and of his part requireth their ſaluation.}
not a God that wilt iniquitie.

\V Neither shal the malignant
\SNote{The wicked and wickednes haue noe conuerſation with God.}
dwel neere thee: neither shal the vniuſt abide
\SNote{In the day of iudgement.}
before thine eies.

\V Thou hateſt al that worke iniquitie: thou wilt
\SNote{By final ſentence of eternal dãnation.}
deſtroy al that ſpeake lie.

\V The bloudie and deceitful man our Lord wil abhorre:

\V But I in the multitude of
\SNote{Not in mans powre, but in Gods mercie muſt the iuſt man truſt.}
thy mercy. I wil enter into thy houſe: I wil adore toward
\SNote{In the Church of God.}
thy holie temple in thy
\SNote{With reuerential feare as in Gods preſence.}
feare.

Lord conduct me in thy iuſtice: becauſe of mine enimies direct my way in
thy ſight.

\V Becauſe there is
\SNote{No true nor ſolide goodnes in the wicked.}
no truth in their mouth: their hart is
\SNote{They thinke nothing but vanitie, and miſchiefe.}
vayne.

\V
\CNote{\XRef{Pſal.~13.}
\XRef{Rom.~3.}}
Their throte is an
\SNote{Yelding lothſome ſtinch, bitternes, and rancor,}
open ſepulchre, they did
\SNote{yet they flatter with feaned good wordes.}
deceitfully with their tongues,
\SNote{Albeit the iuſt deſire the conuerſion of the wicked, yet if they
wil not repẽt, then the iuſt conforme their deſires to Gods iuſt
iudgement: which shal be manifeſted in the end of the world.}
iudge them Ô God.

%%% 1158
%%% o-1050
\V Let them faile of their cogitations, according to the multitude of
their impieties expel them, becauſe they haue prouoked thee Ô Lord.

\V And let al be glad, that hope in thee, they shal reioyce for euer:
and thou shalt dwel in them. And al that loue thy name shal glorie in
thee, becauſe thou wilt
\SNote{The iuſt shal receiue ſentence of eternal glorie.}
bleſſe the iuſt.

\V Lord, as with a shield of thy good wil, thou haſt crowned vs.


\stopChapter


\stopcomponent


%%% Local Variables:
%%% mode: TeX
%%% eval: (long-s-mode)
%%% eval: (set-input-method "TeX")
%%% fill-column: 72
%%% eval: (auto-fill-mode)
%%% coding: utf-8-unix
%%% End:
