%%%%%%%%%%%%%%%%%%%%%%%%%%%%%%%%%%%%%%%%%%%%%%%%%%%%%%%%%%%%%%%%%
%%%%
%%%% The (original) Douay Rheims Bible 
%%%%
%%%% Old Testament
%%%% Psalmes
%%%% Psalme 031
%%%%
%%%%%%%%%%%%%%%%%%%%%%%%%%%%%%%%%%%%%%%%%%%%%%%%%%%%%%%%%%%%%%%%%




\startcomponent psalme-031


\project douay-rheims


%%% 1198
%%% o-1091
\startChapter[
  title={Psalme 31}
  ]

\PSummary{Forgeuenes
\MNote{The ſecond pœnitential Pſalme.

The 7.~key.}
of ſinnes is a happie thing: 3.~wherto manie are brought by affliction
geuing them vnderſtanding, ſo mouing them to confeſſe their ſinnes,
6.~pray for remiſſion, 10.~not deſpaire, but hope in Gods mercie, and ſo
reioyce with ſincere hart.}

\PTitle{To
\SNote{This Pſalme ſheweth how Dauid was brought to vnderſtand his
ſinnes, to confeſſe, bewayle, and obtaine remiſſion of them.}
Dauid him ſelfe
\LNote{Vnderſtanding.}{VVhen
\MNote{Vexation geueth vnderſtanding.}
Dauid had ſinned and ſomewhile neglected to confeſſe his fault, Gods
mercie by affliction made him to vnderſtand his owne eſtate, who then
repenting, confeſſing, and ſorowing for his ſinnes made this Pſalme,
which is therfore intitled \Emph{Vnderſtanding}: or, \Emph{Inſtruction of
Dauid}.
\MNote{Sinne muſt be puniſhed.}
It geueth vs alſo to vnderſtand, and to know, ſaith S.~Auguſtin, that we
muſt neither truſt in our owne merites, nor preſume to eſcape puniſhment
of ſinne. Thy firſt vnderſtanding therfore or leſſon, muſt be, to know
thy ſelf to be a ſinner.
\MNote{Good workes are of grace.}
The next is, that when with faith thou beginneſt to worke wel by loue,
thou attribute not this to thyn owne ſtreingth, but to the grace of
God.}
vnderſtanding.}

\NV \CNote{\XRef{Rom.~4.}
\XRef{1.~Pet.~4.}}
Bleſſed are they,
\SNote{The firſt bleſſing of a ſinner is the forgeuenes of his ſinnes,}
whoſe iniquities are forgeuen: and
%%% !!! This LNote is for two. 
\LNote{VVhoſe ſinnes are couered. 2.~Not imputed.}{Caluin
\CNote{in Epiſt. ad Ro. c.~4.}
\MNote{Proteſtantes expound this place contrary to many other clere
places.}
and his complices gether poyſon of theſe holie wordes, denying that
ſinnes are truly taken away, but only couered, and ſtil remayne ſay they
in the iuſteſt. VVhich ſenſe would make this Scripture contrarie to
other places.
\XRef{Iſaie.~6.}
Thyn iniquitie shal be taken away, and thy ſinne ſhal be cleanſed.
\XRef{Ioan.~1.}
The lambe of God which taketh away the ſinne of the world.
\XRef{Act.~3.}
Be penitent and conuert, that your ſinnes may be put out.
\XRef{1.~Cor.~6.}
You are waſhed, you are ſanctified, you are iuſtified, & the like, which
ſhew the true real taking away of ſinnes, true ſanctification, and
iuſtification.
\MNote{Contrarie to the expoſition of ancient fathers.}
As S.~Ierom (or ſome other ancient authentical
\Fix{autor)}{author)}{obvious typo, fixed in other}
explicateth this place ſaying: Sinnes are ſo \Emph{couered} by baptiſme
& penance, that they are not to be reueled in the day of iudgement, \Emph{nor
imputed} in him that diligently purgeth him ſelfe in this world, or by
martyrdom. S.~Auguſtin teacheth the ſame ſaying: Sinnes are couered, are
wholly couered, are aboliſhed. Neither muſt you vnderſtand (ſaith he)
that ſinnes are couered, as though ſtil they were, and liued. VVhy then
did the prophet ſay: Sinnes are couered? they are not to be
puniſhed. More clerly,
\Cite{li.~1. c.~13. cont. duas Epiſt. Pelag.}
The Pelagians calumniating Catholiques, as if they taught, that ſinnes
are not taken away, but ſhauen, as heares are cut with a raſor; the
rootes remaining in the fleſh, \Emph{vvhich} (he anſwereth) \Emph{none
affirmeth but an infidel}. Likewiſe S.~Gregorie teacheth, that a ſinner
couereth his ſinnes wel, when with contrarie vertues he ouerwhelmeth
former vices, and with good deedes blotteth out former euil deedes. He
couereth them euil, when either for ſhame, or feare, or obſtinacie, or
deſperation he concealeth his ſinnes, omitting to confeſſe them.
\MNote{God couering or not imputing ſinne doth quite take them away.}
God couereth ſinnes, as a phiſition couereth woundes, by applying
medicinal plaſter, which in deede cureth them. Thus ancient, lerned
Fathers expound this text. Further explicating, that albeit thinges
couered, and only therby hidde from men, do remaine as they were before they
were hid, yet whatſoeuer is hid to God, is in dede vtterly taken away,
for nothing that is, can be hid from God.
\MNote{The contrarie doctrin is iniurious to God:}
And the contrarie doctrin of Proteſtants is iniurious either to Gods
powre, if they ſay he can not quite take away ſinnes, or to his mercie,
if he wil not, or to his iuſtice, if he neuer puniſh ſinnes euer
remayning, and to his truth if he repute otherwiſe, then in deede the
thing is.
\MNote{to Chriſt:}
It is alſo iniurious to Chriſt, to ſay, his bloud and death is not
effectual to take away ſinnes;
\MNote{to holie Scriptures:}
iniurious to innumerable places of holie Scripture, which affirme
plainly that ſinnes by Gods grace are vtterly taken away.
\MNote{to glorified Sainctes.}
Finally it is iniurious to Sainctes in heauen, arguing them as ſtil
infected with ſinnes, if in dede ſinnes yet remaine in them, which is
moſt abſurde, and blaſphemie to ſpeake. And yet foloweth by neceſſarie
conſequence. For if the iuſteſt liued & died in ſinne, they ſhould
remaine eternally in ſinne.}
whoſe ſinnes
\SNote{by charitie, which couereth the multitude of ſinnes.
\XRef{1.~Pet.~4.}}
be couered.

\V Bleſſed is the man, to whom
\SNote{Satisfaction being made.}
our Lord hath
%%% !!! This LNote is part of the previous.
%%% \LNote{}{}
not imputed ſinne,
\LNote{Neither is there guile in his ſpirite.}{In
\MNote{Sincere repentance is a neceſſarie diſpoſition to remiſſion of
ſinne.}
remiſſion of ſinnes the penitent neceſſarily muſt ſo cooperate, that he
haue no guile in his ſpirite, or hart, for if he haue, then he faileth
of the forſaide bleſſednes, and his iniquities are not forgeuen, nor his
ſinnes couered to God, but to be imputed and puniſhed. Yet the
repentence of a ſinner be it neuer ſo ſincere, hartie, and without guile
doth not merite remiſſion of ſinne, but only diſpoſeth therto.
\MNote{After remiſſiõ it is ſatisfactorie and meritorious.}
But after remiſſion it is ſatisfactorie for the paine due for ſinnes,
and meritorious of glorie. According as S.~Auguſtin here teacheth ſaying
Good (or meritorious) workes goe not before faith, and remiſſion, but
folow the ſame.}
neither is there
\SNote{VVhen ſinners repent ſincerly without guile, then God forgeueth:
without which cooperation none is iuſtified.}
guile in his ſpirit.

%%% 1199
\V Becauſe
\SNote{Becauſe I acknowledged not my greuous ſinnes, I was ſtil ſore
afflicted,}
I held my peace, my bones are
\TNote{waxed as if they vvere old.}
inueterated,
\SNote{though otherwiſe I ceaſed not to pray but without any fruict or
good effect.}
whiles I cried al the day.

\V Becauſe day and night thy hand is made heauie vpon me: I
\SNote{Thy diuine prouidence reducing me,}
am turned in my anguish, whiles
\SNote{by remorſe of myn owne conſcience which telleth me that I
deſerue al this affliction.}
the thorne is faſtened.

\V I
\SNote{Therfore I do no longer diſsẽble with men nor am ſilent to thee,
but expreſly acknowlege my ſinnes.}
haue made my ſinne knowen to thee: and my iniuſtice I haue not hid.

I ſaid: I wil confeſſe againſt me my iniuſtice to our Lord: and thou
haſt forgeuen the impietie of my ſinne.

\V For this
\SNote{As I do now recal my ſelfe being ſtricken with Gods heauie hand;
ſo muſt euerie one that wil be purged from his ſinnes and ſanctified
\Emph{pray to thee}, when he is afflicted.}
shal euerie holie one pray to thee, in time conuenient. But yet
\SNote{Though calamities be meruelous great like \Emph{to a deluge}:}
in the floud of manie waters, they shal
\SNote{yet they ſhal not opreſſe him, that relieth vpon God.}
not approche to him.

\V Thou art my refuge from tribulation, which hath compaſſed me: my
exultation, deliuer me from them that compaſſe me.

\V I
\SNote{God ſpeaketh: promiſing by theſe tribulation to geue his ſeruants
\Emph{vnderſtanding}, and inſtruction:}
wil geue thee vnderſtanding, and wil inſtruct thee in the way, that thou
shalt goe; I
\SNote{with perpetual protection.}
wil faſten mine eies vpon thee.

\V Doe
\SNote{Be not therfore careles, like to brute beaſtes, but conſideratiue
of your actions.}
not become as horſe and mule, which haue no vnderſtanding.

%%% o-1092
In
\SNote{The Prophet or anie iuſt ſoul beſecheth God to hold this ſtraict
hand of diſcipline ouer ſinners, for their conuerſion.}
bit and bridle binde faſt their cheekes, that approch not to thee.

\V Manie
\SNote{Sinners deſerue much punishment,}
are the ſcourges of a ſinner, but
\SNote{but repenting, and truſting in God shal finde his mercie.}
him that hopeth in our Lord mercie shal compaſſe.

\V Be
\SNote{The end of true penance is ioy to which therfore the prophet
inuiteth al penitents.}
ioyful in our Lord and reioyce ye iuſt, and glorie al ye right of hart.


\stopChapter


\stopcomponent


%%% Local Variables:
%%% mode: TeX
%%% eval: (long-s-mode)
%%% eval: (set-input-method "TeX")
%%% fill-column: 72
%%% eval: (auto-fill-mode)
%%% coding: utf-8-unix
%%% End:
