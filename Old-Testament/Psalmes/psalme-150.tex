%%%%%%%%%%%%%%%%%%%%%%%%%%%%%%%%%%%%%%%%%%%%%%%%%%%%%%%%%%%%%%%%%
%%%%
%%%% The (original) Douay Rheims Bible 
%%%%
%%%% Old Testament
%%%% Psalmes
%%%% Psalme 150
%%%%
%%%%%%%%%%%%%%%%%%%%%%%%%%%%%%%%%%%%%%%%%%%%%%%%%%%%%%%%%%%%%%%%%




\startcomponent psalme-150


\project douay-rheims


%%% 1400
%%% o-1290
\startChapter[
  title={Psalme 150}
  ]

\PSummary{God
\MNote{God moſt excellent and moſt laudable.

The 1.~key.}
abſolutly moſt excellent is to be praiſed, 3.~with al ſortes of
inſtruments, and by al other meanes.}

\PTitle{Alleluia.}

\NV Prayſe ye our Lord
\SNote{Al ye Angels and men that are in the holie and higheſt heauen
praiſe our Lord.}
in his holies: prayſe ye him in
\SNote{Al ye creatures that are in, and vnder the firſt moueable
firmament praiſe our Lord.}
the firmament of his ſtrength.

%%% o-1291
\V Prayſe ye him in
\SNote{And you eſpecially (Gods peculiar people) amongſt whom, and for
whom, diuine miracles haue bene wrought praiſe our Lord,}
his powers: prayſe ye him
\SNote{with al your poſſible endeuoure, for though his infinite
Excellencie excedeth the powre of al creatures to praiſe him
ſufficiently, yet it reſteth that you may infinitly extend your wil and
deſire to praiſe our Lord, according to the multitude of his greatnes.}
according to the multitude of his greatnes.

%%% 1401
\V
\SNote{Out of this your great and infinite deſire, let your tongues
ſound and ſing diuine praiſes, as wel vvith voice, as muſical
inſtruments.}
Prayſe ye him
\SNote{VVherof ſix moſt vſual in the Tabernacle and Temple vvere theſe:
Trumpet, Pſalter, Harpe, Timbrel, Organ, and Cymbal.}
in the ſound of trumpet: prayſe ye him on pſalter, and harpe.

\V Prayſe ye him on timbrel and
\SNote{By the vvay the Pſalmiſt interpoſeth agane tvvo eſpecial thinges,
vvhich make perfect harmonie, vvithout vvhich no inſtrument is gratful
to God: Vnitie amongſt his ſeruants, ſignified by the Quire of conſonant
voices:}
quire: prayſe ye him on
\SNote{and mortification of paſſions, ſignified by Stringes, vvhich are
made of dead beaſtes bovvels.}
ſtringes, and organ.

\V Prayſe ye him on wel ſounding cymbals: prayſe ye him on cymbales of
iubilation: \V let euerie
\SNote{Man created of corruptible bodie and immortal ſoule, is finally
admoniſhed to praiſe our Lord, ouer and aboue the praiſes of al other
corporal creatures; vvho alſo is more eſpecially bond therto then
Angels, becauſe God hath voutſaffed to make himſelf Man, to redeme man
that vvas loſt by ſinne, and to endew him vvith nevv grace, and ſo bring
him to euerlaſting glorie, vvhere vvith holie Angels, men alſo for euer
& euer ſhal praiſe our Lord, vvith hart, voice, and iubilation of
ſpirite, ſinging as the Pſalmiſt concludeth, Alleluia.}
ſpirit prayſe our Lord.
%%% !!! Really a free-floating Annotation
\LNote{Pſalme CL.}{S.~Auguſtin
\MNote{The number of Pſalmes ſignifieth the agrement of the old and nevv
Teſtament.}
in the
\Cite{concluſion of his Enarrations,}
or Sermons vpon the Pſalmes, explicateth a myſterie in the number of an
hundred and fieftie, ſignifying the concord of the two Teſtaments. For
in the old teſtament they kept the Sabbath, which is the ſeuenth day: in
the new we kepe our Lords day, after the ſabbath, that is, the eight:
which ſeuen and eight (making fieftene) multiplied by tenne, ſignifying
the Law of tenne commandments, riſe vnto 150.

Againe ſeuen multiplied by ſeuen make 49. wherto one (to witte the eight)
being added make fieftie, which multiplied by three, ſignifying the
B.~Trinitie, make 150.
\MNote{Three fifeties ſignifie}
Neither ſemed it without cauſe to this great Doctor, that
\MNote{Penance,}
the firſt fieftie end with a Pſalme of Pænance, crauing mercie &
remiſſion of ſinnes:
\MNote{Mercie vvith iuſtice,}
the ſecond with Mercie and Iuſtice, which God ioyneth in the Redemption,
Iuſtification, and Saluation of men:
\MNote{and Praiſes of God.}
the laſt with Diuine Praiſes, ſignifying, that by condemning ſinnes in
our ſelues, through Gods mercie we may be iuſtified, and ſo beginne in
this life, which is to be perfected in the next, to praiſe our Lord, as
S.~Paul admoniſheth with Pſalmes, Hymnes, and Spiritual
Songues. Concluding with the tvvo verſes,
\CNote{\Cite{Colloſs. Circa An.~Do.~380.}}
appointed by S.~Damaſus Pope, to be added in the end of al Pſalmes,
\MNote{Gloria Patri added by tradition.}
and is obſerued euer ſince his time by tradition in the vvhole Church:

Glorie to the Father, and to the Sonne, and to the Holie Ghoſt: As it
was in the beginning, and now, and euer, into worldes of worldes (in
eternitie vvithout end) Amen.}
Alleluia.


\stopChapter


\stopcomponent


%%% Local Variables:
%%% mode: TeX
%%% eval: (long-s-mode)
%%% eval: (set-input-method "TeX")
%%% fill-column: 72
%%% eval: (auto-fill-mode)
%%% coding: utf-8-unix
%%% End:
