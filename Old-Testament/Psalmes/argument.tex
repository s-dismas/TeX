%%%%%%%%%%%%%%%%%%%%%%%%%%%%%%%%%%%%%%%%%%%%%%%%%%%%%%%%%%%%%%%%%
%%%%
%%%% The (original) Douay Rheims Bible 
%%%%
%%%% Old Testament
%%%% Psalmes
%%%% Argument
%%%%
%%%%%%%%%%%%%%%%%%%%%%%%%%%%%%%%%%%%%%%%%%%%%%%%%%%%%%%%%%%%%%%%%




\startcomponent argument


\project douay-rheims


%%% 1138
%%% o-1031
\startArgument[
  title={\Sc{Proemial Annotations vpon the Booke of Psalmes.}},
  marking={Annotations vpon the Booke of Pſalmes.}
  ]

The
\MNote{This booke vndoubtedly is canonical Scripture.}
authoritie of this Booke was euer authentical, and certaine, as the
aſſured word of God, and Canonical Scripture.
But concerning the author, there be diuers opinions. For although it be
manifeſt by the teſtimonie of Philo, and Ioſephus, that in their time,
and alwaies before, only King Dauid was by al Hebrew Doctors holden for
Author of al the Pſalmes: 
\MNote{Late Hebrew Doctors and ſome Catholiques hold diuers authors of
ſundry Pſalmes.}
yet after that lerned Origen, and other Chriſtian doctors, expounded
manie Pſalmes of Chriſt, the Iewes being preſſed therwith, begane to
denie that al were Dauids: alleaging for their new opinion, the titles
of diuers Pſalmes, and ſome other difficulties, miniſtring occaſion of
much needles diſpute, ſtil acknowledging the whole booke to be
Canonical. Wherupon
\CNote{\Cite{Præfat. in Pſal.}}
S.~Ierom, and
\CNote{\Cite{Epiſt.~134.}
\Cite{139.}}
S.~Auguſtin, ſometimes admitted thoſe as authores of certaine Pſalmes,
whoſe names are in the titles thereof. S.~Cyprian, S.~Cyril,
S.~Athanaſius, and others agree in general, that Dauid writte not al:
but differ much in particular, touching other ſuppoſed authores. In ſo
much that \Emph{Melchiſedech, Moyſes, Aſaph, Eman, Idithun, The ſonnes
of Core, Salomon, Ieremie, Ezechiel, Eſdras, Aggæus, and Zacharias, are
al} (with more or leſſe probabilitie) \Emph{reputed authores of ſeueral
Pſalmes.}
\MNote{It is much more probable that Dauid was author of al.}
Neuertheles it ſemeth that S.~Ierom rather related other mens iudgement,
then shewed his owne; as we shal note by and by.
\MNote{Proued by S.~Auguſtin, S.~Chryſoſtom, and greateſt part of
Doctors.}
And S.~Auguſtin
\Cite{(li.~17. c.~14. de ciuit.)}
maturely diſcuſſing this doubt, ſaith plainly, that \Emph{their
Iudgement ſemed to him more credible, vvho attribute al the hundred
fiftie Pſalmes to Dauid alone.}  Further explicating that wheras ſome
Pſalmes
%%% o-1032
haue Dauids owne name in their titles, ſome haue other mens names, ſome
none at al, \Emph{this diuerſitie importeth
%%% 1139
not diuers authores, but ſignifieth other thinges, either perteyning to
the ſame perſons, or by interpretation of their names, belonging to the
preſent matter, as our Lord inſpired him.} Likewiſe S.~Chryſoſtome
reſolutely iudgeth, that only King Dauid was author of this whole
Booke. Moued eſpecially by this argument, for that
\CNote{\XRef{Mat.~22. v.~16.}
\XRef{Act.~4. v.~24.}
\XRef{Ro.~4. v.~6.}
\XRef{Ro.~11. v.~9.}
\XRef{Heb.~4. v.~7.}}
Chriſt and the Apoſtles alleaging the Pſalmes, do oftentimes name Dauid
as author, and neuer anie other.
\MNote{The common voice of Chriſtians & ſome general councels, cal it
Dauids Pſalter.}
Alſo Origen, S.~Baſil, S.~Ambroſe, S.~Hilarie, Theodoretus, Caſſiodorus,
Beda, Eutimius, and moſt part of ancient and late writers, with the moſt
common voice of Chriſtians, cal this booke the Pſalmes of Dauid: and the
General Councels of Carthage, Florence, and Trent, in the Cathologue of
Canonical Scriptures recite this booke, by the name of Dauids Pſalter.

Moreouer it is clere,
\XRef{Act.~2.},
that the ſecond Pſalme, though it want his name, is Dauids.
\MNote{Proued by other Scriptures.}
And other Scriptures
\XRef{2.~Paral.~7. v.~6.}
and
\XRef{1.~Eſd.~3. v.~10.}
ſay plainly, that Dauid made the Pſalmes,
\XRef{104.}
\XRef{105.}
\XRef{106.}
\XRef{117.}
\XRef{135.}
beginning: \Emph{Confeſſe to our Lord, becauſe he is good, becauſe his
mercie is for euer.} VVhich he appointed the Leuites to ſing, or play on
inſtruments:
\XRef{1.~Paral.~15.}
&
\XRef{16.}
and yet they haue not his name in their titles. Againe,
\XRef{2.~Reg.~23.}
\Emph{Dauid} is only intitled \Emph{the egregious}, or
excellent \Emph{Pſalmiſt of Iſrael}. Neither were Aſaph, Eman, and
Idithun anie where called Prophetes (as are al the writers of holie
Scriptures) but only maſters of muſike,
\XRef{1.~Paral.~25.}
And the ſonnes of Core were only porters,
\XRef{1.~Paral.~26.}
Finally S.~Ierom (whoſe iudgement the whole Church ſingularly eſtemeth,
in al queſtions belonging to holie Scriptures) ſemeth as much inclined,
that only the Royal Prophet Dauid was author of this whole booke, as to
the contrarie.
\MNote{S.~Ierom attributeth the ſumme of this booke to Dauid only.}
For in his
\Cite{Epiſtle to Paulinus},
prefixed before the Latin Bible, comprehending the principal arguments
of ſeueral bookes, when he cometh to the Pſalmes, without mention of
other authors, ſaith: \Emph{Dauid our Simonides, Pindarus, and Alceus;
Flaccus alſo, Catullus, and
%%% 1140
Cerenus, ſoundeth out Chriſt, vvith harpe & tenne ſtringed Pſalter,
riſing vp from hel}: ſo attributing the ſumme of this whole booke to the
Royal Prophet Dauid, as if he ſuppoſed no other author.

Touching
\MNote{The Pſalmes are a Summe of al other Scriptures.}
therfore the argument, or contentes of this
%%% o-1033
diuine Pſalter, al Catholique Doctors vniformly agree that it is the
abridgement, ſumme, and ſubſtance of al holie Scriptures, both old and
new Teſtament. As may firſt be probably collected, by that 
\CNote{\XRef{Mat.~5.}
\XRef{7.}
\XRef{11.}
\XRef{22.}
\XRef{Luc.~16.}}
Chriſt himſelfe often comprehending al the old Teſtament by the termes
of the Law and Prophetes, in one place
\XRef{(Luc.~24. v.~44.)}
ſemeth not onlie to reduce al to \Emph{the Lavv and Prophetes iointly},
but alſo to \Emph{the Pſalmes alone}, or ſeuerally. But whether this be
our Sauiours diuine meaning or no in that place, out of this and manie
other places, al the ancient Fathers teach expreſly, that the Pſalmes
are an Epitome of al other holie Scriptures. For example, S.~Denys,
\Cite{li. de Eccleſ. Hierar. contemplat.~2.}
after brief recital of the contents of other holie Scriptures,
ſaith: \Emph{This ſacred booke of diuine Canticles, doth exhibite both a
general ſong, and expoſition of diuine thinges.} S.~Baſil calleth the
\Emph{Pſalmodie of Dauid the common and moſt plentiful ſtorehouſe of al
ſacred doctrine: the treaſure of perfect Theologie.} S.~Ambroſe
accounteth it \Emph{the regiſter of the vvhole Scripture.} Origen,
S.~Cyprian, S.~Ierom, S.~Chryſoſtom,
\CNote{S.~Greg. in Pſal. pœnitent.}
S.~Gregorie, S.~Beda, S.~Bernard, Caſſiodorus, Eutimius, and others vſe
the ſame, or very like termes. S.~Auguſtin particularly diſtinguiſhing
al the Scriptures into foure ſortes of bookes, ſheweth that
\MNote{They cõteine the ſumme of Legal, Hiſtorical,
\Fix{Sapientiential,}{Sapiential,}{obvious typo, fixed in other}
and Prophetical doctrine.}
the Pſalmes conteyne al: \Emph{The Lavv} (ſaith he) \Emph{teacheth
ſomethinges, the Hiſtorie ſomethinges, the Prouerbes alſo and the
Prophetes teach ſomethinges: but the Booke of Pſalmes teacheth al. It
propoſeth the Lavv, recounteth thinges of old, preſcribeth the due
ordering of mens actions, and prophecieth thinges to come. Briefly it is
a common treaſure of good doctrine, aptly adminiſtring that is
neceſſarie to euerie one.} And a litle after, exemplifying in particular
points: \Emph{Is not here} (ſaith
%%% 1141
he) \Emph{al greatnes of vertue, and is not here the right ſquare of
iuſtice? is not the comlines of chaſtitie; the conſummation of prudence?
is not vvhatſoeuer may be called good, lerned in the Pſalmes? Here is
the knovvlege of God; the clere prenounciation of Chriſt to come in
flesh; the hope of general Reſurrection; feare of torments; promiſe of
glorie; reuelation of myſteries. Euen al good thinges are here, as in a
common great treaſure, laide vp and heaped together.}

See
\MNote{Gods prouidence in ſweetly drawing our conſent & cooperation of
free-wil, which is neceſſarie to ſaluation.}
then and obſerue here (Chriſtian reader) the admirable wiſdom, and
goodnes of God. The meanes of mans ſaluation being ſo diſpoſed, that his
owne free conſent, and cooperation is therto neceſſarily required,
according to that moſt approued doctrin of the ſame S.~Auguſtin:
\CNote{\Cite{Ser.~15. de verb. Apoſt.}}
\L{Qui creauit te
%%% o-1034
ſine te; non iuſtificat te ſine te.} \Emph{He that created thee,
vvithout thee, doth not iuſtifie thee vvithout thee}: to helpe our
weaknes, and ſweetly to draw our mindes, otherwiſe auerſe from trauel
and paine,
\CNote{S.~Baſil in prolog.}
the Holie Ghoſt hath ordained that in ſmal rowme, and in pleaſant maner,
we may attaine neceſſarie knowlege of God, & our ſelues, eaſely kepe the
ſame in memorie, and dayly put in practiſe our chiefeſt dutie, in
ſeruing and praiſing God, by ſinging, reading, or hearing theſe diuine
Pſalmes, which one booke (as euerie one shal be able to lerne it, more
or leſſe perfectly) openeth and sheweth the way, to vnderſtand al other
Scriptures, and ſo to finde, & enioy the hidden treaſures of Gods word:
in like maner as a key openeth a lock.
\MNote{Holie Scriptures a ſealed booke.}
For the whole ſacred \Emph{Bible is a ſealed Booke}, and not rightly
vnderſtood,
\CNote{\XRef{Apoc.~5.}}
til the ſeale, or lock be opened, by the key of Gods ſpirite, geuing
knowlege; which the Holie Ghoſt, amongſt other wayes, inſpireth very
often, by ſacred Muſike or Pſalmodie.  As
\CNote{\Cite{li.~4. dialogi. c.~42.}}
S.~Gregorie noteth in holie Scripture
\XRef{(4.~Reg.~3. v.~15.)}
where Eliſeus not yet knowing Gods wil in a particular caſe, called for
a Pſalmiſt (or player on inſtruments) and
\MNote{The Pſalter is the key of other Scriptures.}
\Emph{vvhen the Pſalmiſt ſang, the hand of our Lord came vpon Eliſeus},
and preſently he preſcribing what should be donne, procured plentie of
water without rayne, where
%%% 1142
was none before, and prophecied victorie againſt the enemies. Reaſon
alſo and experience teach, that as 
\CNote{\XRef{Iac.~5.}}
\Emph{men of cheerful hart are apt to ſing}: ſo the exerciſe of reading,
ſinging, or playing Pſalmes, is a conuenient and a ſpecial meanes, to
attaine quietnes or cheerfulnes of mind.

But
\MNote{But itſelf is alſo ſealed.}
as this holie Pſalter is the key of other Scriptures, ſo it ſelfe is
moſt eſpecially a ſealed, and locked Booke, requiring manie keyes.
\Emph{Euerie Pſalme} (ſaith S.~Hilarie) \Emph{hath a peculiar key, and
oftentimes there be ſo manie lockes and keyes of one Pſalme, as there be
diuers perſons that ſpeake, to diuers endes and purpoſes.}
\MNote{But one principal key of ech Pſalme.}
For albeit diuers myſteries are ſometimes connected, and ſo require
ſundrie keyes, yet there is but one principal, & proper key of ech
Pſalme: otherwiſe it should be diuided into manie Pſalmes. Our firſt
endeuour therfore muſt be, to find the proper key of euery Pſalme, that
is, to know what is principally therein conteyned.
\MNote{Tenne keyes of the Pſalter.}
To this purpoſe the lerned Expoſiters of this  booke, haue obſerued
tenne general pointes, or ſeueral matters, to which al the contents may
be reduced, as it were, ſo manie keyes, and
%%% o-1035
meanes of entrance into the ſenſe, and true vnderſtanding of al the
Pſalmes.
\MNote{Alſo tenne ſtringes.}
And the ſame may likewiſe be called the tenne ſtringes of this diuine
inſtrument. Vpon one of which, euery Pſalme principally playeth,
touching the reſt more or leſſe, as cauſe requireth, for more melodious
harmonie, and perfect muſike.

The
\MNote{1.~Key. One God the B.~Trinitie.}
firſt key, or ſtring, is God himſelf: One in Subſtance: Three in
Perſons. Almightie, Alperfect, Powre, VViſdom, Goodnes, Maieſtie,
Iuſtice, Mercie, & other Diuine Attributes.
\MNote{2.~Gods workes.}
The ſecond, is Gods workes of Creation, Conſeruation, and Gouerning of
the whole world.
\MNote{3.~Gods prouidence.}
The third, Gods Prouidence, eſpecially towards man, in protecting and
rewarding the iuſt: in permitting, and punishing the wicked.
\MNote{4.~The Hebrew people.}
The fourth, is the peculiar calling of the Hebrew people, their
beginning in Abraham, Iſaac, and Iacob: their maruelous increaſe in
Ægypt: diuers eſtates,
%%% 1143
manie admirable and miraculous thinges donne amongſt them; with their
ingratitude, reiection, and reprobation.
\MNote{5.~Chriſt our Redemer.}
The fifth principal key, and ſtring is Chriſt, the promiſed Redemer of
mankind: prophecying his Incarnation, Natiuitie, Trauels, Sufferings,
Death, Reſurrection, Aſcenſion, and Glorie.
\MNote{6.~Conuerſion of Gentils, the Catholique Church.}
The ſixt is the propagation of Chriſts name and Religion, with Sacrifice
and 
\Fix{Sacramntes,}{Sacramentes,}{likely typo, same in other}
in the multitude of Gentiles beleeuing in him, euen to the vttermoſt
coaſtes of the earth, the Catholique Church euer viſible.
\MNote{7.~Faith & good workes.}
The ſeuenth is the true maner of ſeruing God, with ſincere faith, and
good workes.
\MNote{8.~Dauids owne actes.}
The eight, holie Dauid interpoſeth manie thinges concerning himſelfe. As
Gods ſingular benefites towards him, for which he rendereth thankes, and
diuine praiſes, recounteth his enimies, dangers, and afflictions of mind
& bodie, namely by Saul, Abſalon and others, humbly beſeeking, and
obtaining Gods protection. He alſo expreſſeth in himſelfe a perfect
image, and patterne of a ſincere and hartie penitent: bewayling,
confeſſing, and puniſhing his owne ſinnes.
\MNote{9.~General Reſurrection, & Iudgement.}
The ninth is the end and renouation of this world, with the general
Reſurrection, and Iudgement.
\MNote{10.~Eternal glory and paine.}
The tenth is eternal felicitie, and punishment, according as euerie one
deſerueth in this life. Theſe are the tenne keyes of this holie Booke;
and tenne ſtringes of this Diuine Pſalter.

Moreouer
\MNote{Foure wayes to find the proper key of euerie pſalme.}
to finde which of theſe is the proper key, and principal ſtring of
euerie Pſalme, lerned Diuines vſe foure
%%% o-1036
eſpecial wayes.
\MNote{1.~By the title.}
Firſt by the title, added by Eſdras, or the Seuentie two Interpreters,
for an introduction to the ſenſe of the ſame Pſalme. So it appeareth
that the third Pſalme treateth literally of Dauids danger, and deliuerie
from his ſonne Abſalon: which is the eight key: though myſtically it
ſignifieth Chriſts Perſecution, Paſſion, & Reſurrection, which is the
fifth key.
\MNote{2.~Allegation in the new Teſtament.}
Secondly, if there be no title, or if it declare not ſufficiently the
key, or principal matter conteyned, it may ſome times be found by
allegation and
%%% 1144
application of ſome ſpecial part thereof in the new Teſtament. So it is
euident
\XRef{Act.~4. v.~25.}
\XRef{c.~13. v.~33.}
\XRef{Heb.~1. v.~5.}
&
\XRef{Heb.~5. v.~5.}
that the ſecond Pſalme perteyneth to Chriſt, impugned and perſecuted by
diuers aduerſaries. VVhich is the fift key.
\MNote{3.~Greatnes of thinges affirmed.}
Thirdly, when greater thinges are affirmed of anie perſon, or people, as
of Dauid, Salomon, Iewiſh Nation, or the like, then can be verified of
them, it muſt neceſſarily be vnderſtood of Chriſt, or his Church, in the
new Teſtament, or in Heauen. So the \Emph{concluſion} of the 14.~Pſalme:
\Emph{He that doth theſe thinges, shal not be moued for euer}, can not
be verified of the tabernacle, nor temple of the Iewes; but of eternal
Beatitude in heauen. VVhich is the tenth key. Though the greater part of
the Pſalme sheweth, that iuſt and true dealing towards our neighboures,
is neceſſarie for attayning of eternal Glorie.
\MNote{4.~Conference of places.}
Fourtly when, both the title and Pſalme, or part thereof ſeme hard and
obſcure, ſome part being more cleare, the true ſenſe of al may be
gathered, by that which is more euident. According to S.~Auguſtins rule,
\Cite{li.~2. c.~9.}
&
\Cite{li.~3. c.~26. Doct. Chriſt.}
\CNote{\Cite{li.~3. c.~4. de pecca. merit.}}
So the title, and former part of the fifth Pſalme, being more obſcure,
are explaned by the laſt verſes, ſhewing plainly that God wil iuſtly
iudge al men, both iuſt and wicked, in the end of this world. VVhich is
the ninth key. By theſe and like meanes the principal key being found,
it wil more eaſily appeare, what other keyes belong to the ſame, and
what other ſtringes are alſo touched. At leaſt the ſtudious may by theſe
helpes make ſome entrance, and for more exact knowlege ſearch the
iudgement of ancient Fathers, and other learned Doctours.

But beſides this ſingular great commoditie, of compendious handling much
Diuine matter in ſmal rowme,
\MNote{The ſtile of this booke is Poetrie.}
this booke hath an other ſpecial excellencie, in the kind of ſtile, and
maner
%%% o-1037
of vttering, which is Meeter, and Verſe, in the original Hebrew
tongue. And though in Greke, Latin, and other languages, the ſame could
not in like forme be
%%% 1145
exactly tranſlated yet the number, and diſtinction of verſes is ſo
obſerued, that it is apt for muſike, as wel voices as inſtruments, and
to al other vſes of Gods ſeruants.
\MNote{Abuſe derogateth not from good thinges.}
Neither is muſical maner of vttering Gods word and praiſes, leſſe to be
eſteemed, becauſe profane Poetes haue in this kind of ſtile vttered
light, vaine, and falſe thinges. For the abuſe of good thinges, doth not
derogate from the goodnes therof, but rather commendeth the ſame, which
others deſire to imitate.
\MNote{Dauids Pſalter more ancient then any profane poetrie now extant.}
And clere it is, that this holie Pſalmodie was before anie profane
poetrie now extant. For Homer the moſt ancient of that ſorte, writte his
poeme, at leaſt two hundred and fourtie yeares after the deſtruction of
Troy: as Apolidorus witneſſeth; others, namely Solinus, Herodotus, and
Cornelius Nepos ſay longer. VVheras
\Fix{kind}{Kind}{obvious typo, fixed in other}
Dauid our Diuine Pſalmiſt, reigned within one hundred years, after the
Troianes warres. There were in dede Amphion, Orpheus, and Muſcus before
Dauid, but their verſes either were not written, or ſhortly periſhed,
only a confuſe memorie remaining of them, recited, altered, and
corrupted by word of mouth: but before them were the ſacred Hiſtorie of
Iob, almoſt al in verſe; and the two Canticles of Moyſes,
\XRef{Exodi~15.}
and
\XRef{Deut.~32.}
\MNote{Muſike very ancient.}
It is moreouer recorded that
\CNote{\XRef{Gen.~4.}}
Iubal (long before Noes floud) was the father of them, that ſang on
harpe, and organ. Muſike therfore is maruelous ancient.
\MNote{Sacred poetrie moſt excellẽt.}
But ſacred Poetrie is in manie other reſpectes moſt excellent, and moſt
profitable. \Emph{This holie Pſalmodie} (ſaith
\CNote{\Cite{Præfat.}}
S.~Auguſtin) \Emph{is a medecine to old ſpiritual ſores, it bringeth
preſent remedie to nevv vvoundes: it maketh the good to perſeuere in
vvel doing, it cureth at once al predominating paſsions, vvhich vexe
mens ſoules.} A little after: \Emph{Pſalmodie driueth avvay euil
ſpirites, inuiteth good Angels to helpe vs, it is a shield in night
terrors, a refreshing of day trauels, a guard to children, an ornament
to yongmen, a comforte to oldmen, a moſt ſeemlie grace to vvemen. Vnto
beginners it is an introduction, an augmentation to them that goe
forvvard in vertue, a ſtable firmament to the perfect: it
%%% 1146
conioyneth the vvhole Church militant in one voice, and is the ſpiritual
eternal ſvvete perfume of the celeſtial Armies, al Sainctes and Angels
in heauen.}

%%% o-1038
To
\MNote{VVhy King Dauid writte diuine poetrie.}
al this we may adde other cauſes, which moued the Royal Prophete to
write this diuine poetrie.
\MNote{The firſt cauſe his natural inclination to muſike.}
Firſt he had from his youth (by Gods ſpecial prouidence) a natural
inclination to Muſike; wherin he ſhortly ſo excelled, that before al the
Muſitians in Iſrael, he was ſelected to recreate king Saul, whom an euil
ſpirite vexed. And his skil, together with his deuotion, had ſuch
effect, that 
\CNote{1.~Reg.~16. v.~23.}
\Emph{vvhen he playde on the harpe, Saul vvas refreshed, and vvaxed
better. For the euil ſpirite departed from him}: ſaith the holie
text. VVherfore he made theſe Pſalmes, that him ſelfe and others might
by ſinging them, imploy this gift of God to his more honour.
\MNote{2.~Verſe more eaſie & more pleſant.}
Secondly, verſe being more eaſie to lerne, more firmly kept in mind, and
more pleaſant in practiſe (for
\CNote{\XRef{Eccli.~40.}}
\Emph{as wine, ſo muſike doth recreate the hart of man}) the Holie Ghoſt
condeſcending to mans natural diſpoſition, inſpired Dauid to write theſe
Pſalmes in meeter,
\CNote{\Cite{S.~Aug.}
\Cite{S.~Baſil. in præfat.}}
\Emph{mixing the povvre of diuine doctrin, vvith delectable melodie of
ſong, that vvhiles the eare is allured vvith ſvvete harmonie of muſike,
the hart is indued vvith heauenlie knovvlege, pleaſant to the mind, and
profitable to the ſoule.}
\MNote{3.~Moſt ſpecial great, and memorable thinges writte in verſe.}
Thirdly, Dauid ſingularly illuminated with knowlege of great, and moſt
diuine Myſteries, indued alſo with moſt gracious diſpoſition of mind,
\Emph{the man choſen according to Gods ovvne hart}
\XRef{(1.~Reg.~13.)}
would vtter the ſame Myſteries, with godlie inſtructions, and praiſes
of God, in the moſt exquiſite kind of ſtile; that is in verſe. For
otherwiſe he was alſo very eloquent in proſe, as wel appeareth by
ſundrie his excellent, and effectual diſcourſes, in the books of Kinges,
and Paralipomenon. For which cauſe
\CNote{\XRef{Exo.~23.}}
Moyſes alſo deſcribed the paſſage of Iſrael forth of Ægypt through the
read ſea in a Canticle, after that he had related the ſame whole
hiſtorie, more at large in proſe; that al might ſing, and ſo render
thankes with melodious voice, and muſical
%%% 1147
inſtruments praiſing God. Likewiſe in an other Canticle
\CNote{\XRef{Deut.~32.}}
he compriſed the whole law, a litle before his death. So alſo
\CNote{\XRef{Iudic.~5.}}
Barac and Debora: and after them
\CNote{\XRef{Iudith.~16.}}
Iudith, ſong praiſes to God for their victories in verſe.
\CNote{\XRef{Prou.~31.}}
Salomon writte the end of his Prouerbes, and a whole booke (intituled
Canticles) &
\CNote{\XRef{1.~Reg.~2.}}
the Prophet Ieremie his Lamentations in verſe. Anna hauing obtained her
prayer for a ſonne, gaue thankes to God with a Canticle.
\CNote{\XRef{Iſa.~38.}}
The like did king Ezechias for recouerie
%%% o-1039
of health. The Prophets
\CNote{\XRef{Iſa.~12,~26.}}
Iſaias,
\CNote{\XRef{Ezech.~38.}}
Ezechiel,
\CNote{\XRef{Ion.~2.}}
Ionas,
\CNote{\XRef{Abac.~3.}}
Abacuc,
\CNote{\XRef{Dan.~3.}}
and the three children in the furnace:
\MNote{Cãticles in the new Teſtament.}
againe in the new Teſtament, the
\CNote{\XRef{Luc.~1.}
\XRef{2.}}
B.~virgin mother, iuſt Zacharie, & deuout Simeon gaue thankes, & ſang
praiſes to God in Canticles.

Fourthly,
\MNote{4.~Both diuine muſike and dittie in Gods temple.}
albeit, the holie King was not permitted, to build the gorgious Temple
for Gods ſeruice, as he greatly deſired to haue done, yet he prouided
both ſtore of muſitians 
\CNote{\XRef{1.~Par.~23,~25.}}
(foure thouſand in number, of which 288.~were maiſters to teach) & made
theſe Pſalmes as godlie ditties, for this holie purpoſe, in al
ſolemnities of feaſtes, and daylie ſacrifice, when the Temple ſhould
afterward be built.

Fiftly,
\MNote{5.~The great vſe of theſe Pſalmes in the Catholique Church.}
he made theſe Pſalmes not only for his owne, & others priuate deuotion,
nor yet ſo eſpecially for the publique Diuine ſeruice in the Temple, and
other Synagogues of the Iewes, but moſt principally for the Chriſtian
Catholique Church, which he knew ſhould be ſpred in the whole
earth. Forſeing the maruelous great, and frequent vſe therof in the
Chriſtian Clergie, and Religious people of both ſexes. As he prophecieth
in diuers Pſalmes. 
\CNote{\XRef{56.}}
\Emph{Al the earth ſing to thee}:
\CNote{\XRef{117.}}
\Emph{ſing Pſalmes to thy name}. Againe,
\CNote{\XRef{65.}}
\Emph{I vvil ſing Pſalmes to thee} (Ô God) \Emph{in the Gentiles, in al
peoples, and Nations.} VVhich him ſelfe neuer did, but his Pſalmes are
euer ſince Chriſt, ſong by Chriſtians, conuerted from gentilitie, as we
ſee in the Churches Seruice.
\MNote{The whole Pſalter in the ordinarie office euerie weke.}
For the whole Pſalter is diſtributed to be ſong, in the ordinarie office
of our Breuiarie euerie weke. And though extraordinarily, for the
varietie of times, and feaſtes, there is often alteration,
%%% 1148
yet ſtil the greater part is in Pſalmes.
\CNote{\XRef{4.}
\XRef{(30. ad v.~7.}
\XRef{53.}
\XRef{62.}
\XRef{66.}
\XRef{90.}
\XRef{94.}
\XRef{118.}
\XRef{133.}
\XRef{148.}
\XRef{149.}
\XRef{150.}}
\MNote{Certaine Pſalmes euerie day.}
Certayne alſo of the ſame Pſalmes, are without change, or intermiſſion
repeted euerie day. And ſuch as haue obligation to the Canonical Houres,
muſt at leaſt read the whole Office priuatly, if they be not preſent
where it is ſong.
\MNote{Many Pſalmes in other Eccleſiaſtical offices.}
The Office alſo of Maſſe, ordinarily beginneth with a Pſalme. In
Litanies, and almoſt al publique Prayers, and in adminiſtration of other
Sacraments, and Sacramentals, either whole Pſalmes, or frequent verſes
are inſerted. Likewiſe the greateſt part of the Offices, of our
B.~Ladie, and for the dead are Pſalmes. Beſides the ſeuen Pœnitential,
and fiftene Gradual Pſalmes, at certaine times. So that Clergie mens
daly office conſiſteth much in ſinging, or reading Pſalmes.
%%% o-1040
And therfore 
\MNote{Bishops bond to be skilful in Dauids Pſalter.}
al Byshops eſpecially, are ſtrictly bond by a particular Canon
\Cite{(Diſt.~38. cap. Omnes pſallentes)}
to be skilful in the Pſalmes of Dauid: and to ſee that 
\MNote{Other Prieſtes to haue competent knowlege therin.}
other Clergie men be wel inſtructed therin. According to the Holie
Ghoſts admonition, by the pen of the ſame Royal Prophet
\XRef{(Pſal.~46.)}
\L{Pſallite ſapienter}; or: \L{intelligenter}, that is, \Emph{Sing
Pſlames vvith knovvlege, and vnderſtanding them}. Not that euerie one is
bond to know, and be able to diſcuſſe al difficulties, but competently,
according to their charge vndertaken in Gods Church. Otherwiſe euerie
one that is, or intendeth to be a Prieſt, may remember what God
denounceth to him, by the Prophet Oſee
\XRef{(c.~4.)}
\Emph{Becauſe thou haſt repelled knovvlege, I vvil repel thee, that thou
do not the function of Prieſthood vnto me.} Thus much touching the
Author, the contentes, the poetical ſtile, & final cauſe of this holie
Pſalter.

As
\MNote{VVhy this booke is called the pſalter.}
for the name, S.~Ierom, S.~Auguſtin, and other Fathers teach, that
wheras amongſt innumerable muſical inſtruments, ſix were more ſpecially
vſed in Dauids time, mentioned by him in the laſt Pſalme. \Emph{Trumpet,
Pſalter, Harpe, Timbrel, Organ, and Cimbal.} This booke hath his name of
the inſtrument called Pſalter, which hath tenne ſtrings, ſignifying the
tenne commandements, and is made in
%%% 1149
forme (as S.~Ierom, and S.~Bede ſuppoſe) of the Greke, letter \G{Δ}
\Emph{delta}, becauſe as that inſtrument rendreth ſound from aboue, ſo
we ſhould attend to heauenlie vertues, which come from aboue:
\MNote{Other inſtrumentes make conſorte with the Pſalter.}
Likewiſe vſing the harpe, which ſignifyeth mortification of the fleſh, &
other inſtruments, which ſignifie and teach other vertues,
\MNote{Al vertues are referred to Gods honour.}
we muſt finally referre al to Gods glorie, reioyce ſpiritually in hart,
and render al praiſe to God.


\stopArgument


\stopcomponent


%%% Local Variables:
%%% mode: TeX
%%% eval: (long-s-mode)
%%% eval: (set-input-method "TeX")
%%% fill-column: 72
%%% eval: (auto-fill-mode)
%%% coding: utf-8-unix
%%% End:
