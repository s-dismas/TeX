%%%%%%%%%%%%%%%%%%%%%%%%%%%%%%%%%%%%%%%%%%%%%%%%%%%%%%%%%%%%%%%%%
%%%%
%%%% The (original) Douay Rheims Bible 
%%%%
%%%% Old Testament
%%%% Psalmes
%%%% Psalme 011
%%%%
%%%%%%%%%%%%%%%%%%%%%%%%%%%%%%%%%%%%%%%%%%%%%%%%%%%%%%%%%%%%%%%%%




\startcomponent psalme-011


\project douay-rheims


%%% 1166
%%% o-1058
\startChapter[
  title={Psalme 11}
  ]

\PSummary{The
\MNote{The ſtate of the Church in the firſt and laſt times of
  Chriſt. The 6.~key.}
Prophet deſcribeth the paucity of iuſt men, and 
\Fix{abundanc}{abundance}{obvious typo, fixed in other}
  of wicked, both at Chriſts firſt coming in flesh, 6.~and ſecond in
  maieſtie, in the end of the world.}

\PTitle{Vnto
\SNote{Chriſts firſt,}
the end for
\SNote{and laſt comming,}
the octaue, the
\SNote{wil bring ioy to the elect.}
Pſalme of Dauid.}

\VV Saue
\SNote{Chriſt calleth his myſtical bodie, him ſelfe.
\XRef{Act.~9. v.~4.}}
me Lord, becauſe the holy hath fayled
\SNote{Falſe and duble dealing hinder from true faith.}
becauſe verities are diminished from among the children of men.

\V They haue ſpoken vaine thinges euerie one to his neighbour,
deiceitful lippes, they haue ſpoken in hart and hart.

\V Our Lord deſtroy al deceitful lippes, & the tongue that ſpeaketh
\SNote{Inſolent & arrogant.}
great thinges.

\V Which haue ſaid: We wil magnifie our tongue, our lippes are of vs,
who is our Lord?

\V For the miſerie of the needie, and mourning of the poore, now eil I
ariſe, ſaith our Lord: I wil put in
\SNote{VVhen ſinne moſt abunded Chriſt came into this world: and in like
caſe wil come to iudge.}
a ſaluation: I wil do confidently in him.

\V
\CNote{Prouerb.~30q.}
Wordes of our Lord, be chaiſt wordes: ſiluer examined by fire, tryed
from the earth, purged ſeuen fold.

\V Thou Lord wilt
\SNote{Yet ſtil there remaine ſome iuſt whom God preſerueth.}
preſerue vs: and keepe vs from this generation for euer.

\V The
\LNote{The impious vvalke round about.}{S.~Auguſtin
\MNote{Temporal deſires hinder the entrance into heauen.}
expoundeth this of worldlie men deſiring temporal thinges, ſignified by
the ſeuen dayes, wherin this whole life is turned about, as in a whele,
not prouiding for the eight day, which is eternitie, after the day of
Iudgement. 
\CNote{\Cite{li.~12. c.~13. ciuit.}}
In an other place he ſheweth alſo, that this ſentence agreeth aptly to
the Platoniſtes, who taught, that 
\MNote{Platoniſtes error.}
this world neuer endeth, but paſſeth
and returneth round about, in a reuolution of manie yeares; ſo that al
thinges ſhould happen againe euen as they did before, contrarie to this,
and manie other Scriptures, affirming that God \Emph{vvil preſerue} the
iuſt, and kepe them \Emph{from this generation for euer}. VVhereas the
reprobate, who ſette their whole mind on temporal thinges, or expect a
reuolution of al, ſhal eternally walke without the kingdome of heauen, &
neuer enter in; though ſome may cal with the fooliſh virgins, ſaith
S.~Ierom (or ſome other learned author) vpon this place: \Emph{Lord, Lord,
open} (the dore) \Emph{to vs: but he vvil anſvver: that I knovv you
not.}
\XRef{Mat.~25.}}
impious walke round about: according to thy highnes thou haſt
\SNote{God ſometimes ſuffereth the wicked to do what euil they deſire.}
multiplied the children of men.


\stopChapter


\stopcomponent


%%% Local Variables:
%%% mode: TeX
%%% eval: (long-s-mode)
%%% eval: (set-input-method "TeX")
%%% fill-column: 72
%%% eval: (auto-fill-mode)
%%% coding: utf-8-unix
%%% End:
