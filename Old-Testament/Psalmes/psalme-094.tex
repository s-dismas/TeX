%%%%%%%%%%%%%%%%%%%%%%%%%%%%%%%%%%%%%%%%%%%%%%%%%%%%%%%%%%%%%%%%%
%%%%
%%%% The (original) Douay Rheims Bible 
%%%%
%%%% Old Testament
%%%% Psalmes
%%%% Psalme 94
%%%%
%%%%%%%%%%%%%%%%%%%%%%%%%%%%%%%%%%%%%%%%%%%%%%%%%%%%%%%%%%%%%%%%%




\startcomponent psalme-94


\project douay-rheims


%%% 1310
%%% o-1201
\startChapter[
  title={Psalme 94}
  ]

\PSummary{An
\MNote{Chriſt our Lord and king.

The 5.~key.}
inuitation to ſerue and adore Chriſt our Lord and Meſſias, 3.~aſwel for
the benefites of creating al thinges, 7.~as for his Incarnation, and not
to harden our hartes as the Iewes did.}

\PTitle{Praiſe
\SNote{Praiſe ſongue with voices:}
of Canticle,
\SNote{inſpired to Dauid, & written by him.}
to Dauid him ſelfe.}

%%% !!! The first verse?
\NV Come, let vs
\SNote{VVith great and ſolemne exultation:}
reioyce to our Lord: let vs make iubilation to God
\SNote{God our Creator, is alſo our Protector & Sauiour.}
our ſauiour.

%%% 1311
\V Let vs
\SNote{Let vs be more diligent, and preuent our accuſtomed time. For no
man can preuent Gods grace with anie good worke, who firſt preuenteth
vs, els we can neither doe, nor thincke anie good thing.}
preuent his face in confeſſion: and
\SNote{Not only in ſinging his praiſe with voice, but alſo with muſical
inſtruments.}
in Pſalmes let vs make iubilation to him.

\V Becauſe our Lord is a great God: and a great King aboue al goddes.

\V Becauſe in his hand are the endes of the earth: and the heightes of
the mountaines be his.

\V Becauſe the ſea is his, and he made it: and his handes formed the
drie land.

\V Come let vs adore,
\SNote{So alſo Iſaias
\XRef{(c.~45. v.~23.)}
and S.~Paul
\XRef{(Philip.~2.)}
teach that kneeling or bowing the knees, as an external religious
ceremonie is acceptable to God.}
and fal downe: and wepe before our Lord, that made vs.

\V Becauſe he is the Lord
\SNote{It is moſt iuſt and neceſſarie that we adore God, becauſe he made
vs, and al this world for vs, hath alſo redemed vs, and made vs his
people, as ſhepe of his paſture, and as a Paſtor feedeth and gouerneth
vs.}
our God; and we the people of his paſture, and the shepe
\SNote{Of his making.}
of his hand.

\V
\SNote{Though ſome haue often repelled, and reſiſted Gods grace, yet if
they receiue it being offered againe, it wil auaile them to remiſſion of
ſinnes.}
To day if ye shal heare his voice,
\LNote{Harden not your hartes.}{VVhatſoeuer
\MNote{It is in mans freewil to reſiſt good motions.}
God propoſeth by preaching, or inſpiration to a ſinner, it reſteth ſtil
in the powre of his freewil, to harden his harte, and to reiect al ſuch
good motions, and ſo he doth not only fruſtrate Gods grace, and hinder
his owne iuſtification, but alſo increaſeth his former ſinnes.
\CNote{\Cite{Concil. Triden. Seſſ.~6. c.~5.}}
But by not reſiſting, when deliberating therupon he could reſiſt, he
diſpoſeth himſelfe and cooperateth to firſt iuſtification. And therfore
the royal Prophet here admoniſheth, and earneſtly exhorteth al men, to
do this which God hath put in our powre, not to harden our owne hartes,
when we heare his voice, by reſiſting and reiecting his grace freely
offered, without al merite of our part.}
harden not your hartes;

\V As in the prouocation according to the day
\SNote{The Iſraelites in the deſert tempted God, by deſiring water, and
fleſh, of voluptuous concupiſcence without neceſſitie. For Manna did
both extinguiſh their thirſt, and taſted vnto them, whatſoeuer they
deſired:
\XRef{Exo.~16.}
That alſo which was left vngathered when the ſunne waxed hotte, melted
\XRef{(v.~21.)}
and ſerued their cattel for drincke. So this tentation was a figure of
thoſe, which require to communicate vnder both kindes, as if one did not
conteine as much as both.}
of the tentation in the deſert: where your fathers tempted me, proued
me, and ſaw my workes.

%%% o-1202
\V
\SNote{By this mention of the offence of fourtie yeares, as long before
paſſed, is conuinced that Moyſes writte not this Pſalme, who died in the
very fourtith yeare of their abode in the deſert. And S.~Paul citing
the wordes of this Pſalme
\XRef{(Heb.~4.)}
manifeſtly acknowlegeth Dauid the writter therof, and that it was
written long after Moyſes time in theſe wordes:
\XRef{(v.~7.)}
Againe he limiteth a certaine day; To day, in Dauid ſaying after ſo long
time, as is aboue ſaide. To day if you ſhal heare his voice, do not
obdurate your hartes. For if Ieſus (that is Ioſue) had geuen them reſt,
he would neuer ſpeake of an other day afterward.}
Fourtie years was I
\SNote{Being greatly offended, I approched nere vnto them, in puniſhing
the offenders.}
offended with that generation, and ſaid: Theſe alwaies erre in hart.

\V And theſe haue not knowne my waies: as I ſware in my wrath:
\SNote{Thoſe that murmured died in the deſert, and entered not into the
promiſed land, euen ſo thoſe that finally offend Chriſt, ſhal not enter
into euerlaſting reſt:
\XRef{Heb.~3. &.~4.}}
if they shal enter into my reſt.


\stopChapter


\stopcomponent


%%% Local Variables:
%%% mode: TeX
%%% eval: (long-s-mode)
%%% eval: (set-input-method "TeX")
%%% fill-column: 72
%%% eval: (auto-fill-mode)
%%% coding: utf-8-unix
%%% End:
