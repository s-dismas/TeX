%%%%%%%%%%%%%%%%%%%%%%%%%%%%%%%%%%%%%%%%%%%%%%%%%%%%%%%%%%%%%%%%%
%%%%
%%%% The (original) Douay Rheims Bible 
%%%%
%%%% Old Testament
%%%% Psalmes
%%%% Psalme 104
%%%%
%%%%%%%%%%%%%%%%%%%%%%%%%%%%%%%%%%%%%%%%%%%%%%%%%%%%%%%%%%%%%%%%%

%%% !!! The CNotes in this Pſalme are likely placed incorrectly.



\startcomponent psalme-104


\project douay-rheims


%%% 1326
%%% o-1216
\startChapter[
  title={Psalme 104}
  ]

\PSummary{The
\MNote{Gods ſpecial benefites towards the Iewes.

The 4.~key.}
Iſraelites are exhorted to ſing praiſes to God, 5.~for his meruelous
benefites towards Abraham, Iſaac, and Iacob. 11.~Whoſe particular
familie, being then ſmal, went from Chanaan into Ægypt (17.~Whither
Ioſeph by Gods prouidence was caried before) there increaſed in number,
was perſecuted, 26.~deliuered by Moyſes and Aaron, working manie great
miracles, 36.~protected, and fedde in the deſert, 44.~and finally
poſſeſſed Chanaan.}

\PTitle{
%%% !!! This isn't right
\SNote{Alleluia ſignifieth more then \L{Laudate Dominum}, Praiſe ye our
Lord. For by theſe two hebrew wordes, \HH{Alleluia}, the Prophet
inuiteth al men to praiſe God, with gladnes, and iubilation, with hart,
voice, and geſture, with inſtruments, and howſoeuer we are able. And
therfore S.~Ierom, S.~Auguſtin, and al Catholique writers kepe the ſame
worde, and tranſlate it not, neither in the titles of Pſalmes,  nor
ordinarily in anie place of holie Scripture. This is the firſt Pſalme
thus titled, and is the ſame Pſalme in ſenſe, and in good part of the
wordes, which the royal Prophet made, and cauſed to be ſongue, when he
brought the Arke of God from the houſe of Obededom into his owne houſe.
\XRef{1.~Par.~16. v.~8.}}
Alleluia.}

\NV
\MNote{For an expoſition of this Pſalme read the places quoted in the
inner margen.}
Confeſſe ye to our Lord, and inuocate his name:
\SNote{How much more gratful is it now to God, that we celebrate the
greater myſteries of the new Teſtament.}
shew forth his workes among the Gentiles.

\V Chaunt to him, and ſing to him: tel ye al his meruelous workes.

\V Prayſe ye him in his holie name: let the hart of them reioice that
ſeeke our Lord.

\V Seeke ye our Lord, and be confirmed: ſeeke
\SNote{His preſent helpe.}
his face alwayes.

\V Remember ye his meruelous workes, which he hath done: his wonders,
and the iudgments of his mouth.

%%% o-1217
\V The ſeede of Abraham, his ſeruantes: the children of Iacob his elect.

\V
\CNote{\XRef{Gen.~12. v.~7.}}
He is the Lord our God: in
\SNote{Not only in Iſrael, but in al the world.}
al the earth are his iudgementes.

\V
\CNote{\XRef{Gen.~17. v.~4.}}
He hath bene mindeful for euer of his teſtament; of the word, which he
commanded
\SNote{For euer, to the end of the world.}
vnto a thouſand generations.

\V
\CNote{\XRef{Gen.~26. v.~3.}}
Which he diſpoſed to Abraham: and of his oath to Iſaac.

\V
\CNote{\XRef{Gen.~28. v.~13.}}
And he appointed it to Iacob for a precept: and to Iſrael for an eternal
teſtament.

%%% 1327
\V
\CNote{\XRef{Gen.~46. v.~26.~27.}}
Saying: To thee wil I geue the land of Chanaan, the corde of your
inheritance.

\V When they were
\SNote{But 70.~perſons.}
of ſmal number, very few and ſeiourners therof:

\V And they paſſed from nation into nation, & from kingdom to an other
people.

\V He leift not a man to hurt them: and he rebuked kings for their ſake.

\V Touch not my annointed, and toward my prophetes be not malignant.

\V
\CNote{\XRef{Gen.~41. v.~54.}}
And
\SNote{By his prouidence ſuffered.}
he called a famine vpon the land: and he deſtroyed al the ſtrength of
bread.

\V
\CNote{\XRef{Gen.~37. v.~28.}}
He ſent a man before them: Ioſeph was ſold to be a ſeruant.

\V They humbled his feete in fetters, yron paſſed through his ſoule, \V
vntil his word came.

The word of our Lord inflamed him: \V
\CNote{\XRef{Gen.~39. &~ſeq.}}
the king ſent, and looſed him; the prince of the people, and releaſed
him.

\V He appointed him lord of his houſe: and prince of al his poſſeſſion.

\V That he might inſtruct his princes as himſelfe: and might teach his
ancientes wiſedom.

\V
\CNote{\XRef{Gen.~46.}}
And Iſrael entered into Ægypt, and Iacob was a ſeiourner in the land
\SNote{Ægypt poſſeſſed by Meſraim Chams ſecond ſonne.
\XRef{Gen.~10. v.~13.}}
of Cham.

\V
\CNote{\XRef{Exo.~1. v.~7.}}
And he increaſed his people excedingly: and ſtrengthned them ouer their
enemies.

\V He
\SNote{May it be vnderſtood, or beleued (ſaith
%%% !!! Cite?
S.~Auguſtin) that God turneth the hart of man to committe ſinnes? Or is
it no ſinne, or is it a ſmal ſinne, to hate the people of God? Or to
worke guile towards his ſeruants? VVho wil ſay this? VVhat then, is God
author of theſe ſo greeuous ſinnes, who is not to be ſuppoſed the author
of a moſt ſmal ſinne? This lerned Father therfore anſwereth, that God
peruerted not a right hart, but turned that was of it ſelfe peruerſe, to
the hatred of his people, where he might vſe that euil wil, not by
making them euil, but by beſtowing vpon his owne people good thinges,
which the euil might eaſily enuie. VVhich hatred of theirs how God vſed
both to the exerciſe of his people (which is profitable to vs) & to the
glorie of his owne name, the thinges that folow do teach vs, which are
here remembred to his praiſe.}
turned their hart, that they hated his people: and to worke guile toward
his ſeruantes.

\V
\CNote{\XRef{Exo.~3.~4. 7.~8.~9.~10.~11.}}
He ſent Moyſes his ſeruant: Aaron,
\SNote{In whom God eſtabliſhed the
\Fix{Prieſtgood}{Prieſthood}{obvious typo, fixed in other}
of Moyſes law.}
him ſelfe whom he choſe.

%%% o-1218
\V He did put in them the wordes of his ſignes, and of his wonders in
the Land of Cham.

\V He ſent
\SNote{The
\Fix{ointh}{ninth}{obvious typo, fixed in other}
plague of the Ægyptians.}
darkenes, and obſcured: and did
\SNote{God willingly, not as one loath or vnwilling, performed al that
he threatned.}
not exaſperate his wordes.

%%% 1328
\V He turned their
\SNote{The firſt plague.}
waters into bloud: and killed their fishes.

\V Their land brought forth
\SNote{The ſecond plague.}
frogges in
\SNote{Dauid knew this by reuelation, or by tradition for it is not in
Exodus.}
the inner chambers of their kinges.

\V He ſayd, and
\SNote{The fourth plague.}
the
\SNote{A ſvvarme of flies.}
cænomyia came: and the
\SNote{The third plague.}
ciniſes in al their coaſtes.

\V He made theyr raynes
\SNote{The ſeuenth plague.}
haile: fire burning in their land.

\V And he ſtroke their vines, and their figtrees: and he deſtroyed the
wood of their coaſtes.

\V He ſaid, &
\SNote{The eight plague.}
the locuſt came, and the
\SNote{A worme that ſpoyleth corne, graſſe, and fruict.}
bruchus wherof there was no number.

\V And it did eate al the graſſe in their land: and it did eate al the
fruicte of their land.

\V And he ſtroke euerie
\SNote{The tenth plague. The fifth & ſixt of peſtilence and boyles are
omitted.}
firſt begotten in their land: the firſt fruictes of al their labour.

\V
\CNote{\XRef{Exod.~12. v.~35.}}
And he brought them forth with gold and ſiluer, and there was not in
their tribes a feeble perſon.

\V Ægypt was glad at their departure: becauſe the feare of them lay vpon
them.

\V
\CNote{\XRef{Exod.~13. v.~21.}}
He ſpred a cloude for their protection, and fire to shine vnto them by
night.

\V
\CNote{\XRef{Exod.~16. v.~13.}}
They made petition, and the quaile came: and he filled them with the
bread of heauen.

\V
\CNote{\XRef{Exod.~17. v.~6.}}
He diuided the rocke, and waters flowed: riuers ranne in the drie
ground.

\V Becauſe he was mindful of his holie word,
\CNote{\XRef{Gen.~12.}}
which he had vttered to Abraham his ſeruant.

\V And he brought forth his people in exultation, and his elect in ioy.

\V
\CNote{\XRef{Ioſue.~6. &~ſeq.}}
And he gaue them the countries of the Nations: and they poſſeſſed the
labours of peoples:

\V That they might keepe his iuſtifications, and ſeeke after his lawe.


\stopChapter


\stopcomponent


%%% Local Variables:
%%% mode: TeX
%%% eval: (long-s-mode)
%%% eval: (set-input-method "TeX")
%%% fill-column: 72
%%% eval: (auto-fill-mode)
%%% coding: utf-8-unix
%%% End:
