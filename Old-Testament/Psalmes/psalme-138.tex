%%%%%%%%%%%%%%%%%%%%%%%%%%%%%%%%%%%%%%%%%%%%%%%%%%%%%%%%%%%%%%%%%
%%%%
%%%% The (original) Douay Rheims Bible 
%%%%
%%%% Old Testament
%%%% Psalmes
%%%% Psalme 138
%%%%
%%%%%%%%%%%%%%%%%%%%%%%%%%%%%%%%%%%%%%%%%%%%%%%%%%%%%%%%%%%%%%%%%


\startcomponent psalme-138


\project douay-rheims


%%% 1385
%%% o-1276
\startChapter[
  title={Psalme 138}
  ]

\PSummary{Gods
\MNote{Gods ſpecial prouidence of his ſeruantes.

The 3.~key.}
knowlege, 7.~and preſence (10.~without the helpe, or hinderance of anie
thing) extendeth to al thinges, times, and places. 17.~He geueth
exceding great honour to his ſainctes, 20.~the wicked, as enimies to God
are iuſtly hated. 23.~The iuſt pray for Gods perpetual direction.}

%%% 1386
\PTitle{Vnto
\SNote{By this part of the title (to the end) is ſignified (as is noted
\XRef{Pſal.~4.)}
that the matter conteined in the Pſalme, perteyneth to the nevv
Teſtament.}
the end, a Pſalme of Dauid.}

\NV Lord thou
\SNote{God vvho knovveth al thinges moſt abſolutly and perfectly,
vvithout diſcourſe or ſearching, yet, as it vvere, maketh experimental
trial of his ſeruants, to make them in ſome ſorte to know him, and to
knovv themſelues. And ſo here, holie Dauid or other faithful man,
acknovvlegeth Gods Omniſcience, that is, perfect knovvlege of al
thinges, vvithout exception, paſt, preſent, & to come: al vvorkes,
vvordes, thoughtes, and vvhat ſoeuer can be, though it neuer vvas nor
ſhal be, in general and in particular.}
haſt proued me, and haſt knowen me: \V thou haſt knowen my ſitting
downe, and my riſing vp.

\V Thou haſt vnderſtood my cogitations far of: my path and
\SNote{The vttermoſt meaſure and reach of myne intention.}
my corde thou haſt ſearched out.

\V And thou haſt foreſene al my wayes: becauſe there is not a word in my
\SNote{The word holden in by the tongue, and not vttered by mouth, is
not hidden from God.}
tongue.

\V Behold ô Lord thou haſt knowen al the laſt thinges, & them of old:
thou haſt formed me, and haſt put thy hand vpon me.

\V Thy knowledge is
\SNote{By experiẽce we ſee that Gods knovvlege excedeth our reach.}
become meruelous of me: it is made great, and I can not reach to it.

\V
\SNote{As Gods knovvlege comprehendeth al thinges, ſo his preſence
extendeth it ſelfe to al places, neither is conteined in place, but
excedeth al place, in his diuine immenſitie.}
Wither shal I goe from thy ſpirit? and wither shal I flee from thy face?

\V If I shal aſcend into heauen, thou art there: if I deſcend into hel,
thou art preſent.

\V If I shal take my winges early, and dwel in the extreme partes of the
ſea:

\V Certes thither alſo shal thy hand conduct me: and thy right hand shal
hold me.

\V
\SNote{The Prophet alſo in the perſon of anie curious imaginatiue man,
examineth and findeth that no darknes, nor couer can hide anie thing
from God.}
And I ſayd: Perhaps darknes shal treade ouer me: and the night is mine
illumination in my delightes.

%%% o-1277
\V For darkenes shal not be darkened from thee, and the night shal be
lightened as the day: as the darkenes therof, ſo alſo the light therof.

\V Becauſe thou haſt poſſeſſed
\SNote{Nothing ſemeth more hidden, then a mans entrals,}
my reynes: thou haſt receiued me from
\SNote{or a child in the mothers vvombe.}
my mothers wombe.

\V I wil confeſſe to thee, becauſe thou art terribly magnified: thy
workes are meruelous, & my ſoule knoweth excedingly.

%%% 1387
\V My
\SNote{Or bones in the fleſh.}
bone is not hid from thee, which thou madeſt in ſecrete: and my
ſubſtance in the lower partes of the earth.

\V
\SNote{Or mans bodilie imperfection before his birth,}
Mine
\TNote{Golem, \L{Embryonem}.}
imperfection thine eies haue ſene, & in thy booke
\TNote{of knovvlege.}
al shal be written:
\SNote{dayly formed by God, not by man.
\XRef{Iob.~10. v.~8.}
\XRef{2.~Mac.~7. v.~22,~23.}}
daies shal be formed, & no man in them.

\V
\SNote{Aboue al conſiderations it moſt excedeth, that God ſo high and
infinite, honoreth his humble poore ſeruants ſo excedingly, that it
ſemeth to themſelues, farre more then can be due. For he revvardeth
euen ouer & aboue merites; which merites alſo are founded in Gods mercie
geuen vvithout merite.}
But to me thy
\SNote{Nevv tranſlaters peruert this place, tranſlating (thoughts) for
(frendes) contrarie to the Hebrevv, Greke, and Latin, and al ancient
Fathers, only pretending that the ſame vvord in the Chaldee tongue alſo
ſignifieth thoughts.}
frendes ô God are become honorable excedingly: their principalitie is
excedingly ſtrengthned.

\V
\SNote{The number alſo of Saintes vvhom God hath choſen, called,
iuſtified and vvil glorifie, excede mans conceipt.
\XRef{Apoc.~7.}}
I wil number them, and they shal be multiplied aboue the ſand:
\SNote{Incenſed vvith this excellent glorie, and deſiring to be of this
innumerable multitude, by thy grace I haue riſen from ſinne, and in
confidence of thy perpetual helpe, I ſtand and hope to perſeuere in thy
ſeruice.}
I roſe vp and I am yet with thee.

\V
\SNote{And if it be ſo, yea ſeing it is ſo, that as thou ô God doeſt
exalt thy Saincts, to exceding and vnſpeakable honour: ſo thou haſt alſo
decreed to puniſh obſtinate ſinners, vvith eternal death and damnation:}
If thou shalt kil ſinners ô God:
\SNote{I renounce al vvicked aſſociation, gette ye avvay from me al
cruel bloudie men, that vvould dravv me into euerlaſting torments.}
ye men of blood depart from me.

\V
\SNote{Avvay from me, you that thinke,}
Becauſe you ſay in thought: they
\SNote{the glorious manſions in heauen, prepared and promiſed to the
iuſt, are vaine hopes, and in vaine expected.}
shal receiue thy cities in vayne.

\V
\SNote{This hate of ſuch ſinners the iuſt ſhal confidently plead, and
happie are they that ſhal be able truly to alleage for themſelues in the
day of iudgement, that they hated al, vvhom God hateth,}
Did not I hate them, that hate thee ô Lord: and
\SNote{yea hated them vvith feruent zele, that are Gods enimies.}
pyned away becauſe of thine enemies?

\V
\SNote{Stil the Prophet inculcateth this neceſſarie perfect hatred,}
With perfect hatred did I hate them: they are become
\SNote{and emnitie tovvards Gods enimies.}
enemies to me.

%%% o-1278
\V
\SNote{For that none in this life (without ſpecial and extraordinarie
reuelation) knoweth certainly their owne ſtate, whether they be worthie
of Gods loue or hatred
\XRef{(Eccle.~9.)}
the iuſt ſubmitte themſelues to Gods examination of their hart and
actions:}
Proue me ô God, and know my hart: examine me, and know my pathes.

\V And
\SNote{humbly praing God, that if they be in the way of iniquitie,}
ſee, if the way of iniquitie be in me:
\SNote{he wil voutſafe to reduce and guid them into the right way of
euerlaſting life.}
and conduct me in the euerlaſting way.


\stopChapter


\stopcomponent


%%% Local Variables:
%%% mode: TeX
%%% eval: (long-s-mode)
%%% eval: (set-input-method "TeX")
%%% fill-column: 72
%%% eval: (auto-fill-mode)
%%% coding: utf-8-unix
%%% End:
