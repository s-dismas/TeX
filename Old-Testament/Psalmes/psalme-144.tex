%%%%%%%%%%%%%%%%%%%%%%%%%%%%%%%%%%%%%%%%%%%%%%%%%%%%%%%%%%%%%%%%%
%%%%
%%%% The (original) Douay Rheims Bible 
%%%%
%%%% Old Testament
%%%% Psalmes
%%%% Psalme 144
%%%%
%%%%%%%%%%%%%%%%%%%%%%%%%%%%%%%%%%%%%%%%%%%%%%%%%%%%%%%%%%%%%%%%%


\startcomponent psalme-144


\project douay-rheims


%%% 1394
%%% o-1284
\startChapter[
  title={Psalme 144}
  ]

\PSummary{God
\MNote{Gods Maieſtie excelleth al thinges.

The 1.~key.}
is, and for euer ought to be praiſed, 3.~for his immenſiue, infinite,
glorious Maieſtie, meruelous workes, merciful benefites; for his powre,
wiſdom, iuſtice, 19.~who wil reward the good, and deſtroy the wicked.}

\PTitle{Prayſing,
%% !!! SNote should go before 'Prayſing'
\SNote{By this title, Eſdras ſignifieth that the Holie Ghoſt, vvho
indited al the Pſalmes to Gods praiſe, more ſpecially in theſe ſeuen
laſt ſuggeſted to Dauid, and by him to al Gods ſeruants, that al their
other ſeruice muſt tend, and be directed to the praiſe of God: and that
therin vve muſt continevv, and finally reſt, as in the ſabbath of the
ſeuenth day, ſignified (as
%%% !!! Cite?
S.~Beda ſuppoſeth) by theſe ſeuen laſt Pſalmes of praiſe, eternally
praiſing our Lord God. For vvhich principal end both Angels and Men, yea
and al other creatures vvere made.}
to Dauid himſelfe.}

\NV I
\LNote{I wil exalt thee.}{As
\MNote{The ſeuen laſt Pſalmes perteyne more ſpecially to prayſes.}
this Pſalme is the firſt of the ſeuen, vvhich conteyne more particular
inſtruction of perpetually praiſing God:
\MNote{This Pſalme and other ſix are compoſed in order of the Alphabet.}
ſo it is the ſeuenth of thoſe, vvhich are compoſed in order of the
Alphabet, to vvitte, the
\XRef{24.}
\XRef{33.}
\XRef{36.}
\XRef{110.}
\XRef{111.}
\XRef{118.}
and this
\XRef{144.}
Of vvhich the three former vvant ſome letters: ſignifying (as
%%% !!! Cite?
Caſſiodorus interpreteth) ſuch in Gods Church, as ſing his praiſes, but
vvith ſome imperfections: the other foure haue the perfect Alphabet,
ſignifying thoſe, that ſing Gods praiſes vvith perfect deuotion. VVhich
only foure S.~Ierom calleth Alphabetical Pſalmes.
\Cite{Epiſt. ad Paulum Vrbicam.}
&
\Cite{Proæm. in Lament. Ierom.}}
wil exalt thee my God
\SNote{King, is the proper epitheton of Chriſt, the Sonne of God, to
whom, in his humanitie, God the Father promiſed the Church of al nations
for his kingdom.
\XRef{Pſal.~2.}
in vvhom alſo the vvhole Bleſſed Trinitie is praiſed.}
the king: and I wil bleſſe thy name
\SNote{Al the time of this vvorld they praiſe God,}
for euer, and
\SNote{& after in eternitie.}
for euer and euer.

\V Euerie day wil I bleſſe thee: and wil praiſe thy name for euer, and
for euer and euer.

%%% o-1285
\V Great is our Lord and exceding laudable, and of his greatnes there is
no end.

\V Generation and generation shal praiſe thy workes: and they shal
pronounce thy powre.

\V They shal ſpeake the magnificence of the glorie of thy holines: and
shal tel thy meruelous workes.

\V And they shal tel the force of thy
\SNote{Of vvonderful and miraculous thinges, vvhich ſtrike terrour into
mens mindes.}
terrible thinges: and shal declare thy greatnes:

\V They shal vtter the memorie of the abundance of thy ſwetnes: and in
thy iuſtice they shal reioyce.

\V Our Lord is pitiful and merciful: patient and very merciful.

\V Our Lord is ſweete to al: and his
\SNote{The effectes of Gods mercie in redeming, and
\Fix{recallidg}{recalling}{obvious typo, fixed in other}
ſinners, are eminent aboue al other workes.}
commiſerations are ouer al his workes.

\V Let al thy workes ô Lord confeſſe to thee: and let thy
\SNote{Therfore the ſanctified haue ſpecial cauſe to praiſe God.}
ſainctes bleſſe thee.

\V They shal tel the glorie of thy kingdom: and shal ſpeake thy might.

\V That they may make thy might knowne to the children of men: and the
glorie of the magnificence of thy kingdom.

\V Thy kingdom is a kingdom
\SNote{Chriſts kingdom the militant Church is magnifical, but much more
the triumphant vvhich is eternal.}
of al worldes: and thy dominion in al generation and generation.

%%% 1395
\V
\LNote{Our Lord is faithful.}{This
\MNote{It is probable that the Hebrevv text novv vvanteth a verſe in
this Pſalme.}
verſe is not novv in the ordinarie Hebrevv text, and therfore either the
ſame is defectiue, or els this Pſalme ſhould ſeme not to be compoſed
vvith a perfect Alphabet in the fountaine tongue. For here it vvanteth
the letter Nun.
\MNote{And therfore is not more certaine then the Greke or Latin.}
But ſeing S.~Ierom counteth this one of the foure
Alphabetical Pſalmes, omitting the other three, vvhich conſiſt of
vnperfect Alphabets, it is very probable that this verſe vvas once in
the Hebrevv text, as it is both in Greke & Latin. VVherby amongſt
other places, appeareth, that there is no certaintie, to correct the
Greke, or Latin Bible by the Hebrevv, vvhich is novv extant; but rather
by them that may be ſupplied, vvhich the Hebrevv vvanteth.}
Our Lord is faithful in al his wordes: and holie in al his workes.

\V Our Lord
\SNote{God is readie of his part to lift vp al.}
lifteth vp al that fal: and ſetteth vp al that are bruiſed.

\V The eies of al hope in thee ô Lord: and thou geueſt their meate in
time conuenient.

\V Thou openeſt thy hand: and filleſt
\SNote{He geueth neceſſarie thinges to al liuing creatures, euen to
brute beaſtes.}
euerie liuing creature with bleſſing.
 
\V Our Lord is iuſt in al his wayes: and holie in al his workes.

\V Our Lord is neere to al that inuocate him: to al that inuocate him in
truth.

\V He wil doe the wil of them that feare him, and wil heare their
prayer, and ſaue them.

\V Our Lord keepeth al that loue him: and he wil deſtroy al ſinners.

\V My mouth shal ſpeake the prayſe of our Lord: and let al flesh bleſſe
his holie name for euer, and for euer and euer.


\stopChapter


\stopcomponent


%%% Local Variables:
%%% mode: TeX
%%% eval: (long-s-mode)
%%% eval: (set-input-method "TeX")
%%% fill-column: 72
%%% eval: (auto-fill-mode)
%%% coding: utf-8-unix
%%% End:
