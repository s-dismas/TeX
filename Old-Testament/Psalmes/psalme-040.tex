%%%%%%%%%%%%%%%%%%%%%%%%%%%%%%%%%%%%%%%%%%%%%%%%%%%%%%%%%%%%%%%%%
%%%%
%%%% The (original) Douay Rheims Bible 
%%%%
%%%% Old Testament
%%%% Psalmes
%%%% Psalme 40
%%%%
%%%%%%%%%%%%%%%%%%%%%%%%%%%%%%%%%%%%%%%%%%%%%%%%%%%%%%%%%%%%%%%%%




\startcomponent psalme-40


\project douay-rheims


%%% 1217
%%% o-1108
\startChapter[
  title={Psalme 40}
  ]

\PSummary{The 
\MNote{Chriſts Paſſion and Reſurrection.

The 5.~key.}
prophet pronounceth them happie that wil beleue in Chriſt, coming in
humilitie and pouertie. 5.~Chriſt deſcribeth his owne poore afflicted
ſtate in this life, by reaſon he is to ſatisfie for the ſinnes of the
world; the malice of his aduerſaries, 10.~eſpecially of Iudas; 11.~and
by way of prayer, prophecieth his owne Reſurrection.}

\PTitle{Vnto
\SNote{Perteyning to the new Teſtament, as appeareth by the
\XRef{10.~verſe.}
\CNote{Io.~13. v.~18.}
alleaged by our Sauiour.

This Pſalme is alſo applied by the Church in the office of the ſick,
whom whoſoeuer aſſiſteth in that caſe, may hope to haue aſſiſtance in
their owne like neceſſitie.}
the end, a Pſalme to Dauid him ſelfe.}

\VV Bleſſed is the man that vnderſtandeth concerning
\SNote{He is happie that is not ſcandalized in Chriſt
\XRef{(Luc.~7. v.~13.)}
coming in pouertie, and ſuffering extreme afflictions.}
the needie, and the poore: in
\SNote{He that truſteth in Chriſt, notwithſtanding the contrarie motiues
of his worldlie miſerie, shal be deliuered by him in al diſtreſſe.}
the euil day our Lord wil deliuer him.

\V Our Lord
\SNote{Our Lord wil geue to ſuch ſeruantes more grace in this life, and
glorie in the next,}
preſerue him, and geue him life, and make him bleſſed in the land: and
\SNote{nor ſuffer him to be ouercome in tentations.}
deliuer him not vnto the wil of his enemies.

\V Our Lord helpe him
\SNote{When ſuch conſtant ſeruantes are ſick to death, Chriſt wil moſt
eſpecially comforte and helpe them.}
vpon the bed of his ſorow: thou haſt turned al his couche in his
infirmitie.

%%% o-1109
\V I ſaid:
\SNote{Chriſt in the behalf of his myſtical bodie confeſſeth their
ſinnes, and prayeth for them.}
Lord
\Fix{haue haue}{haue}{obvious typo, fixed in other}
mercie on me: heale my ſoule, becauſe I haue ſinned to thee.

\V Mine enemies haue ſpoken euils to me: When shal he die,
\SNote{After death ſuffered for mankind Chriſt riſeth, and his name and
kingdom is glorious.}
and his name perish?

\V And if
\SNote{Thoſe that came not of good wil, but of malice to obſerue Chriſts
deedes and wordes, carped at both, ſometimes ſaying, he taught againſt
the law, and againſt Moyſes; ſometymes that he caſt out diuels in the
powre of Beelſebub.}
he came in to ſee, he ſpake vayne thinges: his hart hath gathered
together iniquitie to him ſelfe.

%%% 1218
He went forth and ſpake together.

\V Al mine enemies whiſpered againſt me: they did thinke euils to me.

\V They
\SNote{At laſt they reſolued that he should die.}
haue determined an vniuſt word againſt me:
\SNote{But they could not ſo ſuppreſſe his powre, for he roſe againe in
glorie.}
Shal not he that ſleepeth adde to ryſe againe?

\V 
\CNote{\XRef{Io.~13.}
\XRef{Act.~1.}}
For
\SNote{By our Sauiours application of this verſe, it is certaine that
the traitor Iudas is here deſcribed.
\XRef{Ioa.~13. v.~18.}}
the man alſo of my peace, in whom I hoped: who did eate my breades, hath
greatly troden me vnder foote.

\V But thou ô Lord haue mercie vpon me, and raiſe me vp againe: and I
\SNote{In the day of iudgement Chriſt Iudge of al wil render to euerie
one as they deſerue.}
wil repay them.

\V In this I haue knowen that thou wouldeſt me: becauſe mine enemie shal
not reioyce ouer me.

\V But me thou haſt receiued
\SNote{As before in reſpect of ſinners, Chriſt Iudge of al wil render to
euerie one: ſo here in his owne perſon he auoucheth his owne innocencie,
which made him apt to ſatisfie for others.}
becauſe of innocencie: and thou haſt confirmed me in thy ſight for euer.

\V Bleſſed be our Lord the God of Iſrael
\SNote{For this mercie of Almightie God in ſauing the elect by his
Sonnes death, he is to be praiſed for euer eternally.}
from the beginning of the world, and for euermore:
\SNote{Al the bleſſed agree in this, that God is eternally to be praiſed
and therto ſay \Emph{Amen}. So be it, ſo be it.}
Be it, be it.

%%% !!! Afterword to Psalme?
Some diuide the Pſalmes into fiue bookes, ſuppoſing the firſt booke to
end here with theſe wordes: \Emph{Be it, be it}: not obſeruing that the
laſt Pſalme hath not this ending. S.~Ierom confuteth this opinion by our
Sauiours, and S.~Peters naming it the booke, not bookes of Pſalmes.
\XRef{Luc.~20. v.~42.}
\XRef{Act.~1.}
Moreouer if this were the end of one booke, then the Pſalme folowing
ſhould not be called the 41.~Pſalme, but the firſt Pſalme of the ſecond
booke.


\stopChapter


\stopcomponent


%%% Local Variables:
%%% mode: TeX
%%% eval: (long-s-mode)
%%% eval: (set-input-method "TeX")
%%% fill-column: 72
%%% eval: (auto-fill-mode)
%%% coding: utf-8-unix
%%% End:
