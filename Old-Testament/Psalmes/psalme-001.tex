%%%%%%%%%%%%%%%%%%%%%%%%%%%%%%%%%%%%%%%%%%%%%%%%%%%%%%%%%%%%%%%%%
%%%%
%%%% The (original) Douay Rheims Bible 
%%%%
%%%% Old Testament
%%%% Psalmes
%%%% Psalme 01
%%%%
%%%%%%%%%%%%%%%%%%%%%%%%%%%%%%%%%%%%%%%%%%%%%%%%%%%%%%%%%%%%%%%%%




\startcomponent psalme-01


\project douay-rheims


%%% 1150
%%% o-1042
\startChapter[
  title={Psalme 1}
  ]

\Summary{The
\MNote{The right maner of ſeruing God. The 7.~key.}
Royal prophet Dauid placed this Pſalme as a Preface to the
  reſt, conteyning, 1.~true happines, which conſiſteth in flying ſinnes,
  and ſeruing God. 3.~The good doe proſper, 5.~not the wicked: 6.~as wil
  appeare in the end of this world.}

Bleſſed
\CNote{\XRef{Mat.~5.}}
\SNote{He is in the right way to eternal felicitie.}
is the man, that
\LNote{Hath not gone, not ſtood, not ſitte.}{The
\MNote{They are happie (in hope) that decline from euil.}
Hebrew ſtile, and maner of diſcourſe differeth here from other nations,
in mentioning firſt the leſſe euil, and the greateſt laſt. VVhereas we
would ſay in the contrary order: He is happie that hath
not \Emph{ſitte}, that is, hath not ſetled himſelfe in wickednes, nor
finally perſiſted obſtinate: more happie, that hath not \Emph{ſtood},
anie notable time continued in ſinne: and moſt happie, that hath not
\Emph{gone}, not geuen anie conſent at al to euil ſuggeſtions.}
hath
\SNote{not conſented to euil ſuggeſtions.}
not gone in the counſel of the impious, & hath
\SNote{not continued in ſinne.}
not
%%% !!! This LNote is part of the above.
%%% \LNote{}{}
ſtoode in the way of ſinners, and hath
\SNote{not finally perſiſted in wicked life.}
not
%%% !!! This LNote is part of the above.
%%% \LNote{}{}
ſitte in the chayre of peſtilence:

\V
\CNote{\XRef{Ioſ.~1.}}
But
\SNote{He is wholly occupied & delighted in keeping Gods commandments.}
his
\LNote{His vvil in the vvay of our Lord.}{As
\MNote{Iuſtice conſiſteth in fleing euil and doing good.}
one part of happines conſiſteth in declining from euil: ſo the other is
in doing good; the wil deſiring, and diligently endeuoring to walke in
the way of vertue, and law of God. VVhich is true iuſtice, and right
forme of good life, propoſed in this Pſalme, for attayning eternal
beatitude.}
wil is in the way of our Lord, and in his law he wil meditate day and
night.

\V
\CNote{\XRef{Iere.~17.}}
And he shal be as a tree, that is planted nigh to
\SNote{To him that vſeth Gods grace wel, more grace is continually
geuen.}
the ſtreames of waters, which shal geue his fruite in his time:

\V And
\SNote{Through ſuch grace he shal perſeuer.}
his leafe shal not fal: and
\SNote{Al thinges worke to the good of them that loue God ſincerely.}
al thinges whatſoeuer he shal doe, shal proſper.

\V The impious not ſo: but
\SNote{The wicked are carried with euery light tentation.}
as duſt, which the winde driueth from the face of the earth.

\V Therfore the impious shal
\SNote{Al ryſing at the laſt day, the wicked shal not riſe with hope nor
comforth, but in deſolation.}
not riſe againe in iudgement: nor ſinners in the
\SNote{The happie congregation of the bleſſed.}
councel of the iuſt.

\V For our Lord
\SNote{Approueth & rewardeth.}
knoweth the way of the iuſt, and the way of the impious
\SNote{In eternal damnation.}
shal perish.


\stopChapter


\stopcomponent


%%% Local Variables:
%%% mode: TeX
%%% eval: (long-s-mode)
%%% eval: (set-input-method "TeX")
%%% fill-column: 72
%%% eval: (auto-fill-mode)
%%% coding: utf-8-unix
%%% End:
