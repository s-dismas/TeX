%%%%%%%%%%%%%%%%%%%%%%%%%%%%%%%%%%%%%%%%%%%%%%%%%%%%%%%%%%%%%%%%%
%%%%
%%%% The (original) Douay Rheims Bible 
%%%%
%%%% Old Testament
%%%% Psalmes
%%%% Psalme 136
%%%%
%%%%%%%%%%%%%%%%%%%%%%%%%%%%%%%%%%%%%%%%%%%%%%%%%%%%%%%%%%%%%%%%%


\startcomponent psalme-136


\project douay-rheims


%%% 1383
%%% o-1274
\startChapter[
  title={Psalme 136}
  ]

\PSummary{The
\MNote{The Iewes lamentation in captiuitie.

The 4.~key.}
Prophet deſcribeth how lamentably the people in captiuitie of Babylon,
wil bewaile the want of meanes to ſerue God, and of their natiue ſoyle:
7.~with iuſt deſire of their enimies punishment.}

\PTitle{A Pſalme of Dauid
\SNote{By adding to this title (for Ieremie) the Septuagint Interpreters
ſignifie that this Pſalme treateth of the ſame captiuitie, in which
Ieremias writte his Lamentations.}
for Ieremie.}

%%% 1384
\NV Vpon the riuers
\SNote{Nere the riuers in Chaldea; wherof Babylon was the head citie,}
of Babylon, there we
\SNote{the Iewes remained mourning,}
ſate and wept: whiles we
\SNote{remembring the holie rites and ſeruice of God, which had bene in
Sion, wherof they were depriued in the captiuitie.}
remembred Sion.

\V On the willowes in the middes therof, we hanged vp
\SNote{Al their muſical inſtruments, as hauing no vſe of them.}
our inſtrumentes.

\V Becauſe there they that led vs captiue,
\SNote{Either in earneſt, or in ſcorne the Chaldees willed them to ſing,
as they were accuſtomed in their countrie.}
demanded of vs wordes of ſonges.

And they that led vs away: Sing ye an hymne to vs of the ſonges of Sion.

\V
\SNote{They excuſed themſelues, and refuſed to ſing ſacred Pſalmes
before prophane people, neither had they mind to ſing in that mourning
ſtate of captiuitie.}
How shal we ſing the ſong of our Lord in a ſtrange land?

\V
\SNote{The people ſhew not only their feruent preſent deſire to ſerue
God in Ieruſalem, but alſo their firme purpoſe ſtil to deſire the ſame,
wiſhing that if they forget it, or loſe this affection, their right
handes, or whatſoeuer is moſt deare, or neceſſarie for them, may be
forgotten, not conſerued, but ſuffered to periſh.}
If I shal forget thee ô Ieruſalem, let my right hand be forgotten.

\V
\SNote{If I loſe this affection, let me alſo loſe the vſe of my tongue.}
Let my tongue cleaue to my iawes, if I doe not remember thee:

If I shal not ſet Ieruſalem in the beginning of my ioy.

\V Be mindful ô Lord of
\SNote{The Idomeans incenſed the Chaldees to be cruel againſt the Iewes,
wherof they pray for iuſt reuenge, and withal the Pſalmiſt prophecieth
that it wil be reuenged, which Iſaias alſo prophecieth,
\XRef{c.~21. v.~11.}}
the children of Edom, in
\SNote{for their reioycing in Ieruſalems miſerie.}
the day of Ieruſalem:

That
%%% !!! SNote should go before 'That'
\SNote{The voice of the Idumeans, inciting the Babylonians vtterly to
deſtroy Ieruſalem.}
ſay: Raſe it, raſe it, euen vnto the foundation therof.

\V
\SNote{A prophecie that the people of Babylon ſhould alſo be puniſhed,
for their crueltie againſt the Iewes, wherof Iſaias likewiſe
prophecieth.
\XRef{c.~13.}}
Daughter of Babylon miſerable: bleſſed is he, that shal repay thee thy
payment, which thou haſt payed vs.

%%% o-1275
\V Bleſſed is he,
\SNote{God wil bleſſe, or reward them that ſhal ſeuerly afflict the
Babylonians,}
that shal hold, and
\SNote{not ſparing their children. Morally he is bleſſed, that
mortifieth his owne paſſions, cutteth of firſt il motions, or puniſheth
venial ſinnes, that they grow not ſtrong vvithin his ſoule, and ſo draw
it to committe mortal ſinne.
\Cite{S.~Aug. hic.}
&
\Cite{S.~Greg. in fine expoſ. Pſal.~4. pænit.}}
shal dash thy litle ones againſt the rocke.


\stopChapter


\stopcomponent


%%% Local Variables:
%%% mode: TeX
%%% eval: (long-s-mode)
%%% eval: (set-input-method "TeX")
%%% fill-column: 72
%%% eval: (auto-fill-mode)
%%% coding: utf-8-unix
%%% End:
