%%%%%%%%%%%%%%%%%%%%%%%%%%%%%%%%%%%%%%%%%%%%%%%%%%%%%%%%%%%%%%%%%
%%%%
%%%% The (original) Douay Rheims Bible 
%%%%
%%%% Old Testament
%%%% Psalmes
%%%% Psalme 06
%%%%
%%%%%%%%%%%%%%%%%%%%%%%%%%%%%%%%%%%%%%%%%%%%%%%%%%%%%%%%%%%%%%%%%




\startcomponent psalme-06


\project douay-rheims


%%% 1158
%%% o-1050
\startChapter[
  title={Psalme 6}
  ]

\PSummary{Dauids
\MNote{A pathetical praier of a ſinner & the firſt penitential
  Pſalme. The 7.~key.}
earneſt and hartie praier after he had grieuouſly ſinned. 5.~Which being
grounded in filial, not ſeruile feare, 9.~concludeth with aſſured hope,
and confidence in Gods mercie.}

\PTitle{Vnto
\SNote{This Pſalme perteyneth alſo to penitence in the new Teſtament.}
the end in ſongs, the Pſalme of Dauid
\LNote{For the octaue.}{Literally
\MNote{The octaue
\Fix{hgnifieth}{ſignifieth}{obvious typo, fixed in other}
the world to come.}
it ſemeth that the Pſalmes which haue \Emph{For the octaue}, in their
titles, were to be ſong on an inſtrument of eight ſtringes. So the
Caldee paraphaſis tranſlateth. \L{In citheris octo chordarum}: \Emph{in
Citherus of eight ſtringes}. But prophetically S.~Auguſtin, & others
expound it, to perteine to the Reſurrection in the end of this world. So
Dauid, and al penitent ſinners bewaile their ſinnes, and do penance in
this life, for the octaue, that is for the world to come.}
for the octaue.}

\VV Lord,
\CNote{\XRef{Pſal.~37.}}
\SNote{Condemne me not eternally.}
rebuke me not in
\LNote{In thy furie, nor in thy vvrath.}{By
\MNote{Hel for ſome ſinners.}
\Emph{furie} is ſignified diuine iuſtice, irreuocably condemning the
wicked to eternal damnation: by
\MNote{Purgatorie for others.}
\Emph{vvrath}, Gods fatherlie chaſticement correcting ſinners, whom he
ſaueth. VVherupon S.~Gregorie teacheth, that the faithful ſoule not only
feareth furie, but alſo wrath: becauſe 
\CNote{\Cite{in Pſal.~37.}}
\Emph{after the death of the flesh, ſome are deputed to eternal
torments, ſome paſſe to life through the fire of purgation.} VVhich
doctrine the 
ſame holy father confirmeth, by the iudgement of S.~Auguſtin more
ancient. VVho likewiſe affirmeth,
\CNote{\XRef{1.~Cor.~3.}}
that al thoſe which haue not laide Chriſt their fundation, are rebuked
in furie, becauſe they are tormented in eternal fire: and thoſe which
vpon right fundation (of true faith in Chriſt) haue \Emph{built vvood,
hay, ſtubble}, are chaſtiſed in wrath, becauſe they are brought to reſt
of beatitude, but purged by fire. Let therfore the faithful ſoule
(conſidering what she hath donne, and contemplating what she shal
receiue) ſay: \Emph{Lord rebuke me not, in thy furie: nor chaſtice me in
thy vvrath}. As if ſhe ſaid more plainly: This only with my whole
intention of hart, I craue, this inceſſantly with al my deſires I
couete, that in the dreadful iudgement, thou neither ſtrike me with the
reprobate, nor afflict me with thoſe, that shal be purged in burning
flames. Thus S.~Gregorie,
\Cite{in 1.~Pſalm. penitent. v.~1.}}
thy furie; nor
\SNote{Spare me alſo for part of the temporal paine, which I deſerue.}
chaſtiſe me in
%%% !!! This LNote is part of the above.
%%% \LNote{}{}
thy wrath.

\V Haue mercie on me Lord, becauſe I am weake:
\SNote{Geue me the medicine of grace.}
heale me Lord, becauſe al
\SNote{My ſorow hath inwardly pearced me euen to the bones.}
my bones be trubled.

\V And my ſoule is
\SNote{With feare of thy iuſt wrath.}
trubled exceedingly: but thou Lord
\SNote{Leaueſt thou me in this calamitie?}
how long?

\V
\SNote{Shew againe thy fauorable contenance.}
Turne thee Ô Lord, and
\SNote{From this fearful affliction.}
deliuer my ſoule: ſaue me for
\SNote{Though my ſinnes haue deſerued the contrary yet shew thy mercy.}
thy mercie.

\V Becauſe there is not in
\SNote{This life is the time of repentance, after death no conuerſion.}
death, that is mindful of thee: and in
\SNote{In hel nothing but blaſphemie.}
hel who shal confeſſe to thee?

%%% o-1051
\V I
\SNote{I haue in part lamẽted.}
haue labored in my ſighing, I
\SNote{I wil adde more ſorow & penance.}
wil euerie night washe my bed; I wil
\SNote{I wil perſiſt in my penance, til I be thoroughly waterd with thy
grace.}
water my couche with my teares.

%%% 1159
\V My
\SNote{Myn eyes are dimme with weeping, for feare of thy iuſt
iudgement.}
eye is trubled for furie: I haue waxen
\SNote{My heares are gray with ſorrow,}
old
\SNote{wherat myn enemies reioyce.}
among al myne enemies.

\V
\CNote{\XRef{Mat.~7.}
&
\XRef{25.}
\XRef{Luc.~15.}}
\SNote{After due ſorow the true penitent hath confidence in God, againſt
his enimies.}
Depart from me al ye that worke iniquitie: becauſe our Lord
\SNote{VVil moſt certainly accept of true repentance.}
hath heard the voice of my weeping.

\V Our Lord hath heard my petition, our Lord hath receiued my prayer.

\V Let al myne enemies be
\SNote{Theſe are not imprecations, but threatninges, that the wicked may
amend, or els predictions if they perſiſt in ſinne.}
ashamed, & very ſore trubled: let them be conuerted and ashamed very quicly.


\stopChapter


\stopcomponent


%%% Local Variables:
%%% mode: TeX
%%% eval: (long-s-mode)
%%% eval: (set-input-method "TeX")
%%% fill-column: 72
%%% eval: (auto-fill-mode)
%%% coding: utf-8-unix
%%% End:
