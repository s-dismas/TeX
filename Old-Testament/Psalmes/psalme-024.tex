%%%%%%%%%%%%%%%%%%%%%%%%%%%%%%%%%%%%%%%%%%%%%%%%%%%%%%%%%%%%%%%%%
%%%%
%%%% The (original) Douay Rheims Bible 
%%%%
%%%% Old Testament
%%%% Psalmes
%%%% Psalme 024
%%%%
%%%%%%%%%%%%%%%%%%%%%%%%%%%%%%%%%%%%%%%%%%%%%%%%%%%%%%%%%%%%%%%%%




\startcomponent psalme-024


\project douay-rheims


%%% 1187
%%% o-1080
\startChapter[
  title={Psalme 24}
  ]

\PSummary{A
\MNote{A prayer of the faithful.

The 7.~key.}
general prayer of the faithful againſt al enemies, 4.~with
  deſire to be directed in the way of godlines, 7.~and to be pardoned
  for ſinnes paſt, 9.~acknowledging Gods meeknes, 17.~our weaknes,
  neceſſitie of helpe, and hope in God: 22.~concludeth with prayer for
  the whole Church.}

\PTitle{Vnto
\SNote{This Pſalme perteyneth more properly to the new Teſtament. And
is artificially compoſed: the verſes beginnĩg with diſtinct letters in
order of the Hebrew Alphabet, to the laſt verſe.}
the end, the Pſalme of Dauid.}

%%% 1188
\NV To thee ô Lord I haue lifted vp
\SNote{My mind, to be attẽtiue.}
my ſoule:
\V my God in thee is my confidence, let me
\SNote{Not be fruſtrate of my petition.}
not be ashamed:

\V Neither let mine enemies ſcorne me: for al
\SNote{That patiently expect the time when God wil aſſiſt.}
that expect thee, shal not be confounded.

\V
\SNote{This maner of praying is frequent in the Pſalmes, ſignifying as a
prophecie, that ſo it wil come to paſſe and the conformitie of the iuſt
to Gods iuſtice.}
Let al be confounded that do vniuſt thinges in vayne. Lord shew me thy
wayes: and teach me thy pathes.

\V Direct me
\SNote{In true faith and religion.}
in thy truth, and teach me: becauſe thou art God my Sauiour, and thee
haue I expected
\SNote{Al our life we muſt deſire more and more knowlege of true
doctrin.}
al the day.

\V Remember ô Lord thy commiſerations, and thy mercies: that are from
the beginning of the world.

\V The ſinnes
\SNote{From the firſt vſe of reaſon, at which time manie are careles,}
of my youth, and
\SNote{& negligent to lerne how to ſerue God.}
my ignorances doe not remember.

According to thy mercie remember thou me: for thy goodneſſe ô Lord.

\V Our Lord is
\SNote{As God is ſweete in geuing good motions:}
ſweete, and
\SNote{ſo he is ſeuere to them that reſiſt his grace.}
righteous: for this cauſe he wil geue a law to them that ſinne in the
way.

\V He wil direct the milde in iudgement: he wil teach the meeke his
wayes.

\V Al the wayes of our Lord, be
\SNote{God mercifully preuenteth with his grace:}
mercie and
\SNote{and iuſtly rewardeth good workes.}
truth, to them that ſeeke after his
\SNote{Gods law is his couenant with man:}
teſtament and his
\SNote{and teſtimonie of his wil.}
teſtimonies.

\V For thy name ô Lord thou wilt be propitious to my ſinne: for
\SNote{Sinne in reſpect of auerſion from God is great & nedeth his
grace.}
it is much.

%%% o-1081
\V Who is the man that
\SNote{He that feareth God which is the beginning of wiſdome, receiueth
fiue ſpiritual commodities here mentioned:}
feareth our Lord? he
\SNote{1.~God inſtructeth him by his law:}
appoynteth him a law in the way, that he hath choſen.

\V His ſoule
\SNote{2.~beſtoweth al neceſſaries vpon him:}
shal abide in good things: and
\SNote{3.~others shal imitate his good example:}
his ſeede shal inherite the land.

\V Our Lord is
\SNote{4.~God wil protect him:}
a firmament to them that feare him: &
\SNote{5.~according to Gods couenant he shal enioy the manifeſt ſight of
God for his eternal reward.}
his teſtament that it may be made manifeſt to them.

%%% 1189
\V Myne eies are alwayes to our Lord: becauſe he wil plucke my feete out
of the ſnare.

\V Haue reſpect to me, and haue mercie on me: becauſe I am
\SNote{Mans weaknes without Gods helpe.}
alone and poore.

\V The tribulations of my hart are multiplied: deliuer me from
\SNote{Tribulatiõs can not be auoided, but muſt neceſſarily be ſuffered:
therfore ô God geue vs grace to paſſe through them without ſinne.}
my neceſſities.

\V See my
\SNote{Myn affliction.}
humiliation, and my labour: and
\SNote{Take away the cauſe and affliction wil be mitigated.}
forgeue al my ſinnes.

\V 
\CNote{\XRef{Ioan.~15.}}
Behold mine enemies, becauſe they are multiplied, and with
\SNote{Wicked men of hatred do endeuour to draw others into ſinne.}
vniuſt hatred hated me.

\V Keepe my ſoule, and deliuer me: I shal
\SNote{Thoſe that hope in God shal neuer be confounded.}
not be ashamed, becauſe I hoped in thee.

\V The innocent and righteous haue cleaued to me: becauſe I expected
thee.

\V
\SNote{Al the letters of the Alphebet being complete in this Pſalme,
this laſt verſe beginneth with \HH{Pere}, \Emph{Redeeme}, praying God to
redeme and deliuer Iſrael, that is, the whole Church from tribulations.}
Deliuer Iſrael ô God, out of al his tribulations.


\stopChapter


\stopcomponent


%%% Local Variables:
%%% mode: TeX
%%% eval: (long-s-mode)
%%% eval: (set-input-method "TeX")
%%% fill-column: 72
%%% eval: (auto-fill-mode)
%%% coding: utf-8-unix
%%% End:
