%%%%%%%%%%%%%%%%%%%%%%%%%%%%%%%%%%%%%%%%%%%%%%%%%%%%%%%%%%%%%%%%%
%%%%
%%%% The (original) Douay Rheims Bible 
%%%%
%%%% Old Testament
%%%% Psalmes
%%%% Psalme 037
%%%%
%%%%%%%%%%%%%%%%%%%%%%%%%%%%%%%%%%%%%%%%%%%%%%%%%%%%%%%%%%%%%%%%%




\startcomponent psalme-037


\project douay-rheims


%%% 1211
%%% o-1103
\startChapter[
  title={Psalme 37}
  ]

\PSummary{King
\MNote{The third penitential Pſalme.

The 7.~key.}
Dauid, or anie other penitent, earneſtly prayeth God to remitte his
ſinnes, and mitigate the paines which he acknowledgeth him ſelfe to haue
deſerued, 12.~lamenting the afflictions which he ſuffereth by ſuch as
ſometimes were his freindes, 14.~whoſe tentations he now reſiſteth,
truſting in
%%% 1212
God, reſigning himſelfe to Gods wil, confeſſing his owne iniquitie, and
humbly praying for Gods helpe.}

\PTitle{A Pſalme of Dauid, in
\SNote{In remembrance that by ſinne we loſt the reſt and peace, which
man had in the ſtate of innocencie; ſecondly we loſt the peace of
conſcience; thirdly the reſt and peace of eternal felicitie.}
recordation of the ſabbath.}

\VV Lord
\SNote{Condemne me not to eternal paine:}
rebuke me not in thy furie:
\SNote{nor puniſh me in purgatorie fire; but purge me ſo in this life,
that the purging fire be not needful. By which fire (ſaith S.~Auguſtin)
though ſome ſhal be ſaued (\L{grauiour ramen erti ille ignis, quam
quicquid poteſt home pati in hac vita}) yet that fire ſhal be more
greuous, then whatſoeuer a man can ſuffer in this
\Fix{life.}{llife.}{obvious typo, fixed in other}
S.~Gregory alſo expoundeth this ſame place, as if Dauid ſayd thus: I
know it wil come to paſſe, that after the end of this life, ſome ſhal be
cleanſed by purging flames, ſome ſhal be vnder the ſentence of eternal
damnation. But becauſe I do eſteme that tranſitorie fire more
intolerable then al preſent tribulation, I deſire not only not to be
rebuked in furie of eternal damnation, but alſo I feare to be purged in
the wrath of tranſitorie correption. Thou therfore ô Lord whom I ſerue
in my ſpirite, whom I know to be the Sauiour of al men, rebuke me not in
furie of perpetual damnation, nor
\Fix{chatiſe}{chaſtiſe}{obvious typo, fixed in other}
me in wrath of purging puniſhment. See
\XRef{Annotat. Pſal.~6.}}
nor chaſtiſe me in thy wrath.

%%% o-1104
\V Becauſe
\SNote{Afflictions of mind and bodie ſent by thy iuſt iudgement.}
thy arrowes are faſt ſticked in me: and thou haſt
\SNote{Thou haſt ſtrooke me with an heauie hand.}
faſtened thy hand vpon me.

\V There is
\SNote{I already feele in my fleſh, in al my bones, and powres great
affliction,}
no health in my flesh, at
\SNote{conſidering thy iuſtice,}
the face of thy wrath: my bones haue no peace at
\SNote{and my ſinnes.}
the face of my ſinnes.

\V Becauſe mine iniquities are gone
\SNote{Which are excedingly increaſed, almoſt ouerwhelming my ſpirite.}
ouer my head: and as a
\SNote{Sinnes no waſhed away be penance by their weight carie the ſoule
into more and more wickednes.}
heauie burden are become heauie vpon me.

\V My
\SNote{Stil corrupting thoſe partes which were whole before, as a
peſtered ſore that is not cured.}
ſcarres are putrified and corrupted, becauſe of my folishnes.

\V I am become miſerable, and am made
\SNote{Not able to goe ſtreight to do anie good worke, being guiltie of
greuous ſinne.}
crooked euen to the end: I went ſorowful al the day.

\V Becauſe
\SNote{Concupiſcence ſtriuing in me.}
my loynes are filled with illuſions: and there is no health in my flesh.

\V I am afflicted and am humbled excedingly: I
\SNote{From the ſorrow of my hart, my voice hath broken out into
clamour.}
rored for the groning of my hart.

%%% 1213
\V Lord,
\SNote{Ô God thou knoweſt my deſire, to be reſtored to thy fauour.}
before thee is al my deſire: and my groning is not hid from thee.

\V My hart is trubled, my ſtrength hath forſaken me: and the light of
mine eies, and the ſame is not with me.

\V My
\SNote{Thoſe that were my freindes and companions in ſinne are become
myn enemies, becauſe I forſake them:}
frendes, and my neighbores haue approched, & ſtood againſt me.

And they that were neere me, ſtood far of: \V and they did violence
which ſought my ſoule.

And they that ſought me euils,
\SNote{ſought by al meanes to intangle me againe.}
ſpake vanities: and meditated guiles al the day.

\V But I as
\SNote{I now renoũce al ſinne.}
one deafe did not heare: and as one dumme not opening
%%% o-1105
his mouth.

\V And I became as a man not hearing: and not hauing reproofes in his
mouth.

\V Becauſe
\SNote{I now relie vpon thee ô God.}
in thee ô Lord haue I hoped, thou wilt heare me ô Lord my God.

\V
\SNote{For this cauſe I am returned to thee and do pray that mine
enemies may not preuail againſt me.}
Becauſe I ſaid: Leſt ſometime mine enemies reioyce ouer me: and whiles
my feete are moued, they ſpeake great thinges vpon me.

\V Becauſe I
\SNote{I reſigne my ſelfe to thee.}
am readie for ſcourges: and my ſorow is in my ſight alwaies:

\V Becauſe
\SNote{Though thou knoweſt al yet with mouth confeſſion is made to
ſaluation,}
I wil declare my iniquitie: and I wil
\SNote{and I meditate of that which my ſinne hath deſerued.}
thinke for my ſinne.

\V But mine enemies liue, and are confirmed ouer me: and they are
multiplied that hate me vniuſtly.

\V They that repay euil thinges for good,
\SNote{One kind of detraction is in reueling ſecrete faultes, an other
in feaning and imputing falſe crimes, the third (here mentioned) in
calling vertue vice, as penance, hypochriſie.}
detracted from me: becauſe I folowed goodnes.

\V
\SNote{Graunt me Lord final perſeuerãce in thy grace, and ſeruice.}
Forſake me not ô Lord my God, depart not from me.

Attend vnto my help, ô Lord the God of my ſaluation.


\stopChapter


\stopcomponent


%%% Local Variables:
%%% mode: TeX
%%% eval: (long-s-mode)
%%% eval: (set-input-method "TeX")
%%% fill-column: 72
%%% eval: (auto-fill-mode)
%%% coding: utf-8-unix
%%% End:
