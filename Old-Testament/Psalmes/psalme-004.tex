%%%%%%%%%%%%%%%%%%%%%%%%%%%%%%%%%%%%%%%%%%%%%%%%%%%%%%%%%%%%%%%%%
%%%%
%%%% The (original) Douay Rheims Bible 
%%%%
%%%% Old Testament
%%%% Psalmes
%%%% Psalme 04
%%%%
%%%%%%%%%%%%%%%%%%%%%%%%%%%%%%%%%%%%%%%%%%%%%%%%%%%%%%%%%%%%%%%%%




\startcomponent psalme-04


\project douay-rheims


%%% 1154
%%% o-1046
\startChapter[
  title={Psalme 4}
  ]

\PSummary{The
\MNote{Confidence in God neceſſary. The 7.~key.}
holie prophet teacheth, by his owne example, to flee to
  God in al tribulation. 3.~That other refugies are inſufficient. 9.~And
  Gods helpe moſt aſſured.}

%%% 1155
%%% o-1047
\PTitle{Vnto
\LNote{Vnto the end.}{The
\MNote{The ſignification of this phraſe, \Emph{To the end} in the titles
  of Pſalmes.}
Hebrew word \HH{Lamnatſea}, ſignifieth \Emph{to him that ouer cometh}. And
ſo the Hebrewes interprete, that the Pſalmes, which haue this word in
their titles, were directed either to him, that excelled in ſkil of
muſike;
\CNote{\XRef{1.~Par.~15.}}
or had authoritie ouer other muſitians: or to him, whoſe office
was to ſing victories and triumphes. But the Latin, according to the
Greeke, hath \L{In finem}, \Emph{Vnto the end}, which (moſt commonly
ſignifying perpetuitie, or continuance vnto the end of anie thing) in
the titles of the Pſalmes rather ſignifieth, that the matter conteyned
in the Pſalme, perteineth to future times, or perſons; eſpecially to the
new Teſtament. And ſo S.~Auguſtin expoũdeth it here of \Emph{Chriſt}, who
is
\CNote{\XRef{Rom.~10.}}
\Emph{the end} (or perfection) \Emph{of the lavv}. Not that the
principal contentes belong to Chriſt, in his owne Perſon, but to his
myſtical bodie the Church, and faithful people, whom the Prophete here
teacheth to haue confidence in God, moderation in their affections, &
patience in tribulation, which is the ſeuenth key, propoſing his owne
example, & prophetically Chriſts. The ſame wherto Chriſt exhorteth,
ſaying:
\XRef{Ioan.~16. v.~vlt.}
\Emph{Haue confidence, I haue ouercome the vvorld.}  Signifying that his
ſeruantes, through his grace, may alſo ouercome it.}
the end, in
\SNote{In an inſtrument apt for verſes.}
ſongues, the Pſalme
\SNote{This Pſalme perteyneth to the beloued, ſignified by the word
Dauid.
\Cite{S.~Aug. li.~17. c.~14. ciuit.}
\Cite{S.~Beda in Pſal.}}
of Dauid.}

\VV VVhen
\SNote{VVhen Saul vniuſtly perſecuted iuſt Dauid, God heard his
prayers.}
I inuocated, the God of my iuſtice heard me: in
\SNote{being ſtraictly beſeeged.
\XRef{(1.~Reg.~23.~26.)}}
tribulation thou haſt enlarged to me,
\SNote{Likewiſe helpe me when ſoeuer I shal nede.}
Haue mercie on me, and heare my prayer.

\V Ye ſonnes of men how long are you of
\SNote{Why do you ſtil harden your hartes?}
heauie hart? why loue you
\SNote{honour, and tranſitorie glorie,}
vanitie, and ſeeke
\SNote{falſe and deceiptful riches?}
lying?

\V
\CNote{\XRef{Epheſ.~4.}}
And know ye that our Lord hath made his
\SNote{Euery godly ſoule.}
holie one
\SNote{Rich with vertues.}
meruelous:
\SNote{Euerie iuſt ſoule hath confidence in God, that he wil heare his
crie.}
our Lord wil heare me, when I shal crie to him.

\V Be ye
\SNote{Iuſt anger is good & neceſſarie agaĩſt ſinne.}
angrie, and
\SNote{But then is moſt nede to beware not to excede in paſſion. Haue
therfore a continual purpoſe neuer to ſinne.}
ſinne not: the thinges that you ſay in
\SNote{Euil cogitations.}
your hartes, in your
\SNote{Bewaile & repent before you ſleepe.}
chambers be ye ſorie for.

\V Sacrifice ye the
\LNote{Sacrifice of Iuſtice.}{Not
\MNote{Three ſpiritual ſacrifices neceſſarie.}
only external Sacrifice of diuers kindes, were neceſſarie in the law of
nature, and of Moyſes, and one moſt excellent and complement of al, in
the new Teſtament, but alſo ſpiritual ſacrifice was euer, and is
required, and that of three ſortes.
\MNote{Of penance.}
Firſt, Sacrifice of ſorow, and contrition for ſinnes.
\XRef{(Pſal.~50.)}
\Emph{An afflicted ſpirite is a ſacrifice to God.}
\MNote{Iuſtice.}
The ſecond is ſacrifice of Iuſtice, here mentioned.
\MNote{Praiſe.}
The third is Sacrifice of praiſe
\XRef{(Pſal.~49.)}
\Emph{Immolate to God the ſacrifice of praiſe.} Concerning the ſecond
propoſed in this place: He offereth ſacrifice of Iuſtice, that rendereth
to euery one that is due.
\MNote{VVhat is due to God.}
Firſt to God as our Creator, a reſignation of our ſelues, euen our
liues, at his diuine pleaſure; as to our Maſter, we muſt render faith
and beleefe, in al that he propoſeth; as to our Father, hope,
confidence, & reuerential feare; as to our Lord and King, payment of
tribute, that is obſeruation of his law and commandments; as to our
Captaine, the trauel of warfare in this life; as to our Phiſitian,
patience and toleration, when he cureth our woundes, by chaſticement for
ſinnes; as to our Spouſe, chaſtity of body and mind, fleeing al carnal
and ſpiritual fornication; as to our Freind, frequent conuerſation in al
actes of deuotion.
\MNote{To our ſelues.}
VVe owe to our ſelues, that ſeing we conſiſt of ſoule and bodie, we
keepe due ſubordination, that the ſoule and reaſon command, & the bodie,
and inferiour appetite obey: as the ſeruant muſt obey his maſter, and
the handmaide her miſtris.
\MNote{To our neighbour.}
VVe owe to our neighbour, loue from the hart, inſtruction alſo from the
mouth; and aſſiſtance by our helpe, according to his neceſſitie, and our
abilitie; yea though our neighbour be our enemie.
\MNote{To our enemies.}
But to other enimies contrary thinges are due.
\MNote{The world.}
To the world, contempt: becauſe the goodes of this world are ſmal, few,
ſhorte, vncertaine, deceiptful, not ſatisfying the mind, and mixed with
manie euils and dangers.
\MNote{The fleſh.}
To the fleſh we owe chaſticement, and daylie care, ſo to feede it, that
it ſerue the ſoule, & rebel not.
\MNote{The diuel.}
To the diuel we muſt render the ſhame, that cometh by ſinne, acknowledging
our faults, and al truthes, and ſo returne vpon him \Emph{al vanitie and
lying}, wherwith he allureth & ſeduceth.
\MNote{To ſinne.}
Finally to ſinne it ſelfe, we owe hate, and reuenge, becauſe it is the
only euil, that hurteth vs; and due puniſhment with zele of iuſtice,
becauſe it diſhonoreth God. He that thus offereth \Emph{ſacrifice of
iuſtice}, may iuſtly (as it foloweth in the Pſalme) hope (yet not in him
ſelfe but) in our Lord.
\MNote{Light of reaſon sheweth there is a God that rewardeth.}
And leſt anie ſhould pretend ignorance, ſaying:
\Emph{vvho sheweth} (or teacheth) \Emph{vs good thinges}, as though they
lacked inſtruction, the Prophete preuenteth this vaine excuſe, ſaying:
\Emph{The light of thy countenance Ô Lord} (the light of reaſon, which
is the image of God, wherto we are created like) \Emph{is ſigned vpon
vs}, fixed in our vnderſtanding, that we may ſee there is a God, that
ought to be ſerued, and that he wil revvard his ſeruantes.
\XRef{Heb.~11.}}
ſacrifice of
\SNote{Not only external but moſt eſpecially internal ſacrifice of
iuſtice, and obſeruation of Gods commandments is moſt neceſſarie.}
iuſtice, and hope in our Lord. Manie ſay:
\SNote{The ſolide rewardes promiſed by God?}
Who sheweth vs good thinges?

\V The
\SNote{Reaſon and grace are freely geuen to man, wherby he may know that
God wil reward the iuſt.
\XRef{Heb.~11. v.~6.}}
light of thy countenance Ô Lord is ſigned vpon vs, thou haſt geuen
\SNote{VVherin a iuſt man inwardly reioyceth.}
gladneſſe in my hart.

\V By the fruite of their
\SNote{For example and in figure of heauenlie rewardes, God gaue
temporal wealth in the old Teſtament.}
corne, and
%%% !!! This SNote is part of the above.
%%% \SNote{}
wine, and
%%% !!! This SNote is part of the above.
%%% \SNote{}
oile they are multiplied.

\V In
\SNote{In this confidẽce the iuſt may reſt contented.}
peace in the ſelfe ſame I wil ſleepe, and reſt:

\V Becauſe thou Lord haſt
\SNote{God ſo promiſeth euerie iuſt perſon in particular.}
ſingularly ſetled me in hope.


\stopChapter


\stopcomponent


%%% Local Variables:
%%% mode: TeX
%%% eval: (long-s-mode)
%%% eval: (set-input-method "TeX")
%%% fill-column: 72
%%% eval: (auto-fill-mode)
%%% coding: utf-8-unix
%%% End:
