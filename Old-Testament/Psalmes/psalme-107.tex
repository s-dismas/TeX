%%%%%%%%%%%%%%%%%%%%%%%%%%%%%%%%%%%%%%%%%%%%%%%%%%%%%%%%%%%%%%%%%
%%%%
%%%% The (original) Douay Rheims Bible 
%%%%
%%%% Old Testament
%%%% Psalmes
%%%% Psalme 107
%%%%
%%%%%%%%%%%%%%%%%%%%%%%%%%%%%%%%%%%%%%%%%%%%%%%%%%%%%%%%%%%%%%%%%


\startcomponent psalme-107


\project douay-rheims


%%% 1335
%%% o-1225
\startChapter[
  title={Psalme 107}
  ]

\PSummary{The
\MNote{Dauid ſingeth prayſes for benefites receiued.

The 8.~key.}
royal prophet promiſeth, 5.~and rendereth praiſes to God, 7.~for his
deliuerie from trubles, and aduancement in the kingdom, 13.~praying God
ſtil to helpe mans infirmitie.}

\PTitle{A Canticle
\SNote{This Pſalme was ſongue with inſtruments beginning the muſike, and
voices folowing.}
of Pſalme, to Dauid himſelfe.}

\VV My
%%% Both notes should go before 'My'
\CNote{\XRef{Pſal.~56. v.~8.}}
\SNote{The former part of this Pſalme to the 7.~verſe, is the ſame in
ſenſe, and almoſt in wordes, with the latter part of
\XRef{the~56. from the 8.~verſe.}}
hart is readie ô God, my hart is readie: I wil chaunte, and wil ſing in
my glorie.

\V Ariſe my glorie, ariſe pſalter, and harpe: I wil ariſe early.

\V I
\SNote{King Dauid ſubdued not only ſome partes of Chanaan, not ſubiect to
the Iewes before
\XRef{(2.~Reg.~5.}
\XRef{1.~Par.~11.)}
but alſo brought the Philiſtims, Moabites, Ammonites, Idumeans,
Amalechites, the kinges of Soba, Syria, and Emath, to pay tribute,
\XRef{2.~Reg.~8.}
\XRef{1.~Par.~18.}}
wil confeſſe to thee in peoples ô Lord: and I wil ſing to thee
\SNote{Yet al theſe victories and conqueſtes were but a figure of Chriſts powre
and dominion in al nations. And therfore, the reſt of this Pſalme, by
%%% !!! Cite?
S.~Auguſtin, and other fathers iudgement, was rather prophetically
vttered by Dauid, in the perſon of Chriſt, and more perfectly performed
by Chriſt in his Church, then hiſtorically auerred of Dauid himſelfe.}
in the Nations.

\V Becauſe thy mercie is great aboue the heauens: and thy truth euen to
the cloudes.

\V Be exalted aboue the heauens ô God, and thy glorie ouer al the earth:
\V
\CNote{\XRef{Pſal.~59. v.~7.}}
\SNote{The reſt of this pſalme is the ſame with the latter part of
\XRef{the~59. from the 7.~verſe.}}
that thy beloued may be deliuered.

Saue with thy righthand; and heare me: \V God ſpake in his holie:

%%% 1336
I wil reioyce, and wil diuide Sichem; and I wil meſure the vale of
tabernacles.

\V Galaad is mine, and Manaſſes is mine: and Ephraim the protection of
my head.

Iuda is my king: \V Moab the potte of my hope.

Vpon Idumea I wil extend my shoe: the ſtrangers are made my freindes.

\V Who wil conduct me into a fenſed citie? who wil conduct me into
Idumea?

\V Wilt not thou ô God, which haſt repelled vs, and wilt not thou goe
forth ô God in our hoaſtes?

\V Geue vs helpe out of tribulation: becauſe mans ſaluation is vayne.

%%% o-1226
\V In God we shal doe ſtrength: and he wil bring our enemies to
nothing.


\stopChapter


\stopcomponent


%%% Local Variables:
%%% mode: TeX
%%% eval: (long-s-mode)
%%% eval: (set-input-method "TeX")
%%% fill-column: 72
%%% eval: (auto-fill-mode)
%%% coding: utf-8-unix
%%% End:
