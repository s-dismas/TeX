%%%%%%%%%%%%%%%%%%%%%%%%%%%%%%%%%%%%%%%%%%%%%%%%%%%%%%%%%%%%%%%%%
%%%%
%%%% The (original) Douay Rheims Bible 
%%%%
%%%% Old Testament
%%%% Psalmes
%%%% Psalme 069
%%%%
%%%%%%%%%%%%%%%%%%%%%%%%%%%%%%%%%%%%%%%%%%%%%%%%%%%%%%%%%%%%%%%%%




\startcomponent psalme-069


\project douay-rheims


%%% 1263
%%% o-1154
\startChapter[
  title={Psalme 69}
  ]

\PSummary{An
\MNote{Dauids prayer in perſecutiõ.

The 8.~key.}
other prayer of Dauid, when he was perſecuted by Abſolom: made in a
Pſalme after his deliuerie.}

\PTitle{Vnto
\SNote{An apt prayer alſo for the afflicted in the nevv Teſtament,}
the end, a Pſalme of Dauid, in remembrance, that our
\SNote{from the danger of Abſolom,
\XRef{(2.~Reg.~18.)}
or from anie perſecutor.}
Lord ſaued him.}

\VV O God
\SNote{Al men at al times nede Gods helpe:}
intend vnto my helpe: Lord
\SNote{but moſt preſent nede, in preſent dangers.

The reſt of this Pſalme is conteyned in the 39.~Pſalme, from the
\XRef{15.~verſe.}
but there the whole Church prayeth for helpe, the world being almoſt
drowned in ſinnes; here Dauid, or other particular perſons, or peoples
pray in their ſeueral diſtreſſes.}
make haſt to helpe me.

%%% 1264
\V
\CNote{\XRef{Pſal.~39.}}
Let them be confounded, and be ashamed, that ſeeke my ſoule.

\V Let them be turned away backeward, and be ashamed that wil me euils.

Let them be turned away forthwith ashamed, that ſay to me: Wel, wel.

\V Let al that ſeeke thee reioyce, and be glad in thee, and let them ſay
alwayes: Our Lord be magnified: which loue thy ſaluation.

\V But I am needie and poore: ô God helpe me, thou art my helper, and
deliuerer: ô Lord be not ſlacke.


\stopChapter


\stopcomponent


%%% Local Variables:
%%% mode: TeX
%%% eval: (long-s-mode)
%%% eval: (set-input-method "TeX")
%%% fill-column: 72
%%% eval: (auto-fill-mode)
%%% coding: utf-8-unix
%%% End:
