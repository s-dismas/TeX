%%%%%%%%%%%%%%%%%%%%%%%%%%%%%%%%%%%%%%%%%%%%%%%%%%%%%%%%%%%%%%%%%
%%%%
%%%% The (original) Douay Rheims Bible 
%%%%
%%%% Old Testament
%%%% Numbers
%%%% Argument
%%%%
%%%%%%%%%%%%%%%%%%%%%%%%%%%%%%%%%%%%%%%%%%%%%%%%%%%%%%%%%%%%%%%%%




\startcomponent argument


\project douay-rheims


%%% 0340
%%% o-0306
\startArgument[
  title={\Sc{The Argvment of the Booke of Nvmeri.}},
  marking={The Argument of Nvmeri.}
  ]

%%% !!! There are a bounch of CNotes that I can not easily
%%% figure out where they go. So I am skipping them for now.
%%% They are not in other.
In this booke called \Emph{Numeri, are contained} (ſaith
\CNote{\Cite{Epiſt. ad Paulum.}}
S.~Hierom)
\MNote{Myſteries cõteined in theſe hiſtories.}
\Emph{the Myſteries of al Arithmetike}, or numbering, \Emph{of the
Prophecie of Balaam, and of the fourtie two Manſions} of the
Iſraelites, \Emph{in the deſert}. VVhich myſtical ſenſe the ſame great
Doctor, as alſo
\CNote{\Cite{qq. in Num.}}
S.~Auguſtin and other Fathers do gather of the literal, written by
Moyſes. VVho here proſecuteth the ſacred hiſtorie after Geneſis and
Exodus (Leuiticus alſo containing one moneth) from the ſecond moneth of
the ſecond yeare, after the deliuerie of the Iſraelites out of Ægypt, nere
39.~yeares, to the laſt of Moyſes life.
\MNote{The contents according to the letter.}
Firſt therfore he reporteth how al the men of twelue tribes, of the age
of twentie yeares and vpward, were numbered. Likewiſe the tribe of Leui
was numbered and imployed partly in prieſtly function, the reſt to
aſsiſt the prieſts. He deſcribeth alſo the order of marching and
encamping, the Leuites alwayes next and round about the Tabernacle:
the other twelue tribes in circuite of them on al ſides. He moreouer
recordeth certaine notable murmurings, tumults, ſchiſmes, and rebellions
with the euents therof, and miſerable endes of chief ſeducers. VVhoſe
great iniuries Moyſes mekely ſuſtained with ſingular patience, ſtil
executing his owne function with heroical fortitude. Among which, diuers
precepts and lawes are partly repeated, partly added, as wel concerning
Religion and Gods ſeruice, as godlie policie and ciuil gouernment of the
people, with chaſtiſment of offenders. How alſo their enemies endeuoured
to annoy them, Balac king of Moab procuring Balaam the ſorcerer, ſo much
as in him lay, to curſe them, but al in vaine.
\MNote{VVicked life draweth to Idolatrie.}
Yet by carnal fornication manie were drawen to ſpiritual. Both which
being punished God againe proſpered his people, in diuers encounters
and battailes againſt Infidels. Finally the promiſed Land of Chanaan on
both ſides Iordaine is deſcribed by limites, which they shal parte
amongſt them by lot, the Leuites mingled in euerie tribe, with their
appointed cities and commoditie for habitation, and the tithes, firſt
fruites, oblations and abundant prouiſion for their maintenance. Cities
alſo of refuge are deſigned for caſual manſlayers; and a law eſtablished
that al shal marie within their owne tribes, to auoide confuſion of
inheritances.
\MNote{Three partes of this booke.}
So this booke may be diuided into three partes. In the firſt the
principal and moſt perfect ſort of the people are numbered, and diſpoſed
in order according to diuers ſtates and offices, before they depart from
the deſert of Sinai. In the nine firſt chapters. Then are related
ſundrie thinges, which happened vnto them in the reſt of their iourney,
eſpecially manie and great impediments. Through al which God punishing
ſome, brought the reſidue to enioy the promiſed land. From the
10.~chap. to the end of the 33. Laſtly the countrie of Chanaan is againe
promiſed, with order ſo to poſſeſſe and enioy it, that euerie tribe may
haue and keepe their ſeueral partes. In the three laſt chapters.


\stopArgument


\stopcomponent


%%% Local Variables:
%%% mode: TeX
%%% eval: (long-s-mode)
%%% eval: (set-input-method "TeX")
%%% fill-column: 72
%%% eval: (auto-fill-mode)
%%% coding: utf-8-unix
%%% End:
