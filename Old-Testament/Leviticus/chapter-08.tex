%%%%%%%%%%%%%%%%%%%%%%%%%%%%%%%%%%%%%%%%%%%%%%%%%%%%%%%%%%%%%%%%%
%%%%
%%%% The (original) Douay Rheims Bible 
%%%%
%%%% Old Testament
%%%% Leviticus
%%%% Chapter 08
%%%%
%%%%%%%%%%%%%%%%%%%%%%%%%%%%%%%%%%%%%%%%%%%%%%%%%%%%%%%%%%%%%%%%%




\startcomponent chapter-08


\project douay-rheims


%%% 0294
%%% o-0266
\startChapter[
  title={Chapter 8}
  ]

\Summary{Moyſes conſecrateth Aaron high Prieſt, 13.~and his ſonnes
  Prieſts, 33.~continuing in the tabernacle ſeuen dayes and nights.}

And
\MNote{The ſecond part.

Of conſecrating Prieſts, and their veſtmẽts: with puniſhment of ſome
that tranſgreſſed.}
Our Lord ſpake to Moyſes, ſaying: \V Take Aaron with his ſonnes, their
veſtimentes, and the oyle of vnction, a calfe for ſinne, two Rammes, a
basket with azymes, \V and thou shalt gather al the aſſemblie to the
dore of the tabernacle. \V And Moyſes did as our Lord had
commaunded. And al the multitude being gathered before the
%%% o-0267
dore of the tabernacle, \V he ſaid: This is the word, that our Lord hath
commanded to be done. \V And immediatly he
\LNote{Offered Aaron.}{By
\MNote{Particular calling and conſecration neceſſarie to prieſtlie
offices, & authoritie in ſpiritual cauſes.}
this maner of \Emph{taking, offering}, and \Emph{conſecrating} Aaron
Hiegh Prieſt, S.~Paul ſheweth that none may chalenge to them ſelues, nor
preſume to exerciſe prieſtlie offices, or anie authoritie in ſpiritual
cauſes, but ſuch as be orderly called therto. Yea that Chriſt him ſelf
would not haue exerciſed this function, but that he was alſo called of
God vnto it, ſaying: Euerie Hiegh Prieſt taken from among men, is
appointed for men,
\CNote{\XRef{Heb.~5.}}
\Emph{in thoſe thinges that pertaine to God}. Neither doth anie man take
the honour to himſelf, but he that is called of God, as Aaron. So Chriſt
alſo did not glorifie himſelf, that he might be made a Hiegh Prieſt: but
he that ſpake to him:
\CNote{\XRef{Pſal.~109.}}
\Emph{Thou art a Prieſt for euer, according to the order of
Melchiſedech.} Aarons ſonnes were alſo called, but to lower offices,
dignitie, and authoritie.
\MNote{Ordering of Prieſtes was a Sacrament in the law of Moyſes.}
And both he and they were ordained and conſecrated by a peculiar
Sacrament, to wit, by certaine determinate external ceremonies and
rites, ſignifying grace geuen them by God, for the due performing of
their function. For firſt they were taken from the common ſtate of men,
wherby is deſigned their ordinarie vocation; then purified by certaine
waſhings and ſacrifice for ſinne, ſignifying ſpecial puritie required in
them, afterwardes inueſted with holie and precious garmentes, which
ſignified their ſacred function, and great dignitie, excelling al
temporal dominion and principalitie; finally conſecrated in ſolemne
maner with holie ointment, and bloud of pacifique ſacrifice offered for
this purpoſe; other ſacrifice of holocauſt alſo offered in the ſame
ſolemnitie.}
offered Aaron & his ſonnes: and when he had
\SNote{VVaſhing ſignified puritie required in Prieſts.}
waſhed them, \V
\LNote{Reueſted.}{The
\MNote{Seuẽ precious veſtments for the high Prieſt ſignifiing:}
hiegh Prieſt had ſeuen ſpecial ornaments in his veſture.
\MNote{1.~Puritie.}
Firſt, a ſtraict \Emph{linnen vvhite garment}; ſignifying puritie of
life moſt ſpecially required in Prieſtes.
\MNote{2.~Diſcretion.}
Secondly a girdle, or \Emph{Bavvdrike}, of twiſted ſilke and gold,
embrodered worke, in coloures yelow, blew, purple, and ſcarlet;
ſignifying diſcrete moderation of his actes, to the ſpiritual profite of
al ſortes of people.
\MNote{3.~Good works of edification.}
Thirdly a \Emph{Tunike}, or long robe downe to the foote, of hyacinth,
or blew ſilke, at the skirt therof like pomegranates wrought of twiſted
ſilke, blew, purple and ſcarlet, and litle belles of pureſt (yelow) gold
interpoſed one by the other rownd about, of ech ſorte ſeuentie two;
\CNote{\Cite{S.~Hierom. Epiſt. ad Fabiol.}}
ſignifying heauenlie conuerſation vpon earth, alſo vnion and concord in
faith and maners, with edification by good workes.
\MNote{4.~Toleration of others infirmities.}
Fourthly, an ornament vpon his ſhoulders, called an \Emph{Ephod}, of gold
and twiſted ſilke, embrodered of the former coloures, reaching before
to the girdle, with two precious Onyx ſtones cloſed in gold, one hauing
engrauen ſix names of the tribes of Iſrael, ſet on one ſhoulder, the
other hauing the other ſix names on the other shoulder; for a
remembrance that he muſt ſupporte, and meekly beare the infirmities of
the people.
\MNote{5.~Knowledge and ſinceritie.}
Fiftly, a breaſt plate called \Emph{Rationale}, of the ſame precious
matter, the meaſure of a palme, foure ſquare, embrodered with the ſame
foure coloures, with foure rewes of twelue precious ſtones, and therin
engrauen the names of the twelue tribes. Beſides which were engrauen
alſo \HH{\Sc{Vrim}} and \HH{\Sc{Thvmmim}}, \Emph{Illuminations}
and \Emph{Perfections}, or \Emph{Doctrin} and \Emph{Veritie}, becauſe
the hiegh Prieſt muſt haue knowledge of the truth, and ſincere
intention. Likwiſe in the Ephod and Rationale were ringes, hookes, and
chaines of pureſt gold, to ioyne them faſt together. Al ſignifying the
perpetual ſolicitude and care which he ought to haue in his hart, to
know and teach the truth, that the people may truly ſerue God, to his
honour and their owne ſaluation.
\MNote{6.~Intentiõ directed to God.}
Sixtly, a \Emph{Mitre} of twiſted ſilke, with
little crownes embrodered worke, ſet on his head, to ſignifie that he
muſt direct al his actions to Gods glorie, that ſitteth aboue al.
\MNote{7.~Contemplation of God & his workes.}
Seuenthly, \Emph{A plate of ſacred veneration}, made of the fineſt gold, with
the moſt holie name of God engrauen, ſet on his forhead; to put him ſtil
in remembrance to contemplate God and his workes.}
he
\SNote{Precious veſtiments their dignitie: and holie oile their
authoritie.}
reueſted the high prieſt, with the ſtrait linnen garment, girding him
with a bawdrike, and reueſting him with the tunike of hyacinth, and ouer
it he put the Ephod, \V which he ſtraitening with the girdle, fitted it
to the Rationale, wherin was
\SNote{VVhen the high Prieſt at anie time put the Ephod to the
Rationale, God gaue anſwers to his demandes, in matters
of \Emph{doctrine} and \Emph{veritie} which king Dauid willed Abiathar
to doe.
\XRef{1.~Reg.~13. v.~9.}
Neuer could anie woman weue \Emph{doctrin} & \Emph{veritie}, but
diuine \Emph{vviſdom} did make ſuch garmẽts.
\Cite{S.~Cyril. lib.~6. in Leuit.}}
Doctrine and Veritie, \V with the mitre alſo he couered his head: and
vpon it, againſt the forehead, he put the plate of gold conſecrated in
ſanctification,
%%% 0295
as our Lord had commanded him. \V He
%%% !!! This LNote and the next are out of order in the notes at the end
%%% of the chapter.
\LNote{Tooke oile.}{A third thing that Moyſes was bid to take, beſides
the men and the veſtiments, was the \Emph{holie oile of vnction}, which
he poured only vpon the
\MNote{Aaron annointed high Prieſt.}
hiegh Prieſts head, not on other Prieſts; to ſignifie that powre
deſcended from him to the reſt. But both he and
\MNote{His ſonnes alſo conſecrated.}
they, and their holie veſtiments were ſprinkled with this oile, and with
bloud taken from the altar; their right eares alſo were touched with the
bloud of a ramme, ſacrificed, and their right thumbes, and great toes of
their right handes, and feete; to ſignifie prompt obedience, and right
intention, in offering ſacrifice, according to Gods ordinance, and not
after the maner of infidels, or humane inuention, nor to anie ſiniſter
intent or purpoſe.}
tooke alſo the oyle of vnction, wherwith he anoynted the tabernacle,
with al the furniture therof. \V And ſanctifying them, and hauing
ſprinckled the altar ſeuen times, he anoynted it, and al the veſſel
therof, and the lauer with the foote therof he ſanctified with the
oyle. \V The which pouring vpon Aarons head, he anoynted, and
conſecrated him: \V 
%%% !!! Not marked in text
\LNote{His ſonnes.}{The
\MNote{Other Prieſtes had alſo three ornamentes.}
other Prieſts had three ſpecial ornaments: a \Emph{Linnen vvhite
garment}, a \Emph{Bavvdrike}, and a \Emph{Mitre}, for glorie and bewtie;
to ſignifie the qualities aboue mentioned, \Emph{puritie, diſcretion},
and \Emph{direct intention} alſo required in them.}
his ſonnes alſo after he had offered them, he
reueſted with linnen tunikes, and girded them with bawdrikes, and put
mitres on them, as our Lord had commanded. \V He
\LNote{He offered the calfe.}{Other
\MNote{Al three kindes of ſacrifice offered at the conſecration of
Prieſts.}
thinges which Moyſes was here commanded to take, at the conſecration of
Prieſts, were a calfe, to be offered in ſacrifice for ſinne; two rammes,
the one in holocauſt, the other in pacifique ſacrifice, for the
conſecration of Prieſts; and a basket of vnleauened bread, to be offered
with the two rammes. Al for the greater ſolemnitie of this Sacrament of
Orders.
\CNote{\XRef{Num.~3.}}
\MNote{\Fix{Prieſtood}{Prieſthood}{obvious typo, fixed in other}
and Law changed together.}
By which Aaron and his ſonnes were made the lawful and ordinarie
Prieſts of the law newly deliuered by Moyſes. And ſo Prieſthood was
changed from the firſt borne of euerie familie, and eſtabliſhed only in
Aaron and his ſonnes, and their iſſue male, to be in like ſorte
conſecrated. And the reſt of the Leuites to aſſiſt them.
\MNote{The Sacramẽt of holie Orders prefigured, and the new Law.}
By this alſo
was prefigured the Sacrament of holie Orders in the Church of Chriſt,
with an other change of Prieſthood from the familie & order of Aaron, to
Prieſtes of the new Teſtament, of what familie or nation ſoeuer. And
withal an other change of the law.
\CNote{\XRef{Heb.~7.}}
\Emph{For the Prieſthood being tranſlated, it is neceſſarie} (ſaith
S.~Paul) \Emph{that a tranſlation of the Lavv be alſo made.} And this
Sacrament in dede geueth grace (as by the other it was only ſignified) to
thoſe that are rightly ordered. As the ſame Apoſtle teſtifieth, willing
Timothie to \Emph{reſuſcitate the grace geuen him by impoſition of
handes.}
\XRef{2.~Timot.~1.}
\Cite{S.~Ambroſe in 1.~Timot.~4.}
\Cite{S.~Auguſt. lib. de bono coniugali c.~24.}
&
\Cite{lib.~2. contra Epiſti. Parmen.}
\Cite{Theodoret. q.~48. in lib. Num.}}
offered alſo the calfe for ſinne: and when Aaron and his ſonnes had put
their handes vpon the head therof, \V he did immolate it: drawing the
bloud, and dipping his finger, touched the hornes of the altar round
about. Which being expiated, and ſanctified, he poured the reſt of the
bloud at the botome therof. \V But the fatte that was vpon the
entralles, and the caule of the liuer, and the two little kidneys, with
their little tallow he burnt vpon the altar: \V the calfe with the
skinne, and the fleſh, and the dung, he burnt without the campe, as our
Lord had commanded. \V He offered alſo a ramme for an holocauſt: vpon
the head wherof when Aaron and his ſonnes had put their handes, \V he
did immolate it, and poured the bloud therof in the circuite of the
altar. \V And cutting the ramme it ſelfe into peeces, the head therof,
and the ioyntes, and the fatte he burnt with fire, \V hauing firſt
waſhed the entralles, and the feete, and the whole ramme together he
burnt vpon the altar, becauſe it was an holocauſt of moſt ſwete odour to
our Lord, as he had commanded him. \V He offered alſo the ſecond ramme,
for the conſecration of prieſts: and Aaron, and his ſonnes did putte
their handes vpon the head therof: \V which when
\SNote{As wel by the function of cõſecrating Prieſts, as of offering
Sacrifice it appeareth that Moyſes was a Prieſt. Yea the chiefe and
hiegheſt Prieſt (ſaieth S.~Auguſtin) for his more excellent miniſterie
and extraordinarie calling: Aaron was hiegh prieſt for his Pontifical
inueſture, and ordinarie vocation, which ſhould continew in his
ſucceſſors.
\Cite{q.~23. in Leuit.}}
Moyſes had immolated, taking of the bloud therof, he touched the tippe
of Aarons right eare, and the thumbe of his right hand, in like maner
alſo of his foote. \V He offered alſo the ſonnes of Aaron: and when of
the bloud of the ramme, being immolated, he had touched the tippe of the
right eare of euerie one, and the thumbes of the right hand and foote,
the reſt he poured on the altar, round about: \V but the fatte, and the
rump, and al the fatte that couereth the entralles, and the caule of the
liuer, and the two kidneies with their fatte, & with the right shoulder,
he ſeperated. \V And taking out of the basket
%%% 0296
of azymes, which was before our Lord, a loafe without leauen, and a
manchet tempered with oile, and a wafer he put them vpon the fatte, and
the right shoulder, \V deliuering
%%% o-0268
al to Aaron, and to his ſonnes. Who hauing lifted them vp before our
Lord, \V he tooke them againe of their handes, and burnt them vpon the
altar of holocauſt, becauſe it was the oblation of conſecration, for a
ſwete odoure, of the ſacrifice to our Lord. \V And he tooke of the ramme
of conſecration, the breſt for his portion, eleuating it before our
Lord, as our Lord had commanded him. \V And taking the oyntment, and the
bloud that was vpon the altar, he ſprinckled it vpon Aaron, and his
veſtiments, & vpon his ſonnes and their veſtiments. \V And when he had
ſanctified them in their veſtiments, he commanded them, ſaying: Boile
the flesh before the dore of the tabernacle, and there eate it. Eate ye
alſo the loaues of conſecration, that are laid in the basket, as our
Lord commanded me, ſaying: Aaron and his ſonnes shal eate them: \V and
whatſoeuer shal be left of the flesh, and the loaues, fire shal
conſume. \V Out of the dore alſo of the tabernacle you shal not goe
forth ſeuen daies, vntil the day wherein the time of your conſecration
shal be expired, for in ſeuen dayes the conſecration is finished: \V as
at this preſent it hath bene done, that the rite of the ſacrifice might
be accomplished. \V Day & night shal you tarie in the tabernacle
obſeruing the watches of our Lord, leſt you die: for ſo it hath bene
commanded me. \V And Aaron, and his ſonnes did al thinges which our Lord
ſpake by the hand of Moyſes.


\stopChapter


\stopcomponent


%%% Local Variables:
%%% mode: TeX
%%% eval: (long-s-mode)
%%% eval: (set-input-method "TeX")
%%% fill-column: 72
%%% eval: (auto-fill-mode)
%%% coding: utf-8-unix
%%% End:
