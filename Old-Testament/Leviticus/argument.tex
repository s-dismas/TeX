%%%%%%%%%%%%%%%%%%%%%%%%%%%%%%%%%%%%%%%%%%%%%%%%%%%%%%%%%%%%%%%%%
%%%%
%%%% The (original) Douay Rheims Bible 
%%%%
%%%% Old Testament
%%%% Leviticus
%%%% Argument
%%%%
%%%%%%%%%%%%%%%%%%%%%%%%%%%%%%%%%%%%%%%%%%%%%%%%%%%%%%%%%%%%%%%%%




\startcomponent argument


\project douay-rheims


%%% 0281
%%% o-0254
\startArgument[
  title={\Sc{The Argvment of Leviticvs.}},
  marking={The Argument of Leviticus.}
  ]

VVhen
\CNote{\XRef{Exod.~vlt.}}
\MNote{So ſoone as the Tabernacle was erected God declared the offices
of the Leuites, written in this booke: wherof it is called Leuiticus.}
the Tabernacle was erected, nere to \Emph{Mount Sinai}, the \Emph{firſt
day of the ſecond yeare}, after the children of Iſrael parted from
Ægypt, and was ſo \Emph{repleniſhed} with Gods \Emph{Maieſtie}, that
none, no \Emph{not Moyſes} himſelf \Emph{could enter in}, our Lord
ſpeaking from thence, called Moyſes, and declared to him the offices of
the Leuites;
\CNote{\XRef{Nu.~1.}}
whom only, and no others, he deputed for the adminiſtration, and charge
of ſacred things: wherof this booke (wherin they are written) is called
Leuiticus. \Emph{In which} ſaith
\CNote{\Cite{Epiſt. ad Paulinum.}}
S.~Hierom, \Emph{al and euerie Sacrifice, yea almoſt euerie ſillable,
and Aarons veſtments, and the whole Leuical order breath forth heauenlie
ſacraments}, or myſteries.
\CNote{\XRef{Leuit.~1.}}
\MNote{The contents of this booke.}
For firſt God here preſcribeth what ſacrifices he wil haue, in what
manner, and to what purpoſes.
\CNote{\XRef{8.}}
Then what partes and qualities he requireth in Prieſts; how they shal be
veſted and conſecrated, ſeuerly punishing ſome that tranſgreſſed:
\CNote{\XRef{11.}}
with commandment neither to offer in ſacrifice, nor to eate things
reputed vncleane,
\CNote{\XRef{12.}}
and the maner of purifying ſuch things, and perſons, as by diuers
occaſions were polluted:
\CNote{\XRef{18.}}
Interpoſing alſo ſome moral, and iudicial precepts; 
\CNote{\XRef{23.}}
appointeth certaine ſolemne feaſtes, times of reſt, and Iubilie yeare.
\CNote{\XRef{26.}}
Finally promiſeth rewardes, and threatneth punishments to thoſe that
kepe or breake his commandments:
\CNote{\XRef{27.}}
with particular admonition touching vowes and tithes.
\MNote{Diuided into fiue parts.}
So this booke may be diuided into fiue ſpecial partes. The firſt, of
diuers ſortes of Sacrifices: in the ſeuen firſt chapters. The ſecond, of
conſecrating Prieſts, and their veſtments, with punishment for offering
ſtrange fire, in the three next chapters. The third, of diſtinction
betwen cleane and vncleane, with the maner of purifying certaine legal
vncleanes, and other precepts moral and iudicial, from the 11.~chap. to
the~23. The fourth, of feaſts, times of reſt, and Iubilie with
priuiledges, rewardes, and punishments, from the 23.~chap. to
the~27. The fifth, of vowes, and tithes, in the laſt chapter.


\stopArgument


\stopcomponent


%%% Local Variables:
%%% mode: TeX
%%% eval: (long-s-mode)
%%% eval: (set-input-method "TeX")
%%% fill-column: 72
%%% eval: (auto-fill-mode)
%%% coding: utf-8-unix
%%% End:
