%%%%%%%%%%%%%%%%%%%%%%%%%%%%%%%%%%%%%%%%%%%%%%%%%%%%%%%%%%%%%%%%%
%%%%
%%%% The (original) Douay Rheims Bible 
%%%%
%%%% Old Testament
%%%% Leviticus
%%%% Chapter 11
%%%%
%%%%%%%%%%%%%%%%%%%%%%%%%%%%%%%%%%%%%%%%%%%%%%%%%%%%%%%%%%%%%%%%%




\startcomponent chapter-11


\project douay-rheims


%%% 0301
%%% o-0273
\startChapter[
  title={Chapter 11}
  ]

\Summary{The diſtinction of cleane and vncleane in beaſtes, fish, birdes,
  and other things. 43.~With commandment to be holie, and impolluted.}

And
\MNote{The third part.

Of things cleane and vncleane, with the maner of purifying: & other
precepts moral & iudicial.}
our Lord ſpake to Moyſes and
\SNote{Hitherto God reueled his Law to Moyſes onlie, and by him to the
people. Now alſo to Aaron after he was cõſecrated high Prieſt: yet not
alwayes, for Moyſes was ſtil ſuperior.
\XRef{chap.~12. 14.~16.~17. &c.}}
Aaron, ſaying: \V Say to the children of Iſrael: Theſe are the beaſts
which you ought to eate of al the liuing things of the earth. \V Euerie
one that hath the hoofe diuided, and cheweth the cudde among the cattel,
you ſhal eate. \V But whatſoeuer in dede cheweth the cudde, and hath an
hoofe, but diuideth it not, as the camel, and others, that you ſhal not
eate, and among the
\LNote{Vncleane you shal repute it.}{In
\CNote{\XRef{Gen.~7. et.~8.}}
\MNote{Some things counted vncleane in the law of nature & of Moyſes.}
the firſt age of the world, before Noes floud, and ſo forward by
tradition; and after by the written Law, ſome liuing creatures were
reputed vncleane, and forbid to be eaten or offered in ſacrifice. Not as
euel of themſelues,
\CNote{\XRef{1.~Tim.~4.}}
\Emph{for euerie creature of God is good}, by nature and creation:
\MNote{Three cauſes of this obſeruance.}
but this diſtinction and prohibition was made in the old Teſtament, for
iuſt cauſes, as the ancient fathers note ſpecially three.
\MNote{1.~For inſtruction.}
Firſt, for inſtruction of the people much inclined to idolatrie, God
diſtinguiſhed al beaſtes, birdes, and fishes into cleane and vncleane,
wherby al men might know, that none of them is God. \Emph{For hovv can
anie man of reaſon} (ſaieth lerned Theodoret
\Cite{q.~11. in Leuit.})
\Emph{thinck that to be God, vvhich either he abhorreth as vncleane, or
offereth in ſacrifice to the true God, and eateth therof himſelf?}
\MNote{2.~For exerciſe of obedience.}
Secondly, God commanded this obſeruance to exerciſe his people in
obedience, with precepts not otherwiſe neceſſarie, but becauſe he ſo
commanded. As at firſt he commanded Adam not to eate of the tree of
knowledge of good and euel. The tranſgreſſion wherof brought al mankind
into miſerie.
\CNote{\XRef{Rom.~5,~2.}}
From which againe Chriſt by his obedience redemed vs.
\CNote{\XRef{Mach.~6. &~7.}}
For obſeruation of this law old Eleazarus, and the ſeuen bretheren with
their mother, did geue their liues, rather then they would eate ſwines
fleſh, and for the ſame are glorious Martyrs, as teſtifie S.~Cyprian
\Cite{Epiſt. 56. ad Thibaritanos.}
&
\Cite{li. de exhort. Mart. c.~11.}
S.~Gregorie Nazianzen,
\Cite{orat.~20. de Machab.}
S.~Chryſoſtom
\Cite{de natiuitate ſeptem Machab.}
S.~Ambroſe
\Cite{li.~1. de officijs. c.~4.}
&
\Cite{li.~2. de Iacob. c.~10. &~11.}
and the whole Church celebrating their feaſt, the firſt day of Auguſt.
\MNote{3.~For ſignification.}
Thirdly and moſt ſpecially theſe obſeruation were commanded for
ſignification of vertues to be embraced, and of vices or ſinnes to be
auoided.
\MNote{The things holden for cleane ſignified vertues.}
Such beaſtes therfore were holden for cleane, and allowed for
mans foode, as diuide the hoofe, and ruminate, or chew the cudde,
ſignifying diſcretion betwixt good and euel; and diligent conſideration,
or meditation of Gods law:
\MNote{The vncleane ſignified vices.}
and the beaſtes which lack thoſe two
properties of diuiding the hoofe,
and chewing the cudde, or either of them, were reputed vncleane,
ſignifying ſuch men as care not
whether they do wel or euel, or do not ruminate, and meditate good
things, which they heare or read, forgetting or neglecting, what is
taught them. Likewiſe the fiſhes that haue finnes and ſcales, which
ſignifie eleuating of the mind, and auſteritie of life, were counted
cleane: but thoſe that want either of the ſame were vncleane and
prohibited. Alſo certaine birdes were eſtemed cleane and allowed to be
eaten: others vncleane and forbid. As the Eagle, ſignifying pride; the
griffon, tyrannie; the oſprey, oppreſſion; the kite, fraud; the vultare,
ſedition; al kindes of rauens, carnal voluptouſnes; the oſtrich,
worldlie cares; the owle, ſlouth, or dulnes in ſpiritual things; the
ſterne, duble dealing; al kindes of haukes, crueltie; the ſchritch owle,
luxurie; the diuer, gluttonie; the ſtorke, enuie; the ſwanne,
hypocriſie; the onocratal, auarice; the porphiron, ſelfe wil; the
herodian, a bloudie mind; the caladrion, much babling; the lapwing,
deſolation of mind, or deſperation; the batte, earthlie policie; and the
like in other birdes, beaſtes, and fishes. Al agreable to that time, in
which (ſaieth S.~Auguſtin
\Cite{li.~6. c.~7. cont. Fauſt.})
thoſe things were to be foreſhewed, not only in wordes, but alſo in
factes, which ſhould be reueled in latter time;
\MNote{Chriſtians are not bound to the obſeruances of the old law, but
to that which they ſignified.}
and being now reueled by
Chriſt, and in Chriſt, the burdenous obſeruances are not impoſed to the
faithful gentiles, to whom yet the authoritie of the prophecie is
commended. To the ſame effect,
\Cite{li. cont. Adimant, c.~15.}
&
\Cite{li.~50. homil. ho.~45.}
\Cite{S.~Hierom. in Matt.~15.}
\Cite{Origenes. ho.~7.}
\Cite{S.~Cyril. li.~7. in Leuit.}
\Cite{S.~Gregorie. in Cant.~7.}
\Cite{Procopius in Leuit.~11.}
Out of whom and others S.~Thomas explicateth at large, that which we
haue here briefly noted.
\Cite{1.~2. q.~102. a.~6.}}
vncleane you ſhal repute it. \V Cherogril which cheweth the cudde, and
diuideth not the hoofe, is vncleane. \V The hare alſo: for that alſo
cheweth the cudde, but diuideth not the hoofe. \V And the ſwine: which
though it diuideth the hoofe, cheweth not the cudde. \V The flesh of
theſe you ſhal not eate, nor touch their carcaſſes, becauſe they are
vncleane to you. \V Theſe are the thinges that brede in the waters, and
which it is lawful to eate. Al that hath finnes, and ſcales, aſwel in
the ſea, as in the riuers, and the pooles, you ſhal eate. \V But
whatſoeuer hath not finnes and ſcales, of thoſe that moue and liue in
the waters, ſhal be vnto you abhominable, \V and execrable, their flesh
you shal not eate,
%%% 0302
and their carcaſſes you ſhal
\Fix{avoide.}{auoide.}{possible typo, same in other}
\V Al that haue not finnes and ſcales in the waters, ſhal be
polluted. \V Of birdes theſe are they which you muſt not eate, and are
to be auoided of you: The Eagle, and the griffon, and the oſprey, \V and
the kite, and the vulture according to his kinde, \V and euerie one of
the rauens kinde, according to their ſimilitude, \V the oſtrich, and the
owle, and the ſterne, and the hauke according to his kinde, \V the
ſcritchowle, and the diuer, and the ſtorke, \V and the ſwanne, and the
onocratal, and the porphiron, \V the herodian, and the charadrion
according to his kind, the lapwing alſo, and the batte. \V Of foules
euerie one that goeth vpon foure feete, shal be abominable to you. \V
And whatſoeuer walketh vpon foure feete, but hath the legges behind
longer, wherwith he hoppeth vpon the earth, \V that you ſhal eate, as is
the bruke in his kind, the attake, and the ophiomach, and the locuſt,
euerie one according to their kinde. \V But of foule whatſoeuer hath
foure feete onlie, shal be execrable to you: \V and whatſoeuer ſhal
touch the carcaſſes of them, shal be polluted, and shal be vncleane
\SNote{If in dede this vncleãnes were a ſinne, it ſhould be clenſed by
contritiõ, and neither neceſſarily remaine til night, nor thẽ be taken
away without other meanes.}
vntil euen: \V and if it be neceſſarie that he carie anie of theſe that
be dead, he shal wash his clothes, and shal be vncleane vntil ſunne
ſette. \V Euerie beaſt that hath a hoofe, but diuideth it not, neither
cheweth the cudde, shal be
%%% o-0274
vncleane: and whatſoeuer toucheth it, shal be
defiled. \V That which walketh vpon hands of al beaſts, which goe on
foure feete, shal be vncleane: he that toucheth their carcaſſes, shal be
polluted vntil euen. \V And he that carieth ſuch carcaſſes, shal wash
his clothes, and shal be vncleane vntil euen: becauſe al theſe thinges
are vncleane to you. \V Theſe alſo shal be reputed among polluted
thinges, of al that moue vpon the earth, the weeſel and the mouſe and
the crocodile, euerie one according to their kinde, \V the migale, and
the camelean, and the ſtellion, and the lizard, and the moule: \V al
theſe are vncleane. He that toucheth their carcaſſes, shal be vncleane
vntil euen: \V and that wherupon anie thing of their carcaſſes falleth,
shal be polluted aſwel veſſel of wood and rayment, as skinnes and haire
clothes: and in whatſoeuer veſſel anie worke is done, they shal be
dipped in water, and shal be polluted vntil euen, and ſo afterward shal
be cleane. \V But the earthen veſſel, wherinto anie of theſe falleth
within it, shal be polluted, and therfore is to be broken. \V Al
%%% 0303
meate, which you shal eate, if the water be poured vpon it, shal be
vncleane; and al liquor that is dronke of al veſſel, shal be
vncleane. \V And vpon whatſoeuer ought of ſuch carcaſſes falleth, it
shal be vncleane: whether ouens, or pottes with feete, they shal be
diſtroyed, and shal be vncleane. \V But the fountaines and the
ceſternes, and al collection of waters shal be cleane. He that toucheth
their carcaſſe, shal be polluted. \V If it fal vpon ſeede corne it shal
not pollute it. \V But if any man poure water vpon the ſeede, and
afterward it be touched with the carcaſſes, it shal be forthwith
polluted. \V If a beaſt be dead, of which it is lawful for you to eate,
he that toucheth the carcaſſe therof, shal be vncleane vntil euen: \V
and he that eateth or carieth anie thing therof; shal wash his clothes,
and shal be vncleane vntil euen. \V Al that creepeth vpon the earth,
shal be abhominable, neither shal it be taken for meate. \V Whatſoeuer
goeth vpon the breſt on foure feete, and hath manie feete, or traileth
on the earth, you shal not eate, becauſe it is abhominable. \V Doe not
contaminate your ſoules, nor touch ought therof, leſt you be
vncleane. \V For I am the Lord your God: be holie, becauſe I am
holie. Pollute not your ſoules in anie creeping beaſt, that moueth vpon
the earth. \V For I am the Lord, that brought you out of the Land of
Ægypt, that I might be your God. \V You shal be holie becauſe I am
holie. \V This is the lawe of beaſts and foules, and of euerie liuing
ſoule, that moueth in the waters, and creepeth on the earth, \V that you
may know the differences of the cleane, and the vncleane, and know what
you ought to eate, and what to refuſe.


\stopChapter


\stopcomponent


%%% Local Variables:
%%% mode: TeX
%%% eval: (long-s-mode)
%%% eval: (set-input-method "TeX")
%%% fill-column: 72
%%% eval: (auto-fill-mode)
%%% coding: utf-8-unix
%%% End:
