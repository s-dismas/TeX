%%%%%%%%%%%%%%%%%%%%%%%%%%%%%%%%%%%%%%%%%%%%%%%%%%%%%%%%%%%%%%%%%
%%%%
%%%% The (original) Douay Rheims Bible 
%%%%
%%%% Old Testament
%%%% Ecclesiasticus
%%%% Chapter 21
%%%%
%%%%%%%%%%%%%%%%%%%%%%%%%%%%%%%%%%%%%%%%%%%%%%%%%%%%%%%%%%%%%%%%%




\startcomponent chapter-21


\project douay-rheims


%%% 1537
%%% o-1418
\startChapter[
  title={Chapter 21}
  ]

\Summary{An inuectiue againſt ſinne in general, 5.~and diuers in
  particular.}

Sonne haſt thou ſinned? doe ſo no more: but for the old alſo pray that
they may be forgeuen thee. \V
\SNote{As a ſerpent deceiptfully approcheth & ſtingeth the bodie, ſo al
ſinnes inuegle and hurt the ſoule.}
As from the face of a ſerpent flee from ſinnes: and if thou approch to
them, they wil receiue thee. \V The teeth of a lion the teeth thereof,
killing the ſoules of men. \V Al iniquitie is as a two edged ſword,
there is no remedie for the wound thereof. \V Brawling and iniuries shal
bring the ſubſtance to nothing: and the houſe that is verie rich, shal
be made nothing by pride: ſo the ſubſtance of the proude shal be rooted
out. \V The prayer of the poore out of the mouth shal come to his eares,
and iudgement shal come for him ſpedely. \V He that hateth chaſtiſement,
is
\SNote{It is a ſigne that he is guiltie, who contemneth freindly
admonition.}
the trace of a ſinner: and he that feareth God,
\SNote{He that truly feareth God wil diligently examine his ovvne actes,
& defectes when he is vvarned.}
wil turne to his owne hart. \V The mightie in a bold tongue is knowen a
far of, and a wiſeman knoweth himſelf to fal by him. \V He that
buildeth his houſe at other
\Fix{menes}{mens}{obvious typo, fixed in other}
charges, is as he that gathereth his ſtones
\SNote{As walles of ſtone built in the froſt, ſo riches or good name
vniuſtly gotten wil not cõtinue long.}
in the winter. \V The ſynagogue of ſinners is as tow gathered together,
%%% 1538
and their conſummation a flame of fire. \V The way of ſinners is paued
with ſtones, & in their end, hel, & darkenes, and paines. \V He that
keepeth iuſtice, shal conteine the vnderſtanding therof. \V The
conſummation of the feare of God wiſdom and vnderſtanding. \V He shal
not be taught, that is not wiſe in good. \V But there is wiſdom that
abundeth in euil: and there is no vnderſtanding where bitternes is. \V
The knowlege of the wiſe shal abound as an inundation, and his counſel is
permanent as a fountaine of life. \V The hart of a foole is as a broken
veſſel, and al wiſdom it shal not hold. \V A man of knowlege wil praiſe
whatſoeuer wiſe word he shal heare, and wil applie it to himſelf: the
riotous man hath heard it, and it shal diſpleaſe him, and he wil caſt it
behind his back. \V The
\SNote{Senſeles, or bad talke is tedious to al good men.}
narration of a foole is as a burden in the way: for in
\SNote{VVordes that may edifie are gratful to al godlie eares.}
the lippes of the wiſe shal grace be found. \V The mouth of the prudent
is ſought in the Church, and they wil thinke vpon his wordes in their
hartes. \V As a houſe deſtroied, ſo is wiſdom to a foole: & the knowlege
of the vnwiſe inexplicable wordes. \V Fetters on the feete, doctrine to
a foole, and as manicles vpon the right hand. \V A foole in laughter
exalteth his voice: but a wiſeman wil ſcarſe laugh ſecretly. \V Doctrine
to the prudent is a golden ornament, and as it were a bracelet on the
right arme. \V The foote of a foole goeth eaſely into his neighbours
houſe: & a cunning man wil be abashed at the perſon of the mightie. \V A
foole wil looke from the window into the houſe: but the nurtered wil
ſtand without. \V It is the follie of a man to harken by the dore: and a
wiſeman wil be greued with the contumelie. \V The lippes of the vnwiſe
shal tel foolish thinges: but the wordes
%%% o-1419
of the wiſe shal be pondered in balance. \V The hart of fooles is in
their mouth: and the mouth of wiſemen is in their hart. \V Whiles
\SNote{VVicked men condemning the diuel or anie other wicked, do in dede
condeme them ſelues. And to them agreeth that ſentence of our Sauiour:
By thyne owne mouth I iudge thee, naughtie ſeruant.
\XRef{Luc.~19.}}
the impious curſeth the diuel, he curſeth his owne ſoule. \V The
whiſperer shal defile his ſoule, and shal be hated in al: and he that
shal abide with him, shal be odious: the ſtil man and wiſe shal be
honored.


\stopChapter


\stopcomponent


%%% Local Variables:
%%% mode: TeX
%%% eval: (long-s-mode)
%%% eval: (set-input-method "TeX")
%%% fill-column: 72
%%% eval: (auto-fill-mode)
%%% coding: utf-8-unix
%%% End:


  
