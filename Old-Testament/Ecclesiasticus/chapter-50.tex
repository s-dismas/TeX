%%%%%%%%%%%%%%%%%%%%%%%%%%%%%%%%%%%%%%%%%%%%%%%%%%%%%%%%%%%%%%%%%
%%%%
%%%% The (original) Douay Rheims Bible 
%%%%
%%%% Old Testament
%%%% Ecclesiasticus
%%%% Chapter 50
%%%%
%%%%%%%%%%%%%%%%%%%%%%%%%%%%%%%%%%%%%%%%%%%%%%%%%%%%%%%%%%%%%%%%%




\startcomponent chapter-50


\project douay-rheims


%%% 1580
%%% o-1456
\startChapter[
  title={Chapter 50}
  ]

\Summary{Praiſes of Simon the High Prieſt. 27.~Deteſtation of certaine
  perſecuting aduerſaries. 29.~With concluſion that the obſeruers of
  this doctrine shal be wiſe and happie.}


Simon
\SNote{This Simon called Iuſtus, and Priſcus, was high prieſt when this
booke was written (in the time of Ptolomie the firſt, king of Ægypt) a
very holie man, and dead before it was tranſlated into Greke, about the
time of Ptolomie the third called Euergetes, nere 300.~yeares before
Chriſt.}
the ſonne of Onias, the high prieſt, who in his life held vp the houſe,
and in his daies ſtrengthned the temple. \V
\CNote{\Cite{Ioſephus lib.~12. Antiqui.}}
The height alſo of the temple was founded by him, the duble building and
high walles of the temple. \V In his daies the welles of waters flowed out,
and they were filled as the ſea aboue meaſure. \V Who had care of his
nation, and deliuered it from perdition. \V Who preuailed to amplifie
the citie,
%%% 1581
who obteyned glorie in conuerſing with the nation: and amplified the
entrance of the houſe, and the court. \V As the morning ſtarre in the
middes of a cloude, and as the ful moone he shineth in his dayes. \V And
as the ſunne shining, ſo did he shine in the temple of God. \V As the
rainbow that shineth among the cloudes of glorie, and as a flower of
roſes in the daies of the ſprine, and as the lilies that are in the
paſſage of water, and as frankenſence ſmelling in ſummer daies. \V As
fire gliſtering, and frankenſence burning in the fire. \V As a maſſie
veſſel of gold, adorned with euerie precious ſtone. \V As an oliue tree
budding, and a cypreſſe tree aduancing it ſelf on high, when he tooke
the robe of glorie, and was reueſted to the conſummation of ſtrength. \V
In going vp to the holie altar, he made the veſture of holines,
glorie. \V And in receiuing the portions out of the hand of the
prieſtes, himſelf alſo ſtanding by the altar. About him was the ring of
his bretheren: and as the ceder plant in mount Libanus, \V ſo ſtoode
they about him as boughes of the palme tree, & al the children of Aaron
in their glorie. \V And the oblation of our Lord in their handes, before
al the ſynagogue of Iſrael: and executing the conſummation on the altar,
to amplifie the oblation of the high king, \V he ſtretched forth his
hand in
\TNote{\L{libatione}}
oblation of moiſt ſacrifice, and offered of the blood of the grape. \V
He powred out on the fundation of the altar a diuine odour to the high
prince. \V Then cried out the children of Aaron, they ſounded with
beaten trumpets, and made a great voice to be heard for a remembrance
before God. \V Then al the people together made haſt, and fel on their
face vpon the earth, to adore our Lord their God, and to make prayers to
God omnipotent the Higheſt. \V And the ſingers amplified in their
voices, and in the great houſe the ſound was encreaſed ful of
ſweetenes. \V
\CNote{\XRef{Num.~6. v.~23.}}
And the people in prayer deſired our Lord the Higheſt, vntil the honour
of our Lord was perfected, and they finished their office. \V Then
coming downe, he lifted vp his handes ouer al the congregation of the
%%% o-1457
children of Iſrael, to geue glorie to God from his lippes, and to glorie
in his name, \V and he repeated his prayer, willing to shew the power of
God. \V And now pray ye the God of al, who hath done great thinges in al
the land, who hath encreaſed our daies from our mothers wombe, and hath
done with vs according to his mercie: \V geue he vnto vs ioyfulnes of
%%% 1582
\Fix{euerlaſting:}{hart euerlaſting:}{obvious typo, fixed in other}
\V that Iſrael may beleue that the mercie of God is with vs, to deliuer
vs in his dayes. \V Two nations my ſoule hateth: and the third is
\SNote{Three nations; the Idumeans, Philiſthijmes, and Samaritanes, did
moſt perſecute the Iſraelites: the Samaritanes were no one pure nation,
but mixt of Aſſirians and Iewes: and ſo here called \Emph{no nation}.}
no nation, which I hate: \V they that ſitte in mount Seir, and the
Philiſthijms, and the
\SNote{They are alſo called \Emph{a foolish people}, becauſe they
knowing true religion, mixed idolatrie therwith, according to diuers
ſectes, as appeareth
\XRef{4.~Reg.~17. v.~29.}}
foolish people that dwel in Sichem. \V Ieſus the ſonne of Sirach, a man
of Ieruſalem, wrote the doctrine of wiſdom and diſcipline in this booke,
who renewed wiſdom from his hart. \V Bleſſed is he, that conuerſeth in
theſe good thinges: and he that layeth them in his hart, shal be wiſe
always. \V For if he doe them, he shal be able to doe al thinges:
becauſe his ſteppes are in the light of God.


\stopChapter


\stopcomponent


%%% Local Variables:
%%% mode: TeX
%%% eval: (long-s-mode)
%%% eval: (set-input-method "TeX")
%%% fill-column: 72
%%% eval: (auto-fill-mode)
%%% coding: utf-8-unix
%%% End:


  
