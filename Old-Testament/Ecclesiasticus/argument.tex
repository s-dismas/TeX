%%%%%%%%%%%%%%%%%%%%%%%%%%%%%%%%%%%%%%%%%%%%%%%%%%%%%%%%%%%%%%%%%
%%%%
%%%% The (original) Douay Rheims Bible 
%%%%
%%%% Old Testament
%%%% Ecclesiasticus
%%%% Argument
%%%%
%%%%%%%%%%%%%%%%%%%%%%%%%%%%%%%%%%%%%%%%%%%%%%%%%%%%%%%%%%%%%%%%%




\startcomponent argument


\project douay-rheims


%%% 1507
%%% o-1391
\startArgument[
  title={\Sc{The Argvment of Ecclesiasticvs.}},
  marking={Argument of Ecclesiasticus.}
  ]


In what ſenſe this Booke is ſometimes called Salomons, we haue shewed in
the argument before the Booke of Wiſdom. As likewiſe \Emph{that it is
Canonical} Scripture. Wherto we might adde more teſtimonies of
ancient Fathers: as
\MNote{Particular teſtimonies that this booke is holie Scripture.}
S.~Clement of Alexandris,
\Cite{li.~1. c.~8. Pedagogi,}
Origin,
\Cite{ho.~8. in Numer.}
&
\Cite{ho.~1. in Exech.}
S.~Cyprian
\Cite{de opere & eleemoſ.}
S.~Athaſius
\Cite{in Synopſi,}
&
\Cite{li. de virginitate.}
S.~Baſil
\Cite{in regul. diſput. reſp.~104.}
S.~Gregorie Nazianzen.
\Cite{Orat.~2. aduerſ. Iulian.}
S.~Epiphanius.
\Cite{hær.~76.}
&
\Cite{in Ancorato.}
S.~Hilarie,
\Cite{in Pſal.~144.}
S.~Ambroſe
\Cite{de bono mortis. c.~8.}
&
\Cite{Ser.~22. in Pſal.~118.}
S.~Chryſoſtom
\Cite{ho.~33. ad populum Antioch.}
S.~Auguſtin,
\Cite{li.~2. ca.~8. Doct. Chriſt.}
&
\Cite{li.~17. c.~20. de Ciuit.}
S.~Gregorie the great
\Cite{in Pſal.~50.}
and manie others expreſly cite this booke as holie Scripture. But
chiefly we relie vpon the auctoritie of the Church defining that it is
Canonical.

%%% 1508
It
\MNote{It was written in Hebrew & tranſlated into Greke.}
was written by \Emph{Ieſus the ſonne of Sirach in Hebrew}, about the
time of Simon Iuſtus, otherwiſe called Priſeus: and \Emph{tranſlated
into Greke by the auctors Nephew}, as the ſame Tranſlator teſtifieth in
his Prologue, but expreſſeth not his owne name. It is called
\Emph{Ecclesiasticus}, which ſignifieth \Emph{a Collector} or
\Emph{Gatherer}, as a common title of euerie ordinarie preacher,
inſtructing and exhorting the multitude gathered to a ſermon:
\MNote{Difference betwen Ecclesiasticus, and Ecclesiastes.}
with difference from \Emph{Ecclesiastes}: which ſignifieth
\Emph{The Preacher}, as a greater title of \Emph{the chief} or principal
\Emph{Preacher of anie Church, Citie, or Prouince}, and agreeth moſt
eminently to \Emph{Chriſt our Sauiour}: who preached, and ſendeth
preachers to the whole world.
\MNote{\GG{Panaretos}.}
And for the excellent contents, it \Emph{may} alſo rightly \Emph{be
called} \GG{Panaretos}, that is, a \Emph{Receptacle}, or ſtorehouſe
\Emph{of al vertues}, for the inſtruction of al in general, to cooperate
with Gods grace in this life, and ſo enherite eternal glorie.
\MNote{The contents diuided into two partes.}
In fourtie and three whole chapters, are mixtly the commendations, and
precepts of al ſortes of vertues; ſometimes in particular, but more
often vnder the general names of Wiſdom and Iuſtice. In the other eight
chapters are recited manie excellent examples of moſt renowmed holie
men: with praiſes and thankes to God.


\stopArgument


\stopcomponent


%%% Local Variables:
%%% mode: TeX
%%% eval: (long-s-mode)
%%% eval: (set-input-method "TeX")
%%% fill-column: 72
%%% eval: (auto-fill-mode)
%%% coding: utf-8-unix
%%% End:
