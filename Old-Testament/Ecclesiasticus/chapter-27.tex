%%%%%%%%%%%%%%%%%%%%%%%%%%%%%%%%%%%%%%%%%%%%%%%%%%%%%%%%%%%%%%%%%
%%%%
%%%% The (original) Douay Rheims Bible 
%%%%
%%%% Old Testament
%%%% Ecclesiasticus
%%%% Chapter 27
%%%%
%%%%%%%%%%%%%%%%%%%%%%%%%%%%%%%%%%%%%%%%%%%%%%%%%%%%%%%%%%%%%%%%%




\startcomponent chapter-27


\project douay-rheims


%%% 1547
%%% o-1426
\startChapter[
  title={Chapter 27}
  ]

\Summary{For want, and deſire of riches, manie committe ſinne, 4.~from
  which the feare of God preſerueth. 6.~Tentation proueth, who is iuſt,
  12.~conſtant, and modeſt. 17.~Freindes are bond to ſecreſie, 25.~and
  fidelitie.}


Through pouertie manie haue offended:
\CNote{\XRef{1.~Tim.~6.}}
and he that ſeeketh to be made rich, turneth away his eie. \V As a ſtake
is faſtened in the middes of ſtones compact together, ſo alſo in the
middes of ſelling and buying, ſinne shal be ſtraytened. \V Sinne shal be
deſtroyed with the ſinner. \V If thou hold not thyſelf inſtantly in the
feare of our Lord,
\SNote{The ſoule is kept in good ſtate by fearing God.}
thy houſe shal quickly be ſubuerted. \V As in the shaking of a ſieue the
duſt wil remaine: ſo
\SNote{After that ſinne is purged there remaine reliques in the ſoule,
as duſt in a ſieue, vvhen the chaffe is caſt out, til it be more purged
or washed.
\XRef{Pſal.~50. v.~4.}}
the perplexitie of a man in his cogitation. \V
\CNote{\XRef{Prou.~27.}}
The fornace tryeth the potters veſſels, and the tentation of tribulation
iuſt men. \V As the husbandrie about a tree sheweth the fruite thereof,
ſo a word out of the thought of the hart of man. \V Prayſe not a man
before ful diſcourſe, for this is the trial of men. \V If thou folow
iuſtice, thou shalt apprehend it: and shalt put it on as a long robe of
honour, and thou shalt dwel with it: and it shal protect thee for euer,
and in the day of
\Fix{knowleging}{acknowleging}{likely typo, fixed in other}
thou shalt finde ſtedfaſtnes. \V The ſoules flocke together to their
like: and truth shal returne to them, that worke it. \V The lion alwayes
lyeth in wayte for a pray: ſo ſinnes for them that worke iniquities. \V
A holie man continueth in wiſdom
\SNote{VVhether the ſunne shineth forth or not, it is alvvayes light: ſo
is a vviſman alvvayes vertuous, vvhether it appeare outvvardly or no.}
as the ſunne: for a foole is changed as
\SNote{A foole, or vvicked man, hath no light of vertue in himſelf (like
the moone) but ſometimes ſemeth to haue more light, ſometimes leſſe,
ſometimes none at al.
\Cite{S.~Bernard}}
the moone. \V In the middes of the vnwiſe keepe the word til his time:
but in the middes of deepe conſiderers be continually. \V The naration
of ſinners is odious, & their laughter is in the
\Fix{deligthes}{delightes}{obvious typo, fixed in other}
of ſinne. \V Speach that
%%% o-1427
ſweareth much
%%% 1548
shal make the heare of the head to ſtand vpright: and his lacke of
reuerence is the ſtopping of the eares. \V Sheding of bloud is in the
brawling of the proud: and their curſing is a greeuous hearing. \V He
that diſcloſeth the ſecrete of a freind, loſeth credite, and he shal not
finde a freind to his minde. \V Loue thy neighbour, and be ioyned with
him in fidelitie. \V But if thou diſcouer his ſecrets, thou shalt not
purſew after him. \V For as a man that loſeth his freind, ſo alſo he
that loſeth the freindshipe of his neighbour. \V And as he that letteth
a bird goe out of his hand, ſo haſt thou leaft thy neighbour, & shalt not
take him. \V Folow him not, becauſe he is far abſent, for he is fled, as
a doe out of the ſnare: becauſe his ſoule is wounded. \V Thou canſt no
more blinde him, and of a curſe there is reconciliation: \V but to
diſcloſe the ſecrets of a freind, is the deſperation of an vnhappie
ſoule. \V
\CNote{\XRef{Prou.~10.}}
He that winketh with the eie, forgeth wicked thinges, and no man wil
caſt him of: \V in the ſight of thyne eyes he wil
\Fix{ſweete}{ſweeten}{obvious typo, fixed in other}
his mouth, and wil be in admiration vpon thy wordes: but at the laſt he
wil peruert his mouth, and in thy wordes he wil lay a ſcandal. \V I haue
heard manie thinges, & haue not eſteemed them equal to him, and our Lord
wil hate him. \V He that
\SNote{He that expreſly doth iniurie to an other is iuſtly punished alſo
in this vvorld.}
caſteth a ſtone on high, it wil fal vpon his head: and
\SNote{Hovv ſecretly ſoeuer anie hurteth an other, he vvoundeth his
ovvne conſcience, and can not eſcape Gods iudgement.}
the deceitful ſtroke wil diuide the woundes of the deceitful. \V
\CNote{\XRef{Prou.~16.}
\XRef{Eccle.~10.}}
He that diggeth a pit, shal fal into it: and he that ſetteth a ſtone
for his neighbour, shal ſtumble on it: & he that layeth a ſnare for
an other, shal perish in it. \V To a man that doth moſt wicked counſel,
it shal be turned vpon himſelf, and he shal not know from whence it
cometh to him. \V Deriſion & reproch of the proud, and vengeance as a
lyon shal lie in waite for him. \V They shal perish in a ſnare that are
delighted with the fal of the iuſt: and ſorow shal conſume them before
they die. \V Anger and furie, both are execrable, and the ſinful man
shal be ſubiect to them.


\stopChapter


\stopcomponent


%%% Local Variables:
%%% mode: TeX
%%% eval: (long-s-mode)
%%% eval: (set-input-method "TeX")
%%% fill-column: 72
%%% eval: (auto-fill-mode)
%%% coding: utf-8-unix
%%% End:


  
