%%%%%%%%%%%%%%%%%%%%%%%%%%%%%%%%%%%%%%%%%%%%%%%%%%%%%%%%%%%%%%%%%
%%%%
%%%% The (original) Douay Rheims Bible 
%%%%
%%%% Old Testament
%%%% Ecclesiasticus
%%%% Chapter 12
%%%%
%%%%%%%%%%%%%%%%%%%%%%%%%%%%%%%%%%%%%%%%%%%%%%%%%%%%%%%%%%%%%%%%%




\startcomponent chapter-12


\project douay-rheims


%%% 1525
%%% o-1407
\startChapter[
  title={Chapter 12}
  ]

\Summary{Vſe beneuolence towards good men. 10.~Truſt not enemies ouer much.}


If thou wilt doe good,
\SNote{It is rather crueltie then mercie to nouriſh a wicked man
perſiſting in ſinne: for ſo he runneth ſtil into more wickednes, and
into eternal damnation,}
know to whom thou doeſt it, and there shal be much thanke in thy good
deedes. \V Doe good to the iuſt, and thou shalt finde great rewarde: and
if not of him, aſſuredly of our Lord. \V For it is not wel with him,
that is euer occupied in euil thinges, and that geueth not almes:
becauſe the Higheſt both hateth ſinners, and hath mercie on them
\SNote{but the penitent is to be comforted and aſſiſted.}
that are penitent. \V Geue to the merciful, and receiue not the ſinner:
both to the impious, & to ſinners he wil repay vengeance, keping them
vnto the day of vengeance. \V Geue to the good, and receiue not a
ſinner. \V Doe good to the humble, and geue not to the impious:
prohibite to geue him bread, leſt therin he be mightier then thou: \V for
thou shalt finde duble euils in al the good, whatſoeuer thou shalt do
to him: becauſe the Higheſt hateth ſinners, and wil repay vengeance to
the impious. \V A freind shal not be knowen in proſperitie, and an
enimie shal not be hid in aduerſitie. \V In the proſperitie of a man,
his enimies are in ſorow, and in affliction a freind is knowne. \V
\SNote{Euerie one is bond to loue his enemie of charitie, but in
prudence it behoueth not to credite him. According to our Sauiours rule:
Be wiſe as ſerpents; and ſimple as dooues.
\XRef{Mat.~10.}}
Credite not thyn enemie for euer: for as a braſſe potte his wickednes
ruſteth: \V and if humbling himſelf he goe crouching, be aduiſed in thy
mind, and beware of him. \V Place him not by thee, neither
%%% 1526
let him ſitte on thy right hand, leſt perhaps turning into thy place, he
ſeke after thy ſeate: and at the laſt thou know my wordes, and be
pricked in my ſayinges. \V Who wil haue pittie vpon the inchanter,
ſtricking of a ſerpent, or of anie that come nere to beaſtes? ſo alſo he
that kepeth companie with a wicked man, and is wrapped in his ſinnes. \V
For one houre he wil tarie with thee: but if thou decline, he wil not
abide it. \V In his lippes the enimie ſpeaketh ſwetely, and in his hart
he lyeth in wayte, that he may ouerthrow thee into the pitte. \V In his
eyes the enimie weepeth: and if he may finde a time, he wil not be
ſatisfied with bloud: \V and if euils happen to thee, thou shalt finde
him there firſt. \V In his eyes the enimie weepeth, and as it were
helping thee, he wil vndermine thy feete. \V He wil shake his head, and
clappe his hand, and whiſpering manie thinges he wil change his
countenance.


\stopChapter


\stopcomponent


%%% Local Variables:
%%% mode: TeX
%%% eval: (long-s-mode)
%%% eval: (set-input-method "TeX")
%%% fill-column: 72
%%% eval: (auto-fill-mode)
%%% coding: utf-8-unix
%%% End:


  
