%%%%%%%%%%%%%%%%%%%%%%%%%%%%%%%%%%%%%%%%%%%%%%%%%%%%%%%%%%%%%%%%%
%%%%
%%%% The (original) Douay Rheims Bible 
%%%%
%%%% Old Testament
%%%% Ecclesiasticus
%%%% Chapter 19
%%%%
%%%%%%%%%%%%%%%%%%%%%%%%%%%%%%%%%%%%%%%%%%%%%%%%%%%%%%%%%%%%%%%%%




\startcomponent chapter-19


\project douay-rheims


%%% 1534
%%% o-1415
\startChapter[
  title={Chapter 19}
  ]

\Summary{An admonition againſt drunkennes, luxurie, 4.~light ſuſpition,
  7.~an detraction. 13.~Freindlie correption is neceſſarie, 23.~and
  ſincere humilitie. 27.~Exterior carege is a ſigne of internal
  diſpoſition.}


A workman that is a drunkard shal not be rich: and he
\SNote{The beſt remedie againſt great ſinnes is to auoide ſmal ones, and
not to contemne the leaſt, but diligently to amend al.}
that contemneth ſmal thinges, shal fal by litle and litle. \V
\CNote{\XRef{3.~Reg.~11.}}
Wine and wemen make wiſemen to apoſtatate, and shal reproue the prudent: \V
and he that ioyneth himſelf to harlotes, shal be naught. Rottennes and
wormes shal inherite him, and he shal be lifted vp for a greater
example, and his life shal be taken out of the number. \V
\CNote{\XRef{Ios.~22.}}
He that geueth credite quickly, is light of hart, and shal be leſſened:
and he shal more ouer be counted one that ſinneth againſt his owne
ſoule. \V He that reioyceth in iniquitie, shal be reprehended, and he
that hateth chaſtiſement, shal be diminished of life: and he that hateth
babbling, extinguisheth malice. \V He that ſinneth againſt his owne
ſoule, shal repent: and he that is delighted in
%%% 1535
naughtineſſe, shal be reprehended. \V Iterate not a wicked and hard
word, and thou shalt not be leſſened. \V To freind and foe tel not thy
minde: and if thou haue ſinne,
\SNote{That this document perteyneth to common conuerſation with
worldlie men, appeareth by the next verſe. But to reuele ſecrete ſinnes
to a ſpiritual father, in ſacramental confeſſion, is neceſſarie
vvholeſome and ſecure. It is alſo very commendable and moſt lawful in
holie religious Societies, vvhere they willingly for their owne
ſpiritual good ſubmitte themſelues to ſuch a godlie rule.}
diſcloſe it not. \V For he wil heare thee, and wil watch thee, and as it
were defending the ſinne, he wil hate thee, and ſo wil he be preſent with
thee alwaies. \V Haſt thou heard a word againſt thy neighbour? let it
die together in thee, truſting that it wil not burſt thee. \V At the preſence of
a word the foole traueleth, as the groning of the childbirth of an
infant. \V An arrow ſtickt in the thigh of flesh: ſo is a word in the
hart of a foole. \V
\CNote{\XRef{Leuit.~19.}}
Rebuke a freind, leſt perhapes he hath not vnderſtood, and ſay: I did it
not: or if he did it, that he doe it not againe. \V
\CNote{\XRef{Mat.~18.}}
Rebuke thy neighbour, leſt perhaps he ſaid it not: and if he ſaid it,
leſt perhaps he iterate it. \V Rebuke thy freind: for there is often
a fault committed. \V And beleue not euerie word. There is that
offendeth with the tongue, but not from his hart. \V
\CNote{\XRef{Iac.~3.}}
For who is there that hath not offended in his tongue? Rebuke thy
neighbour before
%%% o-1416
thou threaten. \V And geue place to the feare of the Higheſt: becauſe
the feare of God is al wiſedom, and to feare God is in it, & the
diſpoſition of the law is in al wiſdom. \V And the diſcipline of
wickednes is not wiſedom: and the cogitation of ſinners is not
prudence. \V There is wickednes, and in it execration: and there is a
foole that hath leſſe wiſedom. \V Better is a man that hath leſſe
wiſdom, and lacketh vnderſtanding, in feare, then he that abundeth in
vnderſtanding, and tranſgreſſeth the law of the Higheſt. \V There is an
aſſured ſubtilitie, & the ſame wicked. \V And there is that vttereth an
exact word telling the truth. There is that
\SNote{Falſe pretence of pietie is hypocriſie.}
wickedly humbleth himſelfe, and his inner partes be ful of deceite: \V
and there is a iuſt man
\SNote{And in a Superior, to
\Fix{oppen}{open}{obvious typo, fixed in other}
his ovvne ſecrete fault to his ſubiects is puſillanimitie.}
that ſubmitteth himſelf ouermuch of great humilitie: and there is a iuſt
one
\SNote{Diſcretion auoideth both: by concealing and reueling faultes as
reaſon directeth and iuſtice requireth.}
that boweth his face, and feyneth himſelf not to ſee that which is
vnknowen: \V and if he be forbidden to ſinne for imbecillitie of power,
if he shal finde a time to do euil, he wil do euil. \V A man is knowen
by the ſight, and a wiſeman is knowen by the shew of his face. \V The
clothing of the bodie, and the laughing of the teeth, and the going of
the man tel of him. \V There is a lying chaſtiſement in the anger of a
contumelious perſon: and there is a iudgement, that is not allowed to be
good: and there is that holdeth his peace, and he is wiſe.


\stopChapter


\stopcomponent


%%% Local Variables:
%%% mode: TeX
%%% eval: (long-s-mode)
%%% eval: (set-input-method "TeX")
%%% fill-column: 72
%%% eval: (auto-fill-mode)
%%% coding: utf-8-unix
%%% End:


  
