%%%%%%%%%%%%%%%%%%%%%%%%%%%%%%%%%%%%%%%%%%%%%%%%%%%%%%%%%%%%%%%%%
%%%%
%%%% The (original) Douay Rheims Bible 
%%%%
%%%% Old Testament
%%%% Ecclesiasticus
%%%% Chapter 22
%%%%
%%%%%%%%%%%%%%%%%%%%%%%%%%%%%%%%%%%%%%%%%%%%%%%%%%%%%%%%%%%%%%%%%




\startcomponent chapter-22


\project douay-rheims


%%% 1538
%%% o-1419
\startChapter[
  title={Chapter 22}
  ]

\Summary{An other admonition againſt ſlouth, 3.~diſſolute children,
  6.~and mirth out of ſeaſon. 7.~Fooles are hardly corrected, 10.~more
  to be bewayled then the dead. 14.~Much talke doth not profite
  them. 24.~Offend not, nor feare not a freind. 33.~Kepe alwayes guard
  of thy tongue.}


%%% 1539
The ſluggard is ſtoned
\SNote{Contempt & ignominie is the worldlie puniſhment of the ſlouthful,
beſides his eternal damnation at the day of iudgement.
\XRef{Mat.~25. v.~30.}}
with a durtie ſtone, & al men wil ſpeake of his diſgrace. \V The
ſluggard is ſtoned with the dung of oxen: and euerie one, that shal
touch him, wil shake his handes. \V The confuſion of the father is of a
ſonne without diſcipline: and the daughter shal be made of leſſe
account. \V A wiſe daughter is an inheritance to her husband, for she
that confoundeth, is made a contumelie to her father. \V She that is
bold shameth father and husband, and shal not be inferiour to the
impious: but of them both she shal be dishonored. \V Muſike in mourning
is a tale out of time: ſcourges and doctrine are at al time wiſdom. \V
He that teacheth a foole, is as he that gleweth together a potshard. \V
He that telleth a word to him that heareth not, is as he that raiſeth vp
a man ſleepeing out of an heauie ſleepe. \V He ſpeaketh with him that
ſleepeth, which vttereth wiſdom to a foole: and in the end of the
narration he ſaieth: Who is this? \V Weepe vpon the dead, for his light
hath failed: and weepe vpon
\SNote{In this and other places is not vnderſtood a foole that by defect
of natural vvitte is ignorant, or an ideote, but he that is voide of
grace, ful of malice, and wickednes. For the wicked life of ſuch a one
is worſe then his death.
\XRef{v.~12.}}
a foole, for he faileth in vnderſtanding. \V Weepe a little vpon the
dead, becauſe he is at reſt. \V For the wicked life of the very wicked,
aboue the death of a foole. \V
\CNote{\XRef{Gen.~50.}
\XRef{Prou.~23.}}
The
\Fix{moorning}{mouring}{obvious typo, fixed in other}
of the dead is ſeuen daies: but of a foole and of the impious, al the
daies of their life. \V Speake not much with a foole, and goe not with
the vnwiſe. \V Keepe thy ſelfe from him, that thou haue no moleſtation:
and thou shalt not be defiled with his ſinne. \V Turne aſide from him,
and thou shalt finde reſt; and shalt not be wearied with his follie. \V
What shal be heauier then lead? and what other name hath it but
foole. \V It is eaſier to beare ſand and ſalt, and a maſſe of yron, then
an vnwiſe man, and a foole, and impious. \V
\CNote{\XRef{Prou.~27.}}
A frame of wood bound together in the fundation of a building, shal not
be diſſolued: ſo alſo the hart confirmed in the cogitation of
counſel. \V The cogitation of the wiſe at al time, yea by feare shal not
be depraued. \V As ſtakes in high places, and plaiſteringes laid without
coſt, shal not abide againſt the face of the winde: \V ſo alſo a feareful
hart in the
%%% o-1420
cogitation of a foole shal not reſiſt againſt the violence of
feare. \V As a trembling hart in the cogitation of a foole, al time wil
not feare, ſo alſo he that continueth alwaies in the preceptes of
God. \V He that pricketh the eie, bringeth forth teares: and he that
pricketh the hart, bringeth forth feeling. \V He that caſteth a
%%% 1540
ſtone at fowles, and shal throw them downe: ſo he that ſpeaketh
reprochefully to his freind, diſſolueth freindship. \V
\SNote{A true freind wil not be loſt for temporal damage nor danger.}
Although thou droweſt a ſword at a freind, deſpaire not: for there is
returning to a freind. \V If thou open a ſad mouth, feare not, for there
is agreement:
\SNote{But the vices of deriſion, reproch, and the like violate al
freindſhipe with wiſe and good men.}
except taunt, and reproch and pride, and reuealing of ſecret, and a
traiterous wound: in al theſe thinges a freind wil flee away. \V
Poſſeſſe fidelitie with a freind in his
\Fix{prouertie,}{pouertie}{obvious typo, fixed in other}
that in his goodes alſo thou maiſt reioyce. \V In the time of his
tribulation continew faithful to him, that in his inheritance alſo thou
maiſt be heire with him. \V Before the fire the vapour of the chimney,
and the ſmoke of the fire riſeth on high: ſo alſo before bloud euil
wordes, and contumelies, & threates. \V I wil not be ashamed to ſalute a
freind, from his face I wil not hide myſelf: and if there chance euiles
to me by him, I wil beare it. \V Euerie one that shal heare, wil beware
of him. \V
\CNote{\XRef{Ps.~140.}}
Who wil geue a gard to my mouth, and a ſure ſeale vpon my lippes, that I
fal not by them, and my tongue deſtroy me?


\stopChapter


\stopcomponent


%%% Local Variables:
%%% mode: TeX
%%% eval: (long-s-mode)
%%% eval: (set-input-method "TeX")
%%% fill-column: 72
%%% eval: (auto-fill-mode)
%%% coding: utf-8-unix
%%% End:


  
