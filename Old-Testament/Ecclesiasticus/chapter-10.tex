%%%%%%%%%%%%%%%%%%%%%%%%%%%%%%%%%%%%%%%%%%%%%%%%%%%%%%%%%%%%%%%%%
%%%%
%%%% The (original) Douay Rheims Bible 
%%%%
%%%% Old Testament
%%%% Ecclesiasticus
%%%% Chapter 10
%%%%
%%%%%%%%%%%%%%%%%%%%%%%%%%%%%%%%%%%%%%%%%%%%%%%%%%%%%%%%%%%%%%%%%




\startcomponent chapter-10


\project douay-rheims


%%% 1522
%%% o-1404
\startChapter[
  title={Chapter 10}
  ]

\Summary{Wiſe ſuperiors are very neceſſarie, becauſe the multitude folow
  their example. 6.~Remitte and forget iniuries, deteſt pride,
  iniuſtice, contumelie, and auarice. 12.~Life is short. 14.~Pride is
  the roote of al ſinnes. 23.~Iuſt pouertie is better then ſinful
  riches. 31.~Mekenes and modeſtie are neceſſarie in al men.}


A
\CNote{\XRef{Prou.~29.}}
wiſe iudge shal iudge his people, and the principalitie of the wiſe shal
be ſtable. \V
\SNote{Example of rulers is of great efficacie.}
According to the Iudge of the people, ſo alſo are his miniſters: and
what maner of man the ruler of a citie, ſuch alſo are the inhabitants
therein. \V An vnwiſe king shal deſtroy his people: and cities shal be
inhabited by the vnderſtanding of the prudent. \V The powre of the earth
is in the hand of God, and he wil rayſe vp a profitable ruler for a time
ouer it. \V The proſperitie of man is in the hand of God, & vpon the
face of the ſcribe he wil put his honour. \V Anie iniurie of thy
neighbour remember not, and doe nothing by workes of iniurie. \V
\CNote{\XRef{Dan.~4.}}
Pride is odious before God and men: and al the iniquitie of the nations
is execrable. \V A
\SNote{The cauſes of tranſlating kingdomes, & dominions.}
kingdome is tranſlated from nation vnto nation, becauſe of iniuſtices,
and iniuries, and contumelies, and diuerſe deceites. \V But
\SNote{\Emph{Couetouſnes is the roote of al euiles},
\XRef{1.~Tim.~6.}
in that for lucre manie fal into al ſortes of ſinnes, euen into ſchiſme
and hereſie, \Emph{erring from the faith}.
\XRef{ibidem. v.~10.}}
nothing is more wicked then the couetous man. Why is earth and ashes
proud? \V Nothing is more wicked then to loue money. For he hath his
ſoule alſo to ſel: becauſe in his life he hath caſt forth his moſt
inward thinges. \V Al power is of short life. Long ſicknes greueth the
Phyſicion. \V Short ſicknes the Phyſicion cutteth of at the firſt: ſo
alſo the king is to day, & to morrow he shal die. \V For when a man shal
die, he shal inherite ſerpents, and beaſts, and wormes. \V The begynning
of the pride of man, is to apoſtate from God: \V becauſe his hart is
departed from him that made him, for
\SNote{Neuertheles pride is the beginning of al ſinne, as this text
expreſly teſtifieth, and the reaſon is, for that mans inordinate ſelf
loue is the cauſe of declining from Gods commandments, & they which
runne on in that courſe, caſt themſelues headlong into the depth of al
miſchief, and of eternal miſerie.}
pride is the begynning of al ſinne: he that holdeth it, shal be filled
with curſes, & it shal ſubuert him in the end. \V Therfore hath our Lord
dishonoured the congregations of the euil, & hath deſtroyed them euen to
the end. \V God hath deſtroyed the ſeates of proud princes, and hath
made the meeke ſitte in their ſtead. \V God hath made the rootes of the
proud nations to wither, and hath planted the humble of the nations
themſelues. \V Our Lord hath ſubuerted the landes of the gentiles, and
hath deſtroyed them euen to the fundation.
%%% 1523
\V He hath made of them to wither, and hath deſtroyed them, and hath
made the memorie of them to ceaſe from the earth. \V God hath deſtroyed
the memorie of the proud, and hath left the memorie of them that are
humble in vnderſtanding. \V Pride was not created to men: nor wrath to
the nation of wemen. \V That ſeede of men shal be honoured, which
feareth God: but that ſeede shal be dishonoured, which
tranſgreſſeth the commandments of our Lord. \V In the middes of brethren
their ruler shal be
%%% o-1405
in honour: and they that feare our Lord, shal be in his eyes. \V The
glorie of the rich, of the honourable, and of the poore, is the feare of
God. \V Deſpiſe not the iuſt man that is poore, and magnifie not the
ſinful man that is rich. \V The great one, and the iudge, and the
mightie is in honour, and there is none greater then he, that feareth
God. \V Free men wil ſerue a ſeruant, that is wiſe:
\CNote{\XRef{Prou.~17.}}
and a man that is prudent and hath diſcipline, wil not murmur being
rebuked, and the ignorant shal not be honoured. \V Extol not thyſelf in
doing thy worke, and linger not in the time of diſtreſſe: \V
\CNote{Prou.~12.}
better is he that worketh, and abundeth in al thinges, then he that
glorieth, and lacketh bread. \V Sonne in mildenes keepe thy ſoule, and
geue him honour according to his deſert. \V Him that ſinneth agaynſt his
owne ſoule who shal iuſtifie? and who shal honour him that dishonoureth
his owne ſoule? \V The poore man is glorified by his diſcipline and
feare: & there is a man that is honoured for his ſubſtance. \V But he
that is glorified in pouertie, how much more in ſubſtance? and he that
is glorified in ſubſtance, let him feare pouertie.


\stopChapter


\stopcomponent


%%% Local Variables:
%%% mode: TeX
%%% eval: (long-s-mode)
%%% eval: (set-input-method "TeX")
%%% fill-column: 72
%%% eval: (auto-fill-mode)
%%% coding: utf-8-unix
%%% End:


  
