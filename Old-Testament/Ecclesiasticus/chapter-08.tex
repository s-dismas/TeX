%%%%%%%%%%%%%%%%%%%%%%%%%%%%%%%%%%%%%%%%%%%%%%%%%%%%%%%%%%%%%%%%%
%%%%
%%%% The (original) Douay Rheims Bible 
%%%%
%%%% Old Testament
%%%% Ecclesiasticus
%%%% Chapter 08
%%%%
%%%%%%%%%%%%%%%%%%%%%%%%%%%%%%%%%%%%%%%%%%%%%%%%%%%%%%%%%%%%%%%%%




\startcomponent chapter-08


\project douay-rheims


%%% 1519
%%% o-1402
\startChapter[
  title={Chapter 08}
  ]

\Summary{Contend not with a man of powre, rich, ful of tongue, or very
  ignorant. 6.~Deſpiſe not the penitent, nor old folke. 8.~Reioyce not
  at an enemies death. 9.~Lerne of the elder. 13.~Obſerue diſcretion in
  adminishing, lending, and in being ſuertie. 17.~Reproue not
  Iudges. 18.~Conuerſe not with the furious, foolish, nor with
  ſtrangers.}


Striue not with a mightie man, leſt perhaps thou fal into his handes. \V
Contend not with a rich man, leſt perhaps he make an action againſt
thee. \V For
\SNote{Briberie ſometimes corrupteth kinges much more other inferior
Iudges. And therfore it is better to ſuffer damage then to contend by
law againſt the rich.}
gold and ſiluer hath deſtroyed manie, and hath reached euen to the
%%% 1520
hart of kinges, and hath turned them. \V Striue not with a man ful of
tongue, and thou shalt not heape ſtickes vpon his fyre. \V Commuicate
not with the ignorant man, leſt he ſpeake il of thy progenie. \V
\CNote{\XRef{Gal.~6.}}
Deſpiſe not a man that turneth himſelf from ſinne, nor vpbrayde him
therwith: remember that we are al in ſtate to be blamed. \V Deſpiſe not
a man in his old age: for we alſo shal become old. \V Reoyce not of
thine enemie dead: knowing that we al doe die, and would not that others
should ioy therat. \V Deſpiſe not the narration of wiſe ancients, and in
their prouerbes be thou conuerſant. \V For of them thou shalt lerne
wiſdom, and doctrine of vnderſtanding, and to ſerue great man without
blame. \V Let not the narration of the ancients eſcape thee: for they
lerned of their fathers: \V becauſe of them thou shalt lerne
vnderſtanding, and in time of neceſſitie to geue anſwer. \V Kindle not
the coles of ſinners rebuking them,
\CNote{\XRef{Prou.~26.}}
and be not kindled with the flame of the fire of their ſinnes. \V Stand
not againſt the face of a contumelious perſon, leſt he ſitte as a ſpie
in wayte for thy mouth. \V Lend not to a man mightier then thyſelf, and
if thou doeſt lend, count it as loſt. \V Be not ſuretie aboue thy power:
and if thou be ſuretie, thinke as if thou were to pay it. \V Iudge not
agaynſt a iudge: becauſe he iudgeth according to that which is iuſt. \V
With the audacious goe not on the way, leſt perhaps he burden thee with
his euils: for he goeth according to his owne wil, and thou may perish
together with ths follie. \V
\CNote{\XRef{Prou.~22. v.~24.}}
With an angrie man make no brawle, and with the audacious goe not into
the deſert: becauſe bloud is as nothing before him, and where there is
no helpe, he wil ouerthrow thee. \V Conferre no counſel
\SNote{In al conſultations conferre with the skilful; for the blinde can
not iudge of colours, the deafe of muſike, the ſicke of taiſt: nor the
worldlie men of ſpiritual thinges.}
with fooles, for they can not loue but ſuch as pleaſe them. \V Before a
ſtranger doe no matter of counſel: for thou knoweſt not what he wil
bring forth. \V Make not thy hart manifeſt to euerie man: leſt perhaps
he repay thee falſe kindnes, and ſpeake reprochfully to thee.


\stopChapter


\stopcomponent


%%% Local Variables:
%%% mode: TeX
%%% eval: (long-s-mode)
%%% eval: (set-input-method "TeX")
%%% fill-column: 72
%%% eval: (auto-fill-mode)
%%% coding: utf-8-unix
%%% End:


  
