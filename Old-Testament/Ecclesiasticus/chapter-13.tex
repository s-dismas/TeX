%%%%%%%%%%%%%%%%%%%%%%%%%%%%%%%%%%%%%%%%%%%%%%%%%%%%%%%%%%%%%%%%%
%%%%
%%%% The (original) Douay Rheims Bible 
%%%%
%%%% Old Testament
%%%% Ecclesiasticus
%%%% Chapter 13
%%%%
%%%%%%%%%%%%%%%%%%%%%%%%%%%%%%%%%%%%%%%%%%%%%%%%%%%%%%%%%%%%%%%%%




\startcomponent chapter-13


\project douay-rheims


%%% 1526
%%% o-1408
\startChapter[
  title={Chapter 13}
  ]

\Summary{Conuerſation with the proud, rich, and potent is
  dangerous. 9.~Relie vpon Gods helpe. 11.~Beware of puſillanimitie, &
  of preſumption. 19.~A meane is neceſſarie, and the companie of equals
  is moſt ſecure.}


He
\CNote{\XRef{Deut.~7.}}
that toucheth pitch, shal be defiled with it: and he that communicateth
with the proud, shal put on pride. \V
\SNote{He that conuerſeth with a greater man then himſelf (except it be
with vertuous) is forced often to ſuffer much and to yeld to manie
inconueniences.}
He shal take a burden vpon him that communicateth with one more
honorable then himſelf. And be not companion with one richer then
thyſelf. \V What ſocietie shal the
\Fix{caudron}{cauldron}{likely typo, fixed in other}
haue with the earthen potte? for when they shal knock one againſt the
other, it shal be broken. \V The rich man hath done vniuſtly, and he wil
fume: but the poore man being hurt wil hold his peace. \V If thou geue,
he wil take thee: and if thou haue not, he wil forſake thee. \V If thou
haue, he wil liue with thee, and wil emptie thee, and he wil not be
ſorie for thee. \V If thou be neceſſarie for him, he wil ſupplant thee,
and ſmiling wil put thee in hope, telling thee good thinges, and wil
ſay: What wanteſt thou? \V And he wil confound thee in his meates, til
he emptie thee twiſe, & thriſe, and at the laſt he wil mocke thee: and
afterward ſeeing he wil forſake thee, and wil shake his head at thee. \V
Humble thyſelf to God, and expect his handes. \V Take heede leſt ſeduced
into follie thou be humbled. \V
\SNote{Puſillanimitie in a ſuperior maketh him omite his dutie, &
committe errors, fearing to do that perteyneth to his office, & which
his auctoritie requireth.}
Be not humble in thy wiſdom, leſt humbled thou be ſeduced into
follie. \V Being called of the mightier depart: for by this he wil cal
thee the more. \V Be
%%% 1527
not importune, leſt thou be reiected: and be not farre from him, leſt
thou goe into obliuion. \V Stay not to ſpeake felow-like with him:
neither credite his manie wordes. For by much talke he wil proue thee,
and ſmiling wil examine thee of thy ſecretes. \V His cruel mind wil kepe
thy wordes: and he wil not ſpare for malice, and for bandes. \V Take
heede to thyſelf, and attend diligently to thyn hearing: becauſe thou
walkeſt with thy ſubuerſion. \V But hearing thoſe thinges ſee as it were
in ſleepe, and thou shalt watch. \V Loue God al thy life, and inuocate
him for thy ſaluation. \V Euerie beaſt loueth the like to it ſelf: ſo
alſo euerie man the nereſt to himſelf. \V Al flesh wil match with the
like to it ſelf, and euerie man wil aſſociate himſelf to his like. \V If
the woolf shal at anie time communicate with the lambe, ſo the ſinner
with the iuſt. \V What fellow-shippe hath an holy man with a dogge, or
what part hath the riche with the poore? \V The wilde aſſe in the deſerte
is the lyons pray: ſo the poore are alſo the paſtures of the riche. \V
And as humilitie is abomination to the proude: ſo alſo the poore man is
the execration of the riche. \V The riche man being moued is confirmed
by his frendes: but the humble when he
%%% o-1409
is fallen, shal be thruſt out euen of his familiars. \V To the rich
deceeued there are many recouerers: he hath ſpoken proud wordes, and
they haue iuſtified him. \V The humble was deceiued, he moreouer is
rebuked alſo: he hath ſpoken wiſely, and place was not geuen vnto
him. \V
\SNote{Acceptation of perſons hindereth manie good counſels, & promoteth
manie euil thinges.}
The rich man ſpake, and al helde their peace, and they wil carry his
worde euen to the cloudes. \V The poore man ſpake and they ſay: Who is
this? and if he ſtumble, they wil ouerthrowe him. \V Subſtance is good,
to him that hath no ſinne in his conſcience: and pouertie is moſt
wicked in the mouth of the impious. \V The hart of a man altereth his
countenance, either into good, or into euil. \V The token of a good
hart, and a good countenance thou shalt hardly finde, and with labour.


\stopChapter


\stopcomponent


%%% Local Variables:
%%% mode: TeX
%%% eval: (long-s-mode)
%%% eval: (set-input-method "TeX")
%%% fill-column: 72
%%% eval: (auto-fill-mode)
%%% coding: utf-8-unix
%%% End:


  
