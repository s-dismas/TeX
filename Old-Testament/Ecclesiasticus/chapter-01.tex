%%%%%%%%%%%%%%%%%%%%%%%%%%%%%%%%%%%%%%%%%%%%%%%%%%%%%%%%%%%%%%%%%
%%%%
%%%% The (original) Douay Rheims Bible 
%%%%
%%%% Old Testament
%%%% Ecclesiasticus
%%%% Chapter 01
%%%%
%%%%%%%%%%%%%%%%%%%%%%%%%%%%%%%%%%%%%%%%%%%%%%%%%%%%%%%%%%%%%%%%%




\startcomponent chapter-01


\project douay-rheims


%%% 1509
%%% o-1393
\startChapter[
  title={Chapter 01}
  ]

\Summary{Wiſdom
\MNote{The 1.~part.

Praiſes and preceptes of vviſdom.}
procedeth from God, appeareth in his creatures, 10.~and is geuen in
competent meaſure to al that feare God, 16.~it bringeth al vertues,
27.~excludeth al vices, 33.~and is to be ſought in ſimplicitie of hart.}


Al wiſdom is of our Lord God, & hath bene alwayes with him, & is before
al time. \V The ſand of the ſea, & the droppes of rayne, & the dayes of
the world
\SNote{Mans vviſdom is not able to comprehend the vvorkes of God.}
who hath numbred? The height of heauen, and breadth of the earth, &
profunditie of the depth who hath meaſured? \V The wiſdom of God that
goeth before al thinges who hath ſearched out? \V Wiſdom was created
before al thinges, & the vnderſtanding of prudence from euerlaſting. \V
A fountayne of wiſdom the word of God on high, and the entrance therof
euerlaſting commandments. \V The roote of wiſdom to whom hath it bene
reueled, & the ſubtilties therof who hath knowen? \V The diſcipline of
wiſdom to whom hath it bene reueled, and made manifeſt, and the
multiplication of her entrance who hath vnderſtood? \V There is one moſt
high Creatour omnipotent, and mightie King, and to be feared excedingly,
ſitting vpon his throne and the God of dominion. \V He created her in
the Holie Ghoſt, and hath ſene, and
\Fix{nummbred,}{numbred,}{obvious typo, fixed in other}
and meaſured her. \V And he hath powred her out vpon al his workes, and
vpon al flesh according to his
%%% 1510
gift, and hath geuen her to them that feare him. \V The feare of our
Lord is
\SNote{Eternal glorie is the fruicte of the feare of our Lord: not that
this one vertue ſufficeth, but it is the beginning, grounded in true
faith, and bringeth forth other vertues, diuine giftes, vvith the
fruites of the Holie Ghoſt, & a ioyful crowne in the end.}
glorie, and gloriation, and ioy, and a crowne of exultation. \V The
feare of our Lord shal delight the hart, and shal geue ioy, gladnes in
length of dayes. \V With him that feareth our Lord it shal be wel in the
later end, and in the day of his death he shal be bleſſed. \V The loue
of God is honorable wiſdom. \V But they to whom she shal appeare in
viſion, they loue her in the viſion, and in the agniſing of her great
workes. \V
\CNote{\XRef{Prou.~1.}
&
\XRef{9.}}
The feare of our Lord is the begynning of wiſdom, and was created with
the faythful in the wombe, and goeth with the elect wemen, and is knowen
with the iuſt and faythful. \V The feare of our Lord is religioſitie of
knowlege. \V Religioſitie shal keepe and iuſtifie the hart, shal geue
ioy and gladnes. \V With him that feareth our Lord it shal be wel, and
in the dayes of his conſummation he shal be bleſſed. \V The fulneſſe of
wiſdom is to feare God, and fulneſſe is of the fruites therof. \V Al her
houſe she shal fil with her generations, and the ſtorehouſes with her
treaſures. \V A crowne of wiſdom, the feare of our Lord, replenishing
place, and the fruite of ſaluation: \V and he hath ſene, and numbred
her: but both are the giftes of God. \V Wiſdom shal diſtribute knowlege,
and vnderſtanding of prudence: and exalteth the glorie of them that hold
it. \V The roote of wiſdom is to feare our Lord: for the boughes therof
are of long time. \V In the treaſures of wiſdom is vnderſtanding, &
religioſitie of knowlege, but to ſinners wiſdom is abomination. \V The
feare of our Lord expelleth
%%% o-1394
ſinne: \V for he that is without feare, can not be iuſtified: for the
anger of his animoſitie, is his ſubuerſion. \V Vntil a time the patient
shal ſuſteyne, and after shal be rewarded of ioyfulnes. \V A good
vnderſtanding wil hide his wordes vntil a time, and the lippes of manie
shal shew forth his vnderſtanding. \V In the treaſures of wiſdom is
ſignification of diſcipline: \V but the worshipe of God,
\SNote{Men drowned in ſinne thincke the ſeruice of God a moſt tedious &
loathſome thing.}
is abomination to a ſinner. \V Sonne, coueting wiſdom, keepe iuſtice,
and God wil geue her to thee. \V For the feare of our Lord is wiſdom,
and diſcipline: and that which wel pleaſeth him, \V is fayth, and
meeknes, and he wil fil his treaſures. \V Be not incredulous to the
feare of our Lord: and come not to him with a duble hart. \V Be not an
hypocrite in the ſight of men, and be not ſcandalized in thy lippes. \V
Attend to them, leſt perhaps thou fal, and bring diſhonour to thy
ſoule, \V and God
%%% 1511
reuele thy ſecretes, and in the middes of the ſynagogue caſt thee
downe: \V becauſe thou cameſt to our Lord wickedly, & thy hart is ful of
guile and deceite.


\stopChapter


\stopcomponent


%%% Local Variables:
%%% mode: TeX
%%% eval: (long-s-mode)
%%% eval: (set-input-method "TeX")
%%% fill-column: 72
%%% eval: (auto-fill-mode)
%%% coding: utf-8-unix
%%% End:


  
