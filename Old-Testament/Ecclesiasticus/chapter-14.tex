%%%%%%%%%%%%%%%%%%%%%%%%%%%%%%%%%%%%%%%%%%%%%%%%%%%%%%%%%%%%%%%%%
%%%%
%%%% The (original) Douay Rheims Bible 
%%%%
%%%% Old Testament
%%%% Ecclesiasticus
%%%% Chapter 14
%%%%
%%%%%%%%%%%%%%%%%%%%%%%%%%%%%%%%%%%%%%%%%%%%%%%%%%%%%%%%%%%%%%%%%




\startcomponent chapter-14


\project douay-rheims


%%% 1527
%%% o-1409
\startChapter[
  title={Chapter 14}
  ]

\Summary{Offence of the tongue is a frequent and dangerous
  ſinne. 3.~Riches are hurtful to a couetous, and to an enuious
  mind. 11.~Workes of mercie neceſſarie, 22.~and perſeuerance in
  Wiſdom.}


Bleſſed
\CNote{\XRef{Iac.~3.}}
is the man that hath not offended in a worde out of his mouth, and is
not pricked with the ſorrow
%%% 1528
of ſinne. \V Happie is he, that hath not had heauines of his minde, and
hath not fallen from his hope. \V Subſtance is without reaſon to the
couetous man and niggard, and for the ſpiteful enuious man to what
purpoſe is gold? \V He that heapeth together from his hart vniuſtly,
gathereth for others, and in his goodes an other wil kepe riote. \V He
that is wicked to himſelfe, to what other man wil he be good? and he
shal haue no pleaſure in his goodes. \V
\SNote{He that can not afforde nouriſhment to his owne bodie by ſuch
meanes as he hath, ſinneth againſt God, abuſing his benefites, againſt
himſelf whom he vniuſtly afflicteth and againſt his neighbour whom he
ſcandalizeth.}
He that enuieth himſelfe, nothing is worſe then he, and this is the
reward of his malice: \V and if he doe good, he doth
\Fix{yt}{it}{obvious typo, fixed in other}
ignorantly, and not willing: and at the laſt he manifeſteth his
malice. \V The eye of the enuious is wicked, and turneth away his
face, and deſpiſeth his owne ſoule. \V
\CNote{\XRef{Prou.~27. v.~20.}}
The eye of the couetous man inſatiable in a portion of iniquitie, wil
not be ſatisfied til he conſume his owne ſoule withering it. \V An euil
eye is towards euil thinges: & he shal haue his fil of bread, needie &
in heauines shal he be at his table. \V Sonne if thou haue it, doe good
to thyſelfe, and offer to God worthie oblations. \V Be mindful that
death ſlacketh not, and that
\SNote{In the old teſtament al deſcended into ſome part of hel.}
the couenant of hel hath beene shewed thee: for the couenant of this
world shal dye the death. \V Before death do good to thy freind, and
according to thine abilitie ſtretching out thy hand, geue to the
poore. \V Be not defrauded of thy good day, and let not a litle portion
of a good gift ouerpaſſe
\Fix{the.}{thee.}{obvious typo, fixed in other}
\V Shalt thou not leaue to others thy ſorrowes, & labours in the deuiſion
of the lotte? \V Geue
%%% o-1410
and take, and iuſtifie thy ſoule. \V Before thy death worke iuſtice: for
in hel there can not meat be found. \V 
\CNote{\XRef{Iſa.~40. v.~7.}}
Al flesh shal waxe olde as graſſe, and as the leafe fructifying on a
greene tree. \V Some grow, and ſome are shaken of: ſo the generation of
flesh and bloude, one is ended, and an other is borne. \V Al corruptible
worke shal faile in the end: and he that worketh it shal goe
therwith. \V And
\SNote{There shal be particular reward of euerie good worke.}
euerie excellent worke shal be iuſtified: and he that worketh it, shal
be honoured therin. \V Bleſſed is the man that shal continew in wiſdom,
and that shal meditate in his iuſtice, and in vnderſtanding shal
conſider the prouidence of God. \V He that conſidereth her wayes in his
hart, and hath vnderſtanding in her ſecrets, going after her as a
ſearcher, and conſiſting in her wayes: \V He that looketh through her
windowes, and heareth in her gates: \V He that reſteth by her houſe, &
in her walles faſtening a ſtake wil ſet vp his cotage beſide her handes,
%%% 1529
and good thinges shal reſt in his cottage for euer. \V He shal ſet his
children vnder her couering, and shal abide vnder her boughes: \V he shal
be protected vnder her couering from the heate, and shal reſt in her
glorie.


\stopChapter


\stopcomponent


%%% Local Variables:
%%% mode: TeX
%%% eval: (long-s-mode)
%%% eval: (set-input-method "TeX")
%%% fill-column: 72
%%% eval: (auto-fill-mode)
%%% coding: utf-8-unix
%%% End:


  
