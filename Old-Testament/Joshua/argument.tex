%%%%%%%%%%%%%%%%%%%%%%%%%%%%%%%%%%%%%%%%%%%%%%%%%%%%%%%%%%%%%%%%%
%%%%
%%%% The (original) Douay Rheims Bible 
%%%%
%%%% Old Testament
%%%% Joshua
%%%% Argument
%%%%
%%%%%%%%%%%%%%%%%%%%%%%%%%%%%%%%%%%%%%%%%%%%%%%%%%%%%%%%%%%%%%%%%




\startcomponent argument


\project douay-rheims


%%% 0489
%%% o-0440
\startArgument[
  title={\Sc{The Argument of the Booke of Iosue.}},
  marking={Ioſue.}
  ]

VVhether
\MNote{VVhoſoeuer was author, the authoritie of this booke is certaine.}
Ioſue himſelf writ this booke (which is the
\CNote{\Cite{Hiſtor: Scholaſt.}}
common opinion) or ſome other, it was euer held vndoubtedly by al, for
Canonical Scripture: and according to the diſtribution of the whole
Bible into \Emph{Legal, Hiſtorical, Sapiential, and Prophetical} Bookes,
this is the firſt of the hiſtorical ſorte.
\MNote{Bookes of holie Scripture principally treating of ſeueral
argumentes, yet in the ſame
\Fix{participat}{participate}{likely typo, fixed in other}
ech ſorte with others.}
But as the fiue proſedent called \Emph{Legal}, beſides the \Emph{Law},
comprehend alſo the hiſtorie of the Church, from the beginning of the
world nere 2500.~yeares, and withal conteine much diuine \Emph{Wiſdome},
& \Emph{Prediction} of thinges to come: ſo theſe bookes now folowing
called \Emph{Hiſtorical}, and likewiſe the \Emph{Sapiential} and
\Emph{Prophetical} enſuing after, participate each with others in their
ſeueral argumentes: euerie one more or leſse inducing Gods ſeruantes to
keepe his \Emph{Law}; recording thinges donne; teaching what is moſt
meete to be donne; and foreshewing before hand, thinges donne afterwardes,
or which yet shal come to paſſe.
\MNote{The cõtentes of this booke.}
So this booke doth not only ſet forth
the Actes of Ioſue, who ſucceded Moyſes in tẽporal gouernment of Gods
people, commanding and directing them by lawe and Wiſedome; but alſo the
ſame \Emph{thinges donne by him, and his verie name} (as
\CNote{\Cite{S.~Hiero. Epiſt. ad Paulim.}}
S.~Hierom, &
\CNote{\Cite{S.~Amb. in Pſal.~47.}
\Cite{S.~Aug. li.~12. c.~31.}
&
\Cite{li.~16. c.~19. contra Fauſt. Manich.}}
other Fathers teach) \Emph{prefigure our Lord} \Sc{Iesvs}
\Emph{Chriſt}. For in Hebrew \Sc{Iehosva} is the name both of this
Capitaine General, the leader of The Iſraelites ouer Iordan into the
Land of promiſe, and of our Lord and \Sc{Saviovr}, who by his Baptiſme,
and other Sacramentes bringeth his people of al Nations, into the true
Land of the liuing, where is life and felicitie euerlaſting.
\MNote{Diuided into foure partes.}
Touching therfore the hiſtorie, theſe foure ſpecial thinges are here
deſcribed. Firſt, the paſſage of the Iſraelites ouer Iordan. In the fiue
firſt chapters. Secondly, their conqueſt of the promiſed Land. In the
ſeuen chapters folowing. Thirdly, the partition of the ſame Land amongſt
nine Tribes and a half. From the 13.~chap. to the~22. Fourthly, in the
three laſt chapters, the returne of the other two Tribes and a half to
their poſſeſsions, on the eaſt ſide of Iordan; with Ioſues laſt
admonition to them al, to ſerue God ſincerly; and his, and Eleazars
death.


\stopArgument


\stopcomponent


%%% Local Variables:
%%% mode: TeX
%%% eval: (long-s-mode)
%%% eval: (set-input-method "TeX")
%%% fill-column: 72
%%% eval: (auto-fill-mode)
%%% coding: utf-8-unix
%%% End:
