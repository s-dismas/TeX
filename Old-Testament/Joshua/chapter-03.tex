%%%%%%%%%%%%%%%%%%%%%%%%%%%%%%%%%%%%%%%%%%%%%%%%%%%%%%%%%%%%%%%%%
%%%%
%%%% The (original) Douay Rheims Bible 
%%%%
%%%% Old Testament
%%%% Joshua
%%%% Chapter 03
%%%%
%%%%%%%%%%%%%%%%%%%%%%%%%%%%%%%%%%%%%%%%%%%%%%%%%%%%%%%%%%%%%%%%%




\startcomponent chapter-03


\project douay-rheims


%%% 0493
%%% o-0444
\startChapter[
  title={Chapter 3}
  ]

\Summary{After three dayes abode by the bankes of Iordan, 3.~the Prieſtes
  with the Arke of God entering firſt into the riuer, 15.~the vpper part
  miraculouſly ſtandeth and ſwelleth, the lower running away, they goe
  into the middle chanel, and there ſtay, whiles al the people paſse
  ouer drie foote.}

Ioſue therfore ryſing vp in the night, remoued the campe: and departing
from Setim, they came to Iordan, he, and al the children of Iſrael, and abode
there for three dayes. \V Which being paſſed, the herauldes went through
the middes of the campe, \V and beganne to proclaime:
\SNote{In place of the cloud, and piller of fire, the arke is now caried
for their
\Fix{guid}{guide}{possible typo, fixed in other}
and direction.}
When you ſhal ſee the arke of couenant of our Lord your God, and
\SNote{It perteined to the Leuites office to carie the arke.
\XRef{Num.~4.}
but in this ſpecial ſeruice & miraculous paſſage the Prieſtes did carie
it: ſo the greater may do the office of the leſſe, not
contrariwiſe.}
the prieſtes of the Leuitical ſtocke carying it, ryſe you alſo, and
folow them going before: \V and let there be betwen you and the arke the
ſpace of two thouſand cubites: that you may ſee it a farre of, and know
which way you may goe: becauſe you haue not walked by it before: and
beware you approch not to the arke. \V And Ioſue ſaid to the people: Be
ſanctified: for to morrow our Lord wil doe among you merueilous
thinges. \V And he ſaid to the prieſtes: Take vp the arke of the
couenant, & goe before the people. Who fulfilling his commandementes,
tooke it, and walked before them. \V And our Lord ſaid to Ioſue: This
day wil I beginne to exalt thee before al Iſrael:
\SNote{God ſhewed by this miracle, that Ioſue had ſpecial commiſſion
from him, and that vnder his gouernment the people ſhould proſper.}
that they may know as I was with Moyſes, ſo am I with thee alſo. \V And
doe
%%% 0494
thou
\LNote{Command the prieſtes.}{Becauſe
\MNote{An obiection for laiheadſhip of the Church.}
Ioſue commanded the prieſtes to take the arke, and ſtand with it in
Iordan; for that alſo
\XRef{(chap.~5.)}
he miniſtred, or appointed others to miniſter, the Sacrament of
Circũciſion; likewiſe
\XRef{(chap.~8.)}
bleſſed the people; and
\XRef{(chap.~24.)}
renewed Gods couenant with them; Engliſh Proteſtantes inferre, that
he was chief ſuperior in ſpiritual cauſes; and therfore lay princes are
ſupreme heades, & gouerners of the Church immediatly vnder God.
\MNote{Anſwer.}
But none of theſe action, nor al put together do proue their
purpoſe. For notwithſtanding he very lawfully did theſe, and other like
thinges, yet he had a ſpiritual ſuperior in earth, which was Eleazar the
high prieſt.
\MNote{Moyſes chief both in ſpiritual and temporal authoritie, which was
after diuided betwen the high Prieſt & temporal Prince.}
For Moyſes being extraordinarie ſuperior of al, both in ſpiritual and
temporal cauſes, the ordinarie prieſthood, and ſpiritual ſupremacie was
eſtabliſhed in Aaron and his ſonnes, as appeareth
\XRef{Leuit.~8.}
\XRef{Num.~20.}
and other places: and the temporal gouernment after Moyſes was geuen to
Ioſue, ſucceding to him
\XRef{(Num.~27.)}
not in al, but
\CNote{\XRef{Exod.~4,~5,~6.}
&
\XRef{Deut.~17.}}
\Emph{in part of his glorie} (or authoritie) his whole honour (or power)
being diſtributed betwen the high Prieſt, and the temporal Prince, as
learned Theodoret
\Cite{(q.~48. in Num.)}
noteth vpon the ſacred text; expreſly diſtinguiſhing their offices
\XRef{(v.~21.)}
that \Emph{Eleazar the prieſt should conſult our Lord for him} (and ſo
receiue anſwer \Emph{in doctrin and veritie},
\XRef{Exod.~28.}
\XRef{Leuit.~8.)}
and that \Emph{Ioſue should goe out and goe in, and al the children of
Iſrael vvith him} (that is, lead and gouerne the people) \Emph{at
Eleazars vvord}. 
\MNote{The high prieſt ſuperiour.}
VVhere it is manifeſt that Ioſue was not ſet ouer Eleazar, but Eleazar
ouer him.
\MNote{Ioſue executed Gods wil, not by ſpiritual iuriſdictiõ, but with
ſubordination to the high prieſt.}
That therfore which Ioſue did in ſpiritual affaires, was in
ſubordination to the high prieſt; by whoſe direction, approbation, or
ratihabation, he commanded ſome of the prieſtes to carie the arke, and
with it to goe into Iordan, and coming into the midde chanel to ſtand
there, whiles al the armie and people paſſed ouer:
\CNote{\XRef{Chap.~5.}}
alſo gaue order that al ſhould be circumciſed;
\CNote{\XRef{8.}}
bleſſed the people;
\CNote{\XRef{22.}}
read the law; and after
\CNote{\XRef{23.}}
godlie exhortations,
\CNote{\XRef{24.}}
renewed the couenant betwen God and them; al in way of execution of Gods
wil & cõmandementes, not by anie pretended iuriſdiction in ſpiritual
thinges.

In
\MNote{Other good princes haue alſo much aduanced religion, but not
taken ſupremacie in ſpiritual cauſes.}
like ſorte manie other good temporal Princes, as wel in the old as the
new Teſtament, haue diſpoſed and executed diuers thinges perteining to
Gods ſeruice: their office requiring that they ſhould ſet forward,
maintaine and defend true faith and religion. Eſpecially Chriſtian
Princes, of whom Eſai prophecied
\XRef{(chap.~49.)}
that \Emph{Kinges should be foſter fathers, and Queenes the nources of
the Church}.

Conformably wherto S.~Auguſtin teacheth
\Cite{(li.~3. c.~51. cont. Creſcon)}
that Kinges, in that they are Kinges, ſerue God by commanding good
thinges, and forbidding euel, not only perteining to humaine ſocietie,
but alſo belonging to Gods religion. To this effect Conſtantin the great
did manie religious actes: yea euen thoſe thinges which our aduerſaries
wreſt to their owne ſenſe, ſhew euidently his due ſubmiſſion to his
ſpiritual paſtors. As when vrged by the Donatiſtes peruerſe
importunitie, and being deſirous (as S.~Auguſtin teſtifieth,
\Cite{Epiſt.~166.)}
to bridle ſo great impudencie, he heard and iudged Biſhop Cecilians
cauſe, after other Biſhops ſentence for him againſt the heretikes; where
he both gaue iudgement agreable to the Biſhops, and yet pleading pardon,
%%% !!! Not really a TNote
\TNote{\L{Veniam petiturus}}
excuſed himſelf for this fact. VVhich had not neded, if he had bene the
ordinarie or competent iudge. Optatus alſo writeth
\Cite{(li.~1. cont. Parmen.)}
that the ſame Emperour Conſtantin exclamed againſt the appellantes in
theſe wordes: \L{O rabida furoris audacia! ſicut in cauſis Gentilium
fieri ſolet, appellationem interpoſuerunt.} O outragious boldnes of
furie! like as in cauſes of Gentiles is wont, they haue interpoſed an
appeal. The like good offices did Iuſtinian, and Charles the great, and
manie other Chriſtian Emperours and Kinges; for which they are much
renowmed in the whole Church; and ſome haue benne honoured for their
religious zele, with glorious titles geuen to them and their
ſucceſſors.
\MNote{For maintaining Catholique religion againſt heretikes, the kings
of Spaine haue the title \Emph{Catholique}.}
To the Kinges of Spaine, from the time of Alfonſus King of Caſtil, aboue
eight hundred yeares agone, for expelling the Arians, was geuen the
title of
Fix{\Emph{Cathoque}}{\Emph{Catholique}}{obvious typo, fixed in other}
as Michael Ritius a Neapolitan writeth.
\MNote{The French Kinges, \Emph{moſt Chriſtian}.}
To the French Kinges the title of \Emph{moſt Chriſtian}, from the time
of Philip the Emperour, about 400.~yeares ſince, for expelling the
Albigenſes, as recordeth Nicholaus Gillius.
\MNote{Kinges of England, \Emph{Defenders of the faith}.}
To our King Henrie the eight of England, for his booke of the
Sacramentes againſt Luther,
%%% !!! Not really a CNote
\CNote{An. Do. 1521}
Pope Leo the tenth gaue the title: \Emph{Defender of the faith}.}
command the prieſtes, that carie the arke of the teſtament, and ſay to
them: When you shal be entred into part of the water of Iordan, ſtand in
it. \V And Ioſue ſaid to the children of Iſrael: Come hither, and heare
the word of our Lord your God. \V And againe he ſaid: In this you shal
know that our Lord the liuing God is in the middes of you, and shal
deſtroy in your ſight the Chananeite and Hetheite, the Heueite and
Pherezeite, the Gergeſeite alſo and the Iebuſeite, and the Amorrheite.
\V Behold the arke of the couenant of the Lord of al the earth shal goe
before you into Iordan. \V Prepare twelue men of the tribes of Iſrael,
one of euerie tribe. \V And when the prieſtes that carie the arke of the
Lord of the whole earth shal ſette the ſteppes of their feete in the
waters of Iordan, the waters, that are beneath, shal runne downe and
decay: and thoſe that come from aboue, shal ſtand together in one
heape. \V Therfore the people went out of their tabernacles, to paſſe
ouer Iordan: and the prieſtes, that caried the arke of the couenant,
went on before them.
%%% o-0445
\V And they being entered into Iordan, and their
feete dipped in part of the water (and Iordan in the harueſt time had
filled the bankes of his chanel) \V the waters that came downeward
ſtoode in one place, and like a mountaine ſwelling vp appeared farre
from the citie, that is called Adom to the place of Sarthan: but thoſe
that were beneth, ranne downe into the Sea of the wildernes (which now is
called the dead ſea) vntil they wholy decayed. \V And the people
went againſt Iericho: and the prieſtes that caried the arke of the
couenant of our Lord, ſtoode girded vpon the drie ground in the middes
of Iordan, and al the people paſſed ouer through the drie chanel.


\stopChapter


\stopcomponent


%%% Local Variables:
%%% mode: TeX
%%% eval: (long-s-mode)
%%% eval: (set-input-method "TeX")
%%% fill-column: 72
%%% eval: (auto-fill-mode)
%%% coding: utf-8-unix
%%% End:
