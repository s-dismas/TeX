%%%%%%%%%%%%%%%%%%%%%%%%%%%%%%%%%%%%%%%%%%%%%%%%%%%%%%%%%%%%%%%%%
%%%%
%%%% The (original) Douay Rheims Bible 
%%%%
%%%% Old Testament
%%%% Deuteronomy
%%%% Chapter 24
%%%%
%%%%%%%%%%%%%%%%%%%%%%%%%%%%%%%%%%%%%%%%%%%%%%%%%%%%%%%%%%%%%%%%%




\startcomponent chapter-24


\project douay-rheims


%%% 0464
%%% o-0417
\startChapter[
  title={Chapter 24}
  ]

\Summary{Diuorce permitted to auoide greater euil. 5.~The newly maried
  muſt not goe to warre. 7.~He that traterouſly ſelleth a man muſt be
  ſlaine. 8.~Diſobedience to Prieſtes incurreth leproſie. 10.~Such
  things may not be taken to pledge, as can not be wel ſpared. 14.~Poore
  laborers muſt be preſently payed. 16.~Not one punished for an others
  fault, but right iudgement to al, 18.~and liberal almes to the poore.}

If a man take a wife, and haue her, and she finde not grace before his
eies for
\Fix{ſonne}{ſome}{obvious typo, fixed in other}
lothſomenes: he shal write a bil of diuorce, and shal geue it in her
hand, and
\LNote{Dimiſſe her.}{VVhether
\MNote{VVhether the band of mariage could be looſed or no in the old
law, amongſt Chriſtiãs it can not be diſſolued.}
this diuorce was tollerated as a leſſe ſinne, to auoide a greater, as
S.~Hierom
\Cite{(li.~1. in Mat. c.~5. & li.~3. in c.~19.)}
S.~Chryſoſtom
\Cite{(ho.~12. in Mat.~5.)}
and others teach; or diſpenſed withal, and ſo made lawful to the Iewes,
which is alſo probable, for 
\Fix{hat}{that}{obvious typo, fixed in other}
none of the holie Prophetes did euer
reprehend it; ſure it is, that Chriſt either by correcting a fault, or
by recalling a former diſpenſation, reſtored the inſolubilitie of
mariage to the firſt inſtitution, ſaying:
\XRef{(Mat.~19.)}
\Emph{That vvhich God hath ioyned together, let not man ſeparate.}
Further anſwering the Phareſes, concerning this law: that \Emph{Moyſes
for the hardnes of your hart permitted you to dimiſse you vviues: but
from the beginning it vvas not ſo.} And albeit he alloweth ſeparation of
man and wife for fornication,
\MNote{No not for adultrie.}
yet for no cauſe neither of them can marie againe, ſo long as the other
liueth. As S.~Auguſtin
\Cite{(li.~1. de adulter. coniugijs. c.~11. &~12.)}
by conference of three Euangeliſtes wordes touching this point, plainly
ſheweth, concluding that \Emph{for ſo much as holie Scripture calleth
him} (that taketh a woman ſo dimiſſed) \Emph{not a husband, but an
adulterer, she is ſtil his vvife, by vvhom for fornication she vvas
dimiſſed.} Likewiſe he proueth by S.~Paules doctrin
\XRef{(Rom.~7.}
&
\XRef{1.~Cor.~7.)}
that though diuorce be made for adultrie, yet neither the guiltie nor
innocent partie can marie an other, for the Apoſtle ſaieth: \Emph{a
vvoman is vnder the lavv of her husband, ſo long as he liueth, if her
husband be dead, she is looſed from his lavv. Therfore her husband
liuing, she shal be called an aduoutreſſe, if she be vvith an other
man. If she part let her remaine vnmaried, or be reconciled to her
husband. A vvoman is bond to the lavv ſo long time, as her husband
liueth}, &c. Theſe wordes of the Apoſtle (ſayeth he,
\Cite{li.~2. c.~4.)}
ſo often repeted, ſo often inculcated, are true, are liuelie, are ſound,
are plaine. A woman beginneth not to be the wife of a later huſband,
except ſhe ceaſe to be the wife of the former. And ſhe ceaſeth to be the
wife of the former, if he die, not if he (or ſhe) committe adultrie.
Therfore a wife
is lawfully dimiſſed for fornication, but the bond of the former
remaineth, for which cauſe he is guiltie of adultrie, that marieth her
that is dimiſſed, yea, though it be for fornication. Thus and much more
ſayeth S.~Auguſtin in the ſame, & in other bookes. And al the ancient
fathers, and lerned ſchoolmen teach vniformly, that nothing but bodilie
death can looſe the band of Mariage conſummate;
\MNote{Only before conſummatiõ Mariage is diſſolued by ſolemne vow in
Religion.}
nor of vnconſummate, but death, or ſolemne vow in an opproued rule of
religion.}
dimiſſe her out of his houſe. \V And being departed when she shal haue
married an other husband, \V and he alſo hateth her, and hath geuen her
a bil of diuorce, and hath dimiſſed her out of his houſe, or is deade: \V
the former husband can not take her againe to wife: becauſe she is
polluted, and is made abominable before our Lord: leſt thou make thy
Land to ſinne, which our Lord thy God shal deliuer thee to poſſeſſe. \V
When a man hath lately taken a wife, he shal not goe forth to battel,
neither shal any publique neceſſitie be inioyned him, but he shal attend
to his owne houſe without fault, that one yeare he may reioyce with his
wife. \V Thou shalt not take for a pledge the nether, or the vpper
milſtone: becauſe
\SNote{This hebrew phraſe ſignifieth, that pledging the thing wherin the
meanes of life conſiſteth is as if he pledged his life.}
he hath pledged his life to thee. \V If any man be taken ſoliciting his
brother of the children of Iſrael, and ſelling him take a price, he shal
be ſlaine, and thou shalt take away the euil from the middes of thee. \V
Obſerue diligently that thou incurre not the plague of leproſie, but
thou shalt doe whatſoeuer the prieſtes of the Leuitical ſtocke shal
teach thee, according to that, which I haue commanded them, and fulfil
thou it carefully. \V Remember what our Lord your
%%% 0465
God did to Marie, in the way when you came out of Ægypt. \V When thou
ſhalt require of thy neighbour any thing, that he oweth thee, thou ſhalt
not enter into his houſe to take away a pledge: \V but thou ſhalt ſtand
without, and he shal bring forth to thee that which he hath. \V But if
he be poore, the pledge shal not lodge with thee that night, \V but
forthwith thou shalt reſtore it to him before the going downe of the
ſunne: that ſleeping in his rayment, he may bleſſe thee, & thou mayeſt
haue iuſtice before our Lord thy God. \V Thou shalt not denie the hyre
of the needie, and poore man thy brother, or the ſtranger, that
%%% o-0418
dwelleth with thee in the land, and is within thy gates: \V but the
ſame day thou shalt pay him the price of his labour, before the going
downe of the ſunne,
\SNote{In caſe the laborer ſuſteyneth his life by his dailie wages, then
not to pay him is in effect to kil him: and ſuch ſinne crieth to God for
reuenge.}
becauſe he is poore, and there withal ſuſteyneth his life: leſt he crie
againſt thee to our Lord, and it be reputed to thee for a ſinne. \V The
fathers shal not be ſlaine for the children, nor the children for the
fathers, but euerie one shal die for his owne ſinne. \V Thou shalt not
peruert the iudgement of the ſtranger and the pupil, neither shalt thou
take away the widowes rayment for a pledge. \V Remember that thou didſt
ſerue in Ægypt, and our Lord thy God deliuered thee from
thence. Therfore I command thee that thou doe this thing. \V When thou
haſt reaped the corne in thy field, and forgetting haſt left a ſheafe,
thou shalt not returne to take it away: but thou shalt ſuffer the
ſtranger, and the pupil, and the widow to take it away, that our Lord
thy God may bleſſe thee in al the worke of thy handes. \V If thou haue
gathered the fruites of thy oliue trees, whatſoeuer remaineth on the
trees, thou shalt not returne to gather it: but shalt leaue it to the
ſtranger, the pupil, and the widow. \V If thou make vintage of thy
vineyard, thou shalt not gather the cluſters that remaine, but they shal
goe to the vſes of the ſtranger, the pupil, and the widow. \V Remember
that thou alſo didſt ſerue in Ægypt, and therfore I command thee that
thou doe this thing.


\stopChapter


\stopcomponent


%%% Local Variables:
%%% mode: TeX
%%% eval: (long-s-mode)
%%% eval: (set-input-method "TeX")
%%% fill-column: 72
%%% eval: (auto-fill-mode)
%%% coding: utf-8-unix
%%% End:
