%%%%%%%%%%%%%%%%%%%%%%%%%%%%%%%%%%%%%%%%%%%%%%%%%%%%%%%%%%%%%%%%%
%%%%
%%%% The (original) Douay Rheims Bible 
%%%%
%%%% Old Testament
%%%% Deuteronomy
%%%% Argument
%%%%
%%%%%%%%%%%%%%%%%%%%%%%%%%%%%%%%%%%%%%%%%%%%%%%%%%%%%%%%%%%%%%%%%




\startcomponent argument


\project douay-rheims


%%% 0418
%%% o-0375
\startArgument[
  title={\Sc{The Argvment of Devteronomie.}},
  marking={The Argument of Deuteronomie.}
  ]

Deuteronomi, 
\MNote{This booke is a repetition, explication, and ſuplement of the
Law.}
in English \Emph{The ſecond law}, ſo called not that there be two lawes
of Moyſes,
\CNote{\Cite{S.~Aug. q.~49. in Deuteron.}}
but becauſe the ſame which was firſt geuen in Mount Sinai, fiftie dayes
after the children of Iſrael parted from Ægypt, is here repeted, in the
eleuenth moneth of the fourtith yeare of their abode in the deſert. In
which repetition albeit Moyſes explicateth the ſame law, adding alſo
diuers things not expreſſed before: yet it is but an Abridgement
conceiued and vttered in fewer wordes. VVhereupon S.~Bede
%%% !!! Should this Cite and CNote be combined? They are separate in
%%% both.
\Cite{(in princ. Leuit.)}
\CNote{\Cite{et princ. Deutero.}}
compareth this booke with the foure precedent, as one made of them
al.
\MNote{It prefigured the Goſpel.}
For wheras \Emph{the former foure prefigured the foure Goſpels, this
ſignified the whole Goſpel}, contained in al foure. Likewiſe S.~Hierom
calleth it \Emph{A prefiguration of the Euangelical law: ſo iterating
former things, that al become new of old.}
%%% !!! Should these Cites and CNote be combined? They are separate in
%%% both.
\Cite{(Epiſt ad Paulim. ca.~7.}
&
\Cite{de Mans.~42.)}
\CNote{\Cite{Manſ. vlt.}}
But touching the literal ſenſe, Moyſes here compriſeth foure general
things: vnto which after his death the fifth is added;
\MNote{Conteineth fiue partes.}
and ſo the whole conteineth fiue partes.
Firſt, 
\CNote{\XRef{Chap.~1.}}
he briefly reciteth Gods ſpecial benefites beſtowed on this people, and
their ingratitude, incredulitie, murmurings, and punishments. In the
three firſt chapters.
Secondly he repeteth and explicateth Gods precepts, 
\CNote{\XRef{4.}}
moral, ceremonial, and iudicial, 
\CNote{\XRef{12.}}
with the functions and offices of Prieſts, and Leuites. From the
4.~chap. to the 27. Thirdly 
\CNote{\XRef{27.}}
he denounceth Gods promiſes of manie bleſſings, and thretes of
punishments, for keeping or breaking his commandments. From the
27.~chap. to~31. Fourthly
\CNote{\XRef{31.}}
he exhorteth them to ſerue and loue God, but withal fortelleth, that
they wil often fal to great ſinnes, and for the ſame shal be punished,
and at laſt forſaking Chriſt, shal be forſaken: yet finally bleſſeth
their tribes, in figure of the Gentiles, that shal be called in their
place. Chap.~31.~32. and~33.
\CNote{\XRef{34.}}
Fiftly, in the laſt chapter, Ioſue writeth the death, burial, and
ſingular commendation of Moyſes.


\stopArgument


\stopcomponent


%%% Local Variables:
%%% mode: TeX
%%% eval: (long-s-mode)
%%% eval: (set-input-method "TeX")
%%% fill-column: 72
%%% eval: (auto-fill-mode)
%%% coding: utf-8-unix
%%% End:
