%%%%%%%%%%%%%%%%%%%%%%%%%%%%%%%%%%%%%%%%%%%%%%%%%%%%%%%%%%%%%%%%%
%%%%
%%%% The (original) Douay Rheims Bible 
%%%%
%%%% Old Testament
%%%% Canticle
%%%% Argument
%%%%
%%%%%%%%%%%%%%%%%%%%%%%%%%%%%%%%%%%%%%%%%%%%%%%%%%%%%%%%%%%%%%%%%




\startcomponent argument


\project douay-rheims


%%% 1468
%%% o-1355
\startArgument[
  title={\Sc{The Argvment of the Canticle of Canticles}.},
  marking={Canticle}
  ]


Salomon,
\MNote{King Salomon according to his three names writte and intitled his
three bookes.}
called alſo \Emph{Eccleſiaſtes}, and \Emph{Idida}, according to theſe
three names (as
\CNote{\Cite{Proem. in Eccle.}}
S.~Ierom noteth) writte three bookes of three particular arguments,
directed to three degrees of people, with three diſtinct titles, al
tending to one end, the true ſeruice of God, which bringeth to eternal
felicitie. In the firſt he teacheth the principles of good life, to flee
from vices, and folow vertues: belonging to ſuch as \Emph{beginne} to
obſerue Gods law, wherin true wiſdom conſiſteth: and this booke is
called the \Emph{Prouerbes}, or \Emph{Parables}, that is to ſay, Pithie,
brief, ſentencious precepts;
\MNote{Salomon Pacifier king of Iſrael.}
of \Emph{Salomon}, which ſignifieth \L{Pacificus}, \Emph{Peaceable}, or
Pacifier: \Emph{the ſonne of Dauid, King of Iſrael}. In the ſecond he
exhorteth to contemne this world, shewing that true felicitie conſiſteth
not in anie worldlie or temporal thinges, but in the eternal fruition of
God, which is obtayned by keping his commandments. And this booke he
intitleth:
\MNote{Eccleſiaſtes, Preacher king of Ieruſalem.}
\Emph{The wordes of Eccleſiaſtes}, which is \L{Concionator}, Preacher,
\Emph{Sonne of Dauid, King of Ieruſalem}, becauſe he there exhorteth
ſuch as haue made ſome progreſſe in vertues, called \L{Proficientes},
ſignified by the inhabitants of the Metropolitan citie Ieruſalem;
whereas in the former he ſtiled himſelf king of Iſrael, propoſing
precepts mete for al
%%% 1469
the twelue tribes, and al vulgar men deſirous and beginning to ſerue
God. In both bookes, for more auctoritie ſake, making mention of his
godlie renowmed father the Royal Prophet Dauid, with his owne title alſo
of king.
\MNote{Idida, Beloued.}
But in this third booke he only expreſſeth his proper name Salomon, whom
God ſingularly loued, wherof he was called \Emph{Idada}. Becauſe this
alone, without mention of father or king, was moſt conuenient for the
\Emph{Perfect}, who not as ſeruants, or yong ſcholars are moued by feare
of auctoritie, but as children are ſwetly drawne by loue.
\MNote{This Canticle doth excel other Canticles.}
And this he writte in verſe, intitling it not ſimply a Canticle, but
\Emph{The Canticle of Canticles}, as preeminent aboue other
Canticles. The bridal ſongue for the Mariage, to be ſolemnized betwen
God himſelf and his glorious ſpouſe.
\MNote{Al are not mete to read it.}
For though al holie Scriptures are the ſpiritual bread, and food of the
faithful, yet al are not meate for al, at al ſeaſons. Some parts are not
for ſinners, nor for beginners, nor for ſuch as are yet in the way
towards perfection, but only for the perfect. According to the Apoſtles
doctrine:
\CNote{\XRef{Heb.~5.}}
\Emph{Milke is for children, that are yet vnskilful of the word of
iuſtice. But ſtrong meate is for the perfect, them that by cuſtom, haue
their ſenſes exerciſed to the diſcerning of good and euil.} With what
moderation therfore, and humilitie, this Canticle of Gods perfect ſpouſe
may be read, the diſcrete wil conſider, and not preſume aboue their
reach, but be wiſe with ſobrietie. For here be very high and hidden
Myſteries, as Origen teacheth in his lerned Commentaries (which S.~Ierom
tranſlated into Latin, and ſingularly commendeth) and ſo much harder to
be rightly vnderſtood, for
%%% o-1356
that the feruent ſpiritual loue, of the inward man, reformed in ſoule,
and perfected in ſpirite, is here vttered in the ſame vſual wordes and
termes, wherwith, natural, worldlie, yea and carnal loue of the outward
man, old Adam, corrupted by ſinne, is commonly expreſſed: and are ſo
much more dangerous to be miſtaken, as we are more addicted to proper
wil, & priuate iudgement, or ſubiect to carnal, or paſſionate
motions.
\MNote{Beſt methode in lerning is to beginne with doctrine of good life,
then ſtudie to know natural thinges: and finally contemplate diuine
myſteries.}
Wherfore it ſemeth moſt mete to kepe the ſame order in reading theſe
three bookes, which the auctor wiſe Salomon obſerued in writing
them. And which Philoſophers alſo folow in their forme of
diſcipline. For they firſt lerne and teach Moral Philoſophie, then
Natural, & laſtly Metaphiſikes which is their Diuinitie. As Salomon had
geuen them example: firſt teaching \Emph{precepts of good life}, and
maners, in his \Emph{Prouerbes}: after, diſcourſing of natural thinges,
\Emph{in Eccleſiaſtes}, deduced thence a concluſion, which prophane
Philoſophers wel vnderſtood not, \Emph{to contemne this world}: and
finally cometh to high myſtical Diuinitie, in this ſupereminent
\Emph{Canticle: written} in an other ſtile, \Emph{in verſe}, and
\MNote{A ſacred dialogue or Enterlude.}
\Emph{in forme of a ſacred Dialogue} betwen Chriſt and his ſpouſe: or as
Origen calleth it, in forme of an
\TNote{\L{Formæ dramatis}.}
\Emph{Enterlude}, in reſpect of diuers ſpeakers & actors, & of diuers
perſons, to whom the ſpeaches are directed, and of whom they are
vttered.
\MNote{God & Chriſt the Spous, or Bridgrome.}
For \Emph{by
%%% 1470
the Spous or Bridgrome}, is not only vnderſtood Chriſt as Man, but alſo
as God, and the whole Bleſſed Trinitie; to whom manie prayers, praiſes,
and thankes are offered vp: and by whom manie benefites are geuen,
praiſes returned, & promiſes made to his ſpouſe.
\MNote{Three ſpouſes.}
Likewiſe by the \Emph{Spouſe} or \Emph{Bride}, the ancient fathers
vnderſtood three ſortes of ſpouſes: al eſpouſed to Chriſt, and to God:
to witt,
\MNote{The General.}
his \Emph{General Spouſe}, the whole Church of the old and new
Teſtaments; of al that are, and shal be perfect, making one
\CNote{\XRef{Epheſ.~5.}}
myſtical bodie, free from ſinne, without ſpotte, or wrinkle, ſanctified
in Chriſt. Alſo his
\MNote{The Special,}
ſpecial ſpouſe, which is euerie particular holie ſoule. And his
\MNote{and

Singular.}
ſingular ſpouſe, his moſt bleſſed & moſt immaculate Virgin Mother. This
being the general ſumme of this excellent Canticle, remitting the
reader, for explication therof to the
\CNote{\Cite{Origen.}
\Cite{S.~Ierom.}
\Cite{S.~Aug. lib.~8. de Gen. li.}
\Cite{S.~Greg.}
\Cite{S.~Beda.}
\Cite{S.~Tho.}
\Cite{Arbor. Geneb. Del Rio.}}
lerned deuout Commenters, both of ancient and late writers, we shal alſo
endeuour to gether the ſame contents more particularly, not before the
chapters, becauſe we can not there ſo conueniently diſtinguish the ſame
by verſes, but
\MNote{The particular contents are ſette in the margent of euerie
chapter.}
in the margent. Where we shal eſpecially note the ſpeakers, as ſemeth
more probable of euerie parcel, according to the firſt ſenſe (not hauing
rowme for more) perteyning to the General ſpouſe, the Catholique Church:
which is the great, and euerlaſting holie Citie of God the eternal
King.



\stopArgument


\stopcomponent


%%% Local Variables:
%%% mode: TeX
%%% eval: (long-s-mode)
%%% eval: (set-input-method "TeX")
%%% fill-column: 72
%%% eval: (auto-fill-mode)
%%% coding: utf-8-unix
%%% End:
