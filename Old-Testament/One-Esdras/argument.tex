%%%%%%%%%%%%%%%%%%%%%%%%%%%%%%%%%%%%%%%%%%%%%%%%%%%%%%%%%%%%%%%%%
%%%%
%%%% The (original) Douay Rheims Bible 
%%%%
%%%% Old Testament
%%%% One Esdras
%%%% Argument
%%%%
%%%%%%%%%%%%%%%%%%%%%%%%%%%%%%%%%%%%%%%%%%%%%%%%%%%%%%%%%%%%%%%%%




\startcomponent argument


\project douay-rheims


%%% 0963
%%% o-0865
\startArgument[
  title={\Sc{The Argvment of the Bookes of Esdras.}},
  marking={Arguments of the Bookes of Esdras.}
  ]

Esdras
\CNote{\XRef{1.~Eſdr.~7.}}
\MNote{The two bookes of Eſdras and Nehemias are but one in the Hebrew.}
a holie Prieſt and Scribe, of the ſtocke of Aaron, by the line of
Eleazar, vvriteth the hiſtorie of Gods people, in, and preſently after
their captiuitie in Babilon: vvhich \Emph{Nehemias} an other godlie
Prieſt proſecuteth, vvhoſe booke is alſo called the ſecond of Eſdras,
becauſe in the Hebrevv and Greke they are but one booke, relating the
acts of them both.
\MNote{The third and fourth are not canonical.}
The other two books called \Emph{the third and fourth} of Eſdras,
touching the ſame matter, are not in the Hebrew, nor \Emph{receiued into
the Canon of holie Scripture}, though the Greke Church hold the third
booke as Canonicall, and placeth it firſt, becauſe it conteyneth thinges
donne before the other.

In
\MNote{This hiſtorie hath alſo a ſpiritual ſenſe.}
the two here folowing, vvhich are vndoubtedly holie Scripture
\CNote{\Cite{Epiſt. ad Paulin.}}
S.~Ierom ſayth, that \Emph{Eſdras and Nehemias (to witte the Helper, and
Comforter from God) reſtored the Temple, and built the walles of the
citie}; adding that \Emph{al the troope of the people returning into
their countrie, alſo the deſcription of Prieſtes, Leuites, Iſraelites,
Proſelites, and the workes of walles and towres diuided by ſeueral
families}, \L{aliud in cortice præſerunt, aliud in medulla retinent},
\Emph{ſhew one thing in the barke, kepe an other thing in the marrow}:
ſignifying that this hiſtorie hath both a literal, and a myſtical
ſenſe.
\MNote{Firſt booke diuided into two partes.}
According to the letter, this firſt booke shevveth the reduction of Gods
people from Babylon, in the firſt ſix chapters. In the other foure,
their inſtruction by Eſdras after their returne.


\stopArgument


\stopcomponent


%%% Local Variables:
%%% mode: TeX
%%% eval: (long-s-mode)
%%% eval: (set-input-method "TeX")
%%% fill-column: 72
%%% eval: (auto-fill-mode)
%%% coding: utf-8-unix
%%% End:
