%%%%%%%%%%%%%%%%%%%%%%%%%%%%%%%%%%%%%%%%%%%%%%%%%%%%%%%%%%%%%%%%%
%%%%
%%%% The (original) Douay Rheims Bible 
%%%%
%%%% Old Testament
%%%% Judith
%%%% Argument
%%%%
%%%%%%%%%%%%%%%%%%%%%%%%%%%%%%%%%%%%%%%%%%%%%%%%%%%%%%%%%%%%%%%%%




\startcomponent argument


\project douay-rheims


%%% 1030
%%% o-0926
\startArgument[
  title={\Sc{The Argvment of the Booke of Ivdith.}},
  marking={Argument of the Booke of Iudith.}
  ]

S.~Ierom
\CNote{\Cite{Epiſt.~111.}
\Cite{115. in Prefat. in Iudith.}}
\MNote{S.~Ierom for the authoritie of the councel of Nice, held this
booke to be canonical, which before he did not.}
ſometime ſuppoſed this booke, not to be canonical, but afterwarde
finding that \Emph{the Councel of Nice accounted it in the number of
holie Scriptures}, he ſo eſtemed it; and therupon not only tranſlated it
into Latin, out of the Chaldee tongue, wherin it was firſt written, but
alſo as occaſion required, alleaged the ſame as diuine Scripture, and
ſufficient to conuince matters of faith in controuerſie. For otherwiſe
his oppoſing the authoritie of the Nicen Councel, should proue nothing
at al againſt \Emph{the Iewes}, ſeing they alſo \Emph{acknowledge this
booke amongſt Agiographa} (or holie writtes) \Emph{but leſſe fitte} (ſay
they) \Emph{to ſtreingthen thoſe thinges which come into
contention}. Wherby is clere that S.~Ierom thenceforth held it for
diuine Scripture. As further appeareth in his
%%% o-0927
comentaries
\Cite{in Iſai.~14.}
more expreſly
\Cite{Epiſt. ad Principiam},
he counted it in ranke with other Scriptures, wherof none doubteth,
ſaying: \Emph{Ruth, Eſther, Iudith}, were of ſo great renoume, that they
\Emph{gaue the names to ſacred volumes}. And in
\Cite{this Preface}
doubted not to ſay: that \Emph{the rewarder of Iudithes chaſtitie} (God
himſelf) \Emph{gaue her for imitation not only to wemen, but alſo to
men: gaue her ſuch vertue that she ouerthrew him, whom none
could ouercome, and conquered the inuincible}.
\MNote{Other Doctors both before and after the councel of Nice accounted
this booke canonical.}
Alſo before the Councel, Origen
\Cite{in c.~14. Iudith.}
Tertullian
\Cite{de Monogamia c.~vlt.}
And
%%% 1031
diuers whom S.~Hilarie citeth, and diſſenteth not from them,
\Cite{Prologo in Pſalmos,}
\Emph{held this booke for Canonical}. Manie more
\Fix{writes}{writers}{obvious typo, fixed in other}
likwiſe about the time of the ſame Councel, and after ſo account
it. Prudentius
\Cite{in Phycomachia Prudicitiæ & libidinis:}
Chromatius
\Cite{in c.~6. Mat.}
Paulinus
\Cite{in Natali.~10.}
S.~Chryſoſtom
\Cite{hom.~10. in Math.}
S.~Ambroſe
\Cite{li.~3. Offic. c.~13.}
\Cite{Epiſt.~82.}
et
\Cite{li. de viduis.}
S.~Auguſtin (or ſome other good author) writte two ſermons of Iudith
\Cite{228.}
\Cite{229.}
Caſſiodorus
\Cite{diuini lect. c.~6.}
Fulgentius
\Cite{Epiſt.~2. de ſtatu viduarum.}
Ferrandus Carthaginenſis
\Cite{as Regiũ de re militati.}
Iunilius Africanus
\Cite{li.~1. de partibus diuine legis.}
Sulpitius
\Cite{in hiſtoria.}
S.~Beda
\Cite{de ſex ætatibus.}
Alredus writing
\Cite{the life of Edward our king.}
More are not neceſſarie to reaſonable men.
Concerning the time, and author,
\MNote{VVhen this hiſtorie happened.}
it ſemeth moſt probable that theſe thinges happened when
\CNote{\XRef{2.~Para.~33.}}
Manaſſes king of Iuda was either in priſon in Babylon, or newly reſtored
to his kingdom, who as it ſemeth permitted the gouerment to the high
Prieſt Eliachim
\XRef{(Chap.~4.)}
otherwiſe called Ioachim.
\XRef{(ch.~15.)}
\MNote{And by whom it was written.}
Who alſo writte this booke, as Philos
\Cite{Chronologie li.~2.}
reporteth. From which time they had no warres til the reigne of Ioachaz,
about 80.~yeares, conformable to the long peace mentioned.
\XRef{chap.~16. v.~30.}
\MNote{The cõtentes.}
In ſumme we haue here, not a poetical Comedie (as Martin Luther shameth
not to cal it, in
\Cite{Sympoſiates, c.~29.}
and in his
\Cite{German Preface of Iudith,}
but a ſacred Hiſtorie (as al aforementioned eſtemed it, and the Iewes
confeſſe) of a moſt valiant Matrons fact, deliuering the people of God
from perſecution of a cruel
\Fix{Tyranne.}{Tyrant.}{obvious typo, fixed in other}
\MNote{Diuided into foure partes.}
The firſt three chapters shew the occaſion of this danger: the next
foure deſcribe the difficulties & diſtreſſes therof: other ſeuen with
part of the 15. how Iudith deliuered them from it. In the reſt Iudith is
much prayſed, and she with the whole people praiſe God.


\stopArgument


\stopcomponent


%%% Local Variables:
%%% mode: TeX
%%% eval: (long-s-mode)
%%% eval: (set-input-method "TeX")
%%% fill-column: 72
%%% eval: (auto-fill-mode)
%%% coding: utf-8-unix
%%% End:
