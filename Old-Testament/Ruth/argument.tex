%%%%%%%%%%%%%%%%%%%%%%%%%%%%%%%%%%%%%%%%%%%%%%%%%%%%%%%%%%%%%%%%%
%%%%
%%%% The (original) Douay Rheims Bible 
%%%%
%%%% Old Testament
%%%% Ruth
%%%% Argument
%%%%
%%%%%%%%%%%%%%%%%%%%%%%%%%%%%%%%%%%%%%%%%%%%%%%%%%%%%%%%%%%%%%%%%




\startcomponent argument


\project douay-rheims


%%% 0584
%%% o-0524
\startArgument[
  title={\Sc{The Argvment of the Booke of Rvth.}},
  marking={Rvth.}
  ]

Amongſt
\MNote{The hiſtorie of Ruth is regeſtred in holie Scripture, for the
genealogie of Dauid, and eſpecially of our Sauiour Chriſt.}
other thinges that happened to the people of Iſrael, in the time of the
Iudges, this hiſtorie of Ruth, to witt, her coming from Moab, her
conuerſion to true Religion, godlie conuerſation, and mariage with Booz
of the tribe of Iuda, is recorded, as a more principal matter. For that
not onlie king Dauid, but conſequently alſo our \Sc{Saviovr}, the
Redemer of mankind deſcended from her. VVherby was foreſigned, that as
ſaluation thus proceded from the Gentiles together with the Iewes:
ſo the Gentiles are made partakers of the ſame grace. More clerly
prophecied, as S.~Hierom noteth, by Iſai
\XRef{(cap.~16.)}
ſaying: \Emph{Send forth ô Lord the lambe, the Ruler of the earth, from
the Rocke of the deſert to the mount of the daughter of Sion.} That is,
from Ruth the gentile to Hieruſalem, or rather to the Church.
\CNote{\XRef{Iudic.~12.}}
This mariage of Ruth came to paſſe about the time of Abeſan Iudge. The
booke was written, as is moſt probable, by Samuel: and is diuided into
foure chapters; whoſe contentes folow in their places.


\stopArgument


\stopcomponent


%%% Local Variables:
%%% mode: TeX
%%% eval: (long-s-mode)
%%% eval: (set-input-method "TeX")
%%% fill-column: 72
%%% eval: (auto-fill-mode)
%%% coding: utf-8-unix
%%% End:
