%%%%%%%%%%%%%%%%%%%%%%%%%%%%%%%%%%%%%%%%%%%%%%%%%%%%%%%%%%%%%%%%%
%%%%
%%%% The (original) Douay Rheims Bible 
%%%%
%%%% Old Testament
%%%% Genesis
%%%% Argument
%%%%
%%%%%%%%%%%%%%%%%%%%%%%%%%%%%%%%%%%%%%%%%%%%%%%%%%%%%%%%%%%%%%%%%




\startcomponent argument


\project douay-rheims


%%% 0019
%%% o-0017
\startArgument[
  title={\Sc{The Argvment of the Booke of Genesis.}},
  marking={The Argument of the Booke of Genesis.}
  ]

This
\MNote{Geneſis written by Moyſes.}
firſt Booke of holie
\Fix{Sripture,}{Scripture,}{obvious typo, fixed in other}
called Geneſis, which ſignifieth \Emph{birth} or \Emph{beginning}, was
written by Moyſes, when he was deſigned by God, to inſtruct & rule the
children of Iſrael. As alſo the other foure bookes folowing.
\MNote{Alwayes authentical.}
\MNote{So knowen by Tradition, confirmed by Chriſt. Alleaged alſo by the
Apoſtles.}
The Author and authoritie of al which fiue bookes, were euer acknowledged by
the faithful, both of the old and new Teſtament: and ſo accounted and
eſteemed by tradition, til Chriſt and his Apoſtles:
\CNote{Mat.~19. Heb.~11. Iacob.~2. 1.~Pet.~2. 2.~Pet.~2.}
who alſo confirmed
them by their teſtimonies, and allegations of the ſame, as of holie
Scriptures. From the creation vntil Moyſes writ (which was aboue two
thouſand and foure hundred yeares) 
\MNote{Religion reueled to ſpecial perſons, and ſo obſerued by
Traditions.}
the Church exerciſed Religion by Reuelations made to certaine
Patriarches, and by Traditions from man to man, without anie Scriptures
or Law written.
\MNote{VVhy Scripture was written.}
But the peculiar people of God being more viſibly
ſeparated from other nations, & manie errors abunding in the world,
\CNote{S.~Aug. queſt. vet. & noui Teſtam. cap.~3.}
God
would for correction & confutation therof, haue his wil made further
knowen to his children, and ſo remaine amongſt them in written record,
by his faithful ſeruant and Prophet Moyſes.
\MNote{VVhat Moyſes ſpecially sheweth in this booke.}
VVho therfore declareth the Author and beginning of al thinges, that is,
How al creatures were made by God, and of him haue their being, and by
him only are conſerued. He teacheth expreſly that there is one onlie
God, againſt thoſe that imagined and brought into the phantaſies of men
manie goddes. That the whole or vniuerſal ſubſtance of heauen and earth,
with their ornaments and accidents, were made in time; againſt thoſe
that thought the firſt fundation therof had euer benne. That God doth
gouerne the ſame; againſt thoſe that ſay, al is ruled by deſtenie or by
the ſtarres, and not by the continual prouidence of God. That God is a
rewarder of the good, and a punisher of euil; which ſinners ſeme either
not to know, or groſly to forgete. And that God created al for mans vſe
and benefite, which should make vs grateful.
\MNote{Man moſt particularly deſcribed.}
VVherfore holie Moyſes more particularly deſcribeth the beginning of man;
what he was at firſt; how he fel; how al mankind is come of one man:
\MNote{The right line from Adam to Noe.}
deducing the Genealogie of Adam, eſpecially to Noe.  Then how men being
more and more defiled vpon the earth, with wicked, eſpecially carnal
ſinnes, were by Gods iuſt wrath drowned with an vniuerſal floud.

Againe how a few reſerued perſons multiplied the world anew. But this
ofspring alſo falling into manie ſinnes, eſpecially Idolatrie and
ſpiritual fornication, as thoſe of the firſt age did to carnal offences,
God ſtil conſerued ſome faithful & true ſeruants.
\CNote{Gen.~10.}
Of which Moyſes
ſpecially purſueth the line of Noe by Sem his firſt begotten ſonne.
\MNote{The principal Patriarches from Noe to the 12.\ ſonnes of Iſrael.}
Then deſcribeth the particular vocations, liues, maners, notable ſayings,
and noble factes, with ſincere religion of Abraham, Iſaac, Iacob, Ioſeph, &
other holie Patriarches: who liued before the written lavv. Likevviſe
vpon vvhat occaſion, & in vvhat
%%% 0020
maner, Iacob otherwiſe called Iſrael, with al his progenie, deſcended
from the Land of Canaan into Ægypt, and were there entertayned. So this
booke contayneth the hiſtorie of two thouſand and three hundred & odde
yeares.
\MNote{This booke diuided into eight partes.}
And it may be diuided into eight partes.
\MNote{1.}
The firſt contayneth the
%%% o-0018
Creation of Heauen and Earth, & other Creatures, and laſtly of
Man. chap.~1.\ &~2.
\MNote{2.}
The ſecond part is of the tranſgreſsion & fal of man, & his caſting out
of Paradiſe, of multiplication of men, and of ſinne, though ſtil ſome
were iuſt, of the general floud, that drowned al except eight perſons, &
few other liuing creatures of the earth. from the third chap.\ to the~8.
\MNote{3.}
The third part is of the new increaſe, & multiplication of the
ſame. from the 8.~chap.\ to the 11.
\MNote{4.}
The fourth, of the confuſion of tongues, & the diuiſion of nations. in
the 11.~chap.
\MNote{5.}
The fift relateth Abrahams going forth of his countrie, Gods promiſe,
that in his ſeede al Nations should be bleſſed, & the commandment of
Circumciſion, from the 12.~chap.\ to the 21.
\MNote{6.}
The ſixth part recounteth the progenie, and other bleſsings, eſpecially
the great vertues of Abraham, Iſaac, and Iacob. from the 21.~chap.\ to
the~37.
\MNote{7.}
The ſeuenth part reporteth the ſelling of Ioſeph into Ægypt, and his
aduancement there. from the 37.~chap.\ to the~46.
\MNote{8.}
The eight and laſt part is of Iacob, and his progenies going into Ægypt,
their intertainment there, and of Iacobs, and finally of Ioſephs
death. in the fiue laſt chapters.


\stopArgument


\stopcomponent


%%% Local Variables:
%%% mode: TeX
%%% eval: (long-s-mode)
%%% eval: (set-input-method "TeX")
%%% fill-column: 72
%%% eval: (auto-fill-mode)
%%% coding: utf-8-unix
%%% End:
