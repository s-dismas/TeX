%%%%%%%%%%%%%%%%%%%%%%%%%%%%%%%%%%%%%%%%%%%%%%%%%%%%%%%%%%%%%%%%%
%%%%
%%%% The (original) Douay Rheims Bible 
%%%%
%%%% Old Testament
%%%% Genesis
%%%% Chapter 06
%%%%
%%%%%%%%%%%%%%%%%%%%%%%%%%%%%%%%%%%%%%%%%%%%%%%%%%%%%%%%%%%%%%%%%




\startcomponent chapter-06


\project douay-rheims


%%% 0040
%%% o-0037
\startChapter[
  title={Chapter 6}
  ]

\Summary{Mans ſinnes cauſe of the deluge. 4.~Giants vvere then vpon the
  earth. 8.~Noe being iuſt vvas commanded to build the Arke, 18.~vvherin
  he vvith ſeuen perſons more, and the ſeede of other liuing things
  vvere ſaued.}

And after that men began to be multiplied vpon the earth, & had
procreation of daughters: \V The
\SNote{The profeſſors of true religion were called the ſõnes of God, the
folowers of errors the ſõnes of men.}
%%% !!! LNote not marked in text of either
\LNote{Sonnes of God.}{The
\CNote{lib.~15. c.~23. ciuit. De vera relig. c.~7. con
epiſt. fund. c.~4. Trac.~32. in Ioan.}
\MNote{Sõnes of God and ſonnes of men was then ſuch a diſtinction, as
now Catholiques and Heretikes.}
progenie of Seth, profeſſing true faith & Religion, were called the
ſonnes of God: and thoſe of Cains iſſue and congregation, folowing
erronious and wicked opinions, were called the ſonnes of men. VVhich
were then the diſtinctiue termes of true and falſe Religion, as
afterwardes were the termes of Iewes and Gentiles: after Chriſt,
Chriſtians and Paganes: and laſtly true and falſe Chriſtians are
diſtinguiſhed, by the names of Catholiques and Heretikes. As S.~Auguſtin
teacheth, in his queſtions vpon Geneſis, & other places. VVhich is
confirmed by the like iugement of
\Cite{S.~Ciril Alexandrinus li.~9. aduers. Iulianum.}
\Cite{S.~Ambroſe li. de Noe & arca. c.~4.}
\Cite{S.~Pacianus epiſt. as Symphirianum.}
Theodoret. & manie others vpon this place.}
ſonnes of God ſeing the daughters of men, that they were faire, tooke
to them ſelues wiues out of al, which they had choſen. \V And God ſaid: My
ſpirit ſhal not remaine in man for euer,
%%% 0041
becauſe he is flesh: & his dayes ſhal be
\LNote{An hundred and tvventie yeares.}{Mans
\CNote{Philo. Ioſephus Lactant. Rupert. Toſtatus.}
life was not here ſhortned
to an hundred and twentie yeares, as ſome haue miſunderſtood this
place. For after this diuers liued much longer, as appeareth in the
genealogie of Sem to Abram in the
\XRef{11.~chapter of Geneſis.}
And Abraham liued 175.~yeares
\XRef{(c.~25.)}
Iſaac 180.
\XRef{(c.~35.)}
Iacob 147.
\XRef{(c.~47.)}
and Ioiadas borne 1500.~yeares after, liued 130.~yeares
\XRef{(2.~Par.~24.)}
\MNote{This warning and expectation of repentance ſheweth freewil in
mã.}
But 120.~yeares were granted before the floud for that generation to
repent in, as the
\Cite{Chaldee Edition}
expreſſeth more plainely: \L{Terminum dabitur eis centum viginti annorum
ſi ſorte conuertatur.} The terme of an hundred and twentie yeares ſhal
be geuen them, if perhaps they may conuert. And ſo
\CNote{ho.~22. in Gen.}
S.~Chriſoſtom,
\CNote{Tradit. Heb.}
S.~Hierom, and
\CNote{lib.~15. c.~24. ciuit.}
S.~Auguſtin expound this Scripture. Yet
whether God cut of 20.~of theſe yeares, and brought the floud after a
100. (for Noe had his ſonnes when he was 500.~yeares old, & the floud
came in the 600.~yeare of his age) or that this warning was geuen
twentie yeares before anie of his ſõnes were borne, 
\MNote{Scriptures not eaſie.}
is not ſo eaſely decided by the holie Doctors. How eaſie ſoeuer
Proteſtants ſay al Scriptures are. Though vnder correction of better
iudgement, it ſemeth more probable, that Moyſes by anticipation ioyneth
the birth of Noes ſonnes (when he was 500.~yeares old) to the reſt of
the geneologie of the firſt Patriarkes, in the former chapter, and then
telleth of this admonition, geuen 20.~yeares before their birth. And ſo
God expected the peoples repentance the whole time of 120.~yeares preſcribed.} 
an hũdred & twentie yeares. \V And
\LNote{Giants vvere vpon the earth.}{Some
\MNote{Erronious opinions concerning theſe giants.}
haue thought that theſe giantes were not men, nor begotten by men, but
that either diuels, which fel at firſt from heauen, or other Angels
allured with concupiſcence, begate them of the daughters of Cain. Philo
Iudeus in his booke
\Cite{de Gigantibus},
writeth that thoſe whom Moyſes here called \Emph{Angels}, the
Philoſophers called \Emph{Genios}. \L{Qui ſunt animalia
aërea}, \Emph{vvhich are liuing creatures vvith ayrie bodies}. Ioſephus
\Cite{(li.~1. Antiq.)}
ſaith that Angels begate theſe giants. Tertullian alſo
\Cite{(li. de habitu muliebri)}
holdeth the ſame error, and diuers more otherwiſe good authors.
\MNote{The principal doctors proue that they were men, and begotten of
men.}
But
\Cite{S.~Ciril of Alexandria (li.~9. aduer. Iulian)}
\Cite{S.~Chriſoſtom (homil.~22. in Gen.)}
\Cite{S.~Ambroſe (de Noe & arca. c.~4.)}
\Cite{S.~Auguſtin (li.~15. c.~23. de ciuit)}
\Cite{S.~Hierom (Tradit. Hebraic)}
and other moſt principal Doctors teach it to be vntrue, yea vnpoſſible,
that theſe giants ſhould haue bene begotten by anie other creatures then
by men.
\MNote{Firſt reaſon.}
For that Angels and diuels are mere ſpirits without al natural
bodies.
\MNote{2.}
And if they had ayrie bodies (as they haue not) yet they could
not haue ſuch generation. For the powre or force to engender belongeth
to the vegatatiue ſoule, whoſe proper operations are to turne nutriment
into the ſubſtance of the ſubiect wherin it is, and to engender new
iſſue of ofspring from the ſame, as Ariſtotle ſheweth
\Cite{(li.~2. de anima, textu.~24.)}
And in what bodies ſoeuer there is vegetatiue ſoule, it muſt needes be,
that the ſame was engendred, and muſt ſome times decay and die, and ſo
diuels ſhould be mortal.
\MNote{3.}
Moreouer if they could haue generation togeather with mankind, then ſuch
iſſue ſhould be a diſtinct ſpecies both from man and diuel, as a mule
differeth both from horſe and aſſe.
\MNote{4.}
Againe, if ſpirits had abuſed women in aſſumpted bodies, and ſhape of
men, yet they did not take them to wiues as the Scripture ſaith they
did, who begate theſe giants.
\MNote{5.}
Finally the holie Scripture here expreſly calleth the giants
men. \Emph{Theſe be the} mightie ones, famous \Emph{men}.
\MNote{Giants moſt monſtruous in bodie and in minde.}
The modeſtie
of Scripture terming them famous, whom our common phraſe would cal
infamous being more monſtrous in wickednes of mind, then in hugenes of
bodie. For they were moſt inſolent, laſciuious, couetous, cruel, and in
al kinde of vices moſt impious.}
Giants were vpon the earth in thoſe dayes. For after the ſonnes of God
did companie with the daughters of men, and they brought forth children,
theſe be the mightie of the olde world, famous men. \V And God ſeing the
malice of men was much on the earth, and that 
%%% !!! LNote not marked in text of either
\LNote{Al the cogitation bent to euil.}{Luther (in his
\Cite{21.~article}
condemned by Leo the tenth) would proue by theſe wordes, and the like
folowing, \Emph{Al flesh had corrupted his vvay vpon earth}, that al
workes of men are ſinnes.
\MNote{Luthers argument that al mens workes are ſinnes.}
For (ſaith he) ſeeing the hartes of al men are bent alwaies to euil, and
al humane actions proceede from the hart, it muſt needes by that the
hart as the fountaine being corrupt, the ſtreames alſo iſſuing from the
ſame muſt be corrupted. Againe al fleſh hauing corrupted his way vpon
earth, there is not any iuſt man (ſaith he) nor any man without ſinne:
and with Proteſtantes al ſinnes are mortal.
\MNote{Heretiks like to Siſyphus.}
But Heretikes arguments are like to that the Poëts ſeyne of Siſyphus
laboring to carie a great ſtone to the toppe of an high hil, which when
he hath brought almoſt to the height, it ſtil falleth from him, &
tumbleth againe to the bottome. Euen ſo their arguments that make
greateſt ſhew of prouing their opinions, are nothing but vaine
traueling, when they come to be tried by the true ſenſe of holy
Scripture. In this place Moyſes deſcribeth the enormitie of ſinne that
reigned in the world before the floud, for which God ſent that
deſtruction.
\MNote{The ſinnes before the floud very greuous in foure reſpects.}
For it was haynous in deede, and that eſpecially in foure reſpects.
\MNote{1.}
Firſt the malice and wickednes was \Emph{general}, which is ſignified by
thoſe wordes, \Emph{al flesh hath corrupted his vvay vpon earth.}
\MNote{2.}
Secõdly it was great malice, ſignified by the words \Emph{much},
and, \Emph{al the cogitations of their hart is bent to euil}. For they
committed al maner of wickednes in hautines of pride, in al
laſciuiouſnes of the fleſh, in al crueltie of robbing, ſacking, &
murthering, in al impietie, againſt God & man.
\MNote{3.}
Thirdly, it was of long continuance, and dayly iterated. For Cain once
fallen into damnable ſinne neuer repented, and al his progenie was
exceding wicked and after that Adam and Seth were dead, and Enoch
tranſlated, manie of the faithful fel to the wicked ſorte, and became
worſe and worſe \L{omni tempore}, alwaies, or \Emph{euerie day}.
\MNote{4.}
Fourthly they were obſtinate and obdurate, not repenting when \Emph{Noe}
built the arke, and \Emph{preached iuſtice} (as
\CNote{2.~Pet.~2.}
S.~Peter teſtifieth) and
therfore \Emph{God ſaued him and his familie, bringing in the deluge
vpon the vvorld of the impious.} Al which maketh nothing at al for
Luther.
\MNote{Luthers argument anſwered.}
For although the malice of man, and corruption of fleſh, was
then verie general, great, of long continuance, & obſtinate, yet was it
not ſo vniuerſal, but that God him ſelfe excepted Noe, ſaying to
him \Emph{I haue found thee iuſt in my ſight in this generation},
whereby it is clere that theſe general termes, \Emph{al cogitation}
and \Emph{al flesh}, haue exceptions. As likewiſe other as general
propoſitions in this ſame chapter, concerning the puniſhment threatned,
comprehend not abſolutly al, and euerie one, but almoſt al, very few
excepted. \Emph{I vvil cleane take avvay}, or deſtroy \Emph{man vvhom I
haue created, from the face of the earth. The end of al flesh is come
before me.} Againe, \Emph{that I may deſtroy al flesh vvherein is breath
of life vnder heauen.} Theſe are very general ſpeaches, that al ſhould
be deſtroyed, and yet eight perſons of mankind, that had the ſame
natural fleſh, and amongſt other liuing creatures, that had breath,
diuers payers were ſaued aliue. So that this place (nor anie other in
holie Scripture) wil not proue that Proteſtants paradox, that al mens
actions are mortal ſinnes, or that no man in this life is or can be
iuſt: but
\CNote{Ezech.~1. Luc.~1. &~2. Apoc.~22.}
manie ſcriptures tel vs plainly that ſome men were iuſt, as Noe, Iob,
Daniel, Eliſabeth, Simeon and others. Of Noe ſee more in the next
annotation.}
al the cogitation of their
hart was bent to euil at al times, \V it
\SNote{God who is immutable, & ſubiect to no paſſion, yet by the
enormitie of ſinnes ſemeth prouoked to wrath, and to repent that he had
made man.
\Cite{S.~Amb. li. de Noe & arca. c.~4.}}
repented him that he had made man on the earth. And touched inwardly
with ſorrowe of hart, \V I wil, ſaith he, cleane take away man, whom I
haue created, from the face of the earth, from man euen to beaſtes, from
that which creepeth euen vnto the foules of the ayre, for it repenteth
me that I haue made them.

\V But Noe found grace before our Lord. \V Theſe are the generations of
Noe:
\CNote{Eccl.~44,~17.}
\LNote{Noe vvas a iuſt and perfect man.}{Here
\MNote{Noe iuſt and perfect.}
Noe is not onlie called iuſt, but alſo perfect. The
\Fix{hebrew}{Hebrew}{possible typo, same in both}
word \HH{tamim} of the verbe \HH{tamam} (which ſignifieth to finiſh or
accompliſh) ſheweth that Noe was a perfect or complete man doing al that
he was commanded, and performing the offices of al vertues that
pertained to him; and that not in a vulgar and meane ſorte, but in a
high degree, & heroical maner, as ſundrie ancient Fathers haue gathered
vpon this place. VVe ſhal cite ſome few of their ſayings for example.
\Cite{S.~Hierom (Tradit. Hebraic in Gen.)}
diſtinguiſhing betwen conſummate iuſtice (of the next life) & iuſtice of
this generation (or tranſitorie life) ſaith: \Emph{Noe the iuſt man vvas
perfect in his generations: Noe did vvalke vvith God: that is, did
folovv his ſteppes.}
\Cite{S.~Auguſt. (li.~15. ciuit. c.~26.)}
ſaith the like, that \Emph{Noe vvas called iuſt in his generation, to
vvit, not as the citizens of Gods citie are to be perfected in that
immortalitie, in vvhich they shal be equal to Angels, but as they may
be perfect in this pilgramage.} And in his booke
\Cite{de perfectione contra Cæleſtium.}
he deſcribeth him to be
\MNote{VVho is perfect in this life.}
\Emph{a perfect man, that runneth vvithout blame
tovvards perfection, voide of damnable ſinnes, and is not negligent to
cleanſe venial ſinnes, by almes, prayers}, and other good
workes. S.~Ambroſe alſo teſtifieth,
\Cite{(li. de Noe & arca c.~4.)}
that albeit the world was verie wicked, yet ſome were iuſt,
ſaying: \Emph{By the grace (or fauoure) vvhich Noe found, is shevved
that other mens offence doth not obſcure the iuſt man, vvho is prayſed,
not by the nobilitie of his birth, but by the merit of his iuſtice and
perfection.} S.~Chriſoſt. moſt largely
\Cite{(ho.~23. in Gen.)}
ſetteth forth the iuſtice and perfection of Noe. VVhere after he hath
ſhewed that Noe deſerued in deede the name of a man, becauſe he by
flying vices, and folowing vertues conſerued the image of man, when
others like beaſtes were ledde away and ruled by their wicked luſtes,
proceedeth thus in his commendation. Behold (ſaith he) an other kind of
praiſe: Noe is called, iuſt, which denomination comprehendeth al
vertue. For this name \Emph{iuſt} we vſe to pronounce of them, that
exerciſe al maner of vertue. And that you may lerne, how he ariued to
the very toppe which 
was then alſo required of our nature, the Scripture ſaith, \Emph{he vvas
iuſt, being perfect in his generation}. He performed what thinges ſoeuer it
behoueth one to doe that embraceth vertue, for ſuch a one is perfect, he
intermitted nothing, he halted in nothing, he did not wel in this thing,
and ſinned in that thing, but was perfect in euerie vertue, which was
requiſite for him to haue. Moreouer to make alſo this iuſt man more
conſpicuous to vs in regard of the time, and by comparing him with
others, the Scripture ſaith, \Emph{he vvas perfect in his generation}:
in that time, in that peruerſe generation, which declined vnto euil,
which would not ſo much as pretend anie reſemblance of vertue. In that
generation therfore, in thoſe times, that iuſt man not only pretended,
but arriued to that height of vertue, that he became perfect, and in al
thinges abſolute. And that which I ſaid before, to doe wel amongſt the
enimies of vertue, amongſt them that forbid vertue, doth alwaies
teſtifie a greater poyſe of vertue, ſo by this occaſion the iuſt man got
greater prayſes. Neither doth diuine Scripture here make an end of
praiſing him, but further ſheweth the excellencie of his vertue, and
that he was approued by
\TNote{\L{Diuino calculo.}}
Gods owne cenſure, for beſides ſaying: \Emph{He vvas perfect in his
generation}, it addeth, that \Emph{Noe pleaſed God.} So great was the
renowne of his vertue, that he deſerued to be prayſed of God. \Emph{For
Noe pleaſed God} ſaith the Scripture, that you may know that he was
approued of God. He pleaſed that eye, that can not be deceiued, by his
good workes. Thus farre S.~Chriſoſtom and much more to the ſame
effect. S.~Gregorie the great in his
\Cite{fifth booke of Morales},
and
\Cite{36.~chapter vpon the third chapter of Iob},
recounting certaine principal Patriarches among the reſt
ſaith: \Emph{Noe for that he pleaſed Gods examination vvas ſaued aliue
in the vncleane vvorld}, and after a large catalogue of other iuſt men
in confirmation of this doctrine, that ſome were iuſt in the law of
nature concludeth thus: \Emph{Neither is it to be beleued} (ſaith
he) \Emph{that only ſo manie vvere iuſt before the lavv vvas receiued,
as Moyſes contracteth in his moſt briefe deſcription.}}
Noe was a iuſt and perfect man in
\SNote{In al generations God reſerued ſome iuſt. Much more in the law of
Grace.}
his generations, he did walke with God. \V And he begat three ſonnes,
Sem, Cham, & Iapheth. \V And the earth was corrupted before God, and was
replenished with iniquitie. \V And when God had perceiued that the earth
was corrupted (for al flesh had corrupted his way vpon the earth) \V he
ſaid to Noe: The end of al flesh is come before me, the earth is
replenished with iniquitie from the face of them, & I wil deſtroy them
with the earth. \V Make thee an arke of timber planke: cabinets ſhalt
thou make in the arke, and ſhalt pitch it within, and without with
bitume. \V And thus ſhalt thou make it. The length of the Arke ſhal be
three hundred
\LNote{Three hundred cubites.}{Apelles
\MNote{Appelles an old Heretike, that denied Chriſt to haue true fleſh.}
an old heretike, ſcholar of Marcian, but after leauing him, and amongſt
other new coyned hereſies, reiecting the Law & the Prophetes, would by
this place impugne Moyſes, ſaying it was vnpoſſible that in ſo ſmale
rowme, as was the arke by this deſcriptiõ, the deſigned payers of al
kindes of beaſtes, foule, & ſerpents, ſhould be contained, with the
eight perſons, and al their prouiſion of meate for a whole
yeare. VVherupon he concludeth that this narration (which he calleth a
fable) hath no probabilitie, nor poſſibilitie to be true.
\MNote{A general anſwer to al calumniators of wiſe and learned men.}
To whom & al
ſuch calumniators it may be anſwered, that Moyſes euen in an heretikes
owne cõceipt, if malice obſcured not his ſenſe, muſt needes be thought
wiſe ynough, if he had benne diſpoſed to fayne fables, to frame them
probable, or poſſible, eſpecially when he pretended not to ſignifie a
miracle, in the ſmalnes of the rowme to receiue ſo much, as he
reporteth.
\CNote{ho.~2. in 6.~Gen.}
\MNote{Origens opinion of long cubites not probable.}
Origen to anſwer him ſuppoſeth a cubite here mẽtioned, to haue cõtained
ſix ordinarie cubites: and ſo doubtles the arke might eaſily containe al
thinges that are here ſpoken of, for ſo it were like to a great
citie. But this opinion neither hath good warrant, that euer the
Ægyptians (of whom he ſuppoſeth Moyſes might haue learned it) or any
other nation vſed ſuch long cubites, neither can this meaſure of a
cubite, be agreable to Moyſes meaning, who no doubt ſpeaketh of the like
cubites here, as he doth in other places.
\MNote{Moyſes in other places can not be vnderſtood to ſpeake of
\Fix{o}{ſo}{obvious typo, fixed in other}
long cubites.}
And in
\CNote{Exod.~17.}
Exodus he deſcribeth an Altar to be made fiue cubites long, fiue
broade, and three in height. VVhich would be by Origens meaſure (euerie
cubite contayning ſix ordinarie cubites, that is nine foote at leaſt) in
length, and likewiſe in breadth 45.~foote, and 27.~foot in
height. Againe
\XRef{(Deut.~3.)}
Moyſes telleth of an iron bed of Og King of Baſan, that was nine cubites
long, & foure broad. VVhich make according to Origens meaſure of a
cubite, fourſcore and one foote in length, and in breadth 36.~foote:
which in deede haue no probabilitie. And therfore
\CNote{l.~15. ciuit. c.~27.}
S.~Auguſtin and other
Doctors, ſuppoſing that Moyſes in al theſe bookes, written for
inſtruction of the ſame people, whom he brought forth of Ægypt, ſpeaketh
of one ſorte of cubites, do likewiſe iudge that he meaneth ordinarie &
knowne cubites, which containe a foote & a halfe euerie cubite, as
Vitruuius Agricola and others do proue, or a foote and three quarters of
a foote, which is the greateſt cubite, that ſemeth to be mentioned in
holie Scripture, called a mans cubite, or
\CNote{Deu.~3.}
\Emph{cubite of a mans
hand}. And ſo the Arke was at leaſt in length 450.~foote, in breadth
75. in height 45. or at moſt in length 525.~foote, in breadth~87. and a
halfe: in height 52. and a halfe. And either of theſe capacities was
ſufficient to receiue al the thinges here mentioned, conſidering the
loftes & partitions, that were in the whole arke.}
cubitts: fiftie cubitts the breadth, and thirtie cubitts the height of
it. \V Thou ſhalt make a windowe in the arke, and in a cubit finish the
toppe of it: and the dore of the arke thou ſhalt ſet at the ſide belowe,
middle chambers, and third loftes ſhalt thou make in it. \V Behold I wil
bring the waters of a great floud vpon the earth, that I may deſtroy al
fleſh, wherin there is breath of life vnder heauen. Al thinges that are
in the earth, ſhal be conſumed. \V And I wil eſtablish my couenant with
thee: and thou ſhalt enter into the arke, thou and thy ſonnes, and thy
wife, and the wiues of thy ſonnes with thee. \V And of al liuing
creatures
%%% o-0038
of al flesh, thou ſhalt bring payers into the arke, that they
may liue with thee: of the male ſexe, and the female. \V Of foules
according to their kind, and of beaſtes in their kind, & of al that
creepeth on the earth according to their kind:
%%% 0042
payres of al ſortes ſhal enter in with thee, that they may liue. \V Thou
ſhalt take therfore with thee of al meates, that may be eaten, and thou
ſhalt lay them vp with thee: and they ſhal be meate for thee and
them. \V
\CNote{Heb.~11.}
Noe therfore
\SNote{A right example of a iuſt man.}
did al thinges, which God commanded him.


\stopChapter


\stopcomponent


%%% Local Variables:
%%% mode: TeX
%%% eval: (long-s-mode)
%%% eval: (set-input-method "TeX")
%%% fill-column: 72
%%% eval: (auto-fill-mode)
%%% coding: utf-8-unix
%%% End:
