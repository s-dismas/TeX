%%%%%%%%%%%%%%%%%%%%%%%%%%%%%%%%%%%%%%%%%%%%%%%%%%%%%%%%%%%%%%%%%
%%%%
%%%% The (original) Douay Rheims Bible 
%%%%
%%%% Old Testament
%%%% Genesis
%%%% Chapter 04
%%%%
%%%%%%%%%%%%%%%%%%%%%%%%%%%%%%%%%%%%%%%%%%%%%%%%%%%%%%%%%%%%%%%%%




\startcomponent chapter-04


\project douay-rheims


%%% 0033
%%% o-0030
\startChapter[
  title={Chapter 4}
  ]

\Summary{VVicked Cain killeth holie Abel: 9.~vvhoſe bloud cryeth for
  reuenge. 11.~Cain a curſed
  \Fix{vacabond,}{vagabond,}{likely typo, fixed in other}
  17.~hath much iſſue. 25.~Adam alſo hath Seth, and Seth Enos.}

And Adam knewe Eue his wife: who conceiued and brought forth Cain,
ſaying: I haue gotten a man through God. \V And againe ſhe brought forth
his brother Abel. And Abel was a ſhepehard, & Cain a huſbandman. \V And
it befel after manie dayes that Cain
\LNote{Offered giftes.}{Either God him ſelfe taught Adam, and he his
children, or els they knew by inſtinct of nature, that Sacrifice muſt be
offered to God, to acknowledge therby his ſupreme dominion ouer man, and
mans due ſubiection to his diuine Maieſtie.
\MNote{External Sacrifice due to God in euerie Law.}
And that not only in internal affections, which (as
\CNote{Lib.~10. de ciuit. c.~5.}
S.~Auguſtin, and al
Catholique Doctors teach) is principally required, but alſo in external
things, becauſe we conſiſt of bodie, and not only of ſoule, and haue, by
Gods goodnes, the vſe of corporal things. As here we ſee example in the
law of nature: and the ſame was ordained by written precept in the
\CNote{Leuit.~10.}
law of Moyſes: the
\CNote{Dan.~12. Mal.~1.}
Prophetes alſo foretold, that external Sacrifice ſhould
be offered in the law of grace, and new Teſtament, to wit, the
\CNote{Luc.~22.}
ſame
which Chriſt inſtituted, and left in his Church, to continew to the end
of the world.
\MNote{Sacrifice due to God onlie, and to no creature.}
Moreouer this homage of offering Sacrifice is ſo peculiar to God only,
that albeit manie other exterior rites and ſeruices are vſed both to God
& men, as to be bare head, to bowe, to kneele, & the like before
them,
\CNote{Lib.~10. ciuit. cap.~4.}
\Emph{either of great humilitie} (ſaith S.~Auguſtin) \Emph{or of
peſtiferous flatterie}, to ſuch as are \L{homines colendi, venerandi, ſi
autem eis multum additur, & adorandi}: \Emph{men to be vvorshipped,
reuerenced and if much be geuen them, adored} (for this terme
of \Emph{adoring} is alſo applied to men in holie Scriptures
\XRef{Gen.~23. v.~7.}
\XRef{27. v.~29.})
yet Sacrifice is due to God only, and to no creature how excellent ſo
euer. In ſo much (ſaith the
%%% !!! Aristotle? Same Doctor? Something is confused here or above.
\CNote{Ariſtot. li.~2. Metaphiſ. Ethic.~9. Polit.~7. c.~8.}
ſame Doctor) that as al nations founde it
neceſſarie to offer Sacrifice, ſo none durſt ſacrifice to anie \L{niſi
et, quem Deum aut ſciuit, aut putauit, aut finxit}: but to him whom they
either knew, or thought, or fained to be God.}
offred of the fruites of the earth giftes to our Lord. \V Abel alſo
\CNote{Heb.~11.}
\SNote{A figure of the Lambe that was ſlaine from the beginning of the
world.
\XRef{Apoc.~13. v.~8.}}
offred of the firſt begotten of his flocke, and of their fat: and our
Lord
\LNote{Had reſpect to Abel.}{Both
%%% !!! Where do theſe go?
\CNote{To.~3. q.~4. Quæſt. Hebraic. Lib.~15. ciuit. c.~7. Mala.~1. Hebr.~11.
Leuit.~9. Iudic.~6. 2.~Par.~7. 3.~Reg.~18. 2.~Mac.~1.}
Cain and Abel did wel in offering
external Sacrifice, but they differed much in ſinceritie and maner of
chooſing or diuiding their oblations, touching Gods part and their owne,
as S.~Iuſtinus Martyr, S.~Hierom, S.~Auguſtin and others teach. For Abel
offered of the beſt things, of \Emph{the firſt begotten of his flock,
and of their fatte.} And therfore God reſpected and approued it. But to
Cain and to his giftes he had not reſpect, becauſe he wanted ſincere
deuotiõ.
\MNote{Abels Sacrifice declared acceptable, & not Cains, by ſome
external ſigne.}
VVhich difference of Gods acceptance appeared doubtles, as S.~Hierom and
S.~Auguſtin ſuppoſed, by ſome external ſigne, otherwiſe Cain had not
vnderſtood it. Moſt like it was by fire ſent from God, which inflamed
and conſumed Abels Sacrifice, & not Cains. As we read of diuers other
Sacrifices in holie Scriptures.}
had reſpect to Abel, & to his giftes. \V But to Cain, and to his giftes
he had not reſpect: & Cain was exceeding angrie, and his countenance
abated. \V And our Lord ſaid to him: Why art thou angrie? and why is thy
countinance fallen? \V If thou doe wel,
\LNote{Shalt thou not receiue.}{Reward
\MNote{Reward and puniſhment according to our workes.}
of good workes, and punishment of euil are clerly proued by this
place. God ſaying to Cain: \Emph{If thou doeſt vvel, shalt thou not
receiue againe?} what els but wel for wel doing? as Abel receiued
conſolation of his Sacrifice wel offered, \Emph{but if thou doeſt il,
shal not thy ſinne be preſent forthvvith at the dore?} afflicting thy
conſcience, and not ſuffering thy mind to be in quiet, for remorſe of
thy wicked fact, and feare of iuſt iudgement. For hence it came that
Cains countenance fel, and his ſtomack boyled with angre: puniſhment ſo
beginning euen in this life, & much more in the next world our Sauiour
wil
\CNote{Mat.~16.}
\Emph{render} (as him ſelfe ſaith) to euery man according to his
workes: which the
\CNote{Rom.~2.}
Apoſtle expreſſeth more diſtinctly, \Emph{eternal life, or vvrath &
indignation.}}
ſhalt thou not receiue againe: but if thou doeſt il, ſhal not thy ſinne
forthwith be preſent at the dore? but the luſt therof ſhal be
\LNote{Vnder thee.}{This
\MNote{Freewil in mã alſo after his falle.}
Text ſo plainly ſheweth freewil in man, alſo after his falle, that the
Engliſh
\Fix{Proteſtans}{Proteſtants}{obvious typo, fixed in other}
to auoid ſo clere a truth, for theſe wordes, \Emph{the luſt thereof} (to
wit of ſinne) \Emph{shal be vnder thee, and thou shalt haue dominion
ouer it}, corruptly tranſlate in
\CNote{Bible 1579.}
ſome of their Bibles thus: Vnto
thee \Emph{his} deſire ſhal be ſubiect, and thou ſhalt rule
ouer \Emph{him}. As if God had ſaid, that Abel ſhould be vnder Cain. As
the phantaſtical Manichees peruerted the ſenſe, whoſe abſurditie
\CNote{lib.~15. c.~7. ciuit.}
S.~Auguſtin controlleth maintayning the true conſtruction of the wordes,
\L{Tu dominateris illius; nunquid fratris? abſit. Cuius igitur niſi
peccati?} \Emph{Thou shalt rule ouer: VVhat, ouer thy brother? Not ſo.
Ouer vvhat then but ſinne?} In other Engliſh Editions, namely in the
laſt, which we ſuppoſe they wil ſtand to, it is better, but yet obſcure
thus, \Emph{Vnto thee shal be the deſire therof, and thou shalt haue
rule of it.} Let vs therfore examine the ſenſe, and if
\CNote{Quæſt. Hebraic. in Gen.}
S.~Hierome, the
great ſcripture Doctor did rightly vnderſtand it, God did ſpeake to this
effect to Cain: \Emph{Becauſe thou haſt freevvil, I vvarne thee, that
ſinne haue not dominion ouer thee, but thou ouer ſinne.}
\MNote{The Hebrew alſo & Greeke text proue freewil in Cain.}
The Hebrew hath thus: \L{ad te appetitus eius, et tu dominaberis in
eum}, or \L{ei}. \Emph{Vnto thee the appetite therof, and thou shalt
rule ouer it.} Thargum Hieroſolomitanum concludeth Gods ſpeach to Cain
thus: \Emph{Into thy hand I haue geuen povvre of thy concupiſcence, and
haue thou dominion therof: vvhether thou vvilt to good or to euil.} The
Greke hath thus: \Emph{To thee is the conuerſion therof, and thou shalt
beare rule ouer it}: to wit, appetite, luſt, concupiſcence is vnder thy
wil.
\CNote{S.~Auguſtin. li. de vera Rel. ca.~14.}
\MNote{Freewil teſtified by antiquitie, vniuerſalitie, and conſent of
lerned & reaſonable perſons.}
Finally, al antiquitie vniuerſalitie and vniforme conſent of Chriſtian
Doctors, and other lerned Philoſophers, and reaſonable men hold it for
certaine and an euident truth, that man yea a ſinner hath freewil.
\MNote{Luther abhorred the name of freewil.}
Yet Luther, the father of Proteſtants, ſo abhorred this truth, that he
could not abide the very word, nor voutſafe (when he writ againſt it) to
title his beaſtlie booke, \L{Contra liberum arbitrium}, \Emph{Againſt
freewil}: but, \L{De ſeruo arbitrio}, \Emph{Of ſeruil arbitriment}. And
denieth that man is in aniwiſe free to chooſe, to reſolue, or determine,
but in al things ſeruil, tyed, conſtrained, and compelled to whatſoeuer
he doth, ſaith, or thincketh. Further, that man in al his actiõs is like
to a hackney, that is, forced to goe whither the rider wil haue him. And
knowing the whole world againſt him, ſhameth not to confeſſe, that he
ſetteth them al at naught in reſpect of him ſelfe, concluding
thus:
\CNote{lib. de ſeruo arbitrio.}
\Emph{I haue not} (ſaith he) \Emph{conferred vvith anie in this
booke, but I haue affirmed, and I do affirme. Neither vvil I that anie
man iudge hereof, but I counſail al to obey, or yielde to my opinion.}
\MNote{Caluin alſo miſliketh the word freewil.}
Caluin alſo for his part, conſpireth in this hereſie with Luther, but
more faintly rather wiſheth, then imagineth that men be ſo madde as to
flee from the name of freewil. I (ſaith
\CNote{lib.~2. c.~2. par.~8.}
Caluin) \Emph{neither myſelfe
vvould vſe this vvord, and vvould vvish others, if they aske me
counſaile, to abſtaine from it.} 
\MNote{VVhere is neceſſitie there is
\Fix{nether}{neither}{likely typo, fixed in other}
reward nor puniſhment due.}
But we wil be bold to oppoſe
\CNote{lib.~2. aduerſ. Iouiniam.}
S.~Hieromes
reaſon againſt Luther, Caluin, al Manichees, and others that denie
freewil. \Emph{God made vs} (ſaith he) \Emph{vvith freevvil, neither are
vve dravven by neceſsitie to vertues nor to vices; othervviſe vvhere is
neceſsitie, there is neither damnation nor crovvne.}}
vnder thee, and thou ſhalt haue dominion ouer it.

\V And Caine ſaid to Abel his brother: Let vs goe forth abroad. And when
they were in the filde,
\CNote{Sap.~10.}
Caine roſe vp againſt his brother Abel, and
ſlewe him. \V And our Lord ſaid to Cain: Where is Abel thy brother? Who
anſwered: I know not: am I my brothers keper? \V
\CNote{1.~Io.~3.}
And he ſaid to him: What haſt thou done?
\SNote{VVilful murther is one of the ſinnes that crie to God for
reuenge.}
the voice of thy brothers bloud crieth to me out of the earth. \V Now
therfore curſed ſhalt thou be vpon the earth, which hath opened her
mouth, & receiued the bloud of thy brother at thy hand. \V When thou
ſhalt til
%%% 0034
it, it ſhal not yeld to thee her fruite: a roag and vagabound ſhalt thou
be vpon the earth. \V And Cain ſaid to our Lord: Myne iniquitie is
greater, then that I may deſerue pardon. \V Loe thou doeſt caſt me out
this day from the face of the earth, and from thy face ſhal I be hid,
and I ſhal be a vagabound & fugitiue on the earth: euerie one therfore
that findeth me, ſhal kil me. \V And our Lord ſaid to him: No, it ſhal
not ſo be: but whoſoeuer ſhal kil Cain, shal be punished ſeauen
fould. And our Lord put a marke on Cain, that whoſoeuer found him should
not kil him.

\V And
\LNote{Cain vvent forth.}{It
\CNote{1.~Ioan.~2.}
\MNote{Going forth of the Church a marke of Heretikes.}
is a marke of Heretikes to make breach, and goe forth of the Church. And
commonly it cometh of enuie. \Emph{Some runne into hereſies and
ſchiſmes} (ſaith
\CNote{Tract. de Zelo. & linore.}
S.~Cyprian) \Emph{vvhen they enuie Bishops, vvhileſt one
either complaineth that him ſelfe vvas not rather ordained, or diſdaineth to
ſuffer an other aboue him. Hereupon he kicketh, hereupon he
rebelleth.
\CNote{1.~Ioan.~3.}
Enuie moued Cain to kil his brother, becauſe his ovvne
vvorkes vvere vvicked} and reiected: \Emph{and his brothers iuſt}, and
eſtemed. So going forth became obſtinate, obdurate, and deſperate in
his ſinne, and being reprobate of God, began a wicked Citie, oppoſite to
the Citie of God. VVherfore Moyſes, as
\CNote{lib. de Paſtore. c.~8. & c.~20.}
S.~Auguſtin noteth, intending to
deſcribe, and ſhew the perpetual continuance of Gods Citie, the true
Church, from Adam, which he doth by the line of Seth to Noe, and ſo
forward to his owne time, would not omit to tel alſo the progenie of
Cain, euen to the floode, wherin al his ofſpring was finally drowned and
deſtroyed, that the true Citie of God might appeare more diſtinct, more
cõſpicuous, & more renowned. And that in deede the ſame only (and not
anie broken and interrupted companies or conuenticles) might be knowen
to be the true Church of God.}
Cain went forth from the face of our Lord, and dwelt as a fugitiue on
the earth at the eaſt ſide of Eden. \V And Cain knewe his
%%% o-0031
wife, who conceiued, and brought forth Enoch: And
\SNote{By the increaſe of Abrahãs ſeede (by the line onlie of Iſaac and
Iacob, beſides the iſſues of Iſmael and Eſau) in litle more then
400.~yeares to aboue ſix hundreth thouſand men able to beare armes
\XRef{(Num.~1.)}
it appeareth that Caines progenie in as manie yeares might ſuffice to
people a citie, yea a whole countrie.
\Cite{S.~Aug. l.~15. ciuit. c.~8.}}
he built a citie, & called the name therof by the name of his ſonne,
Enoch. \V Moreouer Enoch begat Irad, and Irad begat Mauiael, and Mauiael
begat Mathuſael, and Mathuſael begat Lamech. \V Who tooke
\SNote{This Lamech of Cains iſſue, is the firſt that is noted in
Scripture, to haue taken two wiues.}
two wiues, the name of the one was Ada, and the name of the other
Sella. \V And Ada brought forth Iabel, who was the father of them that
dwel in tents, and of heardſmen. \V And his brothers name was Iubal: he
was the father of them that ſing on harpe & organes. \V Sella alſo
brought forth Tubalcain, who was a hammerer & worker in al worke of
braſſe & iron. And the ſiſter of Tubalcain was Noema. \V And Lamech ſaid
to his wiues Ada and Sella: Heare my voice ye wiues of Lamech, harken to
my talke: for
\LNote{I haue ſlaine.}{So
\MNote{Scripture hard.}
hard and obſcure is this place, that
\CNote{Tom.~3. ad 1.~quæſt. Damaſi.}
S.~Hierom required by S.~Damaſus
Pope to expound it, dareth not affirme anie one ſenſe for certaine, but
propoſing diuers, which the text may ſeme to beare, wiſheth the Pope
(who was alſo very lerned) to examine al more at large: putting him in
mind that Origen writ his
\Cite{twelfth and thirtenth bookes}
vpon this onlie place.
\MNote{A probable ſenſe according to the Hebrewes Tradition.}
The moſt probable expoſition ſemeth to be gathered out of the Hebrewes
Tradition, that this Lamech of the iſſue of Cain (for there was an other
Lamech of Seths progenie) much addicted to hunting, and his eyes
decaying, vſed in that exerciſe the direction of a young man his nephew,
the ſonne of Tubalcain. VVho ſeing ſomething moue in buſhes, ſuppoſing
it to be a wild beaſt, willed his grandfather to ſhoote at the ſame:
which he did, and ſtroke the marke with a deadlie wound, and approching
to take the prey, found it to be old Cain. VVhereupon ſore amazed,
afflicted, and moued with great paſſion, did ſo beate the young man, for
his il direction, that he alſo died of the drie blowes. After both which
miſhappes, and his paſſion at laſt calmed, Lamech lamenteth as the text
ſaith, that he had killed a man and ſtripling, to wit, the one with a
wound, the other with drie blowes, for which he feared ſeuenfold
puniſhment more then Cain ſuffered for killing Abel. Neuertheles
S.~Hierom &
\CNote{S.~Chriſ. ho.~20. in Gen.}
other Fathers thinke it probable, that Lamech killing the
one of ignorance, the other in paſſion, was not ſo ſeuerly puniſhed as
he feared. And ſo the vnderſtand the reſt of this paſſage, that
ſeuenfold vengance was taken of Cain, by prolongation of his miſerable
life til his ſeuenth generation, when one of his owne iſſue ſlew him,
and an other of the ſame lineage with him.
\CNote{Lib.~1. Antiq. ca.~2. Origen in Gen.}
And Lamech was puniſhed
ſeuentie ſeuen fold when his ſeuentie ſeuen children (for ſo manie he
had, as Ioſephus writeth) and al their ofspring periſhed in the
floud.
\MNote{Myſtical ſenſe.}
Myſtically by ſeuentie ſeuen may be ſignified that the ſinne of
mankind ſhould be puniſhed and expiated in Chriſt our Redemer,
\CNote{Luc.~3.}
who was borne in the ſeuentie ſeuenth generation from Adam.}
I haue ſlaine a mã to the wounding of my ſelfe, and a ſtripeling to mine
owne drie blowe brewſing. \V Seuenfould vengeance shal be taken of Cain:
but of Lamech ſeuentie times ſeuen fould.

\V Adam alſo knewe his wife again: and she brought forth a ſonne, and
called his name Seth, ſaying: God hath giuen me other ſeede for Abel,
whom Cain ſlewe. \V But to Seth alſo was borne a ſonne, whom he called
Enos, this man
\LNote{Begane to inuocate.}{Seth
\CNote{Suidas vocabulo Seth.}
was a moſt holie man, and ſo brought vp
his children, that they were called the ſonnes of God.
\XRef{Gen.~6.}
Adam alſo and Eue were penitent, and became great confeſſors, and are
now Sainctes. And ſo it can not be doubted but amongſt other ſpiritual
exerciſes they prayed and inuocated God. And therfore that which is here
ſaid: He (to wit \Emph{Enos}) \Emph{begane} or (as the Hebrew hath) then
was begune, \Emph{to inuocate the name of our Lord}, can not be
vnderſtood of priuate,
\CNote{Ioſeph. l.~1. Anti. S.~Aug. epiſt.~99. ad Euod.}
\MNote{Publike prayer beſides Sacrifice in the Church of God.}
but of ſome publique prayer of many meeting
togeather, & obſeruing ſome rites & ſet formes in peculiar place
dedicated to diuine Seruice, the Church being now growne to a competent
multitude. And that beſides Sacrifice, which was alſo before, as
appeareth both by Cain & Abel.}
began to inuocate the name of our Lord.


\stopChapter


\stopcomponent


%%% Local Variables:
%%% mode: TeX
%%% eval: (long-s-mode)
%%% eval: (set-input-method "TeX")
%%% fill-column: 72
%%% eval: (auto-fill-mode)
%%% coding: utf-8-unix
%%% End:
