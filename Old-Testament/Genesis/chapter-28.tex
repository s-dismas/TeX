%%%%%%%%%%%%%%%%%%%%%%%%%%%%%%%%%%%%%%%%%%%%%%%%%%%%%%%%%%%%%%%%%
%%%%
%%%% The (original) Douay Rheims Bible 
%%%%
%%%% Old Testament
%%%% Genesis
%%%% Chapter 28
%%%%
%%%%%%%%%%%%%%%%%%%%%%%%%%%%%%%%%%%%%%%%%%%%%%%%%%%%%%%%%%%%%%%%%




\startcomponent chapter-28


\project douay-rheims


%%% 0112
%%% o-0105
\startChapter[
  title={Chapter 28}
  ]

\Summary{Iacob with his fathers bleſsing, and admonition not to take a
  wife of Chanaan, but of the daughters of his vncle Laban, goeth into
  Meſopotamia. 6.~Eſau in the meane time marieth a third wife, his vncle
  Iſmaels daughter. 11.~Iacob ſeeth in ſlepe a ladder reaching to
  heauen, Angels aſcending and deſcending, and our Lord leyning theron
  renewed the promiſes made to Abraham and Iſaac. 16.~And he awayking
  maketh a vow.}

Iſaac therfore called Iacob, and bleſſed him, and commanded him ſaying:
Take not a wife of the ſtocke of Chanaan: \V but goe, and make a iourney
into Meſopotamia of Syria, to the houſe of Bathuel thy mothers father,
and take thee a wife thence of the daughters of Laban thin vncle. \V And
God almightie bleſſe thee, and make thee
%%% 0113
encreaſe and multiplie thee: that thou maieſt be into multitudes of
peoples. \V And
\SNote{Iſaac againe cõfirmeth the bleſſings of Abraham to Iacob, and his
ſede omitting Eſau: yea and God repeteth the ſame.
\XRef{v.~13.}}
geue he thee the bleſſings of Abraham, and to thy ſeed after thee: that
thou mayeſt poſſeſſe the land of thy perigrination, which he promiſed to
thy grandfather. \V And when Iſaac had diſmiſt him, taking his iourney
he came to Meſopotamia of Syria to Laban the ſonne of Bathuel the
Syrian, brother to Rebecca his mother. \V And Eſau ſeing that his father
had bleſſed Iacob, and had ſent him into Meſopotamia of Syria, to marry
a wife thence; and that after the bleſſing he had commanded him,
ſaying: Thou ſhalt not take a wife of the daughters of Chanaan: \V and
that Iacob obeying his parents was gone into Syria: \V hauing tryal alſo
that his father did not willingly ſee the daughters of Canaan: \V he
went to Iſmael, and tooke to wife beſides them, which he had before,
Maheleth the daughter of Iſmael Abrahams ſonne, ſiſter to Nabaioth. \V
Therfore Iacob being departed from Berſabee, went on to Haran. \V And
when he was come to a certaine place, and would reſt in it after ſunne
ſet, he
\LNote{Tooke of the ſtones.}{Iacob
\MNote{VVhy Iacob traueled in poore ſtate.}
traueling into a ſtrange countrie went in ſuch poore ſtate, the better
to hide his departure from Eſau, who otherwiſe might haue killed him by
the way. It was alſo thus diſpoſed by God, that Iacobs faith and
confidence might, to his greatter merite, be exerciſed: and that Gods
prouidence might more manifeſtly appeare, as it did in his returne after
twentie yeares, when with moſt gratful mind he recounted Gods benefites
ſaying
\XRef{(Gen.~32.)}
VVith my ſtaffe I paſſed ouer this Iordan, and now with two troupes I do
returne.}
tooke one of the ſtones that lay there, and putting it vnder his head,
ſlept in the ſame place. \V
\CNote{Sap.~10.}
And he ſaw in his ſleepe
\LNote{A ladder.}{He
\MNote{A notable example of Gods comforth to the afflicted.}
that was in temporal diſtreſſe, was maruelouſly comforted ſpiritually,
by ſeing a ladder that reached from the earth to heauen; Angels paſſing
vp and downe the ſame, and the Sonne of God leaning vpon it, as he that
reigneth both in heauen and earth, who in particular promiſed him, and
his ſede that whole land, that he and his ſede ſhould be bleſſed,
\MNote{Al nations beleuing in Chriſt are bleſſed in him.}
yea that in \Emph{His Sede} al nations ſhould be bleſſed, and that he
would kepe and protect him where ſoeuer he went. How al this was
performed is briefly reherſed in the
\Cite{booke of wiſdom. chap.~10.}}
a ladder ſtanding vpon the earth, and the top therof tooching heauen:
the Angels alſo of God aſcending and deſcending by it, \V and our Lord
leyning vpon the ladder ſaying to him: I am the Lord God of Abraham thy
father, and the God of Iſaac: the Land, wherin thou ſleepeſt, I wil geue
to thee and to thy ſeed. \V And thy ſeed ſhal be as the duſt of the
earth: thou ſhalt be dilated to the Weſt, and to the Eaſt, & to the
%%% o-0106
North, and to the South: and \Sc{in thee and thy seed al the tribes of the earth
shal be blessed.} \V And I wil be thy keeper whither ſo euer thou goeſt,
and wil bring thee backe into this land: neither wil I leaue thee, til I
ſhal haue accompliſhed al things which I haue ſaid. \V And when Iacob
was awaked out of ſleepe, he ſaid: In dede our Lord is in this place,
and I wiſt not. \V And trembling he ſaid: How terrible is this place!
this is none other but the houſe of God, and the gate of heauen. \V And
Iacob ariſing in the morning, tooke the ſtone, which he had laid vnder
his head, and
\LNote{Erected it, povvring oyle.}{To
\MNote{Erecting and annointing of Altares is a religious office being
done to Gods honour.}
erect a ſtone, and powre oyle vpon it, was no wiſe ſuperſticious in
Iacob. Neither did he lerne it of Idolaters: for he abhorred and
deteſted al idolatrical obſeruances.
\MNote{The Church lerneth not rites of Idolaters, but they of the
Church.}
But as S.~Iuſtinus Martyr,
S.~Clement of Alexandria, Origen, Euſebius and others teſtifie,
idolatrical ſuperſtition did rather imitate true religious
ceremonies. For the diuel alwayes affecting that honour, which he ſeeth
done to God, perſwaded thoſe whom he ſeduced, and blinded with errors,
to ſerue him in ſuch maner of external rites, as God was ſerued, that
therby he might either haue like worſhip with God, as it happened among
Painim Idolaters: or els depriue God of this kind of honour, as now we
ſee Proteſtants reiect and pul downe conſecrated Altares, pretending
them to be ſuperſticious. VVherin they ſhew moſt groſſe ignorance, if in
dede they ſo iudge of ignorance, and not of mere malice.
\MNote{Difference in religious, ſuperſticious, & ciuil honour conſiſteth
in the perſons, & intentions.}
For who is ſo ſimple, but he may ſee, that the chiefe difference betwen
Religion and Superſtition in external things, conſiſteth in the perſons
to whom they are done, & in the intẽtion of the doers, & by the ſame
\Fix{differcnee}{difference}{obvious typo, fixed in other}
of perſons ciuil honour is alſo diſtinguiſhed, from both religious and
ſuperſticious. As he that kneeleth to God, religiouſly honoreth
God. Kneeling to the ſunne, moone, or other falſe Gods, ſuperſticiouſly
honoreth the diuel, & kneeling to the King, ciuilly honoreth the
King. Iacob without doubt did al to Gods onlie honour. And that which he
did in this place, is now vſed in the Catholique Church. For ſo Rabanus
a diligent obſeruer and writer of Eccleſiaſtical Rites, Ceremonies, and
Cuſtomes touching the vſe of holie oyle witneſſeth
\Cite{(li.~1. c.~45. Inſtitut. cleric.)}
that the Altar being firſt ſprinkled with water, is annointed with
Chriſme, to the example of the Patriarch Iacob, who after that dreadful
viſion, erected a ſtone for a title (or monument) powring oyle theron,
and calling that place \Emph{The houſe of God}.
\MNote{Two ſortes of holie oyle.}
S.~Cyprian alſo writing of Chriſme, mentioneth the two ſortes of holie
oyle vſed in the Church; one of ſimple oyle conſecrated by a Biſhop,
which is vſed for Catechumes before Baptiſme, perſons poſſeſſed, and the
ſick; the other is made of oyle and balme, alſo conſecrated by a Biſhop,
and this is vſed in Baptiſme, Confirmation, and in conſecrating Altares,
Kings, and Prieſts.}
erected it for a title, powring oyle vpon the toppe. \V And he called
the name of the citie
\TNote{Houſe of God.}
Bethel, which before was called Luza. \V And he
\LNote{Vovved.}{It
\MNote{Vowes are properly of things which are not otherwiſe commanded.}
can not be vnderſtood that Iacob here vowed, or promiſed only to ſerue
God, as the Soueraigne Lord of al creatures, for to that he was bond,
whether he ſhould proſper temporally or no; but that he vowed particular
godlie workes, to which he was not otherwiſe obliged. As here he
expreſſeth two things. Preſuppoſing before al, that the Lord Omnipotent
ſhal be his God, he addeth, firſt \Emph{And this ſtone, vvhich I haue
erected for a title, shal be called the houſe of God.} wherby he
promiſed the building of a Church, performed at his returne
\XRef{(chap.~35.)}
Secondly he added, \Emph{And of al things vvhich thou shalt geue me I
vvil offer tithes to thee.} And this likwiſe was of free deuotion.
\CNote{Gen.~14.}
For tithes alſo in the law of nature were dew to Prieſts, and by
inferior Prieſts to the chiefe Prieſt, as Abraham gaue tithes to
Melchiſedech. And ſo al his tithes were dew to his father, and after his
father him ſelfe was chiefe: yet he promiſed them to God, that is, to
offer them in Sacrifice, and beſtow them in other vſes pertaining to
Gods ſeruice.}
vowed a vowe, ſaying: If God
%%% 0114
ſhal be with me, and ſhal keepe me in the way, by the which I walke, and
ſhal geue me
\SNote{To whom ynough is not ynough, to him nothing is ynough.
\Cite{Aulus Gell.}}
bread to eate, and rayment to put on, \V and I ſhal be returned
proſperouſly to my fathers houſe, the Lord ſhal be my God, \V and this
ſtone, which I haue erected for a title, ſhal be called the Houſe of
God: and of al things that thou ſhalt geue to me, I wil offer tithes to
thee.


\stopChapter


\stopcomponent


%%% Local Variables:
%%% mode: TeX
%%% eval: (long-s-mode)
%%% eval: (set-input-method "TeX")
%%% fill-column: 72
%%% eval: (auto-fill-mode)
%%% coding: utf-8-unix
%%% End:
