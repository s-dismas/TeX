%%%%%%%%%%%%%%%%%%%%%%%%%%%%%%%%%%%%%%%%%%%%%%%%%%%%%%%%%%%%%%%%%
%%%%
%%%% The (original) Douay Rheims Bible 
%%%%
%%%% Old Testament
%%%% Genesis
%%%% Chapter 39
%%%%
%%%%%%%%%%%%%%%%%%%%%%%%%%%%%%%%%%%%%%%%%%%%%%%%%%%%%%%%%%%%%%%%%




\startcomponent chapter-39


\project douay-rheims


%%% 0141
%%% o-0131
\startChapter[
  title={Chapter 39}
  ]

\Summary{Ioſeph being in great credite with his maiſter, hath the whole
  charge of his houſe. 7.~Contemning his myſtris ſolicitation to
  incontinencie, 13.~is falſly accuſed by her to his maiſter: 20.~and
  caſt into priſon, 21.~where againe he getteth credite, and hath the
  charge of al the priſoners.}

Therfore Ioſeph was brought into Ægypt, and Putiphar an Eunich of
Pharao, prince of his armie, a man of Ægypt bought him, at the hand of
the Iſmaelites, by whom he was brought. \V And
\SNote{Ioſeph endued with al vertues was a ſpecial mirrour of
chaſtitie.
\Cite{S.~Amb. li. de Ioſeph. c.~1.}}
our Lord was with him, and he was a man that in al things did
proſperouſly: and he dwelt in his maiſters houſe, \V who knewe verie wel
that our Lord was with him, and that al thinges which he did, were
directed by him in his hand. \V And Ioſeph found grace before his
maiſter, and miniſtred to him: by whom being made ruler ouer al his
thinges, he gouerned the houſe committed to him, and al thinges that
were deliuered vnto him: \V
\SNote{The foure cardinal vertues reigned in him.}
and our Lord bleſſed the houſe of the
Ægyptian for Ioſephes ſake, and multiplied as wel in houſes, as in
landes al his ſubſtance. \V Neither knew he any other thing, but the
bread which he did eate. And Ioſeph was of beautiful countenance, and
comely fauored to behold. \V After manie dayes therfore his maiſtreſſe
caſt her eyes on Ioſeph, and ſaid: Sleepe with me. \V Who
\SNote{Temperance.}
in no wiſe aſſenting to that wicked act, ſaid to her: Behold, my maiſter
hauing deliuered al thinges vnto me, knoweth not what he hath in his
owne houſe: \V neither is there any thing
%%% 0142
which is not in my power, or that he hath not deliuered to me, beſide
thee, that art his wife:
\SNote{Iuſtice.}
how therfore can I do this wicked thing, and ſinne againſt my God? \V
With theſe kinde of wordes
\SNote{Fortitude.}
day by day both the woman was importune vpon the young man: and he
refuſed the aduoutrie. \V And it chanced on a certaine day, that Ioſeph
went into the houſe, and did ſome buſineſſe without anie man with
him: \V and ſhe catching the ſkirte of his garment, ſaid: Sleepe with
me. Who
\SNote{Prudence.}
leauing the cloke in her hand, fled, and went forth abroad. \V And when
the woman ſawe the garment in her handes, and her ſelfe to be
contemned, \V ſhe called to her the men of her houſe, and ſaid to them: See
he hath brought in an Hebrew, to delude vs: he came vpon me, for to lie
with me: and when I had cried out, \V and he heard my voice, he left the
cloake that I held, and fled forth. \V For an argument therfore of her
credite, ſhe reſerued the cloake, and ſhewed it to her huſband
%%% o-0132
returning
home, \V and ſaid: There came vnto me the Hebrew ſeruant, whom thou
dideſt bring hither, for to delude me: \V and when he heard me crie, he
left the cloke which I held, and fled forth. \V His maiſter hearing
theſe thinges, and geuing ouer light credite to his wiues wordes, was
very wrath: \V and deliuered Ioſeph into priſon, where the kinges
priſoners were kept, and he was there ſhut vp. \V And
\SNote{God is more ſpecially with his ſeruants in affliction then in
proſperitie.
\Cite{S.~Amb. li. de Ioſeph. c.~5.}}
our Lord was with Ioſeph, and hauing mercie vpon him gaue him grace in
the ſight of the chiefe of the priſon. \V Who deliuered in his hand al
the priſoners that were kept in cuſtodie: and whatſoeuer was done, was
vnder him. \V Neyther did himſelfe knowe any thing, hauing committed al
things to him: for our Lord was with him, and directed al his workes.


\stopChapter


\stopcomponent


%%% Local Variables:
%%% mode: TeX
%%% eval: (long-s-mode)
%%% eval: (set-input-method "TeX")
%%% fill-column: 72
%%% eval: (auto-fill-mode)
%%% coding: utf-8-unix
%%% End:
