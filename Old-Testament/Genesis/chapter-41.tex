%%%%%%%%%%%%%%%%%%%%%%%%%%%%%%%%%%%%%%%%%%%%%%%%%%%%%%%%%%%%%%%%%
%%%%
%%%% The (original) Douay Rheims Bible 
%%%%
%%%% Old Testament
%%%% Genesis
%%%% Chapter 41
%%%%
%%%%%%%%%%%%%%%%%%%%%%%%%%%%%%%%%%%%%%%%%%%%%%%%%%%%%%%%%%%%%%%%%




\startcomponent chapter-41


\project douay-rheims


%%% 0144
%%% o-0134
\startChapter[
  title={Chapter 41}
  ]

\Summary{Pharao dreaming of fat & leane kine: 5.~alſo of ful and thinne
  eares of corne, 8.~no other being able to interprete, 9.~Ioſeph is
  remembred. 25.~Who interpreting the ſame, 38.~is made ruler ouer al
  Ægypt, 50.~marieth, and hath two ſonnes, Manaſses and Ephraim.}

After two yeares
\SNote{Pharao his dreames, and his Eunuches were prophetical. For by
them God forſhewed things to come:
\XRef{v.~25.}
yet they were no prophets, but Ioſeph: who had the gift to interpret
them.
\Cite{S.~Aug. li.~12. c.~9. de Gen. ad lit.}
\Cite{S.~Greg. li.~12. Moral in c.~13. Iob.}}
Pharao ſaw a dreame. He thought he ſtood vpon a riuer, \V out of the
which came vp ſeuen kine, faire and fat exceedingly: and they fed in
mariſh places. \V Other ſeuen alſo came vp out of the riuer, foule, and
caryan leane: and they fed on the very banke of the riuer, in grene
places: \V and they deuoured them, that had the meruewlous beautie and
good ſtate of bodies. Pharao after he waked, \V ſlept againe, and ſaw an
other dreame:
%%% 0145
Seuen eares of corne grew forth vpon one ſtalke ful and faire: \V there
ſprang alſo other eares as many, thinne and blaſted with aduſtion, \V
deuouring al the beautie of the former. Pharao awaking vp after his
reſt, \V and when morning was come, being frighted with feare, he ſent
to al the interpreters of Ægypt, and to al the wiſe men: and they being
called for, told them his dreame, neither was there anie that could
interprete it. \V Then at length the maiſter of the cupbearers
remembring himſelfe, ſaid: I confeſſe my ſinne: \V The king being angrie
with his ſeruantes, commanded me and the chiefe of the bakers to be caſt
into the priſon of the captaine of the ſouldiers: \V where in one night
both of vs ſaw a dreame portending things to come. \V There was there a
young man an 
\Fix{hebrew,}{Hebrew,}{possible typo, fixed in other}
ſeruant to the ſame captaine of the ſouldiers: to
whom telling our dreames, \V we heard whatſoeuer afterward the euent of
the thing proued to be ſo. For I was reſtored to my office: and he was
hanged vpon a gibbet. \V Forthwith at the kinges commandment, Ioſeph
being brought out of the priſon they polled him: and changing his
apparel, brought him vnto him. \V To whom he ſaid: I haue ſeene dreames,
and there is not anie that can expound them: which I haue heard, thou
doeſt moſt wiſely interprete. \V Ioſeph anſwered: Without me, God ſhal
anſwere proſperous thinges to Pharao. \V Pharao therfore told that he
had ſeene: Me thought I ſtoode vpon the banke of the riuer, \V and ſeuen
kine came vp out of the banke of the riuer, exceeding faire, and ful of
fleſh: which grazed on greene places in a mariſh paſture. \V And behold,
there folowed theſe, other ſeuen kine, ſo paſſing il fauored and leane,
that I neuer ſaw the like in the land of Ægypt, \V which hauing deuoured
and conſumed the former, \V gaue no token of their fulnes: but with the
like leanenes and deformitie,
%%% o-0135
looked heauelie. Awaking, and fallen
againe into a deepe ſleepe, \V I ſawe a dreame: Seuen eares of corne grew
forth vpon one ſtalke, ful and verie faire. \V Other ſeuen alſo thinne
and blaſted, with aduſtion, ſprang of the ſtalke: \V which deuoured the
beautie of the former: I told the dreame to the coniecturers, and there
is no man that can declare it. \V Ioſeph anſwered: The kinges dreame is
one: God hath ſhewed to Pharao
\SNote{Theſe things came to paſſe by Gods particular prouidẽce.
\XRef{Pſalm.~4.}
\Emph{God called} (or cauſed) \Emph{a famine vpon the land.}}
the thinges that he wil doe. \V The ſeuen faire kine, and
%%% 0146
the ſeuen ful eares: be ſeuen yeres of plentifulnes: and both conteine
the ſelfe ſame meaning of the dreame. \V Alſo the ſeuen leane and thinne
kine, that came vp after them, and the ſeuen thinne eares, and blaſted
with the burning winde: are ſeuen yeares of famine to come. \V Which
ſhal be fulfilled in this order. \V Behold there ſhal come ſeuen yeares
of great fertilitie in the whole Land of Ægypt: \V after which ſhal
folowe other ſeuen yeares of ſo great ſterilitie, that al the abundance
before ſhal be forgotten: for the famine shal conſume al the land, \V
and the greatnes of the ſcarſitie, shal deſtroy the greatnes of the
plentie. \V And in that thou dideſt ſee the ſecond time a dreame
perteining to the ſame thing: it is a token of the certeintie, for that
the worde of God shal come to paſſe, and be fulfilled ſpedely. \V Now
therfore let the king prouide a wiſe man and induſtrious, and make him
ruler ouer the Land of Ægypt: \V that he may appointe ouerſeers ouer al
countries: and gether into barnes the fifth part of the fruites, during
the ſeuen yeares of the fertilitie, \V that now preſently shal enſewe:
and let al the corne be laid vp, vnder Pharaoes handes, and let it be
reſerued in the cities. \V And let it be in a readines, againſt the
famine of ſeuen yeares to come, which shal oppreſſe Ægypt, and the land
shal not be conſumed with ſcarſitie. \V The counſel pleaſed Pharao, and
al his ſeruants: \V and he ſpake to them: Can we find ſuch an other man,
that is ful of the ſpirite of God? \V He ſaid therfore to Ioſeph:
Becauſe God hath shewed thee al things that thou haſt ſpoken, can I find
a wiſer and one like vnto thee? \V Thou shalt be ouer my houſe, and at
the commandment of thy mouth al the people shal obey: only in the throne
of the kingdome I wil goe before thee. \V And againe Pharao ſaid to
Ioſeph: Behold, I haue appointed thee ouer the whole land of Ægypt. \V
And he tooke his ring from his owne hand, and gaue it into his hand: and
he put vpon him a ſilke roabe, and put a chaine of gold about his
necke. \V And he made him goe vp into his ſecond chariot, the cryer
proclayming that al should bowe their knee before him, and that they
should know he was
\LNote{Made gouernour.}{It
\CNote{Eccli.~11.}
\MNote{Holie Ioſeph ſuddenly aduanced.}
is eaſie in the eyes of God, ſuddenly to enrich the poore. For who would
haue thought (ſaith
\CNote{li. de Ioſeph.}
Philo) that in one day a bondman ſhould be made a
lord, a poore priſoner the chiefe of the nobilitie, an vnder gaolor the
viceroy, or kings deputie, for a common priſon to haue a kinglie court
of his owne, from extreme ignominie, to aſcend into ſo hiegh a roome of
dignitie!}
made gouernour ouer the whole Land of Ægypt. \V And the king ſaid to
Ioſeph: I am Pharao: without thy commandment no man shal moue hand or
foote in al the land of Ægypt. \V And he turned his
%%% 0147
name, and called him in the Ægyptian
\Fix{togue}{tongue}{obvious typo, fixed in other}
\LNote{Sauiour of the vvorld.}{In the original text the new name and
title geuen by Pharao to Ioſeph is expreſſed by theſe two
wordes, \HH{Saphnath pahanaach}: the former \HH{Saphnath} in Hebrew
ſignifieth a ſecrete or hidden thing, of \HH{ſaphan} to hide: but the
ſignification of the other word \HH{pahanaach}, is more vncertaine, being
found no where els in the holie Bible.
\MNote{Ioſeph truly called the reueler of ſecrets.}
The Rabins do commonly interprete
them both together, \Emph{The man to vvhom ſecretes are reueled},
or, \Emph{The reueler of ſecretes}, and ſo this name agreeth wel to
Ioſeph, in reſpect of the gift of interpreting dreames. But beſides his
interpreting, he alſo gaue moſt wiſe counſel, that tended to the ſaftie
of manie, which, it is like, Pharao ment to expreſſe by this new
name.
\MNote{But more honorably, the Sauiour of the world.}
And S.~Hierom, who doubtles with great diligence, and no leſſe
iudgement, ſearched the true ſignification therof, ſaith, that albeit
this name in Hebrew ſoundeth \Emph{The finder out of ſecrets}, yet ſeing
it was impoſed by an Ægyptian (who knew no Hebrew) the reaſon therof
muſt be had of the ſame tongue; and theſe two wordes in the Ægyptian
language are interpreted \Emph{The Sauiour of the vvorld}: for that he
deliuered the world from the iminent ruine of famine. Thus ſaith
S.~Hierom.
\MNote{Therin a figure of Chriſt.}
And ſo moſt aptly the figure anſwereth to Chriſt, the
true \Sc{Saviovr} of the world.}
the Sauiour of the world. And he gaue him to wife Aſeneth the
daughter of Putiphar
\SNote{\HH{Cohen} ſignifieth prieſt; as not only the latin, but alſo the
70. & Philo and Ioſephus here tranſlate though ſometimes it
ſignifieth \Emph{prince}, as the Chaldey paraphraſis interpreteth,
wherby it is probable that this Putiphar was both a prieſt, and a
prince.}
prieſt of Heliopolis. Ioſeph therfore went forth to the land of Ægypt
(\V and he was thirtie yeares old when he ſtood in the
%%% o-0136
ſight of king
Pharao) and did circuite al the countries of Ægypt. \V And the
fertilitie of the ſeuen yeares came: and the corne being bound vp into
ſheaues was gethered togeather into the barnes of Ægypt. \V Al the
abundance alſo of graine was laid vp in euerie citie. \V And there was
ſo great abundance of wheat, that it became equal to the ſand of the
ſea, and the plentie exceeded meaſure. \V And there were borne vnto
Ioſeph two ſonnes before the famine came: whom Aſeneth the daughter of
Putiphar prieſt of Heliopolis bare him. \V And he called the name of the
firſt begotten
\TNote{Obliuion.}
Manaſſes, ſaing: God made me to forget al my labours, & my fathers
houſe. \V The name alſo of the ſecond he called
\TNote{Fruitful or Grovving.}
Ephraim, ſaing: God hath made me to encreaſe in the land of my
pouertie. \V Therfore when the ſeuen yeares of the plentifulnes, that
had bene in Ægypt were paſſed: \V the ſeuen yeares of ſcarcitie beganne
to come, which Ioſeph foretold: and in the whole world the famine
preuailed, but in al the land of Ægypt there was bread. \V The which
being in hunger, the people cried to Pharao, deſiring foode. To whom he
anſwered: Goe ye to Ioſeph: and whatſoeuer he ſhal ſay to you, that doe
ye. \V And the famine dayly encreaſed in al the land: and Ioſeph opened
al the barnes, and ſold to the Ægyptians: for them alſo the famine had
opreſſed. \V And al prouinces came into Ægypt, to buy victuales, and to
moderate the miſerie of the ſcarſitie.


\stopChapter


\stopcomponent


%%% Local Variables:
%%% mode: TeX
%%% eval: (long-s-mode)
%%% eval: (set-input-method "TeX")
%%% fill-column: 72
%%% eval: (auto-fill-mode)
%%% coding: utf-8-unix
%%% End:
