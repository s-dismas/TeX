%%%%%%%%%%%%%%%%%%%%%%%%%%%%%%%%%%%%%%%%%%%%%%%%%%%%%%%%%%%%%%%%%
%%%%
%%%% The (original) Douay Rheims Bible 
%%%%
%%%% Old Testament
%%%% Genesis
%%%% Chapter 01
%%%%
%%%%%%%%%%%%%%%%%%%%%%%%%%%%%%%%%%%%%%%%%%%%%%%%%%%%%%%%%%%%%%%%%




\startcomponent chapter-01


\project douay-rheims


%%% 0021
%%% o-0019
\startChapter[
  title={Chapter 1}
  ]

\Summary{God
\MNote{The firſt part. Of the creatiõ of al things.}
createth heauen and earth, and al things therin; diſtinguishing and
bevvtyfying the ſame; 26.~laſt of al the ſixth day he createth man: to
vvhom he ſubiecteth al corporal things of this inferior vvorld.}

In
\CNote{Act. 14,~15. 17,~24. Pſalm. 32,~6. 135,~5. Eccli.~10,~1.}
\LNote{In the beginning.}{Holie
\MNote{The Church had only Traditions & no Scripture aboue 2400.~yeares.}
Moyſes telleth what was done in the beginning of the world, and ſo
forward euen til his owne time, writing aboue two thouſand and foure
hundreth yeares after the beginning. Al which being incomprehenſible by
humaine witte or diſcourſe, he knew partly by Reuelations from God, for
he had the gyft of Prophecie in moſt excellent ſorte: partly by
Traditions from his elders, who lerned of their fathers. For vntil that
time the Church had only Traditions of ſuch things, as were reueled to
ſpecial men, wherby we ſee the great authoritie of Traditions, before
there were Scriptures.
\MNote{Traditions neceſſarie for three cauſes.}
And ſince Scriptures were written they are alſo neceſſarie, for three
ſpecial reaſons.
\MNote{1.}
Firſt for that we are only aſſured by Tradition of the Church, that
thoſe bookes are in dede holie Scriptures, which are ſo accounted, and
not by the Scripture it ſelfe, for that were to proue the ſame by the
ſame, vntil we be aſſured of ſome part, that proueth ſome other
partes. And this made S.~Auguſtin to ſay plainly, that
\CNote{cont. Epiſt. ſund. c.~5.}
\Emph{he could
not beleue the Goſpel, except the Church told him vvhich is the Goſpel.}
\MNote{2.}
\MNote{Scripture of moſt eminent authoritie.}
Secondly holie Scriptures being once knowen to be the word of God, and
ſo of moſt eminent authoritie of al writings in the world, as
S.~Auguſtin S.~Ierome, & al other Fathers agree, yet for the true
vnderſtanding of the ſame, both the Scripture it ſelfe, and the ancient
Fathers remitte vs to the Church, namely to thoſe in the Church, that
are \Emph{appointed} by Gods ordinance, \Emph{in the high place that he
hath choſen.} VVhich were the High Prieſts in the old Teſtament, as
appeareth:
\XRef{Deut.~17.}
\XRef{Mat.~23.}
\XRef{Ioan.~11.}
And in the new Teſtament, S.~Peter and his Succeſſors for whom Chriſt
prayed that his faith ſhould not faile: and therfore commanded him to
confirme his bretheren
\XRef{Luc.~22.}
\MNote{3.}
Thirdly for things not expreſſed in particular in holie Scripture, the
Scripture and Fathers do likewiſe remitte vs to Traditions, and to the
iudgement and teſtimonie of the Church. Chriſt ſaying to his
Apoſtles:
\CNote{Luc.~10,~16.}
\Emph{he that heareth you heareth me.} The Apoſtles doubted
not to ſay:
\CNote{Act.~15,~28.}
\Emph{It ſemed good to the Holie Ghoſt and to vs.} And
S.~Paul willed the Theſſalonians
\CNote{2.~Theſſ.~2.}
\Emph{to hold the traditions, vvhich
they had lerned}, whether it were by word, or by his Epiſtle.}
\LNote{In the beginning God made heauen and earth.}{Al
\MNote{Scriptures hard.}
writers
\CNote{Origen. ſuper. Gen.~c.~1. Aug. li.~2. de~Gen. cont. Manich. c.~2.}
ancient and later find ſuch difficulties in theſe firſt
chapters, that ſome otherwiſe very lerned haue thought it not poſſible
to vnderſtand the ſame according to the proper and vſual ſignification
of the wordes, as the letter may ſeme to ſound, but expound al
allegorically, as that by the waters aboue the firmament ſhould be
vnderſtood the bleſſed Angels, by the waters vnder the firmament wicked
ſpirites, and the like. So did Origen and diuers that folow him
therein. Yea S.~Auguſtin in his bookes vpon Geneſis againſt the
Manichees, written ſhortly after his conuerſion, when he could not find
as he deſired a good and probable ſenſe agreable to the wordes, in their
proper ſignification, expounded them myſtically, but afterwards in his
other bookes
\CNote{lib.~1. c.~18. & lib.~8. c.~2.}
\Cite{de Geneſi ad literam},
he gratfully acknowledgeth that God had geuen him further ſight therin,
and that now he ſuppoſed he could interprete al according to the proper
ſignification of the wordes: yet ſo that he durſt not nor would not
addict him ſelfe to one ſenſe, but that he was readie to imbrace an
other, leſt by ſticking to his owne iudgement he might faile. So
likewiſe
\CNote{Baſ. ho.~9. in Geneſ.}
S.~Baſil,
\CNote{Chriſoſt. epiſt.~44.}
S.~Chriſoſtom,
\CNote{Amb. & Beda in examen.}
S.~Ambroſe,
\CNote{Ierom. Epiſtol. ad Euſtoch.}
S.~Ierome,
S.~Bede, and other greateſt Doctors found & confeſſed
great difficulties in theſe firſt chapters, which they with much ſtudie
endeuored to explicate. And therfore it is a wonder to ſee our
Proteſtants & Puritans hold this Paradox, that Scriptures are eaſie to be
vnderſtood. VVheras both by teſtimonie of thoſe that haue in deede
ſtudied & laboured in them, and by a litle due conſideration, the
cõtrarie is moſt euident.
\MNote{Why Scriptures are hard.}
For whoſoeuer wil looke into the holie Scriptures, ſhal find that ſome
times in ſhew
\CNote{Gen.~1. v.~3.~&~14.}
one place ſemeth contrarie to an other;
\CNote{Exo. 20.~5. & 18,~v.~20.}
ſome times the letter & phraſe are obſcure & ambiguous:
\CNote{Ioan.~8,~25. Rom.~8.}
ſome times the ſentences
vnperfect. Againe manie ſpeaches are prophetical, manie parabolical,
metaphorical, and vttered vnder other tropes and figures, and that in
the literal ſenſe.
\MNote{Three ſpiritual ſenſes beſides the Literal.}
Moreouer there are three ſpiritual ſenſes beſides the literal, very
frequent in holie Scripture.
\MNote{Allegorical.}
Allegorical pertayning to Chriſt and the Church;
\MNote{Moral.}
Moral pertayning to maners;
\MNote{Anagogical.}
and Anagogical pertayning to the next life. As this word Ieruſalem
literally ſignifieth the head citie of Iewrie: Morally the ſoule of
man: Allegorically the Church militant: and Anagogically the Church
triumphant. And ſome times this (and the like of others) metaphorically in
the literal ſenſe ſignifieth the Church militant, and not the citie of
Iewrie, as in the
\XRef{12.~chapter to the Hebrewes}:
and ſome times the Church triumphant, as in the
\XRef{21.~of the Apocalips}.}
the beginning God created heauen and earth. \V And the earth was voide &
vacant, and darkenes was vpon the face of the deapth: and the
\LNote{The Spirite of God.}{In the Hebrew it is ſignified, that the
Spirite of God was on the waters to make them fertile, for that fiſhes
and birdes were to be procreated therof; the word
is \HH{merahepheth}, \L{incubabat}, \Emph{ſate vpon}, to produce fruict
(ſaith S.~Ierom) from the waters, as a henne by her heate, produceth life
in the egges. And the ſame
\CNote{Ierom. Epiſtol.~83. ad Ocea.}
S.~Ierom, and before him
\CNote{Tert. de Baptiſ.}
Tertullian teach,
\MNote{A figure of Baptiſme.}
that this was a figure of Baptiſme, which conſiſteth of water and the
Holie Ghoſt. For as water in the beginning of the world receiued a
certain vital vertue of the Holie Ghoſt to produce liuing creatures: ſo
alſo Baptiſme receiueth vertue of the ſame Holie Ghoſt to procreate new
men.
\MNote{Chriſtians called fiſhes.}
VVherupon Tertullian calleth Chriſtians fiſhes, becauſe they are gotten
from the waters, and thence haue their firſt ſpiritual life. \Emph{Let
it not therfore ſeme ſtrange} (ſaith he) \Emph{that in Baptiſme VVaters
geue life.}}
the Spirite of God moued ouer the waters. \V And God ſaid: Be light
made. And light was made. \V And God ſaw the light that it was good: & he
diuided the light from the darkenes. \V
\CNote{Heb.~11,~3.}
And he called the light, Day,
and the darkenes, Night: and there was euening & morning, that made one
day. \V God alſo ſaid: Be
\SNote{The firmament is al the ſpace from the earth to the hiegheſt
ſtarres: the loweſt part diuideth betwene the waters on the earth and
the waters in the ayer.
\Cite{S.~Aug. li.~11. de Gen. ad lit. c.~4.}}
a firmament made amidſt the waters: and let it diuide betwene waters &
waters. \V
\CNote{Iob.~38. Ier.~10,13.}
And God made a firmament, and diuided the waters, that were
vnder the firmament, from thoſe, that were aboue the firmament. And it
was ſo done. \V And God called the firmament,
\SNote{Likewiſe heauẽ is al the ſpace aboue the earth: in whoſe loweſt
part are birdes and waters, in the higher part ſtarres: the hiegheſt is
the Empyrial heauen.
\XRef{Eſa.~66.}}
Heauen: and there was euening & morning that made the ſecond day. \V God
alſo ſaid: Let the waters that are vnder the heauen, be gathered
together into one place: and let the drie land appeare. And it was ſo
done. \V And God called the drie land, Earth: and the gathering of
waters together, he called Seas. And God ſawe that it was good. \V And
ſaid: Let the earth ſhoot forth grene herbes, and ſuch as may
%%% o-0020
ſeede, &
fruite trees yelding fruit after his kinde, ſuch as may haue ſeede in it
ſelfe vpon the earth. And it was ſo done. \V And the earth brought forth
%%% 0022
grene herbe, ſuch as ſeedeth according to his kinde, & tree that beareth
fruite, hauing ſeede eche one according to his kinde. And God ſaw that
it was good. \V And there was euening & morning that made the third
day. \V Againe God ſaid: Be there lightes made in the firmament of
heauen, to diuide the day & the night, and let them be
\SNote{The lights made the firſt day, are diſpoſed the fourth day in
their proper courſes for more diſtinction of times.
\Cite{S.~Dionyſ. ca.~4. de diuin. nom.}
\Cite{S.~Tho. p.~1. q.~67. a.~4. & q.~70. a.~2.}}
for ſignes & ſeaſons, and dayes and yeares: \V to ſhine in the firmament
of heauen, & to giue light vpon the earth. And it was ſo done. \V And
God made
\LNote{Tvvo great lights, and ſtarres.}{Here occurreth an other example
of the hardnes of holie Scripture. For if the two great lights (to wit
the Sunne & the Moone) and alſo the ſtarres, vvere made the fourth day,
and not before, as it may ſeme by the wordes in this place, then what
was that light, and in what ſubiect was it, that was made the firſt day?
\MNote{Light being a accident remayned without ſubiect, by the iudgement
of ſome lerned Fathers.}
S.~Baſil, S.~Gregorie Nazianzen, Theodoret, and ſome others, writing
vpon this place do thinke that the light, which was made the firſt day,
remayned though an accident without his ſubiect til the fourth day. And
albeit moſt other Doctors rather think that the ſubſtance of the Sunne &
Moone, & of other planets and ſtarres were created the firſt day, and
the fourth day ſet in that order and courſe which now they kepe, with
more diſtinction \Emph{for ſignes and ſeaſons, and dayes and yeares}:
\MNote{The accidents of breade and wine can remaine by Gods power
without their ſubiects.}
yet it is clere that the foreſaid ancient Doctors iudged it poſſible,
that
accidents may remaine without their ſubiect, which a Sacramentarie
wil be loath to grant, leſt it might be proued poſſible, as both theſe &
al other Catholique Doctors beleued and taught, that the accidents of bread
and wine remaine in the bleſſed Sacrament of the Euchariſt without their
ſubiects. VVhich Proteſtants denie.}
two
\SNote{The Sũne & Moone: for though the moone be the leaſt viſible
ſtarre except Mercurie, yet it geueth more light on the earth by reaſon
it is nerer, and ſo Moyſes ſpeaketh according to the vulgar capacitie
and vſe of things.
\Cite{S.~Aug. li.~2. de Gen. ad lit. ca.~16.}}
great lights: a greater light, to gouerne the day: and a leſſer light to
gouerne the night: and ſtarres. \V And he ſet them in the firmament of
heauen, to ſhine vpon the earth, \V and to gouerne the day & the night,
and to diuide the light & the darkenes. And God ſawe that it was
good. \V And there was euening and morning that made the fourth day. \V 
God alſo ſaid: Let the waters bring forth creeping creature hauing life,
and flying foule, ouer the earth vnder the firmament of heauen. \V And
God created huge Whales, and al liuing & mouing creature, that the
waters brought forth, according to eche ſorte, & al ſoule according to
their kinde. And God ſaw that it was good. \V And he
%%% !!! extra note? referance to the note for verse 28 with the
%%% same phraſe repeated about man?
%%% \LNote{}{}
bleſſed them ſaying: Increaſe and multiplie, and repleniſh the waters of
the ſea: and let the birds be multiplied vpon the earth. \V And there
was euening & morning that made the fifth day. \V God ſaid moreouer: Let
the earth bring forth liuing creature, in his kind, cattle, & ſuch as
creepe, & beaſtes of the earth according to their kindes: and it was ſo
done. \V And God made the beaſtes of the earth according to their
kindes, and cattle, & al that crepeth on the earth in his kind.

\V And God ſaw that it was good, \V and he ſaid
\LNote{Let vs make man to our Image.}{For
\MNote{Tenne prerogatiues of man in his creation.}
better conſideration of Gods bountie towards vs, and ſturring our ſelues
to gratitude towards him, we may here note tenne prerogatiues beſtowed
on vs, by our Lord & maker in our creation aboue al other earthlie
creatures.
\MNote{1.~Made like to God.}
Firſt, wheras God by an imperial word of commandment made other
creatures, \L{Fiat lux, Fiat firmamentum}: \Emph{Be there light: Be
there a firmament}: intending to make man, he procedeth familiarly, by
way, as it were, of conſultation, and as to his owne vſe and ſeruice to
make man ſaying: \Emph{Let vs make man to our image and likenes}, that
is to ſay, a reaſonable creature with vnderſtanding and free wil, which
beaſtes haue not.
\MNote{2.~The Myſterie of the B.~Trinitie inſinuated in his creation.}
Secondly, in this worke God firſt inſinuateth the high Myſterie of the
B.~Trinitie, or pluralitie of Perſons in one God (becauſe man is to
beleue the ſame) ſignifying the pluralitie of Perſons by the
wordes \Emph{Let vs make}, and \Emph{to our}: and the vnitie in
ſubſtance, by the wordes \Emph{Image and likenes}, the firſt in the
plural number, the later in the ſingular.
\MNote{3.~Produced by God him ſelfe.}
Thirdly, other creatures were produced by the waters and
earth, \Emph{Let the vvaters bring forth} (fiſhe and foule) \Emph{Let
the earth bring forth} (graſſe and cattle, & other beaſtes) but God
brought forth man, not by the earth, though of the earth, nor by water,
nor by heauen, nor by Angels, but by him ſelfe, geuing him a reaſonable
ſoule, not ſenſual only as to beaſtes, and the ſame not produced of anie
creature, but created immediatly of nothing.
\MNote{4.~Placed in paradiſe.}
Fourthly, God gaue man Paradiſe a moſt pleaſant place to dwel in.
\MNote{5.~Lord of al earthlie creatures.}
Fiftly, God gaue man dominion and imperial authoritie ouer al liuing
creatures vnder heauen.
\MNote{6.~Innocencie.}
Sixtly, man was created in that innocencie of life, and integritie of al
vertues, that his mind was wholly ſubiect to God, his ſenſe to reaſon,
his bodie to his ſpirite, and al other liuing creatures obedient to him:
euen the terrible Lions, the cruel Tigers, the huge Elephants, and the
wildeſt birdes.
\MNote{7.~Excellent knowlege.}
Seuently, God brought them al to man, as to do him homage, and to take
their names of him. VVhich by his excellent knowledge he gaue them
conformable to their natures.
\MNote{8.~Powre to liue euer.}
Eightly, God gaue man in ſome ſorte an immortal bodie, that if he had
kept Gods commandment, he had liued long and pleaſantly in this world,
and ſo ſhould haue bene tranſlated to eternal life without dying.
\MNote{9.~Gift of prophecie.}
Ninthly, God did not only adorne man with al natural knowledge, and
ſupernatural vertues, but alſo with the gift of prophecie. VVherby he
knew that Eue was \Emph{a bone of his bones, and flesh of his flesh},
though being a ſlepe he knew not when ſhe was made.
\MNote{10.~God conuerſed familiarly with man.}
Tenthly (which was the chiefe benefite of al) God conuerſed familiarly
with man, and that in ſhape of man, which was a token of his meruelous
great loue to man, and a ſingular incitment of him to loue God. Reade
more, if you pleaſe, of the dignitie of man, and the benefites of God
towards him in his creation, in
\Cite{S.~Bernard vpon the 99.~Pſalme.}
And
\Cite{vpon the 61.~chapter of Eſaie.}}
\CNote{Col.~3,~10.}
Let vs make Man to our image, & likenes: and let him haue dominion ouer
the fiſhes of the ſea, and the foules of the ayre, and the beaſtes, and
the whole earth, and al creeping creature, that moueth vpon the
earth. \V And God created man, to his owne image:
\CNote{Mat.~19,~4.}
to the image of God he created him, male & female he created them. \V
And God bleſſed them, and ſaith:
\LNote{Increaſe and multiplie.}{VVhether
\MNote{Gods bleſſing alwayes effectual.}
this be a commandment or no, at leaſt it is a bleſſing, for ſo the
wordes before conuince, \Emph{God bleſſed them and ſaid: Increaſe and
multiplie.} He ſaid the ſame alſo to brute creatures, which are not
capable of a precept, but by this were made fertile. VVherby we ſee that
Gods bleſſing alwayes worketh ſome real effect: as of fertilitie in this
and other places, of multiplication of the loaues and fiſhes,
\XRef{Ioan~6.}
\MNote{Eſpecially in the holie Euchariſt.}
And ſome real effect Chriſts bleſſing muſt nedes worke alſo in the
bleſſed Sacrament.
\XRef{Mat.~26.}
VVhich can be no other but changing bread and wine into his bodie &
bloud, ſeing him ſelfe expreſly ſayeth: \Emph{This is my bodie, this is
my bloud.}

And though Gods bleſſing in this place, be alſo a precept,
\MNote{Not al men & women commanded to marie.}
yet it is
not to al men for euer; but for the propagation of mankind, which being
long ſince abundantly propagated, the obligation of the precept ceaſeth
the cauſe ceaſing. So S.~Cyprian, S.~Ierome, S.~Auguſtin, and other
Fathers expound this place. And confirme the ſame by the text, for
immediatly God ſignifying to what end he ſpoke, ſaith: \Emph{and
replenish the earth.} VVhich
\Fix{benig}{being}{obvious typo, fixed in other}
repleniſhed, Gods wil is therin fulfilled.}
Increaſe and multiplie, & repleniſh the earth, and ſubdew it, and rule
ouer the fiſhes of the ſea, and the foules of the ayre, & al liuing
creatures, that moue vpon the earth. \V And God ſaid: Behold I haue
giuen you al maner of hearbe that ſeedeth vpon the earth, and al trees
that haue in them ſelues
%%% 0023
ſeede of their owne kinde, to be your meate: \V and to al beaſtes of the
earth, and to euerie ſoule of the ayre, & to al that moue vpon the
earth, and wherein there is life, that they may haue to feede vpon. And
it was ſo done. \V And God ſawe al things that he had made, and
\SNote{Euerie creature in nature is good, but al conſidered together
make the whole world perfect, moſt apt to mans vſe and Gods glorie.
\Cite{S.~Aug. li.~1. de Gen. cont. Manich. ca.~21.}}
they were very good. And there was euening & morning that made the ſixt
day.

\stopChapter


\stopcomponent


%%% Local Variables:
%%% mode: TeX
%%% eval: (long-s-mode)
%%% eval: (set-input-method "TeX")
%%% fill-column: 72
%%% eval: (auto-fill-mode)
%%% coding: utf-8-unix
%%% End:
