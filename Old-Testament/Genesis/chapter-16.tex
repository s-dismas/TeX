%%%%%%%%%%%%%%%%%%%%%%%%%%%%%%%%%%%%%%%%%%%%%%%%%%%%%%%%%%%%%%%%%
%%%%
%%%% The (original) Douay Rheims Bible 
%%%%
%%%% Old Testament
%%%% Genesis
%%%% Chapter 16
%%%%
%%%%%%%%%%%%%%%%%%%%%%%%%%%%%%%%%%%%%%%%%%%%%%%%%%%%%%%%%%%%%%%%%




\startcomponent chapter-16


\project douay-rheims


%%% 0081
%%% o-0077
\startChapter[
  title={Chapter 16}
  ]

\Summary{Sarai geueth her handmaid Agar as a wife to Abram: 4.~who
  conceiuing deſpiſeth her myſtreſſe, is therfore afflicted, & flyeth
  away. 7.~But is warned by an Angel to returne and humble herſelfe,
  15.~which she doth and beareth Iſmael.}

Sarai therfore, the wife of Abram, had brought forth no children: but
hauing an handmaid an Ægyptian named Agar, \V she ſaid to her huſband:
Behold, our Lord hath cloſed me, that I might not beare: Goe in vnto my
handmaid, if happely of her at the leaſt I may haue children. And when
he agreed to her in this requeſt, \V she toke Agar the Ægyptian her
handmaid tenne yeares after that they firſt dwelled in the land of
Chanaan: and gaue her vnto her huſband
\LNote{To vvife.}{The
\CNote{S.Aug. li.~22. c.~47. cont. Fauſt.}
\MNote{Manichees, condemned pluralitie of wiues in the
Patriarches. Luther alloweth it in Chriſtians. Other Proteſtants in ſome
caſe.}
Manichees did calumniat holie Abraham, and other Patriarches for hauing
manie wiues, condemning them of incontinencie and adulterie for the
ſame.
\CNote{Luther propſit. 62.~65. &~66.}
Luther in the contrarie extreme held it not vnlawful, but
indifferent, now in the law of grace, for a man to haue more wiues then
one at once. And ſome Engliſh Proteſtants hold, that for adulterie, the
innocent partie may marie an other, the firſt liuing. But the Catholique
doctrin diſtinguiſhing times and cauſes, ſheweth how pluralitie of wiues
was lawful ſometimes, and at other times, eſpecially ſince Chriſt,
altogether vnlawful, and vndiſpenſable. The ſumme of which veritie is
this. By the firſt inſtitution of Mariage in the ſtate of innocencie,
and law of nature, and by the law of Chriſt, it is vnlawful for anie man
to haue more wiues, and for anie woman to haue more huſbands, then
one. In the one part of which Law notwithſtanding God ſometimes
diſpenſed.
\CNote{S.~Aug. de bono coniugali. ca.~17.}
\MNote{Two ſortes of precepts in the law of nature.}
For there be two kindes of preceptes pertaining to the law of nature. One
ſorte are as firſt principles of the law of nature, in which God neuer
diſpẽſeth, much leſſe anie man. As that one woman may not haue more
huſbands then one, becauſe the ſame would rather hinder procreation, and
ſo were directly againſt the fruict of mariage. The other ſorte are
as concluſions drowne from the firſt principles, in which God ſometimes
diſpenſeth, but neuer anie man. As in this preſent example: ſeeing it is
againſt natural procreation that one woman ſhould haue manie huſbands,
it is conuenient alſo, there being ordinarily as manie men as wemen in
the world, that euerie man likwiſe ſhould be reſtrained to one wife, for
ſo procreation may rather be increaſed, then if ſome men haue manie
wiues, and
others by that occaſion haue none at al, except in ſome ſpecial caſe. As
after the floud, when there was ſcarſetie of people, God diſpenſed with
ſuch men as in deede were like to make greater procreation by pluralitie
of wiues.
\MNote{Pluralitie of wiues ſometimes allowed.}
VVhich appeareth ſufficiently by that Sarai perſwaded her owne
huſband, to marie an other wife, and he a true ſeruant of God agreed
therto, not as a new thing but as a lawful practiſe of thoſe times. And
\CNote{Deut.~21.}
Moyſes here and in other places ſtil ſpeaketh of it, as of a cuſtome
knowne to the people for lawful. \Emph{If a man} (ſaith he) \Emph{haue
tvvo vviues, one beloued and the other hated, and they haue children by
him, and the ſonne of the hated be firſt borne, he can not preferre the
ſonne of the beloued.} Wherby is clere that two wiues were then lawful,
and the children of both legitimate, and that the firſt borne muſt be
preferred, without reſpect of firſt or laſt mariage. Yet this
diſpenſation either ceaſed before Chriſts time, the cauſe ceaſing, when
the world was repleniſhed;
\CNote{Math.~19.}
\MNote{By the law of Chriſt in no caſe lawful.}
or at leaſt our Sauiour tooke it away,
\CNote{Gen.~2.}
reſtoring Matrimonie to the firſt
inſtitution of two in one fleſh. Who pleaſeth to ſee the Doctors that
vnderſtand, and expound the Scriptures to this effect, may read
\Cite{S.~Auguſtin li.~22. c.~30. &.~47. con. Fauſt. Manich.}
\Cite{li.~16. c.~25. &~38. ciuit.}
&
\Cite{li.~1. de adulter. coniugijs.}
\Cite{S.~Chriſtom ho.~56. in Gen.}
\Cite{S.~Amb. li. de Abraham. c.~4.}
Alſo S.~Chriſoſt. S.~Hierom. and S.~Bede in 19.~Mathei.}
to wife. \V Who did companie with her, but she
\SNote{Some obey whileſt they are rude, or in low ſtate, but hauĩg got a
litle knowlege or aduancement diſdaine their aduancers.
\Cite{S.~Gregorie. li.~21. in 1.~Reg.~3.}}
perceauing that she was with childe, deſpiſed her miſtreſſe. \V And
Sarai ſaid to Abram: Thou doeſt vniuſtly againſt me: I gaue my handmaid
into thy boſome, who perceauing herſelf to be with child, deſpiſeth
me. Our Lord iudge betwen me and thee. \V To whom Abram making anſwere:
Behold, ſaith he, thy hãdmaid is in thine owne hand, vſe her as it
pleaſeth thee. When Sarai therfore did afflict her, she ranne away. \V
And an Angel of our Lord hauing found her, beſide a fountaine of water
in the wilderneſſe, which is in the way to Sur in the deſert, \V he ſaid
to her: Agar, the handmaid of Sarai, whence comeſt thou? and whither
goeſt thou? who anſwered: From the face of Sarai my miſtreſſe doe I
flye. \V And the Angel of our Lord ſaid to her: Returne
%%% 0082
to thy miſtreſſe, and humble thy ſelfe vnder her hand. \V And again:
Multiplying, ſayth he, wil I multiplie thy ſeed, and it shal not be
numbred for the multitude therof. \V And againe after that: Behold,
ſaith he, thou art with child, and thou shalt bring forth a ſonne: and
thou shalt cal his name Iſmael, becauſe the Lord hath heard thin
affliction. \V He shal be a wild man: his hand shal be againſt al men,
and al mens hands againſt him: and ouer againſt al his bretheren shal he
pitch his tents. \V And she called the name of our Lord that ſpake vnto
her: Thou the God which haſt ſene me. For she ſaid: verily here haue I
ſene the backe partes of him that hath ſene me. \V Therfore she called
that wel, the wel of him that liueth and ſeeth me. The ſame is betwen
Cadeſſe, and Barad. \V And Agar brought forth a ſonne to Abram: who
called his name Iſmael. \V Eightie and ſixe yeares old was Abram when
Agar brought him forth Iſmael.


\stopChapter


\stopcomponent


%%% Local Variables:
%%% mode: TeX
%%% eval: (long-s-mode)
%%% eval: (set-input-method "TeX")
%%% fill-column: 72
%%% eval: (auto-fill-mode)
%%% coding: utf-8-unix
%%% End:
