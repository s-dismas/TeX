%%%%%%%%%%%%%%%%%%%%%%%%%%%%%%%%%%%%%%%%%%%%%%%%%%%%%%%%%%%%%%%%%
%%%%
%%%% The (original) Douay Rheims Bible 
%%%%
%%%% Old Testament
%%%% Genesis
%%%% Chapter 03
%%%%
%%%%%%%%%%%%%%%%%%%%%%%%%%%%%%%%%%%%%%%%%%%%%%%%%%%%%%%%%%%%%%%%%




\startcomponent chapter-03


\project douay-rheims


%%% 0029
%%% o-0027
\startChapter[
  title={Chapter 3}
  ]

\Summary{By
\MNote{The ſecond part. Of the fal of man, and propogation of man and of
  ſinne.}
  the craft of the Diuel ſpeaking in a ſerpent, our firſt
  parents tranſgreſſed Gods commandment. 7.~Who being ashamed vvould
  hide them ſelues: 9.~but are reproued by God. 14.~And beſides other
  particular punishments (yet with promiſe of a Redemer) are caſt out of
  Paradiſe.}

But
\SNote{Serpẽts moſt craftie to eſcape harme when they hurt men: ſo is
the diuel.}
the ſerpent alſo was more ſubtile then al the beaſts of the
earth, which our Lord God had made. Which ſaid to the woman:
\LNote{VVhy hath God?}{Here
\MNote{Sinne entred among men by the enuie & craft of the diuel, man cõſenting
to his ſuggeſtions. Eue firſt ſinned in thought, then in words laſt in
deedes.}
we may ſee how ſinne came firſt amongſt men.
\CNote{Sap.~2,~24. Ioan.~8,~44. S.~Aug. lib.~14. de
ciuit. c.~11. Rupert. li. de Trinit. & operibus eius c.~4.}
For the diuel enuying mãs
happie ſtate tempted Eue the weaker perſon, beginning with a queſtion,
therby to allure her into conference, and by ſuch a queſtion as might
bring her into ſuſpition of Gods affection towards man,
ſaying: \Emph{VVhy hath God commanded you, that you should not eate of
euerie tree of paradiſe?} inſinuating by theſe words, and withal
internally ſuggeſting, that God dealt hardly with them, abridging their
libertie without cauſe. And when he had got ſo much of her, that ſhe was
diſpleaſed with the precept, which ſhe ſhewed by adding of her owne (to
make it ſeme more greuous) that they were forbidden \Emph{to touch the
tree}: and againe by reporting the puniſhment as doubtful,
ſaying: \Emph{leſt perhaps vve dye}, then the tempter auouched boldly,
and falſly, that they ſhould not dye, and charged God to be enuious of
the benefite they ſhould get by eating of that tree, ſaying \Emph{their
eyes should be opened, and they should be as Goddes, knowing good and
euil.} Vpon which perſwaſion, and liking alſo ſhe had to the fruite, ſhe
did take and eate, and perſwaded Adam alſo to eate.
\CNote{Lib. de vera Religione c.~14. lib.~1. Retract. c.~13.}
\MNote{Bad ſequels of ſinne.}
And forthwith they ſaw that they would not haue ſeene, knew euil which
they had better not to haue knowen, were aſhamed, and endeuored to
couer, and hide them ſelues. Euen thus the diuel dealeth with men euer
ſince, aſſaulting the weaker perſons, and weaker part, as the fleſh and
ſenſualitie, and by them ſetteth vpon the ſtronger and ſuperior part,
\MNote{No ſinne can be without freewil.}
to get conſent of freewil, without which there is no ſinne. According to
that famous ſaying of S.~Auguſtin: \L{Peccatum adeo eſt voluntarium, vt
nullo modo ſit peccatum, ſi non voluntarium.} \Emph{Sinne is ſo
voluntarie, that in no vviſe it can be ſinne, if it be not voluntarie.}
\MNote{Concupiſcẽce no ſinne, but the effect, and occaſion of ſinne.}
Wherfore it was no ſinne in Eue to be tempted by the ſerpent, which ſhe
could not auoide, nor in Adam to be tempted by Eue, but they ſinned when
they conſented to the euil ſuggeſtions.
\CNote{S.~Aug. lib.~1. de nupt. & con. c.~23.}
And now in the regenerate, though
concupiſcence remaine, which is the effect of ſinne paſt, & occaſion of
ſinne in thoſe that yeld againe to tẽtations, yet it is not ſinne, but
puniſhment of ſinne, and matter of exerciſe in the iuſt,
\MNote{Alſo occaſion of merite.}
and if we reſiſt, of merite: and therfore S.~Paul exhorteth vs,
\CNote{Gal.~5.}
\Emph{to vvalke in the ſpirite, and the luſts of the flesh vve shal not
accomplish.} And in an other place ſheweth,
\CNote{2.~Tim.~2.}
\Emph{that he vvhich fighteth lavvfully, shal be crovvned.}} Why hath
God commanded you, that
%%% 0030
you ſhould not eate of euerie tree of Paradiſe? \V To whom the woman
anſwered: Of the fruite of the trees that are in paradiſe, we doe
eate: \V but of the fruite of the tree which is in the middes of
paradiſe, God hath commanded vs that we should not eate: and that we
ſhould not touch it, leſt perhapes we die. \V
\CNote{2.~Cor.~11,~3.}
And the ſerpent ſaid to the
woman: No you ſhal not dye the death. \V For God doth know that in what
day ſoeuer you ſhal eate therof, your eyes ſhal be opened: and you ſhal
be as gods, knowing good & euil.

\V The woman therfore ſawe that the tree was good to eate, and fayre to
the eyes, and delectable to behold:
\CNote{Eccl.~25. 1.~Tim.~2,~14.}
and ſhe tooke of the fruite therof,
and did eate, and gaue to her huſband, who did eate. \V And the eyes of
them both were opened: and when they
\SNote{After ſinne they were aſhamed, not before.
\Cite{S.~Chriſ.}}
perceiued themſelues to be naked, they ſowed togeather leaues of a figge
tree, and made themſelues aprons.

\V And hearing the voice of our Lord God walking in paradiſe at the
after none ayre: Adam hid himſelfe and ſo did his wife from the face of
our Lord God, amidſt the trees of paradiſe. \V And our Lord God called
Adam, and ſaid to him: Where art thou? \V Who ſaid: I heard thy voice in
paradiſe: and I feared, becauſe I was naked, and I hid me. \V To whom he
ſaid: And who hath told thee that thou waſt naked, but that thou haſt
eaten of the tree, whereof I commanded thee that thou ſhouldeſt not
eate? \V And Adam ſaid: The woman, which thou gaueſt me to be my felow
companion, gaue me of the tree, and I did eate.

\V And our Lord God ſaid to the woman: Why haſt thou done this? who
anſwered: The ſerpent deceiued me, & I did eate. \V And our Lord God
ſaid to the ſerpent: Becauſe thou haſt done this thing,
\SNote{Al this curſe perteineth to the diuel that ſpake in the ſerpẽt.
\Cite{S.~Aug. l.~2. de Gen. ad lit. cap.~36.}
\Cite{S.~Beda in hunc locum.}}
accurſed art thou among al catle, & beaſts of the earth: vpon thy breſt
ſhalt thou goe, &
\SNote{Earthlie or worldlie and carnal men.
\Cite{S.~Greg. in Pſal.~101.}}
earth ſhalt thou eate al the dayes of thy life. \V I wil put enmyties
betwen thee & the woman, and thy ſeed and the ſeed of her:
\LNote{She shal bruiſe.}{Proteſtants
\MNote{The Latin text defended againſt Kemniſius and other Proteſtants.}
wil not admitte this reading, \L{ipſa conteret}, ſhe ſhal bruiſe, leſt
our Bleſſed Ladie ſhould be ſaid anie way to bruiſe the ſerpents
head. And Kemniſius amongſt others ſaith, that al ancient Fathers
read, \L{ipſum}, not, \L{ipſa}. But he is conuinced of lying by
\Cite{Claudius Marius Victor. lib.~1. in Gen.}
\Cite{Alcimus Auitus lib.~3. carm. c.~6.}
\Cite{S.~Chriſoſtom hom.~17. in Geneſ.}
\Cite{S.~Ambroſe lib. de fuga ſæculi cap.~7.}
\Cite{S.~Auguſtin lib.~2. de Geneſi. contra Manichæos cap.~18.} &
\Cite{lib.~11. de Geneſi. ad literam cap.~26.}
\Cite{S.~Gregorie lib.~1. Moralium cap.~38.}
\MNote{See
\Cite{Card. Bellarmin li.~2. c.~12. de verbo Dei.}}
And after them S.~Bede, Eucherius, Rabanus, Rupertus, Strabus, and Lira
vpon this place,
\Cite{S.~Bernard ſer.~2. ſuper Miſsus eſt.}
And manie others, who read \L{ipſa} as the Latin text now hath.

But
\MNote{Both readings yeld the ſame ſenſe.}
whether we read, \Emph{she shal bruiſe}, or, \Emph{her ſede}, that is
her ſonne Chriſt, \Emph{shal bruiſe} the ſerpents head, we attribute no
more, nor no leſſe to Chriſt, nor to our Ladie by the one reading, then
by the other: for by the text, \Emph{I vvil put enmities betvven thee and
the vvoman, betvven thy ſeede, and her ſeede.} It is clere, that this
enmitie and battle pertained to the woman and her ſeede on the one
partie, and to this diuel, that ſpake by the ſerpent, and al the wicked,
on the other partie, and that the victorie ſhould happen to
mankind.
\MNote{As Adam was the cauſe, and Eue an occaſion of mans captiuitie: ſo
Chriſt is the true cauſe and his mother an occaſion of our reſtauration.}
VVhich being captiue by Adams ſinne, occaſioned by a woman, ſhould be
redeemed, both ſexes, though in farre different ſorte, concurring
therto. And ſo it is moſt true, that Chriſt by his owne proper powre, and
his bleſſed mother by her moſt immediate cooperating to his Incarnation
(and conſequently to other Miſteries) did bruiſe the ſerpents head,
breake and vanquiſh his powre.
\CNote{S.~Ireneus li.~3. c.~33. & lib.~5. circa
med. S.~Epiph. Hær.~78. S.~Ieron. ep.~22. ad Euſtoch. S.~Aug. (or
S.~Fulgens) ſer.~18. de Sanctis. de fide & Symb. de Agone Chriſtiano.}
As manie ancient Fathers do excellently
diſcourſe: namely S.~Bernard, writing vpon theſe wordes in the
\XRef{Apocalips. cap.~12.}
\Emph{A great ſigne appeared in heauen, a vvoman clothed vvith the
ſunne}: Albeit (ſaith he) by one man and one woman we were greatly
damaged: yet (God be thanked) by one man and one woman al loſſes are
repaired, and that not without great increaſe of graces. For the
benefite doth farre excede the loſſe. Our merciful father geuing vs for
a terreſtrial Adam Chriſt our Redemer, & for old Eue Gods owne mother.
\MNote{Our B.~Ladie reſiſted al euil ſuggeſtions.}
Moreouer as the ſame
\CNote{Ser.~2. ſuper Miſſus eſt.}
S.~Bernard ſheweth, this bleſſed Virgin in ſingular
ſorte bruiſed the ſerpents head, in that ſhe quite vanquiſhed al maner
ſuggeſtions of the wicked ſerpẽt, neuer yelding to, nor taking delight
in anie euil moued by him.}
ſhe ſhal bruiſe thy head in peeces, & thou ſhalt lye in waite
%%% o-0028
\SNote{Though good men reſiſt tentations at the firſt aſſaults, and ſo
bruiſe the ſerpẽts head, yet he ẽdeuoreth ſtil to deceiue eſpecially in
the end of mans life, ſignified by the heele.
\Cite{S.~Gre. in cap.~1. Iob.}}
of her heele. \V To the woman alſo he ſaid: I wil multiplie thy
trauailes, and thy child bearinges: in trauaile ſhalt thou bring forth
children, and
\CNote{1.~Cor.~14.}
thou ſhalt be vnder thy huſbands power, and he ſhal haue
dominion ouer thee. \V And to Adam he ſaid: Becauſe thou haſt heard the
voice of thy wife, and haſt eaten of the tree, whereof I
%%% 0031
cõmanded thee, that thou ſhouldeſt not eate, curſed is the earth in thy
woorke: with
\SNote{Al men trauel one way or other: & ſuch as ſuffer wides to
ouergrowe (in their ſouls) ſhal after this life either
ſuſtaine the fyre of Purgatorie or eternal paine.
\Cite{S.~Aug. li.~2. c.~20. de Gen. cõ. Man.}}
much toyling ſhalt thou eate thereof al the dayes of thy life. \V
Thornes and thyſtles ſhal it bring forth to thee, & thou ſhalt eate the
herbes of the earth. \V In the ſweat of thy face ſhalt thou eate bread,
til thou returne to earth, of which thou waſt taken: becauſe
\LNote{Duſt thou art.}{By theſe wordes Adam was admoniſhed to humble him
ſelfe, conſidering the matter wherof his bodie was made, and into which
he ſhould be reſolued againe.
\CNote{Iob.~42. Eſai.~58. Ierem.~6. Ionæ.~3. Mat.~11.}
Wherupon it came to be a ceremonie amongſt penitents, to caſt aſhes on
their heads. As appeareth in holie Scriptures.
\MNote{The ceremonie of aſhes, on Aſhweneſday.}
For which cauſe the Church now alſo vſeth this ceremonie the firſt day
of Lent, putting aſhes on her childrens heades: willing them to
remember, that duſt they are, and to duſt they ſhal returne, to moue vs
by this meditation to more ſerious penance.}
duſt thou art, and into duſt thou ſhalt returne.

\V And Adam called the name of his wife, Eue: for becauſe
\SNote{She was mother rather of al the dying: but in figure of our
B.~Lady who is mother of Chriſt, life itſelfe, ſhe is called mother
of the liuing.
\Cite{S.~Epiph. her.~78.}}
ſhe was mother of al the liuing. \V Our Lord God alſo made for Adam and
his wife garments of ſkynnes, and clothed them. \V And ſaid: Loe Adam is
become as it were one of vs, knowing good & euil: now therfore
\LNote{Leſt perhaps.}{Notwithſtanding
\MNote{Gods prouidence concurreth with mans free wil.}
Gods eternal decree in diſpoſing al thinges, and his omnipotencie which
nothing can reſiſt, yet he produceth good, and either auoideth or diſpoſeth of
euil which he ſuffereth, by ordinarie meanes, as appeareth
\XRef{Act.~27. v.~31.}
and that becauſe man hath freewil, with which God concurreth, &
deſtroyeth not nor forceth, as
\CNote{de grat. & liber. arb. c.~6. de corrept. & grat. ad art. falſo impoſ.}
S.~Auguſtin teacheth.}
leſt perhapes he reach forth his hand, and take alſo of the tree of
life, & eate, and liue for euer. \V And our Lord God ſent him out of the
paradiſe of pleaſure, to worke the earth of which he was taken. \V And he
caſt out Adam: and
\LNote{Placed Cherubins.}{Man
\MNote{Paradiſe defended by Angels and by fire & ſworde.}
being caſt out of paradiſe, the ſame is defended with duble gard, with
Angels, that are watchful, wiſe, and potent: and with fire and ſword,
moſt terrible armoure to man. Wherby againe we ſee, that God vſeth
ordinarie meanes in his prouidence, as the miniſtrie of Angels & humane
terror, and would neither deſtroy the tree, nor depriue it of the vertue
to prolong life, nor bereue man of freewil, by which he might deſire to
returne:
\MNote{God deſtroyeth not nature.}
but conſeruing nature in al creatures, preuenteth inconueniences
otherwiſe.

Theſe
\CNote{S.~Aug. lib.~11. de Gen. ad lit. c.~40.}
\MNote{Good Angels hinder diuels of their deſires.}
Angels alſo hinder the diuel, that he can not enter paradiſe, leſt
he ſhould take of the fruite of the tree, and geue it to men to prolong
their liues, and therby draw them to his ſeruice.}
placed before the paradiſe of pleaſure Cherubins, & a flaming, and a
turning ſworde, for to keepe the way of the tree of life.


\stopChapter


\stopcomponent


%%% Local Variables:
%%% mode: TeX
%%% eval: (long-s-mode)
%%% eval: (set-input-method "TeX")
%%% fill-column: 72
%%% eval: (auto-fill-mode)
%%% coding: utf-8-unix
%%% End:
