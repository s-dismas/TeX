%%%%%%%%%%%%%%%%%%%%%%%%%%%%%%%%%%%%%%%%%%%%%%%%%%%%%%%%%%%%%%%%%
%%%%
%%%% The (original) Douay Rheims Bible 
%%%%
%%%% Old Testament
%%%% Genesis
%%%% Chapter 02
%%%%
%%%%%%%%%%%%%%%%%%%%%%%%%%%%%%%%%%%%%%%%%%%%%%%%%%%%%%%%%%%%%%%%%




\startcomponent chapter-02


\project douay-rheims


%%% 0026
%%% o-0023
\startChapter[
  title={Chapter 2}
  ]

\Summary{The worke of ſix dayes being finished, God reſted the ſeuẽth
  day & bleſsed it. 8.~Then placing man in paradiſe (planted with bewtiful
  & ſwete trees, & watered with foure riuers) 16.~cõmandeth him not to
  eate of the tree of knowledge of good & euil, 18.~& formed a woman of
  a ribbe of Adam.}

The heauens therfore & the earth were fully finiſhed, and al the
furniture of them. \V And the ſeuenth day God ended his woorke which he
had made: &
\CNote{Exod.~20,~11. Deut.~5,~14. Heb.~4,~4.}
\SNote{God createth not new kindes of creatures, yet ſtil worketh.
\XRef{Io.~5,~17.}
conſeruing & gouerning al things and
\Fix{createh}{createth}{possible typo, same in both}
ſoules, grace, and glorie
of the ſame kind.
\Cite{S.~Aug. li.~4. de Gen. ad lit. c.~12.}}
reſted
\LNote{The ſeuenth day.}{Al creatures
\Fix{benig}{being}{obvious typo, fixed in other}
made in their kindes in ſix dayes, complete and perfect, God not neding
(as men often do in their workes) to perfect, pooliſh, or amend the
ſame, \Emph{reſted the ſeuenth day}: and therfore the natural perfection
of Gods workes is attributed to the ſeuenth day, and the ſupernatural
perfecting of men in eternal life, after the Reſurrection, is attributed
to the eight day, as
\CNote{in Pſal.~6. &~11.}
S.~Auguſtin and other
\Fix{fathers}{Fathers}{possible typo, same in other}
teach.
\MNote{Obſeruation of holie dayes by Gods inſtitution.}
And for this cauſe \Emph{God bleſſed and ſanctified the ſeuenth
day}, and after we haue in the Decalogue, or tenne commandments, that
this day al ſhould reſt and abſtaine from workes, yea and kepe it
feſtiual, occupying them ſelues in ſpiritual exerciſes ſeruice and
ſpecial worſhipe of God,
\CNote{Act.~13,~14. Leuit.~23.}
as the Iewes did euen til Chriſts, and his
Apoſtles time, praying and hearing the word of God read and expounded in
the Sabboth day.
\MNote{Obſeruatiõ of feſtiual dayes is religious, not Iudaical, nor
heathniſh.}
VVherby we ſee that diſtinction of dayes pertayneth to Religion, the
people of God thus obſeruing the Sabboth in memorie of the Creation, &
diuers other feaſtes in memorie of other benefites. And we now kepe the
Sunday holie, in memorie of Chriſts Reſurrection, and other feaſtes in
gratful remembrance of other Myſteries of Chriſts Natiuitie, the coming
of the Holie Ghoſt, and the like. Yea alſo feaſtes of his bleſſed
Mother, and other Sainctes, for the benefites receiued from Chriſt by
them, and for more honour to Chriſt in them. So this Catholique
obſeruation of feaſtes is neither Iudaical (which alſo in the law was
good but now is abrogated) nor heathniſh, for we honour not Iupiter,
nor Iuno, nor anie falſe god or goddeſſe, but our Lord God Creator &
Redemer, & for his ſake, his beſt ſeruants.
\MNote{Honour of Sainctes is to the greater honour of Chriſt.}
VVherof ſee the
\Cite{Annotations in the Engliſh new Teſtament, 4.~chap. to the
Galatians.}
VVherto we here only adde theſe wordes of 
\CNote{Homil. in 40.~Martyres.}
S.~Baſil. VVhich may ſerue for
a general anſwer to the moſt common obiection. \L{Honor ſeruorum
redundat in communem Dominum.} \Emph{The honour of the ſeruantes
redoundeth to the common Lord, or Maiſter.} So, ſaith he, the honour of
Sainctes is the honour of Chriſt their Lord and ours.}
the ſeuenth day, from al woorke that he had done. \V And he bleſſed the
ſeuenth day and ſanctified it: becauſe in it he had ceaſed from al his
woorke which God created to make.

%%% o-0024
\V Theſe are the generations of heauen & earth, when they were created
in the day, when our Lord God made the heauen, and the earth. \V And
euery plant of the filde, before it ſhot vp in the earth. And euerie
herbe of the ground before it ſprang: for our Lord God had not rayned
vpon the earth: and man was not to til the earth: \V But a ſpring roſe
out of the earth, watering al the ouermoſt part of the earth. \V Our
Lord God therfore formed man of the ſlyme of the earth: and
\SNote{Mans ſoule is immediatly created by God, not produced of other
ſubſtance as the ſoules of beaſtes and plants are.}
breathed into his face the breath of life, &
\CNote{1.~Cor. 15,~45.}
man became a liuing ſoule.

\V And our Lord God had
\SNote{VVhether this paradiſe be now extant is vncertayne, though it be
certaine that Enoch and Elias are yet liuing in earth.
\Cite{S.~Aug. li.~2. cont. Pelagi. c.~23.} See
\Cite{Perereus. li.~3. q.~5. & li.~7. q.~vltima.}}
planted a Paradiſe of pleaſure from the beginning: wherin he placed man
whom he had formed. \V And our Lord God brought forth of the ground al
maner of trees, fayre to behold: and pleaſant to eate of: the tree of
life alſo in the middle of Paradiſe: and the tree of knowledge of good &
euil. \V And a riuer iſſued out of the place of pleaſure to water
Paradiſe, which from thence is diuided into four heades. \V The name of
the one is Phiſon: that is it which compaſſeth al the land of Heuilath,
where gold groweth. \V And the gold of that land is very good:
%%% 0027
there is found bdelium, & the ſtone onyx. \V And the name of the ſecond
riuer is Gehon: that is it which compaſſeth al the land of Ethiopia. \V
And the name of the third riuer is Tygris: that ſame paſſeth along by
the Aſſirians. And the fourth riuer, the ſame is Euphrates.

\V Our Lord God therfore tooke man, & put him in the Paradiſe of
pleaſure, to woorke, & keepe it. \V And he commanded him ſaying: Of
euerie tree of Paradiſe eate thou: \V But
\LNote{Of the tree of knovvledge.}{Beſides the law of nature, by which
Man was bound to direct al his actions according to the rule of reaſon;
and beſides the ſupernatural diuine law, by which he was bound to
beleue, and truſt in God, and to loue him aboue al things, hauing
receiued the giftes of faith, hope, and charitie: God gaue him an other
particular law, that \Emph{he should not eate of the tree of knovvlege
of good and euil.}
\MNote{VVhy a particular poſitiue law beſides the general lawes of God &
nature, was geuen to man.}
And that for two ſpecial reaſons, which
\CNote{lib.~8. de Gen. ad lit. c.~11. Pſal.~15.}
S.~Auguſtin noteth vpon this place.
\MNote{Firſt reaſon.}
Firſt, that God might declare him ſelfe to be Lord of man. VVhich was
abſolutely neceſſarie for man, and nothing at al profitable to God, who
nedeth not our ſeruice, but we without his dominion ſhould vtterly fal
to nothing. \L{Nec enim ipſo non creante &c.} \Emph{For he not creating
vs, neither could vve haue bene, nor he not conſeruing vs, could vve
remayne, nor he not gouerning vs, could vve liue rightly. VVherfore he
onlie is our true Lord, vvhom not for him, but for our ovvne profite and
ſaluation vve ſerue.}
\MNote{2.~reaſon.}
The other reaſon was, that God might geue man matter wherin to exerciſe
the vertue of obedience, and to ſhew him ſelfe a ſubiect of God. VVhich
could not be ſo properly and effectually declared by keping other lawes,
nor the enormitie of diſobedience appeare ſo euidently, as by fulfilling
of Gods wil commanding him, or by doing his owne wil, moued to the
contrarie, in a thing of it ſelfe indifferent, & only made vnlawful,
becauſe it was forbid. But let vs heare S.~Auguſtins owne wordes. 
\MNote{The ſinne of diſobedience.}
\L{Nec potuit melius aut diligentius cõmendari quantum malum sit sola
inobedientia, &c.}
Neither could it (ſaith this great Doctor) be better, nor more exactly
ſignified how bad a thing ſole diſobedience is, then where a man became
guiltie of iniquitie, becauſe he touched that thing contrarie to
prohibition, which if he, not forbidden, had touched, he had not ſinned
at al.
\MNote{Ioyned with damage to him that diſobeyeth.}
For he that ſaith, for example ſake, Touch not this herbe,
ſuppoſing it is poyſenful, and doth forwarne one of death, if he touch
it, death aſſuredly falleth on the contemner of the precept: yea though
no man had prohibited, and he had touched, for he ſhould dye becauſe the
ſame thing bereueth him of health and life, whether it had benne
forbidden him or no.
\MNote{Ioyned with damage of him that forbiddeth.}
Alſo when one forbiddeth that thing to be touched, which would not in
dede preiudice him that toucheth, but him that forbiddeth, as if one
take an others money, being forbid by him, whoſe the money is, it is a
ſinne in him that is forbidden, becauſe it is iniurie to him that
forbiddeth. But when that thing is touched which neither ſhould hurt
him that toucheth, nor any other, if it were not forbid, wherfore is it
prohibited, but that the proper goodnes of obedience, and the euil of
diſobedience might appeare? Thus S.~Auguſtin ſheweth, that diſobedience
is a ſinne, becauſe it is againſt a precept, though otherwiſe the thing
that is done were not euil.
\MNote{True obedience is blind and prompt.}
And amongſt other good notes, teacheth that true obedience inquireth
not, wherfore a thing is commanded, but leauing that to the Superior,
promptly doth that is appointed.}
of the tree of knowledge of good & euil 
%%% !!! Note not marked in text.
\LNote{Of the tree eate thou not.}{This
\CNote{Math.~9. Luc.~10. S.~Epiph. in compẽ. fidei Cat. S.~Aug. epiſt.~80.}
\MNote{Not meate, but the diſobedience hurteth him that tranſgreſſeth
the precept of abſtinence.}
example of our firſt parents tranſgreſſion ſheweth, how friuolous an
anſwer it is to ſay; that breaking of commanded faſtes, or eating meates
forbidden can not hurt vs, the meate being good and holſome: for ſo the
fruite of the tree was good, and ſhould haue hurt no man, if it had not
benne forbidden. Euen ſo al meates of their owne nature are good, yet
the precept of faſting (foretold by our Sauiour in general, and
determined by his Church in particular) and ſo of
\MNote{Lawes in things indifferent bind in conſcience.}
anie other like law,
though it be in things otherwiſe indifferent, proceeding from lawful
Superiors, bindeth the ſubiects in conſcience. And the tranſgreſſion is
properly diſobedience, what other ſinne ſoeuer may alſo be mixed
therwith.}
eate thou not. For in what day ſoeuer thou ſhalt eate of it,
\LNote{Thou shalt dye the death.}{Againſt
\CNote{Caluin l.~3. inſt. c.~4. parag.~31. &~32.}
\MNote{Temporal puniſhment due after ſinne is remitted.}
the new doctrine, denying that after ſinne is remitted, anie temporal
puniſhment remaineth for the ſame, this place declareth that death
(wherof God forewarned Adam, if he ſhould eate of the fruite forbidden)
remained due, and was at laſt inflicted vpon him, for his ſinne, which
was preſently remitted vpon his repentance.

Againe
\CNote{Rom.~5.}
\MNote{Death due to al for Original ſinne.}
for ſo much as we are al ſubiect to death, it proueth that we
were al guiltie of this ſinne, by which death came vpon al men, as
S.~Paul teacheth. Els God ſhould puniſh vs without our fault, which is
vnpoſſible that his goodnes ſhould do. 
\MNote{Yea to infants who haue no other ſinne.}
Eſpecially it appeareth in
infants, who dying before they come to vſe of reaſon, can neuer cõmit
other ſinne: for though they were circumciſed, or had Sacrifice offered,
or other remedie vſed for them before Chriſt, or baptiſed ſince Chriſt:
\MNote{Alſo other penalties inflicted vpon infants.}
yet they ſuffer (as
\CNote{lib.~13. de ciuit. c.~6. lib.~2. de pec. mre. & remiſ. c.~34.}
S.~Auguſtin noteth) both death and manie other
penalties, of ſicknes, cold, heate, hunger, and the like, which can
neither be to them matter of merite (as to others it may be) nor profite
them for auoiding of other ſinnes, ſeing they dye in their infancie. Yea
moreouer if they dyed without
\CNote{Gen.~17.}
circumciſiõ, or other remedie of thoſe
former times, their ſoules periſhed from their people; and
\CNote{Ioan.~3. S.~Greg. li.~4. Moral. c.~2.}
now without Baptiſme can neuer enter into the
\Fix{kingdome}{Kingdome}{likely typo, fixed in other}
of
\Fix{heauen,}{Heauen,}{likely typo, fixed in other}
which could not ſtand with Gods iuſtice, if they were not guiltie of
ſinne.}
thou ſhalt dye the death. \V Our Lord God alſo ſaid: It is not good for
man to be alone: let vs make him a helpe like vnto him ſelfe. \V Our
Lord God therfore hauing formed of clay al beaſtes of the earth, and
foules of the ayre, brought them to Adam that he might ſee what to cal
them: for al that Adam called any liuing creature, the ſame is his name.

\V And Adam called al beaſtes by their names, and al foules of the ayre,
and al cattel of the filde: but vnto Adam there was not found an helper
like him ſelfe. \V Our Lord God therfore caſt a dead ſleepe vpon
Adam: and when he was faſt a ſleepe,
\CNote{Mat.~19,~5. Mar.~10,~7. 1.~Cor. 6,~16. Eph.~5,~31.}
he tooke one of his ribbes, & filled vp fleſh for it. \V And our Lord
God
\SNote{As we ſay brick is made of earth, and a houſe is built of bricke: ſo
Adam was made of earth and Eue built of a ribbe of Adam. And that of one
ribbe, as if God should build a houſe of one bricke, or as in dede he
fedde
\Fix{500.}{5000.}{obvious typo, fixed in other}
men with fiue loaues.
\Cite{Chriſ. ho.~15.}
\Cite{S.~Aug. Tract.~24. in Ioan.}
\Cite{S.~Tho. p.~1. q.~92. a.~3.}}
built the ribbe which he tooke of Adam into a woman: and brought her to
Adam. \V And Adam ſaid: This now, is bone of my bones, and fleſh of my
fleſh: ſhe ſhal be called woman, becauſe ſhe was taken out of man. \V
Wherfore man ſhal leaue his father & mother, & ſhal cleaue to his wife,
& they ſhal be
\SNote{Not three, nor foure, nor more, for then two were chãged to an
other number.
\Cite{S.~Ier. li.~1. cont. Ioui.}}
two in one fleſh. \V And they were both naked; to wit Adam & his wife:
and were not aſhamed.


\stopChapter


\stopcomponent


%%% Local Variables:
%%% mode: TeX
%%% eval: (long-s-mode)
%%% eval: (set-input-method "TeX")
%%% fill-column: 72
%%% eval: (auto-fill-mode)
%%% coding: utf-8-unix
%%% End:
