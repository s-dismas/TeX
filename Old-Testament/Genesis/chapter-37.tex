%%%%%%%%%%%%%%%%%%%%%%%%%%%%%%%%%%%%%%%%%%%%%%%%%%%%%%%%%%%%%%%%%
%%%%
%%%% The (original) Douay Rheims Bible 
%%%%
%%%% Old Testament
%%%% Genesis
%%%% Chapter 37
%%%%
%%%%%%%%%%%%%%%%%%%%%%%%%%%%%%%%%%%%%%%%%%%%%%%%%%%%%%%%%%%%%%%%%




\startcomponent chapter-37


\project douay-rheims


%%% 0135
%%% o-0126
\startChapter[
  title={Chapter 37}
  ]

\Summary{Ioſeph
\MNote{The ſeuenth part of this booke. How Ioſeph was ſold into Ægypt,
  and there aduanced.}
informing his father of his brethrens faults, 5.~and
  telling his dreames, is by them more hated. 13.~Being ſent to viſite
  them, 18.~they firſt thinke to kil him, 26.~but by Iudas coũſel ſel
  him to the Iſmaelites, 29.~vnwiting to Ruben. 33.~His father
  lamenteth ſuppoſing him to be ſlaine by ſome wild beaſt. 36.~He is
  ſold againe to Putiphar in Ægypt.}

And Iacob dwelt in the land of Chanaan, wherin his father ſoiourned. \V
And
\SNote{Theſe things folowing hapned to Iacob, in his generations, that
is in his childrẽ. See
\Cite{S.~Chriſoſt. ho.~23. in Gen.}}
theſe are his generations: Ioſeph when he was ſixtene yeares old, fed
the flock with
%%% 0136
his brethren being yet a boy: and he was with the ſonnes of Bala and
Zelpha his fathers wiues: and he accuſed his brethren to his father of
\SNote{That for il life they were infamous, the hebrew word \HH{dibba}
ſignifieth \Emph{infamie.}}
a moſt wicked crime. \V And Iſrael loued Ioſeph aboue al his ſonnes,
becauſe he had begotten him
\LNote{In his old age.}{This
\MNote{The leaſt offenſiue cauſe is alleaged, why Iacob loued Ioſeph
aboue his bretheren.}
being one cauſe why Iacob loued Ioſeph aboue al his other ſonnes, for
that he was the youngeſt of the eleuen (for Beniamin the twelfth was yet
an infant) it is alleaged in holie Scripture (ſaith S.~Chriſoſtom
\Cite{Epiſt. ad Olympian)}
as leaſt offenſiue to his bretheren. For a more ſpecial cauſe was, for
his mother Rachels ſake, but moſt principal cauſe of al was, for his
great vertues, and mature iudgement; for which God alſo preferred him
aboue them al, and now forſhewed the ſame by viſions in ſleepe. VVhich
they enuying and meaning to preuent, did in dede vnwitting cooperate
therto,
\MNote{God turneth euil to good effect.
\Cite{S.~Aug. li.~14. c.~27. ciuit.}}
Gods prouidence turning their euil worke to infinite good. As the ſame
holie Ioſeph truly interpreteth it to them, after their fathers death,
when they iuſtly feared reuenge, for ſo great and inhumane iniuries done
vnto him.
\XRef{chap.~50. v.~20.}}
in his old age: and he made him a coate of diuers coloures. \V And his
brethren ſeing that he was loued of his father, more then al his ſonnes,
they hated him, neither could they ſpeake any thing to him peacably. \V
It chanced alſo that he reported to his brethren a dreame, that he had
ſeene: which occaſion was the ſeed of greater hatred. \V And he ſaid to
them: Heare my dreame which I haue ſeene: \V I thought we bounde ſheaues
in the field: and my ſheafe aroſe as it were, and ſtood, and your ſheaues
ſtanding about did adore my ſheafe. \V His brethren anſwered: What ſhalt
thou be our king? or ſhal we be ſubiect to thy dominion? This occaſion
of his dreames and wordes miniſtred nourishment to the enuie and
hatred. \V He ſawe alſo an other dreame, which telling his brethren, he
ſaid: I ſawe in a dreame, as it were the ſunne, and the moone, and
eleuen ſtarres adore me. \V Which when he had reported to his father,
and brethren, his father rebuked him, and ſaid: What meaneth this dreame
that thou haſt ſeene? why ſhal I and thy mother, and thy brethren adore
thee vpon the earth? \V His brethren therfore enuyed him: but
\SNote{Brothers eaſily enuie eech other: but the parents are glad of
their childrens aduancement.}
his father conſidered the thing with him ſelfe. \V And when his brethren
abode in Sichem, feeding their fathers flockes, \V Iſrael ſaid to him:
Thy brethren feed ſheepe in Sichem: come, I wil ſend thee to them. Who
anſwering, \V I am readie; he ſaid to him: Goe, and ſee if al things be
wel with thy brethren, and the ſheepe: and bring me word againe what
they doe. Being ſent therfore from the Vale of Hebron, he came to
Sichem: \V and a man found him there wandering in the field, and asked
what he ſought. \V But he anſwered:
\SNote{So Chriſt, & al good Paſtors.}
I ſeeke my brethren, ſhew me where they fede the flockes. \V And the
%%% o-0127
man ſaid to him: They are departed from this place: for I heard them
ſay: Let vs goe into Dothain. Ioſeph therfore went forward after his
brethren, and found them in Dothain. \V Who when they had ſeene him a farre
of, before he came nighe them, they deuiſed to kil him: \V and ſpake
among them ſelues: Behold the dreamer commeth, \V come, let vs kil him,
and caſt him into an old
%%% 0137
ceſterne: and we wil ſay: A naughtie wild beaſt hath deuoured him:
\SNote{So the Iewes thinking to preuẽt Chriſts exaltation cooperated
vnwitting therto.
\Cite{Proſper. li. de promiſſ. Dei.}}
and then it ſhal appeare what his dreames doe profite him. \V And Ruben
hearing this, endeuored to deliuer him out of their hands, and ſaid: \V
Do not take away his life, neyther ſheed ye blood: but caſt him into
this ceſterne, that is in the wilderneſſe, and keepe your handes
harmeles: and he ſaid this, deſirous to deliuer him out of their handes,
and to reſtore him to his father. \V As ſoone therfore as he came vnto
his brethren, forthwith they ſtripped him out of his ſide coate, and of
diuers colours. \V And caſt him into the old ceſterne, that had not
water. \V And ſitting to eate bread, they ſaw Iſmaelites wayfaring men
cõming from Galaad, and their camels carying ſpices, and roſen, and
mirrh into Ægypt. \V Iudas therfore ſaid to his brethren: What auaileth
it vs if we kil our brother, and conceale his bloode? \V It is better
that he be ſold to the Iſmaelites, and that our handes be not polluted:
for he is our brother and our fleſh. His brethren aſſented to his
wordes. \V And when the Madianite marchants paſſed by, they drawing him
out of the ceſterne, ſold him to the Iſmaelites, for
\SNote{Some read \Emph{thirtie}: And as the reading is diuers, ſo Chriſt
whom Ioſeph ſignified is more & leſſe eſtimed of diueres.
\Cite{S.~Aug. Ser. 81. de temp.}}
twentie peeces of ſiluer, who brought him into Ægypt. \V And Ruben
returning to the ceſterne, findeth not the boy: \V and renting his
garments went to his brethren, and ſaid: The boy doth not appeare, and
whither shal I goe? \V And they tooke his coate, and dipped it in the
blood of a kidde, which they had killed: \V ſending ſome that should
carie it to their father, and should ſay: This we haue founde: ſee
whether it be thy ſonnes coate, or no. \V Which when the father
acknowledged, he ſaid: It is my ſonnes coate, a naughtie wild beaſt hath
eaten him, a beaſt hath deuoured Ioſeph. \V And tearing his garments,
did on ſackcloth, mourning his ſonne a great time. \V And al his
children being gethered together to aſſwage their fathers ſorowe, he
would not take comforte, but ſaid: I wil deſcend vnto my ſonne
\LNote{Into hel mourning.}{Proteſtants
\MNote{Graue for hel corruptly trãſlated.}
denying more places for ſoules after this life, the Heauen for the iuſt,
and Hel for the wicked, tranſlate the hebrew
word \HH{Sheol}, \Emph{graue} for \Emph{hel}. Becauſe if they ſhould
grant that Iacob, or other holie fathers of the old Teſtament deſcended
into hel, they muſt confeſſe ſome other hel, then where the damned are
tormented, whither no Chriſtian wil ſay that thoſe fathers went. If they
contended only about the ſenſe and meaning of the text, it were more
tolerable, for therin they ſpeake, according to their erronious opinion,
as they thinke.
\CNote{See S.~Hiero. Ep.~119. S.~Aug. li.~20. c.~15. ciuit.}
\MNote{VVilful corruption.}
But knowing as ſome of them doe, that \Emph{Hel} is the true word of the
text, there is no ſinceritie nor moral honeſtie in putting \Emph{Graue},
in place therof. And that they know it, the ſecond table of the Bible,
printed at London 1602. witneſſeth, noting for a common place, that in
the
\XRef{37.~chap. of Geneſis. v.~35.}
\Emph{Hel is taken for graue}, therby confeſſing, that the true Engliſh
word of the holie Scripture in that place is \Emph{Hel}, but that they
would haue it to ſignifie graue. VVherupon anie reaſonable man would
thinke to finde the word \Emph{Hel} in the text, with ſome gloſſe to ſhew
that graue were to be vnderſtood. But in al their Editions, alſo in that
which was printed the yeare next folowing, 1603. wherto the ſame table
is adioyned, they reade \Emph{graue}, and not \Emph{hel} in that place,
though in ſome
\CNote{Nu.~16. 2.~Reg.~22. Iob.~17. Pſal.~15, 17,~85.}
other places, they much diſagree in tranſlating the ſame word.

As for the ſenſe,
\MNote{Iacob ſpake of hel not of graue.}
it can not be that Iacob ment the graue: for when he ſaid he would goe
to his ſonne, he ſuppoſed him to be deuoured by a wild beaſt, and not
buried in a graue. And therfore muſt neceſſarily meane, that he would
goe where he thought the ſoule of his ſonne to be. VVhich was neither in
heauen, for then he would rather haue aſcended thither ioyful, then
deſcended to anie place mourning; but to a lowe place, where the iuſt
ſoules then remained in reſt, which was called or
\CNote{Luc.~16.}
\MNote{Abrahams boſome.}
\L{Limbus Patrum}, or
Abrahams boſome. That is, ſaith S.~Auguſtin, in his anſwere to Biſhop
Euodius
\Cite{(Epiſt.~99.)}
\L{ſecretæ cuiuſdam quietis habitatio.} The habitation of a certaine
ſecret reſt.}
into hel, mourning. And whileſt he perſeuered in weeping, \V the
Madianites ſold Ioſeph in Ægypt to
\Fix{Phutiphar}{Putiphar}{possible typo, fixed in other}
an Eunich of Pharoes maiſter of the ſouldiars.


\stopChapter


\stopcomponent


%%% Local Variables:
%%% mode: TeX
%%% eval: (long-s-mode)
%%% eval: (set-input-method "TeX")
%%% fill-column: 72
%%% eval: (auto-fill-mode)
%%% coding: utf-8-unix
%%% End:
