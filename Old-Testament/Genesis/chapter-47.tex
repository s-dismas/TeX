%%%%%%%%%%%%%%%%%%%%%%%%%%%%%%%%%%%%%%%%%%%%%%%%%%%%%%%%%%%%%%%%%
%%%%
%%%% The (original) Douay Rheims Bible 
%%%%
%%%% Old Testament
%%%% Genesis
%%%% Chapter 47
%%%%
%%%%%%%%%%%%%%%%%%%%%%%%%%%%%%%%%%%%%%%%%%%%%%%%%%%%%%%%%%%%%%%%%




\startcomponent chapter-47


\project douay-rheims


%%% 0161
%%% o-0148
\startChapter[
  title={Chapter 47}
  ]

\Summary{Iacob with his ſonnes being come into Geſsen, Pharao granteth
  them the ſame place to dwel in. 13.~The famine forceth the Ægyptians
  to ſel al their goods, landes, and poſſeſsions to the King, 22.~except
  the Prieſts part, to whom the king aloweth neceſsarie foode, without
  paying for it. 27.~After ſeuentene yeares Iacob adiureth Ioſeph, to
  burie him amongſt his anceſters.}

Ioſeph therfore going in told Pharao, ſaing: My father & brethren, their
ſheepe and heardes, & al thinges that they poſſeſſe, are come out of the
Land of Chanaan: & behold they ſtay in the Land of Geſſen. \V The vtmoſt
alſo of his brethren fiue perſons he preſented before the king: \V whom
he asked: What trade haue you? They anſwered: We thy ſeruantes are
paſtours of ſheepe, both we, and our fathers. \V We are come to ſoiourne
in the land, becauſe there is no graſſe for thy ſeruantes flockes, the
famine being very ſore in the land of Chanaan: and we deſire thee to
command that we thy ſeruantes may be in the Land of Geſſen. \V And the
King therfore ſaid to Ioſeph: Thy father and thy brethren are come to
thee. \V The Land of Ægypt, is in thy ſight: make them to dwel in the
beſt place, and deliuer them the Land of Geſſen. And if ſo be thou knowe
that there are induſtrious men among them, appoint them maiſters of my
cattel. \V After this Ioſeph brought in his father to the King, and ſet
him before him: who bleſſing him, \V and being asked of him: How manie
be the dayes of the yeares of thy life? \V He anſwered: The dayes of
the pilgrimage of my life are an hundred thirtie yeares,
\SNote{Euerie mans life is ſhorte, & repleniſhed with manie miſeries.
\XRef{Iob.~14.}}
few, and
%%% !!! Extra note mark?
%\SNote{}
euil, and they are not come to the dayes of my fathers, in which they
were pilgrimes. \V And bleſſing the king, he went forth.
%%% o-0149
\V But Ioſeph gaue poſſeſſion to his father and his brethren in Ægypt,
in the beſt place of the land, in Rhameſſes, as Pharao had commanded. \V
And he nourished them, and al
%%% 0162
his fathers houſe, alowing victuales to euerie one. \V For in the whole
world there wanted bread, and famine oppreſſed the land, eſpecially of
Ægypt and Chanaan. \V Out of which he gethered together al the money for
the ſelling of corne, and brought it in vnto the kings treaſure. \V And
when the byers wanted money, al Ægypt came to Ioſeph, ſaying: Geue vs
bread: why die we before thee, our money failing? \V To whom he
anſwered: Bring your cattel, and for them I wil geue you victuales, if
you haue not to pay. \V Which when they had brought, he gaue them
ſuſtenance for horſes, and sheepe, and oxen, and aſſes: and he ſuſtayned
them that yeare for the exchange of the cattel. \V And they came the
ſecond yeare, and ſaid to him: We wil not conceale from our lord, that
our money faying, our cattel withal haue fayled: neither art thou
ignorant, that we haue nothing beſides our bodies and land. \V Why
therfore shal we die in thy ſight? both we and our land wil be thyne:
bye vs to be the kings bondmen, and geue vs ſede, leſt for default of
tillers the land be turned into a wildernes. \V Ioſeph therfore bought
al the Land of Ægypt, euery man ſelling his poſſeſſions for the greatnes
of the famine. And he brought it vnder Pharaos handes, \V and al the
people therof from the fardeſt ends of Ægypt, euen to the vttermoſt
coaſts therof, \V
\LNote{Sauing the land of the Prieſts.}{Let
\MNote{The immunitie and care of Prieſts in the law of nature. Yea
amongſt Infidels.}
them heare which now liue (ſaith
\CNote{Ho.~65. in Gen.}
S.~Chryſoſtom) what great care men had
in times paſt of the prieſts of idols: and let them learne at leaſt to
yeeld like honour to true prieſts, to whom the miniſterie of al diuine
offices is committed. For if the Ægyptians, in their errors, had ſo
great care of Idols, thincking them to be more honored, if their
miniſters were reſpected, how great condemnation doe they not deſerue,
that now diminiſh that, which pertaineth to the prouiſion of prieſts?
\MNote{Much more amongſt Chriſtians, Prieſts ought to be reſpected.}
Doe yee not know that the honour pertaineth to God himſelf? Regard not
therfore him to whom the honour is exhibited. For it is not for his
cauſe to whom thou doeſt it, but for his ſake whoſe prieſt he is, that
of him thou maieſt abundantly receiue rewards. VVherfore he
ſaid:
\CNote{Math.~25. &~10.}
\Emph{He that shal doe it to one of theſe, hath done it for me}:
& \Emph{He that receiueth a prophet, in the name of a prophet, shal
receiue the revvard of a prophet.} VVil our Lord reward thee according to
the worthines or meannes of his miniſters? According to thine owne
alacritie, he either crowneth or condemneth. &c. I ſay not this for the
prieſts ſakes, but for yours, deſiring to gaine you in al things. For in
lieu of that litle you geue, you ſhal receiue immortal rewards, and
vnſpeakable good. Let vs conſider theſe things, and haſte to ſerue them,
not looking vpon the coſt, but vpon the gaine, and increaſe that riſeth
therof. &c. For whatſoeuer you beſtow vpon Gods prieſts, he accounteth
it as beſtowed on himſelf. And he that ſo beſtoweth, ſhal not only
receiue like retribution, but manifold greater: our merciful God,
alwayes of the abundance of his mercie, exceeding the things which are
done by vs. Let vs not therfore be worſe then infidels, who for the
error of idols gaue ſo much to their ſeruants; for how much error and
truth do differ, ſo much the difference is there, betwen theirs and Gods
Prieſts. Thus much and ſomething more writeth S.~Chriſoſtom vpon this
place.}
ſauing the land of the
\LNote{Prieſts.}{The
\MNote{\HH{Cohen} in ſome place ſignifieth Prince, but is here tranſlated
\Emph{Prieſt}, in al the Engliſh Bibles.}
Hebrew word \HH{Cohenim} is here vniuerſally tranſlated \Emph{Prieſts},
in al languages and Editions: which
\XRef{(chap.~40. v.~45.)}
ſome tranſlate \Emph{Prince}: and more probably
\XRef{(2.~Reg.~8. v.~vlt.)}
where Dauids ſonnes are called \HH{Cohenim}: who were in dede Princes,
and not properly Prieſts. But in this preſent place it ſignifieth thoſe,
to whom Pharao alowed particular prouiſion in the time of dearth, which
al vnderſtand of Prieſts, and not of Princes.}
Prieſts, which the king had deliuered them:
\SNote{The prieſts, of Ægypt being not forced to laboure for their
liuing, found out the Mathematiques, as witneſſeth Ariſtotle.
\Cite{in princ. Metaph.}}
to whom alſo a certaine alowance of victuals was geuen out of the cõmon
barnes, and therfore they were not driuen to ſel their poſſeſſions. \V
Ioſeph therfore ſaid to the people: Behold as you ſee, Pharao poſſeſſeth
both you and your land: take ſede, and ſowe the fields, \V that you may
haue corne. The fifth part you ſhal geue to the king: the other foure I
am content you ſhal haue for ſede, and for foode to your families and
your children. \V Who anſwered: Our life is in thy hand: only let our
lord haue a reſpect vnto vs, and we wil gladly ſerue the king. \V From
that time vntil this preſent day in the whole land of Ægypt, the fifth
part is paied to the kings, and it became as it were a lawe, ſauing the
land of the prieſts, which was free from this condition. \V Iſrael
therfore dwelt in Ægypt, that is, in the Land of Geſſen, and poſſeſſed
it: and was increaſed, and multiplied exceedingly. \V And he liued in it
ſeuenteene yeares: and al the dayes of his life came to an hundred
fourtie ſeuen
%%% 0163
yeares. \V And when he ſawe that the day of his death approched, he
called his ſonne Ioſeph, and ſaid to him: If I haue found grace in thy
ſight, put thy hand vnder my thigh: and thou ſhalt doe me this mercie
and truth, not to bury me in Ægypt: \V but
\LNote{I vvil ſleepe vvith my fathers.}{Albeit
\MNote{Special place of burial lawfully deſired, and ſpiritually
profitable.}
neither the lack of burial, nor anie crueltie nor contumelie vſed
againſt dead bodies, can annoy the iuſt, for
\CNote{Luc.~12.}
\Emph{thoſe that kil mens
bodies, can aftervvards doe them now more harme}: yet it is both a lawful
natural deſire, and a ſpiritual comfort and profit, to be buried in
ſpecial places, where their owne frends, or holie perſons are buried, or
where God is more ſpecialy ſerued, Sacrifice, and other prayers
offered. And ſo both Iacob and Ioſeph deſired to reſt in the land of
Chanaan, where their parents were buried and where Chriſt ſhould be
borne and redeeme mãkind.
\MNote{But pompe auaileth not the dead.}
But worldlie pompe and honour of funerals, are rather the cõfort of the
liuing, then the reliefe of the departed, as S.~Auguſtin teacheth, in
\XRef{Pſal.~115.}
For in the ſight of men, the troupe of ſeruants (ſaith the ſame
S.~Auguſtin
\Cite{lib.~1. c.~13. de ciuit.})
made ſolemne and glorious exequies to the rich glutton, that was
cloathed in ſilk, and fared delicately in his life, but in the ſight of
God, the Angels miniſterie made far more excellent to poore Lazarus,
though they caried not his bodie into a marble tombe, but his ſoule into
Abrahams boſome.}
I wil ſleepe with my fathers, and take me away out of this land, and
burie me in the ſepulchre of my anceſters. To whom Ioſeph anſwered: I
wil doe that thou haſt commanded. \V And he ſaid: Sweare then to me. Who
ſwearing, Iſrael adored God, turning
\LNote{To the beds head.}{S.~Paul
\MNote{The Septuagint are not contrarie to the Hebrew and Latin text,
but ſupplie that was omitted.}
alleaging this place ſaith:
\CNote{Heb.~11.}
\Emph{Iacob adored the top of}
(Ioſeph) \Emph{his rod}, folowing the Septuagint, who for the ſame
Hebrew word (being without points, that is, without vowels) in this
place ſay, \Emph{rod}, and in the next chapter
\XRef{(v.~2.)}
interpret \Emph{bed}. For \HH{Matteh} ſignifieth \Emph{a rod},
and \HH{Mittah}, \Emph{a bed}. The Latin therfore tranſlating \Emph{bed},
as the Hebrew is pointed, and the Septuagint, and S.~Paule
reading \Emph{rod}, both are true, and both together expreſſe the whole
action, that Iacob taking Ioſephs \Emph{rod} into his hand, and turning
to \Emph{the beds head}, leaned on the top of the rod, and adored not
only God, the Lord and geuer of al good, but alſo his ſonne Ioſeph now
the chiefe ruler and Prince of Ægypt, as S.~Auguſtin expoundeth.
\Cite{q.~162. in Gen.}
And herein ſaith S.~Chriſoſtom
\Cite{(ho.~66.)}
Ioſephs dreame was fulfilled, that the ſunne and moone ſhould adore
him. The like ſaith Theodoret
\Cite{(q.~108. in Gen.)}
\MNote{Adoration of God and creatures is not repugnant.}
And Procopius addeth that Iacob adoring Ioſephs rod, adored alſo Chriſts
kingdome, prefigured by the ſame rod. But how adoration of creatures
redoundeth to the honour of God, more is noted vpon the ſaid place of
S.~Paul.
\XRef{Heb.~11.}}
to the beds head.


\stopChapter


\stopcomponent


%%% Local Variables:
%%% mode: TeX
%%% eval: (long-s-mode)
%%% eval: (set-input-method "TeX")
%%% fill-column: 72
%%% eval: (auto-fill-mode)
%%% coding: utf-8-unix
%%% End:
