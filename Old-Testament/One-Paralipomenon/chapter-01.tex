%%%%%%%%%%%%%%%%%%%%%%%%%%%%%%%%%%%%%%%%%%%%%%%%%%%%%%%%%%%%%%%%%
%%%%
%%%% The (original) Douay Rheims Bible 
%%%%
%%%% Old Testament
%%%% One Paralipomenon
%%%% Chapter 01
%%%%
%%%%%%%%%%%%%%%%%%%%%%%%%%%%%%%%%%%%%%%%%%%%%%%%%%%%%%%%%%%%%%%%%




\startcomponent chapter-01


\project douay-rheims


%%% 0838
%%% o-0751
\startChapter[
  title={Chapter 1}
  ]

\Summary{The
\MNote{The firſt part.

Genealogies partly of other progenies of Adam, but ſpecially of Iacobs
iſſue.}
genealogie of Adam in the right line to Noe, and his three ſonnes, Sem,
Cham, and Iaphet. 5.~The generations of Iaphet, 8.~of Cham, 17.~and of
Sem. 24.~The right line of Sem to Abraham. 26.~Abrahams generations by
the line of Iſmael, 32.~by the ſonnes of Cetura, 34.~and by the line of
Iſaac, and his ſonne Eſau; 43.~with their kinges, 51.~and dukes.}

Adam,
%%% !!! Unmarked Annotation. Applies to whole chapter (and ſucceeding
%%% ones) in general. Alſo no verse number or 'Adam' starts annotation.
\LNote{Adam.}{Becauſe
\MNote{Differences of names, numbers, & times, found in holie ſcriptures,
make them hard to be vnderſtood.}
in diuers holie Scriptures, and eſpecialy in theſe bookes of
Paralipomenon, manie difficulties occurre concerning diuers perſons, and
places; as alſo differences of numbers and times; in reconciling wherof
the holie Fathers and Doctors haue much laboured, making ſometimes large
commentaries to ſatisfie them ſelues, and other diligent ſearchers of
the truth, & to remoue the obloquies of detractors from the authoritie
of holie Scripture, whoſe learned explications of ſuch obſcurities if we
ſhould cite, it would be ouer long, and contrarie to our purpoſe of
brief Annotations; here once for often, we wil preſent to the vulgar
reader, certaine cleare and ordinarie rules, by which the learned
Diuines do reconcile ſuch apparent contradictions.

Firſt,
\MNote{Diuers meanes to reconcile ſeming contradictions in holie
ſcriptures.}
\MNote{1.}
it is euident by ſundrie examples, that manie perſons, places, and ſome
other thinges had diuers names, & ſo are ſometimes called by one name,
ſometimes by an other.
\MNote{2.}
Secondly, (which is more common) manie were called by the ſame names, and
ſo muſt be diſtinguiſhed by the differences of times, places, qualities,
or other circumſtances.
\MNote{3.}
Thirdly, in genealogies and other hiſtories, children are not alwaies
called the ſonnes, or daughters of their natural parentes, but ſometimes
of legal fathers;
\CNote{\XRef{Luc.~3.}}
and ſometimes alſo of thoſe that adopted them for their children; and
ſometimes of their grandfathers, or former progenitors.
\MNote{4.}
Fourthly, ſometime for myſterie ſake, an other number is expreſſed,
being true in the myſtical ſenſe, differing from the preciſe number
according to the hiſtorie.
\CNote{\XRef{Mat.~1.}}
As in the genealogie of Chriſt the Euangeliſt counteth thriſe fourtene
generations from Abraham to our Sauiour, differing from the hiſtorie of
the old Teſtament.
\MNote{5.}
Fiftly, euen in the hiſtorie it ſelf, ſometimes holie Scripture counteth
only the greater numbers, ommitting the leſſer, and in ſome other addeth
alſo the odde numbers.
\MNote{6.}
Sixtly, the Scriptures ſpeake often by tropes, as mentioning part for
the whole, or the whole for the part; ſo by the figure Synechdoche,
Chriſt is ſaid to haue bene three dayes dead, that is, one whole day and
part of other two. And ſome king liuing or reigning ſo manie yeares and
part of an other, and his ſucceſſour reigning the other part, ech part
is counted to each of them for a whole yeare, and ſo a yeare is added,
more then is in the preciſe number.
\MNote{7.}
Seuenthly, ſometimes the ſonnes reigned together with their fathers, as
Ioathan reigned his father Ozias yet liuing
\XRef{4.~Reg.~15.}
& ſo both their reignes are ſometimes counted, ſometimes their ſeueral
yeares, as euerie one reigned alone.
\MNote{8.}
Eightly, the times of vacances, in the gouernment of the Iudges, reignes
of kinges, and the like, are ſometimes omitted in calculation, ſometimes
adioyned to the predeceſſor, or ſucceſſor.
\MNote{9.}
Ninthly, ſometimes the holy Scripture mentioneth the only time that one
liued or reigned wel, as it were blotting out the reſt with obliuion. So
Saul is ſayd to haue reigned two yeares
\XRef{(1.~Reg.~13.)}
vvho wel and euil reigned much longer.
\MNote{10.}
Tenthly, by error in writing, wordes, names, and eſpecially numbers may
eaſely be changed, and can not eaſely be corrected. By theſe or other
like meanes, al the holie Scriptures may be defended,
\MNote{Not priuate but publique ſpirit of the Church expounder of holie
Scripture.}
though none ought to preſume by his priuate ſpirit, to vnderſtand and
expound al Scriptures; which are hard not only by reaſon of their
profound ſenſe, ſurpaſſing mans natural capacitie, but alſo for that in
outward apparence, ſometimes there ſeeme to be contradictions; but in
dede neither are, nor can be vttered by the Holie Ghoſt, the Spirit of
truth,
\CNote{\XRef{2.~Pet.~1. v.~20.}}
inditer of the whole ſacred Bible. And therfore we muſt relie vpon Gods
Spirit, ſpeaking in his ſpouſe the Church, commended vnto vs by thoſe
Scriptures, wherof we are ſufficiently aſſured.}
\SNote{Adam had two other ſonnes before Seth, but Cains race was vtterly
extinguiſhed by the flood, and Abel had no children.}
Seth, Enos, \V Cainan, Malaleel, Iared, \V Henoch, Mathuſale, Lamech, \V
Noe, Sem, Cham, and Iapheth. \V The ſonnes of Iapheth: Gomer, and Magog,
& Madia, and Iauan, Thubal, Moſoch, Thiras. \V Moreouer the
ſonnes of Gomer: Aſcenez, and Riphath, and Thogorma. \V And the ſonnes
of Iauan: Eliſa and Tharſis, Cethim and Dodanim. \V The ſonnes of Cham:
Chus, and Meſraim, and Phut, & Chanaan. \V And the ſonnes of Chus: Saba,
and Heuila, Sabatha, & Regma, and Sabathaca. Moreouer the ſonnes of
Regma: Saba, and Dadan. \V And Chus begat Nemrod: this begane to be
mightie in the earth. \V But Meſraim begat Ludim, and Anamim, and
Laabim, & Nephtuim, \V Phetruſim alſo, and Caſtuim: from whom came
Philiſthijm, & Caphthorim. \V But Chanaan begat Sidon his firſtborne,
the Hetheite alſo, \V and the Iebuſeite, and the Amorrheite, & the
Gergeſeite, \V & the Heueite, and the Araceite, and the Sineite. \V The
Aradium alſo, and the Samareite, and the Hamatheite. \V The ſonnes of
Sem: Aelam, and Aſſur, and Arphaxad, & Lud, and Aram, and Hus, and Hul,
%%% o-0752
and Gether, and Moſoch. \V And Arphaxad begat Sale, who alſo begat
Heber. \V Moreouer to Heber were borne two ſonnes, the name of one was
Phaleg, becauſe in his daies the earth was diuided; and the name of his
brother Iectan. \V And Iectan begat Elmodad, and Saleph, & Aſarmoth, and
Iare, \V Adoram
%%% 0839
alſo, and Vſal, and Decla, \V Hebal alſo, and Abimael, and Saba,
moreouer \V alſo Ophir, and Heuila, and Iobab. Al theſe are the ſonnes
of Iactan: \V
\SNote{As before the right line of Adam to Noe, ſo here from his ſonne
Sem to Abram.}
Sem, Arphaxad, Sale, \V Heber, Phaleg, Ragau, \V Serug, Nachor,
Thare, \V Abram, this is
\SNote{For myſterie ſake God changed his name to Abraham.
\XRef{Gen.~17.}}
Abraham. \V And the ſonnes of Abraham: Iſaac & Iſmael. \V And theſe are
the generations of them. The firſtbegotten of Iſmael, Nabaioth, and
Cedar, and Abdeel, and Mabſam, \V and Maſma, and Duma, Maſſa, Hadad, and
Thema, \V Ietur, Naphis,
Cedma. Theſe are the ſonnes of Iſmahel. \V And the ſonnes of
Cetura Abrahams concubine, which ſhe bare: Zamran, Iecſan, Madan,
Madian, Ieſboc, and Sue. Moreouer the ſonnes of Iecſan: Saba, and Dadan. And the
ſonnes of Dadan: Aſſurim, and Latuſſim, and Laomim. \V And the ſonnes of
Madian: Epha, and Epher, and Henoch, and Abida, and Eldaa. Al theſe the
ſonnes of Cetura. \V And Abraham begat Iſaac: whoſe ſonnes were Eſau,
& Iſrael. \V The ſonnes of Eſau: Eliphaz, Rahuel, Iehus, Ihelom, and
Core. \V The ſonnes of Eliphaz: Theman, Omar, Sephi, Gathan, Cenez,
Thamna, Amalec. \V The ſonnes of Rahuel: Nahath, Zara, Samma, Meza. \V
The ſonnes of Seir: Lotan, Sobal, Sebeon, Ana, Diſon, Eſer, Diſan. \V
The ſonnes of Lotan: Hori, Homam. And the ſiſter of Lotan was Thamna. \V
The ſonnes of Sobal: Alian, and Manahath, and Ebal, Sephi, & Onam. The
ſonnes of Sebeon: Aia & Ana. The ſonne of Ana: Diſon. \V The ſonnes of
Diſon: Hamram, and Eſeban, and Iethran, and Charan. \V The
ſonnes of Eſer: Balaan, and Zauan, and Iacan. The ſonnes of Diſan: Hus
and Aran. \V Theſe be the kinges, that reigned in the Land of Edom,
before there was a king ouer the children of Iſrael: Bale the ſonne of
Beor: and the name of his citie, Deneba. \V And Bale died, and Iobab the
ſonne of Zare of Boſra, reigned for him. \V And when Iobab alſo was
dead, Huſam of the Land of the Themanes reigned for him. \V And Huſam
alſo died, and Adad the ſonne of Badad reigned for him, who ſtroke
Madian in the Land of Moab: and the name of his citie was Auith. \V And
when Adad alſo was dead, Semla of Maſreca reigned for him. \V But Semla
alſo died, and there reigned for him Saul of Rohoboth, which is ſituate
beſides the riuer. \V Saul alſo being dead, Balanan, the ſonne of
Achobor reigned for him. \V But this alſo died, and Adad reigned for
him: whoſe cities name was Phau, and his wife was called Meerabel the
daughter of Matred, the daughter of Mezaab.
%%% 0840
\V And Adad being dead, there began to be dukes in Edom for kinges: duke
Thamna, duke Alua, duke Ietheth, \V duke Oolibama, duke Ela, duke
Phinon, \V duke Cenez, duke Thaman, duke Mabſar, \V duke Magdiel, duke
Hiram. Theſe be the dukes of Edom.


\stopChapter


\stopcomponent


%%% Local Variables:
%%% mode: TeX
%%% eval: (long-s-mode)
%%% eval: (set-input-method "TeX")
%%% fill-column: 72
%%% eval: (auto-fill-mode)
%%% coding: utf-8-unix
%%% End:
