%%%%%%%%%%%%%%%%%%%%%%%%%%%%%%%%%%%%%%%%%%%%%%%%%%%%%%%%%%%%%%%%%
%%%%
%%%% The (original) Douay Rheims Bible 
%%%%
%%%% Old Testament
%%%% One Paralipomenon
%%%% Chapter 19
%%%%
%%%%%%%%%%%%%%%%%%%%%%%%%%%%%%%%%%%%%%%%%%%%%%%%%%%%%%%%%%%%%%%%%




\startcomponent chapter-19


\project douay-rheims


%%% 0871
%%% o-0781
\startChapter[
  title={Chapter 19}
  ]

\Summary{The king of Ammon euil intreating king Dauids men, whom he had
  curteouſly ſent to condole the death of his father, is ouerthrowen in
  battel, 16.~with the Aſsyrians his hyred confederates.}

And it chanced that
\CNote{\XRef{2.~Reg.~10.}}
Naas the king of the children of Ammon died, and his ſonne reigned for
him. \V And Dauid ſayd: I wil do mercie with Hanon the ſonne of Naas:
for his father hath done me pleaſure. And Dauid ſent meſſengers to
confort him vpon the death of his father. Who when they were come into
the land of the children of Ammon, to confort Hanon, \V the princes of
the children of Ammon ſayd to Hanon: Thou thinkeſt perhaps, that Dauid
for honour ſake toward thy father hath ſent ſome that ſhould comfort
thee: neither markeſt thou, that his ſeruantes are come to thee to
eſpie, and ſeeke out, and ſearche thy land. \V Therfore Hanon made the
ſeruantes of Dauid balde, and ſhaued them, and cut away their cotes from
the buttockes to the feete, and ſent them away. \V Who
when they were gone, and had ſent word to Dauid, he ſent to meete them
(for they had ſuſteyned great reproch) and commanded them to tarie in
Iericho, til their beard grewe, and then they ſhould returne. \V And the
children of Ammon ſeing, that they had done iniurie to Dauid, as wel
Hanon as the reſt of the people, they ſent a thouſand talents of ſiluer,
to hyre them chariotes and horſemen out of Meſopotamia, and from Siria
Maacha, and from Soba. \V And they hyred two and
\Fix{thirtre}{thirtie}{obvious typo, fixed in other}
thouſand chariotes, and king Maacha with his people. Who when they were
come, camped ouer agaynſt Medaba. The children of Ammon alſo being
gathered together out of their cities, came to the battel. \V Which when
Dauid had heard, he ſent Ioab, and al the hoſt of valiant men: \V and
the children of Ammon iſſuing forth, put their armie in aray beſide the
gate of the citie: and the kinges, that were come to ayde him, ſtood
apart in the field. \V Ioab therfore vnderſtanding that battel was
%%% 0872
made agaynſt him before and behind, choſe the moſt valiant men of al
Iſrael, and marched on againſt the Syrian. \V And the reſt of the people
he gaue vnder the hand of Abiſai his brother: and they went forth
agaynſt the children of Ammon. \V And he ſayd: If the Syrian ſhal
ouercome me, thou ſhalt ayde me: and if the children of Ammon ſhal
ouercome thee, I wil ayde thee. \V Take courage, and let vs play the men
for our people, and for
%%% o-0782
the cities of our God: and our Lord wil doe that which is good in his
ſight. \V Ioab therfore marched on, and the people that were with him,
agaynſt the Syrian to battel: and he put them to flight. \V Moreouer the
children of Ammon ſeing that the Syrian was fled, themſelues alſo fled
from Abiſai his brother, and went into the citie: and Ioab alſo returned
into Ieruſalem. \V But the Syrian ſeing that he was fallen before
Iſrael, ſent meſſengers, and brought the Syrian, that was beyond the
riuer: and Sophach the General of
\Fix{Aderezers}{Adarezer}{obvious typo, fixed in other}
warre, was their captayne. \V Which when it was told Dauid, he gathered
together al Iſrael, and paſſed Iordan, and fel vpon them, and directed
his armie agaynſt him, they fighting on the contrarie part. \V And the
Syrian fled from Iſrael: and Dauid ſlewe of the Syrians ſeuen thouſand
chariotes, and fourtie thouſand footemen, and Sophach General of the
armie. \V And the ſeruantes of Adarezer ſeing themſelues to be ouercome
of Iſrael, fled to Dauid, & ſerued him: and Syria would no more giue
ayde to the children of Ammon.


\stopChapter


\stopcomponent


%%% Local Variables:
%%% mode: TeX
%%% eval: (long-s-mode)
%%% eval: (set-input-method "TeX")
%%% fill-column: 72
%%% eval: (auto-fill-mode)
%%% coding: utf-8-unix
%%% End:
