%%%%%%%%%%%%%%%%%%%%%%%%%%%%%%%%%%%%%%%%%%%%%%%%%%%%%%%%%%%%%%%%%
%%%%
%%%% The (original) Douay Rheims Bible 
%%%%
%%%% Old Testament
%%%% Esther
%%%% Argument
%%%%
%%%%%%%%%%%%%%%%%%%%%%%%%%%%%%%%%%%%%%%%%%%%%%%%%%%%%%%%%%%%%%%%%




\startcomponent argument


\project douay-rheims


%%% 1055
%%% o-0950
\startArgument[
  title={\Sc{The Argvment of the Booke of Esther.}},
  marking={Argument of the Booke of Esther.}
  ]

Of
%%% !!! Are these first few cites, or the writers who doubted?
\CNote{\Cite{Melito}
\Cite{S.~Atha.}
\Cite{S.~Greg. Nazian.}
%%% !!! Cite for assertion of above writers doubting?
\Cite{Origen apud Euſeb. lib.~6. c.~25. hiſt.}}
\MNote{This whole booke is canonical.}
the authoritie of this booke \Emph{only two or three ancient writers,
doubted}, before the councels of Laodicea, and Carthage declared it to
be Canonical. \Emph{Al the reſt did euer eſteme it as diuine Scripture.}
For albeit S.~Ierom in his time found not certaine partes therof in the
Hebrew, and therfore tranſpoſed the ſame to the end of the booke, as now
we haue them: yet in the Greeke he found al theſe ſixtenne chapters
conteyned in tenne. And it is not vnprobable, that theſe parcels were
ſometimes in the Hebrew, as were diuers whole bookes which are now
loſt. But whether they were at anie time in the Hebrew or no, the Church
of Chriſt accounteth the whole Booke of infallible authoritie, reading
as wel theſe partes, as the reſt in her publique office. And the councel
of Trent
\Cite{(ſeſſ.~4.)}
for more expreſſe declaration defineth \Emph{that al the bookes recited
in the ſame Decree} (amongſt which is Eſther) \Emph{with al the partes
therof, as they are accuſtomed to be read in the Catholique Church, and
be conteyned in the old vulgate latin Edition, are ſacred and Canonical
Scripture.}

It
\MNote{The contentes.}
conteyneth a particular great danger of the people of Iſrael, hapening
(as is probable) shortly after their general relaxation, & returne of
ſome from the captiuitie of Babylon; and their
\Fix{deliuere}{deliuerie}{likely typo, fixed in other}
from it, through the godlie zele and other vertues of Quene Eſther,
directed herein by Mardocheus, who being alſo in imminent danger was
deliuered & aduanced, and finally
\MNote{VVritten by Mardocheas.}
writ the hiſtorie.
%%% 1056
\MNote{Diuided into foure partes.}
Which may be diuided into foure partes, not by order of the chapters: as
\Fix{hey}{they}{obvious typo, fixed in other}
are here tranſpoſed: but in order of time, firſt the author reporteth
ſome thinges going before the peoples danger, in the
11.~1. 2. 12.~chapters, and part of the 3. Secondly their danger and
diſtreſſe, in the reſt of the 3. and part of 13.~chapters. Thirdly their
deliuerie: from the 4.~chapter to the middes of the 9. and reſt of the
13, and in the 14.~15. and~16. Fourtly, the thinges that enſued
hereupon, in the other half of the ninth chapter, the 10.~chapter, and
firſt verſe of the eleuenth.

\hairline

VVho ſoeuer vvil pleaſe to read this hiſtorie, in order of the time as
the thinges happened, adioyning the latter chapters, vvhich are in the
Greke, and not in the Hebrevv, may folovve the letters of the Alphabet,
as here vve haue placed them in the margent, beginning at the ſecond
verſe of the 11.~chapter, vvhere he findeth the letter A and vvhen he
cometh to B returne vvhere the ſame letter is noted. ch.~1. And ſo in
the reſt folovv the ſame direction.

\stopArgument


\stopcomponent


%%% Local Variables:
%%% mode: TeX
%%% eval: (long-s-mode)
%%% eval: (set-input-method "TeX")
%%% fill-column: 72
%%% eval: (auto-fill-mode)
%%% coding: utf-8-unix
%%% End:
