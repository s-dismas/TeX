%%%%%%%%%%%%%%%%%%%%%%%%%%%%%%%%%%%%%%%%%%%%%%%%%%%%%%%%%%%%%%%%%
%%%%
%%%% The (original) Douay Rheims Bible 
%%%%
%%%% Old Testament
%%%% Esther
%%%% Chapter 01
%%%%
%%%%%%%%%%%%%%%%%%%%%%%%%%%%%%%%%%%%%%%%%%%%%%%%%%%%%%%%%%%%%%%%%




\startcomponent chapter-01


\project douay-rheims


%%% 1056
%%% o-0951
\startChapter[
  title={Chapter 1}
  ]

\Summary{King Aſſuerus celebrateth a great banket to shew his glorie,
  10.~calleth quene Vaſthi therto. Who refuſing to come, is by aduiſe of
  his counſel depoſed.}

In
\MNote{The firſt part beginneth in the 11.~ch. A.

{\Large B}}
the daies of Aſſuerus, who reigned from India vnto Æthiopia ouer an
hundred twentie ſeuen prouinces: \V when he ſate in the throne of his
kingdõ, the citie Suſã was in the begynning of his kingdom. \V In the
third yeare therfore of his empyre he made a great feaſt to al the
princes, and to his ſeruantes, to the moſt valiant of the Perſians, and
the nobles of the Medes, and the rulers of the prouinces in his
ſight, \V that he might shew the riches of the glorie of his kingdom,
and the greatnes, & vaunting of his might, a great time, to witte, an
hundred & foure ſcore dayes. \V And when the daies of the feaſt were
accompliſhed, he inuited al people, that was found in Suſan, from the
greateſt to the leaſt: and commanded ſeuen daies a feaſt to be prepared
in the entrance of the garden, and of the wood, which was planted with
royal garniſhing and with hand. \V And there hong on euerie ſide
hangings of skie colour, and grene, and hyacinthine colour, held vp with
cordes of ſilke, and of purple, which were put into rings of yuorie, and
were held vp
%%% 1057
with marble pillers. Beddes alſo of gold and ſiluer, were placed in
order vpon the floore paued with the emerauld,
%%% o-0952
and the touch ſtone: which paynting adorned with meruelous varietie. \V
And they that were inuited, dranke in golden cuppes, and the meates were
brought in change of veſſels. Wine alſo plenteous and the beſt was ſet
downe, as was worthie of a kings magnifence. \V
\SNote{Modeſtie and temperance amongſt heathen people condemneth
Chriſtianes that vrge men to drinke immoderatly, and ſo cauſe them to be
drunke.
\Cite{S.~Au. Ser.~231.~232. de tempore.}}
Neither was there that compelled them to drinke that would not, but as
the king had appointed, making ech of his princes ouerſeer of euerie
table, that euerie man might take what he would. \V Vaſthi alſo the
queene made a feaſt for the wemen in the palace, where king Aſſuerus had
accuſtomed to remayne. \V Therfore the ſeuenth day, when the king was
merier, and after very much drinking was wel warmed with wine, he
commanded Maumam, and Bazatha, and Harbona, and Bagatha, and Abgatha,
and Zethar, and Charchas, the ſeuen eunuches, that miniſtred in his
ſight, \V that they ſhould bring in queene Vaſthi before the king,
the crowne ſet vpon her head, that he might ſhew her beautie to al the
peoples and princes: for ſhe was exceding beautiful. \V Who refuſed, and
contemned to come at the kings commandment, which he had commanded by
the eunuches. Wherupon the king being wrath, and chaffed with
\SNote{The end of immoderate feaſting is commonly browling. Here the
king became furious, and the queene was diuorſed from him.}
exceding furie, \V asked the wiſemen, which after the manner of a king
were alwayes preſent with him, and he did al thinges by their conſel,
which knew the lawes, and rightes of the elders: \V (and the chiefe and
neareſt him were, Charſena, and Sethar, and Admatha, and Tharſis,
and Mares, and Marſana, and Mamuchan, ſeuen dukes of the Perſians, and
of the Medes, which ſaw the face of the king, and were wont to
\Fix{ſirt}{ſit}{obvious typo, fixed in other}
firſt after him) \V to what ſentence Vaſthi the queene ſhould be
ſubiect, that would not do Aſſuerus the kings commandment, which he had
commanded by the eunuches. \V And Mamuchan anſwered, the king hearing,
and the princes:
\SNote{Brentius approueth the ſentence of this paraſite, but Ioſephus
\Cite{li.~11. c.~6.}
Macrobius
\Cite{li.~7. c.~1. Saturn.}
S.~Ierome
\Cite{ad Ruſtic.}
and S.~Ambroſe
\Cite{l. de Elia c.~14.}
iudge the queenes refuſal lawful, and agreable to the Perſians lawes,
which prohibited maried wemẽ to come in ſight of other men in great
aſſemblies: neither had the king iuſt cauſe to break that law, for
pleaſing his phanſie in his drunken humour.
\XRef{v.~10.}
Luther alſo wreſteth this example in fauoure of adulterie.
\Cite{par.~2. de diuortio. folio~177. Editionis vvitember Ano.~1553.}}
Queene Vaſthi hath not only hurt the king, but alſo al peoples, and
princes, that are in al the prouinces of king Aſſuerus. \V For the word
of the queene wil goe forth to al wemen, that they wil contemne their
husbands, and wil ſay: King Aſſuerus commanded that the queene Vaſthi
ſhould come in to him, and ſhe would not. \V And by this example al the
wiues of the princes of the Perſians and the Medes, wil little eſteeme
the commandmentes of their husbandes,
%%% 1058
wherfore the kings indignation is iuſt. \V If it pleaſe thee, let an
edict goe forth from thy face, and let it be written according to the
law of the Perſians and of Medes, which is not lawful to be
tranſgreſſed, that Vaſthi come in no more to the king, but an other,
that is better then ſhe, take her kingdome. \V And let this be publiſhed
into al the empire of thy prouinces (which is moſt large) and let al the
wiues, as wel of the greater as of the leſſer geue honour to their
husbandes. \V His counſel pleaſed the king, and the princes: and the
king did according to the counſel of Mamuchan, \V and he ſent letters to
al the prouinces of his kingdome, as euerie nation could heare and
reade, in diuers languages and characters, that the husbandes ſhould be
princes and maiſters in their houſes: and that this ſhould be publiſhed
through al peoples.


\stopChapter


\stopcomponent


%%% Local Variables:
%%% mode: TeX
%%% eval: (long-s-mode)
%%% eval: (set-input-method "TeX")
%%% fill-column: 72
%%% eval: (auto-fill-mode)
%%% coding: utf-8-unix
%%% End:
