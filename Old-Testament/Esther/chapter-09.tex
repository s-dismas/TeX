%%%%%%%%%%%%%%%%%%%%%%%%%%%%%%%%%%%%%%%%%%%%%%%%%%%%%%%%%%%%%%%%%
%%%%
%%%% The (original) Douay Rheims Bible 
%%%%
%%%% Old Testament
%%%% Esther
%%%% Chapter 09
%%%%
%%%%%%%%%%%%%%%%%%%%%%%%%%%%%%%%%%%%%%%%%%%%%%%%%%%%%%%%%%%%%%%%%




\startcomponent chapter-09


\project douay-rheims


%%% 1068
%%% o-0962
\startChapter[
  title={Chapter 9}
  ]

\Summary{The Iewes kil their enemies which would haue killed
  them, 6.~namely the tenne
  \Fix{ſonns}{ſonnes}{likely typo, fixed in other}
of Aman are hanged on gallowes, 13.~more ſlaine the next day. 17.~The
  day folowing is made holie, and ſo to be kept euerie yeare.}

Therfore
\MNote{\Large M}
in the thirtenth day of the twelth moneth, which we haue ſaid now before
to be called Adar, when ſlaughter was prepared for al the Iewes, and
their enemies gaped after their bloud,
\LNote{The caſe being changed.}{In
\MNote{Great, and maruelous changes by the power of God.}
this whole hiſtorie (as in manie other
\CNote{\XRef{Pſal.~76.}}
paſſages of holie Scripture) we ſee the meruelous change of the right
hand of the higheſt. Firſt in the verie inſtant of extreme danger, the
kinges face with burning eyes ſhewing the furie of his breaſt
\XRef{ch.~15. v.~10.}
was turned into mildnes towardes Eſther.
\XRef{v.~11.}
Secondly proud Aman being aduanced in honour & office, aboue al princes
of the kinges court,
\XRef{ch.~3. v.~1.}
was ſodainly forced publikely to lead Mardocheus his horſe, whom he moſt
hated.
\XRef{ch.~6. v.~10.}
Thirdly, he was conſtrayned with loud voice to proclame his honour, whom
he moſt deſpiſed and threatned. Fourtly, the ſame Aman, before called
the father (as it were the onlie gouernour) of the king,
\XRef{ch.~13. v.~6.}
\XRef{ch.~16. v.~11.}
was forthwith condemned for a traitor.
\XRef{ch.~7. v.~8.}
\XRef{ch.~16. v.~18.}
Fifthly, he that could not abide to ſee Mardocheus,
\XRef{ch.~5. v.~13.}
afterward durſt not looke vpon the king, nor could endure his countenance.
\XRef{ch.~7. v.~6.}
Sixtly, he was hanged on the ſame gallowes, which he had prepared for
Mardocheus.
\XRef{ch.~7. v.~9.}
Seuently, vvheras he was not content with the death of Mardocheus alone,
but procured the kings decree to deſtroy the whole nation,
\XRef{ch.~3. v.~6.}
and ſo feaſted with the king, when the Iewes mourned,
\XRef{v.~15.}
ſhortly after the king ſent new letters for the Iewes ſaftie, geuing
them leaue to kil whom ſoeuer they would of their enemies.
\XRef{ch.~8. v.~8.~11.}
Eightly, the ſame day which was deſigned for deſtruction, was made the
day of ioy and exultation to the children of God.
\XRef{ch.~9. v.~1.~17.}
\XRef{ch.~16. v.~21. &c.}
By which literal ſenſe Gods meruelous prouidence is manifeſtly ſhewed,
neuer ſuffering his church to periſh.
\CNote{\Cite{D.~Tho. prologo in Epiſt. Canonic.}}
It hath moreouer two ſpecial myſtical ſenſes.
\MNote{Either a figure of our B.~ladie,}
Firſt, as ſaftie of temporal life was procured to one nation by Eſthers
interceſſion to king Aſſuerus, ſo general ſaluation is procured to al
mankind by mediation of the bleſſed virgin Marie, cruſhing the ſerpents
head; and the ſentence of death is changed by new letters, granting
euerlaſting life, and glorie to al Gods true ſeruantes.
\MNote{And of the Church.}
Eſther alſo, as likewiſe Iudith, in figure of the Church (ſaith
S.~Ierom,
\Cite{Prologo in Sophon.)}
killed the aduerſaries, and deliuered Iſrael from danger of periſhing.}
the caſe being changed to the contrarie, the Iewes began to be
ſuperiours, and
\SNote{VVhere no more danger remaineth remiſſion of iniuries is more
commendable then reuenge, but where malice continueth, and new danger
may probably enſue, iuſtice is neceſſarie, and afterwards peace may be
made more ſecurely.
\Cite{S.~Bernard ſer.~2. de verb. Apoſt.}}
to reuenge them ſelues of their aduerſaries. \V And they were gathered
together in euerie citie, and towne, and place, to extend their hand
againſt their enemies, and their perſecutors. And none durſt reſiſt,
becauſe the feare of their greatnes did penetrate al peoples. \V For
both the iudges of the prouinces, and captaynes, and lieutenantes, and
euerie dignitie, that was chiefe ouer euerie place and worke, extolled
the Iewes for feare of Mardocheus: \V whom they knew to be prince of the
palace, and to be able to doe very much: the fame alſo of his name
increaſed dayly, and flew abroad through al mens mouthes. \V Therfore
the Iewes ſtroke their enemies with a great ſlaughter, and ſlew them,
repaying them that which they had prepared to doe to them: \V in ſo much
that in Suſan alſo they killed fiue hundred men, beſides the tenne
ſonnes of Aman the Agagite the enemie of the Iewes: whoſe names be
theſe: \V Pharſandatha, and Delphon, and Eſphatha, \V and
%%% 1069
Phoratha, and Adalia, and Aridatha, \V and Phermeſta, and Ariſai, and
Aridai, and Iezatha. \V Whom when they had ſlaine, they would not take
prayes of their goodes. \V And by and by the number of them that were
killed in Suſan, was brought to the king. \V Who ſaid to the queene: In
the citie of Suſan the Iewes haue killed fiue hundred men, beſides the
\SNote{In the firſt ſlaughter Amans tenne ſonnes were ſlayne and
afterwards alſo hanged.
\XRef{v.~14.}}
ten ſonnes of Aman: how great a ſlaughter thinkeſt thou doe they make in
al the prouinces? what askeſt thou more, & what wilt thou that I cõmand
to be done? \V To whom ſhe anſwered: If it pleaſe the king, let there
authoritie be geuen to the Iewes, that as they haue done to day in
Suſan, ſo alſo they may doe to morow, and that the tenne ſonnes of Aman
be hanged on gibbettes. \V And the king commanded that it ſhould be ſo
done. And forthwith the edict hong in Suſan, and the tenne ſonnes of
Aman were hanged. \V The
\Fix{fourthtenth}{fourtenth}{obvious typo, fixed in other}
day of the moneth Adar the Iewes being gathered together, there were
killed in Suſan three hundred men: neither was their ſubſtance ſpoyled
by them. \V Yea and through al prouinces, which were ſubiect to the
kings dominion, the Iewes ſtood for their liues, their enemies and
perſecutors being ſlayne: in ſo much that there was fully ſeuentie fiue
thouſand of them that were killed, and
%%% o-0963
no man tooke any of their goodes.

\V
\MNote{The fourth
\Fix{parth.}{part.}{obvious typo, fixed in other}

Other thinges folowing their deliuerie from danger.}
And the thirtenth day of the moneth Adar was the firſt day with them
al of the ſlaughter, & the fourtenth day they ceaſed to kil. Which they
ordayned to be ſolemne, ſo that in it at al times afterward they gaue
them ſelues to good chere, mirth & bankets. \V But they that made the
ſlaughter in the citie of Suſan, were occupied in the ſlaughter the
thirtenth and fourtenth day of the ſame moneth: and in the fiftenth day
they ceaſed to kil. And therfore they ordayned the ſame a ſolemne day of
good cheere and ioyfulnes. \V But thoſe Iewes, that abode in townes not
walled and villages, ordayned the fourtenth day of the moneth Adar for
bankettes and ioy, ſo that they reioyſe in it, and ſend one an other
portions of bankets and meates. \V Mardocheus therfore wrote al theſe
things, and being compriſed in letters ſent them to the Iewes, that
abode in al the kings prouinces, as wel thoſe that lay neere, as far
of, \V that they ſhould take
\SNote{The Iewes in Suſan kept the fiftenth day holie,
\XRef{v.~18.}
thoſe that dwelt in other places kept the fourtenth day.}
the fourtenth and fiftenth day of the moneth Adar for feaſtes, and the
yeare alwayes returning ſhould celebrate them with ſolemne honour: \V
becauſe
%%% 1070
in the ſame dayes the Iewes reuenged them ſelues of their enemies, and
mourning and ſorrow were turned into mirth and ioy, and that theſe
ſhould be dayes of good cheere and gladneſſe, and they should ſend one
to an other portions of meates, and should geue giftes to the poore. \V
And the Iewes receiued into a ſolemne rite al things, which they had
begune to doe at that time, and which Mardocheus by letters had
commanded to be done. \V For Aman, the ſonne of Amadathi of the ſtocke of
Agag, the enemie and aduerſarie of the Iewes, purpoſed euil againſt
them, to kil them and deſtroy them: and he caſt Phur, which in our
language is turned, a lot. \V And afterward Eſther went in to the king,
beſeching that his endeuours might by the kings letters be made voyde: &
the euil that he had intended againſt the Iewes, might returne vpon his
owne head. Finally they hong both him and his ſonnes vpon the
gallowes, \V and ſince that time theſe dayes are called Phurim, that is,
of Lottes: becauſe Phur, that is, a lot, was caſt into the pot. And al
things, that were done, are contayned in the volume of this epiſtle,
that is, of this booke: \V and the thinges that they ſuſteyned, and that
were afterward changed, the Iewes
\SNote{A feaſt inſtituted by Mardocheus was accepted and obſerued by al
the Iewes, as a conſtitution agreable and not contrarie to the law.
\XRef{Deut.~4. v.~2.}
&
\XRef{12. v.~32.}}
tooke vpon them ſelues and their ſeede, and vpon al, that would be
ioyned to theyr religion, that it should be lawful for none to paſſe
without ſolemnitie theſe dayes: which the writing teſtifieth, and
certaine times require, as yeares continually ſuccede one an other. \V
Theſe are daies, which no obliuion shal euer put out: and al prouinces
in al the world shal celebrate through out al generations: neither is
there any citie, wherein the daies of Phurim, that is, of lottes, muſt
not be obſerued of the Iewes, and of their progenie, which is bound to
theſe ceremonies. \V And Eſther the queene the daughter of Abihail, and
Mardocheus the Iew wrote alſo the ſecond epiſtle, that with al diligence
this day should be eſtablished ſolemne for the time to come. \V And
%%% o-0964
they ſent to al the Iewes, that were in the hundred and ſeuen and
twentie prouinces of king Aſſuerus, that they should haue peace, and
receiue truth, \V obſeruing the Daies of lottes, and in their time
should celebrate them with ioy: as Mardocheus and Eſther had appoynted,
and they tooke vpon them to be obſerued of them ſelues, and of their
ſeede, faſtes, and cries, and the daies of Lottes, \V and al thinges,
which are conteyned in the hiſtorie of this booke, which is called
Eſther.


\stopChapter


\stopcomponent


%%% Local Variables:
%%% mode: TeX
%%% eval: (long-s-mode)
%%% eval: (set-input-method "TeX")
%%% fill-column: 72
%%% eval: (auto-fill-mode)
%%% coding: utf-8-unix
%%% End:
