%%%%%%%%%%%%%%%%%%%%%%%%%%%%%%%%%%%%%%%%%%%%%%%%%%%%%%%%%%%%%%%%%
%%%%
%%%% The (original) Douay Rheims Bible 
%%%%
%%%% Old Testament
%%%% Isaia
%%%% Fourth Part
%%%%
%%%%%%%%%%%%%%%%%%%%%%%%%%%%%%%%%%%%%%%%%%%%%%%%%%%%%%%%%%%%%%%%%




\startcomponent fourth-part


\project douay-rheims


%%% 1584
%%% o-1459
\startArgument[
  title={\Sc{The Fovrth Part of the Old Teſtament Conteining Prophetical
  Bookes.},
  marking={The Fourth Part of the Old Teſtament.}
  ]


The Argument of Prophetical Bookes in General.

Amongſt
\MNote{Gods ſpecial benefite of ſending Prophetes to the people.}
manie great benefites, which God beſtowed vpon his peculiar people in
the old Teſtament, one principal, and very excellent was, that beſides
their ordinarie Paſtors, and gouerners in ſpiritual cauſes, the Prieſtes
of Aarons progenie, and other clergie men of the ſame tribe of Leui, in
Ierarchical ſubordination of one chief, with other ſuperiors and
ſubiectes, diſpoſed in ſacred functions; he alſo gaue them other
extraordinarie Prophetes of ſundrie tribes, as admonitors and guides, to
reduce them from errors of ſinne, into the right way of vertue.
\MNote{The function of Prophetes, to exhorte to repentance with hope of
Gods mercie by Chriſt.}
Which office the ſame Prophetes performed, as wel by threatning the
offenders with Gods wrath, and punishment, as by exhorting them to
repentance, and ſo to truſt in Gods aſſured mercie, that he would geue
them better times, and reliefe from their miſeries. But moſt eſpecially
theſe holie Prophetes did foreſee, and foretel the happie times of Grace
in the New Teſtament. The coming of Meſſias, Chriſt our Redemer and
Sauiour: with the myſteries of his Incarnation, Birth, Paſſion, Death,
Reſurrection, Aſcenſion, Coming of the Holie Ghoſt, Fundation,
Propagation, perpetual Stabilitie of his Church; and finally the General
Iudgement, Eternal Glorie of the bleſſed, and Euerlaſting paine of the
damned. For albeit they preached and prophecied manie thinges, properly
and immediatly perteyning to the particular ſtate, and people of the
Iewes, and other nations, where they conuerſed, yet the principal ſumme
of al the prophetical bookes, is of Chriſt and his Church. Yea al the
old Teſtament is a general prophecie, and forshewing of the New. Which
(as we noted in the beginning) is conteyned, and lieth hid in the
old. Neuertheles ſpeaking more diſtinctly of the proper arguments, or
contents of the foure partes of the old
%%% 1585
Teſtament, the former three more peculiarly ſet forth the Law, the
Hiſtorie, and Sapiential precepts: and this laſt part chiefly conteyneth
Prophecies of thinges to come. Of which the greateſt part is now come to
paſſe, or dayly fulfilled, and the reſt shal likewiſe be performed
in due time. So now in order after the Legal, Hiſtorical, and Sapiential
bookes, folow the Prophetical: and are theſe, according to the names of
the Prophetes that writte them.
%%% o-1460
\MNote{Foure greater Prophetes, and twelue leſſer were auctors of the
prophetical bookes folowing. Baruchs booke being inſerted \Emph{in
Ieremies}.}
\Emph{Iſaie, Ieremie with Baruch, Ezechiel, and Daniel}, commonly called
the greater Prophetes: and the twelue leſſer are \Emph{Oſee, Ioel, Amos,
Abdias, Ionas, Micheas, Nahum, Abacuc, Sophonias, Aggeus, Zacharie, and
Malachie}. Who were al ſingularly inſpired, and gouerned in their
preachings and writinges, by the Holie Ghoſt, that they could not
erre. Yea they were ſo illuminated in their vnderſtanding, that they
clerly ſaw that, which they vttered.
\MNote{Prophecies are called viſions, for their certaintie.}
And therfore their Prophecies are called \Emph{Viſions}, for the aſſured
infallibilitie of truth, which they auouch. For as nothing is more
certaine in vulgar knowlege then that, which we ſee with our corporal
eyes, and therfore of al witneſſes the eye witnes is eſtemed the ſureſt:
and as in al natural knowlege, that is moſt certaine, which is ſene by
diſcourſe of reaſon: ſo in ſupernatural knowlege nothing is more aſſured
then that, which is ſene by ſupernatural light. Whereof there be three
ſortes: the light of Faith, of Prophecie, and of Glorie.
\MNote{Light of prophecie is next to the light of glorie, and more clere
then the light of faith.}
Al three certaine, and vndoubted; but moſt clere and manifeſt is the
viſion by light of glorie: wherby God is ſene in himſelf, and al thinges
in him, that perteyne to the ſtate of euerie glorious Sainct. Next
therto is the viſion by light of prophecie, wherwith God illuminateth
the vnderſtanding of the Prophet by a ſpecial, extraordinarie, and
tranſitorie light of grace, that either he clerly ſeeth the reueled
truthes, or at leaſt perfectly knoweth, that he is moued by the Holie
Ghoſt, though he vnderſtand not al, that the Holie Ghoſt intendeth; and
ſo when, and where it is Gods wil, he vttereth the ſame, for inſtruction
of others. The laſt, which is alſo certaine, but more obſcure, is the
ſupernatural knowlege, which al Catholique Chriſtians haue by light of
faith, aſſuredly beleuing al thinges which God reueleth by his Church.

Concerning
\MNote{Prophecies are hard to be vnderſtood for diuers cauſes.}
therfore this excellent diuine gift of Prophecie, granted to few, for the
benefite of al Gods ſeruants, we are here to informe the vulgar reader,
that wheras theſe prophecies are for moſt part hard to be vnderſtood,
and as S.~Peter teacheth,
\CNote{\XRef{2.~Pet.~3.}}
\Emph{not knowen by priuate interpretation, but muſt be interpreted by
the ſame Spirite, wherwith they were written}, our purpoſe is not to
explicate them, nor yet to produce large explications of the godlie
lerned Fathers, but rather fewer and briffer notes then hertofore, and
for the reſt we remitte the more lerned and ſtudious readers, according
to their capacities, to ſearch the ſame, in the commentaries of ancient
and late Expoſiters: wishing others to content themſelues, with the more
eaſie partes
%%% 1586
of holie Scriptures, and other godlie bookes, and daylie inſtructions of
ſpiritual teachers. And ſuch as do alſo read theſe, may obſerue with vs,
theſe (amongſt other) ſpecial cauſes of the hardnes of the Prophetes.
\MNote{Suddaine tranſition from one thing to an other.}
One cauſe is the frequent interruption of ſentences, with ſuddaine
change from one perſon, or matter to an other, without apparent
coherence. Which
\CNote{\XRef{S.~Ierom. in c.~2.~&3. Nahum.}}
S.~Ierom noteth in ſundrie places. As
\XRef{Iſaie.~7.}
after that the Prophet hath ſeuerely reprehended king Achab, for his
diſtruſt of Gods aſſiſtance againſt his temporal enimies
\XRef{(v.~13.)}
in the next wordes he prophecieth, that \Emph{a Virgin shal conceiue,
and beare a ſonne}, Chriſt our Sauiour, and the like in other places.
\CNote{\Cite{S.~Chryſ. ho.~8. in Math.~2.}}
\MNote{That which is ſpoken of certaine perſons is ment of others.}
An other cauſe is, that the Prophetes ſpeake thinges of ſome perſons,
which are to be fulfilled in others, either of their progenie, or
prefigured by them. As the prophecie of the Iewes and Gentiles,
compriſed in the hiſtorie of Eſau & Iacob. Likewiſe that which Iacob
prophecied
\XRef{(Gen.~49.)}
of Simeon
\Fix{an}{and}{likely typo, same in both}
Leui, not fulfilled in themſelues but in the Scribes, and
%%% o-1461
Prieſtes deſcending of their ſtock.
\CNote{\Cite{S.~Aug. de catech. rudibus c.~3.}}
Alſo much of that which Dauid ſemeth to ſpeake of Salomon,
\XRef{Pſal.~88.}
can only be vnderſtood of Chriſt. Other examples wil occurre in the
Prophetes enſuing.
\MNote{Prophecies are often vttered in figuratiue ſpeaches. Some conſiſt
in thinges done, other are mixt with hiſtories, and temporal thinges
with ſpiritual.}
Briefly, for we can not here expreſſe al the cauſes in few wordes,
prophecies are often times vttered in figuratiue ſpeaches, and often not
in wordes, but in factes;
\CNote{\Cite{Et cont. Mend. c.~10.}}
other times ſo mixed with hiſtories, and temporal thinges with
ſpiritual, againe ſome thinges perteyning to the old Teſtament, ſo
ioyned with myſteries of the new, and the like, that moſt hard it is to
diſcerne, nay not poſſible without ſpecial reuelation, or inſtruction of
others to know, to what purpoſe or thing euerie part perteyneth, or is
to be applied:
\CNote{\Cite{Eutym. in Pſal.~117.}}
for ſome thinges are ſpoken only of the hiſtorie, ſome thinges of
myſteries, manie thinges of both. And the reaſon why the Holie Ghoſt
doth ſo vtter theſe prophecies is noted by S.~Ierom
\Cite{(in Nahum.~3.)}
\MNote{VVhy God would haue them obſcure.}
that the proud and malicious enimies of Religion may not vnderſtand
them: \Emph{leſt} (ſayth he) \Emph{a holie thing should be geuen to
dogges, pearles caſt to ſwine, moſt ſacred myſteries laide oppen before
prophane perſons}. S.~Gregorie alſo alleageth an other reaſon
\Cite{(ho.~17. in Ezech.)}
that occaſion of humilitie \Emph{may be geuen vs by thoſe thinges, which
are hidden in holie Scriptures}. And increaſe alſo of merite by beleuing
more then we vnderſtand, \Emph{becauſe faith hath not merite, where
reaſon geueth experiment}.


\stopArgument


\stopcomponent


%%% Local Variables:
%%% mode: TeX
%%% eval: (long-s-mode)
%%% eval: (set-input-method "TeX")
%%% fill-column: 72
%%% eval: (auto-fill-mode)
%%% coding: utf-8-unix
%%% End:
