%%%%%%%%%%%%%%%%%%%%%%%%%%%%%%%%%%%%%%%%%%%%%%%%%%%%%%%%%%%%%%%%%
%%%%
%%%% The (original) Douay Rheims Bible 
%%%%
%%%% Old Testament
%%%% Isaia
%%%% Argument
%%%%
%%%%%%%%%%%%%%%%%%%%%%%%%%%%%%%%%%%%%%%%%%%%%%%%%%%%%%%%%%%%%%%%%




\startcomponent argument


\project douay-rheims


%%% 1586
%%% o-1461
\startArgument[
  title={\Sc{The Argvment of the Prophecie of Isaie.}},
  marking={The Argument of Isaie.}
  ]


Isaie
\MNote{Iſai of noble lineage and a martir prophecied a long time.}
the ſonne of Amos, and nephew (as S.~Ierom
\CNote{\Cite{Prefat. ad Paul & Euſto.}}
inſinuateth) to king Amaſias, prophecied in the times of Oſias, Ioathan,
Achaz, Ezechias, and in the beginning of Manaſſes, Kinges of Iuda; in al
aboue three ſcore
%%% 1587
yeares and was cruelly put to death, ſawed into partes by commandment of
Manaſſes. 
\CNote{\Cite{S.~Ier. Epiſt. ad Paulin.}
Et
\Cite{in com. Iſa.}
\Cite{S.~Aug. lib.~18. c.~27. ciuit.}
&
\Cite{lib.~9. c.~5. confeſſ.}}
\MNote{Is called the Euangelical Prophet.}
He is commonly called the \Emph{Euangelical Prophet}, for his ample and
particular ſpeaches of Chriſt, more large and more plaine then in anie
other of the old Prophetes.
\MNote{He writte in a high ſtile.}
His ſtile is high and eloquent, according to his liberal education being
of the royal bloud. For ſo it pleaſeth the Holie Ghoſt, to vtter his
diuine prophecies diuerſly according to the qualities, and conditions of
the perſons, by whom he ſpeaketh: by Iſaie in a loftie, and by Amos in a
meane ſtile: as a muſitian ſoundeth the ſame ſongue, by a ſimple pipe, &
by a cornet, trumpet or other muſical inſtrument. Which S.~Paul alſo
witneſſeth, ſaying:
\CNote{\Cite{Heb.~1.}}
\Emph{Diuerſly and by diuers meanes, God ſpake to the fathers in the
Prophetes.}
\MNote{Liued in the kingdom of Iuda.}
Iſaie therfore conuerſing in the kingdom of Iuda, eſpecially in the
Emperial and Metropolitan citie of Ieruſalem, preached & prophecied
manie thinges perteyning to the Tribes of Iuda and Beniamin, as alſo to
the tribe of Leui. Which after the ſchiſme of Ieroboam, repayred in
maner al to the kingdom of Iuda, where God was rightly ſerued. He
prophecied alſo of the tenne Tribes, the kingdom of Iſrael: & of the
future captiuities of them both,
%%% o-1462
and of the reduction of Iuda. Alſo he prophecied of other nations, and
peoples, with whom the Iewes had either emnitie, or freindlie
conuerſation: and of al the world. But moſt eſpecially of the coming of
Chriſt, to redeme, and deliuer mankind from captiuitie of ſinne.

The
\MNote{The contents, diuided into two general partes,}
whole prophecie conteyneth two general partes. Firſt more
principally the Prophet admonisheth, and threatneth the people, that
they shal be punished for their manifold ſinnes in the 39.~former
chapters. In the other 27. he comforteth them, ſignifying that God of
his mercie, wil after
\Fix{chatiſment,}{chaſtiſment,}{obvious typo, fixed in other}
& their repentance, deliuer them from their aduerſaries. Yet ſo that ech
part participateth of the principal contents with the other.
\MNote{and into eight particular.}
More particularly the whole booke may be diuided into eight partes.
\MNote{1.}
In the twelue firſt chapters, the Prophet admonisheth al ſortes in the
kingdom of Iuda, of their ingratitude towards God, with manie other
ſinnes and of iuſt punishment, but mixt with conſolation of Gods mercie,
and thankſegeuing for the ſame.
\MNote{2.}
In eleuen chapters folowing, he directeth his ſpeach to other Nations,
aduerſaries to the Iewes.
\MNote{3.}
In foure more he extendeth his admonitions to al the world, ſtil
intermixing ſome conſolations.
\MNote{4.}
In other foure he reprehendeth both the kingdoms, of Iſrael and Iuda,
for ſeeking helpe of ſtrange nations.
\MNote{5.}
In the next eight chapters he prophecieth of diuers dangers imminẽt to
the kingdom of Iuda, of their captiuitie in Babylon, of Gods benignitie
deliuering them, & very much in euerie part of Chriſt, and his Church.
\MNote{6.}
Then in fiue chapters he prophecieth very particularly of the
comfortable deliuerie from ſinne by Chriſt.
\MNote{7.}
In other foure from temporal captiuitie by Cyrus King of Aſſirians.
\MNote{8.}
And finally in the laſt eightene chapters, he prophecieth largely of
the perfect deliuerie by Chriſt, conuerſion of al Nations, reiection of
the Iewes, til nere the end of the world, when they shal alſo returne to
Chriſt.


\stopArgument


\stopcomponent


%%% Local Variables:
%%% mode: TeX
%%% eval: (long-s-mode)
%%% eval: (set-input-method "TeX")
%%% fill-column: 72
%%% eval: (auto-fill-mode)
%%% coding: utf-8-unix
%%% End:
