%%%%%%%%%%%%%%%%%%%%%%%%%%%%%%%%%%%%%%%%%%%%%%%%%%%%%%%%%%%%%%%%%
%%%%
%%%% The (original) Douay Rheims Bible 
%%%%
%%%% Old Testament
%%%% One Kings
%%%% Chapter 28
%%%%
%%%%%%%%%%%%%%%%%%%%%%%%%%%%%%%%%%%%%%%%%%%%%%%%%%%%%%%%%%%%%%%%%




\startcomponent chapter-28


\project douay-rheims


%%% 0649
%%% o-0582
\startChapter[
  title={Chapter 28}
  ]

\Summary{The
\MNote{The fourth part.

Of the ruine of Saul, and exaltation of Dauid.}
Philiſtians fighting againſt Saul, Dauid promiſeth
  fidelitie to Achis. 3.~Saul deſtroyeth magicians, 6.~but God not
  anſwering him, 7.~ſeeketh a woman that hath a Pithon ſpirite,
  12.~willeth her to raiſe vp Samuel. 15.~Who appearing foretelleth him,
  that he, and his ſonnes shal die the next day.}

And it came to paſſe that in thoſe daies the Philiſtijms gathered
together their companies, that they might be prepared to battel againſt
Iſrael: and Achis ſayd to Dauid: Knowing thou now, that thou shalt goe
forth with me in the campe, thou, and thy men. \V And Dauid ſayd to
Achis: Now thou shalt know what thy ſeruant wil doe. And Achis ſayd to
Dauid: And I wil appoint thee keper of my head al daies. \V And Samuel
was dead, and al Iſrael mourned for him, and buried him in Ramatha his
citie. And Saul tooke al the magicians and ſoothſayers out of the
land. \V And the Philiſtijms were gathered together, and came and camped
in Sunam: and Saul alſo gathered together al Iſrael, and came into
Gelboe. \V And Saul ſaw the campe of the Philiſtijms, and feared, and
his hart was afrayd excedingly. \V And he conſulted our Lord, and he
anſwered him not, neither by dreames, nor by prieſtes, nor by
prophetes. \V And Saul ſayd to his ſeruantes: Seeke me a woman that hath
a pithonical ſpirite, and I wil goe to her, and wil aske by her. And his
ſeruantes ſayd to him: There is a woman that hath a pithonical ſpirite
in Endor. \V He therefore changed his habite and was clothed with other
garmentes, and he went himſelfe, and two men with him, and they came to
the woman in the night, and ſayd to her: Deuine vnto me in the
pythonical ſpirite, and raiſe me vp whom I ſhal tel thee. \V And the
woman ſayd to him: Loe, thou knoweſt what great thinges Saul hath done,
%%% 0650
and how he hath rayſed the magicians and ſoothſayers out of the land:
why therefore doeſt thou lye in waite for my life, that I may be
ſlaine? \V And Saul ſware vnto her in our Lord, ſaying: Our Lord liueth,
there shal no euil happen vnto thee for this thing. \V And the woman
ſayd to him: Whom ſhal I rayſe vp to thee? Who ſayd: Raiſe me vp
Samuel. \V And when the woman had ſeene Samuel, ſhe cried out with a
loud voice, and ſayd to Saul: Why haſt thou deceiued me? for thou art
Saul. \V And the King ſayd to her: Feare not: what ſaweſt thou? And the
woman ſayd to Saul: I ſaw
\SNote{Not manie but one excelent perſon, an old man comelie in apparel.}
Goddes coming out of the earth. \V And he ſayd to her: What maner of
forme hath he? Who ſayd: An old man is come vp, and he is clothed with a
mantel. And
\LNote{Saul vnderſtood that it vvas Samuel.}{It is not defined nor
certaine, whether the ſoule of Samuel appeared, or an euil ſpirit tooke
his ſhape, and ſpake to Saul.
\MNote{S.~Auguſtins opinion whether Samuels ſoule appeared, or no.}
S.~Auguſtin
\Cite{(li.~2. q.~3. ad Simplician)}
propoſeth both the opinions as probable. VVhere firſt he ſheweth, that
Samuels ſoule might appeare; either brought thither by the euil ſpirite,
which is not ſo much to be merueled at, as that our Lord and Sauiour
ſuffered him ſelf to be ſette vpon the pinnacle of the temple, and to be
caried into a high mountaine by the diuel; yea to be taken priſoner,
bound, whipped, and crucified, by the diuels miniſters: or els that the
ſpirite of the holie prophet, was not raiſed by force of the inchantment, or
anie power of the diuel, but by Gods ſecrete ordinance vnknowen to the
pythonical woman, and to Saul, and ſo appeared in the kings preſence,
and ſtroke him with diuine ſentence. Againe he anſwereth, that there may
be a more eaſie and readie ſenſe of this place, to wit, that Samuels
ſpirite (or ſoule) was not in deede raiſed, but an imaginarie illuſion
made by the diuels inchantment, which ſemed to be Samuel, and which the
Scripture calleth by the name of Samuel, as pictures or images are
commonly called thoſe perſons or thinges which they repreſent. So when
we behold pictures in a table, or on a wal, we ſay, this is Cicero, that
is Saluſt, that Achilles, that is Rome. To this effect S.~Auguſtin
diſcourſeth more at large in the place before cited.
\MNote{More probable that his verie ſoule appeared, not compelled by the
euil ſpirite, but obeying Gods ſecrete ordinance.}
But in an other worke written after
\Cite{(de cura pro mortuis gerenda. c.~15.)}
teaching that ſoules of the dead appeare ſometimes to the liuing, he
ſaieth expreſly, \Emph{Samuel the prophet being dead, foretold future
thinges to King Saul yet liuing.} Though ſome be of opinion (ſaieth he)
that Samuel himſelf appeared not, but ſome euil ſpirit tooke his
ſimilitude.

And this laſt iudgement of S.~Auguſtin is much confirmed;
\MNote{Firſt proofe.}
firſt by the wordes of this text, literally and plainly affirming that
Samuel appeared, and ſpake to Saul, and Saul to him, and that Saul
\Emph{vnderſtood} (or \Emph{knevv}, not only thought, imagined, or
ſuppoſed) \Emph{that it was Samuel}.
\MNote{2.}
Secondly, this apparition came ſooner, preuenting the inchantment, and
in better order, then the pithonical woman expected, as appeareth by her
anſwer, ſaying ſhe ſaw God (or an excellent perſon) aſcending in comelie
maner and attyre: whereas euil ſpirites vſed to appeare (as the Rabbins
teſtifie) in vglie bodies, with the heeles into the ayer, and head
downward.
\MNote{3.}
Thirdly, the Author of Eccleſiaſticus
\XRef{(ch.~46.)}
amongſt the prayſes of Samuel the prophet, ſaieth: \Emph{he ſlept}, (or
died) \Emph{and certified the King, and shevved to him, the end of his
life}. VVhere it ſemeth clere, that the ſame perſon that died, denounced
Gods wil and ſentence to Saul. Moreouer if it had bene an illuſion of an
euil ſpirite, it would hardly ſeme anie praiſe at al.
\MNote{4.}
Fourthly, the diuel could not naturally foretel that Saul and his
ſonnes, with manie of the people ſhould be ſlaine the next day, and
Dauid reigne after him: neither is it probable, that God reueiled ſuch
ſecretes to euil ſpirites, wherby men might take more occaſion to folow
nicromancie.
\MNote{5.}
Fiftly, moſt Fathers and Doctors are of the ſame iudgement. S.~Iuſtinus
Martyr
%%% !!! Add CNote to below Cite?
\CNote{\Cite{to.~2. pag.~210.}}
\Cite{Dialogo cum Triphone.}
S.~Ambroſe
\Cite{li.~2. in Luc.~1.}
S.~Hierom
\Cite{in Iſaiæ.~7.}
Ioſephus
\Cite{li.~6. c.~15. Antiq.}
and manie other old and late writers. The chiefeſt argument for the
other opinion is the authoritie of Tertullian
\Cite{li. de animus.}
Procopius and Eucherius
\Cite{vpon this place}
and the vncertaine authors
\Cite{Quæſtionem apud Iuſtinum q.~52.}
\Cite{lib. de mirabil. Sac. Scrip.}
and
\Cite{Quæſt. vet. Teſtamenti, q.~27. apud Auguſtinum. tomo.~3. et.~4.}
\MNote{Soules ſometimes appeareth after death.}
As for the Proteſtantes denying, that ſoules once parted from their
bodies, can appeare to anie aliue,
\CNote{\Cite{loco citato.}}
S.~Auguſtin confuteth them, both by this example of Samuel, ſuppoſing
the booke of Eccleſiaſticus to be Canonical Scripture, and of Moyſes
being dead, and Elias yet liuing (whom they hold alſo to be dead) both
appearing with Chriſt in his transfiguration.
\XRef{Mat.~17.}}
Saul
\TNote{\HH{iadagh} \L{cognouit} \Emph{knevv}.}
vnderſtood that it was Samuel, and he bowed himſelfe vpon his face on
the earth, and
\SNote{Saul adored not Samuel with diuine honour, but with dulia,
reuerence due to a bleſſed ſoule.}
adored.
%%% o-0583
\V And Samuel ſayd to Saul: Why haſt thou diſquieted me, that I
ſhould be raiſed vp? And Saul ſayd: I am in great diſtreſſe: for the
Philiſtijms fight againſt me, and God is departed from me, and would not
heare me, neither in the hand of prophetes, nor by dreames: therefore I
haue called thee, that thou ſhouldeſt ſhew me, what I ſhal doe. \V And
Samuel ſayd: Why askeſt thou, whereas our Lord is departed from thee,
and is paſſed to thine aduerſarie? \V For our Lord wil doe to thee as he
ſpake in my hand, and he wil cut thy kingdome out of thy hand, & wil
geue it to thy neighbour Dauid: \V becauſe thou haſt not obeyed the
voice of our Lord, neither didſt thou the wrath of his furie in
Amalec. Therefore that which thou ſuffereſt hath our Lord done to thee
this day. \V And our Lord wil geue Iſrael alſo with thee into the handes
of the Philiſtijms: and to morow thou and thy ſonnes ſhal be
\SNote{In ſtate of the dead in an other world, not in the ſame
particular ſtate.}
with me: yea the campe alſo of Iſrael wil our Lord deliuer into the
handes of the Philiſthijms. \V And forthwith Saul fel ſtretched forth on
the ground, for he feared much the wordes of Samuel, and there was no
ſtrength in him, becauſe he had not eaten bread al that day. \V That
woman therefore went vnto Saul (for he was very much trubled) and ſayd
to him: Behold thy handmaide hath obeied thy voice, and I haue put my
life in my hand: and I heard the wordes, which thou ſpakeſt to me. \V
Now therefore heare thou alſo the voice of thy handmaide, and I wil ſet
before thee a morſel of bread, that eating thou mayeſt recouer ſtrength,
and be able to goe on thy iourney. \V Who refuſed, and ſayd: I wil not
eate. But
%%% 0651
his ſeruantes and the woman forced him, and at length hearing their
voice, he aroſe from the ground, and ſate vpon the bed. \V And that
woman had a paſture fed calfe in the houſe, and ſhe made haſt, and
killed him: and taking meale kneded it, and baked azimes, \V and ſette
before Saul, and before his ſeruantes, who when they had eaten roſe vp,
and walked al that night.


\stopChapter


\stopcomponent


%%% Local Variables:
%%% mode: TeX
%%% eval: (long-s-mode)
%%% eval: (set-input-method "TeX")
%%% fill-column: 72
%%% eval: (auto-fill-mode)
%%% coding: utf-8-unix
%%% End:
