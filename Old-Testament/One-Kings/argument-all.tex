%%%%%%%%%%%%%%%%%%%%%%%%%%%%%%%%%%%%%%%%%%%%%%%%%%%%%%%%%%%%%%%%%
%%%%
%%%% The (original) Douay Rheims Bible 
%%%%
%%%% Old Testament
%%%% One Kings
%%%% Argument All
%%%%
%%%%%%%%%%%%%%%%%%%%%%%%%%%%%%%%%%%%%%%%%%%%%%%%%%%%%%%%%%%%%%%%%




\startcomponent argument-all


\project douay-rheims


%%% 0591
%%% o-0531
\startArgument[
  title={\Sc{The Argvment of the Bookes of Kinges and Paralippomenon in General.}},
  marking={Argument of Kinges.}
  ]

After the booke of Iudges (wherunto Ruth is annexed) rightly folow the
bookes of Kinges: ſignifying that after the general \Emph{Iudgement}
cometh the \Emph{euerlaſting Kingdome}.
\MNote{Theſe hiſtories are alſo expounded myſtically by the ancient
Fathers.}
As
\CNote{\Cite{qq. in 1.~Reg. c.~1.}}
venerable Beda expoundeth this connexion of bookes, wherin he alſo
explicateth manie other Myſteries of Chriſt & the Church præfigured in
theſe hiſtories. Likewiſe
\CNote{\Cite{Prologo. in 1.~Reg.}}
S.~Gregorie teacheth that beſides the \Emph{hiſtorical & moral ſenſe
expreſſed in the ſimplicitie of the letter, an other myſtical
vnderſtanding is to be ſought in the height of the Allegorie}. In
confirmation wherof he citeth
\CNote{\Cite{li.~17. c.~4. ciuit.}}
S.~Auguſtin and
\CNote{\Cite{Ep. ad Paulin.}}
S.~Hierom; who ſay, that Elcana his two wiues ſignified the Synagogue of
the Iewes, and the Church of Chriſt: & that the death of Heli & Saul,
with tranſlation of
%%% 0592
Prieſthood to Samuel and Sadoch, and of the Kingdome, to Dauid and his
Succeſſors, præfigured the \Emph{new Prieſthood, and new Kingdome of
Chriſt} the old ceaſing which were shadowes therof. So theſe two great
Doctors S.~Gregorie and S.~Beda, inſiſting in the ſteppes of other
lerned holie Fathers, that had gone before them, expound theſe hiſtories
not only hiſtorically but alſo
%%% o-0532
myſtically.
\MNote{The general contents of al the bookes of Kinges &
Paralippomenon.}
The hiſtorie firſt ſetteth forth the changing of the forme of gouernment
from Iudges to Kinges: and then at large what Kinges did reigne ouer the
Hebrew people, as wel in one intire Realme, as ouer the ſame people
diuided into two kingdomes; their more principal Actes; their good and
euil behauiour; alſo the proſperitie, declinations, and final
captiuities of both the Kingdomes. Al which is conteined in \Emph{foure
bookes of Kinges}, with other two partly repeting that was ſaied before,
but eſpecially ſupplying thinges omitted in the whole ſacred hiſtorie
from the beginning of the world, called \Emph{Paralippomenon}.
\MNote{Samuel writ the firſt part, but vncertaine who writ the reſt.}
The two firſt are alſo called the \Emph{Bookes of Samuel}, though he
writ not one of them wholly, for he died before the hiſtorie of the
former ended; but they goe both vnder his name, becauſe he annointed the
two firſt Kinges, and writ a great part of their Actes. Wherto the reſt
was added either by Dauid and Salomon, as ſome thinke, or by Nathan and
Gad, as is probably gathered,
\XRef{1.~Paralip.~29. v.~29.}
The authors alſo of the third and fourth bookes of Kinges, and of the
two of Paralippomenon are vncertaine; yet al haue euer bene receiued and
held for Canonical Scripture.


\stopArgument


\stopcomponent


%%% Local Variables:
%%% mode: TeX
%%% eval: (long-s-mode)
%%% eval: (set-input-method "TeX")
%%% fill-column: 72
%%% eval: (auto-fill-mode)
%%% coding: utf-8-unix
%%% End:
