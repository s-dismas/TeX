%%%%%%%%%%%%%%%%%%%%%%%%%%%%%%%%%%%%%%%%%%%%%%%%%%%%%%%%%%%%%%%%%
%%%%
%%%% The (original) Douay Rheims Bible 
%%%%
%%%% Old Testament
%%%% One Kings
%%%% Chapter 08
%%%%
%%%%%%%%%%%%%%%%%%%%%%%%%%%%%%%%%%%%%%%%%%%%%%%%%%%%%%%%%%%%%%%%%




\startcomponent chapter-08


\project douay-rheims


%%% 0605
%%% o-0543
\startChapter[
  title={Chapter 8}
  ]

\Summary{Samuel growing old, and his ſonnes for bribes peruerting
  iudgement, the people require to haue a king. 7.~To whom by Gods
  commandment, Samuel forsheweth the law of a king, to make them ceaſe
  from their demand; 19.~but they perſiſt therin.}

And it came to paſſe when Samuel waxed old, he appoynted his ſonnes
iudges ouer Iſrael. \V And the name of his firſt begotten ſonne was
Ioel: and the name of the ſecond Abia, iudges in Berſabee. \V And
his ſonnes walked not in his waies: but they declined after auarice, &
tooke bribes, and peruerted iudgement. \V
\SNote{Heli his ſonnes grieuouſly offending in their office before
\XRef{(chap.~2.)}
and now Samuels ſonnes alſo peruerting iudgemẽt gaue occaſion to the
people, to demand a king to iudge their temporal cauſes rightly not
declining to wrong for bribes.}
Therfore al the ancientes of
%%% o-0544
Iſrael being aſſembled, came to Samuel into
Ramatha. \V And they ſayd to him: Behold thou art old, and thy ſonnes
walke not in thy wayes: appoynt vs a king, that he may iudge vs, as alſo
al nations haue. \V And the word was miſliked in the eyes of Samuel,
becauſe they had ſayd: Geue vs a king, that he may iudge vs. And Samuel
prayed to our Lord. \V And our Lord ſayd to Samuel: Heare the voice of
the people in al thinges which they ſpeake to thee. For they haue not
\LNote{Reiected me.}{For
\MNote{VVhy the peoples demand to haue a king is diſliked.}
ſo much as
\CNote{\XRef{Exod.~19.}}
God had choſen Iſrael a peculiar people to him ſelf, and hitherto
\CNote{\XRef{Deut.~17.}}
ruled the ſame by his Prieſtes eſtabliſhed among them, and
\CNote{\XRef{Iudic.~2. v.~16.}}
by Iudges extraordinarily raiſed vp, and ſent by him, to deliuer them in
their diſtreſſes, their demand now to haue a King, who (after the maner
of other nations) ſhould be their Lord, and haue more dignitie, and
authoritie ouer them, then Dukes or Iudges had, is interpreted, as in
effect to reiect God: in that they diſliked, & ſought to change his
forme of gouernment. And therfore this requeſt of the people iuſtly
diſpleaſed both Samuel and God himſelf.}
reiected thee, but me, that I ſhould not reigne ouer them. \V According
to al their workes, which they haue done from the day that I brought
them out of Ægypt vntil this day: as
%%% 0606
they haue forſaken me, and ſerued ſtrange goddes, ſo doe they alſo vnto
thee. \V Now therfore heare their voice: but yet teſtifie to them, and
foretel them the
\SNote{\HH{Miſphat} ſignifieth maner, faſion, or proceding.}
right of the king, that ſhal reigne ouer them. \V Samuel therfore ſpake
al the wordes of our Lord to the people which had deſired a king of
him, \V and ſayd: This ſhal be
\LNote{The right of the King.}{Samuel
\MNote{Kinges ſometimes oppreſſe their ſubiectes by Gods ſufferance, but
vniuſtly.}
here by Gods appointment, to diſwade the people from their deſire of a
king, at leaſt to admoniſh them before hand, what they are like to find
by experience, reciteth ſuch thinges, as Kinges abuſing their powre do
oftentimes practiſe, by reaſon of their high dignitie, and litle feare
of controlment, but vniuſtly and vnlawfully; according to the doctrin of
ancient Fathers. Amongſt others,
\CNote{\Cite{S.~Cyp. li.~3. ep.~9. ſiue.~65.}}
S.~Cyprian calleth the exactions of kinges here recited, \Emph{greuous
iniuries}.
\CNote{\Cite{S.~Hiero. in Oſee.~8.}}
S.~Hierom \L{dura imperia}, & \L{ſeruitutem}, \Emph{rigorous or cruel
gouernmentes}, and \Emph{ſeruititude}.
\CNote{\Cite{S.~Greg. li.~4. c.~2. in 1.~Reg.~8.}}
S.~Gregorie proueth the ſame by two contrarie examples. Seing (ſayeth
he) that which is here foretold, was puniſhed in Achab and Ieſabel
\XRef{(3.~Reg.~21.)}
it ſheweth, that it was not right by diuine iudgement, which they
exacted. And when the elect King Dauid was to build an altar to our Lord
\XRef{(1.~Paral.~21.)}
he would not take part of Ornans field, except he payed a iuſt price for
it. Moreouer the law preſcribing the dutie of Kinges
\XRef{(Deut.~17. v.~16. &c.)}
commandeth them not to multiplie horſes, nor \Emph{to heape riches} nor
to \Emph{take high courage, that their hartes be not lifted vp into
pride ouer their brethren}.
\MNote{Kinges haue prerogatiues aboue, but not contrarie to the lawes.}
Neuertheles Kinges haue great prerogatiues (more then Dukes, and Iudges)
beſides, and aboue, but neuer contrarie to the law: that albeit they can
not take their ſubiectes landes or goodes, neither for themſelues, nor
to geue to their ſeruantes at their pleaſure: yet in diuers caſes
ſubiectes are bound, to contribute of their priuate goodes, to ſupplie
the neceſſitie of the King, or of the commonwealth, as by nature euerie
part muſt ſuffer damage, or danger in defence of the principal member,
or whole bodie. And if anie refuſe ſo to do, they may iuſtly be
compelled.

Furthermore
\CNote{\Cite{Concil. Lateran c.~3. de hæret.}}
\MNote{Euil princes may be depoſed by God & the Church: but not by the
people only.}
in caſe Kinges or other Princes commit exceſſes, and oppreſſe their
ſubiectes, yet are they not by and by to be depoſed by the people, nor
commonwealth, but muſt be tolerated with patience, peace, and meeknes,
til God by his ſouereigne authoritie, left in his Church, diſpoſe of
them: which his diuine wiſdom and goodnes often differreth to do, as here
he expreſly forewarneth, ſaying:
\XRef{(v.~18.)}
\Emph{You shal crie in that day, from the face of your King, and our
Lord vvil not heare you}. And the reaſon is, becauſe he wil puniſh the
ſinnes of the people, by ſuffering euil princes to reigne.
\XRef{Iob.~34. v.~30.}

Of
\MNote{Pointes obſerued in the conſtitution and depoſition of King
Saul.}
which important difficultie, falling ſometimes betwen Princes and their
ſubiectes, who ſo deſireth, may ſearch the iudgement of ancient Fathers,
and ſee S.~Thomas, and other ſchole Doctors,
\Cite{2.~2. q.~12. a.~2.}
Here only for better vnderſtanding of this preſent text, theſe brief
pointes may be obſerued.
\MNote{1.}
Firſt, the people of their owne wil deſired to haue a King.
\MNote{2.}
Secondly, they requeſted the ſame at the handes of Samuel their preſent
Superiour.
\MNote{3.}
Thirdly, this demand diſpleaſed both Samuel and God himſelf.
\MNote{4.}
Fourthly, yet God condeſcended to grant their ſuite, but with an
admonition, and forewarning of the inconueniences, which they ſhould
finde and feele.
\MNote{5.}
Fiftly, God himſelf deſigned the perſon that ſhould be King, reueled him
by viſion, and commanded Samuel to annoint him.
\MNote{6.}
Sixtly, God neuertheles by guiding the lotte, more manifeſtly declared,
and confirmed his election.
\MNote{7.}
Seuently, God depoſed the ſame King, for tranſgreſſing his law,
\XRef{chap.~13. v.~13.}
and diſobeying his commandment,
\XRef{chap.~15. v.~13.}
appointing an other, by the miniſterie of Samuel.
\XRef{chap.~16.}
\MNote{8.}
Eightly, notwithſtanding his depoſition, he remained in his dignitie til
his death, which happened by other meanes.
\XRef{chap.~31.}
By al which it appeareth, that God conſtituted Saul the firſt King of
the Iewes, the people ſuing to haue a King; but depoſed him for euil
behauiour, the people deſiring no ſuch thing, and Samuel the Prophet
much lamenting the ſame. Yet was he not actually bereaued of the crowne
and kingdom during his life.}
the right of the king, that ſhal reigne ouer you: Your ſonnes he wil
take, and put in his chariotes, and wil make them vnto him the horſemen,
and running footmen before his chariotes, \V and wil appoynt them his
tribunes, and centurions, and the plowers of his fieldes, and mowers of
his corne, and makers of his armour and of his chariotes. \V Your
daughters alſo wil he take to make ointementes, and to be cookes, and
bakers. \V Your fieldes alſo, and vineyardes, and the beſt oliuetes he
wil take away, and geue to his ſeruantes. \V Yea and your corne alſo,
and the reuenewes of your vineyardes he wil tithe, to geue his eunuches
and ſeruantes. \V Your ſeruantes alſo and handmaides, and goodlieſt yong
men, and aſſes he wil take away, and put in his worke. \V Your flockes
alſo wil he tithe, you ſhal be his ſeruantes. \V And you ſhal crie in
that day from the face of the king, which you haue choſen you: and our
Lord
\SNote{God alwaies heareth thoſe that truly repent for their ſinnes, but
doth not alwayes deliuer them from afflictions, which are due for
offences, or profitable for probation and merite of his children.}
wil not heare you in that day, becauſe you deſired vnto your ſelues a
king. \V But the people would not heare the voice of Samuel, but ſayd:
Not ſo: for there ſhal be a king ouer vs, \V and we alſo wil be as al
nations: and our king ſhal iudge vs, and ſhal goe forth before vs, and
shal fight our battels for vs. \V And Samuel heard al the wordes of the
people, and ſpake them in the eares of our Lord. \V And our Lord ſaid to
Samuel: Heare their voice, and appoynt a king ouer them. And Samuel ſayd
to the men of Iſrael: Let euerie man goe into his citie.


\stopChapter


\stopcomponent


%%% Local Variables:
%%% mode: TeX
%%% eval: (long-s-mode)
%%% eval: (set-input-method "TeX")
%%% fill-column: 72
%%% eval: (auto-fill-mode)
%%% coding: utf-8-unix
%%% End:
