%%%%%%%%%%%%%%%%%%%%%%%%%%%%%%%%%%%%%%%%%%%%%%%%%%%%%%%%%%%%%%%%%
%%%%
%%%% The (original) Douay Rheims Bible 
%%%%
%%%% Old Testament
%%%% Exodus
%%%% Chapter 07
%%%%
%%%%%%%%%%%%%%%%%%%%%%%%%%%%%%%%%%%%%%%%%%%%%%%%%%%%%%%%%%%%%%%%%




\startcomponent chapter-07


\project douay-rheims


%%% 0189
%%% o-0172
\startChapter[
  title={Chapter 7}
  ]

\Summary{Moyſes being conſtituted as God of Pharao, and Aaron as the
  prophet of Moyſes, they declare Gods commandment to Pharao; 10.~turne
  the rodde into a ſerpent; 17.~& the water into bloud, which is the
  firſt plague. 22.~The magicians doe the like by inchantments, and
  Pharaos hart is indurate.}

And our Lord ſaid to Moyſes: Behold I haue appointed thee
\LNote{The God of Pharao.}{The
\MNote{The name of God attributed to men.}
name of God, which eſſentially is proper only to the three Diuine
Perſons of the B.~Trinitie, and incommunicable to anie creature
\XRef{(Sap.~14.)}
is neuertheles by ſimilitude attributed in holie Scripture to other
perſons.
\MNote{Iudges called goddes.}
As
\XRef{(Exod.~12. v.~8.)}
Iudges, or princes, are called goddes, for the eminent authoritie and
powre which they haue from God.
\MNote{Moyſes the God of Pharao.}
So Moyſes was conſtituted the Iudge and
God of Pharao, not only to puniſh him, for his obſtinacie, and finally
to compel him to diſmiſſe the Iſraelites out of Ægypt, but alſo to
terrifie him ſo in the meane time, that he being otherwiſe a mightie
King, and extremly and often afflicted by Moyſes, yet durſt neuer lay
violent handes vpon him, leſt himſelfe, and al his nation ſhould
preſently haue bene deſtroyed. As S.~Hilarie
\Cite{(lib.~7. de Trinitate.)}
& S.~Gregorie
\Cite{(ho.~8. in Ezech.)}
note vpon this place.
\MNote{Prieſts called goddes.}
Likewiſe Prieſts are called goddes
\XRef{(Exod.~22. v.~28.)}
for their ſacred function, pertaining to Religion and Seruice of God.
\MNote{Other titles of God geuen to men.}
Prophetes alſo are called \L{Videntes}, \Emph{Seers}
\XRef{(1.~Reg.~9.)}
becauſe by participation of diuine knowledge, they ſee ſometimes the
ſecretes of other mens hartes, things ſupernatural, and future
contingent, though properly and naturally onlie God almightie is
\L{Scrutator cordis}, \Emph{the ſearcher of the hart}, and knoweth al
things.
\XRef{(Sap.~1.)}
Againe S.~Peter ſaieth
\XRef{(2.~Epiſt. c.~1.)}
that \Emph{iuſt men are made partakers of diuine nature.} VVhich is
rather more then to participate in name.
\MNote{Moyſes a Holie Prophete, Prieſt, and Prince.}
Al which titles rightly perteined to Moyſes, being in life Holie, in
knowledge a Prophete, in function a
\CNote{\XRef{Pſal.~98. v.~6.}}
Prieſt, and in powre a Prince. In the ſame ſenſe of participation,
Saintes are called our Mediators, Aduocates, Redemers, Deliuerers, and
the like.}
the God of Pharao: and Aaron thy brother ſhal be
\SNote{Aaron alſo was the prophet of God, but ſubordinate vnder Moyſes,
and ouer Pharao.
\Cite{S.~Aug. q.~17. in Exod.}}
thy prophet. \V Thou ſhalt ſpeake to him al thinges that I command thee:
and he ſhal ſpeake to Pharao, that he diſmiſſe the children of
Iſrael out of his land. \V But
\LNote{I vvil indurate.}{According to our purpoſe mentioned in the
\XRef{Annotations vpon the 9.~chap. to the Romanes,}
we ſhal here recite the ſumme of S.~Auguſtins doctrine
\Cite{(Ser.~88. de tempore)}
touching the hard queſtion: How God did indurate Pharaoes hart. And
withal we ſhal briefly explicate, according to the doctrine of the ſame,
& other moſt learned Fathers of the Church, the true ſenſe of this and
like places, by which
\MNote{Proteſtants hold God to be the cauſe that men do ſinne, yet not
the cauſe of ſinne.}
Zuinglius, Caluin, Beza, and other Sectaries,
would proue that God not only permitteth, but alſo commandeth,
inclineth, inforceth, and compelleth men to do that which is ſinne: yea that
God is the author, internal mouer, & inforcer, that man tranſgreſſeth;
though they denie that God ſinneth, or is cauſe of the malice of
ſinne.
\MNote{Zuinglius doctrine.}
For exãple, Zuinglius
\Cite{(Ser. de prouidentia Dei, ca.~5.)}
ſaieth: \L{Numen ipſum auctor eſt eius, quod nobis eſt iniuſtitia, illi
veri nullatenus eſt.} \Emph{The diuine povvre itſelfe is author of that
thing, vvhich to vs is iniuſtice, but to him in no vviſe is.} And a
litle after, \L{Cum igitur Angelum tranſgreſsorem facit, & hominem,
ipſe tamen tranſgreſſor non cõſſtitur.} \Emph{VVhen therfore God maketh
Angel, and man tranſgreſſor, yet himſelf is not made a tranſgreſſor.}
\Cite{Cha.~6.}
\L{Vnum igitur atque idem facimus, puta adulterum aut homicidium,
quantum Dei eſt auctoris, motoris, impulſoris, opus eſt, crimen non eſt:
quantum hominis eſt, crimen ac ſcelus eſt.} Therfore the ſelfsame act,
as adulterie or manſlaughter, as it is of God \Emph{the author, mouer,
inforcer}, is a worke, is not a crime: but as it is of man, is a crime,
& a wicked act.
\MNote{Caluins doctrine.}
Caluin
\Cite{(li.~8. Inſtit. c.~17. para.~11.)}
affirmeth that the diuel, & the whole band of the wicked can not
conceiue, nor endeuoure, nor doe anie miſchief, \L{niſi quantum Deus
Permiſerit, imo niſi quantum ille mandarit.} \Emph{but ſo farre as God
permitteth} (which al Catholiques firmly beleue) \Emph{nay but ſo far as
he cõmandeth}: which al Catholiques abhorre and deteſt. Likewiſe
\Cite{(li.~2. c.~4. para.~4.)}
alleaging Gods wordes, ſaying \Emph{he had aggrauated, and hardned
Pharaoes hart}, affirmeth, that which God did more, beſides not
mollifying his hart, was, \L{quod obſtinatione pectus eius obfirmandum
Satanæ mandauit}, \Emph{that he committed his hart to Satan to be
obdurated vvith obſtinacie}: making God the author, and Satan only the
miniſter of hardning Pharaos hart.
\MNote{Bezas doctrin.}
Beza folowing this race
\Cite{(in Reſpon. ad Caſtallionem, Aphoriſmo~22.)}
ſaieth, God ſo \Emph{vvorketh} by euil inſtruments, that he doth not
only ſuffer them to worke, nor only moderateth the euent, \L{ſed etiam
vt excitet, impellat, moueat, regat, atque adeo (quod omniũ eſt maximum)
etiam creet, vt per illa agat quod conſtituit}: but alſo ſturreth them
vp, driueth them forward, moueth them, ruleth them, and (which is moſt of
al) euen createth them, that by them he may \Emph{vvorke} that which he
appointed. \Emph{Al vvhich} (ſaith he) \Emph{God doth rightly, and
vvithout anie iniuſtice.}
\MNote{By their doctrin it neceſſarily foloweth, that God ſhould be
author of ſinne.}
So in dede theſe men ſay, when they are preſſed with the blaſphemous
abſurditie, that they make God author and cauſe of ſinne, which
neceſſarily and euidently foloweth of their doctrin. For by the very
light of nature, it is clere, that the commander or inforcer is author
of that euil which an other doth, by his cõmandement or inforcement, and
by al law of nature and nations, diuine and humane, is condemned as
culpable and guiltie of the fault, which the other cõmitteth: but theſe
miniſters ſay (in the places aboue cited) \Emph{God cõmandeth,
inforceth, and vvorketh} al that a ſinner doth. Ergo, God by this
doctrin muſt be author, culpable, and guiltie of ſinne. VVhich is ſo
blaſphemous, and horrible to Chriſtian eares, that they dare not ſay it
in expreſſe termes.

Seing
\MNote{The ſtate of the controuerſie.}
then God is ſaid to haue indurated Pharaoes hart, and al confeſſe
that induration of hart is a moſt greuous ſinne, the controuerſie is:
VVhether God commanded, inforced, and wrought the induration in Pharaoes
hart, or only permitted it? or what els God did to Pharao, wherby his
hart was indurate; and finally by whom it was properly indurate, by God,
or by Pharao him ſelfe?
\MNote{S.~Auguſtins doctrin.
\Cite{ſer.~88. de temp.}}
Al which S.~Auguſtin explicateth, laying firſt this ground (which euerie
one is faithfully and firmly to beleue) that
\MNote{God forſaketh not, til he be forſaken.}
God neuer forſaketh any man, before he be firſt forſaken by the ſame
man: yea God alſo long expecteth, that a ſinner which much and often
offendeth,
\CNote{\XRef{Ezech.~33.}}
\Emph{conuert and liue}. But when the ſinner abideth long in his
wickednes, of the multitude of ſinnes riſeth deſperation, of deſperation
is ingendred obduration.
\CNote{\XRef{Pro.~18.}}
\Emph{For vvhen the impious is
\Fix{comen}{come}{obviuos typo, fixed in other}
to the depth of ſinnes, he contemneth.} Obduration therfore cometh not
of Gods powre compelling, but is ingendred by Gods remiſnes, or
indulgence, and ſo not diuine powre, but diuine patience did harden
Pharaoes hart.
\MNote{God by not puniſhing permitted Pharao to indurate him ſelf. And
that for his former ſinnes.}
How often ſoeuer therfore our Lord ſaieth: \Emph{I vvil indurate the
hart of Pharao}, he would nothing els to be vnderſtood, but I wil ſuſpend
my plagues and puniſhments, wherby I wil permit him through mine
indulgence to be obdurate againſt me. Perhaps ſome wil aſke, why did God
by ſparing him, let him be indurate? why did God take from him his
wholſome puniſhment? I anſwer ſecurely: this was done, becauſe Pharao,
for the huge heape of his ſinnes, deſerued not as a child, to be
corrected vnto amendment, but as an enemie was ſuffered to be
indurate. For of them, whom Gods mercie ſuffereth not to be indurate, it
is written:
\CNote{\XRef{Heb.~12.}}
\Emph{God ſcourgeth euerie child vvhom he receiueth.} And in an other
place:
\CNote{\XRef{Apoc.~3.}}
\Emph{VVhom I loue I correct and chaſtiſe.} Againe:
\CNote{\XRef{Prou.~8.}}
\Emph{VVhom God loueth he chaſtiſeth.} Let no man therfore with Paganes
and Manichees preſume to reprehend or blame Gods iuſtice, but certainly
beleue, that not Gods violence made Pharao indurate, but his owne
wickednes, and his vntamed pride againſt Gods precepts.
\MNote{In abſence of grace ſinne obdurateth.}
Againe, what els
is it to ſay, \Emph{I vvil indurate his hart}, but when my grace is
abſent from him, his owne wickednes wil obdurate him?

To know this by examples: water is congeled with vehement cold, but the
heate of the ſunne coming vpon it, is reſolued, and the ſunne departing,
it freezeth againe.
\MNote{Gods grace in the obſtinate, like the heate of the ſunne in cold
water.}
In like maner by the laſines of ſinners, charitie waxeth cold, & they
are hardned as yſe: but when the heate of Gods mercie commeth vpon them,
they are againe ſoftned. So Pharao without pittie or compaſſion
afflicting the Hebrewes, became as hard as yſe, but Gods hand touching
him with afflictions, he made humble ſupplication, that Moyſes and Aaron
would pray to God for him, promiſing what they demanded: againe, when
the plagues were remoued, he was more indurate againſt God and his
people, then before. VVherby we ſee, Gods gentlenes, indulgence, and
ſparing of Pharao, not his rigour, nor his wil or ſet purpoſe, but his
permiſſion, and Pharaoes owne wilful malice hardned his hart, and
brought him to obſtinate contempt of Gods cõmandments.
\MNote{As a father for not puniſhing is ſaied to ſpoile, ſo God to
indurate.}
And therfore God did only indurate him, in that cõmon phraſe of ſpeaking,
as a father, or a maiſter hauing brought vp his child or ſeruant
delicatly, and not ſufficiently puniſhed his frequent faultes, wherby he
becometh worſe and worſe, deſperate and obdurate, at laſt the father or
maiſter ſaieth: I haue made thee thus bad as thou art. I by ſparing thee
and ſuffering thee to haue thine owne pleaſure, haue nouriſhed thy
peruerſnes, and careleſnes: yet he ſaieth not this, as though by his wil
and intention, but by his goodnes and gentlenes the man became ſo
wicked.
\MNote{Al the wicked may iuſtly be damned: but ſome are iuſtified and
ſaued.}
It may here be demanded againe: why did not our Lord ſo mercifully
puniſh Pharao, as wholy to reclame him, for it ſemeth that had benne
greateſt mercie? and God dealeth ſo with ſome, why doth he not with al,
that al might be ſaued? Firſt it is moſt iuſtly and rightly aſcribed to
their iniquitie, which deſerue to be indurate: againe why this ſinner is
reclaimed, and not an other of the ſame il deſerts, is to be referred to
Gods inſcrutable iudgements, which are often ſecrete, neuer vniuſt. Let
it therfore ſuffice piouſly and humbly to beleue, that as Moyſes
teſtifieth:
\MNote{God neuer willeth but only ſuffereth ſĩne.}
\CNote{\XRef{Deut.~32.}}
\Emph{God is faithful and vvithout anie iniquitie, iuſt and right}: and
as the royal Prophet alſo profeſſeth,
\CNote{\XRef{Pſal.~5.}}
\Emph{Thou art not a God that vvilt iniquitie}, and as the Apoſtle
teacheth,
\CNote{\XRef{Rom.~9.}}
\Emph{there is no iniquitie vvith God}. By al which and ſome more to the
ſame effect (which we omit) S.~Auguſtin concludeth againe, that properly
\MNote{Pharao abuſing Gods benefites hardned his owne hart. And wilfully
periſhed.}
Pharao hardened his owne hart, God only by beſtowing benefites vpon him,
which he abuſed, and not plaguing him ſo much, as he deſerued, but
letting him liue, and reigne, and perſecute the Church for the time,
vntil he and al his armie were in the middes of the ſea. VVhither (as
the ſame lerned father noteth
\Cite{ſer.~89.)}
their owne deſperate boldnes drew them, vaine furie through their owne
madnes prouoking them to goe ſo farre, where God not working, but only
ceaſing to continew his miracle, the waters returning to their owne
nature, and meeting together inuolued and drowned them al.

Other
\MNote{Other places of S.~Auguſtin.}
like expoſitions the ſame lerned father hath in other places. As,
\Cite{q.~18. ſuper Exodum,}
he teacheth that Pharao being already ſo wicked through his owne fault,
other things were done to him and his people, which partly were to the
correction of others, and might haue bene to his, but he abuſing al,
became worſe & worſe, by Gods ſuffering and diſpenſation, \Emph{not only
for his iuſt, but euidently iuſt puniſhment}.
\Cite{Li.~5. cont. Iulian c.~3.}
touching the ground of tentation he alleageth the Apoſtle ſaying:
\CNote{\XRef{Iaco.~1.}}
\Emph{Euerie one is tempted of his owne concupiſcence, abſtracted and
allured}: but touching one kind of Gods puniſhing ſome, that are
ouerwhelmed in obſtinate ſinnes, he alleageth the ſaying of an other
\Fix{Aopſtle,}{Apoſtle,}{obvious typo, fixed in other}
\MNote{Gods iuſtice made euident when ſinnes are more notorious.}
\CNote{\XRef{Rom.~5.}}
\Emph{God hath deliuered them into paſsions of ignomie; and into a
reprobate ſenſe, to do al thoſe things that are not conuenient, for God
deliuereth them} (ſaith he) \Emph{conueniently}: that the ſame ſinnes
are made both puniſhments of ſinnes paſt, and are deſerts of puniſhments
to come. Yet he maketh not the willes euil, but vſeth the euil as he
wil, who can not wil anie thing vniuſtly. Againe,
\Cite{q.~24.}
It appeareth (ſaieth he) that the cauſes of induration of Pharaoes hart,
were not only for that his Inchanters did like things (to thoſe which
Moyſes and Aaron did) but the very patience of God, by which he ſpared
him.
\MNote{Gods patience of it ſelf profitable, by euil harts made
vnprofitable.}
Gods patience according to mens hartes is profitable to ſome to
repentance, to ſome vnprofitable to reſiſt God, & perſiſt in euil: yet
not of it ſelfe vnprofitable, but through the euil hart.

Briefly,
\Cite{q.~36.}
\Emph{I haue
\Fix{harned}{hardned}{obvious typo, fixed in other}
Pharaoes hart}, that is, I haue bene patient ouer him and his ſeruants.
\Cite{Epiſt.~105.}
\MNote{Not doing called ſometimes doing the contrarie.}
God doth not indurate by imperting malice, but by not imperting mercie
(or grace).
\Cite{Li. de Prædeſt. & Grat. c.~4.}
God is ſayed to indurate him, whom he wil not mollifie. So, to make him
blinde whom he wil not illuminate. So alſo to repel him, whom he wil not
cal. And
\Cite{c.~6.}
What is that to ſay: \Emph{I vvil indurate his hart}, but I wil not
mollifie it?
\Cite{cap.~14.}
It ought to haue auailed Pharao to ſaluation, that Gods patience
deferring his iuſt and deſerued puniſhment,
\TNote{\L{Miraculorum verbera crebra denſabat.}}
multiplied vpon him frequent ſtripes of miracles, or \Emph{miraculous
puniſhmẽts}.
\Cite{Cap.~15.}
\MNote{Freewil the cauſe of diuers endes in Pharao and Nabucodonoſor.}
Did not Nabucodonoſor repent being puniſhed after innumerable impieties,
and recouered the kingdome which he had loſt? But Pharao by puniſhment
became more obdurate, and periſhed. Both were men, both Kings, both
perſecutors of Gods people, both gently admoniſhed by puniſhments. VVhat
then made their endes diuers, but that the one feeling Gods hand mourned
in remembrance of his owne iniquitie, the other by his freewil fought
againſt Gods moſt merciful veritie?

Neither
\MNote{Other ancient Doctors teach the ſame.}
is this the doctrin of S.~Auguſtin alone, but of other Doctors alſo.
\MNote{Origen.}
Origen
\Cite{(li.~3. Periarch. c. de Libert. arbitrij.)}
ſaieth: the Scripture ſheweth manifeſtly, that Pharao was indurate by
his owne wil. For ſo God ſaied to him:
\CNote{\XRef{Exo.~4.~8.}}
\Emph{Thou vvouldeſt not: If thou vvilt not diſmiſſe Iſrael.}

S.~Baſil
\MNote{S.~Baſil.}
\Cite{(Orat. quod Deus non ſit auctor malorum)}
ſaieth, God beginning with leſſe ſcourges, proceeded with greater and
greater to plague Pharao, but did not mollifie him being obſtinate,
neither yet did puniſh him with death, vntil he drowned himſelfe, when
he preſumed through pride, to paſſe the ſame way, by which the iuſt
went, ſuppoſing the redde ſea would be paſſable to him, as it was to the
people of God.
\MNote{Chryſoſtom.}
S.~Chryſoſtom
\Cite{(ho.~67. in Ioan.)}
God is ſaied in holie Scripture to haue indurate ſome, and deliuered ſome
into reprobate ſenſe, not for that theſe things are done by God (coming
in dede of mans owne proper malice) but becauſe God iuſtly leauing men,
theſe things happen to them. And
\Cite{(in cap.~1. Rom.)}
\Emph{He deliuered} (into reprobate ſenſe) is nothing els, but \Emph{he
permitted}.
\MNote{Damaſcen.}
S.~Damaſcen
\Cite{(li.~4. ca.~20. de fide orthodoxa)}
It is the maner of holie Scripture to cal the permiſſion of God his
act. As,
\CNote{\XRef{Iſa.~6.}}
\Emph{He hath geuen them the ſpirite of compunction};
\CNote{\XRef{Rom.~11. v.~8.}}
\Emph{eyes, that they may not ſee: and eares that they may not heare},
and the like; al which are to be vnderſtood not as proceding of Gods
action, but as of Gods permiſſion, to wit, for mans free powre of
working.
\MNote{Hierom.}
S.~Hierom
\Cite{(Epiſt.~150. reſp. ad q.~10.)}
Not Gods patience is to be accuſed, but their hardnes who abuſe Gods
goodnes to their owne perdition.
\MNote{Theodoret.}
Theodoret
\Cite{(q.~17. in Exod.)}
It is to be noted, that if Pharao had bene euil by nature, he had neuer
changed his minde. And (after diuers mutations recited, how
ſometimes he would diſmiſſe Iſrael, other times he would not) al theſe
(ſaieth he) Moyſes recorded to teach vs, that neither Pharao was of
peruerſe nature, neither did our Lord God make his mind hard and
rebellious. For he that now inclineth to this part, now to that, plainly
ſheweth freewil of the mind.

S.~Gregorie
\MNote{Gregorie the great.}
\Cite{(li.~11. ca.~8. Moral.)}
God is ſaied to indurate by his iuſtice, when he doth not mollifie a
reprobate hart. And
\Cite{(li.~31. c.~11.)}
Our Lord is ſaied to haue indurated Pharaoes hart, not that he brought
the hardnes itſelfe, but for that his deſertes ſo requiring, he did not
mollifie it, with ſenſibilitie of feare infuſed from aboue.
\MNote{Iſidorus.}
S.~Iſidorus
\Cite{(li.~2. ca.~19. de ſummo bono.)}
Sinne is permitted for puniſhment of ſinne, when a ſinner, for his
deſert forſaken of God, goeth into an other worſe ſinne.

Finally conference of holie Scriptures, as in other hard places, ſo in
this, geueth light for better vnderſtanding therof.
\MNote{The act of induration attributed to Pharao himſelf in diuers
places.}
For diuers places
do not only ſhew that in al theſe reſiſtances, mutations of mind, and
obſtinacie of hart, Pharao was neuer depriued of freewil, as the Doctors
before cited do note, but alſo expreſſly attribute the act of induration
to himſelf.
\XRef{Cha.~8. v.~15.}
\Emph{Pharao ſeeing that reſt vvas geuen he hardned his ovvne hart.}
\XRef{v.~32.}
Where the latin readeth in the paſſiue voice, \L{ingrauatum eſt cor
Pharaonis}, \Emph{Pharaos hart vvas hardned}, which is more obſcure,
the Hebrew ſaieth actiuely, 
\CNote{\Cite{Bible 1552.}
\Cite{1577.}
\Cite{1603.}}
& the proteſtantes ſo tranſlate, \Emph{Pharao hardned his hart this time
alſo.} Likewiſe
\XRef{cha.~9. v.~7.}
the Hebrew ſaieth, \Emph{Pharaoes hart hardned it ſelfe.} Alſo
\XRef{v.~35.}
\Emph{He hardned his ovvne hart, he and his ſeruants.}
\XRef{Cha.~13. v.~15.}
\Emph{VVhen Pharao had indurated himſelfe.} And
\XRef{1.~Reg.~6. v.~6.}
\Emph{VVhy do you harden your hartes, as Ægypt and Pharao hardned their
hart?} Al which are reconciled with the other textes, that ſay \Emph{God
indurated Pharaoes hart}, vnderſtanding that phraſe in like ſenſe to
this.
\XRef{(cha.~15. v.~4.)}
\MNote{How it is ſaid, God caſt Pharao into the ſea, when himſelfe ranne
in wilfully?}
\Emph{God hath caſt Pharao his chariotes, and his armie into the ſea.}
VVhere God only permitted, and in no way forced Pharao and his armie, to
follow the Hebrewes betwen the walles of water. As before is here noted
out of S.~Baſil, and
\CNote{\Cite{Ser.~89.}}
S.~Auguſtin, and the text it ſelfe maketh it euident.
\MNote{Not God but man the cauſe of ſĩne: proued by other ſcriptures.}
Againe manie other places confirme, that not God, but the ſinners owne
wilfulnes, is the proper cauſe of his ſinne.
\XRef{Iob.~24. v.~23.}
God hath geuen him place for penance, and he abuſeth it vnto pride.
\XRef{Eccle.~8. v.~11.}
Becauſe ſentence is not quickly pronounced againſt the euil, the
children of men cõmit euils without al feare.
\XRef{Oſee.~13. v.~9.}
Perdition is thine, O Iſrael, only in me thy helpe.
\XRef{Rom.~2. v.~4.}
The benignitie of God bringeth thee to penance: but according to thy
hardnes, and impenitent hart, thou heapeſt to thy ſelfe wrath.
\XRef{Epheſ.~4. v.~19.}
Gentiles haue geuen vp themſelues to impudicitie (\Emph{or vvantonnes.})
And manie like places ſhew, that God is not the mouer, author, nor forcer
of anie thing, as it is ſinne: but man himſelfe is the author by
wilfully conſenting to tentations of the diuel, the fleſh, and the
world, and by abuſing Gods benefites, and reſiſting his grace.}
I wil indurate his hart, and wil multiplie my ſignes and wonders in the
Land of Ægypt, \V and he wil not heare you: and I wil put in my hand
vpon Ægypt, and wil bring forth my armie and people the children of
Iſrael out of the Land of Ægypt, by very great iudgements. \V And the
Ægyptians ſhal know that I am the Lord, which haue ſtretched forth my
hand
%%% o-0173
vpon Ægypt, and haue brought forth the children of Iſrael out of the
middes of them. \V Therfore Moyſes and Aaron did as our Lord had commanded:
ſo did they. \V And Moyſes was eightie yeares old, and Aaron eightie
three, when they ſpake to Pharao. \V And our Lord ſaid to Moyſes and
Aaron: \V When Pharao ſhal ſay vnto you, Shew ſignes: thou ſhalt ſay to
Aaron: Take thy rodde, and caſt it before Pharao, and it ſhal be turned
into a ſerpent. \V Therfore Moyſes and Aaron going in vnto Pharao, did
as our Lord had commanded. And Aaron tooke the rodde before Pharao and
his ſeruantes, the which was turned into a ſerpent. \V And Pharao called
\SNote{Iannes and Mambres
\XRef{2.~Tim.~3.}
knowen by tradition.}
the wiſe men and the enchanters: and
\LNote{They alſo.}{True
\MNote{True miracles do certainly proue the truth.}
\CNote{\XRef{Mar.~16. v.~20.}
\XRef{Heb.~2. v.~4.}}
miracles, being aboue the courſe of al created nature, can not be
wrought but by the powre of God; who is truth it ſelfe, and can not geue
teſtimonie to vntruth, and therfore they certainly proue that to be
true, for which they are done.
\MNote{Some ſtrange things done by ſleight, by deceipt of ſenſes, & by
courſe of nature, eſpecially by diuels.}
\CNote{\Cite{S.~Aug. li.~18. c.~18. ciuit.}}
Other ſtrange things done by enchanters,
falſe prophetes, and diuels, are not in deede true miracles, but either
ſleights, by quicknes and nimblenes of hand, called legier-demain,
conueing one thing away and bringing an other; or falſe preſentations
deceiuing the ſenſes, and imaginations of men, by making things ſeme to
be that they are not; or els are wrought by applying natural cauſes
knowen to ſome, eſpecially to diuels; who alſo by their natural force
can do great thinges, when God permitteth them. And ſo \Emph{by
enchantments and certaine ſecrecies}, theſe ſorcerers either conueyed
away the roddes, and water, and brought dragons, and bloud, in their
place, & more frogges, from other places; or els by the diuels vſing
natural agents turned roddes into ſerpentes, water into bloud, & other
matter into frogges: al which might be done naturally in longer time, &
by the diuel in ſhort time.
\MNote{Manie things aboue the diuels natural powre.}
But manie thinges are wholy aboue the diuels powre: as to deſtroy the
world, to change the general order therof: to create of nothing: to
raiſe the dead to life: to geue ſight to the borne blind: & the like,
which are only in Gods powre.
\MNote{The diuels powre is much reſtrained.}
In things alſo diuels naturally can do, they are much reſtrayned by Gods
goodnes, leſt they ſhould deceiue, or hurt mankind at their pleaſure. So
theſe Enchanters fayled in the fourth attempt, not able to make more
ſciniphes, nor anie more ſuch prodigies: and were only permitted to
produce ſuch ſerpents, as were deuoured by Aarons ſerpent: and to change
water into bloud: and to increaſe the number of frogges, for the greater
plague, and no profite of the Ægyptians. Neither could they remoue anie
plague. Nay themſelues were ſo plagued with boyles, that for paine, or
for ſhame, they could not ſtand before Moyſes.

It
\MNote{Falſe prophets euer faile, when they pretend by miracles to proue
their doctrine.}
is further to be obſerued, that whenſoeuer anie haue attempted to worke
miracles to proue falſe doctrin, they haue failed, and by Gods
prouidence bene confounded.
\CNote{\XRef{3.~Reg.~19.}}
As when Baals falſe prophetes, crying to their falſe goddes from morning
til noone, could not bring fire for their ſacrifice:
\CNote{Iob.~1.}
and yet the diuel brought fire to burne Iobs ſhepe and ſeruants: God
permitting the one, and not the other. God alſo for a time ſuffered
Simon Magus to make ſhew of miracles, and at laſt (as Egeſippus
\Cite{li.~3. de excid. Hieroſol. c.~2.}
and manie others teſtifie) to flie in to the ayer, as though he would haue
aſcended into heauen,
\MNote{Simon Magus confounded.}
but S.~Peter praying to God, the magician, notwithſtanding his wings
wherwith he preſumed to flie, fel downe and broke his legges, that he
could not goe. To omitte manie examples, Gregorius Turonenſis
\Cite{li.~2. hiſt. Franc. c.~3.}
witneſſeth, that one
\MNote{Cyrola an Arian Biſhop detected.}
Cyrola an Arian Patriarch, pretending to obtaine of God ſight to a man,
that feaned him ſelfe blind, the man was preſently blind in deede, and
exclaming cryed: Take here thy money which thou gaueſt me, to deceiue
the world, reſtore me my ſight, which I had euen now, and by thy
perſwaſion, and for this money, I feaned to want.
\MNote{Caluins attempt miſproued and he defamed.}
It happened worſe to one Bruley a poore man in Geneua, whom Caluin with
wordes and money perſwaded to feane him ſelfe dead, and ſo pretending to
raiſe him to life, the man was found dead in dede, and not he but his
wife (hauing conſented to the deuiſe) lamented in earneſt,
\Fix{enuehing}{enueihing}{possible typo, fixed in other}
againſt that falſe Apoſtle, calling him a ſecrete thefe, and a wicked
murderer, that had killed her huſband. So writeth M.~Ierom Bolſeck
\Cite{in vita Caluini.}
And beſides the womans vnexpected outcrie, and aſſeueration, that her
huſband was not dead before, but that, through Caluins perſwaſions, and
promiſes to releue them with almes, they ſo feaned, al Geneua did knowe,
that Caluin endeuoured to raiſe the man, and could not. Theſe and manie
others haue attempted and could do nothing, but againſt them ſelues.

Al 
\MNote{Gods prouidence in moſt danger.}
the danger is when in dede wonders are done that may ſeme to be
miracles. Againſt ſuch therfore Gods prouidence more particularly
aſſiſteth his ſeruantes diuers wayes.
\MNote{1.~His ſpecial warning not to credit preachers of a new Religion,
though they pretend to be prophetes, or to worke wonders.}
Firſt he warneth al to ſtand faſt when ſuch tentations happen.
\XRef{Deut.~13.}
If there riſe among you a prophet, or one that ſaieth, he hath ſene a
dreame, and fortelleth a ſigne, and a wonder, and it cometh to paſſe
which he ſpake, and he ſay to thee: Let vs goe & folow ſtrange goddes,
whom thou knoweſt not, and let vs ſerue them, thou ſhalt not heare the
wordes of that prophet, or dreamer. In like maner our Sauiour
foretelleth that falſe chriſtes, & falſe-prophetes, ſhal by great ſignes
& wonders ſeduce many, warneth al ſaying:
\CNote{\XRef{Mat.~24.}}
Loe I haue fortold you. If therfore they ſhal ſay vnto you: He is in the
deſert, goe not out. Behold in the cloſets, beleue it not.
\MNote{2.~Moſt dangerous ſeducers reigne but ſhort time.}
Secondly God ſuffered not the Enchanters of Ægypt, nor Simon Magus long:
and for the elect, the dayes of Antichriſts dangerous perſecution ſhal
be ſhortned.
\MNote{3.~Notes to know Antichriſt.}
Thirdly holy Scripture ſo deſcribeth Antichriſt, and his actes, as when
he cometh he may be ſooner knowne. Our Sauiour ſaieth:
\CNote{\XRef{Ioan.~5.}}
The Iewes wil receiue him. S.~Paul calleth him
\CNote{\XRef{2.~Theſ.~2.}}
\Emph{the man of ſinne}, importing one ſingular man, and the ſame
replete with al wickednes,
\CNote{\XRef{Apoc.~13.}}
\Emph{extolled aboue al that is called God, or is vvorshipped}. Neither
worſhipping true God, nor other falſe God aboue him ſelfe. He ſhal be
deadly wonded and cured. Not only he ſhal ſhew ſtrange wonders, but alſo
one of his prophetes ſhal bring fire from the firmament, & his image
ſhal ſpeake.
\MNote{4.~Againſt moſt dangerous aſſaltes God ſendeth moſt forcible
reſiſtãce.}
\Fix{Fourtly}{Fourthly}{likely typo, fixed in other}
as our Lord gaue powre and authoritie to his great Prophet Moyſes,
againſt the Ægyptian Enchanters, in the end of the law of nature, before
the written law: and to his firſt chief vicar S.~Peter, in the beginning
of the law of grace, to control & confound Simon Magus:
\CNote{\XRef{Apoc.~11.}}
ſo he wil ſend
\MNote{See the
\XRef{annotations for c.~5. v.~24.}}
his two reſerued great Prophetes Enoch and Elias nere the end of the
world, to reſiſt Antichriſt, and to teach, teſtifie, and confirme with
their bloud the doctrin of Chriſt. For they ſhal be ſlaine, and riſe
againe after three dayes, and aſcend into heauen.
\CNote{\XRef{Apoc.~20.}}
Then Antichriſt holding him ſelfe moſt ſecure, ſhal ſudainly be
deſtroyed.
\XRef{2.~Theſ.~2.}}
they alſo by Ægyptian enchantments and certaine ſecrecies did in like
maner. \V And euery one did caſt forth their roddes, the which were
turned into dragons: but Aarons rodde deuoured their roddes. \V And
Pharaos hart was indurate, and he heard them not, as our Lord had
commanded. \V And our Lord ſaid to Moyſes: Pharaoes hart is aggrauated,
he wil not diſmiſſe the people. \V Goe to him in the morning, behold he
wil goe forth to the waters: and thou ſhalt ſtand to meete him vpon the
banke of the riuer: and the rodde that was
%%% 0190
turned into a dragon, thou ſhalt take in thy hand. \V And thou ſhalt ſay
to him: The Lord God of the Hebrewes ſent me to thee, ſaying: Diſmiſſe
my people to ſacrifice vnto me in the deſert: and vntil this preſent
\SNote{Induration of hart (ſaith S.~Bernard) is neither cut with
remorſe, nor ſoftened with pittie, nor moued with prayers, nor yeldeth to
threates: yea is more hardned by puniſhments.
\Cite{li.~1. de conſid. ad Eugen.}}
thou wouldeſt not heare. \V This therfore ſaith our Lord: In this thou
ſhalt know that I am the Lord: behold I wil ſtrike with the rodde, that
is in my hand, the water of the riuer, and it ſhal be turned into
bloud. \V The fiſhes alſo, that are in the riuer, ſhal dye, and the
waters ſhal putrifie, and the Ægyptians ſhal be afflicted drinking the
water of the riuer. \V Our Lord alſo ſaid to Moyſes: Say vnto Aaron,
Take thy rodde, and ſtretch forth thy hand vpon the waters of Ægypt, and
vpon their floudes, and riuers and pooles, and al the lakes of waters,
that they may be turned into bloud: and be there bloud in al the Land of
Ægypt, as wel in the veſſels of wood as of ſtone. \V And Moyſes and
Aaron did as our Lord had commanded: and lifting vp the rodde he ſtroke
the water of the riuer before Pharao and his ſeruantes:
\MNote{The firſt plague in water, in which the Ægyptiãs drowned the
Hebrewes infants.
\Cite{Theodoret. q.~19. in exod.}
the like
\Cite{Ap.~16.}
Becauſe the wicked ſpil the bloud of Gods Saintes, he wil geue them
bloud to drinke.}
which was turned into bloud. \V And the fiſhes, that were in the riuer,
died: and the riuer putrified, and the Ægyptians could not drincke the
water of the riuer, and there was bloud in the whole Land of Ægypt. \V
And the enchaunters of the Ægyptians with their enchantments did in like
maner: and Pharaoes hart was indurate, neither did he heare them, as our
Lord had commaunded. \V And he turned away him ſelfe, and went into his
houſe, neither did he yet ſet his hart to it this time alſo. \V And al
the Ægyptians digged round about the riuer for water to drinke: for they
could not drinke of the water of the riuer. \V And ſeuen dayes were
fully ended, after that our Lord ſtroke the riuer.


\stopChapter


\stopcomponent


%%% Local Variables:
%%% mode: TeX
%%% eval: (long-s-mode)
%%% eval: (set-input-method "TeX")
%%% fill-column: 72
%%% eval: (auto-fill-mode)
%%% coding: utf-8-unix
%%% End:
