%%%%%%%%%%%%%%%%%%%%%%%%%%%%%%%%%%%%%%%%%%%%%%%%%%%%%%%%%%%%%%%%%
%%%%
%%%% The (original) Douay Rheims Bible 
%%%%
%%%% Old Testament
%%%% Exodus
%%%% Third Age
%%%%
%%%%%%%%%%%%%%%%%%%%%%%%%%%%%%%%%%%%%%%%%%%%%%%%%%%%%%%%%%%%%%%%%




\startcomponent third-age


\project douay-rheims


%%% 0216
%%% o-0197
\startArgument[
  title={\Sc{The continvance of the chvrch and religion in the third
  age}, from Abrahams going forth of Chaldea, to the parting of Iſrael
  out of Ægypt. The ſpace of 430.~yeares.},
  marking={The Third Age}
  ]

One
\MNote{The ſame Church & Religion in this age as in the former.}
and the ſame Church and Religion begunne in the firſt age of the world, and
continued in the ſecond, became more and more conſpicuous in the
third. For in this age not only the ſame principal and particular
pointes of faith, were beleued and profeſſed, but alſo the number of
profeſsors encreaſed and partly by ſeperation of place and abode, and
ſpecially by diuerſitie of maners, outward rites, and conuerſation were
more diſtinct from infidels then before: as vve shal now shew by the
ſacred hiſtorie of that time. VVhich beginneth with Abrahams going forth
of his countrey of Chaldea, about 2024.~yeares from the beginning of the
vvorld, in the 75.~yeare of his age.

From
\MNote{Beleefe in one God.}
which time forward God often appeared to him, and after him to Iſaac and
Iacob, in the title of \HH{\Sc{El Saddai}}, that is, \Emph{God Almightie:
Creator of al things, Lord, God, moſt high, Poſſeſſor of heauen and
earth.}
\XRef{(Gen.~14.)}
To Moyſes more familiarly
\XRef{(Exod.~3.)}
in his moſt proper name, \Sc{He which is}. In the name of \Emph{foure
letters}, which the Iewes count \Emph{ineffable}. And in diuers other
names, al shewing \Emph{One, Eternal, Omnipotent, infinite Maieſtie}, of
whom al other things depend, and haue their being, himſelfe independent
of any other thing.

This
\MNote{Three diuine Perſons.}
one diuine nature, and indiuiſible ſubſtance is (aboue al reach of
reaſon) \Emph{three in Perſons}: repreſented to Abraham
\XRef{(Gen.~18.)}
by three Angels, in forme of men, vvhom, by ſpecial inſtinct of God,
he \Emph{adored as one}: and firſt ſpake vnto them as to
one: \Emph{Lord} if I haue found grace in \Emph{thy ſight}, goe not
paſt \Emph{thy} ſeruant; and by and by as to manie: VVash \Emph{yee
your} feete. In like maner Moyſes ſometimes ſpeaketh plurally as of
manie, \Emph{There appeared to him three men, they ſaied:} VVhere is
Sara? ſometimes ſingularly; \Emph{He ſaid: I wil come.} So Lot
\XRef{(Gen.~19.)}
ſpake to \Emph{two Angels} repreſenting \Emph{the Sonne of God},
and \Emph{the Holie Ghoſt}, one God with \Emph{the Father}, firſt as to
manie, \Emph{I beſech you my Lordes}, turne into the houſe
of \Emph{your} ſeruant; after as to one: I beſech \Emph{thee my Lord},
becauſe \Emph{thy} ſeruant hath found grace before \Emph{thee}. VVho
likewiſe anſwered as one only: \Emph{I haue heard} thy prayer. Againe
Moyſes sheweth diſtinction of Perſons in God, ſaying
\XRef{(v.~24.)}
\Emph{Our Lord rained from our Lord.} Iob alſo (who liued in this age)
and his frendes profeſſed and ſerued the ſame one God, auouching him to
be the  onlie God and Lord, that \Emph{geueth and taketh away.}
%%% o-0198
%%% 1:21?
\XRef{(Chap.~1.~2.)}
He \Emph{the maker} and peculiar \Emph{Keeper of men. He that taketh
away ſinne, and iniquitie.}
\XRef{(c.~7.)}
\Emph{He that doth great things, incomprehenſible, and meruelous,
%%% 0217
wherof there is not number.}
\XRef{(c.~9.)}
And that with termes appropriated to the three diuine Perſons.
\XRef{(c.~26.)}
\MNote{\Emph{Strength} (or \Emph{power}) the Father,
\Emph{vviſdome} the Sonne, \Emph{Spirite} the Holie Ghoſt.}
\Emph{In his ſtrength ſodainly the ſeas are gathered togeather, and with
his wiſedom he ſtroke the proud man. His Spirite hath adorned the
heauens.} The ſame Myſterie of pluralitie of Perſons in one God is more
clere by the Hebrevv text 
\XRef{Chap.~30. v.~11.}
and
\XRef{35. v.~10.}
where the ſame actions are aſcribed to God, as to one, and as to manie.

But
\MNote{Chriſt promiſed to Abraham.}
moſt euident are the promiſes, figures, and prophecies of Chriſt our
Redemer. For beſides preſent abundance of riches, promiſe of great
progenie, and that the ſame should poſseſſe the fruitful Land of Chanaan
(three ſpecial bleſsings of the old Teſtament) God promiſed Abraham a
farre greater thing
\XRef{(Gen.~12.)}
that in his ſeede \Sc{al nations  and kindreds of the earth shovld be
blessed.} In confirmation whereof, God alſo changed his name \Emph{Abram
(high or noble father)} into \Emph{Abraham (Father of manie nations.)}
\XRef{(Gen.~17.)}
And ſo he was natural father of foure great Kingdomes, \Emph{Iſmaelites,
Madianites, Idumeans}, and \Emph{Iſraelites}:
\CNote{\XRef{Rom.~9.}}
but ſpiritual father of manie more, to wit, of al that beleue in Chriſt,
Iewes and Gentiles, from that time to the worlds end.
\MNote{To Iſaac.}
The ſame promiſes of poſseſsing Chanaan and \Emph{of Chriſt} vvere
renevved and confirmed to Iſaac.
\XRef{(Gen.~26.)}
\MNote{And to Iacob.}
In like ſorte to Iacob.
\XRef{(28.)}
For they pertained not to Iſmael, nor to the other ſonnes of Abraham,
nor to Eſau.
\MNote{Chriſt prefigured by Abraham.}
Moreouer Chriſt, our Redemer and deliuerer from ſinne, and captiuitie of
the diuel, was prefigured by Abraham, at laſt deliuering thoſe from
captiuitie, who otherwiſe endeuoring to shake of the yoke of
Cordorlahomor, fel further into ſubiection and bondage.
\XRef{(Gen.~14.)}
\MNote{By Melchiſedech.}
Alſo Melchiſedech, King and Prieſt, of vnknowen generation,
extraordinarie vocation, without predeceſsor, or ſucceſsor, prefigured
Chriſt King and Prieſt for euer, who not by ſucceſsors, but by Prieſts
his vicars, perpetually exerciſeth al Prieſtly functions.
\MNote{By Iſaac.}
Likewiſe \Emph{Iſaac} borne aboue the common courſe of nature
\XRef{(Gen.~21.)}
ſingularly beloued of his father, carying wood on his back for the
ſacrificing of himſelfe.
\XRef{(22.)}
\MNote{Iacob.}
\Emph{Iacob} flying his brother Eſau
\XRef{(27.)}
hardly treated by Laban,
\XRef{(31.)}
yet alwayes inuincible againſt his aduerſaries.
\XRef{(32.)}
\MNote{Ioſeph.}
\Emph{Ioſeph} hated of his brethren, ſold and deliuered to Gentiles,
\XRef{(37.)}
by them alſo perſecuted,
\XRef{(39.)}
but afterwardes aduanced, and called \Emph{the Sauiour of the world}.
\XRef{(41.)}
\MNote{Iob.}
Iuſt \Emph{Iob} vehemently afflicted;
\MNote{Moyſes.}
\Emph{Moyſes} hidden for a while, then expoſed to danger, and thence
deliuered: afterwards manifeſting him ſelfe to his brethren, by them
reiected, bewrayed, and flying from Pharao,
\XRef{(Exo.~2.)}
returning againe
\XRef{(Exod.~3.}
\XRef{4.}
\XRef{&c.)}
and at laſt deliuering the Iſraelites from bondage of Ægypt.
\XRef{(Exo.~14.)}
\MNote{And manie other things.}
And manie other things, as the ramme ſacrificed in place of Iſaac
\XRef{(Gen.~22.)}
the ladder of Iacob
\XRef{(Gen.~28.)}
Ioſephs ſcepter
\XRef{(47.)}
Aarons rodde
\XRef{(Exo.~7.)}
Paſchal lambe
\XRef{(12.)}
prefigured Chriſt, borne of a Virgin; the onlie Sonne of God; ſometimes
hidden, other
%%% 0218
times conuerſant with men, hated, perſecuted, ſold, betraied; vvho
caried his ovvne croſſe, was ſacrificed, vanquished al his enimies,
aduanced, and
%%% o-0199
acknowledged the true \Emph{Sauiour of the world}, Redemer
and deliuerer of mankind, from ſeruitude, ſlauerie, thraldome, and
bondage of ſinne, death, and the diuel.
\MNote{Prophecie of Chriſt.}
Againe Abraham prophecied that of his ſeede Chriſt our Sauiour should be
borne, when he ſaied to his ſeruant
\XRef{(Gen.~24.)}
\Emph{Put thy hand vnder my thigh, that I may adiure thee by our Lord
God of heauen & earth}, that is, by Chriſt, who should come of his
loynes, as S.~Hierom
\Cite{(Tradit, Heb. in Gen.}
\L{et}
\Cite{explic. Pſal.~44.)}
S.~Ambroſe
\Cite{(li.~1. c.~9. de Abraham)}
and S.~Auguſtin
\Cite{(q.~62. in Gen.}
\L{et}
\Cite{li.~16. c.~33. ciuit.)}
expound it. More euidently Iacob
\XRef{(Gen.~49.)}
\Sc{The scepter shal not be taken away from Ivdas, and a dvke of his
thigh, til he do come that is to be sent, and the same shal be the
expectation of the Gentiles.} Iob as planely:
\CNote{\XRef{Iob.~19.}}
\Emph{I know that my redemer liueth.} Moyſes foreknowing that Chriſt the
true Redemer, and chiefe Lawgiuer should be ſent, praied God to haſten
his miſsion, ſaying: \Emph{I beſech thee Lord, ſend whom thou wilt
ſend.}
\XRef{(Exod.~4.)}

External
\MNote{Sacrifice.}
Sacrifice was frequent and ſolemne, as the ſoueraigne homage to God.
\MNote{Altares.}
And manie Altares erected by Abraham for that purpoſe.
\XRef{(Gen.~12.}
\XRef{13.}
\XRef{15.}
\XRef{22.)}
Vnbloudie, in bread and wine by
\Fix{Melchiſdech}{Melchiſedech}{obvious typo, fixed in other}
\XRef{(Gen.~14.)}
other liquide ſacrifices
\XRef{(Gen.~35. v.~14.)}
offered by Iacob,
\MNote{\Fix{Churces,}{Churches,}{obvious typo, fixed in other}
dedicated.}
with dedication of the place called \Emph{Bethel: the houſe of God}:
which he alſo before hand
\MNote{Vowes.}
promiſed by vow.
\XRef{(Gen.~28.)}
Diuers other Sacrifices offered by Iſaac, and Iacob.
\XRef{(Gen.~26.}
\XRef{31.}
\XRef{33.}
\XRef{36.)}
By Iob and his frends
\XRef{(Iob.~1.}
\XRef{&~42.)}
by Moyſes, Aaron, and other ancients of Iſrael.
\XRef{(Exod.~12.)}
\MNote{Prieſthood.}
Al which conſequently shew Prieſthood, whoſe proper office is to offer
Sacrifice, though amongſt al the aboue named, onlie Melchiſedech was
called a Prieſt. And among the gentiles we finde that Putiphar
\XRef{(Gen.~41.)}
and Iethro
\XRef{(Ex.~3.)}
whoſe daughters Ioſeph and Moyſes maried, were called \Emph{Prieſts}, or
as the word \HH{Cohenim} doth alſo ſignifie, \Emph{Princes}, for they
were great and eminent men in their countries.
\MNote{Priuilege of Prieſts.}
At leaſt thoſe that by ſpecial priuilege were exempted from ſelling
their landes to Pharao, and had notwithſtanding prouiſion of mantenance
in time of dearth
\XRef{(Gen.~47.)}
were properly called Prieſts, for ſuch function as they had in ſeruing
their idols.
\MNote{VVhere is no ſacrifice no Prieſt is required.}
For where vvas true and right Sacrifice, there vvere alſo right Prieſts,
and vvhere
\Fix{Idololatrical}{Idolatrical}{obvious typo, fixed in other}
ſacrifice there were like Prieſts, and vvhere no external ſacrifice at al
(as amongſt Proteſtants) there are no Prieſts, but miniſters only.

In 
\MNote{Circumciſion.}
this age alſo (long before Moyſes) the Sacrament of Circumciſion vvas
giuen to Abraham, for diſtinction of Gods ſelected and peculiar people,
and for remedy of original ſinne, in the male ſexe of Abrahams ſeede,
and others of his communitie. In the other ſexe, and other generation,
former remedies of ſacrifice, or other profeſsion of faith were
auailable.
\MNote{Penance.}
For other ſinnes, not
%%% 0219
only internal repentance was neceſsarie, which vvas euer principally
required
\CNote{\XRef{Gen.~44.}}
(& therfore Ioſeph dealt ſo ſeuerly vvith his brethren, til they had
hartie ſorow and contrition for their ſinnes) but alſo certaine external
purifications, as vvashing and changing garments, vvere ordained.
\XRef{(Gen.~35.)}
\MNote{Mariage.}
Mariage though not then a Sacrament, yet was religiouſly regarded, with
ſpecial care of faith and religion in the choiſe of perſons,
\XRef{(Gen.~24.}
\XRef{27.~v.~46.}
\XRef{c.~28. v.~1.)}
\MNote{Degrees of cõſanguinitie.}
and of
%%% o-0200
certaine degrees of conſanguinitie and affinitie. Adulterie was
punishable by death
\XRef{(Gen.~38.)}
and in no wiſe counted lawful, no not among the heathen.
\XRef{(Gen.~12.}
\XRef{20.}
\XRef{24.}
\XRef{26.}
\XRef{29.}
\XRef{34.}
\XRef{39.)}
\MNote{Pluralitie of wiues lawful ſometimes, neuer of huſbands.}
Pluralitie of vviues in ſome perſons and caſes, lawful in the lavv of
nature 
\XRef{(Gen.~16.}
\XRef{25.}
\XRef{29.)}
as alſo afterwards in the lavv of Moyſes, not in the law of grace, nor
euer pluralitie of husbands.

Spiritual
\MNote{Bleſſings.}
bleſsing, a preeminence of greater perſons, ſo Melchiſedech bleſſed
Abraham.
\XRef{(Gen.~14.)}
Iſaac bleſsed Iacob
\XRef{(c.~27.)}
and Iacob his ſonnes
\XRef{(c.~49.)}
and the ſonnes of Ioſeph, with impoſition of handes,
\MNote{Signe of the Croſſe.}
and framing the forme of a croſſe.
\XRef{(48.)}
\MNote{Ceremonies.}
Other Ceremonies of oyle and wine,
\XRef{(Gen.~28.}
\XRef{35.)}
\Emph{ſprinkling the bloud} of the Paſchal lambe, eating the
lambe \Emph{ſtanding} with their loynes \Emph{gyrded, ſhooes} on their
feete, \Emph{ſtaues} in their handes, and with \Emph{ſpeede}.
\XRef{(Exo.~12.)}
\MNote{Muſical inſtruments.}
Muſical inſtruments in Diuine ſeruice.
\XRef{(Exod.~15.)}

Chriſtes
\MNote{Baptiſme prefigured.}
Baptiſme prefigured by Circumciſion,
\XRef{(Gen.~17.)}
for Chriſtians are \Emph{circũciſed} (ſaith S.~Paul) \Emph{in the
Circumciſion of Chriſt, buried with him in Baptiſme}. Alſo by
the \Emph{cloude} vvhich ſtoode betwene the Ægyptians and
Iſraelites, \Emph{lightning the night} on the one ſide (tovvards Gods
people) \Emph{dark} on the other (tovvards their enemies) and by
the \Emph{redde ſea}, vvhich ſaued the children of Iſrael, and drovvned
the Ægyptians.
\XRef{(Exo.~14.)}
\CNote{1.~Cor.~10.}
\Emph{Al were baptiſed in the cloude, and in the ſea.}
\MNote{The B.~Sacrament.}
So the bread and vvine offered by Melchiſedech, the Paſchal lambe, and
vnleauened bread prefigured the B.~Sacrament, and Sacrifice of Chriſts
bodie and bloud, in formes of bread and wine. Iacob alſo prophecied of
this moſt excellent Myſterie.
\XRef{(Gen.~49.)}
\Emph{He ſhal waſh his ſtole in wine, and his cloke in the bloud of the
grape.}
\MNote{Prieſthood of the new Teſtament.}
In like ſorte Melchiſedechs Prieſthood was a plaine figure
of \Emph{Chriſts Prieſthood}, who firſt by himſelf conſecrated and
offered his ovvne bodie and bloud, and ſtil doth the ſame by his Prieſts
handes of the new Teſtament.

Diuers
\MNote{Traditions.}
other Rites were knovven and obſerued by \Emph{Tradition}.
\MNote{Tythes.}
So Abraham paied Tythes to his ſpiritual Superior,
\XRef{(Gen.~14.)}
\MNote{Forme of iuſtice.}
taught his children and familie \Emph{to keepe the way of our Lord}, and
doe iudgement and iuſtice.
\XRef{(Gen.~18. v.~19.)}
Iſaac and Iacob kept and taught the Ordinances, 
\MNote{Precepts.}
Precepts and Ceremonies
of their anceſters, vvithout Lavves or precepts vvritten.
\XRef{(Gen.~26.)}
\MNote{Raiſing ſeede to the brother.}
Iudas commanded his ſecond ſonne to take the widovv of his brother
deceaſed without children.
\XRef{(Gen.~38.)}
\MNote{Abſtinence.}
The children of Iſrael abſtained from eating the ſinevv of the thigh, in
remembrance that the ſinew of Iacobs thigh was shrunke.
\XRef{(Gen.~32.)}

%%% 0220
Freewil
\MNote{Freewil.}
in men proued, by that Ioſephs brethren in ſelling him \Emph{thought
euil}, not moued nor inclined therto by God, who had no part in their
euil thought, but turned it to good,
\XRef{(Gen.~50.)}
by Gods threatning Pharao
\XRef{(Exo.~8.)}
\Emph{If thou wilt not diſmiſſe Iſrael}. VVhich were vniuſt if Pharao
could not doe otherwiſe. Likewiſe by that Pharao often changed his mind,
ſometimes promiſing to diſmiſſe the Hebrewes, and againe refuſing to doe
it, which sheweth (ſaieth Theodoret) freewil of the mind: and by Gods
preuention of tentation, \Emph{leading the Iſraelites not the neereſt
way, but by the deſert, leſt perhaps it would repent them; and they
would returne into Ægypt.}
\XRef{(Exod.~13.)}
Mans conſent therfore is free notwithſtanding Gods wil, direction, and
commaundement.
\MNote{Mans induſtry neceſſarie.}
And ſo his induſtry is required
%%% o-0201
in his dailie affaires, and then to relie on Gods prouidence, otherwiſe
only to expect Gods wil, operation, or protection, man himſelf
endeuoring nothing is to tempt God. Therfore Abraham
\XRef{(Gen.~12.)}
Iſaac
\XRef{(c.~26.)}
Iacob
\XRef{(ca.~32.)}
and the parents of Moyſes
\XRef{(Exo.~2.)}
being in feare and diſtreſse vſed al prudence to auoid imminent dangers,
albeit they had ſpecial reuelations of ſafetie and happie
ſucceſse.
\MNote{God tempteth not to euil.}
Neither doth God euer tempt anie man to ſinne, but proueth his ſeruants
and maketh them knowen to the world for example of others, and their
owne merit.
\XRef{Gen.~22.}
\XRef{Iob.~1.}
\XRef{2.}
\XRef{&c.}

Onlie
\MNote{Faith and good workes together iuſtifie, and are meritorious, but
neither of them alone.}
faith doth not iuſtifie, nor workes without faith, but both together do
iuſtifie, and are meritorious: ſo Abraham \Emph{beleued God} becauſe he
is omnipotent and truth it ſelfe, \Emph{and it was reputed to him vnto
iuſtice}, 
\XRef{(Gen.~15.)}
but this faith was not ſole, for it had hope, loue, obedience, and other
vertues adioyned, and ſo his beleuing was an act of iuſtice. In like
maner
\CNote{\XRef{Iac.~2.}}
\Emph{Abraham was iuſtified by workes, offering Iſaac his ſonne vpon the
Altar},
\XRef{(Gen.~22.)}
but this worke preſuppoſed faith,
\CNote{\XRef{Heb.~11.}}
\Emph{that God is able to raiſe euen from the dead}.
\CNote{\XRef{Heb.~13.}}
\Emph{So by workes faith is conſumate.} By \Emph{hoſpitalitie} Abraham
and Lot vnawares receiued Angels to harbour.
\XRef{(Gen.~18.}
\XRef{19.)}
\MNote{Perfection in this life.}
\Emph{Abraham} was \Emph{perfect} according to perfection of this life,
\XRef{(Gen.~17.)}
\MNote{Foure principal merites of Abraham.}
moſt highly commended for foure more notorious actes proceding of two
ſpecial vertues, faith and obedience.
\MNote{1.~Prompt obedience.}
The firſt was his prompt obedience, in leauing his countrie and kindred,
going he knew not whither, nor how farre, ſimply and cherfully expecting
Gods further direction, when to goe, and where to abide.
\XRef{(Gen.~12.)}
\MNote{2.~Faith without ſtaggering.}
The ſecond was his excellent faith preſently beleeuing Gods promiſe
(which by al humane reaſon ſemed vnpoſsible) that he should haue
innumerable progenie.
\XRef{(Gen.~15.)}
\MNote{3.~Propagation of faith and religion.}
The third was, that he did not only moſt ſincerly and religiouſly ſerue
God, but alſo taught his poſteritie ſo to do, as God himſelf teſtifieth
of him, ſaying: \Emph{I know that he wil command his children, and his
houſe after him, that they kepe the way of the Lord, and do iudgement
and iuſtice.}
\XRef{(Gen.~18.)}
\MNote{4.~Perfect obedience.}
The fourth was that moſt heroical act of obedience, admirable
%%% 0221
to al ages, being readie to kil, and ſacrifice his owne moſt dearly
beloued ſonne Iſaac. \Emph{For which God ſware by himſelf}, that he
would manie waies bleſſe him, \Emph{becauſe} (ſaieth God) \Emph{thou
haſt obeyed my voice}.
\XRef{(Gen.~22.)}
\MNote{Other iuſt men.}
He prayed for Sodom, and had preuailed, if tenne iuſt perſons had benne
found in that citie.
\XRef{(Gen.~18.)}
And Lot was deliuered from thence for Abrahams ſake.
\XRef{(Gen.~19.)}
\MNote{Iſaac.}
Iſaac was alſo of moſt ſincere mind, deuout to God, exerciſed himſelf in
meditation or mental prayer,
\XRef{(Gen.~24.)}
obtained by prayer his deſire of iſsue.
\XRef{(Gen.~25.)}
\MNote{Iacob.}
Likewiſe Iacob is deſcribed in the holie text \Emph{a plaine} (or
ſincere and innocent) \Emph{man},
\XRef{(Gen.~25. v.~27.)}
patient and conſtant in tribulations.
\XRef{(Gen.~29.}
\XRef{31.}
\XRef{32.}
\XRef{33.)}
He lawfully purchaſed Eſaus conſent of the firſtbirthright.
\XRef{(Gen.~25. v.~31.)}
\MNote{He ſpake truth in myſtical ſenſe.}
He neither lied, nor otherwiſe ſinned, when he anſwered his father that
he was \Emph{Eſau his firſt begotten ſonne}
\XRef{(Gen.~27.)}
but ſpake truth in myſtical ſenſe, agreable to Gods wil and ordinance,
who ſo tranſpoſed Iſaacs bleſsing from Eſau to Iacob. VVhich Iſaac at
length vnderſtanding, conformed him ſelf therto, and confirmed the ſame
\XRef{(v.~33.}
\XRef{&~ch.~28.)}
giuing Eſau ſuch contentment as he could of temporal bleſsings.
\MNote{Ioſeph.}
Ioſeph is
%%% o-0202
renowmed for al vertues, euen from his youth to his death.
\XRef{(Gen.~37.}
\XRef{39.}
\XRef{50.)}
\MNote{Iob.}
Iob \Emph{was ſimple and right, fearing God and departing from euil, a
iuſt and innocent man}, both before and in his tribulations, \Emph{not
ſinning with his lippes: neither ſpake he anie foliſh thing againſt
God}
\XRef{(ch.~1.)}
yea more afflicted \Emph{retained innocencie}
\XRef{(ch.~2.)}
and finally God receiued his prayer for others, and reſtored al his
loſses duble.
\XRef{(ch.~42.)}
\MNote{Moyſes.}
Moyſes a moſt ſpecial ſelected Prophet,
\CNote{\XRef{Nu.~12.}}
\Emph{the meekeſt man on the earth}, of ſingular zeale ſeuerly punished
ſinne, but withal
\CNote{Exo.~32.}
moſt charitably prayed God to forgiue the people and conſerue his
Church.

God
\MNote{Election is of Gods mercie.}
of his mere mercie electeth al thoſe, whom he wil iuſtifie and ſaue,
offering al ſufficient grace, iuſtly leaueth ſome obſtinate ſinners in
ſtate of damnation.
\XRef{(Gen.~25.}
\XRef{Exo.~7.)}
\MNote{Predeſtination excludeth not ordinary meanes.}
His predeſtination, foreknowledge and promiſe, do not exclude but
include the meanes, wherby his wil is done in the iuſt.
\XRef{(Gen.~25.}
\XRef{37.}
\XRef{45.}
\XRef{50.)}
Neither is Gods reprobation the cauſe of anie mans damnation, but
\MNote{Sinne is the cauſe of reprobation.}
mans owne ſinne the proper cauſe, both of reprobation & damnation. 
\MNote{Pharao and other Ægyptians hardned their owne harts.}
For
example, Pharao & his people \Emph{enuying}, vainly \Emph{fearing} and
for their religion \Emph{hating}, and perſecuting \Emph{the children of
Iſrael}, by oppreſsing them with vnſupportable laboures, by commanding
ſecretly to kil their infants, and that not ſucceding, by a new decree
to drowne them
\XRef{(Exo.~1.)}
were mercifully after long conniuence, admonished by Gods legates in his
name quietly to permit his people to ſerue him; but they wilfully
contemned this gentle admonition, Pharao proudly and inſolently
anſwering: \Emph{Who is the Lord, that I ſhould heare his voice, and
diſmiſſe Iſrael? I know not the Lord, and Iſrael I wil not diſmiſſe.}
\XRef{(Exo.~5.)}
\MNote{Pharao and other Ægyptians hardned their owne harts.}
So
%%% 0222
they hardned their owne hartes, and more greuouſly afflicted the
faithful.
\MNote{God did only permitte them to obdurate themſelues.}
God permitting the wicked to liue, and proſper for a time in this world,
not punishing them ſo much as they deſerued, nor mollifying their
hartes, nor illuminating their vnderſtanding vnto effectual conuerſion,
but iuſtly permitting them to perſiſt in obſtinacie.
\XRef{(Ex.~7.}
\XRef{8.}
\XRef{9.}
\XRef{10.}
\XRef{&c.)}

Protection
\MNote{Protection & Inuocation of Angels and Patriarches.}
of Angels & inuocation is proued.
\XRef{(Gen.~24.}
\XRef{32.}
\XRef{48.)}
Patriarches names alſo inuocated.
\XRef{(c.~48. v.~16.)}
Iſaac was bleſsed & proſpered for Abrahams ſake,
\CNote{\Cite{S.~Aug. li.~16. c.~36. ciuit.}}
\Emph{becauſe Abraham obeyed Gods voice, kept his precepts &
cõmandements}, obſerued his ceremonies & his lawes.
\XRef{(Gen.~26.)}
\MNote{Adoration of creatures.}
Ioſephs rodde adored by Iacob.
\XRef{(Gen.~47.)}
Moyſes commanded to put of his shooes, becauſe the place was holie.
\XRef{(Exod.~3.)}
\MNote{Swearing by creatures.}
\Emph{Swearing by creatures} lawful, and ſome times more conuenient,
then immediatly by God him ſelfe.
\XRef{(Gen.~42.)}
\MNote{Ominous ſpeach.}
Likewiſe \Emph{Ominous ſpeach}.
\XRef{(Gen.~24.)}
\MNote{Dreames.}
and \Emph{Dreames}
\XRef{(Gen.~37.}
\XRef{40.}
\XRef{41.)}
are ſometimes lawfully obſerued, and are from God.
\MNote{Images.}
Idols alwaies vnlawful, but not al Images.
\XRef{(Gen.~31.}
\XRef{35.)}
\MNote{Reliques.}
Reliques to be reuently vſed, as Ioſephs bodie conſerued in a coffin in
Ægypt,
\XRef{(Gen.~vlt.)}
tranſlated by Moyſes
\XRef{(Exo.~13.)}
\CNote{\XRef{Ioſue~24.}}
and ſo brought into Chanaan, and layed with other Patriarches in
Sichem.
\MNote{Deuotion to holie places.}
Going bare foote to holie places an act of religious reuerence, and
deuotion.
\XRef{(Ex.~3.)}
\MNote{Figure of
\Fix{Chriſt}{Chriſts}{obvious typo, fixed in other.}
croſſe.}
The ſigne of the croſse vſed by Iacob,
\XRef{(Gen.~48.)}
a figure of Chriſts croſſe. The wood caſt by Moyſes into the bitter
water, and making it ſweete
\XRef{(Exo.~15.)}
an other figure therof.

\Emph{Funeral obſequies}
\MNote{Funeral offices.}
were obſerued by Abraham for his wife Sara
%%% o-0203
\XRef{(Gen.~23.)}
with \Emph{mourning} and \Emph{weeping} for her, according to the
qualitie of ſo holie a perſon, who it is like needed not other
ſatisfactorie workes as Saul and Ionathas, and others ſlaine in battel,
\CNote{\XRef{2.~Reg.~1.}}
for whom Dauid and his court did not only mourne and weepe, but alſo
faſted til euen.
\MNote{Place dedicated for burial.}
He alſo bought a field with a duble caue, where he buried her,
dedicating it for this peculiar vſe, and both himſelf, and Iſaac, Iacob,
Rebecca, and Lia were there buried.
\XRef{(Gen.~49. v.~31.)}
\MNote{Mourning 40.~dayes.}
Ioſeph with al his brethren mourned for their father Iacob, firſt
fourtie dayes in Ægypt, then carying him into Chanaan,
\MNote{Exequies of ſeuen dayes.}
\Emph{celebrated the exequies other ſeuen dayes}.
\XRef{(Gen.~50.)}
His particular \Emph{digging of his owne graue}
\XRef{(v.~5.)}
\MNote{Special place of burial rightely deſired.}
and both his and Ioſephs ſpecial charge to be buried amongſt their
anceſters,
\CNote{\XRef{Act.~7. v.~16.}}
and the tranſlation of al the twelue ſonnes of Iacob, into
Sichem, confirme the deſire of burial in one place rather then in an
other, to be agreable to nature, and holie Scriptures.

Touching
\MNote{No ſoule before Chriſt entred into heauen.}
the ſoules departed, euen the moſt perfect, went into the lower partes,
generally called \Emph{Hel}.
\MNote{Diuers places in hel.}
But ſome were in reſt, others in paines, according to their deſertes,
none in heauen before Chriſt. As S.~Hierom
\Cite{(comment in Oſee.~13.}
\L{et}
\Cite{Eccles.~3.)}
proueth by Iacobs vvordes
\XRef{(Gen.~37.)}
\Emph{I wil deſcend vnto my ſonne into hel}; by Iobs lamentation
\XRef{(Ch.~7.}
\L{et}
\XRef{17.)}
\Emph{that al} (good and bad) \Emph{were retained in hel},
ſaying: \Emph{If I ſhal expect,
%%% 0223
hel is my houſe, and in darknes I haue made my bed}. VVhich place or
receptacle of ſuch Saintes, as Iacob and Iob, vvas doubtles farre
diſtant from hel of the damned,
\CNote{\XRef{Luc.~16.}}
for betvven Lazarus in Abrahams boſome and the glutton in torments,
is \Emph{a great chaos} (or large ſpace) and yet the higheſt of theſe
places is called hel.

In
\MNote{Reſurrection.}
reſpect of \Emph{Reſurrection}, the ſame Iacob called his life in this
vvorld a \Emph{pilgrimage}
\XRef{(Gen.~47.)}
and Iob,
\XRef{(ch.~7.)}
\Emph{a warfare vpon earth}: profeſsing expreſsly
\XRef{(ch.~19.)}
\Emph{In the laſt day I ſhal riſe out of the earth. And I ſhal be
compaſſed againe with my skinne, and in my fleſh I ſhal ſee God.} Our
B.~Sauiour alſo proueth the Reſurrection, becauſe \Emph{the God of
Abraham, Iſaac, and Iacob}
\XRef{(Exo.~3.)}
\CNote{\XRef{Mat.~22.}}
\Emph{is God of them, not} as they are \Emph{dead, but} as they
are \Emph{liuing}, and to returne againe to life in bodie and ſoule
together.
\MNote{General Iudgement.}
Of general Iudgement Iob ſaieth
\XRef{(ch.~31.)}
\Emph{What ſhal I doe when God ſhal riſe to iudge? and when he ſhal
aske, what ſhal I anſwere him?} And Eliu
\XRef{(ch.~34.)}
ſaieth: \Emph{The omnipotent wil render a man his worke, and according
to the waies of euerie one, he wil recompence them.} Sodom and Gomorra
\XRef{(Gen.~19.)}
were \Emph{example}
(ſaith
\CNote{\XRef{2.~Pet.~2.}}
S.~Peter and
\CNote{\XRef{Ep.~Iud.}}
S.~Iude)
\MNote{Eternal puniſhment of the wicked: and ioy of the bleſſed.}
\Emph{of eternal puniſhment in hel fire}.

Of eternal life Iacob profeſſed his hope
\XRef{(Gen.~49.)}
ſaying: \Emph{I wil expect thy ſaluation ô Lord}. And \Emph{Moyſes} (as
S.~Paul teſtifieth)
\CNote{Heb.~11.}
\Emph{denied him ſelfe to be the ſonne of Pharaoes
daughter, eſteming the reproch of Chriſt greater riches, then the
treaſure of the Ægyptians. For he looked vnto the reward.}
\MNote{Continuance of the Church notwithſtanding breaches from it.}
Thus much touching particular pointes of Religion. It reſteth to ſee the
viſible knowen members of the Church, with the heades and gouernors
therof, ſucceding without interruption in the ſame age, notwithſtanding
ſome brake and departed from them, and other innumerable ſectes of
Infidels ſtil multiplied in the world.

%%% o-0204
To
\MNote{Abraham neuer contaminate in Religion.}
beginne therfore with Abraham, before the former age was ended, (at which
time he was 75.~yeares old) holie Scriptures ſtil ſpeake of him, as
alwaies vndefiled, and a true ſeruant of God, though his
father \Emph{Thare} and his brother \Emph{Nachor} ſometimes \Emph{ſerued
ſtrange goddes},
\XRef{(Ioſue.~24.)}
\MNote{Thare and Nachor reduced from idolatrie.}
but were reclamed, and the whole familie, (as S.~Auguſtin proueth,
\Cite{lib.~16. c.~13. de ciuit)}
was perſecuted by the Chaldees. VVherupon Thare leauing Chaldea brought
Abraham, Lot, and Sarai, ſo farre as Haran in Meſopotamia
\XRef{(Gen.~11.)}
whither alſo Nachor repaired afterwards, and there made his habitation, as
appeareth. 
\XRef{(Gen.~24.)}
But Abraham vvas ſooner, and more ſpecially perſecuted in Chaldea, as
Ioſephus teſtifieth
\Cite{(li.~1. Antiq.)}
for
\MNote{Abraham publikly profeſſed his faith.}
his clere and publique profeſsion of one God, Creator of al things,
and that by his only goodnes, and not by mens ovvne povver, happines is
attained. Further Suidas
\Cite{(vocab. Abraham)}
vvriteth, that at the age of 14.~yeares, he admonished his father, not
for lucre ſake, to ſeduce men by vvorshipping images
%%% 0224
of falſe goddes, auouching that there is no other, but the celeſtical
God, maker of the whole world. In vvhich ſincere profeſsion hovv he
alvvaies perſeuered is often teſtified, and needles here to be
repeted. Alſo
\MNote{Sem.}
Sem,
\MNote{Sale.}
Sale, and
\MNote{Heber.}
Heber his proper anceſters (the ninth, ſeuenth, and ſixth in right line
before him) were al holie men, and liued al Abrahams time, much of
\Fix{Iſaachs,}{Iſaacs,}{possible typo, fixed in other}
and part of Iacobs dayes.
\MNote{Melchiſedech.}
Likevviſe Melchiſedech King and Prieſt (a diſtinct perſon, of an other
lineage, as vve ſuppoſe, from Sem) liued in the beginning of this age.
\MNote{Manie profeſſors of true Religion.}
Al which being renowmed men had great troupes, or rather countries,
which with them ſerued the only true God. VVherof we haue example, in
that Abraham (being but a ſtranger in Chanaan) vpon a ſuddaine exploite,
\XRef{(Gen.~14.)}
\Emph{made readie of the ſeruants borne in his houſe, three hundreth and
eighteene wel appointed}, men of armes, al of the ſame religion; for
shortly after they were al circumciſed
\XRef{(Gen.~17.)}
yet was king Melchiſedech of more power and authority then he. And the
other here mentioned, except his elder brother Nachor, and his nephevv
Lot, vvere his ovvne direct progenitors, and by liklihood more
potent. Againe from \Emph{Abraham} the ſucceſsion held on right \Emph{to
Aaron and Moyſes}, and the vvhole people of Iſrael, vvhich vvith them
paſſed out of Ægypt through the redde ſea.

But
\MNote{Breaches from the Church.}
in the meane time, diuers alſo of Abrahams kindred and ſeede, brake of
from this communitie: and fel to idolatrie. For albeit Lot, his brothers
ſonne perſeuered in the true ſeruice of God, yet Lots ſonnes, Moab and
Ammon, at leaſt the
\MNote{Moabites and Ammonites.}
Moabites and Ammonites, two nations that came of them
\XRef{(Gen.~19.)}
were infidels and idolaters. Likewiſe though Nachor, and
\MNote{Nachors progenie.}
Bathuel (Nachors ſonne) continued henceforth in true faith and religion,
yet Laban (the ſame Bathuels ſonne) had falſe goddes, vvhich Rachel
tooke away.
\XRef{(Gen.~31.)}
But true religion being not wholly extinguished in theſe families, both
\Fix{Iſaachs}{Iſaacs}{possible typo, fixed in other}
wife Rebecca, and Iacobs wiues Lia and Rachel; with their handmaides
Bala and Zelpha, either beleued rightly, or were more eaſily brought to
true beleefe, and ſeruice of God.
\MNote{Iſmaelites.}
\Emph{Iſmael} Abrahams firſt ſonne was in his youth euil \Emph{diſpoſed}
\XRef{(Gen.~21.)}
and for endeuoring to corrupt Iſaac (vvhich
\CNote{\XRef{Gal.~4.}}
S.~Paul calleth perſecution) was together with his mother
Agar, \Emph{caſt out of Abrahams houſe}, yet \Emph{proſpered in the
deſert}; had \Emph{twelue ſonnes dukes}, ſometimes viſited his father,
and
%%% o-0205
together with Iſaac buried him.
\XRef{(Gen.~25.)}
\CNote{\XRef{2.~Paral.~12.}
\XRef{16.}
\XRef{&~28.}}
And at the age of 137.~yeares \Emph{died and was put to his people},
that is to others like himſelf good or euil. Abraham alſo ſeparated his
other ſonnes begotten of Cetura
\XRef{(v.~6.)}
from Iſaac, to whom only and not to any other, the promiſed land of
Chanaan, and other more ſpecial bleſsings pertained.
\MNote{Madianites.}
Of theſe laſt ſonnes came the people of Madianites, who kept ſome
reſemblance with the people of God in religion, and therin prefigured
heretikes, that deſcend from Catholique race, but falling to ſchiſme &
hereſie, doe not participate eternal enheritance, with the ſpiritual
%%% 0225
children of God as S.~Auguſtin 
\Fix{teaceth.}{teacheth.}{obvious typo, fixed in other}
\Cite{(q.~70. in Gen.)}
In like ſorte of the two ſonnes of \Emph{Iſaac}, onlie \Emph{Iacob had
the ſpiritual bleſſing}, and enheritance therto belonging.
\XRef{(Gen.~27.)}
\MNote{Idumeans.}
\CNote{Heb.~12.}
\Emph{Eſau} though \Emph{prophane} in maners \Emph{ſelling his
birthright}
\XRef{(Gen.~25. v.~32.)}
which was a ſpiritual iuriſdiction wherin he was a figure of the
reprobate, yet it ſemeth he kept the true faith.
\XRef{(Gen.~35. v.~vlt.)}
But whether he did or no, ſure it is, Iob, (who is probably thought to
be of his race)
\XRef{(Gen.~36.)}
was a moſt holie man and a rare example of vertue. But the poſterities
of them both, and al the progenies of Iſmael, and of Abrahams other
ſonnes by Cetura, ſooner of later fel to infidelitie and idolatrie.
\MNote{Idolatrie ſtil increaſing yet the Church continued, yea alſo
increaſed.}
In other nations of the world, ſtil new goddes and goddeſes were
multiplied vpon euerie occaſion, as S.~Auguſtin
\Cite{(li.~18. de ciuit.)}
recounteth diuers. Al which notwithſtanding, the true Church and citie
of God continued moſt viſible and notorious, yea with meruelous
increaſe, eſpecially after they were more hated and afflicted in Ægypt.
\XRef{(Exo.~1.)}
VVhither they were brought by the ſtrange and ſpecial prouidence of God,
more ſtrangely preſerued, and moſt miraculouſly deliuered from thence.

Much
\MNote{The Church of Chriſt in the new Teſtamẽt alwayes viſible and
great.}
more the \Emph{Church of Chriſt} (wherof this was a shadow, and figure)
hath benne and shal be \Emph{euer} moſt \Emph{viſible}, from the firſt
foundation therof to the worlds end. For beſides the promiſes and
predictions in the new Teſtament, al the Scriptures alſo of the old,
which fortel Chriſt, do withal forshew his Church. \L{\Emph{Totum quod
annunciatur de Chriſto}} (ſaieth S.~Auguſtin
\Cite{de vnitate Eccleſ. c.~2.}
\L{\Emph{caput & corpus eſt.}}
\MNote{The ſame Scriptures forſhew Chriſt and his Church.}
Al that is ſpoken of Chriſt is (of) \Emph{the head and the bodie; The
head is the onlie begotten Ieſus Chriſt, the Sonne of the liuing God}:
he \Emph{the Sauiour of the bodie. His bodie the Church.} Againe
\Cite{(c.~4.)}
\L{Totus Chriſtus caput & corpus eſt.} VVhole Chriſt is the head and the
bodie. The head, the onlie begotten Sonne of God, and the bodie his
Church: the bridegrome and the bride, tvvo in one flesh. Yea for no
other cauſe (ſaieth he
\Cite{li. de catech. rud. c.~3.)}
were al thoſe things written, before the coming of our Lord, which we
read in holie Scriptures, but that his coming might be commended, and
the future Church prefigured, that is, \Emph{the people of God through
out al nations, which is his bodie.} The ſame doth S.~Paul teach vs, not
only ſaying
\XRef{(Gal.~3.)}
\Emph{The law was our pedagogue} (or conductor) \Emph{to Chriſt}, but
alſo
\XRef{(1.~Cor.~12.)}
\Emph{that as the} (natural) \Emph{bodie is one and hath manie
members, and al the members of the bodie, wheras they be manie, yet are
one bodie; ſo alſo Chriſt.} And
\XRef{(Coloſſ.~1.)}
that \Emph{Chriſts bodie is the Church.}
\MNote{Multitude of progenie promiſed to Abraham pertaineth to the
Church of Chriſt.}
As therfore the great bleſsing of redemption and ſaluation was promiſed
in Chriſt
\XRef{(Gen.~12.}
\XRef{&c.)}
ſo it was withal expreſsed, that al \Emph{nations}, and \Emph{kindreds
of the earth} should be partakers therof, yea ſo innumerable
\CNote{\XRef{Gen.~13.}}
\Emph{as the duſt of the earth}, the
\CNote{\XRef{15.}
\XRef{17.}}
\Emph{ſtarres of
%%% o-0206
heauen}, and
\CNote{\XRef{&~22.}}
\Emph{ſandes of the ſea}. VVhich S.~Paule ſaieth
\XRef{(Rom.~9.)}
is
%%% 0226
not ment of Abrahams natural children, but of \Emph{the children of
promiſe}, ſuch as the Romane Chriſtians, and others, Ievves and
Gentiles. So S.~Iohn ſavv in a viſion as a certaine number of
\CNote{\XRef{Apoc.~7.}}
\Emph{twelue thouſand ſigned of euerie tribe of Iſrael, but after theſe
a great multitude which no man could number of al nations, tribes,
peoples, and tongues.}
\MNote{Very abſurde to ſay, the Church of Chriſt was at anie time
obſcure.}
To ſay therfore, as ſome old and nevv heretikes doe, that the Church of
Chriſt ſome times conſiſteth of fevv, or, inuiſible perſons, vvere to
ſay God kept not promiſe vvith Abraham
\XRef{(Gen.~17.)}
and to make the bodie and thing figured, more obſcure then the shadovv
and figure; ſeeing in the whole time of the Lavv of nature, that is in
theſe three firſt ages of the vvorld, the Church being but a figure of
that vvhich is novv, yet vvas alvvayes viſible and notorious, as hath
benne declared.
\MNote{Succeſſion of ſpiritual gouerners during the law of nature.}
And that vvith perpetual ſucceſsion of ſupreme heades, rulers and
gouerners. 
%%% !!! Only in Heretikes. Doesn't seem to fit.
%%% \CNote{\XRef{Iob.~19.}}
As is before noted in the firſt age from Adam to Noe: in the ſecond from
Noe to Abraham: ſo in this third, by the right line of \Emph{Abraham,
Iſaac, Iacob, Leui, Caath, and Amran, to Aaron and Moyſes},
\XRef{(Exo.~6.)}
the Holie Ghoſt not there reciting more genealogies being come to the
origin of the
\Fix{Priſtlie}{Prieſtlie}{obvious typo, fixed in other}
Tribe, that is to theſe tvvo vvhom his diuine goodnes ſelected and
ordained, as vvel \Emph{to ſpeake to Pharao} in behalf of the children
of Iſrael, and \Emph{to bring them out of the Land of Ægypt}, as
aftervvards by one of them to giue his people a vvritten Lavv, and in
the other a perpetual prouiſion of ſpiritual paſtors.
\CNote{\XRef{Ex.~28.}
\XRef{Nu.~3.}}
\MNote{Prieſthood of Moyſes law eſtabliſhed in Aarons ſeede.}
For in Aaron the elder brother God eſtablished an ordinarie ſucceſsion
of Prieſthood, from that time to Chriſt, vvhich before pertained to the
firſtborne in euerie familie: adioyning the reſt of Leuites tribe to
aſsiſt them, in adminiſtration of ſacred things.
\MNote{Moyſes chiefe in ſpiritual and temporal gouernment.}
But Moyſes the younger brother vvas extraordinarily called (which God
therfore shewed and confirmed by ſpecial miracles) not onlie to
Prieſthood, but alſo to be as the God of Pharao, Superior of Aaron,
chiefe mediator betwen God and his people, as wel in deliuering them
from the ſeruitude of Ægypt, and in receiuing the Law, and deliuering it
to them, as in al other ſupreme gouernment ſpiritual and temporal during
his life.


\stopArgument


\stopcomponent


%%% Local Variables:
%%% mode: TeX
%%% eval: (long-s-mode)
%%% eval: (set-input-method "TeX")
%%% fill-column: 72
%%% eval: (auto-fill-mode)
%%% coding: utf-8-unix
%%% End:
