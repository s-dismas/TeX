%%%%%%%%%%%%%%%%%%%%%%%%%%%%%%%%%%%%%%%%%%%%%%%%%%%%%%%%%%%%%%%%%
%%%%
%%%% The (original) Douay Rheims Bible 
%%%%
%%%% Old Testament
%%%% Exodus
%%%% Argument
%%%%
%%%%%%%%%%%%%%%%%%%%%%%%%%%%%%%%%%%%%%%%%%%%%%%%%%%%%%%%%%%%%%%%%




\startcomponent argument


\project douay-rheims


%%% 0174
%%% o-0159
\startArgument[
  title={\Sc{The Argvment of the Booke of Exodvs.}},
  marking={The Argument of the Booke of Exodus.}
  ]

Moyſes
\MNote{The continuation of this booke with Geneſis.}
hauing proſecuted in Geneſis, the ſacred hiſtorie of the Church, vnto
\CNote{\XRef{Gen.~50.}}
Ioſephs death, containing the ſpace of 2310.~yeares, continueth the ſame
in Exodus, for 145.~yeares more.
\CNote{\XRef{Exod.~1.}}
VVhere he first briefly recounteth, how a ſmal number of Iſraelites,
eſpecially after the death of Ioſeph, being much increaſed,
\SNote{The increaſe of the Iſraelites was enuyed, feared, and their
religion hated.}
a new King (riſen in the meane time, who knew not Ioſeph) together with
other Ægyptians, \Emph{enuying} their better partes, both of bodie and
minde, and more fortunate progres in wealth; \Emph{fearing} alſo lest
they ſtil multiplying, either by their owne forces, or ioyning with
other foreners, might ſpoile Ægypt, and returne into Chanaan;
and \Emph{hating} their Religion, becauſe they acknowledged, one onlie,
eternal, omnipotent God, denying and deteſting the new imaginarie goddes
of the Ægyptians, reſolued and publickly decreed,
\MNote{Their perſecution.}
by oppreſsion to hinder their increaſing, & to keepe them in bondage and
ſeruitude. But God almightie, who had choſen them for his peculiar
people,
\MNote{Their greater multiplicatiõ.}
did not only ſo conſerue and multiplie them, that of ſeuentie perſons,
\CNote{\XRef{Exod.~2.}
\XRef{Num.~1.}}
which came into Ægypt, in the ſpace of two hundred and fifteene yeres,
there were ſix hundred thouſand men, able to beare armes, beſides wemen,
children, and old men, which by eſtimation might be three millions in
al, but amongſt other moſt ſtrange and miraculous workes,
\CNote{\XRef{Exod.~2.}}
\Fix{eſpecally}{eſpecially}{possible typo, same in both}
deliuered one Hebrew infant from drowning.
\CNote{\XRef{3.}}
Whom afterwards he made the Guide, and ſupreme Gouernour of the ſame
people;
\CNote{\XRef{5.}}
\MNote{The perſecutor admoniſhed, and puniſhed.}
by him admoniſhed the King to ceaſe perſecuting,
\CNote{\XRef{7.}}
and diuers waies plagued him & his people for their obdurat and
obſtinate crueltie.
\CNote{\XRef{12.}}
\MNote{Gods people mightely deliuered.}
In fine called away, and mightily deliuered his owne people,
\CNote{\XRef{14.}}
drowned that king and al his armie, in the red ſea, the Iſraelites
wonderfully paſsing through, as in a drie chanel, the waters ſtanding on
both ſides, like two walles.
\CNote{\XRef{16.}}
\MNote{Miraculouſly ſuſtained in the deſert.}
In the deſert, fed them miraculouſly with Manna, and gaue them al
neceſſaries, defending them alſo from enimies.
\CNote{\XRef{17.}}
Then God, hauing thus ſelected and ſeuered his people from al other
nations, gaue them a
\MNote{Inſtructed with Lawes, Moral, Ceremonial, and Iudicial.}
written law, as wel of Moral, as Ceremonial and Iudicial preceptes,
\CNote{\XRef{20.}}
with the maner of making the Tabernacle, erecting Altares, conſecrating
Prieſts,
\CNote{\XRef{26.}}
with the inſtitution of daylie Sacrifice, and 
\CNote{\XRef{27.}}
of al veſtures, veſelles,
\CNote{\XRef{30. &~ſeq.}}
& other holie things belonging to the ſeruice of God. So this booke may
be diuided into three partes. Firſt is declared the Iſraelites ſeruile
affliction in Ægypt, vvith their deliuerie from thence: in the fiftene
firſt chapters. Then how they were maintained in the deſert, and
prepared to receiue a law: in the foure next chapters. In the other
21.~chapters, the lavv is preſcribed, inſtructing them hovv to liue
tovvards God, and al men.


\stopArgument


\stopcomponent


%%% Local Variables:
%%% mode: TeX
%%% eval: (long-s-mode)
%%% eval: (set-input-method "TeX")
%%% fill-column: 72
%%% eval: (auto-fill-mode)
%%% coding: utf-8-unix
%%% End:
