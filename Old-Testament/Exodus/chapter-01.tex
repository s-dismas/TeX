%%%%%%%%%%%%%%%%%%%%%%%%%%%%%%%%%%%%%%%%%%%%%%%%%%%%%%%%%%%%%%%%%
%%%%
%%%% The (original) Douay Rheims Bible 
%%%%
%%%% Old Testament
%%%% Exodus
%%%% Chapter 01
%%%%
%%%%%%%%%%%%%%%%%%%%%%%%%%%%%%%%%%%%%%%%%%%%%%%%%%%%%%%%%%%%%%%%%

%%% Latin checked by KK.



\startcomponent chapter-01


\project douay-rheims


%%% 0175
%%% o-0160
\startChapter[
  title={Chapter 1}
  ]

\Summary{The ſmal number of Iſraelites much increaſing in Ægypt,
  6.~eſpecially after the death of Ioſeph and his brethren, 8.~a new
  king, that knew not Ioſeph in vaine ſtriueth to hinder their
  multiplication, 11.~by impoſing workes vpon them, 15.~and by
  commanding to kil, 22.~and to drowne al the
  \Fix{mal-children}{male-children}{likely typo, fixed in other}
  of them. God in the meane time rewardeth the midwiues, that fearing
  him, killed not the children.}

Theſe
\MNote{The firſt part of this booke. Of the Iſraelites ſeruile
affliction in Ægypt, and their deliuerie from thence.}
be the names of the children of Iſrael, that entred into Ægypt with
Iacob: they did enter in euerie one with their houſes, \V Ruben, Simeon,
Leui, Iudas, \V Iſſachar, Zabulon, and Beniamin, \V Dan, and Neptali,
Gad, and Aſer. \V Therfore al the ſoules that came out of Iacobs thigh,
were ſeuentie: and Ioſeph was in Ægypt. \V Who being dead, and al his
brethren, and al that generation, \V the children of Iſrael increaſed,
and as it were ſpringing vp did multiplie: and growing ſtrong
exceedingly, filled the land. \V In the meane time there aroſe a new
king ouer Ægypt, that knew not Ioſeph: \V and he ſaid to his people: Behold
%%% !!! (a)
\SNote{Enuie, vaine feare,
\XRef{(v.~10.)}
& hatred of true religiõ
\XRef{(v.~13.)}
are the cauſes why Infidels perſecute the faithful.}
the people of the children of Iſrael is much, and ſtronger then we. \V
Come, let vs wiſely oppreſſe the ſame,
%%% !!! this SNote marker below is a second referance to the (a) SNote
%%% above. How should we handle this? 
%%% \SNote{}
leſt perhaps it multiplie: and if there ſhal be anie warre againſt vs,
it ioyne with our enemies, and we being ouerthrowne, they depart out of
the land. \V Therfore
\SNote{The firſt perſecution was in temporal loſſes and bodilie paines,
by preſſing them with workes.}
he ſet ouer them maiſters of the workes, to afflict them with burdens:
and they built vnto Pharao cities of tabernacles,
%%% o-0161
Phithom, and Rameſſes. \V And the more they did oppreſſe them, ſo much
the more they multiplied, and increaſed: \V and the Ægyptians
%%% !!! this SNote marker below is a third referance to the (a) SNote
%%% above. How should we handle this?
%%% \SNote{}
hated
%%% 0176
the children of Iſrael, and deriding afflicted them: \V and they brought
their life into bitternes with the hard workes of clay, and bricke, and
with al ſeruice, wherewith they were preſſed in the workes of the
earth. \V And the King of Ægypt ſaid to the midwiues of the Hebrewes: of
whom one was called Sephora, the other Phua, \V commanding them:
\SNote{The ſecond was ſecrete murther.}
When you ſhal be midwiues to the Hebrew wemen, and the time of deliuerie
is come: if it be a manchild, kil it: if a woman, reſerue her. \V
\LNote{But the midvviues feared God.}{In commendation of the midwiues
not obeying the kings commandment,
\MNote{God muſt be feared before Princes commanding contrarie things.}
Moyſes oppoſeth the feare of God, to the feare of Princes; ſhewing
therby that when their commandments are contrarie, the ſubiects muſt
feare God, and not do that the Princes commandeth.
\CNote{\XRef{Mat.~10.}}
So did our Sauiour himſelf teach, and that for feare of damnation,
ſaying:
\CNote{\XRef{Luc.~12.}}
\Emph{Feare him vvho hath povver to caſt into hel.} And ſo
\CNote{\XRef{Act.~4.}}
his Apoſtles indued with the Holie Ghoſt, practiſed, anſwering in this
caſe, that they muſt heare God rather then men. Againe,
\CNote{&
\XRef{5.}}
\Emph{God muſt be obeyed rather then men.} Alwayes vnderſtood, when they
are contrarie.
\MNote{Princes muſt be obeyed in lawful things.}
For otherwiſe both
\CNote{\XRef{1.~Pet.~2.}}
S.~Peter and
\CNote{Ro.~13.}
S.~Paul teach vs, that Princes, yea Infidels, of whom they eſpecially
ſpeake, muſt be obeyed.}
But the midwiues feared God, and did not according to the commandement
of the king of Ægypt, but preſerued the menchildren. \V To whom being
called vnto him, the king ſaid: What is this that you ment to do, that
you would ſaue the men-children? \V Who anſwered: The
\LNote{Hebrevv vvemen are not.}{Herein
\MNote{Al lies are ſinnes and vnlawful.}
the midwiues ſinned. For it is neuer lawful to lye. Becauſe
\CNote{\XRef{Pſal.~118. v.~142.}}
\Emph{the lavv of God is truth}, wherby S.~Auguſtin proueth
\Cite{(li. cont. mend. c.~10.)}
that whatſoeuer varieth from truth is vnlawful. VVhen therfore (ſaith
he) examples of lying are propoſed to vs out of holie Scripture, either
they are not lies, but are thought to be, whiles they are not
vnderſtood, or if they be lies, they are not to be imitated, becauſe
they are vnlawful. S.~Gregorie teacheth the ſame
\Cite{(li.~18. Moral. c.~26.)}
\L{Quia profecto ab equitate diſcrepat, quiſquid a veritate
diſcordat.} \Emph{Becauſe aſſuredly vvhatſoeuer diſagreeth from veritie,
differeth from equitie.}
\MNote{Venial ſinnes.}
Yet theſe fathers hold ſuch an officious lye, as this was, to be a leſſe
ſinne, and more eaſily pardoned, and purged by good workes folowing.}
Hebrew wemen are not as the Ægyptian wemen: for they haue the knowledge
to play the
\Fix{mindwife}{midwife}{obvious typo, fixed in other}
them ſelues, and before we come to them, they are deliuered. \V God
therfore did wel to the midwiues: and the people encreaſed, and became
ſtrong exceedingly. \V And
\LNote{Becauſe the midvviues feared God.}{Feare
\MNote{Feare of God meritorious.}
of God as it is properly taken in holie Scripture, is that holie feare,
by which the children of God refraine from ſinne, and that with temporal
danger, leſt they ſhould offend the diuine Maieſtie. So theſe midwiues
endangering their owne liues, by not fulfilling Pharaos commandment, had
the true feare of God, and for the ſame were rewarded, as is moſt
probable, eternally:
\MNote{Temporal rewardes promiſed in the old Teſtamẽt, eternal in the
new.}
though mention be here made only of temporal reward, after the maner of
the old Teſtament. VVhere ſuch promiſes were made to Abraham, and other
moſt godlie Patriarches, for an aſſay only and taiſt of euerlaſting
life, which is more expreſly promiſed in the Goſpel of Chriſt, as
S.~Hierom teacheth,
\Cite{Epiſt. ad Dardanum.}}
becauſe the midwiues feared God, he built them houſes. \V Pharao therfore
commanded al his people, ſaying: Whatſoeuer ſhal be borne of the male
ſex,
\SNote{The third was open murther.}
caſt it into the riuer: whatſoeuer of the female, reſerue it.


\stopChapter


\stopcomponent


%%% Local Variables:
%%% mode: TeX
%%% eval: (long-s-mode)
%%% eval: (set-input-method "TeX")
%%% fill-column: 72
%%% eval: (auto-fill-mode)
%%% coding: utf-8-unix
%%% End:
