%%%%%%%%%%%%%%%%%%%%%%%%%%%%%%%%%%%%%%%%%%%%%%%%%%%%%%%%%%%%%%%%%
%%%%
%%%% The (original) Douay Rheims Bible 
%%%%
%%%% Old Testament
%%%% Exodus
%%%% Chapter 20
%%%%
%%%%%%%%%%%%%%%%%%%%%%%%%%%%%%%%%%%%%%%%%%%%%%%%%%%%%%%%%%%%%%%%%




\startcomponent chapter-20


\project douay-rheims


%%% 0235
%%% o-0215
\startChapter[
  title={Chapter 20}
  ]

\Summary{Moyſes
\MNote{The third part of this booke: containing Diuine Lawes: Moral,
  Ceremonial, and Iudicial.}
receiueth the Decologue or tenne commandments of God, for al the people,
23.~with repetition that they shal not make falſe goddes, nor make
Altares but of earth, or vnhewed ſtone, and without ſteppes.}

%%% 0236
And our Lord ſpake al theſe wordes: \V I am the Lord thy God, which
brought thee forth out of the Land of Ægypt, out of the houſe of
ſeruitude. \V Thou shalt not haue
\LNote{Strange goddes.}{Proteſtants
\MNote{Proteſtants charge al Catholiques to be Idolaters.}
pretend here to proue, that al Catholiques are Idolaters, for honoring
Sainctes, and their Reliques and Images. And they haue ſo defamed
Catholique Religion in this behalfe, that the vulgar ſorte of deceiued
people, otherwiſe knowing Catholiques to be ordinarily of moderate
conuerſation in life, of iuſt dealing towardes their neighboures,
addicted to prayer, faſting, almes, and manie good
\Fix{woorkes,}{workes,}{likely typo, fixed in other}
more wanting among them ſelues:
\MNote{They abuſe their folowers.}
yet ſuppoſing them, notwithſtanding theſe laudable qualities, to be
Idolaters, are therby auerted from Catholique Religion. And ſurely it
were a iuſt cauſe, if it were true. As wel therfore to purge our ſelues
of ſo haynous an imputed crime, as to remoue this dangerous block of
erronious conceipt, we ſhal here note ſome of the Proteſtants egregious
lies, againſt the whole Church militant, and blaſphemous reproches
againſt
\Fix{the the}{the}{obvious typo, fixed in other}
glorious Sainctes: then briefly declare the true and ſincere doctrine,
and practiſe of the Catholike Church in this point.
\MNote{They belie the Church militant.}
Luther in his
\Cite{poſtil vpon the Goſpel of our Lordes Incarnation,}
ſayth:
\L{Papiſta Virginem Mariam Deum conſtituunt: Omnipontentiam ei in cælo,
& in terra tribuunt.} The Papiſts (ſaith he) make the Virgin Marie God:
they attribute to her omnipotencie in heauen and in earth. In Papiſtrie
al expected more fauour and grace from her, then from Chriſt himſelf. His
ſcholar Melancton
\Cite{(in locis communib.)}
poſtilling the firſt Precept, ſaith: Papiſtes inuocate Sainctes, and
worſhip Images in heathniſh maner. Caluin
\Cite{(li. de neceſſ. refor. Eccleſ.)}
ſaith: thoſe of the Emperours religion (meaning al Catholiques) ſo
diuide Gods offices among Sainctes, that they ioyne them to the
Soueraigne God, as collegues; in which multitude God lieth
hidden. Againſt the moſt glorious virgin mother the ſame Luther
\Cite{(ſer. de natali virg. Mar.)}
feared not to ſay, that he eſtemed no more of the prayer
of \Emph{S.~Marie}, then of anie one of the people.
\MNote{Blaſpheme the triumphant.}
And his reaſon is worſe then his wicked aſſertion, for that, ſaith he,
al that beleue in Chriſt are as iuſt, and as holie as the virgin Marie,
or anie other Sainct how great ſoeuer. The Magdeburgian Centuriators
\Cite{(li.~1. Cent.~1.)}
affirme that the virgin Marie ſinned greuouſly, yea compare her imagined
faultes with the ſinne of Eue in paradiſe.
\Cite{(li.~2.)}
They charge S.~Peter and S.~Paul (alſo after their conuerſions) with
diuers great crimes. Caluin
\Cite{(li.~3. Inſt. c.~2. parag.~31.)}
condemneth Sara and Rebecca of great ſinnes,
\Cite{(c.~4.)}
reprehendeth Iudas Machabeus for ſuperſtitious, and prepoſterous zeale,
in cauſing Sacrifice to be offered for the dead. In his commentarie
\Cite{(in 32.~Exod.)}
he accuſeth moſt holie and meke Moyſes of arrogancie and pride. And
\Cite{(li.~3. Inſtit. c.~20. pa.~27.)}
he ſcuruely ſcoffeth at al Sainctes in general, ſaying: If they heare
mortal mens prayers, they muſt haue eares ſo long, as from heauen to
earth. And calleth them not only \L{homines mortuos}, \Emph{dead men},
(which S.~Hierom reproued in
\Cite{Vigilantius)}
but alſo \L{vmbras, laruas, colluuiem}: \Emph{shadovves, night goblins,
ſtincking filth} yet more,
\Cite{(li. de vera refor. Eccleſ. rat.)}
he calleth them \L{Monſtra, carnifices, beſtias}, \Emph{monſters,
hangmen, beaſtes}.
\MNote{Al modeſt mẽ wil condemne theſe blaſphemies.}
Theſe and like blaſphemies modeſt men can not but abhore and
deteſt.
\MNote{Catholique doctrine and practiſe conuince their lies.}
Their lies alſo are conuinced by S.~Hierom, handling this matter of
purpoſe againſt
\Cite{Vigilantius,}
by S.~Auguſtin touching it by occaſion
\Cite{(li.~20. c.~21.)}
againſt Fauſtus the Manachey, Thomas VValdenſis
\Cite{(to.~3. tit.~13. de Sacramentalibus)}
againſt Wiclif, by al Catechiſmes and Chriſtian Inſtructions, teaching
nothing like, but quite contrarie to theſe mens reportes.
\MNote{The true Catholique doctrin.}
In ſumme they al teach, that Sainctes are to be honored with religious
honour, which is greater then ciuil, but infinitly inferiour to diuine,
as the excellencie of God ſurmounteth al excellencie created.

For
\MNote{Honour due to excellencie.}
better declaration wherof, it is to be conſidered, that ſeing by the
law of God and nature, honour is due to excellencie, there muſt be ſo
manie diſtinct kindes of honour, as there be general kindes of
excellencie, which are three.
\MNote{Three kindes of excellencie.}
The firſt of God, infinite, and incomparably aboue al: the ſecond is
ſupernatural but created, as of grace and glorie: the third is humane or
natural, conſiſting in natural giftes, or worldlie powre and dignitie,
al three as diſtinct as God, heauen, and earth.
\MNote{Therfore three kindes of honour.}
To theſe three general kindes of excellencie perteine therfore other
three as diſtinct kindes of honour; to wit, Diuine due to God only,
called by vſe and appropriation of a greeke word \GG{Latria}: the
ſecond \GG{Dulia}, belonging to Sainctes, and other holie things,
eleuated by God aboue the courſe of nature, in diuers degrees, but within
the ranck of creatures: the third is ciuil honour, due to humane and
worldlie excellencie, according to diuers ſtates and qualities of
men. The firſt of theſe which is diuine, may in no caſe be geuen to anie
creature, how excellent ſoeuer. The third which is ciuil, as both
Catholiques and Proteſtants hold for
\Fix{cetraine,}{certaine,}{obvious typo, fixed in other}
is not competent nor agreable to Sainctes, but to mortal worldlie men in
reſpect of temporal excellencie. Al the controuerſie therfore is about
the ſecond.
\MNote{Proteſtants denie anie honour to be due to Saincts.}
VVhich Caluin
\Cite{(li.~1. Inſtit. c.~11. &~12.)}
and al proteſtant writers denie & reiect, and ſo would haue no honour at
al geuen to Sainctes.
\MNote{Their obiection.}
Obiecting as old heretikes did, that Catholiques do al the ſame external
actes, as ſtanding bare head, bowing, kneeling, praying, and the like to
Sainctes, as to God himſelf.
\MNote{Firſt anſwer.}
VVe anſwer, that the diſtinctiõ of honour cõſiſteth not alwayes in the
external action, but in the intention of the mind. For when we do ſuch
external actes of honour to God, we intend therby to honour the Creator
and Lord of al, and ſo it is diuine honour, but doing the ſame external
actes to a Saint, we conceiue of him, as a glorious ſeruant of God, and
ſo we honour him as a ſanctified and glorified creature, Gods ſubiect
and ſeruant.
\MNote{Example of this neceſſary diſtinction.}
VVithout this diuerſitie of intentions in your mind, you can not ſhew
difference, betwen the honour you do to God, and that you do to the
King, by bowing, kneeling, and the like. For it is the ſame external
action: yet no Chriſtian doubteth but he honoreth God with diuine
honour, & the King with ciuil.
\MNote{Second anſwer.}
Againe we anſwer, that we do not al the external actions of honour to
Sainctes, which we doe to God. For Sacrifice is donne only to God, and
to no Sainct; and becauſe Altares perteine to Sacrifice, they are erected
to God only, though oftentimes in memorie of Saincts.

Both
\MNote{S.~Auguſtin declareth this doctrin: and geueth both the former
anſwers.}
which anſwers S.~Auguſtin gaue long ſince, to Fauſtus the Manachie,
arguing that Catholiques by doing the ſame external actes, worſhipped
Martyrs with diuine honour, and ſo turned them into Idols, as that
heretike inferred. VVherupon S.~Auguſtin declareth,
\CNote{\Cite{li.~20. c.~21.}}
that
\MNote{Three cauſes of celebrating Saincts memories.}
Chriſtian people celebrate together the memories of Martyrs
with \Emph{Religious ſolemnitie}, to ſtyr vp imitation, to be partakers
of their merites, and to be holpen by their prayers. Yet ſo that we
erect not Altares (becauſe they are for Sacrifice) to anie Martyr,
though in memorie of Martyrs, but to God of Martyrs. For who euer
ſtanding at the Altar, in places of Sainctes bodies, ſaide: VVe offer to
thee Peter, or Paul, or Cyprian, but that which is offered, is offered
to God, who crowned the Martyrs, at their memories, whom he crowned,
that by commonition of the very places, greater affection may ariſe, to
inkindle charitie, both towards them, whom we may imitate, and towards
him, by whoſe helpe we may. VVe honour Martyrs with that worſhip of loue
and ſocietie, wherwith holie men are worſhipped in this life. VVhoſe
hart we perceiue is prepared to like ſufferance for the Euangelical
veritie: but Martyrs more deuoutly, by how much more ſecurly, after al
vncertainties are ouercome, and with how much more confident praiſe, we
preach them now victours in a more happie life, then others yet fighting
in this.
\MNote{\GG{Latria} is honour proper to God.}
But with that worſhippe, which in greke is called \GG{Latria}, \Emph{a
ſeruice properly due to God}, which in Latin can not be expreſſed by one
word, we neither worſhip, nor teach to be worſhipped but one God.
\MNote{Sacrifice only to God.}
And for ſo much as offering of Sacrifice perteineth to this worſhippe
(wherof they are called Idolaters, that offer ſacrifice to anie Idols)
we by no meanes offer anie ſuch thing, nor teach to be offered, either
to anie Martyr, or bleſſed ſoule, or holie Angel. Thus farre
S.~Auguſtin. The ſame teacheth Theodoret.
\Cite{(li.~8. ad Græcos)}
Our Lord hath depriued falſe goddes of the honour, they had in Temples,
and in place of them cauſed his Martyrs to be honoured: yet not in the
ſame maner, for we neither bring hoſtes, nor libaments to Martyrs, but
honour them, as holie men, and moſt deare freinds of God. It would be to
long to cite manie ancient Fathers, teſtifying and teaching that Saincts
are to be honored.

More
\MNote{Proteſtants confeſſe that the ancient Fathers honored Saincts,
and their Reliques.}
compendiouſly we wil take our aduerſaries confeſſion, the Magdeburgian
Centuriators. VVho
\Cite{(Pref. Cent.~6.)}
holding that the Church was only pure from idolatrie the firſt hundred
yeares of Chriſt, and that it begane to faile in the ſecond and third
age, more in the fourth and fifth, and was vtterly periſhed in the
ſixth, impute the cauſe of her ruine, that the very chiefe men taught
and practiced the honour of Saincts. Firſt of al (ſay they) theſe
horrible and pernicious darknes, as certaine black cloudes couering the
whole firmament, roſe vp in the verie aſſemblie of teachers. For that
partly the very Doctors of the Church, partly other ſuperſticious men,
augmented ceremonies and humane worſhippes in the Temples. 
%%% !!! I don't think this belongs here.
%%% It is not marked in the text, but is in the margin of both.
%%% \SNote{Manna was put in a golden veſſel.
%%% \XRef{Heb.~9.}}
For ſacred houſes began to be built in al places, with great coſte,
altogether in heathniſh maner: not principally to the end, Gods word
might there be taught, but that ſome honour might be exhibited to the
Reliques of Saincts, and that fooliſh people might there worſhip dead
men.
\MNote{How ſaucie are heretikes to ſcoffe at ſo renowmed a Doctor!}
And how
\Fix{pleaſant}{pleaſantly}{likely typo, fixed in other}
eloquent is that Gregorie, called the great, how feruent, when, as from
his three footed ſtoole, he preached the maner of conſecrating theſe
houſes? And a litle after, by this occaſion dead creatures, and bloudles
half wormeaten bones began to be honored, inuocated, and worſhipped with
diuine honour. Al which \Emph{The Doctors of the Church} not only
wincked at, but alſo \Emph{ſet forvvard}. Thus the reader ſeeth,
notwithſtanding their lies, ſcoffes, and blaſphemies, Proteſtants do
confeſſe, that the Church and her chiefe pillers, ſtraight after the
firſt hundred yeares of Chriſt, fiue hundred next folowing, honored
Saincts and their Reliques. Neither wante there
\Fix{autentical}{authentical}{likely typo, fixed in other}
examples of holie Scriptures, wherby the ſame is proued. As
\XRef{Gen.~32.~48.}
\XRef{Exod.3.~32.}
\XRef{Num.~22.}
\XRef{Ioſue~5.}
\XRef{3.~Reg.~18.}
\XRef{4.~Reg.~2.}
\XRef{Pſalm.~98.}
and els vvhere.}
ſtrange goddes before me. \V Thou shalt not make to thee
\LNote{A grauen thing.}{Here
\MNote{Proteſtants haue corrupted the text in al their Engliſh Bibles.}
the ſame falſifiers of Chriſtian doctrin, do not only peruert the ſenſe
of holie Scripture, wreſtling that againſt Images, which is ſpoken
againſt Idols, but alſo ſhamfully corrupt the text, by
tranſlating \Emph{grauen image}, neither folowing the Hebrew, Greke, nor
Latin. For the Hebrew word, \HH{peſel}, is the verie ſame
that \L{ſculptile} in Latin, that is \Emph{a grauen or carued
thing}. The Greke hath \G{εὶδωλον}, \Emph{an idol}. So al Proteſtants
Engliſh Bibles are falſe.

In
\MNote{God commanded to make Images.}
the meane time til they correct their bookes, they may pleaſe to
remember, that God ſhortly after this
\XRef{(Exod.~25.)}
commanded to make Images of Angels, to wit Cherubins. Likewiſe a braſen
ſerpent.
\XRef{(Num.~21.)}
Alſo oxen and Lions
\XRef{(3.~Reg.~6.}
&
\XRef{7.)}
Neither are Puritanes ſo preciſe, but that they engraue, carue, print,
paint, caſt, ſow, embrother, and otherwiſe make, and kepe Images,
portractes, and pictures of men, and other things. As for worſhipping of
ſacred Images the ſecond concel of Nice
\XRef{(Act.~4.)}
The concel of Trent
\Cite{(ſeſſ.~25.)}
S.~Gregorie the great
\Cite{(li.~7. ep.~5.}
&
\Cite{53.)}
S.~Damaſcen in diuers whole bookes, and manie others, and al Catholique
Catechiſmes and Chriſtian Inſtructions teach,
\MNote{Chriſt, and Saincts are honored in their Images.}
that the honour is not
done to the Image for it ſelf, but at the preſence of the Image, to Chriſt,
or Sainct, whoſe Image it is.

An other controuerſie Caluin here maketh, that from theſe
wordes, \Emph{Thou shalt not make}, beginneth the ſecond precept, ſo
counting foure precepts in the firſt table, and ſix in the ſecond. But
being no matter of faith, how they are diuided, ſo al the wordes, and
the number of tenne commandements be acknowledged (for holie Scripture
calleth them tenne,
\XRef{Exo.~34. v.~28.}
\XRef{Deut.~4. v.~13.}
&
\XRef{10. v.~4.)}
we wil not contend:
\MNote{The firſt table containeth three precepts, the ſecond ſeuen.}
but only as more reaſonable we folow the common maner of diuiding the
firſt table into three precepts, directing vs to God, the ſecond into
ſeuen, belonging to our neighbour, approued for the better by
S.~Auguſtin
\Cite{(q.~71. in Exodum)}
and generally receiued of al Catholiques;
\MNote{The firſt can not wel be diuided.}
grounded vpon this reaſon, among others, becauſe to make or haue a
picture, or ſimilitude of anie creature, to the end to adore it as God,
were in dede to haue a ſtrange God, which is forbid in the firſt wordes.
And ſo al that foloweth to the comination and promiſe, forbiddeth falſe
goddes, and appeareth to be but one precept in ſubſtance.
\MNote{The ninth and tenth are as diſtinct, as the ſixth and ſeuenth.}
But the deſire and internal conſent to adultrie, and to theift, differ
altogether as much, as the external actes of the ſame ſinnes; and
therfore ſeing adultrie and theift are forbidden to be comitted, by two
diſtinct precepts, the prohibition of the internal deſire, with mental
conſent to the ſame, doth alſo require two precepts.}
a
\SNote{In Hebrew \HH{Peſel}, in Greke \G{ειδωλον}, in
Latin \L{ſculptile}, in Engliſh \Emph{a grauen thing}.}
grauen thing, nor any ſimilitude that is in heauen aboue, & that is in
the earth beneth, neither of thoſe thinges that are in the waters vnder
the earth. \V Thou shalt not adore them, nor ſerue them: I am the Lord
thy God mightie, ielous,
\SNote{This commination and promiſe annexed to the firſt commandment
perteyneth to euerie one of the nine folowing.
\Cite{Catech. Ro. p.~3. q.~9.}}
viſiting the iniquitie of the fathers vpon the children, vpon the third
and fourth generation of them that hate me: \V and doing mercie vpon
thouſandes to them that loue me, and keepe my preceptes. \V Thou shalt
not take the name of the Lord thy God in vaine, for the Lord wil not
hold him innocent that shal take the name of the Lord his God vainly. \V
Remember that thou ſanctifie the ſabbath day. \V Six dayes shalt thou
worke, and ſhalt doe al thy workes. \V But on the ſeuenth day is the
ſabbath of the Lord thy God: thou shalt doe no worke in it, thou, and thy
ſonne, and thy daughter, thy man ſeruant, and thy woman ſeruant, thy
beaſt, and the ſtranger that is with in thy gates. \V For ſix dayes the
Lord made heauen and earth, and the ſea, and al thinges that are
\Fix{in in}{in}{obvious typo, fixed in other}
them, and reſted in the ſeuenth day, therfore the Lord bleſſed the
ſabbath day, and ſanctified it. \V Honour thy father and thy mother, that
thou mayſt be
\Fix{longliude}{longliued}{obvious typo, fixed in other}
vpon the earth, which the Lord thy God wil geue thee. \V Thou shalt not
murder. \V Thou shalt not committe aduoutrie. \V Thou shalt not
ſteale. \V Thou shalt not ſpeake againſt thy neighbour falſe
teſtimonie. \V Thou shalt not couet thy neighbours houſe: neither shalt
thou deſire his wife, nor ſeruant, nor handmaide, nor oxe, nor aſſe, nor
any thing that is his. \V And al the people ſaw the voices and the
flames, and the ſound of the trumpet, and the mount ſmoking: and being
frighted and ſtroken with feare they ſtoode a farre of, \V ſaying to
Moyſes: Speake thou to vs, and we wil heare: let not our Lord ſpeake to
vs, leſt perhappes we die. \V And Moyſes ſaid to the people: Feare not:
for God came to proue you, and that his terrour might be in you, and you
should not ſinne. \V And the people ſtiide a farre of. But Moyſes went
vnto the darke cloud wherin God was. \V Our Lord ſaid
%%% o-0216
moreouer to Moyſes: This ſhalt thou ſay to the children of Iſrael: You
haue ſeene that from
%%% 0237
heauen I haue ſpoken to you. \V You ſhal not make goddes of ſiluer, nor
goddes of gold ſhal you make to you. \V An Altar
\SNote{This and other ceremonial precepts are determinate lawes, for
obſeruing the cõmandments of the firſt table pertaining to God.}
of earth you ſhal make to me, and you ſhal offer vpon it your holocaſtes and
pacifiques, your ſheepe and oxen in euerie place where the memorie of my
name ſhal be: I wil come to thee, and wil bleſſe thee. \V And if thou
make an Altar of ſtone vnto me, thou ſhalt not build it of hewed ſtones:
for if thou lift vp thy knife ouer it, it shal be polluted. \V Thou
shalt not goe vp by griefes vnto myne Altar, leſt thy turpitude be
diſcouered.


\stopChapter


\stopcomponent


%%% Local Variables:
%%% mode: TeX
%%% eval: (long-s-mode)
%%% eval: (set-input-method "TeX")
%%% fill-column: 72
%%% eval: (auto-fill-mode)
%%% coding: utf-8-unix
%%% End:
