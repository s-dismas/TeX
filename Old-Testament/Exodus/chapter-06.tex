%%%%%%%%%%%%%%%%%%%%%%%%%%%%%%%%%%%%%%%%%%%%%%%%%%%%%%%%%%%%%%%%%
%%%%
%%%% The (original) Douay Rheims Bible 
%%%%
%%%% Old Testament
%%%% Exodus
%%%% Chapter 06
%%%%
%%%%%%%%%%%%%%%%%%%%%%%%%%%%%%%%%%%%%%%%%%%%%%%%%%%%%%%%%%%%%%%%%




\startcomponent chapter-06


\project douay-rheims


%%% 0186
%%% o-0170
\startChapter[
  title={Chapter 6}
  ]

\Summary{God reueling himſelfe more to Moyſes then he had done to former
  Patriarches, 6.~commandeth him to tel the children of Iſrael, that he
  ſeeing their miſeries, wil deliuer them from Ægypt, and geue them
  poſſeſsion of Chanaan. 14.~The genealogies of Ruben, Simeon, and
  eſpecially of Leui are recited, 26.~to shew the origin of Moyſes and
  Aaron.}

And our Lord ſaid to Moyſes: Now thou shalt ſee what thinges I wil doe
to Pharao: for by a mightie hand shal he diſmiſſe them, and in a ſtrong
hand shal he caſt them
%%% 0187
out of his land. \V And our Lord ſpake to Moyſes, ſaying: I am the
Lord \V that appeared to Abraham, to Iſaac and to Iacob, as God
almightie: and
\LNote{My name \HH{Adonai}.}{Here
\MNote{In place of the name of God counted ineffable, is commonly
redde \HH{Adonai}.}
and in manie other places of holie
Scripture in the Hebrew text, is that name of God of foure letters,
which the Iewes ſay is ineffable. Yet ſure it is, that Moyſes heard it
pronounced, and afterwards writte it as he did the reſt in Hebrew
letters (which are al conſonants) without vowels. But the Rabbins that
long after put points or vowels to al other words, put none to this. For
al then redde \Emph{Adonai} in place therof. And ſo the Latin, and al
vulgar Catholique verſions, keepe the ſame word vntranſlated. The
Septuagint in Greke tranſlate
%%% !!! \G{\Sc{Κυριοσ}}
\G{ΚΥΡΙΟΣ}, which in Latin is \L{Dominus}, in Engliſh \Emph{Lord}. So
alſo al ancient Fathers, and (which is moſt of al)
\CNote{\XRef{Mat.~4. v.~7.~10.}
\XRef{Rom.~15. v.~11.}}
our Sauiour, and his Apoſtles, alleaging ſentences of the old Teſtament,
where this name is contained, ſtil expreſſe it by wordes that
ſignifie \Emph{Lord}. Only certaine late writers haue framed a new word,
by putting the points of \HH{Adonai}, to the proper letters of this
vnknowen name, which are \Emph{Iod}, \Emph{He}, \Emph{Vau}, \Emph{He},
and ſo ſound 
it \HH{Iehouah}:
\MNote{\HH{Iehouah} is not the right name of God.}
which was ſcarce heard of before an hundred yeares. As Biſhop Genebrard,
Cardinal Bellarmin, and F.~Pererius proue, for that neither
\CNote{S.~Dionyſc. S.~Hierom. Theodoret. Damaſcen.}
ancient
Fathers, writing whole Treatiſes \Emph{de Diuinis nominibus}, nor the
elder Rabbins, nor later moſt learned Hebricians, as Rabbi Moyſes, Aben
Ezram, Lira, Paulus Burgenſis and many others, neuer mention \HH{Iehouah}
amongſt the Names or titles of God.}
my name
\SNote{Adonai is not the name here vttered to Moyſes but is redde in
place of the vnknowen name.}
\Sc{Adonai} I did not ſhew them. \V And I made a couenant with them, to
geue them the Land of Chanaan, the land of their pilgrimage, wherein
they were ſtrangers. \V And I haue heard the groning of the children of
Iſrael, wherwith the Ægyptians haue oppreſſed them: and I haue remembred
my couenant. \V Therfore ſay to the children of Iſrael: I the Lord who
wil bring you forth out of the worke-priſon of the Ægyptians, & wil
deliuer you from ſeruitude: and redeme you in a high arme, and
%%% o-0171
great iudgements. \V And I wil take you to me for my people, and I wil
be your God: and you shal know that I am the Lord your God, that brought
you forth out of the worke-priſon of the Ægyptians: \V and brought you
into the land, ouer which I lifted vp my hand to geue it to Abraham,
Iſaac, and Iacob: and I wil geue it you to poſſeſſe, I the Lord. \V
Moyſes then told al to the children of Iſrael: who did not hearken vnto
him, for anguish of ſpirit, and moſt painful worke. \V And our Lord ſpake
to Moyſes, ſaying: \V Goe in, and ſpeake to Pharao the king of Ægypt,
that he diſmiſſe the children of Iſrael out of his land. \V And Moyſes
anſwered before our Lord: Behold the children of Iſrael heare me not:
and how wil Pharao heare, eſpecially wheras I am of vncircumciſed
lippes? \V And our Lord ſpake to Moyſes and Aaron, and he gaue them
commandement vnto the children of Iſrael, & vnto Pharao the king of
Ægypt, that they should bring forth the children of Iſrael out of the
land of Ægypt. \V Theſe are the Princes of their houſes by their
families. The ſonnes of Ruben the firſt begotten of Iſrael: Henoch and
Phallu, Heſron and Charmi. \V Theſe are the kinreds of Ruben. The ſonnes
of Simeon: Iamuel and Iamin, and Ahod, and Iachin, and Soar, and Saul
the ſonnes of the Chananiteſſe, theſe are the progenies of Simeon. \V And
theſe are the names of the ſonnes of Leui by their kinreds: Gerſon and
Caath and Merari. And
\SNote{The yeares of Ioſeph dying firſt of Iacobs ſonnes
\XRef{Gen.~50.}
and of Leui liuing longeſt, and none of the reſt, are not without
myſtery, recorded in holie Scriptures.
\Cite{Chronol. Hebr.}}
the yeares of the life of Leui were an hundred thirtie ſeuen. \V The
ſonnes of Gerſon: Lobni and Semi, by their kinreds. \V The ſonnes of
Caath: Amran, and Iſaar, and Hebron and Oziel. The yeares alſo
of Caaths life, were and hundred thirtie three. \V The ſonnes of
Merari: Moholi and Muſi. Theſe be the kinreds of Leui by
%%% 0188
their families. \V And Amran tooke to wife Iocabed
\SNote{See
\XRef{Num.~26. v.~59.}}
his
\TNote{\L{patruelem pro patrua, quæ Latine non dicitur.}}
aunt by the fathers ſide: who bare him Aaron and Moyſes. And the yeares
of Amrans life were an hundred thirtie ſeuen. \V The ſonnes alſo of
Iſaar: Coree, and Nepheg, and Zechri. \V The ſonnes alſo of Oziel:
Mizael, and Elizaphan, and Sethi. \V And Aaron tooke to wife Elizabeth
the daughter of Aminadab, ſiſter of Nahaſon, who bare him Nadab, and
Abiu, and Eleazar, and Ithamar. \V The ſonnes alſo of Core: Aſer, and
Eleana, & Abiſaph. Theſe be the kinreds of the Corites. \V But Eleazar
the ſonne of Aaron tooke a wife of the daughters of Phutiel: who bare
him Phinees.
\SNote{It perteined not to Moyſes preſent purpoſe, to proſecute the
genealogies of Iacobs other ſonnes, being come to the origin of the
Prieſtlie tribe in Leui the third ſõne.
\Cite{S.~Aug. q.~15. in Exod.}}
Theſe are the heads of the Leuitical families by their kinreds. \V
This is Aaron and Moyſes, whom our Lord commanded that they ſhould bring
forth the children of Iſrael out of the land of Ægypt by their
troupes. \V Theſe are they that ſpake to Pharao the king of Ægypt, that
they might bring forth the children of Iſrael out of Ægypt: this is
Moyſes, and Aaron, \V in the day when our Lord ſpake to Moyſes in the
land of Ægypt. \V And our Lord ſpake to Moyſes, ſaying: I the Lord:
ſpeake to Pharao the king of Ægypt, al thinges which I ſpeake to
thee. \V And Moyſes ſaid before our Lord: Loe I am of vncircumciſed
lippes, how wil Pharao heare me?


\stopChapter


\stopcomponent


%%% Local Variables:
%%% mode: TeX
%%% eval: (long-s-mode)
%%% eval: (set-input-method "TeX")
%%% fill-column: 72
%%% eval: (auto-fill-mode)
%%% coding: utf-8-unix
%%% End:
