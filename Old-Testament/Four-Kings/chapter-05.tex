%%%%%%%%%%%%%%%%%%%%%%%%%%%%%%%%%%%%%%%%%%%%%%%%%%%%%%%%%%%%%%%%%
%%%%
%%%% The (original) Douay Rheims Bible 
%%%%
%%%% Old Testament
%%%% Four Kings
%%%% Chapter 05
%%%%
%%%%%%%%%%%%%%%%%%%%%%%%%%%%%%%%%%%%%%%%%%%%%%%%%%%%%%%%%%%%%%%%%




\startcomponent chapter-05


\project douay-rheims


%%% 0789
%%% o-0708
\startChapter[
  title={Chapter 5}
  ]

\Summary{Naaman the general captaine of Syria is clenſed of
\Fix{deproſie,}{leproſie,}{obvious typo, same in both}
by washing him ſelf as Eliſeus appointeth him ſeuen times in Iordan:
15.~profeſseth his belief in one God promiſing to ſerue him. 20.~Giezi
taketh giftes of Naaman, 25.~and is ſtricken with leproſie.}

Naaman the General of the warfare of the king of Syria, was a great man
with his lord, and honorable: for by him our Lord gaue health to Syria:
and he was a valiant man and rich, but a leper. \V Moreouer out of Syria
there were come forth robbers, and had led away captiue out of the Land
of Iſrael a litle girle, which wayted vpon the wife of Naaman. \V Who
ſaid to her maiſtreſſe: I would my lord had bene with the prophete, that
is in Samaria: ſurely he would haue cured him from the leproſie, which
he hath. \V Naaman therfore went in to his lord, and told him ſaying:
Thus and thus hath the wench of the Land of Iſrael ſpoken. \V And the
king of Syria ſayd to him: Goe, & I wil ſend letters to the king of
Iſrael. Who when he was ſette forward, and had taken with him ten
talentes of ſiluer, and ſix thouſand peces of gold, and ten change of
rayment, \V he brought the letters to the king of Iſrael, in theſe
wordes: When thou shalt receiue this letter, know that I haue ſent to
thee Naaman my ſeruant, that thou mayeſt cure him of his leproſie. \V
And when the king of Iſrael had read the letters, he rent his garmentes,
and ſaid: Am I God, that I can kil, and geue life, becauſe this man hath
ſent to me, that I should cure a man of his leproſie? marke, and ſee
that he ſeeketh occaſions againſt me. \V Which when Eliſeus the man of
God had heard, to witte, that the king of Iſrael had rent his garmentes,
he ſent to him, ſaying: Why haſt thou rent thy garmentes? let him come
to me and let him
%%% 0790
know that there is a prophet in Iſrael. \V Naaman therfore came with
horſes and chariotes, and ſtood at the doore of the houſe of Eliſeus: \V
and Eliſeus ſent a meſſenger to him, ſaying: Goe and be washed ſeuen
times in Iordan, and thy flesh shal receiue health, and thou shalt be
cleane. \V Naaman being angrie departed, ſaying: I thought he would come
out to me, and ſtanding would inuocate the name of the Lord his God, and
touch with his hand the place of the leproſie, and cure me. \V What
are not Abana, and Pharphar the riuers of Damaſcus, better then al the
waters of Iſrael, that I may be washed in them, and be made cleane?
Therfore when he had turned him ſelf, and went away with indignation, \V
his ſeruantes came vnto him, & ſpake to him: Father, & if the prophet
had ſayd a great thing to thee, certes, thou shouldeſt haue done it: how
%%% o-0709
much more wheras now he ſayd vnto thee: Be washed, and thou shalt be
cleane? \V He went downe, & washed in Iordan ſeuen times according to
the word of the man of God, and his flesh was reſtored, as the flesh of
a litle childe, & he was made cleane. \V And returning to the man of God
with al his trayne, he came, and ſtood before him, & ſayd: In very deede
I know that there is no other God in al the earth, but only in Iſrael. I
beſech
\Fix{the}{thee}{obvious typo, fixed in other}
therfore to take a benediction of thy ſeruant. \V But he anſwered: Our
Lord liueth, before whom I ſtand, I wil not take it. And when he would
haue forced him, he did in nowiſe agree. \V And Naaman ſayd: As thou
wilt, but I beſech thee: graunt vnto me thy ſeruant, that I may take of
\SNote{In reſpect of Gods ſpecial electing and ſãctifying the land of
Chanaan, by his 
true religion, Naaman rightly eſtemed that earth fitter for an altar
then the earth of his owne countrie.}
the earth the burden of two mules: for thy ſeruant wil no more make
holocauſt, or victimes to ſtrange goddes, but to the Lord. \V But this
onlie is it, for which thou shalt beſech the Lord for thy ſeruant, when
my maiſter shal goe into the temple of Remmon, to adore: and he leaning
vpon my hand, if I shal adore in the temple of Remmon, he adoring in the
ſame place, that the Lord pardon me thy ſeruant for this thing. \V Who
ſayd to him:
\LNote{Goe in peace.}{Schiſmatiques,
\MNote{The caſe of going to heretical ſeruice, and Naamans going to the
temple of an idol differ in diuers reſpectes.}
as they are commonly (but improperly) now called in England, which being
in mind and iudgement Catholiques, goe ſometimes to Proteſtantes common
prayers, or ſermons, draw an excuſe of their fact, from this warrant of
the Prophet, permitting a Nobleman of Syria to goe and ſerue his king in
the temple, when he adored an Idol. But whoſoeuer wil duly conſider this
example, ſhal find great difference in reſpect of the times, places,
perſons, and of the very doubtes propoſed, betwen this mans caſe and
ours.
\MNote{Difference of times.}
For before Chriſts Goſpel was promulgate, neither al Articles of faith
were ſo expreſly taught, nor the external profeſſion therof ſo ſtrictly
commanded, as now they are in time of more grace, which geueth more ayde
to mans weaknes, wherin alſo more perfection is required, and therfore
our Sauiour exacteth of al
\CNote{\XRef{Mat.~10.}}
\Emph{to confeſſe him}, and his Religion, \Emph{before men: els he vvil
denie them before his Father}.
\MNote{Of places.}
Likewiſe in the place, where this Nobleman dwelt, his preſence in the
temple, and ſeruice to the king, could not be accounted a reuolt from
true religion, which was neuer profeſſed there, nor be ſcandalous to
anie man being al Infidels: but in a chriſtian countrie, where al beare
the name of Chriſtians, eſpecially where men are at controuerſie about
the true Chriſtian religion, al that frequent, or repaire to the ſame
aſſemblies, for publique ſeruice of God, are reputed to  be of the ſame
religion; or els diſſemblers, as it were to haue no care of religion,
\CNote{\XRef{Rom.~1.}
\XRef{2.~Pet.~1.}}
knowing God, and not glorifying him as God, and reuolting from the
truth which they had lerned.
\MNote{Of perſons.}
The difference alſo of perſons is great. For this Nobleman hauing before
his conuerſion ſerued his king, in the office of ſuſtayning him, when he
bowed to the Idol, if he ſhould haue refuſed to do the ſame, it would
rather haue bene ſuppoſed, that he diſdained his Maiſter, or ſhewed
diſloyaltie, then thought, that he refrayned for religion: wheras in
our caſe, verie few do ſuch temporal ſeruice, about the king in the
church: and ſuch as doe carrie the ſword, ſcepter, or the like, are
accounted of that religion, which is there practiſed; except they
manifeſt the contrarie, as this man did, and our men commonly do
not. Yea if anie do ſay they are Catholiques, and yet goe to the
Proteſtantes church, they are counted of that rank, S.~Paul ſpeaketh of,
which
\CNote{\XRef{Tit.~2.}}
\Emph{confeſse they knovv God, but denie him in their deedes}. And thoſe
which refuſe ſuch an office, can not be iudged diſloyal, becauſe it is
ſufficiently knowne, that Catholiques refuſe of mere conſcience.
\MNote{The thinges demanded differ much.}
An other moſt eſpecial difference is in the thinges demanded. This
Syrian promiſing expreſly before the Prophet, and his owne great trayne,
that he would neuer againe ſerue falſe goddes, and that he would ſerue
the onlie true God, and for that purpoſe caried earth with him, to make
an Altar for Sacrifice, and returning home preached the miracle wrought
in himſelf, deſired not to doe anie thing, wherby he might ſeme to ſerue
an Idol:
\CNote{\Cite{D.~Briſtons Motino.~23.}}
but that when the king leaning vpon him, ſhould adore Remmon,
he might bow with his maiſter, not adoring the Idol, for he reſolued and
promiſed the contrarie, but adoring God Almightie, in whom now he
beleued. And this the Prophet approued, in that time, place, and perſon
to be lawful.
\MNote{Perſonal preſence at heretical ſeruice in England, a diſtinctiue
ſigne of conformity to hereſie.}
But thoſe that now in England goe to Proteſtants ſeruice, or ſermons, do
neither publikly renounce al hereſies, nor profeſſe to frequent Maſſe,
the true Sacrifice of the Chriſtian Church, nor auouch the erecting of
an Altar, but goe to church, to ſhew them ſelues obedient to the
Parlament law, which abandoned the true Diuine Seruice, and in place
therof appointed & commanded al to be preſent, at a new forme of common
prayer, thereby making it a diſtinctiue ſigne of conformitie, and
participation in that religion, which theſe diſſemblers in their
conſciences know to be falſe.

This example therfore doth in no ſorte warrant their going to the
heretical church, but contrariwiſe admoniſheth al to take reſolution in
our caſe (as Naaman did in his) of our Eliſeus, or ſpiritual Superior,
and if he should ſay: Goe in peace, then might they pleade an excuſe,
but he ſayth: None can goe without incurring greuous ſinne, and eternal
damnation. The caſe being ſo much different from Naamans.
\MNote{A caſe very like to ours.}
It is in deede more like to that of Eleazarus, and other Machabees, who
were commanded
\CNote{\XRef{2.~Machab.~6.~7.}}
\Emph{by eating ſvvines flesh, to depart from the lavv of God, and their
fathers}. VVhich by no meanes was lawful to doe, nor to make ſhew of
doing it, but rather to dye, as they did moſt gloriouſly.}
Goe in peace. He therfore went from him in the ſpring time of the
earth. \V And
\SNote{Giezi prefigured Iudas the falſe Apoſtle of Chriſt, and al thoſe
that buy or ſel ſpiritual thinges for money who by their auarice loſe
Gods grace, and gaine infamie in this world, and eternal damnation in
the next.
\Cite{S.~Aug. ſer.~208. de tempore.}}
Giezi the ſeruant of the man of God, ſayd: My maiſter hath ſpared Naaman
this Syrian, that he tooke not of him the thinges which he brought: Our
Lord liueth, I wil runne after him, and wil take ſome thing of him. \V
And Giezi folowed at the backe of Naaman: whom when he ſaw running
toward him, he lept downe from his chariote to mete him, and ſaid: Are
al thinges wel? \V And
%%% 0791
he ſaid: Wel. My maiſter hath ſent me to thee, ſaying: Euen now there
are come to me two yong men from mount Ephraim, of the children of the
prophetes: geue them a talent of ſiluer, and two change of rayment. \V
And Naaman ſayd: It is better that thou take two talentes. And he forced
him, & bound the two talentes of ſiluer in two bagges, and the duble
rayment, & layd it vpon two of his ſeruantes, who alſo caried it before
him. \V And when he was come now in the euening, he tooke it out of
their hand, & layd it vp in the houſe, & diſmiſſed the men, and they
departed. \V And himſelf going in, ſtood before his maiſter. And Eliſeus
ſaid: From whence comeſt thou Giezi? Who anſwered: Thy ſeruant hath not
gone any whither. \V But he ſayd: Was not my hart preſent, when the man
returned out of his chariote to meete thee? Now therfore thou haſt
receiued ſiluer, and taken rayment, to bye oliuetes, and vineyardes, and
sheepe, and oxen, and ſeruantes, and handmaides. \V But the leproſie alſo
of Naaman shal cleaue to thee, and to thy ſeede, for euer. And he went
out from him a leper as it were ſnow.


\stopChapter


\stopcomponent


%%% Local Variables:
%%% mode: TeX
%%% eval: (long-s-mode)
%%% eval: (set-input-method "TeX")
%%% fill-column: 72
%%% eval: (auto-fill-mode)
%%% coding: utf-8-unix
%%% End:
