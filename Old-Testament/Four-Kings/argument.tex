%%%%%%%%%%%%%%%%%%%%%%%%%%%%%%%%%%%%%%%%%%%%%%%%%%%%%%%%%%%%%%%%%
%%%%
%%%% The (original) Douay Rheims Bible 
%%%%
%%%% Old Testament
%%%% Four Kings
%%%% Argument
%%%%
%%%%%%%%%%%%%%%%%%%%%%%%%%%%%%%%%%%%%%%%%%%%%%%%%%%%%%%%%%%%%%%%%




\startcomponent argument


\project douay-rheims


%%% 0780
%%% o-0699
\startArgument[
  title={\Sc{The Argvment of the Fovrth Booke of Kinges.}},
  marking={Argument of Four Kinges.}
  ]

This fourth booke proſecuteth the hiſtorie of the two Kingdomes of Iuda
and Iſrael, to the ſeueral captiuities of them both. Shewing manie
ſpecial vertues and heroical actes of good Kinges, Prophetes, and other
godlie perſons: and diuers crimes of the wicked.
\MNote{The kingdom of Iuda conſerued in Dauids ſeede.}
For in Iuda were ſome good kinges, highly commended; ſome euil, whom God
ſpared in this world for Dauids ſake. So that in both ſortes King Dauids
ſeede continued in his throne, and royal ſtate (firſt in the twelue
tribes, afterward in two) nere foure hundred fourſcore yeares. And after
the captiuitie (as wil appeare in the age enſuing) it was conſerued in
honour and eſtimation, til Chriſt our Sauiour.
\MNote{Many royal families begũne and deſtroyed in the kingdom of
Iſrael.}
But in the Kingdome of Iſrael (or tenne tribes) which ſtood about two
hundred fiftie yeares, was great change, by rayſing and extirpating
royal families. Al their kinges were bad, yet partly were ſet vp by God
himſelf, partly ſuffered to reigne; and in both Kingdomes, were true and
falſe prophetes, God vſing the miniſterie of al, to his owne glorie, the
good of his Church, and punishment of others, and ſometimes of
themſelues.
\MNote{This booke diuided into two partes.}
So this booke may be diuided into two partes. In the ſeuentene former
chapters, are recorded ioyntly and mixtly the principal thinges donne in
both kingdomes, til the captiuitie of the tenne tribes. The other eight
chapters conteine other thinges donne in Iuda, vntil their captiuitie in
Babylon.


\stopArgument


\stopcomponent


%%% Local Variables:
%%% mode: TeX
%%% eval: (long-s-mode)
%%% eval: (set-input-method "TeX")
%%% fill-column: 72
%%% eval: (auto-fill-mode)
%%% coding: utf-8-unix
%%% End:
