%%%%%%%%%%%%%%%%%%%%%%%%%%%%%%%%%%%%%%%%%%%%%%%%%%%%%%%%%%%%%%%%%
%%%%
%%%% The (original) Douay Rheims Bible 
%%%%
%%%% Old Testament
%%%% Proverbs
%%%% Chapter 16
%%%%
%%%%%%%%%%%%%%%%%%%%%%%%%%%%%%%%%%%%%%%%%%%%%%%%%%%%%%%%%%%%%%%%%




\startcomponent chapter-16


\project douay-rheims


%%% 1426
%%% o-1315
\startChapter[
  title={Chapter 16}
  ]

\Summary{}


It
\LNote{It perteineth to man to prepare the hart.}{For
\MNote{A general rule that one place of holie Scripture is not contrarie
to an other.}
the better vnderſtanding of this and other hard places of holie
Scripture, this general rule euer approued by al Chriſtians, is moſt
neceſſarie, that al holie Scripture is true and certaine, as being al
indited by the Holie Ghoſt, the Spirite of truth: and ſo one place is
neuer contrarie to an other, though at firſt ſight they may ſo ſeme to
mans vnderſtanding. For by this place the Pelagians would proue, that
man can of himſelf, without the helpe of Gods grace, beginne a good
thing, though without this helpe he can not performe it, becauſe it is
here ſayd: that it perteyneth to man to prepare the hart: that is, to
beginne a good thing.
\CNote{li.~2. c.~8. cont. duas Epiſtolas Palag.}
But S.~Auguſtin ſheweth that it may not be ſo vnderſtood: becauſe ſo it
ſhould be contrarie to that ſaying of our Sauiour:
\CNote{\XRef{Ioan.~15.}}
VVithout me you can do nothing: and that of S.~Paul:
\CNote{\XRef{2.~Cor.~3.}}
VVe are not ſufficient to thinke anie thing of our ſelues, as of our
ſelues: but our ſufficiencie is of God. VVhich tvvo euident places,
vvith other like, do ſhevv that this place hath an other different
ſenſe, from that vvhich the Pelagians gather.
\MNote{Gods grace is neceſſarie both to begine and to proſecute anie
good worke.}
And ſo S.~Auguſtin, and other Fathers teach, that the vviſman here
affirmeth not, that man of himſelf can prepare his hart, or beginne a
good vvorke, but that it perteineth to man to prepare his hart,
preſuppoſing
helpe of Gods grace, and hauing ſo begunne, God alſo gouerneth the
tongue, and by more grace directeth it to ſpeake thoſe thinges vvel,
vvhich the hart purpoſed, and diſpoſed to be vttered, vvithout vvhich
helpe none is able, neither to beginne anie meritorious vvorke, nor to
proſecute, nor perfect that is vvel begunne. The like ſentence
foloweth in the
\XRef{9.~verſe.}
The hart of man diſpoſeth his vvay: but it perteyneth to our Lord to
direct his progreſſe: ſignifying in both places, that after a thing is
vvel begunne (vvhich can not be vvithout the helpe of Gods grace) yet it
can not procede vvel, vvithout more grace, ſtil directing
and ſtreingthning mans freevvil.}
\MNote{Gods grace neceſſarie in euerie good action.}
perteyneth to man to prepare the hart: and to our Lord to gouerne the
tongue.

\V
\MNote{Mans iudgement is not ſecure.}
Al the wayes of man are open to his eies: our Lord is the weigher of
ſpirites.

\V
\MNote{Commend thyne affayres to God.}
Reueale thy workes to our Lord: and thy cogitations shal be directed.

\V
\MNote{Gods prouidence.}
Our Lord hath wrought al thinges for himſelf: the impious alſo to the
euil day.

\V
\MNote{Puniſhment of ſinne.}
Euerie arrogant man is an abomination to our Lord: although hand shal be
to hand he is not
\TNote{vnpunished}
innocent.

The
\MNote{Equitie.}
beginning of a good way, is to doe iuſtice: and it is more acceptable
with God, then to immolate hoſtes.

%%% 1427
\V
\MNote{Mercie.}
By mercie and truth iniquitie is redemed: and in the feare of our Lord
euil is auoided.

\V
\MNote{Deuotion.}
When the wayes of man shal pleaſe our Lord, he wil conuert alſo his
enemies to peace.

\V
\MNote{Iuſt gaine.}
Better is a litle with iuſtice, then much fruite with iniquitie.

\V
\MNote{Neceſſitie of Gods grace.}
The hart of man diſpoſeth his way: but it perteyneth to our Lord to
direct his progreſſe.

\V
\MNote{God aſſiſteth ſuperiors in gouerning their ſubiectes.}
Diuination is in the lippes of the king, his mouth shal not erre in
iudgement.

\V
\MNote{Iuſt balance do pleaſe God & good kinges.}
Weight and balance are iudgements of our Lord: and his worke al the
ſtones of the bagge.

\V
They are abominable to the king that doe impiouſly: becauſe the throne
is eſtablished by iuſtice.

%%% o-1316
\V
\MNote{Righteouſnes.}
The wil of kinges are iuſt lippes: he that ſpeaketh right thinges shal
be beloued.

\V
\MNote{Feare and reuerence of authoritie.}
The kings indignation, meſſengers of death: and the wiſe man wil pacifie
it.

\V In the cherfulnes of the kings countenance is life: and his clemencie
is as the later showre.

\V
\MNote{Loue of wiſdom.}
Poſſeſſe wiſdom, becauſe it is better then gold: and gette prudence,
becauſe it is more precious then ſiluer.

\V
\MNote{Iuſtice in general.}
The path of the iuſt auoideth euils: the keper of his ſoule kepeth his
way.

\V
\MNote{Humilitie.}
Pride goeth before deſtruction, and before ruine the ſpirit shal be
exalted.

\V
\MNote{Meknes.}
It is better to be humbled with the meeke, then to diuide ſpoyles with
the proude.

\V
\MNote{Hope in God.}
The lerned in word shal finde good thinges: and he that hopeth in our
Lord, is bleſſed.

\V
\MNote{Mildnes.}
He that is wiſe in hart, shal be called prudent: and he that is ſweete
in ſpeach shal finde greater thinges.

\V
\MNote{Teaching others.}
A fountaine of life the lerning of him that poſſeſſeth it: the doctrine
of fooles foolishnes.

\V
\MNote{Sincere hart.}
The hart of the wiſe shal inſtruct his mouth: and shal adde grace to his
lippes.

\V
\MNote{Swetnes in conuerſation.}
Wel ſet wordes are a honie combe: ſwetnes of the ſoule the health of the
bones.

\V
\CNote{\XRef{ch.~14. v.~2.}}
\MNote{True faith & Religion.}
There is a way that ſeemeth to a man right: and the later endes therof
lead to death.

\V
\MNote{Proper induſtrie.}
The ſoule of him that laboureth doth labour to himſelf, becauſe his
mouth hath compelled him.

%%% 1428
\V
\MNote{Charitie.}
The impious man diggeth euil, and in his lippes fire burneth.

\V
\MNote{Common good.}
A peruerſe man raiſeth contentions: and one ful of wordes ſeparateth
princes.

\V
\MNote{True freindſhipe.}
An vniuſt man allureth his frende: and leadeth him by a way not good.

\V
\MNote{Sincere thoughts.}
He that with
\Fix{aſtoinied}{aſtonished}{obvious typo, fixed in other}
eies thinketh wicked thinges, byting his lippes bringeth euil to paſſe.

\V
\MNote{Holie old age.}
A crowne of dignitie old age, which shal befound in the wayes of
iuſtice.

\V
\MNote{Patience.}
Better is the patient then a ſtrong man: and he that ruleth his minde,
then the ouerthrower of cities.

\V
\MNote{Gods prouidence.}
Lottes are caſt into the boſome, but they are ordered of our Lord.


\stopChapter


\stopcomponent


%%% Local Variables:
%%% mode: TeX
%%% eval: (long-s-mode)
%%% eval: (set-input-method "TeX")
%%% fill-column: 72
%%% eval: (auto-fill-mode)
%%% coding: utf-8-unix
%%% End:


  
