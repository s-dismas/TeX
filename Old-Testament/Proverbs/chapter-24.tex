%%%%%%%%%%%%%%%%%%%%%%%%%%%%%%%%%%%%%%%%%%%%%%%%%%%%%%%%%%%%%%%%%
%%%%
%%%% The (original) Douay Rheims Bible 
%%%%
%%%% Old Testament
%%%% Proverbs
%%%% Chapter 24
%%%%
%%%%%%%%%%%%%%%%%%%%%%%%%%%%%%%%%%%%%%%%%%%%%%%%%%%%%%%%%%%%%%%%%




\startcomponent chapter-24


\project douay-rheims


%%% 1439
%%% o-1327
\startChapter[
  title={Chapter 24}
  ]

\Summary{}


Emulate
\MNote{Flee euil companie, leſt thou be alured to vice.}
not euil men, neither deſire thou to be with them: \V becauſe their mind
doth meditate robberies, and their lippes ſpeake deceites.

\V
\MNote{VViſdom and vertues, not wickednes, do proſper temporally and
ſpiritually.}
By wiſedom the houſe shal be built, and by prudence it shal be
ſtrengthened.

\V In doctrine the cellars shal be replenished with al precious, and
moſt beautiful ſubſtance.

\V A wiſeman is ſtrong: and a lerned man, ſtrong and valiant.

\V
\MNote{Counſel in warres, and other great affayres.}
Becauſe warre is managed by due ordering: & there shal be ſaluation
where manie counſels are.

\V Wiſedom is high for a foole, in the gate he shal not open his mouth.

\V
\MNote{Good purpoſes.}
He that thinketh to doe euils, shal be called a foole.

\V
\MNote{Report wil of others.}
The cogitation of a foole is ſinne: and a detracter the abomination of
men.

\V
\MNote{Fortitude.}
If thou deſpaire being wearie in the day of diſtreſſe: thy ſtrength shal
be diminished.

\V
\MNote{VVorkes of mercie, according to our abilitie.}
Deliuer them that are led to death: and thoſe that are drawen to death
ceaſe not to deliuer.

\V If thou ſay: I am not of force: he that ſeeth into the hart, he
vnderſtandeth, and nothing deceiueth the keeper of thy ſoule, and he
shal render to a man according to his workes.

\V
\MNote{VViſdom is ſwete:}
Eate honie my ſonne, becauſe it is good, and the honiecombe moſt ſweete
to thy throte: \V
\MNote{and geueth hope.}
ſo alſo the doctrine of wiſedom to thy ſoule: which when thou shalt
finde, thou shalt haue hope in the later end, and thy hope shal not
perish.

%%% o-1328
\V
\MNote{Toleration of others imperfections:}
Lie not in wayte, nor ſeeke impietie in the houſe of the iuſt, nor
ſpoile his reſt.

%%% 1440
\V
\MNote{vvithout which none liueth.}
For
\LNote{Seuen times shal the iuſt fal.}{A
\MNote{A iuſt man falling into venial ſinnes is not therby vniuſt, nor
Gods enimie.}
iuſt man, that is to ſay, Gods true ſeruant, free from mortal ſinne, is
ſubiect during this life, to manie tentations, imperfections, and may
often fal into venial ſinnes, and not loſe iuſtice, nor the true title
of a iuſt man (as here he is called) nor become the diuels ſeruant, nor
Gods enemie: but through Gods grace helping his weaknes, he riſeth
againe from ſmal ſinnes, ſtil perſeuering in Gods fauoure:
wheras contrariwiſe the impious falleth into euil, to witte, into more
and more ſinne, through malice, and lacke of grace, & riſeth not ſo
eaſily. And therfore the wiſman here admoniſheth, not to lie in waite,
nor calumniouſly to ſeke impietie in the houſe (that is, in the ſoule)
of the iuſt.
\MNote{Al ſinnes are not mortal.}
For though he committe ſome faultes, yet he riſeth againe, and is not
impious, vniuſt, nor guiltie of mortal crime, as the wicked man is.}
ſeuen
\TNote{often-times}
times shal the iuſt fal, and shal riſe againe: but the impious shal fal
into euil.

\V
\CNote{\Cite{S.~Aug. li.~11. c.~31. ciuit.}}
\MNote{Charitie towards enemies.}
When thine enemie shal fal, be not glad, and in his ruine let not thy
hart reioyce: \V Leſt perhaps our Lord ſee, and it diſpleaſe him, and he
take away his wrath from him.

\V
\MNote{Haue peace with al:}
Contend not with the moſt wicked, nor emulate the impious: \V
\MNote{ſo much as may be.}
becauſe euil men haue not hope of thinges to come, and the lampe of the
impious shal be extinguished.

\V
\MNote{Loialtie to God & king.}
Feare our Lord, my ſonne, and the king: & with detracters medle not: \V
becauſe their perdition shal ſodenly riſe: and the ruine of both who
knoweth?

\V
\MNote{Equitie in iudgement, condemning the guiltie, & deliuering the
innocent, is very gratful to al.}
Theſe thinges alſo to the wiſe: to know a perſon in iudgement is not
good.

\V They that ſay to the impious: Thou art iuſt: peoples shal curſe them,
and tribes shal deteſt them.

\V They that rebuke him, shal be prayſed: and bleſſing shal come vpon
them.

\V He shal kiſſe the lippes, who anſwereth right wordes.

\V
\MNote{Order in al affaires.}
Prepare thy worke abrode, and diligently til thy ground: that afterward
thou mayſt build thy houſe.

\V
\MNote{Diſcretion in bearing witnes.}
Be not witnes without cauſe againſt thy neighbour: neither alure any man
with thy lippes.

\V
\MNote{Reuenge not.}
Say not: As he hath done to me, ſo wil I doe to him: I wil render to
euerie one according to his worke.

\V
\MNote{Diligent labour, and vigilance to prouide neceſſaries, and to
auoide beggerie.}
I paſſed by the filde of a ſlothful man, and by the vineyard of a foolish
man: \V and behold nettels had filled it wholy, and thornes had couered
the face therof, and the wal of ſtones was deſtroyed.

\V Which when I had ſeene, I layd it in my hart, and by the example I
lerned diſcipline.

\V A litle I ſay, thou shalt ſleepe, a litle thou shalt ſlumber, a litle
thou shalt ioyne thy handes together, to reſt: \V and as a poſte,
pouertie shal come to thee, & beggerie as a man armed.


\stopChapter


\stopcomponent


%%% Local Variables:
%%% mode: TeX
%%% eval: (long-s-mode)
%%% eval: (set-input-method "TeX")
%%% fill-column: 72
%%% eval: (auto-fill-mode)
%%% coding: utf-8-unix
%%% End:


  
