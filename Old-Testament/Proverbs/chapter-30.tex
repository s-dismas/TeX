%%%%%%%%%%%%%%%%%%%%%%%%%%%%%%%%%%%%%%%%%%%%%%%%%%%%%%%%%%%%%%%%%
%%%%
%%%% The (original) Douay Rheims Bible 
%%%%
%%%% Old Testament
%%%% Proverbs
%%%% Chapter 30
%%%%
%%%%%%%%%%%%%%%%%%%%%%%%%%%%%%%%%%%%%%%%%%%%%%%%%%%%%%%%%%%%%%%%%




\startcomponent chapter-30


\project douay-rheims


%%% 1448
%%% o-1336
\startChapter[
  title={Chapter 30}
  ]

\Summary{A
\MNote{The fourth part.

Other ſingular precepts, with prayſe of a prudent woman.}
right wiſeman thinketh humbly of himſelf, 4.~knowing that Gods workes
are inſcrutable, and perfect: 8.~deſireth truth in al thinges, &
mediocritie in riches. 11.~Abhorreth certaine ſortes of men, 15.~&
certaine execrable thinges. 18.~Noteth certaine thinges hard to be
knowen: 21.~other thinges intolerable: 24.~others admirable. 32.~The
tongue dangerous.}


The
\LNote{The vvordes of the Gatherer.}{Some
\MNote{Some ſuppoſe one \HH{Agur} to be auctor of this chapter.}
Interpreters take theſe foure Hebrew wordes, \HH{Agur}, \HH{Iache},
\HH{Ithiel}, and \HH{Vcal}, conteined in this firſt verſe, to be proper
names of men, ſuppoſing that a certaine wiſeman named \HH{Agur}, the
ſonne of \HH{Iache}, ſpake the ſentences folowing in this chapter, to
his ſonnes or ſcholars, called \HH{Ithiel} & \HH{Vcal}. And ſo this
ſuppoſed \HH{Agur}, not Salomon, ſhould be the auctor of this
chapter.
\MNote{But it ſemeth more probable that only Salomon is auctor of this
whole booke.}
But the old Interpreter, whom S.~Ierom approueth and foloweth,
tranſlated the ſame wordes as noones appellatiues. Neither doth anie
ancient Father account this \HH{Agur}, amongſt the writers of holie
Scriptures. And if there were a peculiar auctor of this chapter, it is
like the ſame ſhould haue bene placed laſt, and not before that which now
foloweth, and is by al men confeſſed to be Salomons. And therfore we
thinke it more probable; with S.~Beda, and the common opinion, that
there vvas no other auctor of anie part of this booke, beſides King
Salomon. VVho is here called \Sc{Congregans}, the Gatherer,
\MNote{VVhy he is called Gatherer.}
becauſe he gathered theſe excellent Parables, and Prouerbes; as the
ſonne of the Holie Ghoſt, ſignified by the word \Sc{Iache}, povvring
forth diuine ſentences, for inſtruction of \Sc{Ithiel} & \Sc{Vcal}, that
is, of al thoſe vvith vvhom God is by his grace, and vvho are
ſtreingthened by God abiding vvith them.}
wordes of the Gatherer the ſonne of Vomiter. The viſion, that the man
ſpake, with whom God is, and who being ſtrengthened by God abiding with
him, ſayd: \V I am
\SNote{The wiſeſt man beſt knoweth that he wanteth much of perfect
wiſdom yet in his humilitie ſuppoſeth, that others haue attained ſome
what more then himſelf.}
moſt foolish of men, & the wiſedom of men is not with me. \V I haue not
learned wiſedom, and haue not knowen the ſcience of ſaints. \V
\SNote{Chriſt the Sonne of God is wiſdom it ſelf, and as the Sonne of
man hath perfect wiſdom.}
Who hath aſcended into heauen and deſcended? who hath conteyned the
ſpirit in his handes? who hath bound the waters together as in a
garment? who hath rayſed vp al the borders of the earth? what is his name,
and what is the name of his ſonne, if thou know? \V Euerie word of God
tryed by fyre, is a buckler to them that hope in him. \V Adde not any
thing to his wordes, and ſo thou be reproued and found a lyer. \V Two
thinges I haue asked thee, denie them not to me before I dye. \V
Vanitie, and lying wordes make far from me. Beggerie, and riches geue me
not: geue only things neceſſarie for my ſuſtenance: \V leſt perhaps
being filled I be allured to denie, and may ſay: Who is the Lord? or
being compelled by pouertie I may ſteale, and forſweare the name of my
God. \V
\SNote{Bondſlaues are to be pitied, and not affliction added to the
afflicted.}
Accuſe not a ſeruant to his maſter, leſt perhaps he curſe thee, and thou
fal.

\V There is
\SNote{Foure execrable vices:

Ingratitude.

Hypocriſie.

Inſolencie.

Oppreſſion of the poore.}
a generation that curſeth their father, and that bleſſeth not their
mother. \V A generation, that ſemeth to itſelf cleane, & yet is not
washed from their filthines. \V A generation, whoſe eies are loftie, and
the eielids therof ſet vp on high. \V A generation, that for teeth hath
ſwordes, and chaweth with theyr grinding teeth, that they may eate the
needie out of the earth, and the poore from among men.

%%% 1449
\V The horſeleach hath
\SNote{Cõcupiſcence of the fleſh, & of the eyes.}
two daughters that ſay: Bring, bring. Three things are vnſatiable, the
fourth neuer ſayth it ſufficeth. \V
\SNote{Enuie, Luxurie, Auarice, & Ambition.}
Hel, and the mouth of the matrice, & the earth which is not ſatisfied
with water: but
%%% !!! Extra SNote?
%%% \SNote{}
the fyre neuer ſayth it ſufficeth. \V
\SNote{Diſhonour of parents shal be ſeuerely puniſhed.}
The eie, that ſcorneth his father, & that deſpiſeth the trauail of his
mother, in bearing him, let the rauens of the torrents pick it out, and
the young of the eagle eate it.

\V Three thinges are hard to me, and of the fourth I am vtterly
ignorant. \V The way of an eagle in the ayre, the way of a ſerpent vpon
a rocke, the way of a shippe in the middes of the ſea, and
\SNote{Youngmen folowing carnal appetite, can no more geue account of
their actiõs, then of the vvayes vvhich an eagle, a ſerpent, and a
ſhippe haue paſſed.}
the way of a man in youth. \V Such is alſo the way of an adulterous
woman, which eateth, and wyping her mouth ſayth: I haue done no euil.

%%% o-1337
\V By three thinges the earth is moued, and the fourth it can not
ſuſteyne. \V By a ſeruant when he shal reigne: by a foole when he shal
be filled with meate: \V By an odious woman when she shal be taken in
matrimonie: & by
\LNote{A bondvvoman vvhen she shal be heyre.}{Of
\MNote{VVhy God ſuffereth hereſie to reigne.}
al thinges in this world, it ſemeth moſt abſurde, that hereſie doth
dominiere ouer Catholique religion: vvhich God ſome times, and in ſome
places ſuffereth, for the greater merite of his elect.}
a bondwoman when she shal be heyre to her miſtreſſe.

\V There are
\SNote{By theſe examples are commended foure vertues:

Induſtrie,

Prudence,

Concord,

Humilitie.}
foure the leaſt thinges of the earth, and they are wiſer then the
wiſe. \V The antes, a weake people, which prepareth in the harueſt meate
for themſelues. \V The leueret, a people not ſtrong, which placeth his
bed in the rocke. \V The locuſt hath no king, and they go out al by
their troopes. \V The ſtellion ſtayeth on his handes, & tarieth in kings
houſes. \V
\SNote{Other foure:

Fortitude,

Chaſtitie,

Order,

Iuſtice.}
There are three thinges which go wel, and the fourth that goeth
happely. \V The lyon, the ſtrongeſt of beaſtes shal feare at the meeting
of none: \V the cocke gyrded about the loines, and the ramme: alſo the
king, againſt whom none can reſiſt.

\V There is that
\SNote{Fooles ought not to gouerne.}
hath appeared a foole after that he was lifted vp on high: for if he had
vnderſtood, he would haue layd his hand vpon his mouth. \V And he that
\SNote{Moderation is neceſſarie in al actions.}
ſtrongly preſſeth the pappes to wring out milke, ſtrayneth out butter,
and he that violently
\Fix{cleanceth}{cleanſeth}{likely typo, fixed in other}
his noſe, wringeth out bloud: & he that prouoketh angers, bringeth forth
diſcordes.


\stopChapter


\stopcomponent


%%% Local Variables:
%%% mode: TeX
%%% eval: (long-s-mode)
%%% eval: (set-input-method "TeX")
%%% fill-column: 72
%%% eval: (auto-fill-mode)
%%% coding: utf-8-unix
%%% End:


  
