%%%%%%%%%%%%%%%%%%%%%%%%%%%%%%%%%%%%%%%%%%%%%%%%%%%%%%%%%%%%%%%%%
%%%%
%%%% The (original) Douay Rheims Bible 
%%%%
%%%% Old Testament
%%%% Proverbs
%%%% Third Part
%%%%
%%%%%%%%%%%%%%%%%%%%%%%%%%%%%%%%%%%%%%%%%%%%%%%%%%%%%%%%%%%%%%%%%




\startcomponent third-part


\project douay-rheims


%%% 1402
%%% o-1292
\startArgument[
  title={\Sc{The Third Part of the Old Testament, Conteining Sapiential Bookes.}},
  marking={The Third Part of the Old Testament.}
  ]

\Sc{The Argument of Sapiential Bookes}


Hitherto
\MNote{The coherence of this part with the reſt.}
\Emph{the Law}, and \Emph{Hiſtorie of Gods peculiar people} are ſet
forth in the former partes of the holie Bible: after which folowed the
\Emph{Booke of Pſalmes}, which in maner of ſtile, being al in verſe, is
a diſtinct part, but in ſubſtance of matter, is \Emph{an Epitome} or
briefe Summe \Emph{of al holie Scripture}: moſt conueniently therfore
placed in the middes of the reſt, as the Sunne amongſt other Planetes, a
shining great light in a large houſe. Now enſueth the third part,
conteining \Emph{Diuine Inſtructions}, or \Emph{Rules of good life}. A
doctrine moſt agreable to Gods hiegh wiſdom, and moſt fitly commended to
Man, his reaſonable creature in earth.
\MNote{The contents of Sapiential bookes.}
But beſides this principal ſubiect,
\CNote{\XRef{Preface before Ioſue.}}
as before is noted (that each part participateth with others in their
proper contents) ſo here be manie \Emph{precepts of the Law renewed};
ſundrie \Emph{examples} of men, and thinges paſt \Emph{repeted}, and
diuers \Emph{prophecies vttered} of thinges to come: though in this part
more ſpecially is shewed \Emph{the ground}, and as it were, the very
\Emph{life or ſoule of the Law}, which is \Emph{Reaſon}, the true Rule
or Directorie wherin al good lawes are grounded.

For it both sheweth what ought to be done, or auoided, & directeth mans
iudgement to embrace that is good, and to flee from al euil, not only
illuminating the vnderſtanding to ſee that is right and iuſt, but alſo
diſpoſing the internal affection to deſire, loue, chooſe, and preferre
the right path of Gods law, before whatſoeuer otherwiſe ſemeth pleaſant
or profitable: & ſo, notwithſtanding al dangers, difficulties,
diſtreſſes, worldlie calamities, and death itſelf, effectually
perſwading to perſeuere to the end in holie conuerſation.
\MNote{Why they are ſo called.}
Al which by a general name is called \Emph{Wiſdom: compriſing} in one
word, \Emph{al good deſires, holie vertues, ſupernal giftes, godlie
endeuoures, and the whole meanes wherby God is rightly knowen, & duly
ſerued}; wherof theſe fiue \Emph{Bookes}, teaching this moſt excellent
and moſt neceſſarie maner of life, \Emph{are called
Sapiential}. Neuertheles foure of them haue alſo other particular names,
as
\Fix{appareth}{appeareth}{obvious typo, fixed in other}
in their titles. Only the fourth is called the \Emph{Booke of Wiſdom},
by appropriation of the general name.

%%% 1403
%%% o-1293
\Emph{Al fiue}
\MNote{They are al Canonical Scripture.}
are \Emph{Canonical} and aſſured holie \Emph{Scripture}: as is shewed
\CNote{\XRef{Proem. Annot. Prefac. Tobiæ.}}
before: and may be further proued of the two later, which Proteſtants
denie.
\MNote{Salomon is auctor of the three firſt.}
It is alſo euident that King \Emph{Salomon} was \Emph{Auctor of
the three former}: as
\CNote{\Cite{S.~Iero, in Proem.}}
S.~Ierom.
\CNote{\Cite{S.~Aug. li.~17. c.~20. Ciuit.}}
S.~Auguſtin, and other Fathers proue by the holie text it ſelfe. As it
is likewiſe certaine that he either writte, or at leaſt by diuine
inſpiration vttered,
\MNote{Other bookes of Salomon not extant.}
much more then is now extant. For the holie Scripture
\XRef{(3.~Reg.~4.)}
teſtifieth, that he ſpake \Emph{three thouſand Parables}: and
\Emph{his Songes} were \Emph{a thouſand and fiue}. He diſputed of the
trees from the ceder that is in Libanus, vnto the hyſſop which cometh
out of the wal: and he diſcourſed of beaſtes, and foules, and creeping
wormes, and fishes. Ioſephus
\Cite{(li.~8. c.~2. Antiq.)}
folowing ſome other Edition, ſaith his ſonges were fiue thouſand, and
parables (as the ordinarie text hath) three thouſand. For he deduced a
parable (ſaith Ioſephus) throughout euerie kinde of trees, from the
hyſſop to the ceder. In the ſame maner he treated of beaſtes, and other
liuing creatures of the earth, water, and ayre. For he was not ignorant
of anie natural thing, neither omitted to treate therof, but clerly
explicated al their natural proprieties. Moſt briefly
\CNote{\Cite{Prologo galeato.}}
S.~Ierom declareth both the Auctor, and matter of theſe three bookes,
ſaying: \Emph{Salomon the Peaceable, and amiable of our Lord}
\MNote{A brief ſumme of theſe three:

a Prouerbs,

b Eccleſiaſtes,

c Canticles.}
a \Emph{correcteth maners}: b \Emph{teacheth the nature} (of creatures)
c \Emph{ioyneth the Church and Chriſt; and ſingeth the ſwete bridal ſong
of the holie Mariage.}


\stopArgument


\stopcomponent


%%% Local Variables:
%%% mode: TeX
%%% eval: (long-s-mode)
%%% eval: (set-input-method "TeX")
%%% fill-column: 72
%%% eval: (auto-fill-mode)
%%% coding: utf-8-unix
%%% End:
