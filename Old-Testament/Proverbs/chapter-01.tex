%%%%%%%%%%%%%%%%%%%%%%%%%%%%%%%%%%%%%%%%%%%%%%%%%%%%%%%%%%%%%%%%%
%%%%
%%%% The (original) Douay Rheims Bible 
%%%%
%%%% Old Testament
%%%% Proverbs
%%%% Chapter 01
%%%%
%%%%%%%%%%%%%%%%%%%%%%%%%%%%%%%%%%%%%%%%%%%%%%%%%%%%%%%%%%%%%%%%%




\startcomponent chapter-01


\project douay-rheims


%%% 1404
%%% o-1294
\startChapter[
  title={Chapter 01}
  ]

\Summary{Parables
\MNote{The firſt part.

An inuitation to ſeeke vviſdom: vvith ſome general precepts.}
are profitable to thoſe that loue and wil lerne
  wiſdom. 10.~Al are admonished not to folow the alurements of ſinners:
  20.~but to embrace wiſdome; 24.~and ruine is threatned to the
  contemners.}

The Parables of Salomon, the ſonne of Dauid, king of Iſrael. \V
\SNote{By theſe ſentencious ſimilitudes the ſtudious may better conceiue
and vnderſtand true vviſdom, and the vertues belonging therto.}
To know
\LNote{VViſdom.}{As
\MNote{Three kindes of vviſdom.}
wel in theſe Sapiential bookes, as in other holie Scriptures and ſacred
writers, the vvord vviſdom hath three ſignifications. Sometimes it
importeth the Diuine Attribute called Gods wiſdom; ſometimes
ſupernatural wiſdom geuen to men by the Holie Ghoſt; and ſometimes it
ſignifieth mere humane vviſdom, gotten by the natural light of reaſon
and mans induſtrie.
\MNote{Diuine Attributes are not qualities in God, but his ſubſtance.}
The firſt, as likevviſe other Diuine Attributes, Gods Povvre, Goodnes,
Iuſtice, Truth, Mercie and the like, are not qualities, or other
accidents in God, as the ſame termes ſignifie in creatures. For in God
there is no Accident, but al in him is
\Fix{this}{his}{obvious typo, fixed in other}
Diuine Subſtance and Eſſence, vvhoſe diuers Excellences are called by
ſuch names as mans capacitie can better conceiue:
\MNote{VViſdom increated is God himſelfe.}
and ſo Gods vviſdom is God himſelfe; and is appropriated to the ſecond
Perſon of the bleſſed Trinitie, as Povvre is appropriated to God the
Father, and Goodnes to the Holie Ghoſt. In this ſenſe:
\XRef{chap.~3 v.~16.}
is ſaide: \Emph{Our Lord by vviſdom founded the earth &c.}
\MNote{VViſdom the gifte of the Holie Ghoſt.}
The ſecond is called
\XRef{(Sap.~3. v.~25.)}
\Emph{the vapore of Gods povvre, and a pure emanation} (or influence)
\Emph{of the glorie of Almightie God}, and ſo is a participation of Diuine
increated wiſdom called alſo diuine, according to a certaine anologie,
or ſimilitude of Gods owne wiſdom, and is the principal gifte of the
Holie Ghoſt, by vvhich God is rightly knovvne, and duly ſerued,
including al other ſupernal giftes and vertues, vvherof is treated in
theſe bookes, and ſo vvhich al men are inuited, vvith aſſured promiſe of
celeſtial and eternal revvard.
\MNote{Humane vviſdom.}
The third vviſdome is mere humane, gotten by natural vvitte and ſtudie,
ſuch as Philoſophers haue, knovving manie truthes, but mixt vvith manie
errors, and much ignorance, truly called vvorldlie vviſdom, ſeruing only
for this vvorld. But the ſecond kind, vvhich is as
\Fix{aſparlecle}{a ſparkle}{obvious typo, fixed in other}
of Gods vviſdom, maketh men, othervviſe ignorant and of ſmal capacitie,
rightly vviſe in dede, the true ſeruants of God, and enheriters of the
kingdom of heauen, as theſe bookes do moſt copiouſly teach.}
wiſdom, and diſcipline: \V to vnderſtand the wordes of prudence: and to
receiue inſtruction of doctrine, iuſtice, and iudgement, and equitie: \V
that
\SNote{Profound and ſolide vvitte.}
subtilitie may be geuen to litle ones, knowlege and vnderſtanding to the
youngman. \V The
\SNote{Not only yongmen and inexperienced but alſo the vviſe may
lerne more vviſdom by theſe parables.}
wiſe man hearing shal be wiſer: and he that vnderſtandeth,
\SNote{Shal be fitte to gouerne others.}
shal poſſeſſe gouernementes. \V He shal vnderſtand a parable, and
interpretation, the wordes of the wiſe, and their darke ſayings. \V
\SNote{Feare of our Lord, that is, reuerence of his diuine Maieſtie
vvith deſire duly to ſerue him, and neuer to offend him, is the firſt degree
in aſcending to perfect vviſdom: vvhich conſiſteth not only in the
vnderſtanding but alſo in action.}
The feare of our Lord is the begynning of wiſdom. Fooles deſpiſe wiſedom
and doctrine. \V My ſonne,
\SNote{The firſt precept is to lerne of our elders.}
heare the diſcipline of thy father, and leaue not the lawe of thy
mother: \V that grace may be added to thy head, and a cheyne of gold to
thy necke. \V My ſonne,
\SNote{The ſecond to reſiſt euil ſuggeſtions.}
if ſinners shal entiſe thee, condeſcend not to them. \V If they shal
ſay: Come with vs, let vs lye in waite for bloud, let vs hide ſnares
againſt the innocent without cauſe: \V let vs swalow him aliue as hel,
and whole as one deſcending into the lake. \V We shal finde al precious
ſubſtance, we shal fil our houſe with ſpoiles. \V Caſt in thy lot with
vs, let there be one purse of vs al. \V My ſonne, walke not with them,
ſtay thy foote from their pathes. \V For their feete runne to euil, and
make haſte to shede bloud. \V But
\SNote{The proper remedie againſt ſuch alurements is to be vvatchful,
and to flee from them.}
a nette is caſt in vayne before the eies of them that haue winges. \V
Themſelues alſo lye in wayte againſt their owne
%%% 1405
bloud & practiſe deceites againſt their owne ſoules. \V So the pathes of
euerie couetous man, take violently the ſoules of the poſſeſſors. \V
Wiſdom preacheth
%%% o-1295
abrode, she geueth her voice in the ſtreates. \V In the head of
multitudes she cryeth, in the doores of the gates of the citie she
vttereth her wordes, ſaying: \V O children how long doe you loue
infancie, and fooles couer thoſe thinges, which are hurtful to them ſelues, and
the vnwiſe hate knowlege? \V Turne ye at my correption: behold I wil
vtter my ſpirite to you, and wil shewe you my wordes. \V
\LNote{Becauſe I called and you refuſed.}{God
\MNote{Four benefites of God:

Vocation,

Helpe,

Inſtruction,

Reprehenſion.}
voutſaffeth foure benefites of grace to euerie man, al neceſſarie and
ſufficient for his ſaluation: 1.~He calleth al by preaching, or good
inſpiration. 2.~He offereth helpe. 3.~He inſtructeth the ignorant what is
good, that they may chooſe it if they wil. 4.~And reprehendeth euil, that
they may ſhunne it. They therfore that neglect this manifold grace in
this life, ſhal without al remedie be damned, being to late to repent in
an other world. For then they ſhal crie and not be heard.
\XRef{v.~28.}}
Becauſe I called, and you refuſed: I ſtretched out my hand, and there
was none that regarded. \V You haue deſpiſed al my counſel, and haue
neglected my reprehenſions. \V I alſo wil laugh in your deſtruction, and
wil ſcorne, when that shal come to you, which you feared. \V When ſoden
calamitie shal fal on you, and deſtruction, as a tempeſt shal be at
hand: when tribulation, and diſtreſſe shal come vpon you. \V Then shal
they inuocate me, and I wil not heare: in the morning shal they ariſe,
and shal not finde me: \V for that they haue hated diſcipline, and not
receiued the feare of our Lord, \V nor conſented to my counſel, &
detracted from al my correption. \V They shal eate therfore the fruites
of their way, and shal be filled with their owne counſels. \V The
auerſion of litle ones shal kil them, and the proſperitie of fooles shal
deſtroy them. \V
\LNote{But he that shal heare me.}{Contrariwiſe
\MNote{Reward of workes.}
thoſe that accept Gods grace, and cooperate therwith, ſhal haue eternal
reſt and ioy. The very ſame, which S.~Paul teacheth,
\XRef{2.~Cor.~5. v.~10.}
Euerie one ſhal receiue the proper thinges of the bodie, according as he
hath done, either good or euil.}
But he that shal heare me, shal reſt without terrour, and shal enioy
abundance, feare of euils being taken away.


\stopChapter


\stopcomponent


%%% Local Variables:
%%% mode: TeX
%%% eval: (long-s-mode)
%%% eval: (set-input-method "TeX")
%%% fill-column: 72
%%% eval: (auto-fill-mode)
%%% coding: utf-8-unix
%%% End:


  
