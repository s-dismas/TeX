%%%%%%%%%%%%%%%%%%%%%%%%%%%%%%%%%%%%%%%%%%%%%%%%%%%%%%%%%%%%%%%%%
%%%%
%%%% The (original) Douay Rheims Bible 
%%%%
%%%% Old Testament
%%%% Proverbs
%%%% Chapter 29
%%%%
%%%%%%%%%%%%%%%%%%%%%%%%%%%%%%%%%%%%%%%%%%%%%%%%%%%%%%%%%%%%%%%%%




\startcomponent chapter-29


\project douay-rheims


%%% 1446
%%% o-1334
\startChapter[
  title={Chapter 29}
  ]

\Summary{}


The
\MNote{Loue to be corrected.}
man, that with ſtiffe necke contemneth him that rebuketh, ſoden
deſtruction shal come vpon him: and health shal not folow him.

\V
\MNote{Chooſe godlie magiſtrates.}
In the multiplication of iuſt men, the common people shal reioyce: when
the impious shal take princedom, the people shal mourne.

%%% 1447
\V
\MNote{Loue wiſdom.}
A man that loueth wiſedom, maketh his father glad: but he that
maintaineth harlots, shal deſtroy his ſubſtance.

\V
\MNote{Iuſt and liberal magiſtrates.}
A iuſt king ſetteth vp the land, a couetous man shal deſtroy it.

\V
\MNote{Sincere amitie.}
A man, that with fayre, and fayned wordes ſpeaketh to his frend,
ſpreadeth a nette to his ſteppes.

\V
\MNote{Iuſt ioy of the wickeds fal.}
A ſnare shal intangle the wicked man ſinning: and the iuſt shal praiſe
and reioyce.

\V
\MNote{Compaſſion of the poore.}
The iuſt knoweth the cauſe of the poore: the impious is ignorant of
knowlege.

\V
\MNote{Care of common good.}
Peſtilent men diſſipate a citie: but the wiſe turne away furie.

\V
\MNote{Contend not with a foole.}
A wiſe man, if he contend with a foole, whether he be angrie, or whether
he laugh, shal not finde reſt.

\V
\MNote{Defend the iuſt.}
Men of bloud hate the ſimple: but iuſt men ſeeke his ſoule.

\V
\MNote{Diſcretion in anger.}
A foole vttereth al his ſpirit: a wiſeman differreth, and reſerueth til
afterward.

%%% o-1335
\V
\MNote{Deteſtation of lying.}
A prince that gladly heareth wordes of lying, hath al his ſeruants
impious.

\V
\MNote{Contentment in ſtate of life.}
The pooreman and the creditour haue mette one an other: our Lord is
illuminatour of both.

\V
\MNote{Equitie in iudgement.}
The king, that iudgeth the poore in truth, his throne shal be
replenished for euer.

\V
\MNote{Chaſtiſment of youth.}
Rod and rebuke geueth wiſedom: but the childe, that is left to his owne
wil, confoundeth his mother.

\V
\MNote{God neuer permitteth al to be euil.}
In the multiplication of the impious, wickednes shal be multiplied, and
the iuſt shal ſee the ruines of them.

\V
\MNote{Inſtruct children.}
Nurter thy ſonne, and he shal refresh thee, and shal geue delightes to
thy ſoule.

\V
\MNote{Pray to haue good Paſtors.}
When prophecie shal fayle, the people shal be diſſipated: but he that
keepeth the Law, is bleſſed.

\V
\MNote{Compel the froward to obey.}
A ſeruant can not be taught by wordes: becauſe he vnderſtandeth that
which thou ſayſt, and contemneth to anſwer.

\V
\MNote{Conſideration in ſpeaking.}
Haſt thou ſeene a man ſwifte to ſpeake? follie is rather to be hoped,
then his amendement.

\V
\MNote{Subdue the fleſh to the ſpirite.}
He that nourisheth his ſeruant delicatly from his childhood, afterward
shal feele him ſtubburne.

\V
\MNote{Meeknes.}
An angrie man prouoketh brawles: and he that is eaſie to indignation,
shal be more prone to ſinne.

\V
\MNote{Humilitie.}
Humiliation foloweth the proude: and glorie shal receiue the humble of
ſpirite.

\V
\MNote{Participate not with ſinne.}
He that is partaker with a theefe, hateth his owne ſoule: he heareth one
adiuring, and telleth not.

%%% 1448
\V
\MNote{Feare not men in Gods cauſe remembring that God is Iudge of al.}
He that feareth man, shal ſoone fal: he that truſteth in our Lord shal
be lifted vp.

\V Manie ſeeke after the face of the prince: & the iudgement of euerie
one commeth forth from our Lord.

\V
\MNote{Hate al wickednes.}
The iuſt abhorre an impious man: & the impious abhorre them that are in
the right way.

The
\MNote{Obey ſpiritual and temporal parents.}
ſonne that keepeth the word, shal be out of perdition.


\stopChapter


\stopcomponent


%%% Local Variables:
%%% mode: TeX
%%% eval: (long-s-mode)
%%% eval: (set-input-method "TeX")
%%% fill-column: 72
%%% eval: (auto-fill-mode)
%%% coding: utf-8-unix
%%% End:


  
