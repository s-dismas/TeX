%%%%%%%%%%%%%%%%%%%%%%%%%%%%%%%%%%%%%%%%%%%%%%%%%%%%%%%%%%%%%%%%%
%%%%
%%%% The (original) Douay Rheims Bible 
%%%%
%%%% Old Testament
%%%% Proverbs
%%%% Argument
%%%%
%%%%%%%%%%%%%%%%%%%%%%%%%%%%%%%%%%%%%%%%%%%%%%%%%%%%%%%%%%%%%%%%%




\startcomponent argument


\project douay-rheims


%%% 1403
%%% o-1293
\startArgument[
  title={\Sc{The Argvment of the Proverbes.}},
  marking={Argument of the Proverbes.}
  ]


The
\MNote{VVhy this booke is called Prouerbes and Parables.}
firſt booke called \Emph{Prouerbes}, that is, \Emph{common & vſual
pithie ſentences}, shorte in wordes, ample in ſenſe;
and \Emph{Parables}, ſignifying likenes or \Emph{ſimilitudes}, wherby
more important thinges are vnderſtood then expreſſed; inſtructeth and
exhorteth new beginners, to lerne, and practiſe al ſortes of vertues,
the only right way to true Wiſdome and eternal happines.
\MNote{The contents.

Diuided into foure partes.}
It may be diuided into foure partes. In the firſt nine chapters the
auctor interpoſing certaine general preceptes, produceth wiſdom her
ſelfe inuiting al men to ſeeke her, for the ſpiritual profite, they shal
therby enioy. From thence to the 25.~chap. he geueth ſundrie more
particular precepts, as wel for embracing vertues, as shunning of
vices. In the next fiue chapters, more like precepts of the ſame auctor,
are added by the care of King Ezechias. In the two laſt chapters, either
an other Auctor, or rather the ſame vnder an other title, commendeth to
al men certaine moſt excellent precepts, receiued of his mother; wherto
he adioyneth the praiſe of a right wiſe woman: prophetically the
Catholique Church.


\stopArgument


\stopcomponent


%%% Local Variables:
%%% mode: TeX
%%% eval: (long-s-mode)
%%% eval: (set-input-method "TeX")
%%% fill-column: 72
%%% eval: (auto-fill-mode)
%%% coding: utf-8-unix
%%% End:
