%%%%%%%%%%%%%%%%%%%%%%%%%%%%%%%%%%%%%%%%%%%%%%%%%%%%%%%%%%%%%%%%%
%%%%
%%%% The (original) Douay Rheims Bible 
%%%%
%%%% Old Testament
%%%% Proverbs
%%%% Chapter 15
%%%%
%%%%%%%%%%%%%%%%%%%%%%%%%%%%%%%%%%%%%%%%%%%%%%%%%%%%%%%%%%%%%%%%%




\startcomponent chapter-15


\project douay-rheims


%%% 1424
%%% o-1313
\startChapter[
  title={Chapter 15}
  ]

\Summary{}


A
\MNote{Meeknes.}
ſoft anſwer breaketh anger: and a hard word rayſeth vp furie.

\V
\MNote{Diſcretion.}
The tongue of the wiſe adorneth knowlege: the mouth of fooles boyleth
forth follie.

%%% 1425
\V
\MNote{Gods omniſcience.}
In euerie place the eies of our Lord behold the good and the euil.

\V
\MNote{Calmnes of ſpeach.}
A peaceable tongue is a tree of life: but that which is immoderate, shal
breake the ſpirite.

%%% o-1314
\V
\MNote{Loue to be corrected.}
A foole ſcorneth the diſcipline of his father: but he that regardeth
reprehenſions, shal become more prudent.

In
\MNote{Deſire to fulfil al iuſtice.}
abundant iuſtice there is greateſt force: but the cogitations of the
impious shal be rooted out.

\V
\MNote{Diligence in teaching others.}
The houſe of the iuſt is very much ſtrength: and in the fruites of the
impious is pertubation.

\V
The lippes of the wiſe shal ſow knowlege: the hart of fooles shal be
vnlike.

\V
\MNote{Puritie of hart.}
The victimes of the impious are abominable to our Lord: the vowes of the
iuſt are acceptable.

\V
The way of the impious is abomination to our Lord: he that foloweth
iuſtice is beloued of him.

\V
\MNote{Lerne of good men.}
The doctrine is euil of them that forſake the way of life: he that
hateth reprehenſions shal dye.

\V
\MNote{Al ſecretes knowen to God.}
Hel, and perdition are before our Lord: how much more the hartes of the
children of men?

\V
\MNote{Harken to good admonitions.}
The peſtilent man loueth not him that rebuketh him: nor goeth to the
wiſe.

\V
\MNote{A
\Fix{cherful}{cheerful}{likely typo, fixed in other}
hart is deſirous to lerne.}
A glad hart cheereth the face: in penſifnes of minde the ſpirit is
caſt downe.

\V The hart of the wiſe ſeeketh doctrine: and the mouth of fooles is fed
with vnskilfulnes.

\V
\MNote{A quiet mind.}
Al the dayes of the poore are euil: a ſecure minde is as it were a
continual feaſt.

\V
\MNote{Contentment with ſufficiencie.}
Better is a litle with the feare of our Lord, then great treaſures and
vnſatiable.

\V It is better to be called to herbes with charitie: then to a fatted
calfe with hatred.

\V
\MNote{Patience.}
An angrie man ſtirreth brawles: he that is patient appeaſeth
thoſe that are raiſed.

\V
\MNote{Diligence.}
The way of the ſlothful is as an hedge of thornes: the way of the iuſt
is without offence.

\V
\MNote{Honour of parents.}
A wiſe ſonne maketh the father ioyful: and the foolish man deſpiſeth his
mother. 

\V
\MNote{Diſcretion.}
Follie is ioy to a foole: and the wiſeman directeth his ſteppes.

%%% 1426
\V
\MNote{Deſire to liue wel.}
Cogitations are diſſipated where there is no counſel: but where manie
counſellers are, they are confirmed.

\V
\MNote{Counſel in ſeaſon.}
A man reioyceth in the ſentence of his mouth: and a word in due time is
beſt.

\V
\MNote{To him that is wel trained in good workes, heauen gates are
open.}
The path of life aboue the lerned, that he may decline from the
\Fix{laweſt}{loweſt}{obvious typo, fixed in other}
hel.

\V Our Lord wil deſtroy the houſe of the proude: and wil make ſure the
borders of the widow.

\V
\MNote{Honeſt thoughts and wordes.}
Euil cogitations are an abomination to our Lord: and pure ſpeach moſt
beautiful shal be confirmed of him.

%%% o-1315
\V
\MNote{A liberal mind.}
He that purſueth auarice diſturbeth his houſe: but he that hateth giftes
shal liue.

By
\MNote{Mercie and iuſtice.}
mercie and faith ſinnes are purged: and by the feare of our Lord
euerie one declineth from euil.

\V
\MNote{Obedience.}
The minde of the iuſt meditateth obedience: the mouth of the impious
redoundeth with euils.

\V
\MNote{God aſſiſteth the iuſt.}
Our Lord is far from the impious: and he wil heare the prayers of the
iuſt.

\V
\MNote{VVordes of edification.}
The light of the eies reioyceth the ſoule: a good name fatteth the
bones.

\V
\MNote{Loue of diſcipline.}
The eare that heareth the reprehenſions of life, shal abide in the
middes of the wiſe.

\V
\MNote{Admonition.}
He that reiecteth diſcipline, deſpiſeth his ſoule: but he that yeldeth
to reprehenſions, is a poſſeſſour of the hart.

\V
\MNote{Humilitie.}
The feare of our Lord, the diſcipline of wiſdom: and humilitie goeth
before glorie.


\stopChapter


\stopcomponent


%%% Local Variables:
%%% mode: TeX
%%% eval: (long-s-mode)
%%% eval: (set-input-method "TeX")
%%% fill-column: 72
%%% eval: (auto-fill-mode)
%%% coding: utf-8-unix
%%% End:


  
