%%%%%%%%%%%%%%%%%%%%%%%%%%%%%%%%%%%%%%%%%%%%%%%%%%%%%%%%%%%%%%%%%
%%%%
%%%% The (original) Douay Rheims Bible 
%%%%
%%%% Old Testament
%%%% Proverbs
%%%% Chapter 14
%%%%
%%%%%%%%%%%%%%%%%%%%%%%%%%%%%%%%%%%%%%%%%%%%%%%%%%%%%%%%%%%%%%%%%




\startcomponent chapter-14


\project douay-rheims


%%% 1422
%%% o-1312
\startChapter[
  title={Chapter 14}
  ]

\Summary{}


A
\MNote{Frugal induſtrie.}
wiſe woman buildeth her houſe: the vnwiſe wil with her handes deſtroy
that alſo which is built.

%%% 1423
\V
\MNote{Feare of God.}
He that walketh in the right way, & feareth God is deſpiſed of him, that
goeth an infamous way.

\V
\MNote{Guard of the tongue.}
In the mouth of a foole the rod of pride: but the lippes of the wiſe
keepe them.

\V
\MNote{Diligent labores.}
Where oxen are not, the ſtal is emptie: but where much corne is, there
is the oxes ſtrength manifeſt.

\V
\MNote{Truth in al ſpeach.}
A faithful witneſſe wil not lie: but a deceitful witneſſe vttereth a
lie.

\V
\MNote{Seke wiſdom modeſtly.}
A ſcorner ſeeketh wiſdom and findeth it not: the doctrine of the prudent
is eaſie.

\V
\MNote{%
\Fix{Feloſhipe}{Felowſhip}{likely typo, fixed in other}
with the wiſe.}
Goe againſt a foolish man, and he knoweth not the lippes of prudence.

\V
\MNote{Knowlege of ourſelues.}
The wiſdom of a diſcrete man is to vnderſtand his way: and the imprudence
of fooles erreth.

\V
\MNote{Deteſtation of ſinne.}
A foole wil laugh at ſinne, & among the iuſt grace shal abide.

\V
\MNote{Internal comforth.}
The hart that knoweth the bitternes of his ſoule, in his ioy shal not
the ſtranger be mingled.

\V
\MNote{Contempt of this world.}
The houſe of the impious shal be raſed: the tabernacles of the iuſt shal
ſpring.

\V
\LNote{A vvay vvhich ſemeth iuſt.}{If
\MNote{VVithout true faith none can be ſaued.}
anie Iewes, Turkes, or Heretikes lead a moral good life in this world,
it ſemeth both to themſelues, and to other rude people, that they are in
a right way of ſaluation, but their error in faith leadeth them to
eternal damnation.}
\MNote{The Catholique faith.}
There is a way, which ſeemeth to a man iuſt: but the later endes therof
lead to death.

\V
\MNote{Spiritual ioy.}
Laughter shal be mingled with ſorow, and mourning occupieth the later
endes of ioy.

\V
\MNote{Reward of workes.}
A foole shal be replenished with his wayes, and the good man shal be
aboue him.

\V
\MNote{Beleue not al reportes.}
The innocent beleueth euerie word: the diſcrete man conſidereth his
ſteppes.

\V
\MNote{Mature conſideration.}
A wiſe man feareth and declineth from euil: the foole leapeth ouer and
is confident.

\V
\MNote{Patience.}
The impatient man shal worke folie: and the ſubtel man is odious.

\V
\CNote{\XRef{1.~Cor.~14. v.~20.}}
\MNote{Deſire of ſolide knowlege.}
The childish man shal poſſeſſe folie, and the prudent shal expect
knowlege.

\V
\MNote{Pietie ſhal be revvarded.}
The euil shal lie downe before the good, and the impious before the
gates of the iuſt.

\V
\MNote{Compaſſion of the poore.}
The poore shal be odious euen to his neighbour: but the freindes of the
rich be manie.

\V
\MNote{Almes dedes.}
He that deſpiſeth his neighbour, ſinneth: but he that hath pitie on the
poore, shal be bleſſed.

\V
\MNote{Mercie and veritie.}
They erre that worke euil: mercie and truth prepare good thinges.

%%% 1424
\V
\MNote{Good dedes with few wordes.}
In euerie worke there shal be abundance: but where manie wordes are,
there is oftentimes pouertie.

%%% o-1313
\V
\MNote{Right vſe of riches.}
The crowne of the wiſe, their riches: the follie of fooles, imprudence.

\V
\MNote{True teſtimonie as
\XRef{v.~5.}}
A faithful witnes deliuereth ſoules: and the
\TNote{\L{versipellis}
%%% ??? can't read next word
or turnecoate}
diſſembler vttereth lyes.

\V
\MNote{Feare to offend God preuenteth puniſhment.}
In the feare of our Lord is confidence of ſtrength, and to his children
there shal be hope.

\V
The feare of our Lord a fountaine of life, that he may decline from the
ruine of death.

\V
\MNote{Procure loue and fidelitie in ſubiectes.}
In the multitude of people the dignitie of the king: and in fewnes
of people the ignominie of the prince.

\V
\MNote{Patience.}
He that is patient, is gouerned with much wiſdom: but he that is
impatient, exalteth his follie.

\V
\LNote{Health of the hart.}{As
\MNote{Sincere intention excuſeth ſome errors.}
ſoundnes of the hart conſerueth the reſt of the ſame bodie in life, ſo a
pure intention often excuſeth from mortal ſinne, as in errours committed
of probable, not of groſſe, nor affected ignorance.
\CNote{\Cite{S.~Greg. li.~5. c.~34. Moral.}}
But ſecrete enuie in the hart infecteth and putrifieth mans actions, and
deſtroyeth the workes that ſemed good: which can no more endure ſtrict
examination in the day of iudgement, then a rotten cloth can abide
washing.}
\MNote{Sincere intention.}
Health of hart, the life of the flesh: enuie, the putrefaction of the bones.

\V
\MNote{Compaſſion.}
He that doth calumniate the needie, vpbraideth his maker: but he honoreth
him, that hath pitie on the poore.

\V
\MNote{Confidence in iuſtice.}
The impious shal be expelled in his malice: but the iuſt hopeth in his
death.

\V
\MNote{Inſtruction of the ignorant.}
In the hart of the prudent reſteth wiſdom, & it shal inſtruct al the
vnlerned.

\V
\MNote{Publique iuſtice.}
Iuſtice aduanceth a nation: but ſinne maketh peoples miſerable.

\V
\MNote{Induſtrie in euerie man.}
A ſeruant that vnderſtandeth is acceptable to the king: he that is
vnprofitable shal ſuſteyne his anger.


\stopChapter


\stopcomponent


%%% Local Variables:
%%% mode: TeX
%%% eval: (long-s-mode)
%%% eval: (set-input-method "TeX")
%%% fill-column: 72
%%% eval: (auto-fill-mode)
%%% coding: utf-8-unix
%%% End:


  
