%%%%%%%%%%%%%%%%%%%%%%%%%%%%%%%%%%%%%%%%%%%%%%%%%%%%%%%%%%%%%%%%%
%%%%
%%%% The (original) Douay Rheims Bible 
%%%%
%%%% Old Testament
%%%% Proverbs
%%%% Chapter 25
%%%%
%%%%%%%%%%%%%%%%%%%%%%%%%%%%%%%%%%%%%%%%%%%%%%%%%%%%%%%%%%%%%%%%%




\startcomponent chapter-25


\project douay-rheims


%%% 1441
%%% o-1329
\startChapter[
  title={Chapter 25
  \MNote{\tfx{The 3.~part.}}}
  ]

\Summary{}


Theſe
\MNote{More Parables of Salomon written by others.}
alſo are the parables of Salomon, which the men of Ezechias king of Iuda
wrote out.

\V
\MNote{Gods workes are not al reueled.}
It is the glorie of God to conceale the word, and the glorie of kinges
to ſearch the ſpeach.

\V
\MNote{Kinges haue ſome ſecretes.}
The heauen aboue, and the earth beneth, and the hart of kinges is
vnſcrutable.

\V
\MNote{Publique iuſtice and puniſhment of ſinne.}
Take away the ruſt from ſiluer, and there shal come forth a moſt pure
veſſel: \V Take away impietie from the kings countenance, and his throne
shal be eſtablished with iuſtice.

\V
\MNote{Modeſtie,}
Appeare not glorious before the king, and in the place of great men
ſtand not.

\V
\MNote{and Humilitie.}
For it is better that it be ſaid to thee: Come vp hither, then that thou
be humbled before the prince.

\V
\MNote{Care of others fame.}
The thinges which thy eies haue ſene, vtter not quickly in a brawle:
leſt afterward thou canſt not amend it, when thou haſt dishonoured thy
frend.

\V
\MNote{Compoſe controuerſies ſecretly, rather then contend in publique
court.}
Treate thy cauſe with thy frend, and reueale not a ſecret to a
ſtranger: \V leſt perhaps he inſult againſt thee, when he heareth, and
ceaſe not to vpbraide thee.

Grace and frendshipe deliuer:
\TNote{from feare}
which kepe to thyſelf, leſt thou become reprochful.

\V
\MNote{Speake in due time.}
Apples of gold in ſiluer beddes, he that ſpeaketh a word in his time.

\V
\MNote{Prudent admonition.}
A golden earlet, and a shining precious ſtone, he that rebuketh a
wiſeman, and an obedient eare.

\V
\MNote{Diligence in publique affaires.}
As the cold of ſnow in the day of harueſt, ſo a faithful legate to him,
that ſent him, maketh his ſoule to reſt.

\V
\MNote{Performance of promiſed induſtrie.}
Cloudes, and winde, and no rayne folowing, a glorious man, and not
accomplishing his promiſes.

\V
\MNote{Meeknes.}
By patience the prince shal be pacified, and a ſoft tongue shal breake
hardnes.

\V
\MNote{Temperance.}
Thou haſt found honie, eate that which ſufficeth thee, leſt perhaps
being filled thou vomite it vp.

\V
\MNote{Modeſtie.}
Withdraw thy foote from the houſe of thy neighbour, leſt ſome time
hauing his fil he hate thee.

%%% 1442
\V
\MNote{True teſtimonie.}
A dart, and ſword, and a sharpe arrow, a man that ſpeaketh falſe
teſtimonie
%%% o-1330
againſt his neighbour.

\V
\MNote{Truſt not a diſſembler.}
A rotten tooth, and wearie foote, he that hopeth vpon the vnfaithful in
the day of diſtreſſe, \V and that loſeth his cloke in the day of
cold.

Vineger
\MNote{Striue not with the incorrigible.}
in
\TNote{A conſuming ſaltish hard earth.}
nither, he that ſingeth ſonges to a naughtie hart.
\MNote{Alacritie.}
As a moth the garment, and a worme the woode: ſo the ſadnes of a man
hurteth the hart.

\V
\MNote{Charitie towards enimies.}
If thine enemie shal hunger, geue him meate: if he thirſt, geue him
water to drinke: \V
\CNote{\XRef{Rom.~2.}}
for thou shalt heape hote coales vpon his head, and our Lord wil reward
thee.

\V
\MNote{Heare not detraction.}
The northwinde diſſipateth raynes, & a ſad looke the tongue that
detracteth.

\V
\MNote{Domeſtical peace.}
It is better to ſitte in a corner of the houſe toppe, then with a
brawling woman, and in a common houſe.

\V
\MNote{Ioy of wel doing.}
Cold water to a thirſtie ſoule, and good tydings from a far countrie.

\V
\MNote{Profeſſion of truth.}
A fountaine trubled with the foote, and a vaine corrupted, the iuſt
falling before the impious.

\V
\MNote{Humilitie in knowlege.}
As he that eateth much honie, it is not good for him: ſo he that is a
ſearcher of the maieſtie, shal be oppreſſed of the glorie.

\V
\MNote{Gouernment of the tongue.}
As a citie being open and without compaſſe of walles, ſo a man that can
not repreſſe his ſpirit in ſpeaking.


\stopChapter


\stopcomponent


%%% Local Variables:
%%% mode: TeX
%%% eval: (long-s-mode)
%%% eval: (set-input-method "TeX")
%%% fill-column: 72
%%% eval: (auto-fill-mode)
%%% coding: utf-8-unix
%%% End:


  
