%%%%%%%%%%%%%%%%%%%%%%%%%%%%%%%%%%%%%%%%%%%%%%%%%%%%%%%%%%%%%%%%%
%%%%
%%%% The (original) Douay Rheims Bible 
%%%%
%%%% Old Testament
%%%% Proverbs
%%%% Chapter 26
%%%%
%%%%%%%%%%%%%%%%%%%%%%%%%%%%%%%%%%%%%%%%%%%%%%%%%%%%%%%%%%%%%%%%%




\startcomponent chapter-26


\project douay-rheims


%%% 1442
%%% o-1330
\startChapter[
  title={Chapter 26}
  ]

\Summary{}


As
\MNote{Aduance not the vicious.}
ſnow in the ſummer, and rayne in the harueſt: ſo is glorie vndecent for
a foole.

\V
\MNote{Patience in falſe ſlander.}
As a birde flying to other places, & a ſparow going whither he liſt: ſo a
curſe vttered in vaine shal light vpon ſome man.

\V
\MNote{Chaſtiſment.}
A whippe for a horſe, and a ſnaffle for an aſſe, and a rod on the back
of the vnwiſe.

\V
\MNote{Anſwer a foole wiſely:}
Anſwer not a foole according to his follie, leſt thou be made like to
him.

\V
\MNote{detecting his follie.}
Anſwer a foole according to his follie, leſt he ſeme to himſelf to be
wiſe.

\V
\MNote{Place fitte men in office:}
Lame of feete, and drinking iniquitie, he that ſendeth wordes by a
foolish meſſenger.

\V
\MNote{vviſe men in authoritie:}
As a lame man hath fayre legges in vaine: ſo a parable is vndecent in
the mouth of fooles.

\V
\MNote{vertuous in honour:}
As he that caſteth a ſtone into the heape of Mercurie: ſo he that geueth
honour to the vnwiſe.

%%% 1443
\V
\MNote{and lerned to teach.}
As if a thorne should grow in the hand of the drunkard: ſo a parable in
the mouth of fooles.

\V
\MNote{Make fooles to kepe ſilence.}
Iudgement determineth cauſes: and he that putteth a foole to ſilence,
apeaſeth angers.

%%% o-1331
\V
\CNote{\XRef{2.~Pet.~2.}}
\MNote{Returne not to former ſinnes.}
As a dog that returneth to his vomite, ſo the vnwiſe that reiterateth
his follie.

\V
\MNote{Humble opinion of thyſelf.}
Haſt thou ſene a man ſeeme to himſelfe wiſe? the foole shal haue hope
rather then he.

\V
\MNote{Fortitude.}
The ſlothful ſayth: A lyon is in the way, and a lyoneſſe in the
iourneis: \V
\MNote{Profitable laboures.}
as a doore turneth on his hinge ſo the ſlothful in his bed.

\V
\MNote{At leaſt ſome good worke.}
The ſlothful hideth his hand vnder the armehole, and is greeued if he
turneth it to his mouth.

\V
\MNote{Loue not idlenes.}
The ſlothful ſeemeth wiſer to himſelfe, then ſeuen men ſpeaking
ſentences.

\V
\MNote{Intermedle not in brawles.}
As he that taketh a dog by the eares, ſo he that paſſeth by impatient,
and medleth with an other mans brawle.

\V
\MNote{Vnfained frendſhipe, eſpecially in familiar acquaintance.}
As he is hurtful that shooteth arrowes, and ſpeares vnto death: \V ſo a
man, that hurteth his frende fraudulently: and when he is taken with al
ſayth: I did it in ieſt.

\V
\MNote{Puniſh batemakers.}
When wood fayleth, the fire shal be extinguished: and the whiſperer
taken away, brawles ceaſe.

\V
\MNote{Pacifie the wrathful.}
As coles to burning coles, and wood to fire, ſo an angrie man rayſeth
brawles.

\V
\MNote{Heare not whiſperers of euil reportes.}
The wordes of the whiſperer as it were ſimple, and the ſame come to the
inmoſt partes of the bellie.

\V
\MNote{Flee from hypocrites.}
As if thou wouldeſt adorne an earthen veſſel with droſſie ſiluer, ſo
ſwelling lippes ioyned with a moſt wicked hart.

\V
\MNote{VVhoſe flaterie and ſoft ſpeach are ſuſpicious, they wil fal at
laſt into their owne trappes, truth preuailing.}
An enemie is perceiued by his lippes, when he shal handle deceites in
his hart.

\V When he shal ſubmit his voyce, beleue him not: becauſe there are
ſeuen miſchiefes in his hart.

\V
He that couereth hatred fraudulently, his malice shal be reuealed in the
councel.

\V He that diggeth a pit, shal fal into it: and he that rolleth a ſtone,
it shal returne to him.

\V A deceitful tongue loueth not truth, and a ſlipper mouth worketh
ruines.


\stopChapter


\stopcomponent


%%% Local Variables:
%%% mode: TeX
%%% eval: (long-s-mode)
%%% eval: (set-input-method "TeX")
%%% fill-column: 72
%%% eval: (auto-fill-mode)
%%% coding: utf-8-unix
%%% End:


  
