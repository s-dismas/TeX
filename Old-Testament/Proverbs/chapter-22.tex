%%%%%%%%%%%%%%%%%%%%%%%%%%%%%%%%%%%%%%%%%%%%%%%%%%%%%%%%%%%%%%%%%
%%%%
%%%% The (original) Douay Rheims Bible 
%%%%
%%%% Old Testament
%%%% Proverbs
%%%% Chapter 22
%%%%
%%%%%%%%%%%%%%%%%%%%%%%%%%%%%%%%%%%%%%%%%%%%%%%%%%%%%%%%%%%%%%%%%




\startcomponent chapter-22


\project douay-rheims


%%% 1436
%%% o-1324
\startChapter[
  title={Chapter 22}
  ]

\Summary{}


Better
\MNote{Honeſt fame.}
is a good name, then much riches: aboue ſiluer and gold, good grace.

\V
\MNote{Contentment with our ſtate.}
The rich and poore haue mette one an other: our Lord is the maker of
both.

\V
\MNote{Prudence.}
The ſubtel ſaw euil, and hyd himſelf: the innocent paſſed by, and was
afflicted with damage.

\V
\MNote{Pouertie of ſpirite.}
The end of modeſtie the feare of our Lord, riches and glorie and life.

\V
\MNote{Care of the ſoules health.}
Armour and ſwordes in the way of the peruerſe: but the keper of his owne
ſoule departeth far from them.

\V
\MNote{Good cuſtome in vertues.}
It is a prouerbe: A yongman according to his way, when he is old, wil
not depart from it.

\V
\MNote{Diligent trauel.}
The richman ruleth ouer the poore: and he that boroweth is the ſeruant
of him that lendeth.

\V
\MNote{Equitie.}
He that ſoweth iniquitie, shal reape euils, and with the rod of his
wrath, he shal be conſumed.

\V
\MNote{Almes dedes.}
He that is prone to mercie, shal be bleſſed: for of his breades he hath
geuen to the poore.

He
\MNote{Better to geue then to take.}
that geueth giftes shal purchaſe victorie and honour: but he that
receiueth taketh away the ſoule of the geuer.

%%% o-1325
\V
\MNote{Expel the incorrigible.}
Caſt out the ſcorner, and brawling shal goe forth with him, and cauſe
shal ceaſe and contumelies.

\V
\MNote{Cleanes of hart.}
He that loueth cleanes of hart, for the grace of his lippes, shal haue
the king his frend.

\V
\MNote{The godlie proſper.}
The eies of our Lord keepe knowlege: and the wordes of the iuſt are
ſupplanted.

\V
\MNote{Fortitude.}
The ſlothful ſayth: A lyon is without, in the middes of the ſtreates I
am to be ſlayne.

\V
\MNote{Care of chaſtitie.}
A deepe pitte the mouth of a ſtrange woman: he with whom our Lord is
angrie, shal fal into it.

\V
\MNote{Chaſtiſment.}
Follie is tyed together in the hart of a childe, and the rod of
diſcipline shal driue it away.

%%% 1437
\V
\MNote{Compaſſion.}
He that doth calumniate the poore, to increaſe his riches, himſelf shal
geue to a richer, and shal be in neede.

\V
\MNote{Rules of wiſdom are neceſſarie, profitable, and vpon practiſe
found pleaſant: rightly directing al our
\Fix{thougts}{thoughts}{likely typo, same in both}
wordes and dedes.}
Incline thine eare, and heare the wordes of wiſemen: and ſet thy hart to
my doctrine: \V which shal be beautiful for thee, when thou shalt kepe
it in thy bellie, and it shal flow in thy lippes.

\V
That thy confidence may be in our Lord, wherfore I haue shewed alſo it
to thee this day.

\V Behold I haue deſcribed it to thee three maner of wayes, in
cogitations and knowledge: \V that I might shew thee the ſtabilitie, and
the wordes of truth, out of theſe to anſwer them, that ſent thee.

\V
\MNote{Care of the poore, becauſe they are deare to God.}
Doe not violence to the poore, becauſe he is poore: neither oppreſſe the
needie in the gate: \V becauſe our Lord wil iudge his cauſe, and wil
pearſe them, that haue pearſed his ſoule.

\V
\MNote{Flee from euil companie which may corrupt thee.}
Be not frend to an angrie man, nor walke with a furious man: \V leſt
perhaps thou lerne his pathes, and take ſcandal to thy ſoule.

\V
\MNote{Auoide ſuretiſhipe, leſt thou fal into diſtreſſe.}
Be not with them, that ſticke downe their handes, and that offer
themſelues ſureties for debts: \V for it thou haue not wherewith to
reſtore, what cauſe is there, that he should take the couering from thy
bed?

\V
\MNote{Kepe ancient traditions.}
Trangreſſe not the ancient boundes, which thy fathers haue put.

\V
\MNote{Diligent trauel.}
Haſt thou ſene a man quicke in his worke? he shal ſtand before kinges,
neither shal
\Fix{be}{he be}{obvious typo, fixed in other}
before the vnnoble.


\stopChapter


\stopcomponent


%%% Local Variables:
%%% mode: TeX
%%% eval: (long-s-mode)
%%% eval: (set-input-method "TeX")
%%% fill-column: 72
%%% eval: (auto-fill-mode)
%%% coding: utf-8-unix
%%% End:


  
