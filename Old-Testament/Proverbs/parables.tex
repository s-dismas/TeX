%%%%%%%%%%%%%%%%%%%%%%%%%%%%%%%%%%%%%%%%%%%%%%%%%%%%%%%%%%%%%%%%%
%%%%
%%%% The (original) Douay Rheims Bible 
%%%%
%%%% Old Testament
%%%% Proverbs
%%%% Parables
%%%%
%%%%%%%%%%%%%%%%%%%%%%%%%%%%%%%%%%%%%%%%%%%%%%%%%%%%%%%%%%%%%%%%%




\startcomponent argument


\project douay-rheims


%%% 1416
%%% o-1306
\startArgument[
  title={\Sc{The Parables of Salomon.}},
  marking={The Parables of Salomon.}
  ]


This
\MNote{Sentencious moral precepts.}
repetition of the title ſignifieth, that the ſentences which folow are
more properly called Parables, then the former.
\MNote{How theſe Parables folowing differ from the former.}
From vvhich they alſo differ in maner of vtterance, by the figure
Antitheſis, for moſt part oppoſing in comparing contrarie vertues and
vices, ſhevving their contrarie effectes; vvith great elegancie,
eſpecially in the original tongue; vvhich could not be ſo fully
expreſſed in Greke, nor Latin, much leſſe in vulgar language. But are
the ſame in ſenſe, though often obſcure by reaſon of the Hebrevv phraſe,
ſhortnes of ſentences, and ſo vvithout anie certaine connexion, that we
can not with perſpicuitie, comprehend the ſumme therof in briefe
contentes, after the ordinarie maner before the chapters.
\MNote{VVhy the contents of the twentie chapters folowing are put in the
margent.}
And therfore haue thought it better for the vulgar reader, to ſet downe
in the margent of the twentie chapters next folowing, in briefe termes,
the vertues, or other good thinges (rather then the bad, not hauing
place for both) commended in euerie ſentenſe. For though the ſame be not
alwayes expreſſed in the text, yet they may be vnderſtood by their
oppoſite vices. VVhoſoeuer deſireth further explication, may finde manie
of theſe diuine ſentences, excellently expounded by S.~Ierom,
S.~Auguſtin, S.~Gregorie and other Fathers in ſeueral places. Or read
S.~Bedas Commentaries vpon this whole booke:
\Cite{To.~4. vel. apud S.~Ierom To.~7.}
Or amongſt late writers our lerned contriman D.~Randuplhus Bainus:
Biſhop Ianſenius: and F.~Peltanus.


\stopArgument


\stopcomponent


%%% Local Variables:
%%% mode: TeX
%%% eval: (long-s-mode)
%%% eval: (set-input-method "TeX")
%%% fill-column: 72
%%% eval: (auto-fill-mode)
%%% coding: utf-8-unix
%%% End:
