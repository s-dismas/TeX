%%%%%%%%%%%%%%%%%%%%%%%%%%%%%%%%%%%%%%%%%%%%%%%%%%%%%%%%%%%%%%%%%
%%%%
%%%% The (original) Douay Rheims Bible 
%%%%
%%%% Old Testament
%%%% Proverbs
%%%% Chapter 02
%%%%
%%%%%%%%%%%%%%%%%%%%%%%%%%%%%%%%%%%%%%%%%%%%%%%%%%%%%%%%%%%%%%%%%




\startcomponent chapter-02


\project douay-rheims


%%% 1406
%%% o-1296
\startChapter[
  title={Chapter 02}
  ]

\Summary{Gaining of wiſdom bringeth much good, 10.~and auoydeth much euil:
  16.~deliuering from error of Idolaters and Hæretikes.}


My ſonne,
\SNote{This frequent maner of propoſing the vvay and meanes to vviſdom,
If thou vvilt receiue my vvordes, &c. ſhevveth moſt euidently the povvre
of mans free vvil.}
if thou wilt receiue my wordes, and wilt hide my commandments with thee,
\V that thyne eare may heare wiſdom: incline thyne hart to knowe
prudence. \V For if thou shalt cal for wiſdom, and incline thyne hart to
prudence: \V If thou shalt ſeeke her
\SNote{Not euerie deſire, or ſleight ſeeking of vviſdom
\Fix{ſufficieth,}{ſufficeth,}{obvious typo, fixed in other}
but ſuch laborious ſeeking is required, as a couetous man ſeeketh
treaſure vvhich he knoweth to be hid in the ground.}
as money, and as treaſures shalt dig her vp: \V then shalt thou
vnderſtand the feare of our Lord, and shalt finde the knowlege of God. \V
Becauſe our Lord geueth wiſdom: and out of his mouth prudence and
knowlege. \V He wil keepe the ſaluation of the righteous, & protect them
that walke ſimply. \V Keeping the pathes of iuſtice, & garding the wayes
of ſaints. \V
\CNote{\XRef{Sap.~3. v.~32.}
\XRef{10.~v.~9.}}
Then shalt thou vnderſtand iuſtice, and iudgement, and
equitie, and euerie good path. \V If wiſdom shal enter into thy hart,
and knowlege pleaſe thy ſoule: \V counſel shal keepe thee, and prudence
shal preſerue thee, \V that thou mayſt be deliuered from the euil way,
and from the man, that ſpeaketh peruerſe thinges: \V
%%% !!! These four LNotes are all joined into one in the text.
\LNote{VVho leaue the right vvay.}{Generally
\MNote{Foure markes of an heretike.

1.~He forſaketh the knowen faith.}
this deſcription of wicked men, agreeth to al that committe and perſiſt
in mortal ſinne, whether they walked right at anie time before or no; but moſt
eſpecially ſheweth the properties of heretikes:
\CNote{\XRef{Iſaia.~35 v.~8}}
who forſake and leaue the direct, ancient, beaten, knovvne vvay of the
Catholique Church, and teach nevv obſcure doctrines, not heard of, or not
approued in our forefathers time.}
who
\SNote{A deſcription of peruers ſinners eſpecially of heretikes.}
leaue the right way, and walke by darke wayes: \V
\LNote{Who are glad when they haue done euil.}{Secondly
\MNote{2.~He glorieth in his ovvne invention.}
they glorie in their ovvne deuiſes, and reioyce in moſt vvicked thinges,
as in ſeducing multitude of peoples, to rebel againſt their Catholique
Princes, and other Superiors ſpiritual and temporal; in breaking vovves;
in deſpiſing good vvorkes; truſting to only faith, and that not the
Catholique faith of al true Chriſtians, but euerie one his particular
perſvvaſion, that himſelf is iuſt, & ſhal be ſaued, vvhich by their
ovvne doctrin, none is bond to beleue of an others ſtate, but of his
owne only. In ſo much that the chiefeſt point of a Proteſtants imagined
faith, is not a general Article, which al do or ſhould beleue, but a
moſt particular and ſingular phantaſie, which each one muſt conceiue of
himſelf, or herſelf.}
who are glad when they haue done euil, and reioyce in moſt wicked
thinges: \V whoſe wayes are peruerſe, and their ſteppes infamous. \V
That thou mayſt be deliuered
%%% 1407
from
\LNote{The ſtrange woman.}{Thirdly
\MNote{3.~Teacheth pleaſing thinges.}
Hereſie, called here the ſtrange and forrene woman, tempereth her
vvordes, to pleaſe the itching eares of her auditorie, framing her
doctrine to the humour of thoſe, vvhom she ſeeketh to peruert. The ſame
\Fix{vvich}{which}{obvious typo, fixed in other}
the
\CNote{\XRef{Rom.~16. v.~18.}}
Apoſtle ſaith in other vvordes, by ſvvete ſpeaches and benedictions they
ſeduce the hartes of innocents.}
the ſtrange woman, and from the forener, which mollifieth her wordes, \V
forſaketh the guide of her youth, \V and hath forgotten the couenant of
her God. For her houſe is bowed downe to death, and her pathes to
hel. \V Al
\LNote{That goe in vnto her.}{Fourthly
\MNote{4.~Admitteth no iudge but himſelf.}
Thoſe that do enter into error of hereſie, ſhal not returne, that is,
very hardly and rarely returne into the right vvay of life; the reaſon
whereof the ſame
\CNote{\XRef{Tit.~3. v.~11.}}
Apoſtle yeldeth, becauſe an heretike is condemned by his ovvne
iudgement. For being in error, and admitting no iudge but himſelf, he
parteth from the Church, excludeth the meanes of better inſtruction, &
through his erronious iudgement, remaineth in damnable opinion, and ſo
in the ſtate of damnation.}
that goe in vnto her, shal not returne
%%% o-1297
neither shal they apprehend the
pathes of life. \V That thou mayſt walke in a good way: and mayſt keepe
the pathes of the iuſt. \V For they that are right, shal dwel in the
earth, and the ſimple shal continue in it. \V But the impious shal be
deſtroyed from the earth: and they that doe vniuſtly shal be deſtroyed
from the earth: and they that doe vniuſtly shal be taken away from it.


\stopChapter


\stopcomponent


%%% Local Variables:
%%% mode: TeX
%%% eval: (long-s-mode)
%%% eval: (set-input-method "TeX")
%%% fill-column: 72
%%% eval: (auto-fill-mode)
%%% coding: utf-8-unix
%%% End:
