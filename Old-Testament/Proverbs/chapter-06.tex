%%%%%%%%%%%%%%%%%%%%%%%%%%%%%%%%%%%%%%%%%%%%%%%%%%%%%%%%%%%%%%%%%
%%%%
%%%% The (original) Douay Rheims Bible 
%%%%
%%%% Old Testament
%%%% Proverbs
%%%% Chapter 06
%%%%
%%%%%%%%%%%%%%%%%%%%%%%%%%%%%%%%%%%%%%%%%%%%%%%%%%%%%%%%%%%%%%%%%




\startcomponent chapter-06


\project douay-rheims


%%% 1411
%%% o-1301
\startChapter[
  title={Chapter 06}
  ]

\Summary{He that is ſuretie for an other, muſt haue care to diſcharge
  that he promiſeth. 6.~The ſlouthful muſt lerne diligence of the
  emmotte. 12.~The deſcription of an Apoſtate. 16.~Aboue other ſix bad
  thinges, God deteſteth the ſower of diſcord. 20.~Al are exhorted to
  kepe Gods law, 24.~namely to flee fornication, and al occaſions
  therof.}


My ſonne,
\SNote{The vviſman doth not abſolutly diſvvade from al maner of
ſuretiſhippe, but from raſhly, or vnaduiſedly anſvvering for others. And
eſpecially exhorteth to vſe al diligence in performing, or cauſing
others to performe that vvhich is promiſed or couenanted.}
if thou shalt be ſuretie for thy freind, and haſt made faſt thy hand to
a ſtranger, \V thou art entrapped with the wordes of thy mouth, & caught
with thyne owne wordes. \V Doe therfore my ſonne that which I ſay, and
deliuer thyſelfe, becauſe thou art fallen into the hand of thy
neighbour. Runne diuers wayes, make haſt, rayſe thy freind. \V Geue not
ſleepe to thine eies, neither let thine eieliddes ſlumber. \V Deliuer
thyſelfe as a litle doa from the hand, and as a bird from the hand of
the fowler. \V Goe to the emmote ô ſluggard, and conſider her wayes, and
lerne wiſdom. \V Who wheras she hath no guide, nor maſter, nor captaine,
\V prepareth meate for herſelfe in the ſummer, and gethereth in the harueſt
for to eate. \V How long wilt thou ſleepe ô ſluggard? when wilt thou
riſe out of thy ſleepe? \V Thou shalt sleepe a litle, a litle shalt thou
ſlumber, a litle shalt thou ioyne thy handes to ſleepe: \V and a penurie shal come
to thee, as a wayfaring man, and pouertie as a man armed. But if thou be
not ſluggish, thy harueſt shal come as a fountaine, and penurie shal
flee farre from thee. \V A man that is an
\SNote{Euerie one that ſinneth vvittingly and of malice refuſing to obey
God, imployeth his mouth, eyes, feete, handes and al partes vvith a
vvicked hart and intention to peruerte others: moſt proper to heretikes,
apoſtates from the faith.}
Apoſtata, a man vnprofitable, goeth with peruerſe mouth, \V winketh with
the eies, treadeth with the foote, ſpeaketh with the finger, \V with
wicked hart he deuiſeth euil, and at al time he ſoweth brawles. \V To
him his deſtruction shal come forthwith, and he shal ſodenlie be
deſtroyed, neither shal he haue remedie any more. \V Six thinges there
are, which our Lord hateth, and the ſeuenth his ſoule deteſteth: \V
Loftie eies, a
%%% 1412
lying tongue, handes that shede innocent bloud, \V a hart that deuiſeth
moſt wicked deuiſes, feete ſwift to runne into euil, \V a deceitful
witneſſe that vttereth lies, and him that among
\Fix{brether}{brethren}{obvious typo, fixed in other}
\SNote{The former ſix are al damnable, but this ſeuenth is moſt
deteſtable, becauſe it is oppoſite to the chief vertue charitie, it
breaketh vnitie, & is the proper ſinne of the diuel.}
ſoweth diſcordes. \V My ſonne keepe the preceptes of thy father, and
leaue not the lawe of thy mother. \V Bynde them in thy hart continualy,
and put them about thy throte. \V When thou shalt walke, let them goe
with thee: when thou shalt ſleepe, let them kepe thee, and awaking talke
with them. \V Becauſe the commandment is a lampe, and the lawe a light,
and the way of life the increpation of diſcipline: \V that they may kepe
thee from the euil woman, and from the faire ſpoken tongue of the
ſtranger. \V Let not thy hart
%%% o-1302
couet her beautie, be not caught with her beckes: \V for the price of an
harlot is ſcarſe worth one loafe: but a woman catcheth the precious
ſoule of man. \V
\SNote{Al occaſions of ſinne, eſpecially probable are to be ſhunned.}
Can a man hide fyre in his boſome, that his garmentes burne not? \V Or
walke vpon hote coales, that his ſoales be not burnt? \V So he that
goeth in vnto his neighbours wife, shal not be cleane when he shal
touche her. \V It is
\SNote{Theft is alſo mortal ſinne, againſt the ſeuenth cõmandment, but
not ſo great as adulterie.}
no greate fault, when a man shal haue ſtollen: for he ſtealeth to fil his
hungrie ſoule: \V alſo being taken he shal reſtore ſeuenfold, and shal
geue vp al the ſubſtance of his houſe. \V But he that is an aduouterer,
for penurie of hart shal deſtroy his owne ſoule: \V shame and ignominie
he gethereth to
\Fix{himſefe,}{himſelfe,}{obvious typo, fixed in other}
& his reproch shal not be blotted out. \V Becauſe the zele and furie of
the husband wil not ſpare in the day of reuenge, \V neither wil he yeld
to any mans prayers, neither wil he take for redemption verie many
giftes.


\stopChapter


\stopcomponent


%%% Local Variables:
%%% mode: TeX
%%% eval: (long-s-mode)
%%% eval: (set-input-method "TeX")
%%% fill-column: 72
%%% eval: (auto-fill-mode)
%%% coding: utf-8-unix
%%% End:
