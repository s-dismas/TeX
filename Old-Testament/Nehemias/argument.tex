%%%%%%%%%%%%%%%%%%%%%%%%%%%%%%%%%%%%%%%%%%%%%%%%%%%%%%%%%%%%%%%%%
%%%%
%%%% The (original) Douay Rheims Bible 
%%%%
%%%% Old Testament
%%%% Nehemias
%%%% Argument
%%%%
%%%%%%%%%%%%%%%%%%%%%%%%%%%%%%%%%%%%%%%%%%%%%%%%%%%%%%%%%%%%%%%%%




\startcomponent argument


\project douay-rheims


%%% 0982
%%% o-0882
\startArgument[
  title={\Sc{The Argvment of the Booke of Nehemias.}},
  marking={Argument of the Booke of Nehemias.}
  ]

\Emph{This
\MNote{Duble title of this booke.}
booke} beareth Title both of the author \Emph{Nehemias}, who writ it,
and of \Emph{the ſecond booke of Eſdras}, who in the former writ the
hiſtorie of the Iſraelites after theyr relaxation from captiuitie, to
the building againe of the Temple, with other thinges done the ſame
time.
\MNote{The cõtentes.}
VVhereunto Nehemias ioyneth thinges ſucceding, eſpecially the \Emph{new
erection of walles and towers about the citie of Ieruſalem}.
\CNote{S.~Ierom. Epiſt. ad Paulin.}
\MNote{Diuided into three partes.}
And it may be diuided into three partes. In the two firſt chapters, he
sheweth his compaſsion of his countries miſſerie: and his cõming to
aſſiſt them. In the tenne folowing, he reciteth the good effectes in
repayring, and ſtrengthning the citie with walles and people. In the
laſt chapter, the correction of errors & euil maners, which he found
amongſt them.


\stopArgument


\stopcomponent


%%% Local Variables:
%%% mode: TeX
%%% eval: (long-s-mode)
%%% eval: (set-input-method "TeX")
%%% fill-column: 72
%%% eval: (auto-fill-mode)
%%% coding: utf-8-unix
%%% End:
