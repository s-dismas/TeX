%%%%%%%%%%%%%%%%%%%%%%%%%%%%%%%%%%%%%%%%%%%%%%%%%%%%%%%%%%%%%%%%%
%%%%
%%%% The (original) Douay Rheims Bible 
%%%%
%%%% Old Testament
%%%% Wisdom
%%%% Chapter 14
%%%%
%%%%%%%%%%%%%%%%%%%%%%%%%%%%%%%%%%%%%%%%%%%%%%%%%%%%%%%%%%%%%%%%%




\startcomponent chapter-14


\project douay-rheims


%%% 1497
%%% o-1382
\startChapter[
  title={Chapter 14}
  ]

\Summary{Foolish men intending to ſaile, honour woodden idols, in regard
  of the profite they receiue by shippes: 6.~by which ſome were ſaued in
  the general diluge. 8.~Idols, and idolmakers are curſed. 12.~They were
  not from the beginning, 15.~but were deuiſed for memorie of the dead,
  and worshipped with diuine honour. 22.~So men forgetting God,
  proceeded in idolatrie, with other abominable, and cruel enormities.}

Agayne an other thinking to ſayle, and begynning to make a iorney
through the fierce waues, inuocateth wood
\SNote{Great madnes to inuocate a
\Fix{wodden}{woodden}{obvious typo, fixed in other}
idol more baſe & commonly more corruptible then the wood of a ſhippe.}
more fraile then the wood that carieth him. \V For couetouſnes of
getting inuented it, and the crafteſman by his wiſdom framed it. \V But
thy prouidence, ô Father, doth gouerne: becauſe thou haſt geuen a way
euen in the ſea, and among the waues a moſt ſure path, \V shewing that
thou art able to ſaue out of al thinges, yea
\SNote{As the Iſraelites went through the redſea.}
if a man goe to the ſea without art. \V But that thy workes might not be
voyde of wiſdom: for this cauſe alſo men commit their liues euen to a
little wood, and paſſing ouer the ſea are deliuered by shippe. \V But
from the begynning alſo when the prowde giants perished, the hope of the
world fleing to a shippe, rendered to the world ſeede of natiuitie,
which was gouerned by thy hand. \V For
\SNote{The auctor prophetically alludeth to the wood of the Croſſe, on
which our Sauiour redemed mankind.}
bleſſed is the wood, by the which
\SNote{From whoſe death procedeth mans iuſtification.}
iuſtice is made. \V But the idol that is made by handes, curſed is both
it,
%%% 1498
and he that made it: becauſe he in deede wrought it: and the ſame being
fraile, was called god. \V But to God the impious and his impietie are
odious alike. \V For that which is made, with him that made it, shal
ſuffer torments. \V For this cauſe alſo in the idol of the nations there
shal be no reſpect: becauſe the creatures of God were made to hatred,
and for tentation to the ſoules of men, and for a ſnare to the feete of
the vnwiſe. \V For the begynning
\SNote{Inuention of Idols brought men to ſpiritual fornications, &
corruption of maners.}
of fornication is the deuiſing of idols: and the inuenting of them is the
corruption of life. \V For neither were they from the begynning, neither
shal they be for euer. \V For this vanitie of men came into the world:
and therfore there is found a short end of them. \V For
\LNote{The father made vnto himſelf the image of his ſonne.}{Caluin
\MNote{Caluin falſly chargeth this booke vvith error.}
here chargeth this booke with error, in affirming that idolatrie begane
by ſuperſticiouſly honoring images of the dead. Againſt which he
alleageth that
\CNote{\XRef{Gen.~31.}}
Labans idoles, and others more ancient, were before anie images of
dead men were honoured. But he argueth vpon a falſe ground. For Labans
idols were images, as the Hebrew word \HH{Teraphim} ſignifieth, and is
ſo tranſlated in the
\Cite{Engliſh Bibles (1552. and 1577.)}
but becauſe they were images of falſe goddes, and for that Laban called
them his goddes, a later
\Cite{Bible (1603)}
tranſlateth it better, 
\MNote{Images of falſe goddes are rightly called idols.}
\Emph{idoles}, as the Latin and Greek haue \L{idola}. It is alſo
certaine that Ninus king of Aſſirians long before Laban, yea before
Abraham, ſette vp the image of his Father Belus (otherwiſe called
Iuppiter) to be publikly honored by the people as S.~Cyril ſheweth.
\Cite{li.~3. in Iulianum, nere the end,}
and S.~Ambroſe, or another graue Auctor writeth the ſame in
\Cite{cap.~1. ad Romanos.}
Likewiſe S.~Cyprian
\Cite{li.~de Idolarum vanitate.}
S.~Chryſoſtom
\Cite{ho.~87. in Matth.}
and Egeſippus,
\Cite{apud S.~Ieronym li. de Viris Illuſtrib.}
teſtifie, that the making of mens images, in memorie of the dead, was
the occaſion, and beginning of idolatrie, according as this place
reporteth, that
\MNote{Idolatrie begane by vvorſhipping images of dead men vvith diuine
honour.}
a Father ſorovving for the death of his ſonne, made an image in his
memorie, & begane to worshippe him as a god, cauſing his ſeruants alſo
to honour his dead ſonne, vvith rites and ſacrifices. VVhich priuate
idolatrie vvas abſolutely the firſt, that is recorded in holie
Scripture, or anie other good auctor. And the firſt publique is counted
by moſt auctors, that of Ninus, vvorſhipping the image of his father
Belus, vvith diuine honour, who alſo pardoned al offenders, how enormious
ſoeuer their crimes were, that fled vnto that image.
\MNote{Priuate idolatrie was before publique.}
VVhich allurment together vvith ſo great a kinges auctoritie, drevv
innumerable to publique idolatrie. VVherupon S.~Ierom noteth
\Cite{(in cap.~2. Oſee.)}
that Ninus became ſo great and glorious, as to make his father to be
honored as a god.}
the father being ſorowful with bitter moorning, made vnto himſelf the
image of his ſonne quickly taken away: and him, that then was a dead
man, now
\SNote{This firſt idolatrie was only priuatly exerciſed by the father
and his ſeruants at their maſters cõmandment, by which occaſion publique
idolatrie came into the world, wicked cuſtom in time preuailing.}
he began to worshipe as god, and appointed holie thinges and ſacrifices
among his ſeruants. \V 
%%% !!! SNote marked in both, but absent in both
%%% \SNote{}
Afterward by ſucceſſion of time, the wicked cuſtom preuayling, this
errour was kept as a law, and thinges grauen were worshipped by the
commandement
\Fix{af}{of}{obvious typo, fixed in other}
tyrants. \V And thoſe, whom openly men could not honour, for that they
were far of, their figure being brought from afar, they made an euident
image of the king, whom they would honour: that by their carefulnes they
might honour as preſent, him that was abſent. \V And to
%%% o-1383
the worshipping of theſe, the excellent diligence alſo of the artificer,
holpe them forward, that were ignorant. \V For he willing to pleaſe him,
that entertained him, laboured by his art, to fashion the ſimilitude in
better ſort. \V And the multitude of men caried away by the beautie of
the worke, him that a little before had bene honoured as a man, now they
eſtemed for a god. \V And this was the deceyuing of mans life: becauſe
men ſeruing either affection, or kinges, gaue the name that is
\SNote{The name \Sc{God} in the proper ſignification, can not be geuen
to anie creature.}
not communicable to ſtones and wood. \V And it was not ſufficient that
they erred about the knowlege of God, but alſo liuing in a great battail
of ignorance ſo manie and ſo great euils they cal peace. \V For
\SNote{Manie enormous crimes procede from idolatrie.}
either ſacrificing their children, or making abſcure ſacrifices, or
hauing watches ful of madnes, \V they now neither keepe life, nor
mariage cleane, but one killeth an other by enuie, or playing the
adulterer maketh him ſorowful: \V and al thinges are mingled together,
bloud, manſlaughter, theft and fiction, corruption, and infidelitie,
truble, and periurie,
%%% 1499
diſquieting of the good, \V forgetfulnes of God, inquination of ſoules,
immutation of natiuitie, inconſtancie of mariage, diſorder of adulterie,
and vnchaſtnes. \V For the worshippe of idols not to be named, is the
cauſe of al euil, and the beginning and end. \V For either when they
reioyce, they are madde: or certes prophecie falſe thinges, or liue
vniuſtly, or quickly forſweare themſelues. \V For whiles they truſt in
idols, which are without ſoule, ſwearing amiſſe they hope not to be
hurt. \V
\SNote{Two ſortes of periurie: ſwearing by falſe goddes, and ſwearing
vntruthes.}
Two euil thinges therfore shal happen to them worthely, becauſe they
haue thought euil of God, attending to idols, and haue ſworne vniuſtly,
in guile contemning iuſtice. \V For it is not the powre of them, that
are ſworne by, but the punishment of them that ſinne, goeth alwayes
through the trangreſſion of the vniuſt.


\stopChapter


\stopcomponent


%%% Local Variables:
%%% mode: TeX
%%% eval: (long-s-mode)
%%% eval: (set-input-method "TeX")
%%% fill-column: 72
%%% eval: (auto-fill-mode)
%%% coding: utf-8-unix
%%% End:


  
