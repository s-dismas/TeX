%%%%%%%%%%%%%%%%%%%%%%%%%%%%%%%%%%%%%%%%%%%%%%%%%%%%%%%%%%%%%%%%%
%%%%
%%%% The (original) Douay Rheims Bible 
%%%%
%%%% Old Testament
%%%% Wisdom
%%%% Argument
%%%%
%%%%%%%%%%%%%%%%%%%%%%%%%%%%%%%%%%%%%%%%%%%%%%%%%%%%%%%%%%%%%%%%%




\startcomponent argument


\project douay-rheims


%%% 1477
%%% o-1364
\startArgument[
  title={\Sc{The Argvment of the Booke of Wisdom.}},
  marking={Argument of the Booke of Wisdom.}
  ]


As
\MNote{Both the auctor & auctoritie of this booke were ſometimes
doubtful.}
wel of the auctor, as of the auctoritie of this booke, there haue bene
diuers opinions among the lerned. But in proceſſe of time, the firſt is
probably diſcuſſed, the other is clerly decided by the Church. For
concerning the former doubt, \Emph{Manie ancient Fathers alleage
ſentences of this Booke, as the ſayinges of Salomon.} Namely S.~Ireneus
\Cite{(apud Euſebium, lib.~5. c.~8. Hiſt.)}
S.~Clement of Alexandria,
\Cite{li.~5. &~6. Stromat.}
Origen,
\Cite{ho.~12. in Leuit.}
&
\Cite{li.~8. in Epiſt. ad Rom.}
S.~Athanaſius,
\Cite{in Synopſi.}
&
\Cite{Orat.~2. cont. Arrian.}
S.~Baſil,
\Cite{li.~5. cont. Eunomianos}
S.~Epiphanius,
\Cite{hereſi.~67.}
S.~Gregorie Nazianzen,
\Cite{lib. de Fide.}
S.~Gregorie Niſſen,
\Cite{in Teſtimonijs ex vet. Teſtam. cap. de Natiuitiate ex Virgine.}
S.~Chriſoſt.
\Cite{hom.~33. &~34. in Mat.}
S.~Cyril of Alexandria,
\Cite{li.~10. c.~4.}
Alſo S.~Cyprian
\Cite{li.~de
%%% 1478
exhortat. Martyrum. c.~12.}
&
\Cite{li.~3. c.~59. ad Quirin.}
&
\Cite{li. de Mortalitate.}
S.~Hilarie
\Cite{in Pſal.~117.}
S.~Ambroſe,
\Cite{li. de Salomone. c.~1.}
and diuers others ſuppoſe Salomon to be auctor of this booke.
\MNote{The ſame doubt is of Eccleſiaſticus.}
To whom likewiſe ſome of them aſcribe the booke of Eccleſiaſticus. But
S.~Ierom
\Cite{Præftione in libros Salomonis}
teſtifieth that ſome ancient writers affirme this booke to be written by
Philo a Iew, and the other by Ieſus the ſonne of Sirach. And S.~Auguſtin
very plainly
\Cite{(li.~17. c.~20. de ciuit.)}
ſaith, \Emph{cuſtom preuailed, that the bookes of Wiſdom, &
Eccleſiaſticus, for ſome ſimilitude of ſpeach are called Salomons: but
the more lerned aſſuredly iudge that they are not his.}
\MNote{It is moſt probable, that Philo a Iew writte this booke
collecting manie ſentences of Salomons.}
What then shal we ſay, ſeing ſo manie other ancient lerned Doctors cite
them as his. The anſwer is eaſie. 
\CNote{\XRef{Argum. lib. Reg.}}
And ſufficiently inſinuated by S.~Auguſtin, that theſe two bookes being
like vnto the other three, which are Salomons, were alſo called
his. VVherto we may adde a like example in the two firſt bookes of
Kinges, which are called the bookes of Samuel, though he writte not al
the firſt, nor anie part of the ſecond. Moreouer al theſe
\Emph{fiue} are called by one general title \Emph{Sapiential bookes}.
\MNote{Fiue Sapiential bookes of the old teſtament.}
In ſo much that the Church readeth in the Sacred Office before al
Epiſtles, taken out of anie of theſe fiue bookes, not \L{Lectio
Prouerbiorum}, or \L{Eccleſiaſtæ}, &c. but ſtil \L{Lectio libri
Sapientiæ}. The ſolution therfore is very probable, that this booke of
wiſdom was written by Philo Iudeus, not he that liued after Chriſt, but
an other of the ſame name, nere two hundred yeares before. And
Eccleſiaſticus by Ieſus the ſonne of Sirach. Who not only imitated
Salomon, but alſo compiled their bookes, for moſt part of Salomons
ſentences; conſerued til their times by tradition, or in ſeparated
ſcrolles of papers;
\CNote{\XRef{Chap.~7.~8.~9.}}
yea they ſo vtter ſome ſentences in his perſon, as if himſelf had
written them.
\MNote{The Iewes denie theſe bookes to be Canonical.}
As touching \Emph{the auctoritie of theſe two bookes}, and ſome others,
it is euident that \Emph{the Iewes refuſe} them. And therfore manie
ancient Fathers writing againſt them, ſpared ſometimes to vrge ſuch
bookes, as they knew would be reiected. Eſpecially hauing abundant
teſtimonies of other holie Scriptures, for deciding matters of
%%% o-1365
faith againſt them. Euen as
\CNote{\XRef{Mat.~22.}}
our Sauiour himſelf proued the Reſurrection of the dead againſt the
Sadduces, out of the
\CNote{\XRef{Exo.~3.}}
bookes of Moyſes, which they confeſſed for Canonical Scripture, denying
other partes, where the ſame point might otherwiſe haue bene more
euidently shewed. And ſo S.~Ierom in reſpect of the Iewes ſaide theſe
bookes were not Canonical. Neuertheles he did often alleage teſtimonies
of them, as of other diuine Scriptures: ſometimes with this parentheſis
\L{[ſi cui tanem placet librum recipere]} in
\Cite{cap.~8. &~12. Zachariæ}
other times, eſpecially in his laſt writinges, abſolutly without ſuch
reſtrictions, as in
\Cite{cap.~1. &~56, Iſaiæ}
& in
\Cite{18.~Ieremiæ.}
Where he profeſſeth to alleage none but Canonical Scripture.
\MNote{They are iudged by very manie ancient fathers, and afterwards
defined by the Church to be Canonical Scriptures.}
As for al the other ancient fathers here aboue mentioned, aſcribing this
booke to Salomon, and manie others cited by Doctor Iodocus Coccius
\Cite{(To.~1. Theſauri. li.~6. art.~9.)}
they make
%%% 1479
no doubt at al, but that it is Canonical Scripture, as appeareth by
their expreſſe termes, \Emph{Diuine Scripture, Diuine word, Sacred
letters, Prophetical ſaying, the Holie Ghoſt ſaith}, & the like. Finally
as wel ancient General counſels, namely that of Charthage, an.~D.~419.
with others, as the later of Florence, and Trent haue declared this
booke to be Canonical. And that conformably to the moſt ancient, and
lerned Fathers, as S.~Auguſtin, not only iudgeth  himſelf, but alſo
plainly teſtifieth
\CNote{Et \Cite{li.~17. c.~20. Ciu.}}
\Cite{(li. de Prædeſtinat. Sanct. c.~14.)}
ſaying: \Emph{The ſentence of the booke of wiſdom ought not to be
reiected} (by certaine inclining to Pelagianiſme) \Emph{which hath bene
ſo long publiquely read in the Church of Chriſt, and receiued of al
Chriſtians, Byshops, and others, euen to the laſt of the Laitie,
Penitents, and Catecumes} \L{(cum veneratione diuina auctoritatis)}
\Emph{with veneration of diuine auctoritie. Which alſo the excellent
writers, next to the Apoſtles times, alleaging for witnes} \L{(nihil ſe
adhibere, niſe diuinum teſtimonium crediderunt)} \Emph{thought they
alleaged nothing but} diuine teſtimonie.

The
\MNote{The contents.}
ſumme and contents of this booke is an Inſtruction, and Exhortation to
Kinges and al Magiſtrates, to miniſter iuſtice in the commonwealth,
teaching al ſortes of vertues vnder the general names of Iuſtice &
Wiſdom, with frequent Prophecies of Chriſts Coming, Paſſion,
Reſurrection, & other Chriſtian Myſteries.
\MNote{Diuided into three parts.}
Al may be commodiouſly diuided into three partes. In the ſix firſt
chapters, the auctor admonisheth al Superiors to loue and exerciſe
iuſtice and wiſdom. In the next three, he teacheth that Wiſdom procedeth
only from God, & is procured by prayer & good life. In the other tenne
chapters, he sheweth the excellent effects, and vtilitie of Wiſdom and
Iuſtice.


\stopArgument


\stopcomponent


%%% Local Variables:
%%% mode: TeX
%%% eval: (long-s-mode)
%%% eval: (set-input-method "TeX")
%%% fill-column: 72
%%% eval: (auto-fill-mode)
%%% coding: utf-8-unix
%%% End:
