%%%%%%%%%%%%%%%%%%%%%%%%%%%%%%%%%%%%%%%%%%%%%%%%%%%%%%%%%%%%%%%%%
%%%%
%%%% The (original) Douay Rheims Bible 
%%%%
%%%% Old Testament
%%%% Judges
%%%% Chapter 11
%%%%
%%%%%%%%%%%%%%%%%%%%%%%%%%%%%%%%%%%%%%%%%%%%%%%%%%%%%%%%%%%%%%%%%




\startcomponent chapter-11


\project douay-rheims


%%% 0560
%%% o-0503
\startChapter[
  title={Chapter 11}
  ]

\Summary{Iephte reiected by his brethrens, is intreated by the ancientes
  of Galaad to returne and fight for them againſt the Ammonites:
  12.~with whom he firſt pleadeth the cauſe of Iſrael by iuſt reaſons,
  26.~and long preſcription. But they perſiſting obſtinate, he
  (30.~inconſideratly vowing) 32.~ouerthroweth them, 34.~and ſacrificeth
  his onlie daughter.}

There was at that time Iephte the Galaadite a moſt valiant man and a
warrier, the ſonne of a woman that was
\SNote{The hebrew word \HH{Zonah} ſignifieth alſo \Emph{an in keeper}.}
an harlot, who was borne of Galaad. \V And Galaad had a wife of whom he
had ſonnes: who after they were growen, caſt out
\Fix{Iepthte,}{Iephte,}{obvious typo, fixed in other}
ſaying: Thou canſt not be heyre in the houſe of our father, becauſe thou
art borne of an other mother. \V Whom he fleeing and auoyding, dwelt in
the Land of Tob: and there were gathered to him needie men, and
theeuiſh, and folowed him as their prince. \V In thoſe dayes the
children of Ammon fought againſt Iſrael. \V Who preſſing ſore vpon them,
the ancientes of Galaad went to take Iephte out of the Land of Tob to
helpe them: \V and they ſaied to him: Come and be our prince, and fight
againſt the children of Ammon. \V To whom he anſwered: Are not you they
that hated me, and caſt me out of my fathers houſe, and now are come to
me forced by neceſſitie. \V And the princes of Galaad ſaid to Iephte: For
this cauſe be we now
\SNote{If they had not concurred to his expulſion, it might haue
ſufficed to haue ſent for him, but in this caſe the ancientes iudged it
meete to goe in perſon, and to intreat him. So Chriſt was reiected by
the Iewes, and returneth not to them til in the end of the world they
ſhal ſeeke vnto him.
\Cite{S.~Aug. q.~49. in Iudic. poſt mediũ.}}
come to thee, that thou goe forth with vs, and fight againſt the
children of Ammon, and be the captaine of al that dwell in Galaad. \V
Iephte alſo ſaid to them: If you be come to me ſincerly, that I ſhould
fight for you againſt the children of Ammon, and if our Lord deliuer
them into my handes, ſhal I be your prince? \V Who anſwered him: Our
Lord which heareth theſe thinges, him ſelfe is mediatour and witnes that
we wil doe as we haue promiſed. \V Iephte therfore went with the princes
of Galaad, and al the people made him their prince. And Iephte ſpake al
his wordes before our Lord in Maſpha. \V And he ſent meſſengers to the
king of the children of Ammon, which should ſay in his perſon: What is
betwen me and thee, that thou art come againſt me, to waſt my Land? \V
To whom he anſwered: Becauſe Iſrael tooke my land, when he aſcended out
of Ægypt, from the coaſts of Arnon vnto Iaboc and Iordan: now therfore
with peace reſtore the ſame to me. \V By whom Iephte againe ſent word,
and commanded them that they should ſay
%%% 0561
to the king of Ammon: \V Thus ſayth Iephte: Iſrael did not take the
Land of Moab, nor the Land of the children of
%%% o-0504
Ammon: \V but when they aſcended out of Ægypt, he walked through the
deſert vnto the Readſea, and came into Cades. \V
\CNote{\XRef{Num.~20.}}
And he ſent meſſengers to the king of Edom, ſaying: Suffer me that I may
paſſe through thy land. Who would not condeſcend to his requeſtes. He
ſent alſo to the king of Moab, who alſo him ſelfe contemned to geue
paſſage. He abode therfore in Cades, \V and compaſſed the Land of Edom
at the ſide, and the land of Moab: and came againſt the Eaſt quarter of
the Land of Moab, and camped beyond Arnon: neither would he enter the
boundes of Moab: for Arnon is the border of the Land of Moab. \V Iſrael
therfore ſent meſſengers to Sehon the king of the Ammorrheites, who
dwelt in Heſebon, and they ſaid to him: Suffer me to paſſe through thy
land vnto the riuer. \V Who alſo him ſelfe deſpiſing the wordes of
Iſrael, ſuffered him not to paſſe through his borders: but gathering an
infinite multitude went forth againſt him into Iaſa, and reſiſted
ſtrongly. \V And our Lord deliuered him into the handes of Iſrael with
al his armie, and he ſtroke him, and poſſeſſed al the Land of the
Ammorrheite the inhabiter of that countrie, \V and al the coaſtes therof
from Arnon vnto Iaboc, & from the wildernes vnto Iordan. \V Our Lord
therfore the God of Iſrael ſubuerted the Amorrheite, his people of
Iſrael fighting againſt him, and wilt thou now poſſeſſe his land? \V Are
not thoſe thinges which
\SNote{In the opinion of infidels, it ſemed that they poſſeſſed countries
by the helpe of falſe goddes, and ſo they thought them ſelues to haue
iuſt title. Much more iuſt is the title when God almighty geueth
victorie of conqueſt.
\Cite{S.~Aug. q.~48. in Iudic.}}
Chamos thy god poſſeſſed, dew to thee by right? But the thinges that our
Lord God hath obteyned conquerour, shal come to our poſſeſſion: \V
vnleſſe perhaps thou be better then Balac the ſonne of Sephor the king
of Moab: or canſt shew, that he wrangled againſt Iſrael, and fought
againſt him, \V when he dwelt in Heſebon, and the litle townes therof,
and in Aroer, and the townes therof, or in al the cities nere Iordan,
for
\SNote{He argueth vpon preſcription of 300.~yeares being nere ſo much,
for there wanted ſcarce thirtie: being from the conqueſt made by Moyſes
\XRef{(Num.~21.)}
til the time of Iephte about 270.~yeares.}
three hundred yeares. Wherfore haue you ſo long attempted nothing for
reclaime? \V Therfore I doe not ſinne againſt thee, but thou doeſt euil
againſt me, denouncing me vniuſt warres. Our Lord be iudge the arbiter
of this day betwen Iſrael, and betwen the children of Ammon. \V And the
king of the children of Ammon would not harken to the wordes of Iephte,
which he ſent him by the meſſengers. \V Therfore the ſpirite of our Lord
came vpon Iephte, and circuiting Galaad, and Manaſſes,
%%% 0562
Maſpha alſo of Galaad, and thence paſſing to the children of Ammon, \V
he vowed a vow to our Lord, ſaying: If thou wilt deliuer the children of
Ammon into my handes, \V
\SNote{This vow was vnlawful, for the law forbiddeth to offer man or
woman in ſacrifice.
\XRef{Exo.~34. v.~20.}
\XRef{Deut.~12. v.~31.}}
whoſoeuer ſhal firſt come forth out of the doores of my houſe, and shal
meete me returning with peace from the children of Ammon, him wil I
offer an holocauſte to our Lord. \V And Iephte paſſed to the children of
Ammon, to fight againſt them: whom our Lord deliuered into his
handes, \V and he ſtroke from Aroer til thou come to Mennith, twentie
cities, and as farre as Abel, which is ſette with vineyardes, with a
very great plague, and the children of Ammon were humbled by the
children of Iſrael. \V But Iephte returning into Maſpha to his houſe,
his onlie begotten daughter mette him with tymbrels and daunces. For he
had
%%% o-0505
not other children. \V Whom when he ſaw, he rent his garmentes, and
ſaid: Wo is me my daughter thou haſt deceiued me, and thy ſelf art
deceiued: for I haue opened my mouth to our Lord, and I can doe no other
thing. \V To whom ſhe anſwered: My father, if thou haſt opened thy mouth
to our Lord, do vnto me whatſoeuer thou haſt promiſed, the reuenge and
victorie of thyne enemies being granted to thee. \V And ſhe ſaid to her
father: This only graunt me which I deſire: Suffer me that two monethes
I may goe about the mountaines, and
\SNote{In the old teſtament mariage was ordinarily preferred before
ſingle life but in the new, it is better to kepe virginity.
\XRef{1.~Cor.~7. v.~38.}}
bewayle my virginitie with my felowes. \V To whom he anſwered: Goe. And
he diſmiſſed her two monethes. And when she was gone with her felowes
and companions, ſhe mourned her virginitie in the mountaines. \V And the
two monethes being expired, ſhe returned to her father, and he
\LNote{Did to her as he had vovved.}{VVhether Iephte did wel or no in
ſacrificing his daughter, hauing vowed to offer in ſacrifice whoſoeuer
(or whatſoeuer) ſhould firſt mete him returning with victorie, as it
hapened ſhe did, is a great and hard queſtion, ſaieth S.~Auguſtin
\Cite{(q.~49. in lib. Iudic.)}
and not eaſily decided, the holie ſcripture neither approuing nor
reprouing his fact.
\MNote{Iephte offended in vowing vndiſcretly. But not in performing his
vow as ancient fathers thinke more probable.}
\MNote{S.~Auguſtin.}
Neuertheles by conference of other ſcriptures and diſcourſe of reaſon,
he iudgeth it moſt probable that Iephte offended in vowing without
ſpecial warrant from God, to ſacrifice, that which by the law was not
ſacrificable; yet ſinned not in performing his vow, but rather pacified
God therby, whoſe wil it ſemed to be, that for puniſhment of his ſinne
he ſhould ſacrifice his daughter, becauſe by his diuine prouidence ſhe
firſt mette him: and the omiſſion might rather haue benne for his
natural loue towards his onlie childe, then for the vnlawfulnes of the
ſacrifice: ſeing it once pleaſed God to command Abraham to immolate his
ſonne Iſaac, though when it came to execution, he forbade the ſame,
appointing an other hoſte in place of the childe, which here he did
not. Neither was it iniurious to the daughter, ſeing ſhe, as al
mankinde, muſt once die when God appointeth. Yea further ſhe offered her
ſelf freely (which ſemed to be by Gods inſtinct) willing her father to
do to her whatſoeuer he had promiſed to God. This is the ſumme of
S.~Auguſtins large diſcourſe.
\MNote{S.~Ambroſe.}
Likewiſe S.~Ambroſe
\Cite{(li.~3. de Officiis c.~12.)}
ſuppoſeth aſſuredly that this prince Iephte offended in vowing
vnaduiſedly, for it alſo repented him, when his daughter firſt mette
him: yet that with \Emph{godlie feare and dreade} he performed to his
owne bitter paine that which he had promiſed: inſtituting an
anniuerſarie lamentation of his daughter, for a warning to poſteritie of
more circumſpection in making vowes.
\MNote{S.~Hierom.}
S.~Hierom alſo
\Cite{(li.~1. aduerſ. Iouinian.)}
approueth their opinion that ſay: It was Gods ordinance Iephte ſhould
feele the errour of his vnaduiſed vow, by the death of his daughter, for
a document to others.
\MNote{S.~Chryſoſtom.}
The very ſame teacheth S.~Chryſoſtom,
\Cite{(ho.~14. ad pop. Antioch.)}
that God would haue this errour to be thus puniſhed, that others might
be warned from vowing the like.
\MNote{S.~Gregorie Nazianzen.}
S.~Gregorie Naziazen
\Cite{(orat. de Machabæis)}
preferring the martyrdome of the ſeuen brothers and their mother, before
this ſacrifice of Iephte as \Emph{more aduiſed, and more honorable}, yet
condemneth not this, but recounteth it amongſt other commendable actes.
\MNote{Theodoret.}
Theodoret
\Cite{(q.~19. in Iudic.)}
and al the aforeſaid fathers do highly commend the daughters promptnes
in offering her ſelf to be ſacrificed, which either much extenuated her
fathers fault, or wholly iuſtified his fact. Thus the ancient fathers
moderate their cenſures.
\CNote{\Cite{Bible 1603.}}
\MNote{Proteſtants cenſure.}
Yet a new gloſſe of the Engliſh Bible without ſcruple ſayeth, that by
his raſh vow, \Emph{and vvicked performance his victorie vvas defaced};
and againe, that he was ouercome \Emph{vvith blinde zele, not
conſidering} whether the vow was lawful or no.}
did to her as he had vowed, who knew not man. Thence forth a faſhion in
Iſrael, and a cuſtome was kept: \V that after the compaſſe of a yeare
the daughters of Iſrael aſſemble together, and mourne the daughter of
Iephte the Galaadite foure dayes.


\stopChapter


\stopcomponent


%%% Local Variables:
%%% mode: TeX
%%% eval: (long-s-mode)
%%% eval: (set-input-method "TeX")
%%% fill-column: 72
%%% eval: (auto-fill-mode)
%%% coding: utf-8-unix
%%% End:
