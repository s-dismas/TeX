%%%%%%%%%%%%%%%%%%%%%%%%%%%%%%%%%%%%%%%%%%%%%%%%%%%%%%%%%%%%%%%%%
%%%%
%%%% The (original) Douay Rheims Bible 
%%%%
%%%% Old Testament
%%%% Two Kings
%%%% Argument
%%%%
%%%%%%%%%%%%%%%%%%%%%%%%%%%%%%%%%%%%%%%%%%%%%%%%%%%%%%%%%%%%%%%%%




\startcomponent argument


\project douay-rheims


%%% 0656
%%% o-0589
\startArgument[
  title={\Sc{The Argvment of the Second Booke of Kinges.}},
  marking={Argument of Two Kinges.}
  ]

Beſides
\MNote{This booke is wholly of Dauid.}
a great part of the firſt booke, and beginning of the third, this ſecond
booke is wholly of King Dauid. VVhoſe manie laudable Actes, as alſo his
faultes (which were fewer) with his true repentance, and punishment are
related, not in ſuch method, as may eaſily be diuided into diſtinct
partes, in order of the chapters; but according to the diſtinction of
thinges conteined,
\MNote{His ſucceſſion to the kingdom.}
his ſucceſsion to the royal crowne, firſt in Iuda, and after in al
Iſrael, with the declination and death of his competitour Iſboſeth, are
recorded in the 2. 3. 4. and 5.~chapters.
\MNote{His vertues.}
His vertues, and praiſes, to wit, his ſolemne mourning for Saul and that
familie, his deuotion, fortitude, pietie, and gratitude are ſpecially
touched in the 1. 6. 7. 8. 9. and 10.~chapters.
\MNote{Faultes.}
His ſinnes of adulterie with Bethſabee, of killing her husband Vrias, of
pride in numbering his people, with his hartie repentance, and temporal
punishment for the ſame, are written from the 11.~chapter to the 21,
together with the 24.
\MNote{Thankes, and Prophecie.}
The 22. and 23.~chapters conteyne his thankeſgeuing to God for benefites
receiued, and prophecie of thinges to come, with a catalogue of valiant men.


\stopArgument


\stopcomponent


%%% Local Variables:
%%% mode: TeX
%%% eval: (long-s-mode)
%%% eval: (set-input-method "TeX")
%%% fill-column: 72
%%% eval: (auto-fill-mode)
%%% coding: utf-8-unix
%%% End:
