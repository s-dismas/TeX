%%%%%%%%%%%%%%%%%%%%%%%%%%%%%%%%%%%%%%%%%%%%%%%%%%%%%%%%%%%%%%%%%
%%%%
%%%% The (original) Douay Rheims Bible 
%%%%
%%%% Old Testament
%%%% Ecclesiastes
%%%% Argument
%%%%
%%%%%%%%%%%%%%%%%%%%%%%%%%%%%%%%%%%%%%%%%%%%%%%%%%%%%%%%%%%%%%%%%




\startcomponent argument


\project douay-rheims


%%% 1452
%%% o-1340
\startArgument[
  title={\Sc{The Argument of Ecclesiastes}.},
  marking={Ecclesiastes}
  ]


King
\MNote{This booke called Ecclesiastes, teacheth to contemne this
vvorld. Becauſe felicitie conſiſteth not in anie temporal thing: but in
the eternal ſight of God.}
\Emph{Salomon} a diuine \Emph{Preacher}, wherof this Booke is called
\Emph{Ecclesiastes}, exhorteth al ſuch as haue lerned the principles of
good life, to contemne this world: becauſe al thinges therin are vaine,
and inſufficient to geue repoſe to mans ſoule: shewing that true
\Emph{felicitie}, which al men deſire, \Emph{conſiſteth not} in natural
knowlege, gotten by witte and induſtrie; nor in worldlie pleaſures, much
leſſe in carnal; nor in riches; nor in auctoritie or dominion; nor
\Emph{in anie} other \Emph{temporal thing}; as diuers diuerſly thinke:
\Emph{but} only \Emph{in the true ſeruice of God}, by flying from ſinne,
and doing good workes, as in the meritorious cauſe, \Emph{and}
eſſentially \Emph{in the clere viſion of God}: the proper end, for which
man was created.

And
\MNote{Diuided into three parts.}
ſo this Booke conteyneth three principal parts. Firſt this diuine
preacher confuteth al their opinions, that imagine a falſe felicitie in
humane, worldlie, or temporal thinges: to the beginning of the
7.~chapter. In the reſt of that chapter, and three folowing, he teacheth
that true felicitie conſiſteth in the eternal fruition of God: and is
procured by declining from vices, and embracing vertues. In the two laſt
chapters, he exhorteth al to beginne ſpedily to ſerue God, and to
perſeuere therin to the end of this life.


\stopArgument


\stopcomponent


%%% Local Variables:
%%% mode: TeX
%%% eval: (long-s-mode)
%%% eval: (set-input-method "TeX")
%%% fill-column: 72
%%% eval: (auto-fill-mode)
%%% coding: utf-8-unix
%%% End:
