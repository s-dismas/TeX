%%%%%%%%%%%%%%%%%%%%%%%%%%%%%%%%%%%%%%%%%%%%%%%%%%%%%%%%%%%%%%%%%
%%%%
%%%% The (original) Douay Rheims Bible 
%%%%
%%%% Old Testament
%%%% Ecclesiastes
%%%% Chapter 10
%%%%
%%%%%%%%%%%%%%%%%%%%%%%%%%%%%%%%%%%%%%%%%%%%%%%%%%%%%%%%%%%%%%%%%




\startcomponent chapter-10


\project douay-rheims


%%% 1464
%%% o-1351
\startChapter[
  title={Chapter 10}
  ]

\Summary{Conſidering the great difference betwen wiſdom and follie,
  4.~it behoueth to reſiſt vehement tentations diligently. 5.~As when
  euil, & ignorant men haue auctoritie ouer the wiſe. 8.~The wicked
  often fal into their owne ſnares, 10.~are hard, yet not vnpoſſible to
  be corrected. 11.~Detracters are like ſerpents. 12.~Wiſe graue princes
  are profitable; childish are hurtful to the commonwealth; 18.~which by
  their negligence tendeth to ruine: 20.~yet ſubiectes ought not to
  iudge euil of them.}


Flies
\SNote{Baſe & vicious men mixt with the good corrupt the whole companie:
much more a mortal ſinne in a mans ſoule deſtroyeth al the vertues, that
were there before.}
dying marre the ſweetnes of ointment. Wiſdom and glorie is more
precious, then a litle and temporal follie. \V The hart of a wiſeman is
in his righthand, and the
%%% 1465
hart of a foole is in his lefthand. \V Yea and the foole walking in the
way, wheras himſelf is vnwiſe, eſtemeth al men fooles. \V If the ſpirite
of him that hath powre, aſcend vpon thee, leaue not thy place: becauſe
carefulnes wil make the greateſt ſinnes to ceaſe. \V There is an euil
that I haue ſene vnder the ſunne, as it were by errour proceding from
the face of the prince: \V A foole ſet in high dignitie, and the rich to
ſitte beneth. \V I haue ſene
\SNote{Euil men aduanced ſeme to proſper:}
ſeruants vpon horſes: and princes walking on the ground as ſeruants. \V
He
\SNote{but they fal into their owne trappes.}
that diggeth a pitte, shal fal into it: and he that breaketh the hedge,
a ſerpent shal bite him. \V He that remoueth ſtones,
shal be afflicted in them: and he that cutteth trees, shal be wounded of
them. \V If the iron shal be blunt, and that not as before, but shal be
made blunt, it shal be sharpened by great labour, and after
induſtrie shal wiſdom folow. \V If a ſerpent bite in ſilence, nothing
leſſe then it hath he, that detracteth ſecretly. \V The wordes of the
mouth of a wiſeman grace: and the lippes of the vnwiſe shal throw him
downe headlong. \V The beginning of his wordes is follie, and the later
end of his mouth is moſt wicked errour. \V A foole multiplieth wordes. A
man is ignorant what hath bene before him: and what shal be after him,
who can tel him? \V The labour of fooles shal afflict them, that know not
to goe into
\SNote{Such as ſeke by ſenſe and reaſon to obtaine true knowlege, enter not
into the citie, the Church. They labour in vaine and are afflicted in
ſtudie of Scriptures, when they walke in the deſert, and can not finde
the citie.
\Cite{S.~Ierom.}}
the citie. \V
\LNote{VVoe to thee ô land, vvhoſe king is a childe.}{S.~Ierom
\MNote{This text, and manie others, haue two ſenſes.}
(as in moſt part of his commentaries vpon this booke) expoundeth this
paſſage in two ſenſes: ſimply according to the firſt apparance of the
letter; and myſtically concerning the Church.
\MNote{1.

In kinges and al ſuperiors are required mature age, & diligent
care of the cõmon good.}
The wiſeman ſemeth in dede (ſayth he) to reproue the principalitie of
yongmen, and to condemne luxurious iudges; for that in the one by want
of age is infirme wiſdom; in the other, mature age is weakened by
delicacies. And contrarywiſe he approueth a prince of good partes, &
liberal education; & commendeth thoſe Iudges, which do not preferre
voluptuouſnes before publique affayres: but after great labour, and
adminiſtration of the commonwealth, are conſtrained as by neceſſitie to
take meate.
\MNote{2.

Antiquitie in matter of faith and religion is to be folowed, not
noueltie.}
Yet to me (ſaith this great Doctor) ſomething more ſacred ſemeth to lye
hidde in the letter: that in Scripture they are called yongmen, who
forſake old auctoritie, and contemne ancient precepts of forefathers;
who neglecting Gods commandment, deſire to eſtablish traditions of
men. Touching which points,
\CNote{\XRef{Iſa.~8.}}
our Lord threatneth Iſrael by Iſaias, for that this people hath refuſed
the water of Siloe, that runneth with ſilence, and hath turned away the
old fiſhpond, chooſing the ſtreames of Samaria, and gulfes of Damaſcus,
I wil geue yongmen to be their princes, and deluders shal rule ouer
them. Read Daniel:
\CNote{\XRef{Dan.~7.}}
Thou shalt finde God ancient of dayes. Read the Apocalips of
\Fix{S.~Ioſu:}{S.~Iohn:}{obvious typo, fixed in other}
\CNote{Apoc.~1.}
Thou shalt finde the head of our Sauiour white as ſnow, and as white
wool.
\CNote{\XRef{Iere.~1.}}
Ieremie alſo becauſe he was wiſe and grauitie was reputed in his wiſdom,
was forbid to cal himſelf a childe. VVoe therfore to the land, whoſe
king is the diuel, who alwayes coueting nouelties, rebelled in Abſalom
againſt the father.
\MNote{Alſo mortification and labour is required in Paſtors, not
delicacie nor eaſe.}
VVoe to that land whoſe Iudges, and Princes loue the pleaſures of this
vvorld. VVho, vntil the day of death come, ſay: Let vs eate and drinke,
for to morrow we shal dye. Contrarivviſe bleſſed is the land of the
Church, vvhoſe King is Chriſt, the Sonne of the freeborne, deſcending
from Abraham, Iſaac, and Iacob, the ſtock of Prophetes, and of al
Saintes, ouer vvhom ſinne ruled not: and for that cauſe they vvere truly
free:
\MNote{The B.~Virgin more free from ſinne then the Patriarches.}
of vvhom vvas borne the holie Virgin Marie more free: hauing no ſhrubbe,
nor branch out of the ſide, but her vvhole fruite ſprung forth into a
floure: ſaying in the Canticles:
\CNote{\XRef{Cant.~2.}}
I am the floure of the filde, the lillie of the
\Fix{valles.}{valleyes.}{obvious typo, fixed in other}
The princes alſo of this land are the Apoſtles, and al ſainctes, vvho
haue their king the ſonne of the freeborne, the ſonne of the freevvoman,
not of the bondvvoman Agar, but borne of the freedom of Sara. Neither do
they eate in the morning, nor quickly. For they ſeke not pleaſure in
this preſent vvorld; but shal eate in their due time, vvhen the time of
revvard shal come, and they shal eate in fortitude, and not in
confuſion. Al the good of this preſent vvorld is confuſion: but of the
future vvorld is perpetual fortitude. Thus farre S.~Ierom. VVhoſe
diſcourſe vve haue here cited at large for a taſte of his profound
expoſition of this vvhole booke; that ſuch as haue apportunitie, may
read the reſt in the auctor himſelf.
\Cite{To.~7.}}
Woe to thee ô land, whoſe king is a childe, and whoſe princes eate in
the morning. \V Bleſſed is the land, whoſe king is noble, & whoſe
princes eate in their time to refection, and not to
%%% o-1352
riotouſnes. \V In ſlouthfulnes the roofe of the houſe shal goe to ruine,
& in the infirmitie of the handes the houſe shal droppe through. \V They
make bread for laughter, and wine that liuing they may make merie: and
to money al thinges obey. \V In thy cogitation detract not from the
king, and in the ſecret of thy chamber curſe not the richman: becauſe
euen the birdes of the ayre wil carie thy voice, and he that hath winges
wil declare the ſentence.


\stopChapter


\stopcomponent


%%% Local Variables:
%%% mode: TeX
%%% eval: (long-s-mode)
%%% eval: (set-input-method "TeX")
%%% fill-column: 72
%%% eval: (auto-fill-mode)
%%% coding: utf-8-unix
%%% End:


  
