%%%%%%%%%%%%%%%%%%%%%%%%%%%%%%%%%%%%%%%%%%%%%%%%%%%%%%%%%%%%%%%%%
%%%%
%%%% The (original) Douay Rheims Bible 
%%%%
%%%% Old Testament
%%%% Tobias
%%%% Annotations
%%%%
%%%%%%%%%%%%%%%%%%%%%%%%%%%%%%%%%%%%%%%%%%%%%%%%%%%%%%%%%%%%%%%%%




\startcomponent argument


\project douay-rheims


%%% 1009
%%% o-0906
\startArgument[
  title={\Sc{Annotations} Concerning the bookes of \Sc{Tobias, Ivdith,
  Wisdom, Ecclesiasticvs,} and \Sc{Machabees}.},
  marking={Annotations.}
  ]

Proteſtantes and other Sectaries of this time denie theſe bookes to be
diuine Scripture, becauſe they are not in the Iewes Canon, nor were
accepted for canonical in the primitiue Church.
\MNote{Heretikes denie ſome ſcriptures becauſe they cõuince their
errors.}
But in deede the chiefe cauſe is, for that ſome thinges in theſe bookes,
are ſo manifeſt againſt their opinions, that they haue no other anſwere,
but to reiect their authoritie. An old ſhift noted and refuted by
\CNote{\Cite{Lib. de Peædeſt. Sanct. c.~14.}}
S.~Auguſtin touching the Booke of VViſdome, which ſome refuſed,
pretending that it was not canonical, but in deede becauſe it conuinced
their errors. For otherwiſe who ſeeth not, that
\MNote{The Churches canon of more authority thẽ the Iewes.}
the Canon of the Church of Chriſt is of more authority with al true
Chriſtians, then the Canon of the Iewes? And that the Church of Chriſt
numbreth theſe Bookes amongſt others of diuine and infallible authoritie,
is euident by the teſtimonie and diffinition, not only of later general
Councels; of Trent,
\Cite{Seſſ.~4.}
and Florence
\Cite{Inſtructione Armenorum},
of Pope Innocentius,
\Cite{Epiſt. ad Exuperium},
and Gelaſius,
\Cite{Decreto de libris ſacris};
but alſo the Councel of Carthage
\Cite{An. Dom.~419.}
S.~Auguſtin
\Cite{lib.~2. Doct. Chriſt cap.~8.}
Iſidorius
\Cite{lib.~6. Etymol. cap.~1.}
Caſſidorus
\Cite{lib.~1. Diuinarum Lectionum.}
Rabanus,
\Cite{lib.~2. de Inſtitutione Clericorum},
and others teſtifie the ſame, as we ſhal further note ſeuerally of euerie
booke, in their particular places. And for ſo much as our aduerſaries
acknowlege theſe Bookes to be Holie, and worthie to be read in the
Church, but not ſufficient to proue, and confirme pointes of faith: the
ſtudious reader may conſider that the Councel of Carthage calleth them
\Emph{Canonical, and Diuine}, which ſheweth that they are of infallible
authoritie.
\MNote{A canon is an infallible rule of direction.}
For a Canon is an aſſured rule and warrant of direction, whereby (ſayth
S.~Auguſtin 
\Cite{lib.~11. contra Fauſtum. cap.~5.}
et
\Cite{lib.~2. contra Creſconium. cap.~32.})
the infirmitie of our defect in knowlege is guided, and by which rule
other bookes are likewiſe knowne to be Gods word.
\MNote{The Goſpel is knowne by the Church.}
His reaſon is, becauſe we haue no other aſſurance that the bookes of
Moyſes, the foure Goſpels, and other bookes are the true word of God,
but by the Canon of the Church. VVherevpon the ſame great Doctor vttered
that famous ſaying: that \Emph{he vvould not beleue the Goſpel, except
the authoritie of the Catholique Church moued him thervnto}.
\Cite{contra. Epiſt. Fundamenti. ca.~5.}

True
\MNote{Bookes doubted of before the Churches definition are not doubtful
after.}
it is that ſome Catholique Doctors doubted whether theſe bookes were
Canonical or no, becauſe the Church had not then declared that they
were; but ſince the Churches declaration no Catholique doubteth. So
S.~Ierom
\CNote{\Cite{Præfat. in Iudith.}}
teſtifieth, that the Booke of Iudith (among the reſt) ſemed to him not
canonical, til the Councel of Nice declared it to be. Likewiſe the
Epiſtle to the Hebrewes, the Epiſtle of S.~Iames, the ſecond of
S.~Peter, the ſecond and third of S.~Iohn, S.~Iudes Epiſtle, and the
Apocalyps were ſometimes doubted of, yet were afterwardes declared to be
Canonical. And moſt Proteſtantes, namely Engliſh admitte them al, as the
aſſured word of God, though they were not alwaies ſo reputed by al, but
as S.~Ierom affirmeth of S.~Iames Epiſtle,
\CNote{\Cite{De viris illuſtrib. verbo Iacobus.}}
\L{Paulatim tempore procendente meruit authoritatem}. By litle and litle
in proceſſe of time merited authoritie.


\stopArgument


\stopcomponent


%%% Local Variables:
%%% mode: TeX
%%% eval: (long-s-mode)
%%% eval: (set-input-method "TeX")
%%% fill-column: 72
%%% eval: (auto-fill-mode)
%%% coding: utf-8-unix
%%% End:
