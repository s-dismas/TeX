%%%%%%%%%%%%%%%%%%%%%%%%%%%%%%%%%%%%%%%%%%%%%%%%%%%%%%%%%%%%%%%%%
%%%%
%%%% The (original) Douay Rheims Bible 
%%%%
%%%% Old Testament
%%%% Tobias
%%%% Argument
%%%%
%%%%%%%%%%%%%%%%%%%%%%%%%%%%%%%%%%%%%%%%%%%%%%%%%%%%%%%%%%%%%%%%%




\startcomponent argument


\project douay-rheims


%%% 1010
%%% o-0907
\startArgument[
  title={\Sc{The Argvment of the Booke of Tobie.}},
  marking={Argument of the Booke of Tobie.}
  ]

Beſides
\MNote{Other teſtimonies, that this Booke is canonical.}
the teſtimonies of Councels and Fathers before mentioned, S.~Cyprian,
\Cite{de Oratione Dominica},
alleaging this booke
\Cite{(cap.~11.)}
ſaith: \Emph{Diuine Scripture inſtructeth vs, that prayer is good with
faſting and almes}. S.~Ambroſe
\Cite{(li. de Tobia, c.~1.)}
calleth this booke by the common name of Scripture, ſaying: \Emph{he wil
briefly gather the vertues of Tobie, which the Scripture in hiſtorical
maner layeth forth at large}. VVhere he alſo calleth this hiſtorie
Prophetical, and Tobie a Prophet. And
\Cite{lib.~3. offic. cap.~14.}
alleageth this booke as he doth other holie Scriptures, to proue that
the vertues of Gods ſeruants farre excel the Moral
Philoſophers. S.~Chryſoſtom
\Cite{ho.~15. ad Heb.}
\CNote{\Cite{chap.~13.}}
alleageth Tobias as Scripture denouncing curſe to
contemners. S.~Auguſtin made a ſpecial ſermon of Tobias, as he did of
Iob, which is the
\Cite{226.~ſermon de tempore}.
S.~Gregorie
\Cite{parte~3. paſtor curæ admon.~21.}
alleageth it as holie Scripture. And
\CNote{\Cite{Toma~4.}
et
\Cite{in 1.~Reg.~10.}}
Venerable Beda expoundeth this whole booke myſtically, as he doth other
holy Scriptures.
\MNote{It was written in Chaldee.}
S.~Ierom tranſlated it out of the Chaldee language, wherein it was
written, \Emph{iudging it more mete to diſpleaſe the Phariſaical Iewes,
who reiect it}, then not to ſatisfie the wil of holie Bishops, vrging to
haue it.
\Cite{Epiſt. ad Chromat. & Heliodorum to.~3.}

The author is vncertaine: but S.~Athanaſius
\Cite{(in Synopſi)}
reporteth the contentes at large. And S.~Auguſtin
\Cite{(li. queſt. ex vtroque teſtamento q.~119.)}
deliuereth both the contentes, and cauſe of writing, briefly thus.
\MNote{The cõtentes.}
\Emph{The ſeruant of God, holie Tobias is geuen to vs after the law, for
an example, that we might know how to practiſe the thinges, which we
reade. And if tentations come vpon vs, not to depart from the feare of
God, nor expect helpe from anie other then from him.}
\MNote{Diuided into three partes.}
It may be diuided into three partes. The firſt foure chapters shew the
holie and ſincere maner of life of old Tobias. The eight folowing relate
the iorney, and affayres of yong Tobias,
\Fix{accompained}{accompanied}{obvious typo, fixed in other}
and directed by the Angel Raphael. In the two laſt chapters, they praiſe
God. And old Tobias prophecieth better ſtate of the commonwealth.


\stopArgument


\stopcomponent


%%% Local Variables:
%%% mode: TeX
%%% eval: (long-s-mode)
%%% eval: (set-input-method "TeX")
%%% fill-column: 72
%%% eval: (auto-fill-mode)
%%% coding: utf-8-unix
%%% End:
