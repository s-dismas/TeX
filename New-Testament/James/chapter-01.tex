%%%%%%%%%%%%%%%%%%%%%%%%%%%%%%%%%%%%%%%%%%%%%%%%%%%%%%%%%%%%%%%%%
%%%%
%%%% The (original) Douay Rheims Bible 
%%%%
%%%% New Testament
%%%% Epistles
%%%% James
%%%% Chapter 01
%%%%
%%%%%%%%%%%%%%%%%%%%%%%%%%%%%%%%%%%%%%%%%%%%%%%%%%%%%%%%%%%%%%%%%

%%% Latin checked by KK.




\startcomponent chapter-01


\project douay-rheims


%%% 2880
%%% o-2741
\startChapter[
  title={Chapter 01}
  ]

\Summary{We haue to reioyce in perſecution (but if we be patient, and
  withal abſteine from al mortal ſinne) 9.~conſidering how we shal be
  exalted and crowned for it, when the perſecutour (who enricheth
  himſelf with our ſpoiles) shal fade away. 13.~But if any be tempted to
  fal, or to any other euil, let him not ſay, God is the Authour of it,
  who is the Authour of al good only. 19.~Such points of the Cath. faith
  we muſt be content to learne without contradiction & anger, and to doe
  accordingly. 26.~Becauſe otherwiſe we may talke of Religion, but
  indeed it is no Religion.}

Iames the ſeruant of God and of our Lord \Sc{Iesvs} Chriſt, to the
twelue
\Fix{Tibes}{Tribes}{obvious typo, fixed in other}
that are in diſperſion, greeting.

\V Eſteeme it, my Brethren, al ioy, when you ſhal fal into diuers
tentations: \V knowing that
\CNote{\XRef{Ro.~5,~3.}}
the probation of your faith worketh patience. \V And let patience haue a
perfect worke: that you may be perfect & entire, failing in nothing. \V
But if any of you lacke wiſedom, let him aske of God who giueth to al
men aboundantly, and vpbraideth not: and it ſhal be giuen him. \V But
\CNote{\XRef{Mt.~21,~22.}
\XRef{Mr.~11,~24.}}
let him
\LNote{Aske in faith nothing doubting.}{The
\MNote{What faith is required in praier.}
Proteſtants would proue by this, that no man ought to pray without
aſſurance that he shal obtaine that which he asketh. Where the Apoſtle
meaneth nothing els, but that the asker of lawful things may not either
miſtruſt God's power & hability, or be in diffidence and deſpaire of his
mercie: but that our doubt be only in our owne vnworthineſſe or vndue
asking.}
aske in faith nothing doubting. For he that doubteth, is like to a waue
of the ſea, which is moued & caried about by the wind. \V Therfore let
not that man thinke that he ſhal
%%% o-2742
receiue any thing of our Lord. \V A man double of mind is inconſtant in al
his waies.

\V But let the humble Brother glorie, in his exaltation: \V and the
rich, in humilitie, becauſe
\CNote{\XRef{Pſ.~102,~15.}
\XRef{Eccl.~14,~18.}
\XRef{Eſ.~4,~6.}
\XRef{1.~Pet.~1,~24.}}
as the floure of graſſe ſhal he paſſe: \V for the ſunne roſe with heat,
& parched the graſſe, and the floure of it fel away, and the beautie of
the ſhape therof periſhed: ſo the rich man alſo ſhal wither in his
waies. \V
\CNote{\XRef{Iob.~5,~17.}}
Bleſſed is the man that ſuffereth tentation:
%%% 2881
for when he hath been proued, he ſhal receiue the crowne of life, which
God hath promiſed to them that loue him.

\V
\LNote{Let no man ſay that he is tempted of God.}{We
\MNote{God is not Authour of euil.}
ſee by this, that when the Scriptures (as in the \L{Pater noſter} and
other places) ſeeme
to ſay, that God doth ſometimes tempt vs, or lead vs into tentation;
they meane not, that God is any waies the Authour, cauſer, or mouer of
any man to ſinne, but only by permiſſion, and becauſe by his gratious
power he keepeth not the offender from tentations. Therfore the
blaſphemie of Heretikes, making God the Authour of ſinne, is
intolerable. See
\Cite{S.~Auguſt. ſer.~9. de diuerſ. c.~9.}}
Let no man when he is tempted, ſay that he is tempted of God. For
\LNote{God is not a tempter of euils.}{The
\MNote{Partial & wilful tranſlation.}
Proteſtants as much as they
may, to diminish the force of the Apoſtles concluſion againſt ſuch as
attribute euil tentations to God (for other tentations God doth ſend to
trie mens patience and proue their faith) take and tranſlate the word
paſſiuely, in this ſenſe, that God is not tempted by our euils. Where
more conſonantly to the letter & circumſtance of the words before &
after, & as agreably
\TNote{\G{ἀπείραστός κακῶν}}
to the Greeke, it should be taken actiuely as it is in the Latin, that
God is no tempter to euil. For being taken paſſiuely, there is no
coherence of ſenſe to the other words of the Apoſtle.}
God is not a tẽpter of euils, and he tẽpteth no man. \V But
\SNote{The ground of tentation to ſinne, is our cõcupiſcence, and not
God.}
euery one is tempted of his owne concupiſcence abſtracted and
allured. \V Afterward
\LNote{Concupiſcence when it hath conceiued.}{Concupiſcence
\MNote{Concupiſcence of it-ſelf no ſinne.}
(we ſee here) of it-ſelf is not ſinne, as Heretikes falſely teach: but
when by any conſent of the mind we doe obey or yeald to it, then is
ſinne ingendred and formed in vs.}
concupiſcence when it hath conceiued, bringeth forth ſinne. But
\LNote{Sinne conſummate ingendreth death.}{Here
\MNote{Not euery ſinne mortal.}
we ſee that not al ſinne nor al conſent vnto concupiſcence is mortal or
damnable, but when it is conſummate, that is, when the conſent of mans
mind fully and perfectly yealdeth to the committing or liking of the
acte or motion whereunto concupiſcence moueth or inciteth vs.}
ſinne when it is conſummate, ingendreth death.

\V Doe not erre therfore, my deareſt Brethren. \V Euery beſt guift, and
euery perfect guift, is from aboue, deſcending from the Father of lights,
with whom is no tranſmutation, nor ſhadowing of alteration. \V
Voluntarily hath he begotten vs by the word of truth, that we may be
ſome beginning of his creature. \V You know, my deareſt Brethren, And
\CNote{\XRef{Prou.~17,~27.}}
let euery man be ſwift to heare, but ſlow to ſpeake, and ſlow to anger.
\V For the anger of man worketh not the iuſtice of God.

\V For the which thing caſting away al vncleanneſſe and aboundance of
malice, in meekneſſe receiue the engraffed word, which is able to ſaue
your ſoules. \V But
\CNote{\XRef{Mat.~7,~21.}
\XRef{Ro.~2,~13.}}
be doers of the word, and not hearers only, deceauing your ſelues. \V
For if a man be a hearer of the word, and not a doer, he ſhal be
compared to a man beholding the countenance of his natiuitie in a
glaſſe. \V For he conſidered himſelf, and went his way, and by and by
forgat what an one he was. \V But he that hath looked in
\LNote{The law of perfect libertie.}{The
\MNote{What is the law of libertie in the New Teſtament.}
law of the Ghoſpel and grace of Chriſt, is called the law of libertie,
in reſpect of the yoke and burden of the old carnal ceremonies, and
becauſe Chriſt hath by his bloud of the new Teſtament deliuered al that
obey him, from the ſeruitude of ſinne & the Diuel. But not as the
Libertines and other Heretikes of this time would haue it, that in the
new Teſtament euery man may follow his owne liking & conſcience, and
may chooſe whether he wil be vnder the lawes & obedience of Spiritual
or Temporal Rulers, or no.}
the law of perfect libertie, and hath remained in it, not made a
forgetful hearer, but a doer of the worke; this man ſhal be
\SNote{Beatitude or ſaluation
\Fix{conſiteth}{conſiſteth}{obvious typo, fixed in other}
in wel-working.}
bleſſed in his deed. \V And if any man thinke himſelf to be religious,
not bridling his tongue, but ſeducing his hart, this man's religion is
vaine. \V
\LNote{Religion cleane.}{True
\MNote{Good workes a part of mans iuſtice.}
religion ſtandeth not only in talking of the Scriptures, or only faith,
or Chriſtes iuſtice: but in puritie of life, and good workes, ſpecially
of charitie and mercie done by the grace of Chriſt. This is the
Apoſtolical doctrine, and farre from the Heretical vanitie of this
time.}
Religion cleane and vnſpotted with God and the Father,
%%% o-2743
is this, to viſit pupilles and widowes in their tribulation: and to keep
himſelf vnſpotted from this world.


\stopChapter


\stopcomponent


%%% Local Variables:
%%% mode: TeX
%%% eval: (long-s-mode)
%%% eval: (set-input-method "TeX")
%%% fill-column: 72
%%% eval: (auto-fill-mode)
%%% coding: utf-8-unix
%%% End:

