%%%%%%%%%%%%%%%%%%%%%%%%%%%%%%%%%%%%%%%%%%%%%%%%%%%%%%%%%%%%%%%%%
%%%%
%%%% The (original) Douay Rheims Bible 
%%%%
%%%% New Testament
%%%% Epistles
%%%% James
%%%% Note
%%%%
%%%%%%%%%%%%%%%%%%%%%%%%%%%%%%%%%%%%%%%%%%%%%%%%%%%%%%%%%%%%%%%%%




\startcomponent note


\project douay-rheims


%%% 2880
%%% o-2741
\startChapter[
  title={}
  ]

The
\MNote{The Proteſtãts abhorre the word \Emph{Catholike}.}
word Catholike, though in the title of this Epiſtle & the reſt following
(called, the Catholike Epiſtles) it be not wholy in the ſame ſenſe as it
is in the Creed, yet the Proteſtants ſo feare and abhorre the word
altogether, that in ſome of their Bibles they leaue it cleane out,
although it be in the Greek, and in ſome they had rather tranſlate
ridiculouſly thus, \Emph{The General Epiſtle, &c.} whereas theſe are
famouſly knowen and
\CNote{\Cite{Euſeb. li.~2. hiſt. c.~22.}}
ſpecified in antiquitie by the name of Catholike Epiſtles, for that they
are written to the whole Church, not to any peculiar people or perſon,
as S.~Paules are.


\stopChapter


\stopcomponent


%%% Local Variables:
%%% mode: TeX
%%% eval: (long-s-mode)
%%% eval: (set-input-method "TeX")
%%% fill-column: 72
%%% eval: (auto-fill-mode)
%%% coding: utf-8-unix
%%% End:

