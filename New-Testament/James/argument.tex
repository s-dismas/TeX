%%%%%%%%%%%%%%%%%%%%%%%%%%%%%%%%%%%%%%%%%%%%%%%%%%%%%%%%%%%%%%%%%
%%%%
%%%% The (original) Douay Rheims Bible 
%%%%
%%%% New Testament
%%%% Epistles
%%%% James
%%%% Argument
%%%%
%%%%%%%%%%%%%%%%%%%%%%%%%%%%%%%%%%%%%%%%%%%%%%%%%%%%%%%%%%%%%%%%%

%%% Latin checked by KK.




\startcomponent argument


\project douay-rheims


%%% 2879
%%% o-2740
\startArgument[
  title={\Sc{The Argvment of the Epistle of S.~Iames}},
  marking={Argument of James}
  ]

This Epiſtle (as the reſt following) is directed ſpecially, as
%%% !!! Cite?
S.~Auguſtin ſaith, againſt the errour of only faith, which ſome held at
that time alſo, by miſconſtruing S.~Paules words. Yea not only that,
but many other errours (which then alſo were annexed vnto it, as they
are now) doth this Apoſtle here touch expreſly.

He ſaith therfore, that not only faith, but alſo good workes are
neceſſarie: that not only faith, but alſo good workes doe iuſtifie: that
they are acts of Religion, or ſeruice and worship of God: that to keep
al the commandements of God, and ſo to abſtaine from al mortal ſinne, is
not impoſsible, but neceſſarie: that God is not author of ſinne, no not
ſo much as of tentation to ſinne: that we muſt ſtay our ſelues from
ſinning, with feare of our death, of the Iudgement, of hel: and ſtirre
our ſelues to doing of good, with our reward that we shal haue for it in
heauen. Theſe points of the Catholike faith he commendeth earneſtly vnto
vs, inueighing vehemently againſt them that teach the contrarie
errours. Howbeit he doth withal admonish not to neglect ſuch, but to
ſeeke their conuerſion, shewing them how meritorious a thing that
is. Thus then he exhorteth generally to al good workes, & dehorteth from
al ſinne. But yet alſo namely to certaine, and from certaine: as, from
acception of perſons, from detraction and rash iudging, from
concupiſcence and loue of this world, from ſwearing: and to praier, to
almes, to humilitie, confeſsion and penance: but moſt comiouſly to
patience in perſecution.

Now,
\MNote{Which Iames wrot this Epiſtle.}
who this Iames was: It is not he, whoſe feaſt the Church keepeth the
25.~of Iulie, which was S.~Iohns brother, and whoſe martyrdom we haue
\Fix{\XRef{Actor.~12.}}{\XRef{Act.~12.}}{probable typo, fixed in other}
but he, whom the Church worshippeth the firſt of Maie, who is called
\L{Frater Domini}, \Emph{our Lordes brother}, and brother to Iude, and
which was the firſt Bishop of Hieruſalem, of whom we read,
\XRef{Act.~15.}
and
\XRef{21.}
and alſo
\XRef{Gal.~2.}
of whoſe wonderful auſteritie and puritie of life, the Eccleſiaſtical
ſtories doe report.
\Cite{Euſeb. li.~2. c.~22.}
\Cite{Hiero. in Catalogo.}

Therfore as the old High-Prieſt had power and charge ouer the Iewes, not
only in Hieruſalem and Iurie, but alſo diſperſed in other Countries (as
we vnderſtand
\XRef{Act.~9. v.~1. &~2.)}
ſo S.~Iames likewiſe, being Bishop of Hieruſalem, and hauing care not
only of thoſe Iewes with whom he was reſident there in Iurie, but of al
the reſt alſo, writeth this Epiſtle,
\CNote{\XRef{Ia.~1.}}
\Emph{To the twelue Tribes that are in diſperſion}. And in them, to al
Chriſtians vniuerſally diſperſed through the world.


\stopArgument


\stopcomponent


%%% Local Variables:
%%% mode: TeX
%%% eval: (long-s-mode)
%%% eval: (set-input-method "TeX")
%%% fill-column: 72
%%% eval: (auto-fill-mode)
%%% coding: utf-8-unix
%%% End:
