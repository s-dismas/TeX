%%%%%%%%%%%%%%%%%%%%%%%%%%%%%%%%%%%%%%%%%%%%%%%%%%%%%%%%%%%%%%%%%
%%%%
%%%% The (original) Douay Rheims Bible 
%%%%
%%%% New Testament
%%%% Epistles
%%%% James
%%%% Chapter 02
%%%%
%%%%%%%%%%%%%%%%%%%%%%%%%%%%%%%%%%%%%%%%%%%%%%%%%%%%%%%%%%%%%%%%%

%%% Latin checked by KK.




\startcomponent chapter-02


\project douay-rheims


%%% 2882
%%% o-2743
\startChapter[
  title={Chapter 02}
  ]

\Summary{Againſt acception of perſons. 10.~From al and euery ſinne we
muſt abſteine, hauing in al our words and deeds, the Iudgement before
our eyes: wherin workes of mercie shal be required of vs, 14.~and only
faith, shal not auaile vs. 18.~And that the Catholike by his workes
sheweth that he hath faith: whereas the Heretike hath no more faith
then the Diuel, talke he of faith neuer ſo much, and of iuſtification
thereby only, by the example of Abraham
\XRef{Ro.~4.}
For Abraham indeed was iuſtified by workes alſo, 25.~and likewiſe
Rahab.}

%%% o-2744
My Brethren,
\CNote{\XRef{Leu.~19,~15.}
\XRef{Deu.~1,~16.}
\XRef{Pro.~24,~23.}
\XRef{Eccl.~42,~1.}}
Haue not the faith of our Lord \Sc{Iesvs} Chriſt of glorie
\LNote{In acception of perſons.}{The
\MNote{Scripture abuſed by the Anabaptiſtes to make no diſtinction of
perſons.}
Apoſtle meaneth not, as the Anabaptiſts and other ſeditious perſons
ſometime gather hereof that there should be no difference in
Common-weales or aſſemblies betwixt the Magiſtrate and the ſubiect, the
free man and the bond, the rich and the poore, betwixt one degree &
another: for God and nature, and the neceſſitie of man, haue made ſuch
diſtinctions, and men are bound to obſerue them.
\MNote{What the Apoſtle meaneth by acception of perſons.}
But it is meant only, or ſpecially, that in ſpiritual guifts and graces,
in matters of faith, Sacraments, and ſaluation, and beſtowing the
ſpiritual functions and charge of ſoule, we muſt eſteeme of a poore man
or a bond man, no leſſe then of the rich man and the free, then of the
Prince or the Gentleman: becauſe as Chriſt himſelf calleth al, and
endoweth al ſorts with his graces; ſo in ſuch and the like things we
muſt not be partial, but count al to be fellowes, Brethren, and members
of one head. And therfore the Apoſtle ſaith with a ſpecial clauſe, That
we should not hold or haue the Chriſtian faith with or in ſuch
differences or partialities.}
in acception of perſons. \V For if there ſhal enter into your aſſemblie
a man hauing a golden ring in goodly apparel, and there ſhal enter in a
poore man in homely attire, \V and you haue reſpect to him that is
clothed with the goodly apparel, and ſhal ſay to him, Sit thou here wel:
but ſay to the poore man, Stand thou there, or ſit vnder my
foot-ſtoole: \V doe you not iudge with your ſelues, and are become
iudges of vniuſt cogitations? \V Heare, my deareſt Brethren: hath not
God choſen the poore in this world, rich in faith, and heires of the
Kingdom which God hath promiſed to them that loue him? \V But you haue
diſhonoured the poore man. Doe not the rich oppreſſe you by might: and
thẽſelues draw you to iudgemẽts? \V Doe not they blaſpheme the good name
that is inuocated vpon you? \V If not-withſtanding you fulfil the roial
law according to the Scriptures,
\CNote{\XRef{Leu.~19,~18.}
\XRef{Mat.~22,~39.}
\XRef{Rom.~13.}}
\Emph{Thou shalt loue thy neighbour as
\Fix{they}{thy}{obvious typo, fixed in other}
ſelf}, you doe wel: \V But if you accept perſõs, you worke ſinne,
reproued of the Law as
\Fix{trãgreſſours.}{tranſgreſſours.}{obvious typo, fixed in other}
\V And
\CNote{\XRef{Leu.~19,~37.}
\XRef{Deu.~1,~18}}
whoſoeuer ſhal keep the whole Law, but offẽdeth in one,
\LNote{Is made guilty of al.}{He
\MNote{How he that offendeth in one commandement, is guilty of al.}
meaneth not, that whoſoeuer is a theefe, is alſo a murderer, or that
euery murderer is an aduouterer alſo: or that al ſinnes be equal,
according to the Stoikes & the Hereſie of Iouinian: much leſſe, that he
shal haue as great damnation that tranſgreſſeth one commandement, as if
he had offended againſt euery precept: but the ſenſe is, that it shal
not auaile him to ſaluation, that he ſeemeth to haue kept certaine & not
broken al the commandements: ſeeing that any one tranſgreſſion of the
law, proueth that he hath not obſerued the whole, which he was bound to
doe, ſo farre as is required, & as is poſſible for a man in this
life. S.~Auguſtin diſputing profoundly in his
\Cite{29.~Epiſtle to S.~Hierom,}
of this place of S.~Iames, expoundeth it thus: that he which offendeth
in one, that is, againſt the general and great commandement of loue or
charitie (becauſe it is in a manner al, as being the ſumme of al, the
plenitude of the law, and the perfection of the reſt) breaketh after a
ſort and tranſgreſſeth al, no ſinne being committed but either againſt
the loue of God, or of our neighbour.}
is made guilty of al. \V For he that ſaid, Thou ſhalt not commit
aduoutrie, ſaid alſo, Thou ſhalt not kil. And if thou doe
%%% 2883
not commit aduoutrie, but ſhal kil; thou art made a tranſgreſſour of the
Law. \V So ſpeake ye, and ſo doe, as beginning to be iudged by the law
of libertie. \V For
\LNote{Iudgement without mercie.}{Nothing
\MNote{Workes of mercie exceeding grateful to God.}
giueth more hope of mercie in the next life, then the workes of almes,
charitie, and mercie, done to our neighbours in this life. Neither shal
any be vſed with extreme rigour in the next world, but ſuch as vſed not
mercie in this world.
\Cite{Auguſt. de pec. merit. li.~2. c.~3.}
Which is true, not only in reſpect of the iudgement to euerlaſting
damnation, but alſo of the temporal chaſtiſement in Purgatorie, as
S.~Auguſtin ſignifieth, declaring that our venial ſinnes be washed away
in this world with daily workes of mercie, which otherwiſe should be
chaſtiſed in the next. See
\Cite{epiſt.~29. aforeſaid in fine,}
and
\Cite{li.~21. de Ciu. Dei. c.~17. in fine.}}
iudgement without mercie to him that hath not done mercie. And mercie
\TNote{\G{κατακαυχᾶται}}
exalteth it-ſelf aboue iudgement.

\V
\LNote{What shal it profit, if a man ſay he hath faith?}{This
\MNote{The proud and impudent dealing of the heretikes againſt this
Epiſtle, becauſe it is ſo plaine againſt only faith.}
whole paſſage of the Apoſtle is ſo cleere againſt iuſtification or
ſaluation by only faith, damnably defended by the Proteſtants, & ſo
euident for the neceſſitie, merit, & concurrence of good workes, that
their firſt Authour Luther and ſuch as exactly follow him, boldly (after
the manner of Heretikes) when they can make no shift nor falſe gloſſe for
the text, deny the booke to be Canonical Scripture. But Caluin and his
companions diſagreeing with their Maiſters, confeſſe it to be holy
Scripture. But their shiftes & fond gloſſes for anſwer of ſo plaine
places, be as impudent as the denying of the Epiſtle was in the other:
who would neuer haue denied the booke, thereby to shew themſelues
Heretikes, if they had thought thoſe vulgar euaſions that the Zuinglians
and Caluiniſts doe vſe (wherof they were not ignorant) could haue
ſerued. In both ſorts the Chriſtian Reader may ſee, that al the
Heretikes vanting of expreſſe Scriptures & the word of God, is no more
but to delude the world. Whereas indeed, be the Scriptures neuer ſo
plaine againſt them, they muſt either be wreſted to ſound as they ſay,
or els they muſt be no Scriptures at al. And to ſee Luther, Caluin,
Beza, & their fellowes, ſit as it were in iudgement of the Scriptures to
allow or diſallow at their pleaſures, it is the moſt notorious example
of Heretical pride & miſerie that can be. See their prefaces and
cenſures vpon this Canonical Epiſtle, the Apocalypſe, the Machabees, and
other.}
What ſhal it profit, my Brethren, if a man ſay he hath faith, but hath
not workes? Shal faith be able to ſaue him? \V And
\CNote{\XRef{Io.~3,~17.}}
if a Brother or Siſter be naked, and lacke daily food, \V and one of you
ſay to them, Goe in peace, be warmed and filled; but you giue them not
the things that are neceſſarie for the bodie; what ſhal it profit? \V So
faith alſo, if it haue not workes, is dead in it-ſelf. \V But ſome man
ſaith, Thou haſt faith, and I haue workes: ſhew me thy faith without
workes; and I wil ſhew thee by workes my faith. \V Thou beleeueſt that
there is one God. Thou doeſt wel: the Diuels
%%% o-2745
alſo beleeue and tremble. \V But wilt thou know,
\SNote{He ſpeaketh to al heretikes that ſay, faith only without workes
doth iuſtifie, calling them vaine men, and comparing them to Diuels.}
ô vaine man, that faith without workes is
\Var{idle?}{dead?}
\V
\LNote{Abraham, was he not iuſtified by workes?}{It
\MNote{Only faith, an old hereſie.}
is much to be noted that S.~Auguſtin in his booke
\Cite{de fide & operibus c.~14.}
writeth, that the hereſie of only faith iuſtifying or ſauing, was an old
Hereſie euen in the Apoſtles time, gathered by the falſe interpretation
of ſome of S.~Paules profound diſputation in the
\XRef{Epiſtle to the Romans,}
wherin he commended ſo highly the faith in Chriſt, that they thought
good workes were not auailable: adding further, that
\MNote{S.~Iames & the reſt inculcate good workes againſt the errour of
only faith falſely gathered of S.~Paules words.}
the other three
Apoſtles, Iames, Iohn, and Iude, did of purpoſe write ſo much of good
workes, to correct the ſaid errour of only faith, gathered by the
miſconſtruction of S.~Paules words. Yea when S.~Peter
\XRef{(Ep.~2. c.~3.)}
warneth the faithful that many things be hard in S.~Paules writings, and
of light vnlearned men miſtaken to their perdition; the ſaid S.~Auguſtin
\CNote{\Cite{loco citato.}}
affirmeth, that he meant of his diſputation concerning faith, which ſo
many Heretikes did miſtake to condemne good workes. And in the
\Cite{preface of his commentarie vpon the 81.~Pſalme,}
he warneth al men, that this deduction vpon S.~Paules
ſpeach, \Emph{Abraham was iuſtified by faith, therfore workes be not
neceſſarie to ſaluation}: is the right way to the gulfe of Hel and
damnation.

And
\MNote{S.~Auguſtines whole diſputation in this point very notable, &
directly againſt only faith.}
laſtly (which is in it-ſelf very plaine) that we may ſee this Apoſtle
did purpoſely thus commend vnto vs the neceſſitie of good workes, & the
inanity and inſufficiencie of only faith, to correct the errour of ſuch
as miſconſtrued S.~Paules words for the ſame:
\CNote{Li.~83. q.~q.~76.}
the ſaid holy Doctour noteth that of purpoſe he tooke the very ſame
example of Abraham, whom S.~Paul ſaid to be iuſtified by faith, and
declareth that he was iuſtified by good workes, ſpecifying the good
worke for which he was iuſtified and bleſſed of God, to wit, his
obedience and
\Fix{immalation}{immolation}{likely typo, fixed in other}
of his only ſonne. But how S.~Paul ſaith that Abraham was iuſtified by
faith, ſee the
\XRef{Annotations vpon that place, Ro.~4. v.~1.}}
Abraham our Father was he not iuſtified by workes
\CNote{\XRef{Gn.~22,~10.}}
offering Iſaac his ſonne vpon the altar? \V Seeſt thou that
\LNote{Faith did worke with.}{Some
\MNote{Hereſies againſt good workes.}
Heretikes hold, that good workes are pernicious to ſaluation and
iuſtification: other, that though they be not hurtful but required, yet
they be no cauſes or workes of ſaluation, much leſſe meritorious, but
are as effects and fruits iſſuing neceſſarily out of faith. Both which
fictions, falshoods, & flights from the plaine truth of God's word, are
refuted by theſe words, when the Apoſtle ſaith,
\MNote{Workes concurre with faith as cauſe of iuſtification.}
That faith worketh together with good workes: making faith to be a
coadiutour or cooperatour with workes, and ſo both ioyntly concurring as
cauſes and workers of iuſtification: yea afterward he maketh workes the
more principal cauſe, when he reſembleth faith to the body, and workes
to the ſpirit or life of man.}
faith did worke with his workes: and by the workes the faith was
conſummate? \V And the Scripture was fulfilled, ſaying,
\CNote{\XRef{Gen.~15,~6.}
\XRef{Ro.~4,~3.}
\XRef{Gal.~3.}}
\Emph{Abraham beleeued God, and it was reputed him to iuſtice, and he
was called
\LNote{The freind of God.}{By
\MNote{Workes make vs iuſt indeed before God.}
this alſo another falſe and friuolous
euaſion of the Heretikes is ouertaken, when they feine, that the Apoſtle
here when he ſaith, workes doe iuſtifie, meaneth that they shew vs iuſt
before men, and auaile not to our iuſtice before God. For the Apoſtle
euidently declareth that Abraham by his workes was made or truely called
the freind of God, and therfore was not (as the Heretikes ſay) by his
workes approued iuſt before man only.}
the freind of God.} \V Doe you ſee that by workes a man is iuſtified; &
\LNote{Not by faith only.}{This
\MNote{The Proteſtãts ſay \Emph{by faith only}: S.~Iames cleane
contrarie, \Emph{Not by faith only}.}
propoſition or ſpeach is directly oppoſit or contradictorie to that
which the Heretikes hold. For the Apoſtle ſaith, Man is iuſtified by
good workes, and not by faith only. But the Heretikes ſay, Man is not
iuſtified by good workes, but by faith only. Neither can they pretend
that there is the like contradiction or contrarietie betwixt S.~Iames
ſpeaches and S.~Paules. For though S.~Paule ſay, man is iuſtified by
faith, yet he 
neuer ſaith, by faith only, nor euer meaneth by that faith which is
alone, but alwaies by that faith which worketh by charitie,
\CNote{\XRef{Gal.~5.}}
as he expoundeth himſelf. Though concerning workes alſo, there is a
difference betwixt the firſt iuſtification, whereof S.~Paul ſpecially
ſpeaketh: and the ſecond iuſtification, whereof S.~Iames doth more
ſpecially treate. Of which thing
\SNote{See the
\XRef{annot. vpon the epiſtle to the Romans c.~2. v.~11. }}
els-where there is enough ſaid.

The
\MNote{The manifold meaning of certaine Fathers, when they ſay,
\Emph{Only faith}.}
Fathers indeed vſe ſometimes this excluſiue, \L{ſola}, \Emph{only}, but
in farre other ſenſe then the Proteſtants. For ſome of them thereby
exclude only the workes of Moyſes law, againſt the Iewes: ſome, the
workes of nature and moral vertues without the grace or knowledge of
Chriſt, againſt the Gentils: ſome, the neceſſitie of external good
workes where the parties lacke time and meanes to doe them, as in the
caſe of the penitent theefe: ſome, the falſe opinions, ſectes, and
religions contrarie to the Catholike faith, againſt Heretikes and
miſcreants: ſome exclude reaſon, ſenſe, and arguing in matters of faith
and myſterie, againſt ſuch as wil beleeu nothing but that they ſee or
vnderſtand: ſome the merit of workes done in ſinne before the firſt
iuſtification: ſome, the arrogant Phariſaical vanting of man's owne
proper workes and iuſtice, againſt ſuch as referre not their actions and
good deeds to God's grace. To theſe purpoſes the holy Doctours ſay
ſometimes, that only faith ſaueth and ſerueth: but neuer (as Proteſtants
would haue it) to exclude from iuſtification and ſaluation, the
cooperation of mans free-wil, diſpoſitions and preparations of our harts
by praiers, penance, and ſacramẽts, the vertues of hope and charitie,
the purpoſe of wel-working and of the obſeruation of God's
commandements: much leſſe, the workes and merits of the children of God,
proceeding of grace and charitie, after they be iuſtified & are now in
his fauour: which are not only diſpoſitions and preparations to iuſtice,
but the meritorious cauſe of greater iuſtice, and of ſaluation.}
not by faith only? \V And in like manner alſo
\CNote{\XRef{Ioſ.~2,~1.~18.}
and
\XRef{6,~22.}}
\LNote{Rahab.}{This
\MNote{S.~Paul nameth faith & S.~Iames workes, cauſes of iuſtification:
but neither the one, faith only, nor the other, workes only.}
Apoſtle alleageth the good workes of Rahab by which she was iuſtified,
and S.~Paul
\XRef{\XRef{(11.~Heb.)}}
ſaith she was iuſtified by faith. Which are not contrarie one to the
other: for both is true that she was ſaued by faith, as one ſaith, and
that she was ſaued by her workes, as the other ſaith. But it were
vntruely ſaid, that she was ſaued either by only faith as the Heretikes
ſay; or by only good workes, as no Catholike man euer ſaid. But becauſe
ſome Iewes and Gentil Philoſophers did affirme; they, that they should
be ſaued by the workes of Moyſes law; theſe, by their moral workes:
therfore S.~Paul to the Romans diſputed ſpecially againſt both, prouing
that no workes done without or before the faith of Chriſt, can ſerue to
iuſtification or ſaluation.}
Rahab the harlot, was not ſhe iuſtified by workes, receiuing the
meſſengers, and putting them forth another way? \V For euen as the bodie
without the ſpirit is dead: ſo alſo
\LNote{Faith without workes is dead.}{S.~Iames
\MNote{Faith without workes is a true faith, but not auailable: as the
body without the ſpirit is a true body, though it be dead.}
(as the Proteſtants feine) ſaith that faith without good workes is no
faith, and that therfore it iuſtifieth not, becauſe it is no faith; for
he ſaith that it is dead without workes as the bodie is dead without the
ſoule, and therfore being dead hath no actiuity or efficacie to iuſtifie
or ſaue. But it is a great difference, to ſay that the body is dead, and
to ſay that it is no body: euen ſo it is the like difference, to ſay
that faith without workes is dead, and to ſay that faith without workes
is no faith. And if a dead body be not-withſtanding a true body, then
according to S.~Iames compariſon here, a dead faith is not-withſtanding
a true faith, but yet not auailable to iuſtification, becauſe it is
dead, that is, becauſe it is only faith without good workes.

And
\MNote{What faith the Apoſtle ſpeaketh of: & that he knew no ſpecial
faith.}
therfore it is a great impudencie in Heretikes, and a hard shift, to ſay
that the faith of which the Apoſtle diſputeth al this while, is no true
or properly called faith at al. It is the ſame faith that S.~Paul
defined and commended in al the
\XRef{11.~chapter to the Hebrewes,}
and the ſame which is called the Catholike faith, and the ſame which
being formed & made aliue by charitie, iuſtifieth. Mary true it is, that
it is not that ſpecial faith which the Heretikes feine only to iuſtifie,
to wit, when a man doth firmely beleeue as an article of his faith, that
himſelf shal be ſaued. This ſpecial faith it is not whereof the Apoſtle
here ſpeaketh. For neither he, nor S.~Paul, nor any other ſacred Writer
in al the holy Scriptures euer ſpeake or knew of any ſuch forged faith.}
faith without workes is dead.


\stopChapter


\stopcomponent


%%% Local Variables:
%%% mode: TeX
%%% eval: (long-s-mode)
%%% eval: (set-input-method "TeX")
%%% fill-column: 72
%%% eval: (auto-fill-mode)
%%% coding: utf-8-unix
%%% End:

