%%%%%%%%%%%%%%%%%%%%%%%%%%%%%%%%%%%%%%%%%%%%%%%%%%%%%%%%%%%%%%%%%
%%%%
%%%% The (original) Douay Rheims Bible 
%%%%
%%%% New Testament
%%%% James
%%%% Chapter 05
%%%%
%%%%%%%%%%%%%%%%%%%%%%%%%%%%%%%%%%%%%%%%%%%%%%%%%%%%%%%%%%%%%%%%%




\startcomponent chapter-05


\project douay-rheims


%%% 2888
%%% o-2750
\startChapter[
  title={Chapter 05}
  ]

\Summary{By the damnation to come vpon the vnmerciful rich, he exhorteth
  the perſecuted to patience, & by their owne reward, and by
  examples. 12.~Not to ſweare at al in common talke. 13.~In affliction,
  to pray: in proſperitie, to ſing: in ſicknes, to cal for the Prieſts,
  and that they pray ouer them & anoile them with oile: and that the
  ſicke perſons confeſſe their ſinnes. 19.~Finally, how meritorious it
  is to conuert the erring vnto the Catholike faith, or the ſinner to
  amendement of life.}

Goe to now ye rich men, weep,
\SNote{A feareful deſcription of the miſeries that shal befal in the
next life to the vnmerciful couetous men.}
howling in your miſeries which ſhal come to you. \V Your riches are
corrupt; and your garments are eaten of moths. \V Your gold and ſiluer
is ruſted; and their ruſt ſhal be for a teſtimonie to you, and ſhal eate
your fleſh as fire. You haue ſtored to your ſelues wrath in the laſt
daies.

\V Behold
\LNote{The hire.}{To
\MNote{The ſinnes crying to Heauen.}
with-hold from the poore or labourer the hire or wages that is due or
promiſed to him for his ſeruice or worke done, is a great iniquitie, and
one of thoſe fiue ſinnes which in holy writ be ſaid to cal for vengeance
at God's hand, as we ſee here. They be called in the Catechiſme,
\Emph{Sinnes crying to heauen}. The other foure be, murder,
\XRef{Gen.~18. v.~20.}
Vſurie,
\XRef{Exod.~22. v.~27.}
The ſinne againſt nature,
\XRef{Gen.~18. v.~20.}
The oppreſſion and vexation of widowes, pupilles, ſtrangers, and ſuch
like.
\XRef{ib.}
&
\XRef{Exod.~3. v.~9.}}
the hire of the worke-men that haue reaped your fields, which is
defrauded of you, crieth:
%%% !!! XRef?
and their crie hath entred into
\Fix{the the}{the}{obvious typo, fixed in other}
eares of
the Lord of Sabboth. \V You haue made merie vpon the earth: and in
riotouſnes you haue nouriſhed your harts in the day of ſlaughter. \V You
haue
\Var{preſented}{condẽned}
and ſlaine the iuſt one: and he reſiſted you not.

\V Be patient therfore, Brethren, vntil the comming of our Lord. Behold,
the husband-man expecteth the pretious fruit of the earth: patiently
bearing til he receiue
\SNote{He meaneth either fruit or raine.}
the timely and the lateward. \V Be you alſo patient, and confirme your
harts: becauſe the comming of our Lord
\Var{wil approch.}{is at hand.}
\V Grudge not, Brethren, one againſt another: that you be not
iudged. Behold, the iudge ſtandeth before the gate. \V Take an example,
Brethren, of
%%% !!! Really a Var for 'labour'
\TNote{\L{exitus mali}}
labour and patience, the Prophets, which ſpake in the name of our
Lord. \V Behold we account them bleſſed that haue ſuffered. The
ſufferance of Iob you haue heard, and the end of our Lord you haue ſeen,
becauſe our Lord is merciful and pitieful. \V But before al things, my
Brethren,
\CNote{\XRef{Mt.~5,~34.}}
\LNote{Sweare not.}{He
\MNote{What othes are lawful, what are not.}
forbiddeth not al othes, as the Anabaptiſts falſely ſay. For in iuſtice
and iudgemẽt we may be by our lawful Magiſtrate put to ſweare, and may
lawfully take an othe, as alſo for the aduantaging of any neceſſarie
truth when time and place require. But the cuſtom of ſwearing, and al
vaine, light, & vnneceſſarie othes in our daily ſpeach doe diſpleaſe God
highly, and are here forbidden by the Apoſtle, as alſo by our Sauiour.
\XRef{Mat.~5.}}
ſweare not, neither by heauen,
%%% 2889
nor by earth; nor other othe whatſoeuer. But let your talke be, yea,
yea: no,
%%% o-2751
no: that you fal not vnder iudgement.

\V Is any of you in heauineſſe? let him pray. Is he of cheereful hart?
let him ſing. \V Is any man ſicke among you?
\LNote{Let him bring in the Prieſts.}{The
\MNote{Heretical trãſlation againſt Prieſthood.}
Proteſtants for their ſpecial hatred of the holy order of Prieſthood, as
els-where often, ſo here they corrupt the text euidently, tranſlating
\GG{Preſbyteros}, Elders. As though the Apoſtle had meant men of age,
and not ſuch as were by holy office, Prieſts. S.~Chryſoſtom who knew the
ſenſe and ſignification of the Greeke word according to the
Eccleſiaſtical vſe and the whole Churches iudgement, better then any
Proteſtant aliue, taketh it plainely for \L{Sacerdotes}, that is,
Prieſts.
\Cite{li.~3. Sacerdotie prope initium.}
And if they confeſſe that it is a word of office with them alſo, though
they cal them Elders, and not Prieſts; then we demand whether the
Apoſtle meane here men of that function which they in their new Churches
cal Elders.
\MNote{Neither their \Emph{Elders} (ſo called) nor their Miniſters, can
be thoſe whom the Apoſtle here calleth, \GG{Preſbyteros}.}
If they ſay no, as they muſt needs (for Elders with them are not deputed
ſpecially to publike praying or adminiſtration of the Sacraments, ſuch
as the Apoſtle here requireth to be ſent for) then they muſt needs
grant, that their Elders anſwer not to the function of thoſe which in
the new Teſtament are called \GG{Preſbyteri} in Greek and Latin, and
therfore both their tranſlation to be falſe and fraudulent, and alſo
their naming of their new degrees or orders to be fond and incongruous.

If
\MNote{They haue no reaſon to cal their Miniſters by that name.}
they ſay their Miniſters be correſpondent to ſuch as were called
\GG{Preſbyteri} in holy writ & in the Primitiue Church, & that they are
the man whom the Apoſtle willeth to be called for to anoile the ſicke &
to pray for him, why doe they not thẽ tranſlate \GG{Preſbyteros}
Miniſters? which they might doe with as good reaſon, as cal ſuch as they
haue taken inſteed of our Catholike Prieſts, Miniſters. Which word being
in large acception cõmon to al that haue to doe about the celebration of
diuine things, was neuer appropriated by vſe either of Scripture or of
the holy Church, to that higher function of publike adminiſtration of
the Sacraments and Seruice, which is Prieſthood: but to the order next
vnder it, which is Deaconship.
\MNote{Their Deacons ſhould rather be called Miniſters.}
And therfore if any should be called Miniſters, their Deacons properly
should be ſo termed. And the Proteſtants haue no more reaſon to keep
the ancient Greek word of Deacons, appropriated to that office by the vſe
of antiquity, then to keep the word Prieſt, being made no leſſe peculiar
to the ſtate of ſuch only as Miniſter the holy Sacraments, & offer the
Sacrifice of the Altar. But theſe fellowes follow neither God's word,
nor Eccleſiaſtical vſe, nor any reaſon, but mere phantaſie, noueltie,
and hatred of God's Church.
\MNote{They ſhould keep the name Prieſt, as wel as Deacon.}
And how litle they follow any good rule or reaſon in theſe things may
appeare by this, that here they auoid to tranſlate \Emph{Prieſts}, and
yet in their Communion booke, in their order of viſiting the ſicke, they
commonly name the Miniſter, \Emph{Prieſt}.}
let him bring in the Prieſts of the Church, and let them pray ouer him,
\CNote{\XRef{Mr.~6,~13.}}
\LNote{Anoiling with oile.}{Here
\MNote{The Sacrament of \Sc{Extreme Vnction}.}
is the Sacrament of extreme Vnction ſo plainely promulgated (for it was
inſtituted, as al other Sacraments of the new Teſtament, by our Sauiour
Chriſt himſelf, and, as Venerable Bede thinketh and other ancient
Writers, the anoiling of the ſicke with oile
\XRef{Marc.~6.}
pertaineth thereunto) that ſome Heretikes, for the euidence of this
place alſo (as of the other for good workes) deny the Epiſtle. Other (as
the Caluiniſts) through their confidence of cunning shifts and gloſſes,
confeſſing that S.~Iames is the Authour, yet condemne the Church of God
for vſing and taking it for a Sacrament. But what diſhonour to God is it
(we pray them) that a Sacrament should be inſtituted in the matter of
oile, more then in the element of water? Why may not grace and remiſſion
of ſinnes be annexed to the one as wel as to the other, without
derogation to God?

But
\MNote{The Heretikes obiections againſt the ſaid Sacrament, anſwered:
and withal it is proued to be a Sacrament.}
they ſay, Sacraments endure for euer in the Church, this but for a
ſeaſon in the Primitiue Church. What Scripture telleth them that this
general and abſolute preſcription of the Apoſtle in this caſe should
endure but for a ſeaſon? When was it taken away, abrogated, or
altered? They ſee the Church of God hath alwaies vſed it vpon this
warrant of the Apoſtle, who knew Chriſt's meaning and inſtitution of it
better then theſe deceiued men, who make more of their owne fond gueſſes
& coniectures, grounded neither on Scripture nor vpon any circumſtãce of
the text, nor any one authentical Authour that euer wrote, then of the
expreſſe word of God. It was (ſay they) a miraculous practiſe of healing
the ſicke, during only in the Apoſtles time, and not long after. We aske
them whether Chriſt appointed any certaine creature or external element
vnto the Apoſtles generally to worke miracles by. Himſelf vſed ſometimes
clay & ſpittle, ſometimes he ſent thẽ that were diſeaſed, to waſh
themſelues in waters: But that he appointed any of thoſe or the like
things for a general medicine or miraculous healing only, that we read
not. For in the beginning, for the better inducing of the people to
faith and deuotion, Chriſt would haue miracles to be wrought by ſundry
of the Sacraments alſo. Which miraculous workes ceaſing, yet the
Sacraments remaine ſtil vnto the worldes end.

Againe we demand, whether euer they read or heard that men were
generally commanded to ſeeke for their health by miraculous meanes?
Thirdly, whether al Prieſts, or (as they cal them) Elders, had the guift
of miracles in the primitiue Church? No, it can not be. For though ſome
had, yet al theſe indifferently of whom the Apoſtle ſpeaketh, had not
the guift: and many that were not Prieſts had it, both men and women,
which yet could not be called for as Prieſts were in this caſe. And
though the Apoſtle and others could both cure men and reuiue them
againe, yet there was no ſuch general precept for ſicke or dead men, as
this, to cal for the Apoſtles to heale or reſtore them to life againe.
\MNote{Remiſſion of ſinnes annexed to creatures.}
Laſtly had any external element or miraculous practiſe, vnles it were a
Sacrament, the promiſe of remiſſion of al kind of actual ſinnes ioyned
vnto it? Or could S.~Iames inſtitute ſuch a ceremonie himſelf, that
could ſaue both, body and ſoule by giuing health to the one, and grace
and remiſſion to the other?
\MNote{Holy water.}
At other times theſe contentious wranglers, raile at God's Church, for
annexing only the remiſſion of venial ſinnes to the element of water,
made holy by the Prieſts bleſſing thereof in the name of Chriſt, and his
word: and loe here they are driuen to hold that S.~Iames preſcribed a
miraculous oile or creature which had much more power & efficacie. Into
theſe ſtraits are ſuch miſcreants brought that wil not beleeue the
expreſſe word of God, interpreted by the practiſe of God's vniuerſal
Church.

Venerable
\MNote{Holy oile bleſſed by the Bishop.}
Bede
\Cite{in 9.~Luc.}
ſaith thus. \Emph{It is cleere that this cuſtome was deliuered to the
holy Church by the Apoſtles themſelues, that the ſicke should be
anointed with oile conſecrated by the Bishops bleſſing.} See for this &
for the aſſertion & vſe of this Sacrament,
\Cite{S.~Innocentius ep.~1. ad Decentium Eugubinæm c.~8. to.~3. Conc.}
&
\Cite{l.~2. de viſitatione infirmorum in S.~Auguſtin c.~4.}
\Cite{Concil. Cabilonenſe~2. cap.~48.}
\Cite{Concil. Wormationſe cap.~72. to.~3. Conc.}
\Cite{Aquiſgra. c.~8. Florentium,}
and other later Councels.
\Cite{S.~Bernard in the life of Malachie in fine.}
\MNote{The peoples deuotion toward ſuch hallowed creatures.}
This holy oile becauſe the faithful ſaw to haue ſuch vertue in the
primitiue Church, diuers caried it home and occupied it in their
infirmities, not vſing it in the Sacramental ſort which the Apoſtle
preſcribeth, as the Aduerſaries vnlearnedly obiect vnto vs: but as
Chriſtians now doe (and then alſo did) concerning the water of Baptiſme,
which they vſed to take home with thẽ after it was hallowed, & to giue
it their diſeaſed to drinke.}
anoiling him with oile in the name of our Lord. \V And
\LNote{The praier of faith.}{He
\MNote{The Sacramental words.}
meaneth the forme of the Sacrament, that is, the words ſpoken at the
ſame time when the partie is anoiled, which no doubt are moſt ancient &
Apoſtolike. Not that the word or praier alone should haue that great
effect here mentioned, but ioyned with the foreſaid vnction, as is
plaine.}
the praier of faith
\LNote{Shal ſaue.}{The
\MNote{The three effects of this Sacrament.}
firſt effect of this Sacrament is, to ſaue the ſoule, by giuing grace &
comfort to withſtand the terrours and tentations of the enemie, going
about (ſpecially in that extremitie of death) to driue men to
deſperation or diſtreſſe of mind and other damnable inconueniences. The
which effect is ſignified in the matter of this Sacrament ſpecially.}
ſhal ſaue the ſicke: and our Lord
\LNote{Shal lift
\Fix{vp him.}{him vp.}{obvious typo, fixed in other}}
{When it shal be good for the ſaluation of the partie, or agreable to
God's honour, this Sacrament reſtoreth alſo a man to bodily health
againe, as experience often teacheth vs. Which yet is not done by way of
miracle, to make the partie ſodenly whole, but by God's ordinarie
prouidence & vſe of ſecond cauſes, which otherwiſe should not haue had
that effect, but for the ſaid Sacrament. This is the ſecond effect.}
ſhal lift him vp: and if he be in ſinnes,
\LNote{They shal be remitted him.}{What ſinnes ſo euer remaine
vnremitted, they shal in this Sacrament and by the grace thereof be
remitted, if the perſons worthily receiue it. This is the third
effect. S.~Chryſoſtom of this effect ſaith thus:
\MNote{Prieſts (and not Elders) are the Miniſters of this Sacramẽt.}
\Emph{They} (ſpeaking of Prieſts) \Emph{doe not only remit ſinnes in
Baptiſme, but afterward alſo, according to the ſaying of S.~Iames: If
any be ſicke, let him bring in the Prieſts &c.}
\Cite{Li.~3. de Sacord. prope initium.}
Let the Proteſtants marke that he
calleth \GG{Preſbyteros}, \L{Sacerdotes}: that is \Emph{Prieſts}, and
maketh them the only Miniſters of this Sacrament, and not elders or
other lay-men. By al which you ſee this Sacrament of al other to be
maruelous plainely ſet forth by the Apoſtle. Only ſicke men and (as
\TNote{\G{ἀσθενεῖ τις}}
the Greek word giueth) men very weake muſt receiue it: only Prieſts
muſt
\Fix{by}{be}{obvious typo, fixed in other}
the Miniſters of it: the matter of it is holy oile: the forme is praier,
in ſuch ſort as we ſee now vſed: the effects be as is aforeſaid. Yet
this ſo plaine a matter and ſo profitable a Sacrament, the enemie by
Heretikes would wholy abolish.}
they ſhal be remitted him. \V
\LNote{Confeſſe therfore.}{It
\MNote{Confeſſion.}
is not certaine that he ſpeaketh here of ſacramental Confeſſion, yet the
circumſtance of the letter wel beareth it, and very probable it is that
he meaneth of it: and Origen doth ſo expound it
\Cite{ho.~2. in Leuit.}
& Venerable Bede writeth thus,
\CNote{\Cite{In hunc locum.}}
\Emph{In this ſentence} (ſaith he) \Emph{there muſt be this diſcretion,
that our daily & litle ſinnes we cõfeſſe one to another, vnto our
equals, and beleeue to be ſaued by their daily praier. But the vncleannes
of the greater leproſie let vs according to the law open to the Prieſt,
and at his pleaſure in what manner and how long time he shal command,
let vs be careful to be purified.} But the Proteſtants flying from the
very word \Emph{Confeſsion} in deſpite of the Sacrament tranſlate thus,
\Emph{Acknowledge your faults one to another.} They doe not wel like to
haue in one ſentence, Prieſts, praying ouer the ſicke, anoiling them,
forgiuing them their ſinnes, confeſſion, and the like.}
\SNote{The Heretikes tranſlate, \Emph{Acknowledge your ſinnes &c.} So
litle they can abide the very word of \Emph{confeſsion}.}
Confeſſe therfore your ſinnes one to another: and pray one for another,
that you may be ſaued. For the continual praier of a iuſt man
auaileth much. \V
\CNote{\XRef{3.~Reg.~17.}
\XRef{Ecc.~48.}
\XRef{Luc.~4,~25.}}
Elias was a man like vnto vs, paſſible: and with praier
\LNote{He praied.}{The
\MNote{Truths vnwritten & knowen by tradition.}
Scriptures to which the Apoſtle alludeth, make no mention of Elias
praier. Therfore he knew it by tradition or reuelatiõ. Whereby we ſee
that many things vnwritten be of equal truth with the things written.}
he praied that it might not raine vpon the earth, and it rained not for
three yeares and ſixe moneths. \V And
\CNote{\XRef{3.~Reg.~18,~41.}}
he praied againe: and the heauen gaue raine, and the earth yealded her
fruit.

\V My Brethren, if any of you ſhal erre from the truth, and a man conuert
him: \V he muſt know that he
\LNote{Maketh to be conuerted.}{Here
\MNote{Conuerting of ſoules.}
we ſee the great reward of ſuch as ſeeke to conuert Heretikes or other
ſinners from errour and wickednes: and how neceſſarie an office it is,
ſpecially for a Prieſt.}
which maketh a ſinner to be conuerted from the errour of his way,
\LNote{Shal ſaue.}{We
\MNote{Our ſaluation attributed to men, without derogation to Chriſt.}
ſee, it derogateth not from God, to attribute our ſaluation to any man
or Angel in heauen or earth, as to the workers thereof vnder God, by
their praiers, preaching, correction, counſel, or otherwiſe. Yet the
Heretikes are ſo foolish and captious in this kind, that they can not
heare patiently, that our B.~Lady or others should be counted meanes or
workers of our ſaluation.}
ſhal ſaue his ſoule from death, and
\SNote{He that hath the zeale of conuerting ſinners, procureth thereby
mercie & remiſſion to himſelf which is a ſingular grace.}
couereth a multitude of ſinnes.


\stopChapter


\stopcomponent


%%% Local Variables:
%%% mode: TeX
%%% eval: (long-s-mode)
%%% eval: (set-input-method "TeX")
%%% fill-column: 72
%%% eval: (auto-fill-mode)
%%% coding: utf-8-unix
%%% End:

