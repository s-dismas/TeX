%%%%%%%%%%%%%%%%%%%%%%%%%%%%%%%%%%%%%%%%%%%%%%%%%%%%%%%%%%%%%%%%%
%%%%
%%%% The (original) Douay Rheims Bible 
%%%%
%%%% New Testament
%%%% Front matter
%%%% Books
%%%%
%%%%%%%%%%%%%%%%%%%%%%%%%%%%%%%%%%%%%%%%%%%%%%%%%%%%%%%%%%%%%%%%%




\startcomponent books


\project douay-rheims


%%% 2285
%%% o-2098
\startBooks[
  title={\Sc{The Books of the New Testament, according to the covnt of
  the Catholike Chvrch.}},
  marking={\Sc{Books of the New Testament}}
  ]


%%% !!! Format this better!
\Sc{4.~Ghospels.}

The Ghoſpel of S.~Matthew.

The Ghoſpel of S.~Marke.

The Ghoſpel of S.~Luke.

The Ghoſpel of S.~Iohn.

The \Sc{Acts} of the Apoſtles.

\Sc{S.~Pavles Epiſt.~14.}

The Epiſtle to the Romanes.

The 1.~Epiſtle to the Corinthians.

The 2.~Epiſtle to the Corinthians.

The Epiſtle to the Galatians.

The Epiſtle to the Epheſians.

The Epiſtle to the Philippians.

The Epiſtle to the Coloſſians.

The 1.~Epiſtle to the Theſſalonians.

The 2.~Epiſtle to the Theſſalonians.

The 1.~Epiſtle to the Timothee.

The 2.~Epiſtle to the Timothee.

The Epiſtle to Titus.

The Epiſtle to Philemon.

The Epiſtle to the Hebrewes.

\Sc{The 7.~Cathol. Epiſtles.}

The Epiſtle of S.~Iames.

The 1.~Epiſtle of S.~Peter.

The 2.~Epiſtle of S.~Peter.

The 1.~Epiſtle of S.~Iohn.

The 2.~Epiſtle of S.~Iohn.

The 3.~Epiſtle of S.~Iohn.

The Epiſtle of S.~Iude.

The \Sc{Apocalypse} of S.~Iohn the Apoſtle.

\Emph{1.~The infallible authoritie and excellencie of them aboue al
other writings.}

The
\CNote{\Cite{S.~Aug. li.~11. cont. Fauſt. c.~5.}}
excellencie of the Canonical authoritie of the old and New Teſtament, is
diſtincted from the books of later Writers: which being confirmed in the
Apoſtles times, by the ſucceſſion of Bishops, and propagations of
Churches, is placed as it were in a certaine throne on high, wherunto
euery faithful and godly vnderſtanding muſt be ſubiect and
obedient. There, if any thing moue or trouble thee as abſurd, thou
maieſt not ſay: The Authour of this book held not the truth: but, either
the copie is faultie, or the Tranſlatour erred, or thou vnderſtandeſt
not. But in the workes of them that wrote afterward, which are conteined
in infinit books, but are in no caſe equal to that moſt ſacred
authoritie of Canonical Scriptures, in which ſoeuer of them is found
euen the ſame truth, yet the authoritie is farre vnequal.

\Emph{2.~The diſcerning of Canonical from not Canonical, and of their
infallible truth, & ſenſe, commeth vnto vs, only by the credit we giue
vnto the Catholike Church through whoſe commendation we beleeue both the
Ghoſpel & Chriſt himſelf. Wheras the Sectaries meaſure the matter by
their fantaſies and opinions.}

I
\CNote{\Cite{S.~Aug. cont. Epiſt. fundamenti cap.~5.}}
for my part, would not beleeue the Ghoſpel, vnles the authoritie of the
Catholike Church moued me. They therfore whom I
\Fix{obeiect}{obeied}{obvious typo, fixed in other}
ſaying, Beleeue the Ghoſpel; why should I not beleeue them ſaying,
Beleeue not
\SNote{Luther, Caluin.}
Manichæus? Chooſe whether thou wilt. If thou wilt ſay, Beleeue the
Catholikes: loe they warne me that I giue no credit
%%% 2286
vnto you: and therfore beleeuing them, I muſt needs not beleeue thee. If
thou ſay: Beleeue not the Catholikes: it is not the right way, by the
Ghoſpel to driue me to the faith of Manichæus, becauſe I beleeued
%%% 0-2099
the Ghoſpel it-ſelf by the preaching of Catholikes.

I
\CNote{\Cite{Againe li. de vtilit. credend. c.~14.}}
ſee that concerning Chriſt himſelf, I haue beleeued none, but the
confirmed and aſſured opinion of Peoples and Nations: and that theſe
Peoples haue on euery ſide poſſeſſed the Myſteries of the Catholike
Church. Why should I not therfore moſt diligently require, ſpecially
among them, what Chriſt commanded, by whoſe authoritie I was moued to
beleeue, that Chriſt did command ſome profitable thing? Wilt thou (ô
Heretike) tel me better what he ſaid, whom I would not thinke to haue
been at al, or to be, if I muſt beleeue, becauſe thou ſaieſt it? What
groſſe madnes is this, to ſay, Beleeue the Catholikes that Chriſt is to
be beleeued: and learne of vs, what he ſaid.

Thou
\CNote{\Cite{Againe cont. Fauſtum l.~11. cap.~2.}}
ſeeſt then in this matter what force the authoritie of the Catholike
Church hath, which euen from the moſt grounded and founded Seats of the
Apoſtles, is eſtablished vntil this day, by the line of Bishops
ſucceeding one another, and by the conſent of ſo many peoples. Wheras
thou ſaieſt, This is Scripture, or, this is ſuch as Apoſtles, that is
not; becauſe this ſoundeth for me, and the other againſt me. Thou then
art the rule of truth. Whatſoeuer is againſt thee, is not true.

\Emph{3.~No Heretikes haue right to the Scriptures, but are vſurpers: the
Catholike Church being the true owner and faithful keeper of them,
Heretikes abuſe them, corrupt them, and vtterly ſeeke to abolish them,
though they pretend the contrarie.}

Who
\CNote{\Cite{Tertullian li. De præſcriptionibus},
bringeth in the Catholike Church ſpeaking thus to al Heretikes.}
are you, when, and from whence came you? what doe you in my poſſeſſion,
that are none of mine? By what right (Marcion) doeſt thou cut downe my
wood? Who gaue thee licence (
\SNote{ô Luther, Zwinglius, Caluin.}
ô Valentine) to turne the courſe of my fountaines? By what authoritie
(Apelles) doeſt thou remoue my bounds? And
\SNote{Their ſcholers & followers.}
you the reſt, why doe you ſow and ſeed for theſe companions at your
pleaſure? It is my poſſeſſion, I poſſeſſe it of old, I haue aſſured
origins therof, euen from thoſe Authours whoſe the thing was. I am the
heire of the Apoſtles. As they prouided by their Teſtament, as they
committed it to my credit, as they adiured me, ſo doe I hold it. You
ſurely they diſherited alwaies and haue caſt you off as forainers, as
enemies.

Encountering
\CNote{\Cite{Againe in the ſame book.}}
with ſuch by Scriptures, auaileth nothing, but to ouerturne a man's
ſtomake or his braine. This hereſie receiueth not certaine Scriptures: and
if it doe receiue ſome, yet by adding and taking away, it peruerteth the
ſame to ſerue their purpoſe: and if it receiue any, it doth not receiue
them wholy: and if after a ſort it receiue them wholy, neuertheles by
diuiſing diuers expoſitions, it turneth them cleane another way, &c.

\Emph{4.~Yet doe they vant themſelues of Scriptures exceedingly, but
they are neuer the more to be truſted for that.}

Let
\CNote{\Cite{S.~Hierom aduerſus Luciferianos in fine.}}
them not flatter themſelues, if they ſeem in their owne conceit to
affirme that which they ſay, out of the chapters of Scripture; wheras
the Diuel alſo ſpake ſome things out of the Scriptures: and the
Scriptures conſiſt not in the reading, but in the vnderſtanding.

%%% o-2100
Here
\CNote{\Cite{Vincentius Lirenſis l. cont. prophanas hæreſum Nouationes.}}
perhaps ſome man may aske, whether Heretikes alſo vſe not the
teſtimonies of diuine Scripture. Yes indeed doe they, and that
vehemently. For thou shalt ſee them flie through euery one of the Sacred
books of the Law, through Moyſes, the books of the Kings, the Pſalmes,
the Apoſtles, the Ghoſpels, the Prophets. For whether among their owne
fellowes, or ſtrangers; whether priuatly, or publikely; whether in
talke, or in their books; whether in bankets, or in the ſtreets: they (I
ſay) alleage nothing of their owne, which they endeauour not to shadow
with the words of Scripture alſo. Read the workes of Paulus Samoſatenus,
of Priſcillian, of Eumonian, of Iouinian,
\SNote{Of Caluin, of Iuel, of the reſt.}
of the other plagues and peſtilences: thou shalt find an infinit heap of
examples, no page in a manner omitted
%%% 2287
or void, which is not painted and coloured with the ſentences of the new
or old Teſtament. But they are ſo much the more to be taken heed of, and
to be feared, the more ſecretly they lurke vnder the shadowes of God's
diuine Law. For they know their ſtinkes would not eaſily pleaſe any man
almoſt, if they were breathed out nakedly & ſimply themſelues alone, &
therfore they ſprinkle them as it were with certaine pretious ſpices of
the heauenly word: to the end that he which would eaſily deſpiſe the
errour of man, may not eaſily contemne the Oracles of God. So that they
doe like vnto them, which when they wil prepare certaine bitter potions
for children, doe firſt anoint the brimmes of the cup with honie, that
the vnwarie age, when it shal firſt feel the ſweetnes, may not feare the
bitternes.

\Emph{5.~The cauſe why, the Scriptures being perfit, yet we vſe other
Eccleſiaſtical writings and traditions.}

Here
\CNote{\Cite{Vincentius Lirinenſus in his golden booke before cited,
aduerſus prophanas hæreſum Nouationes.}}
ſome man perhaps may aske, for as much as the Canon of the Scriptures is
perfit, and in al points very ſufficient in itſelf, what need is there,
to ioyne thervnto the authoritie of the
\SNote{So he calleth the Churches ſenſe, & the Fathers interpretatiõs of
Scriptures.}
Eccleſiaſtical vnderſtanding? For this cauſe ſurely, for that al take
not the holy Scripture in one and the ſame ſenſe, becauſe of the deepnes
therof: But the ſpeaches therof, ſome interpret one way, and ſome
another way; ſo that there may almoſt as many ſenſes be picked out of
it, as there be men. For Nouation doth expound it one way, & Sabellius
another way, otherwiſe Donatus, otherwiſe Arius, Eunomius, Macedonius,
otherwiſe Photinus, Appolinaris, Priſcillianus, otherwiſe Iouinian,
Pelagius, Celeſtius, laſtly otherwiſe Neſtorius.
\SNote{Otherwiſe Wicliffe, Luther, Caluin, Puritanes.}
And therfore very neceſſarie it is becauſe of ſo great windings and
turnings of diuers
\Fix{errous,}{errours,}{obvious typo, fixed in other}
that the line of Prophetical & Apoſtolical interpretation, be directed
according to the rule of the Eccleſiaſtical and Catholike ſenſe or
vnderſtanding.

Of
\CNote{\Cite{S.~Baſil li. de Spiritu Sancto. cap.~27.}}
ſuch articles of religion as are kept & preached in the Church, ſome
were taught by the written word, other-ſome we haue receiued by the
tradition of the Apoſtles, deliuered vnto vs as it were from hand to
hand in myſterie ſecretly: both which be of one force to Chriſtian
religion: and this no man wil deny that hath any litle skil of the
Eccleſiaſtical rites or cuſtomes. For if we goe about to reiect the
cuſtomes not conteined in Scripture, as being of ſmal force, we shal
vnwittingly & vnawares mangle the Ghoſpel it-ſelf in the principal parts
therof, yea rather, we shal abridge the very preaching of the Ghoſpel,
and bring it to a bare name.


\stopBooks


\stopcomponent


%%% Local Variables:
%%% mode: TeX
%%% eval: (long-s-mode)
%%% eval: (set-input-method "TeX")
%%% fill-column: 72
%%% eval: (auto-fill-mode)
%%% coding: utf-8-unix
%%% End:
