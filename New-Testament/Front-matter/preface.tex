%%%%%%%%%%%%%%%%%%%%%%%%%%%%%%%%%%%%%%%%%%%%%%%%%%%%%%%%%%%%%%%%%
%%%%
%%%% The (original) Douay Rheims Bible 
%%%%
%%%% New Testament
%%%% Front matter
%%%% Preface
%%%%
%%%%%%%%%%%%%%%%%%%%%%%%%%%%%%%%%%%%%%%%%%%%%%%%%%%%%%%%%%%%%%%%%




\startcomponent preface


\project douay-rheims


%%% 2267
%%% o-2075
\startPreface[
  title={\Sc{The Preface to the Reader Treating of These Three Points:}
    of the tranſlation of Holy Scriptures into the vulgar tongues,
    and namely into English; of the cauſes why this New Teſtament is
    tranſlated according to the ancient vulgar Latin text; and of the
    manner of tranſlating the ſame.},
  marking={\Sc{Preface to the Reader}}
  ]

The holy Bible long ſince tranſlated by vs into English, and the old
Teſtament lying by vs for lack of good meanes to publish the whole in
ſuch ſort as a work of ſo great charge and importance requireth; we haue
yet through God's goodnes at length fully finished for thee (moſt
Chriſtian Reader) al the \Sc{new testament}; which is the principal,
moſt profitable, & comfortable peece of holy Writ: and, as wel for al
other inſtitution of life and doctrine, as ſpecially for deciding the
doubts of theſe daies, more proper and pregnant then the other part not
yet printed.

Which
\MNote{Tranſlation of the Scriptures into the vulgar tõgues, not
  abſolutely neceſſarie or profitable, but according to the time.}
tranſlation we doe not for al that publish, vpon erroneous opinion 1.~of
neceſſitie, that the holy Scriptures should alwayes be in our mother
tongue, or 2.~that they ought, or were ordained by God, to be read
indifferently of al, or 3.~could be eaſily vnderſtood of euery one that
readeth or heareth them in a knowen language; or 4.~that they were not
often, through man's malice or infirmitie, pernicious and much hurtful
to many; 5.~or that we generally and abſolutely deemed it more
conuenient in it-ſelf, & more agreable to God's word and honour, or
edification, of the faithful, to haue them turned into vulgar tongues,
then to be kept & ſtudied only in the Eccleſiaſtical learned languages:
Not for theſe nor any ſuch like cauſes doe we tranſlate this ſacred
Booke; but vpon ſpecial conſideration of the preſent time, ſtate, and
condition of our countrie, vnto which diuers things are either
neceſſarie, or profitable and medicinable now, that otherwiſe in the
peace of the Church were neither much requiſit, nor perchance wholy
tolerable.

1.~In
\MNote{The Churches wiſedom and moderation concerning vulgar
  tranſlation.} 
this matter, to marke only the wiſedom & moderation of holy Church and
the Gouernours therof on the one ſide, and the indiſcrete zeale of the
popular,
%%% o-2076
and their factious leaders, on the other, is a high point of
prudence.  Theſe later, partly of ſimplicitie, partly of curioſitie, and
ſpecially of pride & diſobedience, haue made claime in this caſe for
the common people, with plauſible pretences many, but good reaſons none
%%% 2268
at al.
The other,
\CNote{\XRef{Mt.~24,~45.}}
to whom Chriſt hath giuen charge of our ſoules, the
\CNote{\XRef{1.~Cor.~4,~1.}}
diſpenſing of God's myſteries and treaſures (among which, holy Scripture
is no ſmal ſtore) and the feeding his familie in ſeaſon with food fit
for euery ſort, haue neither of old nor of late, euer wholy condemned al
vulgar verſions of Scripture, nor haue at any time generally forbidden
the faithful to reade the ſame: yet they haue not by publike authoritie
preſcribed, commanded, or authentically euer recommended any ſuch
interpretation to be indifferently vſed of al men.

The
\MNote{The Scriptures in the vulgar languages of diuers Natiõs.}
Armenians ſay they haue the
\CNote{\Cite{Bib. Sanct. li.~4.}}
Pſalter and ſome other peeces tranſlated by
S.~Chryſoſtom into their language, when he was banished among them: and
George the Patriarch, in writing his life, ſignifieth no leſſe.  The
Slauonians affirme they haue the Scriptures in their vulgar tongue,
turned by
\CNote{\Cite{Hiero. ep.~134.}}
S.~Hierom; and ſome would gather ſo much by his owne wordes in his
epiſtle to Sophronius, but the place indeed proueth it not.
Vulpilas ſurely gaue the Scriptures to the Goths in their owne tõgue, &
that before he was an Arrian.
\MNote{Ancient Catholike trãſlations of the Bible into the Italian,
  Frẽch, & English tongue.}
\CNote{\Cite{Bib. Sanct. lib.~4.}}
It is almoſt three hundred yeares, ſince Iames Archbishop of Genua, is ſaid
to haue tranſlated the Bible into Italian.  More then two hundred yeares
agoe, in the daies of Charles the fifth, the French King, was it put
forth faithfully in French, the ſooner to shake out of the deceiued
peoples hands, the falſe heretical tranſlations of a Sect called
\Emph{Waldenſes}.
\CNote{\Cite{Li.~1. hiſt. Angl. c.~1.}}
In our owne countrie, notwithſtanding the Latin
tongue was euer (to vſe Venerable Bede's wordes) common to al the
Prouinces of the
ſame for meditation or ſtudie of Scriptures, & no vulgar tranſlation
commonly vſed or occupied of the multitude,
\MNote{Li.~1. c.~47.}
yet they were extant in
English euen before the troubles that Wicleffe & his followers raiſed
in our Church, as appeareth, as wel by the teſtimonie of Malmesburie
recording that V.~Bede tranſlated diuers partes into the vulgar tongue
of his time, & by ſome peeces yet remaining; as
\MNote{An anciẽt prouincial conſtitution in England concerning
  English tranſlations.  
  \Emph{See Linwood.\ li.~5.\ tit.~de Magiſtris.}}
by a prouincial Conſtitution of Thomas Arundel Archbishop of
Canturburie, in a Councel holden at Oxford: where ſtrait prouiſion was
made, that no heretical verſion ſet forth by Wicleffe, or his adherents,
should be ſuffered, nor any other in or after his time be published or
permitted to be read, being not approued & allowed by the Dioceſan
before: alleaging S.~Hierom for the difficultie and danger of
interpreting the holy Scripture out of one tongue into another, though
by learned & Catholike men.  So alſo it is there inſinuated, that
neither the Tranſlations ſet forth before that Heretikes time, nor other
afterward being approued by the lawful Ordinaries, were euer in our
countrie wholy forbidden, though they were not (to ſay the truth) in
quiet and better times (much leſſe when the people were prone to
alteration, hereſie, or noueltie) either haſtily admitted, or ordinarily
read of the vulgar, but vſed only, or ſpecially, of ſome deuout
religious & contemplatiue perſons, in reuerence, ſecrecie, and ſilence,
for their ſpiritual comfort.

Now
\MNote{The like Catholike and vulgar tranſlations in many
  countries, ſince Luther's time.}
ſince Luther's reuolt alſo, diuers learned Catholikes, for the more
ſpeedy abolishing of a number of falſe and impious tranſlations put
forth by ſundry Sects, and for the better preſeruation or reclaime of
many good ſoules endangered thereby, haue published the Bible in the
ſeueral languages of almoſt al the principal Prouinces of the Latin
Church: no other books in the world being ſo pernicious as heretical
tranſlations of the Scriptures, poiſoning the people vnder colour of
diuine authoritie, & not many other remedies
%%% o-2077
being more ſoueraigne againſt the ſame (if it be vſed in order,
diſcretiõ, & humilitie) then the true, faithful, and ſincere
interpretation oppoſed therevnto.

2.~Which
\MNote{The Churches order & determination concerning the reading
  of Catholike tranſlations of the Bible in vulgar tongues.}
cauſeth the holy Church not to forbid vtterly any Catholike tranſlation,
though she allow not the publishing or reading of any abſolutely &
without exception, or limitation: knowing by her diuine and moſt ſincere
wiſedom, how, where, when, and to whom theſe her Maiſters and Spouſes
guifts are to be beſtowed to the moſt good of the faithful: and therfore
neither generally permitteth that which muſt needs doe hurt to the
vnworthy, nor abſolutely condemneth that which may doe much good to the
worthie.  Wherevpon,
\CNote{\Cite{Ind. lib. prohibit. regul.~4.}}
the order which many a wiſe man wished for before,
was taken by the Deputies of the late famous Councel of Trent in this
behalfe, and confirmed by ſupreme authoritie, that the holy Scriptures,
though truly and Catholikely tranſlated into vulgar tongues, yet may not
be indifferently read of al men, nor of any other then ſuch as haue
expreſſe licence therunto of their lawful Ordinaries, with good
teſtimonie from their Curates of Confeſſours, that they be humble,
diſcrete, and deuout perſons, and like to take much good, and no harme
thereby.  Which preſcript, though in theſe daies of ours it can not be
ſo preciſely obſerued, as in other times and places, where
%%% 2269
there is more due reſpect of the Churches authoritie, rule, and
diſcipline: yet we truſt al wiſe and godly perſons wil vſe the matter in
the meane while, with ſuch moderation, meeknes, and ſubiection of hart,
as the handling of ſo ſacred a Book, the ſincere ſenſes of God's truth
therin, and the holy Canons, Councels, reaſon, and religion doe
require. 

Wherin, though for due preſeruation of this diuine worke from abuſe and
prophanation, and for the better bridling of the intolerable inſolencie
of proud, curious, and contentious wittes, the Gouernours of the Church
guided by God's Spirit, as euer before, ſo alſo vpon more experience of
the maladie of this time then before, haue taken more exact order both
for the Readers and Tranſlatours in theſe later Ages, then of old: yet
\MNote{The holy Scriptures neuer read of al perſons indifferently,
  at their pleaſure.}
we muſt not imagin that in the primitiue Church, either euery one that
vnderſtood the learned tongues wherin the Scriptures were written, or
other languages into which they were tranſlated, might without
reprehenſion, read, reaſon, diſpute, turne and toſſe the Scriptures: or
that our Forefathers ſuffered euery Schole-maiſter, ſcholer, or
Grammarian that had a litle Greeke or Latin, ſtraight to take in hand
the holy Teſtament: or that the tranſlated Bibles into the vulgar
tongues, were in the hands of euery huſband-man, artificer, prentice,
boies, girles, miſtreſſe, maid, man: that they were ſung, plaied,
alleaged, of euery tinker, tauerner, rimer, minſtrel: that they were for
rable talke, for ale-benches, for boats and barges, and for euery
prophane perſon and companie: No, in thoſe better times men were neither
ſo il, nor ſo curious of themſelues, ſo to abuſe the bleſſed book of
Chriſt: neither was there any ſuch eaſy meanes before printing was
inuented, to diſperſe the copies into the hands of euery man, as now
there is.

They
\MNote{Where and in whoſe hands the Scriptures were in the
  primitiue Church.}
were then in Libraries, Monaſteries, Colledges, Churches, in Bishops,
Prieſts, and ſome deuout principal Lay-mens houſes and hands: who vſed
them with feare and reuerence, and ſpecially ſuch parts as perteined to
good life and manners, not medling, but in pulpit and ſchooles (and that
moderately too) with the hard and high myſteries and places of greater
difficultie.  The poore plough-man, could then in labouring the ground,
ſing the Hymnes
%%% o-2078
and pſalmes either in knowen or vnknowen languages, as they heard them
in the holy Church, though they could neither read nor know the ſenſe,
meaning, and myſteries of the ſame.  Such
\MNote{How the laytie of thoſe daies did read thẽ: with what
  humilitie and religion, and information of life and manners.}
holy perſons of both ſexes, to whom Saint Hierom in diuers Epiſtles to
them, commendeth the reading and meditation of holy Scriptures, were
diligent to ſearch al the godly hiſtories and imitable examples of
chaſtitie, humilitie, obedience, clemencie, pouertie, penance,
renouncing the world: they noted ſpecially the places that did breed the
hatred of ſinne, feare of God's iudgement, delight in ſpiritual
cogitation: they referred themſelues in al hard places, to the iudgement
of the Ancient Fathers and their Maiſters in religion, neuer preſuming
to contend, controule, teach or talke of their owne ſenſe and phantaſie,
in deep queſtions of diuinitie.  Then the Virgins did meditate vpon the
places and examples of chaſtitie, modeſtie and demureneſſe; the married,
on coniugal faith and continence; the parents, how to bring vp their
children in faith and feare of God; the Prince, how to rule; the
ſubiect, how to obey; the Prieſt, how to teach; the people, how to
learne.

3.~Then
\MNote{The Fathers sharply reprehend as an abuſe, that al
  indifferently should read, expound, & talke of the Scriptures.}
the ſcholer taught not his Maiſter, the sheep controuled not the
Paſtour, the yong ſtudent ſet not the Doctour to ſchoole, nor reproued
their Fathers of errour and ignorance.  Or if any were in thoſe better
daies (as in al times of hereſie ſuch muſt needs be) that had itching
eares, tikling tongues and wittes, curious and contentious diſputers,
hearers, and talkers rather then doers of God's word: ſuch the Fathers
did euer sharply reprehend, counting them vnworthy and vnprofitable
Readers of the holy Scriptures.  Saint Hierom in his
\CNote{\Cite{Hier. ep.~103. c.~6.}}
Epiſtle to
Paulinus, after declaration that no handycraft is ſo baſe, nor liberal
ſcience ſo eaſy, that can be had without a Maiſter (which S.~Auguſtin
alſo affirmeth, 
\Cite{De vtilitate cred. cap.~7.})
nor that men preſume
in any occupation to teach that they neuer learned, \Emph{Only} (ſaith
he) \Emph{the art of Scripture is that which euery man chalengeth: this
  the chatting old wife, this the doting old man, this the brabling
  Sophiſter, this on euery hand, men preſume to teach before they
  learne it.}  Againe, \Emph{Some with poiſe of lofty words deuiſe of
  ſcripture matters among women: otherſome (fy vpon it) learne of women,
  what to teach men, and leſt that be not enough, by facilitie of
  tongue, or rather audacitie, teach that to others, which they
  vnderſtand neuer a whit themſelues, to ſay nothing of ſuch as be of my
  facultie: who ſtepping from ſecular
%%% 2270
learning to holy ſcriptures, & able to tickle the eares of the
multitude with a ſmooth tale, thinke al they ſpeake, to be the Law of
God.}  This he wrote then, when this maladie of arrogancie and
preſumption in diuine matters, was nothing ſo outragious as now it is. 

S.~Gregorie Nazianzen made an
\CNote{\Cite{In orat. de doſeratio. in diſputa. ſeruãda.}}
oration of the moderation that was to be
vſed in theſe matters: where he ſaith, that ſome in his time thought
themſelues to haue al the wiſedom in the world, when they could once
repeat two or three words, and them il couched together, out of
Scriptures.  But he there diuinely diſcourſeth of the orders and
differences of degrees: how in Chriſtes myſtical body, ſome are ordeined
to learne, ſome to teach: al are not Apoſtles, al Doctours, al
Interpreters, al of tongues and knowledge, not al learned in Scriptures
& diuinitie: that the people went not vp to talke with God in the
mountaine but Moyſes, Aaron, & Eleazar: nor they neither but by the
difference of their callings: that they that rebel againſt this
ordinance, are guilty of the conſpiracie of Core & his Complices: that
\MNote{The Scriptures muſt be deliuered in meaſure & diſcretiõ,
  according to each man's need and capacitie.}
in Scripture there is both milke for babes, and meat for men, to be
diſpenſed, not according to euery one's
%%% o-2079
greedines of appetit, or wilfulnes, but as is moſt meet for each one's
neceſsitie
and capacitie: that as it is a shame for a Bishop or Prieſt to be
vnlearned in God's myſteries, ſo for the common people it is oftentimes
profitable to ſaluation, not to be curious, but to follow their Paſtours
in ſinceritie and ſimplicitie: whereof excellently ſaith S.~Auguſtin,
\CNote{\Cite{De agone Chriſt. c.~53.}}
\L{Fidei ſimplicitate & ſinceritate lactati, nutriamur in Chriſto;
  & cum parui ſumus, maiorum cibos non appetamus},
that is, \Emph{Being fed with the ſimplicitie and ſinceritie of faith,
  as it were with milke, ſo let vs be nourished in Chriſt: and when we
  are litle ones, let vs not count the meates of the elder ſort.}  Who
\CNote{\Cite{De bono perſeuer. c.~16.}}
in another place teſtifieth, that the word of God can not be preached
nor certaine myſteries vttered to al men alike, but are to be deliuered
according to the capacitie of the hearers, as he proueth both
\CNote{\XRef{1.~Cor.~3.}}
by S.~Paules example, who gaue not to euery ſort ſtrong meate, but milke
to many, as being not ſpiritual, but carnal and not capable: and
\CNote{\XRef{Io.~16.}}
by our
Lord's alſo, who ſpake to ſome plainely, & to others in parables, and
affirmed that he had many things to vtter which the hearers were not
able to beare.

How much more may we gather, that al things that be written, are not for
the capacitie and diet of euery of the ſimple Readers, but that very many
myſteries of holy Writ, be very farre aboue their reach, & may and
ought to be (by as great reaſon) deliuered them in meaſure and meane
moſt meet for them?  Which indeed can hardly be done, when the whole
book of the Bible lieth before euery man in his mother tongue, to make
choice of what he liſt.  For
\MNote{The Iewes law for not reading certaine bookes of holy
  Scripture vntil a time.}
which cauſe the ſaid
\CNote{\Cite{In orat. de mode. in diſp. ſerua. in fine.}}
Gregorie Nazianzen wisheth the Chriſtians had as
good a law as the Hebrewes of old had: who (as
\CNote{\Cite{Hiero. in proœm. commen. in Ezec.}}
S.~Hierom alſo
witneſſeth) tooke order among themſelues that none should read the
\Emph{Cantica Canticorum} nor certaine other peeces of hardeſt
Scriptures, til they were thirtie yeares of age.

And truely there is no cauſe why men should be more loth to be ordered
and moderated in this point by God's Church and their Paſtours, then
they are in the vſe of holy Sacraments: for which as Chriſt hath
appointed Prieſts and Miniſters, at whoſe hands we muſt receiue them,
and not be our owne caruers: ſo hath he giuen
\CNote{\XRef{Eph.~4.}}
vs Doctours, Prophets,
Expounders, Interpreters, Teachers and Preachers, to take the law and
our faith at their mouthes: becauſe our faith and religion commeth
not to vs properly or principally by reading of Scriptures, but (as
\CNote{\XRef{Ro.~10.~17.}}
the
Apoſtle ſaith) by hearing of the Preachers lawfully ſent: though reading
in order and humilitie, much confirmeth and aduanceth the ſame.
Therfore this holy Booke of the Scriptures, is called of S.~Ambroſe, 
\L{Liber ſacerdotalis},
\Emph{the booke of Prieſtes}, at whoſe hands and diſpoſition we muſt
take and vſe it.
\Cite{Li.~2. ad Grat.}

4.~The
\MNote{The popular obiections of withholding the Scriptures from
  the people, anſwered.}
wiſe wil not here regard what ſome wilful people doe mutter, that the
Scriptures are made for al men, and that it is of enuie that the Prieſts
doe keep the holy Booke from them.  Which ſuggeſtion commeth of the ſame
ſerpent
\CNote{\XRef{Gen.~3.}}
that ſeduced our firſt parents, who perſuaded them, that God had
forbidden them that tree of knowledge, leſt they should be as cunning as
himſelf, and like vnto the Higheſt.  No,
\MNote{Why the Church permitteth not euery one at their pleaſure to
  read the Scripture.}
no, the Church doth it to keep them from blind ignorant preſumption, and
from that which the Apoſtle calleth
\CNote{\XRef{1.~Tim.~6,~20.}}
\L{falſi nominis ſcientiam},
\Emph{knowledge falſely ſo called}: and not to embarre them from the
true knowledge of Chriſt.  She would haue al wiſe, but
\CNote{\XRef{Ro.~12,~3.}}
\L{vſque ad
ſobrietatem}, \Emph{vnto ſobrietie}, as the Apoſtle ſpeaketh: she
knoweth the Scriptures be ordained for euery ſtate, as meates, elements,
fire, water, candle, kniues, ſword, and the like; 
%%% o-2080
which are as needful (moſt of them) for children as old folkes, for the
ſimple as the wiſe: but yet
%%% 2271
would marre al, if they were at the guiding of other then wiſe men, or
were in the hands of euery one, for whoſe preſeruation they be
profitable.  She forbiddeth not the reading of them in any language,
enuieth no man's commoditie; but giueth order how to doe it to
edification, and not deſtruction: how
\MNote{The holy Scriptures to carnal men & Heretikes, are as
  pearles to ſwine.}
to doe it without caſting
\CNote{\XRef{Mat.~7,~6.}}
\Emph{the holy to dogs}, or \Emph{pearles to ſwine}: (See 
\Cite{S.~Chryſoſt. ho.~24. in Matth}
declaring theſe hogs & dogs to be carnal men & Heretikes, that take no
good of the holy myſteries, but thereby doe both hurt themſelues &
others:) how to doe it agreably to the ſoueraigne ſinceritie, maieſtie,
and depth of Myſterie conteined in the ſame.  She would haue the
preſumptuous Heretike, notwithſtanding he alleage them neuer ſo faſt,
flying as it were through the whole Bible, and coting the Pſalmes,
Prophets, Ghoſpels, Epiſtles, neuer ſo readily for his purpoſe, as
Vincentius Litinenſis ſaith ſuch mens fashion is: yet she would
according to Tertullian's rule,
\CNote{\Cite{Li. de preſcriptionibus.}}
haue ſuch mere vſurpers quite diſcharged
of al occupying and poſſeſsion of the holy Teſtament, which is her old
and only right and inheritance, and belongeth not to Heretikes at al,
whom Origen calleth
\CNote{\Cite{Orig. in 2.~ad Ro.}}
\L{Scripturarum fures}, \Emph{theeues of the
Scriptures}.  She would haue the vnworthy repelled, the curious
repreſſed, the ſimple meaſured, the learned humbled, and al ſorts ſo to
vſe them or abſteine from them, as is moſt conuenient for euery ones
ſaluation: with this general admonition, that none can vnderſtand the
meaning of God in the Scriptures
\CNote{\XRef{Luc.~24.}}
except Chriſt open their ſenſe, & make
them partakers of his holy Spirit in the vnitie of his myſtical bodie:
and for the reſt, she committeth it to the Paſtour of euery prouince and
people, according to the difference of time, place, and perſons, how and
in what ſort the reading of the Scriptures is more or leſſe to be
procured or permitted.

5.~Wherin,
\MNote{S.~Chriſoſtoms exhortations to the reading of holy
  Scriptures; & when the people is ſo to be exhorted.}
the varietie of circumſtances cauſeth them to deale diuerſly: as we ſee
by S.~Chryſoſtom's people of Conſtantinople, who were ſo delicate, dul,
worldly, and ſo much giuen to dice, cardes, ſpecially ſtage-plaies or
theaters (as S.~Gregorie Nazianzen
\CNote{\Cite{In vita Athanaſij.}}
witneſſeth) that the Scriptures &
al holy lections of diuine things were lothſome vnto them: whereby their
holy Bishop was forced
\CNote{\Cite{Ho.~2. in Mat.}
&
\Cite{ho.~3. de Laza.}
&
\Cite{ho.~3. in 2.~ad Theſſ.}
&
\Cite{alibi ſape.}}
in many of his ſermons to crie out againſt their
extreme negligence and contempt of God's word, declaring, that not only
Eremites and Religious (as they alleaged for their excuſe) but ſecular
men of al ſorts might read the Scriptures, and often haue more need
therof in reſpect of themſelues, then the other that liue in more
puritie and contemplation; further inſinuating, that though diuers
things be high and hard therin, yet many godly hiſtories, liues,
examples, & precepts of life and doctrine be plaine; and finally, that
when the Gentils were ſo cunning and diligent to impugne their faith, it
were not good for Chriſtians to be too ſimple or negligent in the
defenſe thereof: as (in truth) it is more requiſite for a Catholike man
in theſe daies when our Aduerſaries be induſtrious to empeach our
beleefe, to be skilful in Scriptures, then at other times when the
Church had no ſuch enemies.

To
\MNote{S.~Chryſoſtom maketh nothĩg for the popular and licentious
  reading of Scriptures vſed amõg the Proteſtants now adaies.}
this ſenſe ſaid S.~Chryſoſtom diuers things, not as a Teacher in
ſchoole, making exact and general rules to be obſerued in al places &
times, but as a pulpit man, agreably to that audience and his peoples
default: nor making it therfore (as ſome peruerſly gather of his words)
a thing abſolutely needful for euery poore artificer to read or ſtudie
Scriptures, nor any whit fauouring the preſumptuous, curious, and
contentious iangling and ſearching of God's ſecrets, reproued by the
foreſaid Fathers, much leſſe approuing the exceſsiue pride and
%%% o-2081
madnes of theſe daies, when euery man and woman is become not only a
Reader, but a Teacher, controuler, & iudge of Doctours, Church,
Scriptures and al: ſuch
\MNote{Euery ſimple artificer amõg them readeth much more the
  deepeſt & hardeſt queſtiõs of holy Scripture, then the moral parts.}
as either contemne or eaſily paſſe ouer al the moral parts, good
examples, and precepts of life (by which as wel the ſimple as learned
might be much edified) and only in a manner, occupie themſelues in
dogmatical, myſtical, high, and hidden ſecrets of God's counſels, as of
Predeſtination, reprobation, election, preſcience, forſaking of the
Iewes, vocation of the Gentils, and other incomprehenſible myſteries,
\CNote{\XRef{1.~Tim.~6.}}
\Emph{Languishing about queſtions} of only faith, fiduce, new phraſes
and figures,
\CNote{\XRef{2.~Tim. c.~3.}}
\Emph{euer learning}, but \Emph{neuer comming to
knowledge}, reading and toſsing in pride of wit, conceit of their owne
cunning, and vpon preſumption of I can not tel what ſpirit, ſuch bookes
ſpecially and Epiſtles, as S.~Peter
\CNote{\XRef{2.~Pet.~3.}}
foretold that the vnlearned and
inſtable would depraue to their owne damnation.

They
\MNote{They preſuppoſe no difficulties, which al the learned
  Fathers felt to be in the Scriptures.}
delight in none more then in the Epiſtle to the Romans, the
\Emph{Cantica Canticorum}, the Apocalypſe, which haue in them as many
myſteries as words.  They find no difficultie in the ſacred Booke
\CNote{\XRef{Apoc.~5,~1.}}
claſped with ſeuen ſeales.  They aſke for no Expoſitour
\CNote{\XRef{Act.~8.}}
with the
%%% 2272
holy Eunich.  They feele no ſuch depth of God's ſcience in the
ſcriptures, as S.~Auguſtin did when he cried out:
\CNote{\Cite{Confeſs. lib.~12. cap.~14.}}
\L{Mira profunditas eloquiorum tuorum, mira profunditas (Deus meus)
  mira profunditas! horror eſt intendere in eam, horror honoris, &
  tremor amoris};
that is, \Emph{O wonderful profoundnes of thy wordes; wonderful
  profoundnes, my God, wonderful profoundnes! it maketh a man quake to
  looke on it: to quake for reuerence, and to tremble for the loue
  thereof.}  They regard not that which the ſame Doctour affirmeth,
\CNote{See \Cite{ep.~3. Aug.}}
that
the depth and profunditie of wiſedom, not only in the words of holy
Scripture, but alſo in the matter & ſenſe, is ſo wonderful, that, liue
a man neuer ſo long, be he of neuer ſo high a witte, neuer ſo ſtudious,
neuer ſo feruent to attaine the knowledge therof, yet when he endeth, he
shal confeſſe he doth but begin.  They feele not with S.~Hierom,
\CNote{\Cite{Hiero. ep.~13. c.~4.}}
that
the text hath a hard shel to be broken before we come to the kernel.
\CNote{\Cite{Ruff. Ec. hiſt. li.~2. c.~9.}}
They wil not ſtay themſelues in only reading the ſacred Scriptures
thirteen yeares together, with S.~Baſil & S.~Gregorie Nazianzene,
before they expound them, nor take the care (as they did) neuer
otherwiſe to interpret them, then by the vniforme conſent of their
Forefathers and tradition Apoſtolike.

If our new Miniſters had had this cogitation and care that theſe and al
other wiſe men haue, and euer had, our countrie had neuer fallen to this
miſerable ſtate in religion, and that vnder pretence, colour, and
countenance of God's word: neither
\MNote{Manners & life nothing amended, but much worſe, ſince this
  licentious toſsing of holy Scriptures.}
should vertue and good life haue been ſo pittifully corrupted in time of
ſuch reading, toiling, tumbling and tranſlating the Booke of our life
and ſaluation: wherof the more precious the right and reuerent vſe is,
the more pernicious is the abuſe and prophanation of the ſame: which
euery man of experience by theſe few yeares proofe, and by comparing the
former daies and manners to theſe of ours, may eaſily trie.

Looke whether your men be more vertuous, your women more chaſt, your
children more obedient, your ſeruants more truſtie, your maids more
modeſt, your freinds more faithful, your laytie more iuſt in dealing,
your Clergie more deuout in praying: whether there be more religion,
feare of God, faith and conſcience in al ſtates now, then of old, when
there was not ſo much reading, chatting, and iangling of God's word, but
much more ſincere dealing, doing, and keeping the ſame.  Look whether
through this diſorder, women teach not their husbands, children their
parents, yong fooles their old and wiſe fathers, the ſcholers their
Maiſters, the sheep their Paſtour, and the People 
%%% o-2082
the Prieſt.
\MNote{Scriptures as profanely cited as heathẽ Poetes.}
Looke whether the moſt chaſt and ſacred ſentences of God's holy word, be
not turned of many, into mirth, mockerie, amorous ballets & deteſtable
letters of loue and leudnes: their delicate rimes, tunes, and
tranſlations much encreaſing the ſame.

This fal of good life & prophaning the diuine myſteries, euerybody
ſeeth: but the great corruption & decay of faith hereby, none ſee but
wiſe men, who only know, that, were
\MNote{Scriptures erroneouſly expoũded according to euery wicked
  man's priuat fanſie.}
the Scriptures neuer ſo truely tranſlated, yet Heretikes and il men that
follow their owne ſpirit and know nothing but their priuate fantaſie,
and not the ſenſe of the holy Church and Doctours, muſt needs abuſe them
to their damnation: and that the curious, ſimple, and
\CNote{\XRef{1.~Cor.~2.}}
ſenſual men which
haue no taſt of the things that be of the Spirit of God, may of infinit
places take occaſion of pernicious errours.  For though the letter or
text haue no errour, yet (ſaith S.~Ambroſe) the Arrian, or (as we may
now ſpeake) the Caluinian interpretation hath errours. 
\Cite{lib~2. ad Gratianum ca.~1.}
and Tertullian ſaith: \Emph{The ſenſe adulterated is as perilous as the
  ſtyle corrupted.}
\Cite{De Præſcrip.}
S.~Hilarie alſo ſpeaketh thus: \Emph{Hereſie riſeth about the
  vnderſtanding, not about the writing. The fault is in the ſenſe, not
  in the word.}
\Cite{lib.~2. de Trinit. in principio}
And S.~Auguſtin ſaith, that many hold the Scriptures as they doe the
Sacraments, \L{ad ſpeciem, & non ad ſalutem}, \Emph{to the outward shew,
and not to ſaluation.}
\Cite{de Baptis cont. Donat. li.~3. ca~19.}
Finally
\MNote{Al Heretikes pretend Scriptures.}
al Sect-maiſters and rauening wolues, yea
\CNote{\XRef{Mat.~4.}}
the Diuels themſelues
pretend Scriptures, alleage Scriptures, and wholy shroud thẽſelues in
Scriptures, as in the wool & fleece of the ſimple sheep.  Whereby the
vulgar, in theſe daies of general diſputes, can not but be in extreme
danger of errour, though their books were truely tranſlated, & were
truely in thẽſelues God's owne word indeed.

But
\MNote{The Scriptures haue beẽ falſely and heretically tranſlated
  into the vulgar tongues, and ſundrie other waies ſacrilegiouſly
  abuſed, and ſo giuen to the people to read.}
the caſe now is more lamentable: for the Proteſtants and ſuch as S.~Paul
calleth
\CNote{\XRef{2.~Cor.~4.}}
\L{ambulantes in aſtutia}, \Emph{walking in deceitfulnes}, haue
ſo abuſed the people, and many other in the world, not vnwiſe, that by
their falſe tranſlations they haue inſteed of God's Law and Teſtament,
and for Chriſtes written wil and word, giuen them their owne wicked
writing and phantaſies, moſt shamefully in al their verſions, Latin,
English, and other tongues, corrupting both the letter and ſenſe by
falſe tranſlation, adding, detracting, altering, tranſpoſing, pointing,
and al other guileful meanes: ſpecially where it
%%% 2273
ſerueth for the aduantage of their priuate opiniõs.  For which they are
bold alſo partly to diſauthorize quite, partly to make doubtful, diuers
whole books allowed for Canonical Scripture by the vniuerſal Church of
God this thouſand yeares and vpward: to alter al the authentical and
Eccleſiaſtical words vſed ſithence our Chriſtianitie, into new prophane
nouelties of ſpeaches agreable to their doctrine: to
\SNote{Al this their dealing is noted (as occaſiõ ſerueth) in the
  Annotations vpon this Teſtament: and more at large in the
  \Sc{discoverie} of heretical tranſlations wherof we haue added a table
  in this edition.}
change the titles of workes, to put out the names of Authours,
\CNote{\Cite{Beza annot. in c.~1. Luc.~1. v.~78.}}
to charge
the very Euangeliſt with following vntrue tranſlation, to adde whole
ſentences proper to their Sect, into their pſalmes in meter,
\CNote{See the \Cite{tenth article of their Creed in meter.}}
euen into
the very Creed in rime.  Al which the poore deceiued people ſay and ſing
as though they were God's owne word, being indeed through ſuch
ſacrilegious treacherie, made the Diuels word.

To ſay nothing of their intolerable liberty and licence to change the
accuſtomed callings of God, Angel, men, places, & things vſed by the
Apoſtles and al antiquitie, in Greek, Latin, and al other languages
of Chriſtian Nations, into new names, ſometimes falſely, and alwaies
ridiculouſly and for oſtentation taken of the Hebrewes: to frame and fine
the phraſes of holy Scriptures after the forme of prophane Writers,
ſticking not, for the ſame to ſupply, adde, alter, or diminish as freely
as if they tranſlated Liuie, Virgil, or Terence.
%%% o-2083
Hauing no religious reſpect to keep either the maieſtie or ſincere
ſimplicitie of that venerable ſtyle of Chriſtes ſpirit, as S.~Auguſtin
ſpeaketh, which kind the holy Ghoſt did chooſe of infinit wiſedom to
haue the diuine myſteries rather vttered in, then any other more
delicate, much leſſe in that meretricious manner of writing that ſundrie
of theſe new tranſlatours doe vſe: of
\MNote{Caluin cõplaineth of the new delicate
  tranſlatours, namely Caſtaliõ: himſelf and Beza being as bad or
  worſe.}
which ſort Caluin himſelfe and his
pue-fellowes ſo much complaine,
\CNote{\Cite{Pref. in N.~Teſt. Gal.~1567.}}
that they profeſſe, Satan to haue gained
more by theſe new interpreters (their number, leuitie of ſpirit, and
audacitie encreaſing daily) then he did before by keeping the word from
the people.  And for a paterne of this miſcheefe, they giue Caſtalion,
adiuring al their churches and ſcholers to beware of his tranſlation, as
one that hath made a very ſport and mockery of God's holy word.  So they
charge him:
\CNote{\Cite{Ioſsias Simlerus in vita Bullingers.}}
themſelues (and the Zuinglians of Zurick, whoſe tranſlations
Luther therfore abhorred)
\Fix{or }{}{obvious typo, fixed in other}% to avoid potential paragraph break.
handling the matter with no more fidelitie, grauitie, or ſinceritie,
then the other: but rather with much more falſification, or (to vſe the
Apoſtles wordes)
\MNote{2.~Cor.~2,~17.}
\Emph{cauponation} and \Emph{adulteration} of God's
word, then they.  Beſides many wicked gloſſes, prayers, confeſſions of
faith, conteining both blaſphemous errours
\SNote{See the 4.~article of their Creed in meter, where they
  profeſſe that Chriſt deſcended to deliuer the Fathers, & afterward in
  their confeſsiõ of their faith, they deny \L{Limbus Patrum}.}
and plaine contradictions to themſelues and among themſelues al
priuileged and authorized to be ioyned to the Bible, and to be ſaid and
ſung of the poore people, and to be beleeued as articles of faith &
wholy conſonant to God's word.

We
\MNote{The purpoſe & commoditie of ſetting forth this Catholike
  edition.} 
therfore hauing compaſſion to ſee our beloued Countriemen, with extreame
danger of their ſoules, to vſe only ſuch prophane tranſlations, and
erroneous mens mere phãtaſies, for the pure and bleſſed word of truth;
much alſo moued therunto by the deſires of many deuout perſons; haue ſet
forth, for you (benigne Readers) the new Teſtament to begin withal,
truſting that it may giue occaſion to you, after diligent peruſing
thereof, to lay away at leaſt ſuch their impure verſiõs as hitherto you
haue beẽ forced to occupie.  How wel we haue done it, we muſt not be
iudges, but referre al to God's Church and our Superiours in the ſame.
To them we ſubmit our ſelues, & this, & al other our labours, to be in
part, or in the whole, reformed, corrected, altered, or quite abolished:
moſt humbly deſiring pardon if through our ignorance, temeritie, or
other humane infirmitie, we haue any where miſtaken the ſenſe of the
holy Ghoſt.  Further promiſing, that if here-after we eſpie any of our
owne errours, or if any other, either freind of good wil, or aduerſarie
for deſire of reprehenſion, shal open vnto vs the ſame; we wil not
(as Proteſtants doe) for defenſe of our eſtimation, or of pride and
contention, by wrangling words wilfully perſiſt in them, but be moſt
glad to heare of them, & in the next editiõ or otherwiſe to correct
them: for it is truth that we ſeeke for, and God's honour: which being
had either by good intention, or by occaſion, al is wel.  This
\MNote{The religious care & ſinceritie obſerued in this
  tranſlatiõ.}
we profeſſe only, that we haue done our endeauour with praier, much
feare and trembling, leſt we should dangerouſly erre in ſo ſacred, high,
and diuine a worke: that we haue done it with al faith, diligence, and
ſinceritie: that we haue vſed no partialitie for the diſaduantage of our
aduerſaries, nor no more licence then is ſufferable in tranſlating of
holy Scriptures: continually keeping our-ſelues as neer as is poſſible,
to our text to the very words and phraſes which by long vſe are made
venerable, though to ſome prophane or delicate eares
%%% 2274
they may ſeeme more hard or barbarous,
\CNote{See \Cite{S.~Auguſt. li.~3. confeſ. c.~5.}}
as the whole ſtyle of Scripture
doth lightly to ſuch at the beginning: acknowledging with S.~Hierom,
that in other writings it is enough to giue in tranſlation, ſenſe for
ſenſe, but that in Scriptures, leſt we miſſe the ſenſe, we muſt keep the
very
%%% o-2084
words.
\Cite{Ad Pammach. epiſtola.~10.~1. ca.~2. in princip.}
We muſt, ſaith S.~Auguſtin, ſpeake according to a ſet rule, leſt licence
of words breed ſome wicked opinion concerning the things conteined vnder
the words.
\Cite{De ciuitate lib.~10. cap~12.}
Wherof our holy Forefathers and ancient Doctours had ſuch a religious
care, that they would not change the very barbariſmes or incongruities
of ſpeach which by long vſe had preuailed in the old readings or
recitings of ſcriptures. as,
\CNote{\XRef{Mt.~22.}}
\L{Neque nubent neque nubentur}, in 
\Cite{Tertullian. li.~4.}
in 
\Cite{Marcion.}
in 
\Cite{S.~Hilarie in c.~22. Mat.}
and in al the Fathers.
\CNote{\XRef{Mar.~8.}}
\L{Qui me confuſus fuerit, confundar & ego eum}, in
\Cite{S.~Cyprian cp.~63. nu.~7.}
\L{Talis enim nobis decebat ſacerdos} (which was an elder tranſlation
then the vulgar Latin that now is) in 
\Cite{S.~Ambroſe c.~3. de fuga ſeculi.}
\CNote{\XRef{Hebr.~7.}}
and S.~Hierom himſelf, who otherwiſe corrected the Latin tranſlation
that was vſed before his time, yet keepeth religiouſly (as himſlef
profeſſeth
\Cite{Præfat. in 4.~Euang. ad Damaſum})
theſe and the like ſpeaches,
\CNote{\XRef{Mat.~6.~20.~22.}}
\L{Nonne vos magis pluris eſtis illis?}
and, \L{filius hominis non venit miniſtrari, ſed miniſtrare}:
and, \L{Neque nubent, neque nubentur}: in his commentaries vpon theſe
places: and,
\CNote{\XRef{Lu.~13.}}
\L{Non capit Prophetam perire extra Hieruſalem}, in his
commentaries in
\Cite{c.~2. Ioel. ſub finem.}
And S.~Auguſtin, who is moſt religious in al theſe phraſes, counteth it
a ſpecial pride and infirmitie in thoſe that haue a litle learning in
tongues, and none in things, that they eaſily take offenſe of the ſimple
ſpeaches or ſoleciſmes in the ſcriptures.  
\Cite{de doctrina Chriſt. li.~2. cap~13.}
See alſo the ſame holy Father
\Cite{li.~3. de doct. Chriſt. c.~3.}
and
\Cite{tract.~2. in Euang. Ioan.}
But of the manner of our tranſlation more anone.

Now,
\MNote{Of the \Sc{annotations}, why they were made, & what matter
  they conteine.}
though the text thus truely tranſlated, might ſufficiently, in the ſight
of the learned and al indifferent men, both controule the aduerſaries
corruptions, and proue that the holy Scripture wherof they haue made ſo
great vantes, maketh nothing for their new opinions, but wholy for the
Catholike Churches beleefe and doctrine, in al the points of difference
betwixt vs: yet knowing that the good and ſimple may eaſily be ſeduced
by ſome few obſtinate perſons of perdition (whom we ſee giuen ouer into
a reprobat ſenſe, to whom the Ghoſpel, which in it-ſelf is the
\CNote{\XRef{2.~Cor.~2.}}
odour of
life to ſaluation, is made the odour of death to damnation, ouer whoſe
eyes for ſinne and diſobedience God ſuffereth a veile or couer to lie,
whiles they read the new Teſtament, euen as
\CNote{\XRef{2.~Cor.~2.}}
the Apoſtle ſaith the Iewes
haue til this day, in reading of the old, that as the one ſort can not
find Chriſt in the Scriptures, read they neuer ſo much, ſo the other can
not find the Catholike Church nor her doctrine there neither) and
finding by experience this ſaying of S.~Auguſtin to be moſt true:
\CNote{\Cite{De doctr. Chriſt. lib.~3. cap.~10.}}
\Emph{If the preiudice of any erronious perſuaſion preoccupate the mind,
  whatſoeuer the Scripture hath to the contrarie, men take it for a
  figuratiue ſpeach}: for theſe cauſes, and ſomewhat to help the
faithful Reader in the difficulties of diuers places, we haue alſo ſet
forth reaſonable large \Sc{Annotations}, thereby to shew the ſtudious
Reader in moſt places perteining to the controuerſies of this time, both
the heretical corruptions and falſe deductions, & alſo the Apoſtolike
tradition, the expoſitions of the holy Fathers, the decrees of the
Catholike Church and moſt ancient Councels: which meanes whoſoeuer
truſteth not, for the ſenſe of holy Scriptures, but had rather follow
his priuate iudgement or the arrogant ſpirit of theſe Sectaries, he shal
worthily through his owne wilfulnes be deceiued: beſeeching al men to
looke with diligence, ſinceritie, and indifferencie, into the caſe that
concerneth no leſſe then euery ones eternal ſaluation or damnation.

Which
\MNote{Hereſies make Catholikes more diligent to ſearch and find
  the ſenſes of holy Scripture for refelling of the ſame.}
if he doe, we doubt not but he shal to his great contentment, find the
holy Scriptures moſt cleerely and inuincibly to proue the articles of
%%% o-2085
Catholike doctrine againſt our aduerſaries, which perhaps he had thought
before this diligent ſearch, either not to be conſonant to God's
words, or at leaſt not conteined in the ſame, and finally he shal proue
this ſaying of S.~Auguſtin to be moſt true:
\CNote{In \Cite{Pſal.~67. prope. finem.}}
\L{Multi ſenſus &c.}
\Emph{Many ſenſes of holy Scriptures lie hidden, & are knowen to ſome
  few of greater vnderſtanding: neither are they at any time auouched
  more commodiously and acceptably then at ſuch times, when the care to
  anſwer heretikes doth force men therunto.  For then, euen they that be
  negligent in matters of ſtudie and learning, shaking of ſluggishnes,
  are ſtirred vp to diligent hearing, that the Aduerſaries may be
%%% 2275
  refelled. Againe, how many ſenſes of holy Scriptures, concerning
  Chriſtes Godhead, haue been auouched againſt Photinus: how many, of
  his Manhood, againſt Manichæus: how many, of the Trinitie, againſt
  Sabellius: how many, of the vnitie in Trinitie, againſt the Arrians,
  Eunomians, Macedonians, how many, of the Catholike Church diſperſed
  throughout the whole world, and of mixture of good and bad in the ſame
  vntil the end of the world, againſt the Donatiſtes and Luciferians
  and other of the like errour: how many againſt al other heretikes,
  which it were too long to rehearſe?  Of which ſenſes and expoſitions
  of holy Scripture the approued Authors and auouchers, should otherwiſe
  either not be knowen at al, or not ſo wel knowen, as the
  contradictions of proud heretikes haue made them.}

Thus he ſaith of ſuch things as not ſeeming to be in holy Scriptures
to the ignorant or heretikes, yet indeed be there.  But in other points
doubted of, that indeed are not decided by Scripture, he giueth vs this
goodly rule to be followed in al, as he exemplifieth in one.  \Emph{Then
  doe we hold} (ſaith he) \Emph{the verity of the Scriptures, when we
  doe that which now hath ſeemed good to the Vniuerſal Church, which the
  authoritie of the Scriptures themſelues doth commend: ſo that, for
  aſmuch as the holy Scripture can not deceiue, whoſoeuer is afraid to
  be deceiued with the obſcuritie of queſtions, let him therin aſke
  counſel of the ſame} \Sc{Chvrch}, \Emph{which the holy Scripture moſt
  certainely and euidently sheweth and pointeth vnto.}
\Cite{Aug. li.~1. cont. Creſcon. c.~13.}

Now
\MNote{Many cauſes why this new Teſtament is tranſlated according
  to the ancient vulgar Latin text.}
to giue thee alſo intelligence in particular, moſt gentle Reader, of
ſuch things as it behoueth thee ſpecially to know concerning our
Tranſlation: We tranſlate the old vulgar Latin text, not the common
Greek text, for theſe cauſes.

1.~It
\MNote{It is moſt ancient.}
is ſo ancient, that it was vſed in the Church of God aboue 1300.\ yeares
agoe, as appeareth by the Fathers of thoſe times.

2.~It
\MNote{Corrected by S.~Hierom.}
is that (by the common receiued opinion and by al probabilitie) which
S.~Hierom afterward corrected according to the Greek, by the appointment
of Damaſus then Pope, as he maketh mention in his Preface before the
foure Euangeliſts, vnto the ſaid Damaſus: and in
\Cite{Catalogo in fine},
and
\Cite{ep.~102.}

3.~Conſequently
\MNote{Commended by S.~Auguſtin.}
\CNote{\Cite{Ep.~10.}}
it is the ſame which S.~Auguſtin ſo commendeth and alloweth in an
Epiſtle to S.~Hierom.

4.~It
\MNote{Vſed and expounded by the Fathers.}
is that, which for the moſt part euer ſince hath been vſed in the
Churches ſeruice, expounded in ſermons, alleaged and interpreted in the
Commentaries and writings of the ancient Fathers of the Latin Church.

5.~The
\MNote{Only authentical, by the holy Councel of Trent.}
\CNote{\Cite{Seff.~4.}}
holy Councel of Trent, for theſe and many other important
conſiderations, hath declared and defined this `only' of al other Latin
tranſlations, to be authentical, and ſo only to be vſed and taken in
publike leſſons, diſputations, preachings, and expoſitions, and that no
man preſume vpon any pretence to reiect or refuſe the ſame.

6.~It
\MNote{Moſt graue, leaſt partial.}
is the graueſt, ſincereſt, of greateſt maieſtie, leaſt partialitie, as
being without al reſpect of controuerſies and contentions, ſpecially
theſe of our
%%% o-2086
time, as appeareth by thoſe places which Eraſmus and others at this day
tranſlate much more to the aſuantage of the Catholike cauſe.

7.~It
\MNote{Preciſe in following the Greek.}
is ſo exact and preciſe according to the Greek, both the phraſe and the
word, that delicate Heretikes therfore reprehend it of rudenes.  And
that it followeth the Greek farre more exactly then the Proteſtants
tranſlations, beſide infinit other places, we appeale to theſe.
\XRef{Tit.~3.~14.}
\L{Curent bonis operibus præeſſe}, \G{προίϛασθαι}.
\Cite{Engl. bib. 1577},
\Emph{to mainteine good workes}, and
\XRef{Heb.~10,~20.}
\L{Viam nobis initiauit}, \G{ἐνεκαίνισεν}.
\Cite{English Bib.}
\Emph{be prepared}.  So in theſe words, \Emph{Iuſtifications,
  Traditions, Idols, &c.}  In al which they come not neer the Greek,
but auoid it of purpoſe.

8.~The
\MNote{Preferred by Beza himſelf.}
Aduerſaries themſelues, namely Beza, preferre it before al the reſt.
\Cite{InPræfat. no. Teſt. an.~1556.}
And againe he ſaith, that the old Interpreter tranſlated very religiouſly
\XRef{Annot. in 1.~Luc v.~1.}

9.~In
\MNote{Al the reſt miſliked of the Sectaries thẽſelues, each
  reprehending another.}
the reſt, there is ſuch diuerſitie and diſſenſion, and no end of
reprehending one another, and tranſlating euery man according to his
fantaſie, that
\CNote{\Cite{Cochla. c.~11. de Cano. Script. authoritate.}}
Luther ſaid, If
%%% 2276
the world should ſtand any long time, we muſt receiue againe (which he
thought abſurd) the Decrees of Councels, for preſeruing the vnitie of
faith, becauſe of ſo diuers interpretations of the Scripture.  And Beza
(in the place aboue mentioned) noteth the itching ambition of his
fellow-tranſlatours, that had much rather diſagree and diſſent from the
beſt, then ſeem themſelues to haue ſaid or written nothing.  And Beza's
tranſlation itſelf, being ſo eſteemed in our countrie, that the Geneua
\CNote{\Cite{The new Te. printed the yeare~1580. in the title.}}
English
Teſtaments be tranſlated according to the ſame, yet ſometime goeth ſo
wide from the Greek & from the meaning in the holy Ghoſt, that
themſelues which proteſt to tranſlate it, dare not follow it.  For
example,
\XRef{Luc.~3.~36.}
They haue put theſe words \Emph{The ſonne of Cainan}, which he wittingly
and wilfully left out: and
\XRef{Act.~1,~14.}
they ſay, \Emph{With the women}, agreably to the vulgar Latin: where he
ſaith, \L{Cum vxoribus}, \Emph{with their wiues}.

10.~It
\MNote{It is truer then the vulgar Greek text itſelf.}
is not only better then al other Latin tranſlations, but then the Greek
text it-ſelf in thoſe places where they diſagree.

The proofe hereof is euident, becauſe moſt of the ancient Heretikes were
Grecians, and therfore the Scriptures in Greek were more corrupted by
them, as the ancient Fathers often complaine.  Tertullian
\CNote{\Cite{Li.~5. cõt. Marcio.}}
noteth the Greek text which is at this day
\XRef{(1.~Cor. 15,~47.)}
to be an old corruption of Marcion the Heretike, and the truth to be as
in our vulgar Latin, \L{Secundus homo de cælo cæliſtis}, \Emph{The
ſecond man from heauen heauenly}.  So
\MNote{The ancient Fathers for proofe therof, and the Aduerſaries
  themſelues.}
read other
\MNote{Ambr. Hierom.}
ancient Fathers, and Eraſmus thinketh it muſt needs be ſo,
and Caluin himſelf followeth it
\Cite{Inſtit. li.~2. c.~13. parag.~2.}
Againe S.~Hierom
\CNote{\Cite{Li.~1. cõt. Iou. c.~7.}}
noteth that the Greek text
\XRef{(1.~Cor.~7,~33.)}
which is at this day, is not the
\Emph{Apoſtolical veritie}
or the true text of the Apoſtle: but that which is in the vulgar Latin,
\L{Qui cum vxore ſet, ſolicitus eſt qua ſunt mundi, quamodo placeat
vxori, & diuiſis eſt}, \Emph{He that is with a wife, is careful of
worldly things, how he may pleaſe his wife, and is diuided or
diſtracted}.  The Eccleſiaſtical hiſtorie called the Tripartite,
\CNote{\Cite{Li.~12. c.~4.}}
noteth the Greek text that now is
\XRef{(1.~Io.~4,~3.)}
to be an old corruption of the ancient Greek copies, by the Neſtorian
Heretikes, and the true reading to be as in our vulgar Latin,
\L{Omnis ſpiritus qui diſſoluit \Sc{Iesvm}, ex Deo non eſt}, \Emph{Euery
ſpirit that diſſolueth \Sc{Iesvs}, is not of God}: & Beza
\CNote{\Cite{Li.~7. c.~32.}}
confeſſeth
that Socrates in his Eccleſiaſtical Hiſtorie readeth ſo in the
Greek, \G{πᾶν πνεύμα ὅ λύει τὸν χριϛὸν} &c.

But
\MNote{The Caluiniſts themſelues often forſake the Greek as
  corrupt, and tranſlate according to the ancient vulgar latin text.}
the proofe is more pregnant out of the Aduerſaries themſelues.  They
forſake the Greek text as corrupted, and tranſlate according to the
vulgar Latin, namely Beza and his ſcholers the English tranſlatours of
the Bible, in theſe places.
\XRef{Hebr. chap.~9. vers.~1.}
ſaying, \Emph{The firſt couenant}, for that which is in the Greek,
%%% RBK:
\TNote{\G{δικαιώματα σκηνή}} % δικαιώμα ?
\Emph{The firſt tabernacle}  Where they put, \Emph{couenant}, not as of
the text, but in another letter, as to be vnderſtood, according to the
vulgar 
%%% o-2087
Latin, which moſt ſincerely leaueth it out altogether, ſaying:
\L{Habuit quidem & prius iuſtificationes &c.} \Emph{The former alſo
indeed had iuſtifications &c.}  Againe 
%%% \XRef{Ro.~11. vers.~21.} ???
\XRef{Ro.~12. vers.~11.}
they tranſlate not according to the Greek text,
\TNote{\G{καιρῷ}}
\L{Tempori ſeruientes},
\Emph{ſeruing the time}, which Beza ſaith muſt needs be a corruption:
but according to the vulgar Latin,
\TNote{\G{κυρίῳ}}
\L{Domino ſeruientes}, \Emph{ſeruing
our Lord}.  Againe, 
\XRef{Apoc.~11. vers.~2.}
they tranſlate not the Greek text, \L{Atrium quod intra templum eſt},
\Emph{the court which is within the temple}, but cleane contrarie,
according to the vulgar Latin, which Beza ſaith is the true reading,
\L{Atrium quod eſt foris Templum}, \Emph{the court which is without the
Temple}.  Only in this laſt place, one English Bible of the yeare 1562.\
followeth the errour of the Greek. Againe,
\XRef{2~Tim.~2. vers.~14.} 
they adde, \Emph{but}, more then is in the Greek, to make the ſenſe more
commodious and eaſie, according as it is in the vulgar Latin.  Againe
\XRef{Ia.~5.~12.}
they leaue the Greek, and follow the vulgar
Latin ſaying,
\TNote{\G{εἰς ὑποκρίσιν}}
\Emph{Leſt you fal into condemnation}.  \Emph{I doubt} not
(ſaith Beza) \Emph{but this is the true and ſincere reading, and I
  ſuſpect the corruption in the Greek came thus &c.}  It were infinit
to ſet downe al ſuch places, where the Aduerſaries (ſpecially Beza)
follow the old vulgar Latin & the Greek copie agreable therunto,
condemning the Greek text that now is, of corruption.

Againe,
\MNote{Superfluities in the Greek which Eraſmus calleth trifling
  and rash additions.}
Eraſmus the beſt tranſlatour of al the later, by Beza's iudgemẽt, ſaith
that the Greek ſometime hath ſuperfluities corruptly added to the text
of holy Scripture, as
\XRef{Mat.~6.}
to the end of the \Emph{Pater noſter}, theſe
words, \Emph{Becauſe thine is the Kingdom, the power and the glorie, for
euer-more}. Which he calleth, \L{nugas}, trifles rashly added to our
Lord's praier, & reprehendeth Valla for blaming the old vulgar Latin
becauſe it hath it not.  Likewiſe 
\XRef{Ro.~11.~6.}
theſe words in the Greek, and not in the vulgar Latin: \Emph{But if of
workes, it is not now grace: otherwiſe the worke is no more a worke}:
and 
\XRef{Mar.~10.~29.}
theſe words, \Emph{or wife}, and ſuch like.
\CNote{See \Cite{No. Teſt. gr. Ro. Stephan. in folio, & Criſpins.}}
Yea the Greek text in theſe
ſuperfluities condemneth it-ſelf, and iuſtifieth the
%%% 2277
vulgar Latin exceedingly; as being marked throughout in a number of
places, that ſuch & ſuch words or ſentences are ſuperflous.  In al
which places our vulgar Latin hath no ſuch thing, but is agreable to the
Greek which remaineth after the ſuperfluities be taken away.  For
example, that before mentioned in the end of the \Emph{Pater noſter},
hath a marke of ſuperfluitie in the Greek text thus `'.  and 
\XRef{Marc 6.~11.}
theſe words, \Emph{Amen I ſay to you; it shal be more tolerable for the
land of Sodom and Gomorrhe in the day of iudgement, then for that
citie}; and 
\XRef{Mat.~19.~22.}
theſe words, \Emph{And be baptized with the Baptiſme that I am baptized
with?}  Which is alſo ſuperflouſly repeated againe 
\XRef{vers 23.}
and ſuch like places exceedingly many: which being noted ſuperfluous in
the Greek, and being not in the vulgar Latin, proue the Latin in thoſe
places to be better, truer, and more ſincere then the Greek.

Wherupon we conclude of theſe premiſſes, that it is no derogation to the
vulgar Latin text, which we tranſlate, to diſagree from the Greek text,
wheras it may notwithſtanding be not only as good, but alſo better.
And
\MNote{The vulgar latin tranſlation agreeth with the beſt Greek
  copies, by Beza's owne iudgement.}
this the Aduerſarie himſelf, their greateſt and lateſt tranſlatour of
the Greek, doth auouch againſt Eraſmus in behalfe of the old vulgar Latin
tranſlation, in theſe notorious words:
\CNote{\Cite{Beza præf. N.~Teſtam. 1556.}
See him alſo
\Cite{Annotat. 13.~Act. v.~20.}}
\Emph{How vnworthily and without
cauſe} (ſaith he) \Emph{doth Eraſmus blame the old Interpreter as
differing from the Greek?  He diſſented, I grant, from thoſe Greek
copies which he had gotten: but we haue found, not in one place, that
the ſame 
%%% o-2088
interpretation which he blameth, is grounded vpon the authoritie of
other Greek copies, & thoſe moſt ancient.  Yea in ſome number of
places we haue obſerued, that the reading of the Latin texts of the
old Interpreter, though it agree not ſometime with our Greek copies,
yet it is much more conuenient, for that it ſeemeth he followed ſome better
and truer copie.}  Thus farre Beza.  In which words he vnwittingly,
but moſt truely, iuſtifieth and defendeth the old vulgar Tranſlation
againſt himſelf and al other cauillers, that accuſe the ſame, becauſe it
is not alwaies agreable to the Greek text: Wheras
\MNote{When the Fathers ſay, that the Latin text muſt yeald to the
Greek and be corrected by it, they meane the true & vncorrupted
Greeke text.}
it was tranſlated out of other Greek copies (partly extant, partly not
extant at this day) either as good and as ancient, or better and more
ancient, ſuch as S.~Auguſtin ſpeaketh of, calling them
\L{doctiores & diligentiores}, \Emph{the more learned and diligent Greek
copies}, wherunto the latin tranſlations that faile in any place, muſt
needs yeald.
\Cite{Li.~2 de doctr. Chriſt. c.~15.}

And if it were not too long to exemplifie and proue this, which would
require a treatiſe by it-ſelf, we could shew by many & moſt cleere
examples throughout the new Teſtament, theſe ſundrie meanes of
iuſtifying the old tranſlation.

Firſt
\MNote{The vulgar latin tranſlation, is many waies iuſtified by
  moſt ancient Greek copies, & the Fathers.}
if it agree with the Greek text (as cõmonly it doth, & in the greateſt
places concerning the controuerſies of our time, it doth moſt
certainely) ſo farre the Aduerſaries haue not to complaine: vnles they
wil complaine of the Greek alſo, as they doe
\XRef{Ia.~4 v.~2.}
and
\XRef{1.~Pet.~3.\ v.~21.}
where the vulgar Latin followeth exactly the Greek text, ſaying,
\L{Occiditis}; and, \L{Quod vos ſimilis forme}, &c.  But Beza in both
places correcteth the Greek text alſo as falſe.

2.~If it diſagree here and there from the Greek text, it agreeth with
another Greek copie ſet in the margent, wherof ſee examples in the
foreſaid Greek Teſtaments of Robert Steuens and Criſpin throughout:
namely
\XRef{2.~Pet.~1,~10.}
\L{Satagite vt per bona opera certam veſtram vocationem faciatis} \G{διὰ
τῶν ἁγαθῶν ἒργῶν}; &
\XRef{Marc.~8.\ v.~7.}
\L{Et ipſos benedixit}, \G{ἐυλογήσας ἀυτὰ}.

3.~If theſe marginal Greek copies be thought leſſe authentical then the
Greek text, the Aduerſaries thẽſelues tel vs the cõtrarie, who in their
tranſlations often follow the marginal copies, and forſake the Greek
text: as in the examples aboue mentioned
\XRef{Rom.~11.}
\XRef{Apoc.~11.}
\XRef{2.~Tim.~2.}
\XRef{Iac.~5.}
&c. it is euident.

4.~If al Eraſmus Greek copies haue not that which is in the vulgar
Latin, Beza had copies which haue it, and thoſe moſt ancient (as he
ſaith) & better. And if al Beza's copies faile in this point and wil
not help vs, Gagneie the French Kings Preacher, and he that might
command in al the Kings Libraries, he found Greek copies that haue iuſt
according to the vulgar Latin: & that in ſuch place as would
ſeeme otherwiſe leſſe probable: as
\XRef{Iac.~3. v.~5},
\L{Ecce quantus ignis quam magnã ſiluã incendit?} 
\Emph{Behold how much fire what a great wood is kindleth}: A man would
thinke it muſt be rather as in the Greek text, \Emph{A litle fire what a
great wood is kindleth}: But an
\CNote{\Cite{Codex veronenſis.}}
approued ancient Greek copie alleaged
by Gagneie, hath as it is in the vulgar Latin.  And if Gagneis copies
alſo faile ſometime, there Beza and Criſpin ſupply Greek copies fully
agreable to the vulgar Latin. as 
\XRef{ep.~Iude vers~5.}
\L{Scientes ſemel \Emph{omnia}, quoniam \Sc{Iesvs} &c.} and
\XRef{vers.~19.}
\L{Segregant \Emph{ſemetipſos}}: likewiſe
\Fix{\XRef{2.~Ephes.~2.}}{\XRef{2.~Thes.~12.}}{There is no
  2. Ephes.  Further the latin appears in Thes., not Ephes.}
%%% 2278
\L{Quod elegerit vos primitias}: \G{ἀπαρχὰς} in ſome Greek copies.
\Cite{Gagn. & 2.~Cor.~9.}
\L{Veſtra amulatio}, \G{ὁ ὑμῶν ζῆλος}
ſo hath one Greek copie.  Beza.

5.~If
\MNote{The Greek Fathers.}
al their copies be not ſufficient, the ancient Greek Fathers had copies
and expounded them agreable to our vulgar Latin, as
\XRef{1.~Tim.~6,~20.}
\TNote{\G{κενοφωνίας}}
\L{Prophanas vocum nouitates}. So readeth S.~Chryſoſtom and expoundeth
it againſt Heretical and erroneous nouelties.  Yet now we know no Greek
copie that readeth ſo. 
%%% o-2089
Likewiſe 
\XRef{Io.~10,~29}
\L{Pater meus quod mihi dedit maius omnibus eſt.} So readeth S.~Cyril
and expoundeth it 
\Cite{li.~7. in Io. c.~10.}
likewiſe,  
\XRef{1.~Io.~4,~3.}
\L{Omnis Spiritus qui ſoluit \Sc{Iesvm}, ex Deo non eſt.} So readeth 
\Cite{S.~Irenæus li.~3. c.~18.}
\Cite{S.~Auguſtin tract.~6. in Io.}
\Cite{S.~Leo epiſt.~10. c.~5.}
beſide Socrates in his 
\Cite{Eccleſiaſtical hiſtorie li.~7 c.~22.}
and the 
\Cite{Tripartite li.~12 c.~4.}
who ſay plainely, that this was the old and the true reading of this
place in the Greek.  And in what Greek copie extant at this day is there
this text
\XRef{Io.~5.~2.}
\TNote{\G{ἐπὶ προβατικῇ}}
\L{Eſt autem Hieroſolymis probatica piſcina?} and yet S.~Chryſoſtom,
S.~Cyril, and Theophylacte read ſo in the Greek, and Beza ſaith it is
the better reading.  And ſo his the Latin text of the Romane Maſſe-book
iuſtified, and eight other Latin copies, that read ſo.  For our vulgar
Latin here, is according to the Greek text, \L{Super probatica.} and
\XRef{Ro.~5. v.~17.}
\L{Donationis & Iuſtitia.} So readeth Theodorete in Greek.  &
\XRef{Lu.~2 v.~14.}
Origen and S.~Chryſoſtom read, \L{Hominibus bonæ voluntatis},
and Beza liketh it better then the Greek text that now is.

\Fix{6.~Were}{6.~Where}{obvious typo, fixed in other.}
there is no ſuch ſigne or token of any ancient Greek copie in the
Fathers, yet theſe later interpreters tel vs, that the old Interpreter
did follow ſome other Greek copie.  As
\XRef{Marc~7,~3.}
\L{Niſi crebro lauerint.} Eraſmus thinketh that he did read in the Greek
\G{πυκνῆ} \Emph{often}: and Beza and others commend his coniecture, yea
and the English Bibles are ſo tranſlated.  Whereas now it is \G{πυγμῆ}
which ſignifieth the length of the arme vp to the elbow.  And who would
not thinke that the Euangeliſt should ſay; The Phariſees wash often, becauſe
otherwiſe they eate not, rather then thus, \Emph{Vnles they wash vp to
the elbow, they eate not?}

7.~If
\MNote{The Latin Fathers.}
\CNote{See \Cite{Annot. Louan. in N.~Teſt. & anno. Luca Brugen. in biblia.}}
al ſuch coniectures, and al the Greek Fathers help vs not, yet the Latin
Fathers with great conſent wil eaſily iuſtifie the old vulgar
tranſlation, which for the moſt part they follow and expound.  As
\XRef{Io.~7.~39.}
\L{Nondum erat ſpiritus datus.} So readeth S.~Auguſtin
\Cite{Li.~4. de Trinit. c.~20.}
and 
\Cite{li.~83. Queſt. q.~62.}
and
\Cite{tract.~52. in Ioan.}
\Cite{Leo ſer.~2. de Pentecoſte.}
Whoſe authoritie were ſufficient, but indeed Didymus alſo a Greek
Doctour readeth ſo
\Cite{li.~2. de Sp.~ſancto},
tranſlated by S.~Hierom, and a Greek copie in the Vaticane, and the
Syriake new Teſtament.  Likewiſe
\XRef{Io.~21.~22.}
\L{Sic eum volo manere}. So read S.~Ambroſe, in
\XRef{Pſal.~45.}
&
\Cite{Pſal.~118. octonario Resp.}
S.~Auguſtin and Vene.\ Bede vpon S.~Iohns Ghoſpel.

8.~And laſtly, if ſome other Latin Fathers of ancient time, read
otherwiſe, either here or in other places, not al agreeing with the text
of our vulgar Latin, the cauſe is, the great diuerſitie and multitude,
that was then of Latin copies, 
\CNote{\Cite{Præfat. in 4.~Eu. ad Damaſum.}}
(wherof S.~Hierom complaineth) til this
one vulgar Latin grew only into vſe.  Neither doth their diuers reading
make more for the Greek, then for the vulgar Latin, differing oftentimes
from both.  As when S.~Hierom in this laſt place readeth, \L{Si ſic eum
volo manere},
\Cite{li.~1. adu.~Ionin.}
It is according to no Greek copie now extant.  And if yet there be ſome
doubt, that the readings of ſome Greek or Latin Fathers, differing from
the vulgar Latin, be a check or condemnation to the ſame: let Beza: that
is, let the Aduerſarie himſelf, tel vs his opinion in this caſe alſo.
\CNote{\Cite{Præfat. citata.}}
\Emph{Whoſoeuer}, ſaith he, \Emph{shal take vpon him to correct theſe
things} (ſpeaking of the vulgar Latin tranſlation) \Emph{out of the
ancient Fathers writings, either Greek or Latin, vnles he doe it very
circumſpectly & aduiſedly, he shal ſurely corrupt al rather then amend it,
becauſe it is not to be thought, that as often as they cited any
place, they did alwaies looke into the book, or number euery word.}
As if he should ſay: We may not by and by thinke that the vulgar Latin
is faultie and to be corrected, when we read otherwiſe in the Fathers
either Greek or Latin, becauſe they did not alwaies exactly cite the
words, but followed ſome 
%%% o-2090
commodious and godly ſenſe therof.

Thus
\MNote{The few and ſmal faults negligently crept into the vulgar
  Latin tranſlation.}
then we ſee that by al meanes the old vulgar Latin tranſlation is
approued good, and better then the Greek text it-ſelf, and that there is
no cauſe why it should giue place to any other text, copies, or
readings.  Marie if there be any faults euidently crept in by thoſe that
heretofore, wrote or copied out the Scriptures (as there be ſome) them
we grant no leſſe, then we would grant faults now adaies committed by
the Printer, and they are exactly noted of Catholike Writers, namely in
al Plantins
%%% 2279
Bibles ſet forth by the Diuines of Louan: and the holy
\CNote{\Cite{Seſſ.~4.}}
Councel of Trent
willeth that the vulgar Latin text be in ſuch points throughly mended,
and ſo to be moſt authentical.  Such faults are theſe \L{In fide},
for, \L{in fine}: \L{Præſcientiam}, for, \L{præſentiam}: \L{Suſcipiens},
for, \L{Suſpiciens}: and ſuch like very rare.  Which are euident
corruptions made by the copiſtes, or growen by the ſimilitude of words.
Theſe being taken away, which are no part of thoſe corruptions and
differences before talked of, we tranſlate that text which is moſt
ſincere, and in our opinion and as we haue proued, incorrupt.  The
Aduerſaries contrarie, tranſlate that text which themſelues confeſſe
both by their writings and doings, to be corrupt in a number of places,
and more corrupt then our vulgar Latin, as is before declared.

And
\MNote{The Caluineſts confeſſing the Greek to be moſt corrupt yet
  tranſlate that only, and hold that only for authentical Scripture.}
if we would here ſtand to recite the places in the Greek which Beza
pronounceth to be corrupted, we should make the Reader to wonder, how
they can either ſo plead otherwiſe for the Greek text, as though there
were no other truth of the new Teſtament but that: or how they tranſlate
only that (to deface, as they thinke, the old vulgar Latin) which
themſelues ſo shamfully diſgrace, more then the vulgar Latin, inuenting
corruptions where none are, nor can be, in ſuch vniuerſal conſent of al
both Greek and Latin copies.  For example,
\XRef{Mat.~10.}
\Emph{The firſt Symon, who is called Peter.}  I thinke
\CNote{In \Cite{Annot. No. Teſt. an.~1556.}}
(ſaith Beza) this
word \G{πρῶτος}, \Emph{firſt}, hath beẽ added to the text of ſome that
would eſtablish Peters Primacie.  Againe
\XRef{Luc.~22.}
The Chalice \Emph{that is shed for you}.  It is moſt likely (ſaith he)
that theſe words being ſometime but a marginal note, came by corruptiõ
out of the margẽt into the text.  Againe
\XRef{Act.~7.}
Figures which they made, \Emph{to adore them}.  It may be ſuſpect (ſaith
he) that theſe words, as many other, haue crept by corruption into the
text out of the margent.  And
\XRef{1.~Cor.~15.}
He thinketh the Apoſtle ſaid not \G{νῖκοσ},
\Emph{victorie}, as it is in al Greek copies,
but \G{νεῖκοσ}, \Emph{contention}.  And  
\XRef{Act.~13.}
he calleth it a manifeſt errour, that in the Greek that is, \Emph{400
  yeares}, for, \Emph{300}.  And 
\XRef{Act.~7. v.~16.}
he rekneth vp a whole catalogue of corruptions: namely
\XRef{Marc~12. v.~42.}
\G{ὅ ἐϛι κοδράντης}, \Emph{which is a farthing}: and \G{ἁυτη ἐϛίν
ἔρημος} 
\XRef{Act.~8. vers.~26.}
\Emph{This is deſert}.  And
\XRef{Act.~7. v.~16.}
the name of Abraham, and ſuch like.  Al which he thinketh to haue been
added or altered into the Greek text by corruption.

But among other places, he laboureth exceedingly to proue a great
corruption
\XRef{Act.~7 v.~14.}
where it is ſaid (according to the \Emph{Septuaginta}, that is, the
Greek text of the old Teſtament) that Iacob went downe into Aegypt with
75.~ſoules.  And he thinketh theſe words \G{τοῦ καινὰν},
\Emph{which was of Cainan}, to be ſo falſe, that he leaueth them cleane
out in
\CNote{\Cite{An. Do.~1556. &~1565.}}
both his editions of the new Teſtament: ſaying, that he is bold
ſo to doe, by the authoritie of Moyſes.  Whereby he wil ſignifie, that
it is not in the Hebrew text of Moyſes or of the old Teſtament, and
therfore it is falſe in the Greek of the new Teſtament.
\MNote{They ſtanding preciſely vpon the Hebrew of the old, and
  Greek text of the new Teſtament, muſt of force denie the one of them.}
\XRef{Luc.~3. v.~36.}
Which conſequence of theirs (for it is common among them and concerneth
al Scriptures) if it were true, al places of the Greek text of the new
Teſtament, cited out of the old according to the Septuaginta, and not
%%% o-2091
according to the Hebrew (which they know are very many) should be falſe,
and ſo by tying themſelues only to the Hebrew in the old Teſtament, they
are forced to forſake the Greek of the new: or if they wil mainteine the
Greek of the new, they muſt forſake ſometime the Hebrew in the old.  But
this argument shal be forced againſt them elſwhere.

By this litle, the Reader may ſee what gay patrones they are of the
Greek text, and how litle cauſe they haue in their owne iudgements to
tranſlate it, or vant of it, as in derogation of the vulgar Latin
tranſlation, & how eaſily we might anſwer them in a word why we
tranſlate not the Greek: forſooth becauſe it is ſo infinitly corrupted.
But
\MNote{They ſay the Greek is more corrupt thẽ we wil grant thẽ.}
the truth is, we doe by no meanes grant it ſo corrupted as they ſay,
though in compariſon we know it leſſe ſincere and incorrupt then the
vulgar Latin, and for that cauſe and others before alleaged we preferre
the ſaid Latin, and haue tranſlated it.

If yet there remaine one thing which perhaps they wil ſay, when they can
not anſwer our reaſons aforeſaid; that we preferre the vulgar Latin
before the Greek text, becauſe the Greek maketh more againſt vs: we
\MNote{We preferre not the vulgar Latin text, as making more for
  vs.}
proteſt that as for other cauſes we preferre the Latin, ſo in this
reſpect of making for vs or againſt vs, we allow the Greek as much as
the Latin, yea
\MNote{The Greek maketh for vs more then the vulgar Latin.}
in ſundrie places more then the Latin, being aſſured that they haue not
one, and that we haue many aduantages in the Greek more then in the Latin, as
by the Annotations of this new Teſtament shal euidently appeare: namely
in al ſuch places where they dare not tranſlate the Greek, becauſe it is
for vs and againſt them.  As when they tranſlate,
\G{δικαιώματα},
\Emph{ordinances}, and not \Emph{iuſtifications}, and that of purpoſe
%%% 2280
as Beza confeſſeth
\XRef{Luc.~1,~6.}
\G{παραδὸσεις}, \Emph{ordinances} or \Emph{inſtructions}, and
not \Emph{traditions}, in the better part.
\XRef{2~Theſs.~2,~15.}
\G{πρεσβυτέρους} \Emph{Elders}, and not \Emph{Prieſts}: \G{ἐιδωλα}, 
\Emph{images} rather then \Emph{idols}  And
\MNote{For the real preſence.}
eſpecially when
\CNote{\XRef{Luc.~22. v.~20.}}
S.~Luke in the Greek ſo maketh for vs (the vulgar Latin
being indifferent for them & vs) that Beza ſaith it is a corruption
crept out of the margent into the text.  What need theſe abſurd diuiſes
and falſe dealings with the Greek text, if it made for them more then
for vs, 
\Fix{eya}{yea}{obvious typo, fixed in other}
if it made not for vs againſt them?  But that the Greek maketh more for
vs, ſee
\XRef{1.~Cor.~7.}
In
\MNote{For faſting.}
the Latin, \Emph{Defraud not one another, but for a time, that you giue
  your ſelues to praier}: in the Greek, \Emph{to faſting and prayer}.
\XRef{Act.~10,~30.}
in the Latin, Cornelius ſaith, \Emph{From the fourth day paſt vntil this
  houre I was praying in my houſe, and behold a man &c.} in the Greek,
\Emph{I was faſting, and praying}.
\XRef{1.~Io.~5,~18.}
in
\MNote{For free-wil.}
the Latin: \Emph{We know that euery one which is borne of God ſinneth
  not: But the generation of God preſerueth him &c.} In the Greek,
\Emph{But he that is borne of God preſerueth himſelf.}
\XRef{Apoc.~22,~14}
in
\MNote{Againſt only faith.}
the Latin, \Emph{Bleſſed are they that wash their garments in the bloud
  of the Lamb &c.} in the Greek, \Emph{Bleſſed are they that doe his
  commandements} 
\XRef{Rom.~8,~38.}
\L{Certus ſum &c.}
\MNote{Againſt ſpecial aſſurance of ſaluation.}
\Emph{I am ſure that neither death nor life, nor other creature is able
  to ſeparate vs from the charitie of God}: as though he were aſſured or
we might and should aſſure our-ſelues of our predeſtination: in the
Greek, \G{πέπεισμαι},
\Emph{I am probably perſuaded that neither death nor life &c.} In
\MNote{For the Sacrifice of Chriſt's body and bloud.}
the Euangeliſts about the Sacrifice and B.~Sacrament, in the Latin thus:
\Emph{This is my bloud that shal be shed for you}: and in S.~Paul,
\Emph{This is my body which shal be betraied or deliuered for you}: both
being referred to the time to come & to the Sacrifice on the Croſſe: in
the Greek, \Emph{This is my bloud which is shed for you}, and, \Emph{my
  body which is broken for you}: both being referred to that preſent
time when Chriſt gaue his body and bloud at his ſupper, then sheading
the one and breaking the other, that is ſacrificing it Sacramentally and
myſtically.  Loe theſe & the like our aduantages in the Greek more then
in the Latin.
%%% o-2092

But
\MNote{The Proteſtãts condemning the old vulgar tranſlation as
  making for vs, condemne thẽſelues.}
is the vulgar tranſlation, for al this Papiſtical, and therfore doe we
follow it?  for ſo ſome of them cal it, and ſay it is
\CNote{\Cite{Againſt D. Sand. Rocke pag.~147.}
See
\Cite{Kem. in exam. Concil. Trident. Seſſ.~4.}}
the worſt of al
other.  If it be, the Greek (as you ſee) is more, and ſo both Greek and
Latin and conſequently the holy Scriptures of the new Teſtament is
Papiſtical.  Againe if the vulgar Latin be Papiſtical, Papiſtrie is very
ancient, and the Church of God for ſo many hundred yeares wherin it hath
vſed and allowed this tranſlation, hath been Papiſtical.  But wherin is
it Papiſtical?  forſooth in theſe phraſes and ſpeaches,
\CNote{\XRef{Mt.~3.}
&
\XRef{11.}}
\L{Pænitetiam agite},
\CNote{\XRef{Eph.~5.}}
\L{Sacramentum hoc magnum eſt},
\CNote{\XRef{Luc.~1.}}
\L{\Sc{Ave gratia plena}},
\CNote{\XRef{Heb.~13.}}
\L{Talibus hoſtiis promeratur Deus}; and ſuch like.  Firſt, doth not the
Greek ſay the ſame?  See the Annotations vpon theſe places.  Secondly,
could he tranſlate theſe things Papiſtically or partially, or rather
prophetically ſo long before they were in controuerſie?  Thirdly, doth
he not ſay for, \L{pænitentiam agite}, in another place,
\CNote{\XRef{Mar.~1.}}
\L{pœnitemini}: 
and doth he not tranſlate other myſteries by the word \L{Sacramentum}, 
as
\XRef{Apoc.~17.}
\L{Sacramentum mulieris} and as he tranſlateth one word,
\TNote{\G{κεχαριτωμένη}}
\L{Gratia plena}, ſo doth he not tranſlate the very like word,
\TNote{\G{εἱλκωμένος}}
\CNote{\XRef{Luc.~16. v.~20.}}
\L{plenus vlceribus}, which themſelues doe follow alſo?  Is this alſo
Papiſtrie? When he ſaid,
\XRef{Hebr.~10.~29.}
\L{Quantum deteriora merebitur ſupplicia &c.} & they like it wel enough,
might he not haue ſaid according to the ſame Greek words,
\L{Vigilate vt mereamini fugere iſta omnia & ſtare ante filium
  hominis.}
\XRef{Luc.~21,~36.}
and, \L{Qui merebuntur ſæculum illud & reſurrectionem ex mortuis &c.}
\XRef{Luc.~20,~35.}
and \L{Tribulationes quas ſuſtinetis,vt mereamini regnum Dei, pro quo &
patimini.} 
\XRef{2.~Theſſ.~1,~5.}
Might he not (we ſay) if he had partially effectated the word merits,
haue vſed it
\Fix{is}{in}{obvious typo, fixed in other.}
al theſe places, according to his and
\CNote{\Cite{No. Te.~1580.}}
your owne tranſlation of the ſame
Greek word
\XRef{Heb.~10,~29}?
Which he doth not, but in al theſe places ſaith ſimply \L{Vt digni
habeamini}, and, \L{Qui digni habebuntur}. And how can it be iudged
Papiſtical or partial, when he ſaith, \L{Talibus hoſtiis promeretur
Deus}, 
\XRef{Heb.~23}?
Was Primaſius alſo, S.~Auguſtines ſcholer, a Papiſt, for
\CNote{\Cite{in ep. ad Heb.}}
vſing this text,
and al the reſt that haue done the like?  Was S.~Cyprian a Papiſt, for
\CNote{\Cite{Ep.~14.}
&
\Cite{18.}}
vſing ſo often this ſpeach, \L{promereri Dominum iuſtis operibus,
pænitentia &c}? or is there any difference, but that S.~Cyprian vſeth it
as a deponent more latinly the other as a paſſiue leſſe finely?  Was
\MNote{The Papiſtrie therof (as they terme it) is in the very
  ſentẽces of the Holy Ghoſt, more then in the trãſlation.}
it Papiſtrie, to ſay, \L{Senior} for \L{Preſbiter}, \L{Miniſtrantibus} 
for \L{ſacrificantibus} or \L{liturgiam celebrantibus}, \L{ſimulachris}
for \L{idolis}, \L{fides tua te ſaluam fecit} ſometime for \L{ſanum
fecit}? Or shal we thinke he was a Caluiniſt for tranſlating thus, as
they thinke he was a Papiſt, when any word ſoundeth for vs?

Againe, was he a Papiſt in theſe kind of words only, and was he not in
whole sentences? as,
\CNote{\XRef{Mat.~16.}}
\L{Tibi dabo claues, &c.} \L{Quis quid ſolueris in terra, erit ſolutum &
in cælis}: and,
\CNote{\XRef{Io.~20.}}
\L{Quorum
%%% 2281
remiſeritis peccata, remittuntur eis}; and,
\CNote{\XRef{Mat.~16.}}
\L{Tunc reddet vnicuique ſecundum opera ſua}; and,
\CNote{\XRef{Iac.~2.}}
\L{Nunquid poterit fides ſaluare eum}?
\L{Ex operibus iuſtificatur homo & non ex fide tantum}; and,
\CNote{\XRef{1.~Tim.~5.}}
\L{Nubere volunt, damnationem habentes, quia primam fidem irritam
ſecerunt}; and,
\CNote{\XRef{1.~Io.~5.}}
\L{Mandata eius grauia non ſunt}; and, 
\CNote{\XRef{Heb.~11.}}
\L{Aſpexit in
remunerationem}. Are al theſe and ſuch, Papiſtical tranſlations, becauſe
they are moſt plaine for the Catholike faith which they cal Papiſtrie?
Are they not word for word as in the Greek, and the very words of the
holy Ghoſt? And if in theſe there be no accuſation of Papiſtical
partiality, why in the other?  Laſtly, are the Ancient Fathers, General
Councels, the Churches of al the weſt part, that vſe al theſe ſpeaches
and phraſes now ſo many hundred yeares, are they al Papiſtical?  Be it
ſo, and let vs in the name of God follow them, ſpeake as they ſpake,
tranſlate as they tranſlated, interpret as they interpreted, becauſe we
beleeue as they beleeued.  And thus farre for defenſe of the old vulgar
Latin tranſlation, and why we tranſlated it before al others: Now of the
manner of tranſlating the ſame.

In
\MNote{The manner of this tranſlatiõ and what hath been obſerued
  therin.} 
this our tranſlation, becauſe we wish it to be moſt ſincere, as
becommeth a Catholike tranſlation, & haue endeauoured ſo to make it: we
are very preciſe & religious in following our copie, the old vulgar
approued Latin; not only in ſenſe, which we hope we alwaies doe, but
ſometime in the very words alſo and phraſes: which may ſeeme to the
vulgar Reader & to common English eares not yet aquainted therewith,
rudeneſſe or ignorance: but to the diſcret Reader that deeply weigheth
and conſidereth the importance of ſacred words and ſpeaches, and how
eaſily the voluntarie Tranſlatour may miſſe the true ſenſe of the Holy
Ghoſt, we doubt not but our conſideration and doing therin, shal ſeem
reaſonable and neceſſarie: yea and that al ſorts of Catholike Readers
wil in short time thinke that familiar, which at the firſt may ſeem
ſtrange, & wil eſteem it more, when they shal otherwiſe be taught to
vnderſtand it, then if it were the common knowen English.

For
\MNote{Certaine wordes not English nor as yet familiar in the
  English tõgue.}
example, we tranſlate often thus, \Emph{Amen, amen, I ſay vnto you};
which as yet ſeemeth ſtrange.  But after a while it wil be as familiar,
as \Emph{Amen} in the end of al praiers and Pſalmes.  And euen as when
we end with, \Emph{Amen}, it ſoundeth farre better then, \Emph{So be
  it}: ſo in the beginning, \Emph{Amen, Amen}, muſt needs by vſe and
cuſtom ſound farre better, then, \Emph{Verily verily}.  Which indeed doth
not expreſſe the aſſeueration and aſſurance ſignified in this Hebrew
word.  Beſides that it is the ſolemne and vſual word of our Sauiour
\CNote{See
\XRef{ãnot. Io. c.~8. v.~14.}
&
\XRef{Apoc. c.~19. v.~4.}}
to
expreſſe a vehement aſſeueration, and therfore is not changed, neither
in the Syriake, nor Greek, nor vulgar Latin Teſtament, but is preſerued
and vſed of the Euangeliſts and Apoſtles themſelues, euen as Chriſt
ſpake it \L{propter ſanctiorem authoritatem} as S.~Auguſtin ſaith of
this and of \Emph{Allelu-ia, for the more holy and ſacred authoritie
therof}. 
\Cite{li.~2. Doct. Chriſt. c.~11.}
And therfore doe we keep the word \Emph{Allelu-ia}.
\XRef{Apoc.~19.}
as it is both in Greek and Latin, yea and in al the English
tranſlations, though in their books of common praier they tranſlate it, 
\Emph{Praiſe ye the Lord}.  Againe if \Emph{Hoſanna, Raca, Belial}, and
ſuch like be yet vntranſlated in the English Bibles, why may not we ſay,
\Emph{Corbana}, and \Emph{Paraſceue}: ſpecially when they Englishing
this later thus,
\CNote{\Cite{No. Teſt. an.~1580.}
\Cite{Bibl. an.~1577.}}
\Emph{the preparation of the Sabboth}, put three words
more into the text, then the Greek word doth ſignifie.
\XRef{Mat.~27,~62.}
And others ſaying thus: After the day \Emph{of preparing}, make a cold
tranſlation and short of the ſenſe: as if they should tranſlate,
Sabboth, \Emph{the reſting}: For,
\CNote{\XRef{Mat.~14. v.~42.}}
\Emph{Paraſceue} is as ſolemne a word
for the Sabboth eue, as \Emph{Sabboth} is for the Iewes ſeauenth day,
and now among Chriſtians much more ſolemner, taken for Good-friday
only.  Theſe words then we thought farre better to keep in the text, &
to tel their ſignification in the margent or in a table
\SNote{See in the end of this Book after al the Tables, an explication
  of ſuch words as are not familiar to the vulgar Reader.}
for that purpoſe, then to diſgrace both the text and them with
tranſlating them.  Such are alſo theſe words, \Emph{The Paſch, The feaſt
  of Azymes, The bread of Propoſition}.  Which they tranſlate:
\CNote{\Cite{Bibl.~1577. Mat.~26,~17.}}
\Emph{The Paſſe-ouer, The feaſt of ſweet bread, The shew bread}.  But if
\Emph{Pentecoſt} 
\XRef{Act.~2.}
be yet vntranſlated in their Bibles, and ſeemeth not ſtrange; why should
not \Emph{Paſch} and \Emph{Azymes} ſo remaine alſo, being ſolemne
feaſts, as Pentecoſt was?  or why should they english one rather then
the other?  ſpecially wheras \Emph{Paſſe-ouer} at the firſt was as
ſtrange, as \Emph{Paſch} may ſeem now, and perhaps as many now
vnderſtand \Emph{Paſch}, as \Emph{Paſſe-ouer}.  And as for \Emph{Azymes},
when they english it, \Emph{the feaſt of ſweet bread}, it is a falſe
interpretation of the word, and nothing expreſſeth that which belongeth
to the feaſt, concerning vnleauened bread.  And as for their terme of
\Emph{shew bread}, it is very ſtrange and ridiculous.  Againe,
if \Emph{Proſelyte} be a receiued word in the English Bibles 
\XRef{Mat.~23.}
\XRef{Act.~2.}
why may not we be bold to ſay, \Emph{Neophyt}.
\XRef{1.~Tim.~3}?
ſpecially when they tranſlating it into English, doe falſely expreſſe
the
%%% o-2094
ſignification of
%%% 2282
the word thus, \Emph{a yong ſcholer}.  Whereas it is a peculiar word to
ſignifie them that were lately baptized, as \Emph{Catechumenus},
ſignifieth the newly inſtructed in faith not yet baptized, who is alſo a
yong ſcholer rather then the other, and many that haue been old
ſcholers, may be \Emph{Neophyts} by differing Baptiſme.  And
if \Emph{Phylacteries} be allowed for English 
\XRef{Mat.~23.}
we hope that \Emph{Didrachmes} alſo, \Emph{Prepuce, Paraclete}, and ſuch
like, wil eaſily grow to be currant and familiar.  And in good ſooth
there is in al theſe ſuch neceſſitie, that they can not conueniently be
tranſlated.  As when S.~Paul ſaith,
\CNote{\XRef{Phi.~3.}}
\L{conciſio, non circumciſio}; how
can we but follow his
\Fix{wery}{very}{obvious typo, fixed in other}
words and alluſion?  And how is it poſſible to expreſſe \L{Euangelizo},
but as we doe, \Emph{Euangelize}? for \L{Euangelium} being the Ghoſpel,
what is \L{Euangelizo} or to \Emph{Euangelize}, but to shew the glad
tydings of the Ghoſpel, of the time of grace, of al Chriſt's benefits?
Al which ſignification is loſt, by tranſlating as the English Bibles
doe, \Emph{I bring you good tydings.}
\XRef{Luc.~2.~10.}
Therfore we ſay \Emph{Depoſitum},
\XRef{1.~Tim.~6.}
and, He \Emph{exinanited} himſelf,
\XRef{Philip.~2.}
and, You haue \Emph{reflorished},
\XRef{Philip.~4.}
and, \Emph{to exhauſt}.
\XRef{Hebr.~9,~28}
becauſe we can not poſſibly attaine to expreſſe theſe words fully in
English: and we thinke much better, that the Reader ſtaying at the
difficultie of them, should take an occaſion to looke in their table, or
otherwiſe to aſke the ful meaning of them, then by putting ſome vſual
English words that expreſſe them not, ſo to deceiue the Reader.
Sometime alſo we doe it for another cauſe.  As when we ſay, \Emph{The
  aduent of our Lord},
\MNote{Why we ſay, \Emph{our Lord}, not, \Emph{the Lord} (but in
  certaine caſes) ſee the Annot. 1.~Tim.~6.}
and, \Emph{Impoſing of hands}, becauſe one is a ſolemne time, the other a
ſolemne action in the Catholike Church: to ſignifie to the people, that
theſe & ſuch like names come out of the very Latin text of the
Scripture.  So
\MNote{Catholike termes proceeding from the very text of
  Scripture.}
did \Emph{Penance, doing penance, Chalice, Prieſt, Deacon, Traditions,
  Altar, Hoſt}, and the like (which we exactly keep as Catholike termes)
proceed euen from the very words of Scripture.

Moreouer, we preſume not in hard places to mollifie the ſpeaches or
phraſes, but religiouſly keep them word for word, and point for point,
for feare of miſſing, or reſtraining the ſenſe of the holy Ghoſt to our
phantaſie.  As
\MNote{Certaine hard ſpeaches and Phraſes.}
\XRef{Eph.~6.}
\Emph{Againſt the ſpirituals of wickednes in the celeſtials}: and,
\CNote{\XRef{Io.~2.}}
\Emph{What to me and thee woman?} wherof ſee the Annotation vpon this
place: and 
\XRef{1.~Pet.~2.}
\Emph{As infants euen now borne, reaſonable, milke without guile deſire
  ye.}  We doe ſo place, \Emph{reaſonable}, of purpoſe, that it may be
indifferent both to infants going before, as in our Latin text; or to
milke that followeth after, as in other Latin copies and in the Greek.
\XRef{Io.~3.}
we tranſlate, \Emph{The ſpirit breatheth where he wil, &c} leauing it
indifferent to ſignifie either the holy Ghoſt, or wind: which
\MNote{The Proteſtãts preſumptuous boldnes and libertie in
  tranſlating.} 
the Proteſtants tranſlating, \Emph{wind}, take away the other ſenſe more
common and vſual in the Ancient Fathers.  We tranſlate
\XRef{Luc.~8.~23.}
\Emph{They were filled}, not adding of our owne, \Emph{with water}, to
mollifie the ſentence, as the Proteſtants doe: and 
\XRef{c.~22.}
\Emph{This is the chalice, the New Teſtament, &c} and not, \Emph{This
  chalice is the New Teſtament: &c.} likewiſe,
\XRef{Mar.~13.}
\Emph{Thoſe daies shal be ſuch tribulation}, not as the Aduerſaries,
\Emph{in thoſe daies}, both our text and theirs being otherwiſe:
likewiſe 
\XRef{Iac.~4,~6.}
\Emph{And giueth greater grace}, leauing it indifferent to the
\Emph{Scripture}, or to the \Emph{holy Ghoſt}, both going before.
Wheras the Aduerſaries 
\Fix{tooto}{to to}{obvious typo, fixed in other}
boldly & preſumptuouſly adde, ſaying: \Emph{The Scripture giueth},
taking away the other ſenſe, which is farre more probable.  Likewiſe
\XRef{Hebr.~12,~21.}
we tranſlate, \Emph{So terrible was it which was ſeen, Moyſes ſaid,
  &c.} neither doth Greek or Latin permit vs to adde, \Emph{that}
Moyſes ſaid, as the Proteſtants preſume to doe.  So we ſay \Emph{Men
  Brethren, A widow woman, A woman a ſiſter, Iames of Alphæus}, and the
like.  Sometime alſo we follow of purpoſe the Scriptures phraſe: as,
\CNote{\XRef{Mat.~5.}}
\Emph{The hel of fire}, according to Greek and
%%% o-2095
Latin; which we might ſay perhaps,
\Emph{the firy hel}, by the Hebrew
phraſe in ſuch ſpeaches, but not,
\TNote{Gehenna ignis.}
\Emph{hel fire}, as commonly it is tranſlated.  Likewiſe
\XRef{Luc.~4,~36.}
What \Emph{word} is this, that in power and authoritie he commandeth the
vncleane ſpirits? as alſo,
\XRef{Luc.~2.}
Let vs paſſe ouer, and ſee the \Emph{word} that is done.  Where we might
ſay, \Emph{thing}, by the Hebrew phraſe; but there is a certaine
maieſtie and more ſignification in the ſpeaches, and therfore both Greek
& Latin keep them, although it is no more the Greek & Latin phraſe,
then it is the English.  And why should we be ſquamish at new words or
phraſes in the Scripture, which are neceſſarie: when we doe eaſily admit
and follow new words coyned in court and in courtly or other ſecular
writings? 

We
\MNote{The Greek added often in the margent for many cauſes.}
adde the Greek in the margent for diuers cauſes.  Sometime when the
ſenſe is hard, that the learned Reader may conſider of it and ſee if he
can help himſelf better then by our tranſlation.  As
\XRef{Luc.~11.}
\L{Nolite extolli.} \G{μὴ μετεωρίζεσθε}. and againe \L{Quod ſupereſt
date eleemoſynam.} \G{τὰ ενόντα}. Sometime to take away the ambiguitie
of the Latin or English;
%%% 2283
as
\XRef{Luc.~11.}
\L{Et domus ſupra domum cadet.} Which we muſt needs english,
\Emph{and houſe vpon houſe shal fal.}  By the Greek, the ſenſe is not,
one houſe shal vpon another; but if one houſe riſe vpon it-ſelf, that is
againſt it-ſelf, it shal perish.  According as he ſpeaketh of a Kingdom
deuided againſt it-ſelf, in the words before.  And
\XRef{Act.~14.}
\L{Sacerdos Ionis qui erat}, in the Greek, \Emph{qui}, is referred to
Iupiter.  Sometime to ſatisfie the Reader, that might otherwiſe conceiue
the tranſlation to be falſe. As 
\XRef{Philip.~4, v.~6.}
\Emph{But in euery thing by praier}, &c.
\G{ἐν παντὶ προσευχῆ},
not \Emph{in al prayer}, as in the Latin it may ſeem.  Sometime when the
Latin neither doth, nor can reach to the ſignification of the Greek
word, we adde the Greek alſo as more ſignificant.
\CNote{\XRef{Mat.~4.}}
\L{Illi ſoli ſeruies},  \Emph{him only shal thou ſerue},
\G{λατρεύσεις} And
\XRef{Act.~6.}
Nicolas a \Emph{ſtranger} of Antioch, \G{προσήλυτος} &
\XRef{Ro.~9}
\Emph{the ſeruice} \G{ὴ λάτρέια}. & 
\XRef{Eph.~10.}
to \L{perfite, inſtuarare omnia in Chriſto}, \G{ἀνακεφαλαιώσασθαι}.
And, \Emph{Wherin he hath gratified vs}, \G{ἐχαρίτωσεν}.
\Fix{Et}{&}{obvious typo, fixed in other.}
\XRef{Eph.~6.}
\Emph{Put on the armour}, \G{πανοπλίαν}: 
and a number the like.  Sometime, when the Greek hath two ſenſes, and
the Latin but one, we adde the Greek.
\XRef{2.~Cor.~1.}
\Emph{By the exhortation wherwith we alſo are exhorted}: the Greek
ſignifieth alſo \Emph{conſolation, &c}  And
\XRef{2.~Cor.~10.}
\Emph{But hauing hope of your faith increaſing, to be, &c.} where the
Greek may alſo ſignifie, \Emph{at} or \Emph{when your faith
  increaſeth}.  Sometime for aduantage of the Catholike cauſe, when the
Greek maketh for vs more then the Latin: as
\CNote{\XRef{Act.~15.}}
\L{Seniores}, \G{πρεσβυτέρους}.
\CNote{\XRef{2.~Theſ.~2.}}
\L{Vt digni habeamini}, \G{ἱνα ἀξιωθήτε}.
\CNote{\XRef{1.~Cor.~11.}}
\L{Qui effundetur}, \G{τό ἐκχυνόμενον}, \L{Præcepta},
\G{παραδόσεις}. And
\XRef{Io.~23.}
\G{ποίμαινε}, \L{Paſce & rege}. And Sometime to shew the falſe
tranſlation of the Heretike.  As when Beza ſaith, \L{Hoc peculum in meo
ſanguine qui}, \G{τό ποτήριον ἐν τῶ ὲμῶ αἵματι τὸ ἐκχυνόμενον}.
\XRef{Luc.~22.}
&. \L{Quem opertet cælo contineri}, \G{ὃν δεῖ ὀυρανὸν δέξασθαι}, 
\XRef{Act 3}
Thus we vſe the Greek diuers waies, & eſteem of it as it is worthie,
and take al commodities therof for the better vnderſtanding of the
Latin, which being a tranſlation, can not alwaies attaine to the ful
ſenſe of the principal tongue, as we ſee in al tranſlations.

Item
\MNote{The Latin text ſometime noted in the margent.}
we adde the Latin word ſometime in the margent, when either we can not
fully expreſſe it, (as
\XRef{Act.~8.}
They tooke order for Steuens funeral, \L{Curauerunt Stephanum},
and, Al take not this word, \L{Non omnes capiunt}.)
or when the Reader might thinke, it can not be as we tranſlate; as,
\XRef{Luc~8.}
A ſtorme of wind deſcended into the lake, and \Emph{they were filled},
\L{& complebantur}: and
\XRef{Io.~5.}
when Ieſus knew that he had now a long time, \L{quia iam multum tempus
haberet}; meaning, in his infirmitie.

This
\MNote{In the beginning of Ghoſpels Matthew, Mark, &c. not
  S.~Matthew, S.~Mark, &c.}
preciſe following of our Latin text, in neither adding nor diminishing,
is the cauſe why we ſay not in the title of the Ghoſpels in the firſt
page, 
%%% o-2096
S.~Matthew, S.~Mar.\ S.~Iohn: becauſe it is ſo neither in Greek
nor Latin: though in the tops of the leaues following, where we may be
bolder, we adde, S.~Matthew, &c. to ſatisfie the Reader: Much vnlike to
the Proteſtants our Aduerſaries, which make no ſcruple to leaue out the
name of Paul in the title of the Epiſtle to the Hebrewes, though it be
in euery Greek book which they tranſlate.  And their moſt authorized
\CNote{\Cite{Bab. an.~1579.}
\Cite{1580.}
\Cite{an.~1577.}
\Cite{1562.}}
English Bibles leaue out (Catholike) in the title of S.~Iames Epiſtle
and the reſt, which were famouſly known in the primitiue Church by the
name of \L{Catholicæ Epiſtolæ}.
\Cite{Euſeb. hiſt. Eccl. li.~2. c.~22.}

Item
\MNote{Another reading in the margent.}
we giue the Reader in places of ſome importance, another reading in the
margent, ſpecially when the Greek is agreable to the ſame, as
\XRef{Iohn.~4.}
\L{tranſiet de morte ad vitam.} Other Latin copies haue, \L{tranſit},
and ſo it is in the Greek.

We
\MNote{The pointing ſometime altered.}
bind not our-ſelues to the points of any one copie, print, or edition of
the vulgar Latin, in places of no controuerſie, but follow the pointing
moſt agreable to the Greek and to the Fathers commentaries.  As
\XRef{Col.~1.~10.}
\L{Ambulantes digne Deo, per omnia placentes}. \Emph{Walking worthy of
God, in al things pleaſing.} \G{ἀξίως τοῦ κυρίου, εὶς πᾶσαν ἀρέσκειαν}.
\XRef{Eph.~1.~17.}
We point thus, \L{Deus Domini noſtri Ieſu Chriſti, pater gloria}:
as in the Greek, and S.~Chryſoſtom, & S.~Hierom both in text and
commentaries.  Which the Catholike Reader ſpecially muſt marke, leſt he
find fault, when he ſeeth our tranſlation diſagree in ſuch places from
the pointing of Latin Teſtament.

We
\MNote{The margent reading ſometime preferred before the text.}
tranſlate ſometime the word that is in the Latin margent, and not that
in the text, when by the Greek or the Fathers we ſee it is a manifeſt fault
of the writers heretofore, that miſtook one word for another.  As, \L{in
fine}, not, \L{in fide},
\XRef{1.~Pet.~3. v.~8.}
\L{præſentiam}, not, \L{præſcientiam}, 
\XRef{2.~Pet.~1. v.~16}
\XRef{Heb.~13.}
\L{latuerunt}, not, \L{placuerunt}.

Thus we haue endeauoured by al meanes to ſatisfie the indifferent Reader,
& to  help his vnderſtanding euery way, both in the text, and by
Annotations: and withal to deale moſt ſincerely before God and man, in
tranſlating & expounding the moſt ſacred
%%% 2284
text of the holy Teſtament.  Fare wel good Reader, and if we profit thee
any whit by our poore paines, let vs for God's ſake be partakers of thy
deuout prayers, & together with humble and contrit hart cal vpon our
Sauiour Chriſt to ceaſe theſe troubles and ſtormes of his deareſt
Spouſe: in the meane time comforting ourſelues with this ſaying of
S.~Auguſtin: \Emph{That Heretikes, when they receiue power corporally to
  afflict the Church, doe exerciſe her patience: but when they oppugne
  her only by their euil doctrins or opinions, then they exerciſe her
  wiſedom.}
\Cite{De ciuit. Dei li.~18. ca.~51.}


\stopPreface


\stopcomponent


%%% Local Variables:
%%% mode: TeX
%%% eval: (long-s-mode)
%%% eval: (set-input-method "TeX")
%%% fill-column: 72
%%% eval: (auto-fill-mode)
%%% coding: utf-8-unix
%%% End:
