%%%%%%%%%%%%%%%%%%%%%%%%%%%%%%%%%%%%%%%%%%%%%%%%%%%%%%%%%%%%%%%%%
%%%%
%%%% The (original) Douay Rheims Bible 
%%%%
%%%% New Testament
%%%% Front matter
%%%% Introduction
%%%%
%%%%%%%%%%%%%%%%%%%%%%%%%%%%%%%%%%%%%%%%%%%%%%%%%%%%%%%%%%%%%%%%%




\startcomponent introduction


\project douay-rheims


%%% 0000
%%% o-0000
\startPreface[
  title={\Sc{Read This}},
  marking={}
  ]

%%% If the document is singlesided, then the 'outside' is the right side
%%% and the 'inside' is the left side.
%%% MNote outer, right
%%% TNote inner, left
%%% CNote inner, left
%%% Var   inner, left
%%%
%%% SNote footnote marked with letters
%%% LNote end-note marked with musical note ♪

%%% Warning! Keep this in sync with that in the Old Testament!

This is a newly typeset edition of the \Emph{Original} Douay-Rheims
Bible of 1582 (New Testament) and 1609 (Old Testament). It is a work in
progress — you can find out more about this project at
saint-dismas.com. As of this writing the work is incomplete; in
addition, the typography and layout still leave much to be desired.

The only other available editions of the original Douay-Rheims Bible
that I am aware of are difficult-to-read copies of copies of the
original printings from between 1582 and 1610. There is also an edition
by Dr. William G. von Peters where he has transliterated the text into
modern English. While from what I have heard, he has done a wonderfull
job, this has always seemed dangerous to me — it is too easy to forget
that the English language has changed since Shakespeare's time 400 years
ago.  Finally, there is Bishop Challoner's 1749-1752 revision of the
Douay-Rheims. Most modern editions of the Douay-Rheims Bible are really
this revised version. If you just want to read the Douay-Rheims Bible,
that is probably a better choice than this work. However, the (highly
polemical) commentary here is excellent, and well worth the extra effort
required.

\vskip 0.5cm

\noindent {\tfa \bf Warning}

\vskip 0.5cm

\noindent Not only the spelling, but the meaning of many English words
has changed in the last four hundred years. If you truly want to study
the Bible and do not know Latin and Greek, you should always compare
multiple translations. The modern translation by Msgr. Ronald Knox
(available at newadvent.com), although a bit loose at points, is a
particularly fine one.

\vskip 0.5cm

\noindent {\tfa \bf Some Notes on the Text Itself}

\vskip 0.5cm

\noindent Not only the language, but the typography has changed in the
last 400 years. Here are some notes that may assist you:

\startitemize
  \item The `long s' (`ſ') is an older form of the lower case letter s. It
  was used at the beginning or in the middle of a word. Thus, `ſinfulneſs'
  for `sinfulness'.
  \item There was not the modern distinction made between the lower case
  letter forms `u' and `v'. A `v' was used at the beginning of a word,
  and a `u' elsewhere. Thus, `vſed' for `used' and `heauen' for
  `heaven'. Also, only the `V' was used for upper case letters. Finally,
  since the letter `w' was uncommon in France where these were
  type-set, they sometimes used `vv'. Thus, `lavvful' for `lawful'.
  \item There was not the modern distinction made between the lower case
  letter forms `i' and `j'. The `i' was used almost universally.
  \item In order to better fit a line of text into the available space,
  the type-setters occasionally used abbreviations such as: `oftẽ'. Here
  the `~' above the vowel indicates that either an `n' or an `m' has
  been elided. 
  \item There are six kinds of notes:
    \startitemize
      \item End Notes: These appear at the end of a chapter, and are
  marked with a musical note, ♪. The marker appears at the beginning of
  the notated passage.
      \item Foot Notes: These appear at the bottom of a page, and are
  marked with a letter. The marker appears at the beginning of
  the notated passage.
      \item Marginal Notes: These appear in the \SorP{right}{outer}
  margin.
      \item Textual Notes: These appear in the \SorP{left}{inner}
  margin, and generally give the original Latin or Greek.
      \item Citations: These appear in the \SorP{left}{inner} margin, and
  indicate a cross-reference to another part of the Bible, or to some
  commentary from one of the Church Fathers. 
      \item Variant Readings: These appear in the \SorP{left}{inner}
  margin, and indicate a variant reading for the text.
    \stopitemize
\stopitemize

\vskip 0.5cm

\noindent {\tfa \bf How You Can Help}

\vskip 0.5cm

\noindent From most to least helpful:
\startitemize
  \item Pray for me to the Lord our God.
  \item Check the citations and cross-references for accuracy, and
  reformulate them so they can be easily used by the modern reader.
  \item Proofread the Latin, Greek, or Hebrew texts against the
  original.
  \item Proofread the current text against the original.
  \item Give me money.
\stopitemize

\noindent You can contact me at destiny6ATmacDOTcom, or\\
Robert Krug\\
P.O. Box 788\\
Columbia, Ken., 42728\\
All thoughts, suggestions, comments, or complaints will be appreciated.

\noindent --- Robert Krug


\stopPreface


\stopcomponent


%%% Local Variables:
%%% mode: TeX
%%% eval: (long-s-mode)
%%% eval: (set-input-method "TeX")
%%% fill-column: 72
%%% eval: (auto-fill-mode)
%%% coding: utf-8-unix
%%% End:
