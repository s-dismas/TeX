%%%%%%%%%%%%%%%%%%%%%%%%%%%%%%%%%%%%%%%%%%%%%%%%%%%%%%%%%%%%%%%%%
%%%%
%%%% The (original) Douay Rheims Bible 
%%%%
%%%% New Testament
%%%% Front matter
%%%% Summe of the New Teſtament
%%%%
%%%%%%%%%%%%%%%%%%%%%%%%%%%%%%%%%%%%%%%%%%%%%%%%%%%%%%%%%%%%%%%%%




\startcomponent summe


\project douay-rheims


%%% 2289
%%% o-2101
\startSumme[
  title={\Sc{The Svmme of the New Testament.}},
  marking={\Sc{Svmme of the New Testament}}
  ]

That which was the ſumme of the Old Teſtament, to wit, Chriſt and his
Church, as 
\CNote{\Cite{Aug. de cat. rud. cap.~3.~4.}}
S.~Auguſtin ſaith, catechizing the ignorant: the very ſame is the ſumme
of the New Teſtament alſo. For (as the ſame S.~Auguſtin ſaith againe)
\CNote{\Cite{Super Exod. q.~73.}}
In the Old Teſtament there is the occultation of the New: and in the New
Teſtament there is the manifeſtation of the Old. And in an other place:
In the Old doth the New lye hidden, and in the New doth the Old lye
open. And therupon our Sauiour ſaid:
\CNote{\XRef{Mat.~5.}}
I am not come to breake the Law or the Prophets, but to fulfil them. For
aſſuredly I ſay vnto you, til Heauen and earth paſſe, one iot or one
title ſhal not paſſe of the Law, til al be fulfilled. In which wordes he
sheweth plainely, that the New Teſtament is nothing els but the
fulfilling of the Old.

Therfore to come to the parts: The \Sc{Ghospels} doe tel of Chriſt him
ſelfe (of whom the Old Teſtament did fortel) and that euen from his
coming into the world, vnto his going out therof againe. The \Sc{Actes
of the Apostles} doe tel of his Church beginning at Hieruſalem the
head-citie of the Iewes, and of the propagation therof to the Gentils
and their head-citie Rome. And the \Sc{Apocalypse} doth prophecie of it,
euen to the conſummation therof, which shal be in the end of the
world. The \Sc{Epistles of the Apostles} do treat partly of ſuch
queſtions as at that time were moued, partly of good life and good
order.

\Sc{The Svmme of the Foure Ghospels.}

The Ghoſpels doe tel hiſtorically the life of our Lord \Sc{Iesvs},
shewing plainely,
\CNote{\XRef{Io.~20,~31.}}
that he is Chriſt or the king of the Iewes, whom vntil then, al the time
of the Old Teſtament, they had expected: and withal, that they of their
owne meere malice and
\Fix{blindne}{blindnes}{obvious typo, fixed in other}
(the iniquitie
\Fix{beginnig}{beginning}{obvious typo, fixed in other}
of the Seniors, but at the length the multitude alſo conſenting) would
not receaue him, but euer ſought his death: which for the Redemption of
the world, he at length permitted them to compaſſe, they deſeruing
thereby moſt iuſtly to be refuſed of him, and ſo his Kingdom or Church
to be taken away from them, and giuen to the Gentils. For the gathering
of which Church after him, he chooſeth Twelue, and appointed one of them
to be the cheefe of al, with inſtructions both to them, and him
accordingly.

The ſtorie hereof is written by foure: Who in
\CNote{\XRef{Eze.~1.}}
Ezechiel and in the
\CNote{\XRef{Apoc.~4.}}
Apocalypſe are likened to foure liuing creatures, euery one according as
his booke beginneth. S.~Matthew
%%% 2290
to a Man, becauſe he beginneth with the pedegree of Chriſt as he is
man. S.~Marke to a Lion, becauſe he beginneth with the preaching of
S.~Iohn Baptiſt, as it were the roaring of a Lion in the
wildernes. S.~Luke to a Calfe, becauſe he beginneth with a prieſt of the
Old Teſtament (to wit, Zacharie the father of S.~Iohn Baptiſt) which
Prieſthood was to ſacrifice Calues to God. S.~Iohn to an Egle, becauſe
he beginneth with the Diuinitie of Chriſt, flying ſo high as more is not
poſsible.

%%% o-2102
The firſt three do report at large what Chriſt did in Galilee, after the
impriſonment of S.~Iohn Baptiſt. Wherfore S.~Iohn the Euangeliſt writing
after them al, doth omit his doings in Galilee (ſaue only one, which
they had not written of al, the wonderful bread which he told the
Capharnaites he could and would giue,
\XRef{Io.~6.})
and reporteth firſt, what he did whiles Iohn Baptiſt as yet was
preaching and baptizing: then after Iohns impriſoning, what he did in
Iurie euery yeare about Eaſter. But of his Paſsion al foure do write at
large.

Where it is to be noted, that from his baptizing (which is thought to
haue been vpon Twelfthday, what time he was beginning to be about
30.~yeare old,
\XRef{Luk. c.~3.})
vnto his Paſsion are numbred three moneths and three yeares, in which
there were alſo 4.~Eaſters.


\stopSumme


\stopcomponent


%%% Local Variables:
%%% mode: TeX
%%% eval: (long-s-mode)
%%% eval: (set-input-method "TeX")
%%% fill-column: 72
%%% eval: (auto-fill-mode)
%%% coding: utf-8-unix
%%% End:
