%%%%%%%%%%%%%%%%%%%%%%%%%%%%%%%%%%%%%%%%%%%%%%%%%%%%%%%%%%%%%%%%%
%%%%
%%%% The (original) Douay Rheims Bible 
%%%%
%%%% New Testament
%%%% Titus
%%%% Argument
%%%%
%%%%%%%%%%%%%%%%%%%%%%%%%%%%%%%%%%%%%%%%%%%%%%%%%%%%%%%%%%%%%%%%%




\startcomponent argument


\project douay-rheims


%%% 2834
%%% o-2694
\startArgument[
  title={\Sc{The Argvment of the Epistle of S.~Pavl to Titvs.}},
  marking={Argument of Titis}
  ]

That Titus was a Gentil, and not a Iew, and that he was in S.~Paules
traine, at the leaſt the 14.~yeare after his conuerſion, if not before,
we vnderſtand by the
\XRef{Epiſtle to the Galatians c.~2.}
And that he continued with him to the very end, appeareth in the
\XRef{ſecond to Timothee c.~4.}
Where he maketh mention that he ſent him from Rome into Dalmatia, when
himſelf was shortly after to be put to death.

And therfore although S.~Luke neuer name him in the Actes, as neither
himſelf, yet no doubt he comprehendeth him commonly, when he ſpeaketh
thus in the firſt perſon plural: \Emph{Forthwith we ſought to goe into
Macedonia.}
\XRef{Act.~16.}
For S.~Paul alſo ſent him to Corinth, between the writing of his
1.~&~2. to the Corinthians (which time concurreth with
\XRef{Act.~19.)}
by occaſion whereof he maketh much and honourable mention of him in the
ſaid ſecond Epiſtle
\XRef{c.~2.}
&
\XRef{c.~7.}
and againe
\CNote{\XRef{2.~Corinth.~3.}}
he ſent him with the ſame Epiſtle: both times about great matters: ſo
that no doubt he was euen them alſo a Bishop, and receiued accordingly
of the Corinthians, \Emph{with feare and trembling}.
\XRef{2.~Cor.~7. v.~15.}
But the ſame is plainer in this Epiſtle to himſelf.
\XRef{c.~1. v.~5.}
Where the Apoſtle ſaith: \Emph{for this cauſe I left thee at Crete, &c.}
By which words it is manifeſt alſo, that this Epiſtle was not written
during the ſtorie of the Actes (ſeeing that no mention is there of
S.~Paules being in the ile of Crete) but after his diſmiſsion at Rome
out of his firſt trouble, and before his ſecond or laſt trouble there,
as is euident by theſe words: \Emph{When I shal ſend to thee Artemas or
Tychicus, make haſt to come to me to Nicopolis, for there I haue
determined to winter.}
\XRef{Tit.~3.}

Therfore he inſtructeth him (and in him al Bishops) much like as he doth
Timothee, what qualities he muſt require in them that he shal make
Prieſts and Bishops, in what ſort to preach, and to teach al ſorts of
men, to commend good workes vnto them: finally, himſelf to be their
example in al goodnes.


\stopArgument


\stopcomponent


%%% Local Variables:
%%% mode: TeX
%%% eval: (long-s-mode)
%%% eval: (set-input-method "TeX")
%%% fill-column: 72
%%% eval: (auto-fill-mode)
%%% coding: utf-8-unix
%%% End:
