%%%%%%%%%%%%%%%%%%%%%%%%%%%%%%%%%%%%%%%%%%%%%%%%%%%%%%%%%%%%%%%%%
%%%%
%%%% The (original) Douay Rheims Bible 
%%%%
%%%% New Testament
%%%% Epistles
%%%% Titus
%%%% Chapter 03
%%%%
%%%%%%%%%%%%%%%%%%%%%%%%%%%%%%%%%%%%%%%%%%%%%%%%%%%%%%%%%%%%%%%%%

%%% Latin checked by KK.




\startcomponent chapter-03


\project douay-rheims


%%% 2838
%%% o-2698
\startChapter[
  title={Chapter 3}
  ]

\Summary{To teach them obedience vnto Princes, and meeknes towardes al
  men, conſidering that we alſo were as they, til God of his goodnes
  brought vs to baptiſme. 8.~To teach good workes, 9.~and to auoid vaine
  queſtions, 10.~and obſtinate Heretikes.}

Admonish them to be ſubiect to Princes and Poteſtates, to obey at a
word, to be ready to euery good worke, \V to blaſpheme no man, not to be
litigious, but modeſt: ſhewing al mildnes toward al men. \V For we alſo
were ſometime vnwiſe, incredulous, erring, ſeruing diuers deſires and
voluptuouſneſſes, liuing in malice and enuie, odible,
hating one another. \V But when the benignitie and
\TNote{\G{φιλανθρωπία}}
kindnes toward man of our Sauiour God appeared: \V
\CNote{2.~Tim.~1,~9.}
not by the workes of Iuſtice which we did, but according to his mercie
he hath ſaued vs
\SNote{As before in the Sacrament of holy Orders
\XRef{(1.~Tim.~4.}
and
\XRef{2.~Tim.~1.)}
ſo here it is plaine that Baptiſme giueth grace, & that by it as by an
inſtrumental cauſe we be ſaued.}
by the lauer of regeneration and renouation of the Holy Ghoſt, \V whom
he hath powred vpon vs aboundantly by \Sc{Iesvs} Chriſt our Sauiour: \V
that being iuſtified by his grace, we may be heires according to hope of
life euerlaſting.

\V
%%% !!! Marked in both, but present only in other
\CNote{\XRef{1.~Tim.~4.}}
It is a faithful ſaying, and of theſe things I wil haue thee auouch
earneſtly: that they which beleeue in God, be careful to excel in good
workes. Theſe things be good and profitable for men. \V But
\CNote{\XRef{2.~Tim.~2,~23.}}
fooliſh queſtions, and genealogies, and contention, and controuerſies of
the Law auoid. For they are vnprofitable and vaine.

\V A man that is
\LNote{A man that is an Heretike.}{Not
\MNote{Who is properly an Heretike, and who is not.}
euery one that erreth in religion, is an Heretike, but he only that
after the Churches determination wilfully and ſtuburnly ſtandeth in his
falſe opinion, not yealding to decree of Councel or the cheefe Paſtours
of the Church therein. \Emph{They} (ſaith S.~Auguſtin
\Cite{ep.~162.)}
\Emph{that defend
their ſentence (though falſe and peruerſe) with no ſtubburne ſtomake or
obſtinate hart, ſpecially if it be ſuch as themſelues by bold
preſumption broched not, but receiued it of their deceiued parents, and
doe ſeeke the truth warily and carefully, being ready to be reformed if
they find it, ſuch are not to be reputed among Heretikes.} And againe
\Cite{li.~18. de Ciuit. c.~51.}
\MNote{Deſcriptions or markes how to know an Heretike.}
\Emph{They that in the Church of Chriſt haue any craſed or peruerſe
opinion, if being admonished to be of a ſound and right opinion, they
reſiſt obſtinately, and wil not amend their peſtiferous opinions, but
perſiſt in defenſe of them, are thereby become Heretikes: and going
forth out of the Church, are counted for enimies that exerciſe vs.}
Againe
\Cite{li.~4. de Bapt. cont. Donat. c.~16.}
\Emph{He is an Heretike that, when the doctrine of the Catholike faith
is made plaine and manifeſt vnto him, had rather reſiſt it, and chooſe
that which himſelf held &c.} And in diuers places he declareth that
S.~Cypriã, though he held an errour, yet was no Heretike becauſe he
would not haue defended it after a general Councel had declared it to be
an errour.
\Cite{li.~2. de Bapt. c.~4.}
So Poſſidonius in the life of S.~Auguſtin
\CNote{Vit. Aug. c.~18.}
reporteth, how, after the determination of the See Apoſtolike that
Pelagius opinion was heretical, al men eſteemed Pelagius an Heretike,
and the Emperour made lawes againſt him as againſt an Heretike. Againe
S.~Auguſtin ſaith, \Emph{He is an Heretike in my opinion, that for ſome
temporal commodity, and ſpecially for his glorie and principalitie,
coineth or els followeth falſe or new opinions.}
\Cite{de vtilit. credendi cap.~1.}

Let
\MNote{The former markes agree to the Proteſtants.}
our Proteſtants behold themſelues in this glaſſe, and withal let them
marke al other properties that old Heretikes euer had, and they shal
find al definitions and markes of an Heretike to fal vpon
themſelues. And therfore they muſt not maruel if we warne al Catholike
men by the words of the Apoſtle in this place to take heed of them,
\MNote{Their bookes, ſeruice, and preaching muſt be auoided.}
and to shun their preachings, bookes, couenticles and companies. Neither
need the people be curious to know what they ſay, much leſſe to confute
them: but they muſt truſt Gods Church, which doth refute and condemne
them. And it is enough for them to know that they be condemned, as
S.~Auguſtin noteth in the
\Cite{later end of his booke de hereſibus.}
And S.~Cyprian ſaith notably to Antonianus demanding curiouſly what
hereſies Nouatianus did teach
\CNote{Ep.~52. nu.~7.}
\Emph{No matter}, ſaith he, \Emph{what hereſie he hath or preacheth, when
he ſearcheth without}: that is to ſay, out of the Church.}
an heretike after the firſt and ſecond
\SNote{Theſe admonitions or correptions muſt be giuen to ſuch as erre,
by our Spiritual Gouernours and Paſtours: to whom if they yeald not,
Chriſtian men muſt auoid them.}
admonition auoid: \V knowing that he that is ſuch an one, is
\LNote{Subuerted.}{Heretikes
\MNote{The Church ſeeketh the ammendement of the moſt obſtinate
Heretikes.}
be often incorrigible yet the Church of God ceaſeth not by al meanes
poſſible to reuoke them. Therfore S.~Auguſtin ſaith
\Cite{ep.~162.}
\Emph{The Heretike himſelf though ſwelling with odious & deteſtable
pride, and mad with the frowardnes of wicked contention, as we admonish
that he be auoided leſt he deceiue the weaklings and litle ones, ſo we
refuſe not by al meanes poſsible to ſeeke his amendement and
reformation.}}
ſubuerted, and ſinneth, being condemned
\LNote{By his owne iudgement.}{Other
\MNote{Heretikes cut themſelues from the Church.}
grieuous offenders be ſeparated by excommunication from communion of
Saints and the fellowship of God's Church, by the ſentence of their
Superiours in the ſame Church: but Heretikes more miſerable and
infortunate then they runne out of the Church of their owne accord, and
ſo giue ſentence againſt their owne ſoules to damnation.}
by his owne iudgement.

\V When I ſhal ſend to thee Artemas or Tychicus, haſten to come vnto me
to Nicopolis. For there I haue determined to winter. \V Set forward
Zenas the lawyer and Appollos carefully, that nothing be wanting to
them. \V And let our men alſo learne
\TNote{\G{προίστασθαι} \L{præesse}}
to excel in good workes to neceſſarie vſes: that they be not
vnfruitful. \V Al that are with me, ſalute
%%% o-2699
thee: ſalute them that loue vs in the faith. The grace of God be with
you al. Amen.
 

\stopChapter


\stopcomponent


%%% Local Variables:
%%% mode: TeX
%%% eval: (long-s-mode)
%%% eval: (set-input-method "TeX")
%%% fill-column: 72
%%% eval: (auto-fill-mode)
%%% coding: utf-8-unix
%%% End:

