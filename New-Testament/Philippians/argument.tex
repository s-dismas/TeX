%%%%%%%%%%%%%%%%%%%%%%%%%%%%%%%%%%%%%%%%%%%%%%%%%%%%%%%%%%%%%%%%%
%%%%
%%%% The (original) Douay Rheims Bible 
%%%%
%%%% New Testament
%%%% Epistles
%%%% Philippians
%%%% Argument
%%%%
%%%%%%%%%%%%%%%%%%%%%%%%%%%%%%%%%%%%%%%%%%%%%%%%%%%%%%%%%%%%%%%%%




\startcomponent argument


\project douay-rheims


%%% 2767
%%% o-2625
\startArgument[
  title={\Sc{The Argvment of the Epistle of S.~Pavl to the Philippians.}},
  marking={Argument of Philippians}
  ]

How S.~Paul was called by a viſion into Macedonia, we read
\XRef{Act.~16.}
and how he came to Philippi being the firſt citie therof, and of his
preaching, miracles, and ſuffering there. And agains
\XRef{Act.~19.}
\Emph{Paul purpoſed in the Spirit, when he had paſſed through Macedonia
and Achaia, to goe to Hieruſalem, ſaying: After I haue been there, I
muſt ſee Rome alſo.} Which purpoſe he executed
\XRef{Act.~20.}
taking his leaue at Epheſus. And being afterward come into Achaia,
\Emph{He had counſel to returne through Macedonia}, and ſo at length
from Philippi he began his nauigation toward Hieruſalem, and from
Hieruſalem being caried priſoner to Rome
\XRef{(Act.~28.)}
he wrote from thence this Epiſtle to the Philippians: or rather in his
ſecond apprehenſion; about 10.~yeares after the firſt.

In it he confirmeth them (as he did
\CNote{\XRef{Eph.~3.}}
the Epheſians alſo about the ſame time) againſt the tentation that they
might haue in hearing that he were executed. Therfore he firſt ſaith:
\CNote{\XRef{Phil.~1. v.~12.}}
\Emph{And I wil haue you know, Brethren, that the things about me, are
come to the more furtherance of the Ghoſpel: ſo that my bands were made
manifeſt in Chriſt in al the Court &c.} Secondly he ſignifieth that his
deſire is,
\CNote{\XRef{23.}}
\Emph{to be diſſolued and to be with Chriſt}. But yet (leſt they should
be diſcomforted) that he hopeth
\CNote{\XRef{26.}}
\Emph{to come againe} to them. Wherof, notwithſtanding that he hath yet
no certaintie, he ſignifieth in ſaying:
\CNote{\XRef{Phil.~2. v.~23.}}
\Emph{I hope to ſend Timothee vnto you immediately as I ſhal ſee the
things that concerne me.} Thirdly therefore he prepareth them againſt
the worſt, ſaying:
%%% !!! both have 24,17 ???
\CNote{\XRef{17.}}
\Emph{I hope to come againe to you: but and if I be immolated, vpon the
ſacrifice and ſeruice of your faith, I reioyce and congratulate with you
al, and the ſelf-ſame thing doe you alſo reioyce and congratulate with
me.}

Moreouer he partly warneth them (as
\CNote{\XRef{Phil.~3.}}
he had done before) of thoſe Iudaical Falſe-Apoſtles who preached
circumciſion and Moyſes law to the Chriſtian Gentils: partly he
exhorteth them to ſuffer perſecution, to liue wel, and ſpecially to
humble themſelues one to another, rather then by any pride to breake the
peace & vnitie of the Church.


\stopArgument


\stopcomponent


%%% Local Variables:
%%% mode: TeX
%%% eval: (long-s-mode)
%%% eval: (set-input-method "TeX")
%%% fill-column: 72
%%% eval: (auto-fill-mode)
%%% coding: utf-8-unix
%%% End:
