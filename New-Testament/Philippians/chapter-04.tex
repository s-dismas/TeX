%%%%%%%%%%%%%%%%%%%%%%%%%%%%%%%%%%%%%%%%%%%%%%%%%%%%%%%%%%%%%%%%%
%%%%
%%%% The (original) Douay Rheims Bible 
%%%%
%%%% New Testament
%%%% Epistles
%%%% Philippians
%%%% Chapter 04
%%%%
%%%%%%%%%%%%%%%%%%%%%%%%%%%%%%%%%%%%%%%%%%%%%%%%%%%%%%%%%%%%%%%%%




\startcomponent chapter-04


\project douay-rheims


%%% 2774
%%% o-2632
\startChapter[
  title={Chapter 4}
  ]

\Summary{He exhorteth them to perſeuerance, and certaine by name to
  vnitie, 5.~to modeſtie, 6.~to peace without ſolicitude or careful
  anxietie, 8.~to al that good is, 9.~to ſuch things as they ſee in
  himſelf. 10.~That he reioyced in their contribution, not for his owne
  need, but for their merit.}

%%% o-2633
Therfore, my deareſt Brethren and moſt deſired, my
\LNote{My ioy.}{He
\MNote{The reward of Preachers.}
calleth them his ioy and crowne, for that he expected the crowne of
euerlaſting life as a reward of his labours towards them. Wherby we may
learne alſo, that beſides the eſſential glorie which shal be in the
viſion and fruition of God, there is other manifold felicitie incident
in reſpect of creatures.}
ioy and my crowne: ſo ſtand in our Lord, my deareſt. \V
\Var{Euchodia}{Eudoia}
I deſire and Syntyche I beſeech to be of one mind in our Lord. \V Yea
and I beſeech thee my
\LNote{Sincere companion.}{The
\MNote{Suſpitious tranſlation.}
English Bibles with one conſent interpret the Greek
words, \Emph{faithful yoke-fellow}, perhaps to ſignifie (as ſome would
haue it) that the Apoſtle here ſpeaketh to his wife: but they muſt
vnderſtãd that their Maiſters Caluin & Beza miſlike that expoſition, and
%%% !!! Bad Cites
\CNote{\Cite{S.~Chryſ.}
\Cite{Theodore.}
\Cite{Occum.}
\Cite{Theophyl.}}
al the Greek Fathers almoſt much more reiect it: and it is againſt
S.~Paules owne words ſpeaking to the vnmarried, That it is good for them
to remaine ſo, euen as himſelf did.
\XRef{1.~Cor.~7,~8.}
\MNote{S.~Paul had no wife.}
Whereby it is euident he had no wife, and therfore meaneth here ſome
other his coadiutour & fellow-labourer in the Ghoſpel.}
ſincere Companion, help thoſe women that haue laboured with me in the
Ghoſpel with
\SNote{This Clement was afterward the 4.~Pope of Rome from S.~Peter, as
S.~Hierom writeth, according to the cõmon ſupputation.}
Clement, and the reſt my Coadiutours, whoſe names are in the booke of
life. \V Reioyce in our Lord alwaies: againe I ſay reioyce. \V Let your
modeſtie be knowen to al men. Our Lord is nigh. \V Be nothing careful:
but
\TNote{\G{ἐν παντὶ τῇ προσευχῇ}}
in euery thing by praier & ſupplication with thankes-giuing let your
petitions be knowen with God. \V And the peace of God which paſſeth al
vnderſtanding, keep your harts and intelligences in Chriſt \Sc{Iesvs}.

\V For the reſt, Brethren, what things ſoeuer be true, whatſoeuer
honeſt, whatſoeuer iuſt, whatſoeuer holy, whatſoeuer
\Fix{aimable,}{amiable,}{obvious typo, fixed in other}
whatſoeuer
of good fame, if there be any vertue, if any praiſe of diſcipline, theſe
things thinke vpon. \V Which you haue both learned, and receiued, and
heard, & ſeen in me; theſe things doe ye, and the God of peace ſhal be
with you. \V And I reioyced in our Lord exceedingly, that once at the
length you haue
\SNote{This reflourishing is the reuiuing of their old liberalitie,
which for a time had been ſlacke & dead.
\Cite{S.~Chryſ.}}
reflouriſhed to care for me, as you did alſo care: but you were
occupied. \V I ſpeake not as it were for penurie. For I haue learned, to
be content with the things that I haue. \V I know both to be brought
low, I know alſo to abound: (euery-where, and in al things I am
inſtructed) both to be ful, & to be hungrie, both to abound, and to
ſuffer penurie. \V I can al things in him that ſtrengthneth me. \V
Neuertheleſſe you haue done wel, communicating to my tribulation.

%%% 2775
\V And you alſo know, ô Philippians, that in the beginning of the
Ghoſpel, when I departed from Macedonia, no Church communicated vnto me
in the account of guift and
\SNote{He counteth it not mere almes or a free guift that the people
beſtoweth on their Paſtours or Preachers, but a certaine mutual traffike
as it were, and enterchange: the one giuing ſpiritual, the other
rẽdering tẽporal things for the ſame.}
receit, but you only: \V For vnto Theſſalonica alſo, once and twiſe you
ſent to my vſe. \V Not that I ſeeke the guift, but I ſeeke the fruit
abounding in your account. \V But I haue al things, and abound: I was
filled after I receiued of Epaphroditus the things that you ſent, an
odour of ſweetnes, an
\LNote{Acceptable.}{How
\MNote{Almes giuen religiouſly.}
acceptable almes are before God, we ſee here: namely when it is giuen
for religion to deuout perſons for a recõpenſe of ſpiritual
benefits. For ſo it putteth on the condition of an oblation or
Sacrifice offered to God, and is moſt acceptable and ſweet in his
ſight.}
acceptable Hoſt, pleaſing God. \V And my God ſupply
%%% o-2634
al your
\TNote{\G{Χριστῷ}}
lack according to his riches in glorie, in Chriſt \Sc{Iesvs}. \V And to
God & our Father be glorie world without end. Amen.

\V Salute ye euery Saint in Chriſt \Sc{Iesvs}. \V The Brethren that are
with me, ſalute you. Al the Saints ſalute you: but eſpecially they that
are of Cæſars houſe. \V The grace of our Lord \Sc{Iesvs} Chriſt be with
your ſpirit. Amen.


\stopChapter


\stopcomponent


%%% Local Variables:
%%% mode: TeX
%%% eval: (long-s-mode)
%%% eval: (set-input-method "TeX")
%%% fill-column: 72
%%% eval: (auto-fill-mode)
%%% coding: utf-8-unix
%%% End:

