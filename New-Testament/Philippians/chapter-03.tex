%%%%%%%%%%%%%%%%%%%%%%%%%%%%%%%%%%%%%%%%%%%%%%%%%%%%%%%%%%%%%%%%%
%%%%
%%%% The (original) Douay Rheims Bible 
%%%%
%%%% New Testament
%%%% Epistles
%%%% Philippians
%%%% Chapter 03
%%%%
%%%%%%%%%%%%%%%%%%%%%%%%%%%%%%%%%%%%%%%%%%%%%%%%%%%%%%%%%%%%%%%%%

%%% Latin checked by KK.




\startcomponent chapter-03


\project douay-rheims


%%% 2772
%%% o-2630
\startChapter[
  title={Chapter 3}
  ]

\Summary{He warneth them of the Falſe-Apoſtles, 4.~shewing that himſelf
  had much more to brag of in Iudaiſme then they: but that he maketh
  price of nothing but only of Chriſt, and of Chriſtian iuſtice, and of
  ſuffering with him (12.~wherin yet he acknowledgeth his imperfection)
  17.~exhorting them to beare Chriſtes Croſſe with him, and not to
  imitate thoſe belly-Gods.}

From hence-forth, my Brethren, reioyce in our Lord. To write the ſame
things vnto you, to me ſurely it is not tedious, and to you it is
neceſſarie. \V See the dogs, ſee the euil workers, ſee the
\TNote{\G{κατατομή}}
conciſion. \V For we are the
%%% !!! This SNote also applies to the same place as the other TNote
%%% above. 
\SNote{By alluſion of words, he calleth the carnal Chriſtiã Iewes that
yet boaſted in the circumciſion of the flesh, \Emph{conciſion}; &
himſelf & the reſt that circumcided their hart and ſenſes ſpiritually,
the true \Emph{circumciſion}.
\Cite{S.~Chryſ.}
\Cite{Theophylact.}}
\TNote{\G{περιτομή}}
circumciſion, which in ſpirit ſerue God: and we glorie in Chriſt
\Sc{Iesvs}, and not hauing confidence in the flesh, \V albeit I alſo
haue confidence in the flesh, if any other man ſeeme to haue confidence
in the flesh, I more, \V circumciſed the eight day of the ſtocke of
Iſrael, of the tribe of Beniamin,
\CNote{\XRef{2.~Cor.~11,~22.}}
an Hebrew of Hebrewes:
\CNote{\XRef{Act.~23,~6.}}
according to the Law, a Phariſee: \V according to emulation, perſecuting
the Church of God: according to the iuſtice that is in the Law,
conuerſing without blame. \V But
%%% o-2631
the things that were gaines to me, thoſe haue I eſteemed for Chriſt,
detriments. \V Yea but I eſteeme al things to be detriment for the
paſſing knowledge of \Sc{Iesvs} Chriſt my Lord: for whom I haue made al
things as detriment, and doe eſteeme them as dung, that I may gaine
Chriſt: \V and may be found in him not hauing
\LNote{My iuſtice.}{Diuers
\CNote{Magdeburg. cent.~1. li.~2. c.~4. pag.~222.}
\MNote{The obiection againſt inherent iuſtice, anſwered.}
Lutherans in their tranſlations doe shamefully mangle this ſentence by
tranſpoſing the words, and falſe pointing of the parts therof, to make it
haue this ſenſe, That the Apoſtle would haue no iuſtice of his owne, but
only that iuſtice which is in Chriſt: Which is a falſe and heretical
ſenſe of the words, and not meant by S.~Paul: who calleth that a man's
owne iuſtice, which he chalengeth by the workes of the Law or nature
without the grace of Chriſt: and that God's iuſtice (as S.~Auguſtine
expoundeth this place) not which is in God, or by which God is iuſt, but
that which is in man from God and by his guift.
\Cite{li.~3. cont. 2.~ep. Pelag. c.~7.}
\Cite{de Sp. & lit. c.~9.}}
my iuſtice which is of the Law, but that which is of the faith of
Chriſt, which is of God, iuſtice in faith: \V to know him, and the
vertue of his reſurrection, and the ſocietie of his paſſions,
configured to his death, \V
\SNote{If S.~Paul ceaſed not to labour ſtil, as though he were not ſure
to come to the marke without continual endeauour; what ſecuritie may we
poore ſinners haue of Heretikes perſuaſions & promiſes of ſecuritie and
ſaluation by only faith?}
if by any meanes I may come to the reſurrectiõ which is from the
dead. \V
\LNote{Not that now.}{No
\MNote{Double perfection: here, and in the life to come.}
man in this life can attaine the abſolute perfectnes either of iuſtice
or of that knowledge which shal be in heauen: but yet there is alſo
another perfectnes, ſuch as according to this ſtate a man may reach
vnto, which in reſpect of the perfection in glorie, is ſmal, but in
reſpect of other leſſe degrees of man's iuſtice and knowledge in this
life, may be called perfectnes. And in this ſenſe the Apoſtle in the
next ſentence calleth himſelf and others perfect, though in reſpect of
the abſolute perfectnes in Heauen, he ſaith here, he is not yet perfect
nor hath yet attained therunto.}
Not that now I haue receiued, or now am perfect: but I purſue, if I may
comprehend, wherin I am alſo comprehẽded of Chriſt \Sc{Iesvs}. \V
Brethren, I doe not account that I haue comprehended. Yet one thing:
forgetting the things that are behind,  but ſtretching forth my ſelf to
thoſe that are before, \V I purſue to the marke,
\TNote{\L{ad brauium}}
to the prize of the ſupernal vocation of God in Chriſt \Sc{Iesvs}. \V
Let vs therfore as many as are perfect, be thus minded: and if you be
any
\LNote{Otherwiſe minded.}{When
\MNote{The heretikes foolish defenſe of their diſſenſions and diuiſions
among themſelues.}
Catholike men now a-daies charge Heretikes with their horrible
diuiſions, diſſenſions, combates, contentions, and diuerſities among
themſelues, as the Catholikes of al other Ages did chalenge their
Aduerſaries moſt truely and iuſtly for the ſame, (both becauſe where the
Spirit of God is not, nor any order or obedience to Superiours, there
can be no peace nor vnitie, and ſpecially for that it is, as S.~Auguſtin
ſaith
\Cite{(li. de agone Chriſt. c.~29.)}
the iuſt iudgement of Gods, that they which ſeek nothing els but to
diuide the Church of Chriſt, should themſelues be miſerably diuided
among themſelues) therfore (I ſay) when men charge the Proteſtants with
theſe things, they fly for their defence to this, that the old Fathers
were not al of one iudgement in euery point in religion: that S.~Cyprian
ſtood againſt others, that S.~Aug. and S.~Hier. wrote earneſtly in a
certaine matter one againſt another, that our Dominicans and
Franciſcans, our Thomiſts & Scotiſts be not al of one opinion in diuers
matters, and therfore diuiſions and contentions should not be ſo
preiudicial to the Zuinglians and Lutherans, as men make it.
\MNote{The difference between the diſagreeing of ancient Fathers or
other Catholikes, and the Heretikes diſſenſiõs among themſelues.}
Thus they defend themſelues: but ridiculouſly and againſt the rule of
S.~Paul here, acknowledging that in this imperfection of mens ſcience in
this life, euery one can not be free from al errour, or thinke the ſame
that another thinketh: wherupon may riſe differences of vnderſtanding,
opinion, and iudgement, in certaine hard matters which God hath not
reuealed or the Church determined, and therfore that ſuch diuerſitie is
tolerable and agreable to our humane condition and the ſtate of the way
that we be in: alwaies prouided, that the controuerſie be ſuch and in
ſuch things, as be not againſt the ſet knowen rule of faith, as he here
ſpeaketh, & ſuch as breake not mutual ſocietie, fellowship, & communion
in praier, ſeruice, Sacraments, and other offices of life and
religion. For ſuch diuiſions and differences come neuer but of Schiſme
or Hereſie; and ſuch are among the Heretikes, not only in reſpect of vs
Catholikes, but among themſelues:
\MNote{The ſpiteful writings of Heretikes, one Sect againſt another.}
as they know that be acquainted with the writings of Luther againſt
Zwinglius, or Weſtphalus againſt Caluin, or the Puritans againſt the
Proteſtants, not only charging one another with Hereſie, Idolatrie,
Superſtition, and atheiſme, but alſo cõdemning each others ceremonies or
manner of adminiſtratiõ, til it come to excõmunication, and banishment,
yea ſometimes burning one of another. Thus did not S.~Cyprian,
S.~Auguſtin, S.~Hierom, the Dominicans, Franciſcans, Thomiſts, Scotiſts,
who al agree in one rule of faith, al of one communion, al moſt deare
one to another in the ſame, al (thankes be to God) come to one holy
Maſſe & receiue the ſame Sacraments, and obey one Head throughout al the
world. S.~Auguſtin
\Cite{li.~2. de Bapt. c.~5.}
shal make vp this matter with this notable ſentence:
\MNote{A notable place of S.~Auguſtin.}
\Emph{We are men} (ſaith he) \Emph{and therfore to thinke ſomewhat
otherwiſe then the thing is, is an humane tentation: but by louing our
owne ſentence too much, or by enuying our betters, to proceed vnto the
ſacriledge of diuiding the mutual ſocietie, and of making ſchiſme, or
hereſie, is diuelish preſumption: in nothing to haue other opinion then
the truth is, that is Angelical perfection.} And a litle after: \Emph{If
you be any otherwiſe minded; this God wil reueale: but to them only}
(ſaith he) \Emph{that walke in the way of peace, and that ſtray aſide
into no diuiſion or ſeparation.} Which ſaying would God al our deare
Countrie-men would marke, and come into the Church, where only, God
reuealeth truth.}
otherwiſe minded, this alſo God
\Var{hath reuealed}{wil reueale}
to you. \V Neuertheleſſe wherunto we are come, that we be of the ſame
mind, let vs continue in the ſame rule.

\V Be followers
\SNote{It is a goodly thing when the Paſtour may ſo ſay to his
flocke. Neither is it any derogation to Chriſt, that the people should
imitate their Apoſtles life & doctrine, & other holy men,
%%% !!! Cites?
S.~Auguſtin, S.~Benedict, S.~Dominike, S.~Francis.}
of me, Brethren, & obſerue them that walke ſo as you haue our forme. \V
For
\CNote{\XRef{Ro.~16,~17.}}
many walke whom often I told you of
%%% 2773
(and now weeping alſo I tel you) the enemies of the croſſe of Chriſt: \V
Whoſe end is deſtruction: whoſe God, is the belly: and their glorie in
their confuſion, which mind worldly things. \V But our conuerſation is
in Heauen: whence alſo we expect the Sauiour, our Lord \Sc{Iesvs}
Chriſt, \V who wil reforme the body of our humilitie, configured to the
body of his glorie, according to the operation whereby alſo he is
able to ſubdue al things to himſelf.


\stopChapter


\stopcomponent


%%% Local Variables:
%%% mode: TeX
%%% eval: (long-s-mode)
%%% eval: (set-input-method "TeX")
%%% fill-column: 72
%%% eval: (auto-fill-mode)
%%% coding: utf-8-unix
%%% End:

