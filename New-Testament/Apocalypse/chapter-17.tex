%%%%%%%%%%%%%%%%%%%%%%%%%%%%%%%%%%%%%%%%%%%%%%%%%%%%%%%%%%%%%%%%%
%%%%
%%%% The (original) Douay Rheims Bible 
%%%%
%%%% New Testament
%%%% Apocalypse
%%%% Chapter 17
%%%%
%%%%%%%%%%%%%%%%%%%%%%%%%%%%%%%%%%%%%%%%%%%%%%%%%%%%%%%%%%%%%%%%%




\startcomponent chapter-17


\project douay-rheims


%%% 2967
%%% o-2829
\startChapter[
  title={Chapter 17}
  ]

\Summary{The harlot Babylon clothed with diuers ornaments, 6.~and
  drunken of the bloud of Martyrs, ſitteth vpon a beaſt that hath ſeuen
  heads and ten hornes: 7.~al which things the Angel expoundeth.}

And there came one of the ſeuen Angels which had the ſeuen vials, &
ſpake with me, ſaying: Come, I wil ſhew thee
\SNote{The final damnation of the whole cõpanie of the reprobate, called
here the great whore.}
the damnation of the great harlot, which ſitteth vpon
\SNote{Theſe many waters are many peoples.
\XRef{v.~15.}}
many waters, \V with whom the Kings of the earth haue fornicated, & they
which inhabit the earth haue been drunke of the
wine of her whoredom. \V And he tooke me away in ſpirit into the
deſert. And I ſaw a woman ſitting vpon a ſcarlet coloured beaſt, ful of
names of blaſphemie, hauing ſeuen heads, and ten hornes. \V And the
woman was clothed round about with purple and ſcarlet, and gilted with
gold, and pretious ſtone, and pearles, hauing a golden cup in her hand,
ful of the abomination & filthines of her fornication. \V And in her
forehead a name written,
%%% !!! LNote out of order in both
\LNote{Myſterie.}{S.~Paul
\MNote{Myſterie.}
calleth this ſecret and cloſſe working of abomination, the myſterie of
iniquitie
\XRef{2.~Theſſal.~2.}
and it is called a litle after in this chapter
\XRef{verſ.~7.}
\Emph{the Sacrament} (or myſterie) of the woman, and it is alſo the
marke of reprobation and damnation.}
\Emph{Myſterie}:
\LNote{Babylon.}{In
\MNote{The Proteſtãts here wil needs haue Babylõ to be Rome, but not in
S.~Peters epiſtle.}
the
\XRef{end of S.~Peters firſt Epiſtle,}
where the Apoſtle dateth it at Babylon which the ancient Writers (as we
there noted) affirme to be meant of Rome: the Proteſtãts wil not in any
wiſe haue it ſo, becauſe they would not be driuen to confeſſe that Peter
euer was at Rome. But here, for that they thinke it maketh for their
opinion, that the Pope is Antichriſt, and Rome the ſeat & citie of
Antichriſt, they wil needs haue Rome to be this Babylon, this great
whore, and this purple
\Fix{harlat.}{harlot.}{likely typo, fixed in other}
For ſuch fellowes, in the expoſition of holy Scripture, be led only by
their preindicate opinions and hereſies, to which they draw al things
without al indifferencie and ſinceritie.

But
\MNote{By Babylon (according to al the Fathers) is ſignified, partly the
whole ſocietie of the wicked, partly the citie of Rome, only in reſpect
of the terrene and heathenish ſtate of them that perſecuted the Church.}
S.~Auguſtin, Aretas, and other Writers, moſt commonly expound it,
neither of Babylon it-ſelf a citie of Chaldæa or Ægypt, nor of Rome, or
any one citie, which may be ſo called ſpiritually, as Hieruſalem before
\XRef{chap.~11.}
is named ſpiritual Sodom and Ægypt; but of the general ſocietie of the
impious, and of thoſe that preferre the terrene Kingdom & commodity of
the world, before God & eternal felicitie. The Authour of the
\Cite{Commentaries vpon the Apocalypſe ſet forth in S.~Ambroſe name,}
writeth thus: \Emph{This great whore ſometime ſignifieth Rome,
ſpecially which at that time when the Apoſtle wrote this, did perſecute
the Church of God. But otherwiſe it ſignifieth the whole citie of the
Diuel, that is, the vniuerſal corps of the reprobate.} Tertullian alſo
taketh it for Rome, thus.
\CNote{Li. aduer. Iudæos.}
\Emph{Babylon} (ſaith he) \Emph{in S.~Iohn is a figure of the citie of
Rome, being ſo great, ſo proud of the Empire, and the deſtroier of the
Saints.} Which is plainely ſpoken of that citie, when it was heathen,
the head of the terrene dominion of the world, the perſecutour of the
Apoſtles & their Succeſſours, the ſeat of Nero, Domitian, and the like,
Chriſts ſpecial enemies, the ſinke of idolatrie, ſinne, and falſe
worship of the Pagan Gods. Then was it Babylon, when S.~Iohn wrot this,
and they was Nero and the reſt figures of Antichriſt, & that citie the
reſemblance of the principal place (whereſoeuer it be) that Antichriſt
shal reigne in, about the later end of the world.

Now to apply that to the Romane Church and the Apoſtolike See, either
now or then, which was ſpoken only of the terrene ſtate of that citie, as
it was the ſeate of the Emperour, and not of Peter, when it did flea
aboue 30.~Popes, Chriſts Vicars, one after another, & endeauoured to
deſtroy the whole Church: that is moſt blaſphemous and foolish.

The
\MNote{The Church of Rome is neuer called Babylõ.}
Church in Rome was one thing, & Babylon in Rome another thing. Peter
ſate in Rome, and Nero ſate in Rome. But Peter, as in the Church of
Rome: Nero, as in the Babylon of Rome. Which diſtinction the Heretikes
might haue learned by S.~Peter himſelf
\XRef{ep.~1. chap.~5.}
writing thus: \Emph{The Church ſaluteth you, that is in Babylon,
coelect.} So that the Church & the very choſen Church was in Rome, when
Rome was Babylon. Whereby it is plaine, that whether Babylon or the
great whore doe here ſignifie Rome or no, yet it can not ſignifie the
Church of Rome: which is now, and euer was, differing from the terrene
Empire of the ſame. And if, as in the beginning of the Church, Nero and
the reſt of the perſecuting Emperours (which were figures of Antichriſt)
did principally ſit in Rome, ſo alſo the great Antichriſt shal haue his
ſeat there, as it may wel be (though others thinke that Hieruſalem
rather shal be his principal citie:) yet euen then that neither the
Church of Rome, nor the Pope of Rome be Antichriſt, but shal be
perſecuted by Antichriſt, and driuen out of Rome, if it be
poſſible. For, to Chriſts Vicar and the Romane Church he wil beare as
much good wil as the Proteſtants now doe, and he shal haue more power to
perſecute him and the Church, then they haue.

S.~Hierom
\Cite{ep.~17. c.~7.}
to Marcella, to draw her out of the citie of Rome to the holy land,
warning her of the manifold allurements to ſinne and il life, that be in
ſo great and populous a citie, alludeth at length to theſe words of the
Apocalypſe, & maketh it to be Babylon, & the purple whore. But ſtraight
way, leſt ſome naughtie perſon might thinke he meant that of the Church of
Rome, which he ſpake of the ſocietie of the wicked only, he
addeth: \Emph{There is there indeed the holy Church, there are the
triumphant monumẽts of the Apoſtles & Martyrs, there is the true
confeſsion of Chriſt, there is the faith praiſed
\CNote{\XRef{Ro.~1.}}
of the Apoſtle, & Gentilitie troden vnderfoot, the name of Chriſtian
daily aduancing it-ſelf on high.} Whereby you ſee that whatſoeuer may be
ſpoken or interpreted of Rome, out of this word \Emph{Babylon}, it is
not meant of the Church of Rome, but of the terrene ſtate, in ſo much
that the ſaid holy Doctour
\Cite{li.~2. aduerſ. Iouinian c.~19.}
ſignifieth that the holines of the Church there, hath wiped away the
blaſphemie written in the forehead of her former iniquitie. But of the
difference of the old ſtate and dominion of the Heathen there, for which
it is reſembled to Antichriſt, and the Prieſtly ſtate which now it hath,
read a notable place in S.~Leo
\Cite{ſerm.~1. in natali Petri & Pauli.}}
Babylon the great, mother of the fornications and the abominations of
the earth. \V And I ſaw the woman
\LNote{Drunken of the bloud.}{It
\MNote{This woman ſignifieth al perſecutours of Saints.}
is plaine that this woman ſignifieth the whole corps of al the
perſecutours that haue & shal shead ſo much bloud of the iuſt: of the
Prophets, Apoſtles, and other Martyrs from the beginning of the world to
the end. The Proteſtants foolishly expound it of Rome, for that there
they put Heretikes to death, and allow of their punishment in other
countries:
\MNote{Putting heretikes to death, is not to shead the bloud of Saints.}
but their bloud is not called the bloud of Saints, no more thẽ the bloud
of eues, man-killers, and other malefactours: for the sheading of which by
order of iuſtice, no Common-wealth shal anſwer.}
drunken of the bloud of the Saints, and of the bloud of the Martyrs of
\Sc{Iesvs}. And I marueled when I had ſeen her, with great
admiration. \V And the Angel ſaid to me: Why doeſt thou maruel? I wil
tel thee the myſterie of the woman, and of the beaſt that carieth her,
which hath the ſeuen heads and the ten hornes.

\V The beaſt which thou ſaweſt,
\SNote{It ſignifieth the short reigne of Antichriſt, who is the cheefe
horne or head of the beaſt.}
was, and is not, and ſhal come vp out of the bottomles depth, and goe
into deſtruction: and the inhabitants on the earth (whoſe names are not
written in the booke of life from the making of the world) ſhal maruel,
ſeeing the beaſt that was, and is
%%% o-2830
not. \V And here is vnderſtanding, that hath wiſedom. The ſeuen heads,
are 
\LNote{Seuen hilles.}{The
\MNote{The Proteſtãts madnes in expounding the 7.~hilles of Rome: the
Angel himſelf expounding thẽ otherwiſe.}
Angel himſelf here expoundeth the 7.~hilles to be al one with the
7.~heads and the 7.~Kings: and yet the Heretikes blinded exceedingly
with malice againſt the Church of Rome, are ſo mad to take them for the
ſeuen hilles literally, vpon which in old time Rome did ſtand: that ſo
they might make the vnlearned beleeue that Rome is the ſeat of
Antichriſt. But if they had any conſideration, they might marke that the
Prophets viſions here are moſt of them by Seuens, whether he talke of
heads, hornes, candleſtickes, Churches, Kings, hilles, or other things:
and that he alluded not to the hilles, becauſe they were iuſt ſeuen, but
that \Emph{Seuen} is a myſtical number, as ſometime \Emph{Ten} is,
ſignifying vniuerſally al of that ſort whereof he ſpeaketh: as, that the
ſeuen heads, hilles, or Kingdoms (which are here al one) should be al
the Kingdoms of the world that perſecute the Chriſtians: being heads and
mountaines for their height in dignitie aboue others. And ſome take it,
that there were ſeuen ſpecial Empires, Kingdoms, or States that were or
shal be the greateſt perſecutours of Gods people: as of Ægypt, Chanaan,
Babylon, the Perſians, and Greeks, which be fiue: ſixtly of the Romane
Empire, which once perſecuted moſt of al other, and which (as the
Apoſtle here ſaith) \Emph{yet is}, or ſtandeth. But the ſeuenth, then
when S.~Iohn wrote this, was not come, neither is yet come in our daies:
which is Antichriſts ſtate, which shal not come ſo long as the Empire of
Rome ſtandeth, as S.~Paul did prophecie.
\XRef{2.~Theſſal.~2.}}
ſeuen hilles, vpon which the woman ſitteth, and they are ſeuen Kings. \V
Fiue are fallen, one is, and another is not yet come: and when he ſhal
come, he muſt tarie a ſhort time. \V And the beaſt which was, and is
not:
\LNote{The ſame is the eight.}{The
\MNote{What is the eight beaſt.}
beaſt it-ſelf being the cõgregation of al theſe wicked perſecutours,
though it conſiſt of the foreſaid ſeuen, yet for that the malice of al
is cõplete in it, may be called the eight. Or, Antichriſt himſelf,
though he be one of the ſeuẽ, yet for his extraordinary wickednes shal
be counted the odde perſecutour or the accõplishment of al other, &
therfore is named the eight. Some take this beaſt called the eight, to
be the Diuel.}
the ſame alſo is the eight, and is of the ſeuen, & goeth into deſtruction. \V
And the ten hornes which thou ſaweſt, are
\SNote{Some expound it of ten ſmal Kingdõs, into which the Roman Empire
shal be deuided, which shal al ſerue Antichriſt both in his life and a
litle after.}
ten Kings, which haue not yet receiued Kingdom, but
\Var{ſhal}{doe}
receiue power as Kings one houre after the beaſt. \V Theſe haue one
counſel and force: and their power they ſhal deliuer to the beaſt. \V
Theſe ſhal fight with the Lamb, and the Lamb ſhal ouercome them, becauſe
\CNote{\XRef{1.~Tim.~6,~15.}
\XRef{Apo.~19,~16.}}
he is Lord of Lords, and King of Kings, and they that are with him,
called, and elect, and faithful. \V And he ſaid to me: The waters which
thou ſaweſt where the harlot ſitteth, are peoples, and Nations, and
tongues. \V And the ten hornes which thou ſaweſt in the beaſt: theſe
ſhal hate the harlot, and
%%% 2968
ſhal make her deſolate and naked, and ſhal eate her fleſh, and her they
shal burne with fire. \V For
\SNote{Not forcing or mouing any to follow Antichriſt, but by his iuſt
iudgement, & for punishment of their ſinnes, permitting thẽ to beleeue
and cõſent to him.}
God hath giuen into their harts, to doe that which pleaſeth him: that
they giue their kingdom to the beaſt, til the words of God be
cõſummate. \V And the woman which thou ſaweſt: is
\LNote{The great citie.}{If
\MNote{The double interpretation of Babylon.}
it be meant of any one citie, and not of the vniuerſal ſocietie of the
reprobate which is the citie of the Diuel, as the Church & the vniuerſal
fellowship of the faithful is called the citie of God, it is moſt like
to be old Rome, as ſome of the Greeks expound it, from the time of the
firſt Emperours, til Cõſtantines daies, who made an end of the
perſecution. For by the authoritie of the old Romane Empire, Chriſt was
put to death firſt, & afterward the two cheefe Apoſtles, & the Popes
their Succeſſours, & infinit Catholike men throughout the world by
leſſer Kings which then were ſubiect to Rome. Al which Antichriſtian
perſecutions ceaſed, when Conſtantine reigned, & yealded vp the citie to
the Pope, who holdeth not the Kingdom or Empire ouer the world, as the
Heathen did, but the fatherhood and ſpiritual rule of the
Church. Howbeit the more probable ſenſe is the other, of the citie of
the Diuel, as the Authour of the
\Cite{homilies vpon the Apocalypſe in S.~Auguſtin,}
declareth.}
the great citie, which hath Kingdom ouer the Kings of the earth.


\stopChapter


\stopcomponent


%%% Local Variables:
%%% mode: TeX
%%% eval: (long-s-mode)
%%% eval: (set-input-method "TeX")
%%% fill-column: 72
%%% eval: (auto-fill-mode)
%%% coding: utf-8-unix
%%% End:

