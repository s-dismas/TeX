%%%%%%%%%%%%%%%%%%%%%%%%%%%%%%%%%%%%%%%%%%%%%%%%%%%%%%%%%%%%%%%%%
%%%%
%%%% The (original) Douay Rheims Bible 
%%%%
%%%% New Testament
%%%% Apocalypse
%%%% Argument
%%%%
%%%%%%%%%%%%%%%%%%%%%%%%%%%%%%%%%%%%%%%%%%%%%%%%%%%%%%%%%%%%%%%%%




\startcomponent argument


\project douay-rheims


%%% 2934
%%% o-2796
\startArgument[
  title={\Sc{The Argument of the
  \Fix{Apocalyspe}{Apocalypse}{obvious typo, fixed in other}
  of S.~John.}},
  marking={argument of the Apocalypse}
  ]

That which the old Testament foretold of Christ himself, the Apostles
could report the fulfilling thereof in the new Testament, by way of an
historie, even from his Conception to his Glorification. But of his
Church, they could not doe the like: because in their time it did but
begin: being to continue long after them, even to the end of the world,
and then at length to be glorified, as Christ her Spouse al-readie
is. Hereupon God would have S.~Luke to report in the Actes of the
Apostles the storie of the Churches beginning, and for the rest of it to
the end, (that we might receive this benefit also by the Apostles hands)
he would S.~John to tel us of it in this booke by way of a prophecie.

Of which booke S.~Jierome saith:
\CNote{\Cite{Jier. ad Paulin.}}
\Emph{The Apocalypse of S.~John hath as many sacraments or mysteries, as
words.} Yea more then that, \Emph{In every word there are hid manifold
and sundrie senses.} Therfore it is very litle that can here be noted,
in respect. Yet to give the good Catholike (whose comfort is here) some
litle help, the booke may be devided into five partes.

The
\CNote{\XRef{Ca.~1.~2.~3.}}
\MNote{1.~part.}
first (after the Proœme) conteineth seven Epistles from Christ now in
glorie, to seven Churches of Asia, or (for, these he maketh al one) to
the seven Bishops of those Churches: meaning not to those only, but to
al his Churches and Bishops through-out the world: saying therfore in
every one of them, to al in general:
\Emph{He that hath an eare, let him heare what the Spirit saith to the
Churches.} As also in every one he exhorteth us to fight manfully (in
this spiritual warfare of ours against sinne) for the victorie, and in
every one accordingly promiseth us a reward in Heaven. But before this,
in the beginning of every one, he partly commendeth, partly
reprehendeth, and exhorteth to penance. Where this is much to be noted
and feared, that among so many, he reproveth some-what in al, save only
in two, which are the
\XRef{second}
& the
\XRef{sixt.}
In the beginning also of every one, he taketh some peece out of the
apparition going before, to frame thereof his style agreably to the
matter of each Epistle.

After
\CNote{\XRef{Ca.~4. to the 8.}}
\MNote{2.}
this admonition to Pastours and their flocks, the second part followeth,
wherein the Church and whole course thereof from the beginning to the
end, is expressed in the opening of a booke in God's hand, and the seven
seales thereof, by Christ. For the which, he seeth praise sung now in
Heaven, and earth, not only to the Godhead, as before, but also (after
a new manner) to Christ according to his Manhood. And here, when he is
come to the opening of the last seale, signifying Domes-day, he letteth
that matter alone for a while,
\CNote{\XRef{Ca.~8. to the 12.}}
and to speake more fully yet of the said course of the Church, he
bringeth in another pagent (as it were) of seven Angels with seven
Trumpets. The effect of both the Seales and Trumpets, is this: That the
Church beginning and proceeding, there should be raised against it,
cruel persecutions, and pestilent heresies: and at
%%% 2935
length after al heresies, a certaine most blasphemous Apostasie, being
the next preparative to the
%%% o-2797
comming of Antichrist: After al which, Antichrist himself in person shal
appeare in the time of the sixt seale, and sixt trumpet, persecuting and
seducing (for the short time of his reigne) more then al before him. The
Church notwithstanding shal stil continue, and wade through al, because
Christ her Spouse is stronger then al these adversaries. Who also
straight after the sayd sixt time, shal in the seventh come in maiestie
and iudge al.

Of
\CNote{\XRef{Ca.~12,~13,~14.}}
\MNote{3.}
the which iudgement, differing yet a while to speake at large, he doth
first in the third part intreat more fully of the Divels working by
Antichrist and his companie against the Church, that the iustice of
Christ afterward in iudging may be more manifest.

At
\CNote{\XRef{C.~15. to the 21.}}
\MNote{4.}
length therefore in the fourth part he commeth to the seven last
plagues, the seventh of them conteining the final damnation of the whole
multitude, societie or corps of the wicked, from the beginning of the
world to the end. Which multitude, in the
\XRef{Ghospel}
and
\XRef{first Epistle of this same S.~John}
(as also in the other Scriptures commonly) is often called
\CNote{\XRef{1.~Jo.~2.}
\XRef{Apoc.~17.}}
\L{Mundus}, \Emph{the world}. And here he calleth it partly,
\L{Meretricem}, \Emph{a whore or harlot}, because with her concupiscence
she entiseth the carnal and earthly men away from God: partly,
\L{Civitatem Babylon}, \Emph{the Citie of Babylon}, because it maketh
warre against Jierusalem the Citie of God, and laboureth to hold God's
people captive in sinne, as it was shadowed in Nabuchodonosor and his
Babylonians, leading and holding the Jewes with their Jierusalem, in
captivitie, until Cyrus (in figure of Christ) delivered them. But
whether al these seven plagues should be understood (as the seventh) of
Domes-day it-self, it is hard to define. More like it is, that the first
sixe are to goe before Domes-day: but whether corporally and literally,
(so as Moyses plagued Ægypt) or rather spiritually, it is more hard to
define. Yet it seemeth more easie, to understand them corporally, as
also the plagues wherewith Elias and his fellow shal in the time of
Antichrist plague the wicked (which peradvanture shal be the same last
plagues) whereof we read in this booke
\XRef{c.~11. v.~6.}
But not content to have described thus the damnation of the whole
adulterous and bloudy societie, he doth also expresly report of their
three grand Captaines damnation, which are these, Antichrist, and his
False-prophet, and the Divel himself the Authour of al this mischiefe.

Finally,
\CNote{\XRef{C.~21.~22.}}
\MNote{5.}
on the other side, in the fifth part he reporteth the unspeakeable and
everlasting glorie, that the Church after al this suffering shal by
Christ her glorious Spouse be assumpted unto. And so concludeth the
booke.


\stopcomponent


%%% Local Variables:
%%% mode: TeX
%%% eval: (long-s-mode)
%%% eval: (set-input-method "TeX")
%%% fill-column: 72
%%% eval: (auto-fill-mode)
%%% coding: utf-8-unix
%%% End:
