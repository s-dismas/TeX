%%%%%%%%%%%%%%%%%%%%%%%%%%%%%%%%%%%%%%%%%%%%%%%%%%%%%%%%%%%%%%%%%
%%%%
%%%% The (original) Douay Rheims Bible 
%%%%
%%%% New Testament
%%%% Apocalypse
%%%% Chapter 19
%%%%
%%%%%%%%%%%%%%%%%%%%%%%%%%%%%%%%%%%%%%%%%%%%%%%%%%%%%%%%%%%%%%%%%




\startcomponent chapter-19


\project douay-rheims


%%% 2971
%%% o-2834
\startChapter[
  title={Chapter 19}
  ]

\Summary{1.~The Saints glorifying God for the iudgement pronounced vpon
  the harlot, 7.~the marriage of the Lamb is prepared. 10.~The Angel
  refuſeth to be adored of S.~Iohn. 11.~There apeareth one (who is the
  Word of God, and the King of Kings and Lord of Lords) ſitting on a
  horſe, with a great armie, and fighting againſt the beaſt and the
  Kings of the earth and their armies: 17.~the birds of the aire being
  in the meane time called to deuoure their flesh.}

After theſe things I heard as it were the voice of many multitudes in
heauen ſaying,
\MNote{\Sc{Allelvia}.}
\Emph{Allelu-ia}. Praiſe, and glorie and power is to our God: \V becauſe
true & iuſt are his iudgements which hath iudged of the great harlot,
that hath corrupted the earth in her whoredom, and hath reuenged the
bloud of his ſeruants, of her hands. \V And
\SNote{This often repeating of \Emph{Allelu-ia} in times of reioycing,
the Church doth follow in her Seruice.}
againe they ſaid, \Emph{Allelu-ia}. And her ſmoke aſcendeth for euer and
euer. \V And the foure and twentie Seniours fel downe, and the foure
beaſts, & adored God ſitting vpõ the throne, ſaying:
\LNote{Amen, Alleluya.}{Theſe
\MNote{\Emph{Amen, Alleluia} not tranſlated.}
two
\TNote{\H{הללויה אמן}}
Hebrew words (as other els-where) both in the Greeke and Latin text are
kept religiouſly, and not tranſlated, vnles it be once or twiſe in the
Pſalmes. Yea and the Proteſtants themſelues keep them in the text of
their English Teſtaments in many places: and maruel it is why they vſe
them not in al places, but ſometimes turne, \Emph{Amen}, into,
\Emph{verily}, whereof ſee the
\XRef{Annotation Ioan.~8. v.~34:}
and in their Seruice booke they tranſlate, \Emph{Alleluia}, into
\Emph{Praiſe ye the Lord}; as though \Emph{Alleluia} had not as good a
grace in the acte of ſeruing God, (where it is indeed properly vſed) as
it hath in the text of the Scripture.

The
\MNote{\Emph{Alleluia} often vſed in the Church, ſpecially in Eaſter
time.}
Church Catholike doth often and ſpecially vſe this ſacred word, to ioyne
with the Church triumphant, conſiſting of Angels and Saints, who here
are ſaid to laud and praiſe God with great reioycing, by this word,
\Emph{Alleluia}, and by often repetition thereof: as the Catholike
Church alſo vſeth, namely in Eaſter time euen til Whit-ſontide, for the
ioy of Chriſts reſurrection, which (as S.~Auguſtin declareth
\CNote{\Cite{Epiſt. ad Ian. c.~17. & c.~15.}}
\Cite{ep. ad Ianuarium)}
was the general vſe of the primitiue Church, making a greater myſterie
and matter of it, then our Proteſtants now doe. At other times of the
yeare alſo he ſaith it was ſung in ſome Churches, but not in al. And
S.~Hierom numbereth it among the hereſies of Vigilantius, that
\Emph{Alleluia} could not be ſung but at Eaſter.
\Cite{Aduerſ. Vigilant. c.~1.}

The
\MNote{It ſignifieth more then (as the Proteſtãts trãſlate it)
\Emph{praiſe ye the Lord}.}
truth is, by the vſe of the Scriptures it hath more
\Fix{it}{in it}{obvious typo, fixed in other}
then, \Emph{Praiſe ye the Lord}, ſignifying with laud, glorifying, and
Prayſing of God a great reioycing withal, mirth, and exultation of hart
in the ſingers thereof. And that is the cauſe why the holy Church ſaith,
\L{Laus tibi Domine}, \Emph{Praiſe to thee, ô Lord}, in Lent and times
of penance and mourning, but not \Emph{Alleluia}. Which (as S.~Auguſtin
alſo declareth) is a terme of ſignification and myſterie, ioyned with
that time, and then vſed ſpecially in the Church of God, when she
repreſenteth to vs in her Seruice, the ioyes and beatitude of the next
life: which is done ſpecially at Eaſter, by the ioyful celebrating of
Chriſts glorious Reſurrection and Aſcenſion, after the penal time of
Lent which repreſenteth the miſerie of this life. See S.~Auguſtin
\Cite{Ser.~1.}
&
\Cite{5. c.~9.}
&
\Cite{6. c.~9.}
\Cite{de Diuerſis to.~10.}
and his
\Cite{enarration vpon the 148.~Pſalme.}
For in the titles and ends of diuerſe holy Pſalmes this \Emph{Alleluia} is
ful of myſterie and ſacred ſignification.
\MNote{Falſe tranſlation.}
Where we muſt aske the Proteſtants, why they haue left it out
altogether, being in the Hebrew, ſaying neither \Emph{Alleluia}, nor
\Emph{Praiſe ye the Lord}, in the
\Cite{Bible~1577:}
and that nine times in the ſixe laſt Pſalmes.

Moreouer
\MNote{\Emph{Amen} and \Emph{Alleluia} should not be tranſlated into
vulgar tõgues.}
the ſaid holy Doctour
\Cite{(li.~2. de doct. Chriſt. c.~11.)}
affirmeth that \Emph{Amen} and \Emph{Alleluia} be not tranſlated into
any other language \L{propter ſanctiorem authoritatem}, for the more
ſacred authoritie of the words ſo remaining. And
\Cite{ep.~178.}
he ſaith that it is not lawful to tranſlate them. \L{Nam ſciendum eſt
&c.}
\MNote{Al Nations in the Primitiue Church ſang \Emph{Amen} and
\Emph{Alleluia}.}
\Emph{For it is knowen} (ſaith he) \Emph{that al Nations doe ſing Amen
and Alleluia in the Hebrew words, which neither the Latin man nor the
Barbarous may tranſlate into his owne language.} See S.~Hierom alſo
\Cite{Epiſt.~1.~7.}
And namely for our Nation, S.~Gregorie wil beare vs witnes that our
countrie receiued the word \Emph{Alleluia} with their Chriſtianitie,
ſaying thus,
\Cite{li.~27. Moral. c.~6.}
\L{Lingua Britaniæ quæ nihil aliud nouerat quam barbarum frendere,
iandudum in Diuines laudibur Hebræum capit reſonare Alleluia},
that is, \Emph{The Britan tongue, which knew nothing els but to mutter
barbarouſly, hath begun of late in God's diuine lauds and praiſes to
ſound the Hebrew Alleluia}. And for Iurie S.~Hierom
\Cite{ep.~17. c.~7.}
writeth, that the husbandmen at the plough ſang \Emph{Alleluia}, which
was not then their vulgar ſpeach. Yea he ſaith that in Monaſteries the
ſinging of \Emph{Alleluia} was inſteed of a bel to cal them together
\Cite{ad Collectam in Epitaph. Paul c.~10.}

This
\MNote{The Proteſtãts profane this word by tranſlating it, & diminish
the ſignification thereof.}
word is a ſacred, Chriſtian, myſtical, and Angelical ſong: and yet in
the new ſeruice booke it is turned into, \Emph{Praiſe ye the Lord}, and
\Emph{Alleluia} is quit gone, becauſe they liſt neither to agree with
the Church of God, nor with the vſe of holy Scriptures, no nor with
their owne tranſlations. But no maruel, that they can not ſing
\CNote{Pſ.~136.}
\Emph{the ſong of our Lord} and of Angels \Emph{in a ſtrange countrie},
that is, out of the Catholike Church in the captiuitie of ſchiſme and
hereſie. Laſtly, we might aske them whether it be al one to ſay
\XRef{Mat.~21.}
\Emph{Hoſanna}, and \Emph{Saue vs we beſeech thee}? whereas Hoſanna is
withal a word of exceeding congratulation and ioy which they expreſſed
toward our Sauiour. Euen ſo \Emph{Alleluia} hath another manner of ſenſe
and ſignification in it, then can be expreſſed by, \Emph{Praiſe ye the
Lord}.}
\Emph{Amen, Allelu-ia}. \V And a voice came out from the throne, ſaying:
Say praiſe to our God al ye his ſeruants: and you that feare him, litle
and great. \V And I heard as it were the voice of a great multitude, and
as the voice of many waters, & as the voice of great thunders, ſaying,
\Emph{Allelu-ia}: becauſe our
%%% 2972
Lord God the omnipotent hath reigned. \V Let vs be glad and reioyce, and
giue glorie to him: becauſe
\SNote{At this day shal the whole Church of the elect be finally and
perfectly for euer ioyned vnto Chriſt in marriage inſeparable.}
the marriage of the Lamb is come, & his wife hath prepared herſelf. \V
And it was giuen to her that ſhe clothe her ſelf with ſilke glittering
and white. For the ſilke are
\LNote{Iuſtifications of Saints.}{Here
\MNote{Iuſtifications are good workes, not as the effects of faith
iuſtifying, but becauſe themſelues alſo with faith iuſtifie a man.}
the Heretikes in their tranſlations could not alter the word
\Emph{iuſtifications} into \Emph{ordinances}, or \Emph{conſtitutions},
as they did falſely in the
\XRef{firſt of S.~Luke,}
whereof ſee the
\XRef{Annotation there verſ.~6.}
but they are forced to ſay in Latin, \L{iuſtificationes}, as Beza: and
in English, \Emph{righteouſnes}, (for \Emph{iuſtifications} they wil not
ſay in any caſe for feare of inconuenience,) yea and they can not deny
but theſe iuſtifications be the good workes of Saints. But where
\CNote{\Cite{Beza.}}
they make this gloſſe, that they be ſo called, becauſe they are the
fruits or effect of faith and of the iuſtice which we haue by only
faith, it is moſt euidently falſe, and againſt the very text, and nature
of the word. For there is no cauſe why any thing should be called a mans
iuſtification, but for that it maketh him iuſt. So
that, \Emph{iuſtifications}, be the vertues of faith, hope, charitie,
and good deeds, iuſtifying or making a man iuſt, and not effects of
iuſtification. Neither faith only, but they altogether be the very
ornaments and inward garments, beauty, and iuſtice of the ſoule, as here
it is euident.}
the iuſtifications of Saints.

\V And he ſaid to me: Write,
\CNote{\XRef{Mt.~22.}
\XRef{Lu.~14.}}
Bleſſed be they that are called to the
\SNote{That is the feaſt of eternal life prepared for his ſpouſe the
Church.}
ſupper of the marriage of the Lamb. And he ſaid to me: Theſe wordes of
God, be true. \V
\LNote{And I fel.}{The
\MNote{S.~Iohns adoring of the Angel explicated againſt the Proteſtants
abuſing the ſame.}
Proteſtants abuſe this place, and the example of the Angels forbidding
Iohn to adore him being but his fellow-ſeruant, and appointing him to
adore God, againſt al honour, reuerence, and adoration of Angels,
Saints, or other ſanctified creatures, teaching that no religious
worship ought to be done vnto them. But in truth it maketh for no ſuch
purpoſe, but only warneth vs that Diuine honour and the adoration due to
God alone, may not be giuen to any Angel or other creature.
\Cite{S.~Aug. de vera relig. cap. vltimo.}
And when the Aduerſaries replie that ſo great an Apoſtle, as Iohn was,
could not be ignorant of that point, nor would haue giuen diuine honour
vnto an Angel (for ſo he had been an Idolater) and therfore that he was
not reprehended for that, but for doing any religious reuerence or other
honour whatſoeuer to his fellow-ſeruant:
\MNote{The Proteſtãts are refelled by their owne reaſon.}
we anſwer that by the like reaſon, S.~Iohn being ſo great an Apoſtle, if
this kind of reuerence had been vnlawful and to be reprehended, as the
Proteſtants hold it is no leſſe then the other, could not haue been
ignorant thereof, nor would haue done it.

Therfore they might much better haue learned of S.~Auguſtin
\Cite{(q.~61. in Geneſ.)}
how this fact of S.~Iohn was corrected by the Angel, and wherein the
errour was. In effect it is thus,
\MNote{S.~Iohn erred only in the perſon, myſtaking the Angel to be
Chriſt himſelf, & ſo adoring him as God.}
That the Angel being ſo glorious and ful of maieſtie, preſenting Chriſts
Perſon, and in his name vſing diuers wordes proper to God, as,
\CNote{\XRef{Apoc. c.~1.}}
\Emph{I am the firſt and the laſt, and aliue and was dead}, and ſuch
like, might wel be taken of S.~Iohn, by errour of his Perſon, to be
Chriſt himſelf, and that the Apoſtle preſuming him to be ſo indeed,
adored him with Diuine honour: which the Angel correcting, told him he
was not God, but one of his fellowes, and therfore that he should not ſo
adore him, but God. Thus then we ſee, Iohn was neither ſo ignorant, to
thinke that any vndue honour might be giuen to any creature: nor ſo il,
to commit idolatrie by doing vndue worship to any Angel in heauen: and
therfore was not culpable at al in this fact, but only erred materially
(as the Schole-men cal it) that is, by miſtaking one for another,
thinking that which was an Angel, to haue been our Lord: becauſe he knew
that our Lord himſelf is alſo
\CNote{\XRef{Eſa.~3. in Graco.}
\XRef{Malac.~3.}}
called an Angel, and hath often appeared in the viſions of the faithful.

And
\MNote{S.~Iohn ſinned not in this adoration.}
the like is to be thought of the Angel appearing in the
\XRef{22. of the Apocalypſe,}
whether it were the ſame or another, for that alſo did ſo appeare, that
Iohn could not tel whether it were Chriſt himſelf or no, til the Angel
told him. Once this is certaine, that Iohn did not formally (as they
ſay) commit idolatrie, nor ſinne at al herein, knowing al dueties of a
Chriſtian man, no leſſe then an Angel of heauen, being alſo in as great
honour with God, yea and in more then many Angels.
\MNote{Another explication of this place.}
Which perhaps may be the cauſe (and conſequently another explication of
this place) that the Angel knowing his great graces and merits before
God, would not accept any worship or ſubmiſſion at his hands, though
Iohn againe of like humilitie did it, as alſo immediately afterward
\XRef{chap.~22.}
which belike he would not haue done, if he had been preciſely aduiſed by
the Angel but a moment before, of errour and vnduetifulnes in the
fact. Howſoeuer that be, this is euident, that this the Angels refuſing
of adoration, taketh not away the due reuerence and reſpect we ought to
haue to Angels or other ſanctified perſons and creatures; and ſo theſe
wordes, \Emph{See thou doe it not}, ſignifie rather an earneſt refuſal
then any ſignification, of crime to be commited thereby.

And
\MNote{The Proteſtãts by conference of Scriptures might find religious
adoration of creatures.}
maruel it is that the Proteſtants making themſelues ſo ſure of the true
ſenſe of euery doubtful place by conference of other Scriptures, follow
not here the conference & comparing of Scriptures that themſelues ſo
much or only require. We wil giue them occaſion & a methode ſo to
doe. He that doubteth of this place, findeth out three things of
queſtiõ, which muſt be tried by other Scriptures.
\MNote{Three points herein examined & proued by Scriptures.}
The firſt, whether there ought to be or may be any religious reuerence
or honour done to any creatures: taking the word \Emph{religion} or
\Emph{religious worship} not for that ſpecial honour which is properly
and only due to God, as
\CNote{\Cite{Aug. de vera relig. c.~55.}}
S.~Auguſtin ſometimes vſeth it, but for reuerence due to any thing that
is holy by ſanctification or application to the ſeruice of God. The
ſecond thing, is whether by vſe of Scriptures, that honour be called
\Emph{adoration} in Latin, or by a word equiualent in other languages,
%%% !!! Hebrew only in other
\TNote{\H{השתחוה}}
Hebrue,
\TNote{\G{προσκυνέω}}
Greeke, or English. Laſtly, whether we may by the Scriptures fal
downe proſtrate before the things, or at the feete of perſons that we ſo
adore. For of ciuil duty done to our Superiours by capping, kneeling, or
other courteſie, I thinke the Proteſtants wil not ſtand with vs: though
indeed, their arguments make as much againſt the one as the other.

But
\MNote{1.

Religious worship of creatures.}
for religious worship of creatures (which we ſpeake of) let them ſee in
the Scriptures both old and new: firſt, whether
%%% !!! Attach these to the specific examples?
\CNote{\XRef{Pſ.~5,~137.}
\XRef{Dan.~6.}
\XRef{3.~Reg.~8.}
\XRef{Ioſ.~7.}
\XRef{Pſ.~98.}
\XRef{131.}}
the Temple, the tabernacle, the Arke, the propitiatorie, the Cherubins,
the altar, the bread of propoſition, the Sabboth, and al their holies,
were not reuerenced by al ſignes of deuotion and religion: whether the
Sacraments of Chriſt, the Prieſt of our Lord, the Prophets, of God, the
Ghoſpel, Scriptures, the name of \Sc{Iesvs}, and ſuch like (which be by
vſe, ſignification, or ſanctification made holy) are not now to be
reuerenced: and they shal find al theſe things to haue been reuerenced
of al the faithful, without any diſhonour of God, and much to his
honour.
\MNote{2.

The ſame is called adoration.}
Secondly, that this reuerence is named \Emph{adoration} in the
Scriptures, theſe ſpeaches doe proue
\XRef{Pſ.~98.}
\Emph{Adore ye his foot-ſtool, becauſe it is holy}; and
\XRef{Hebr.~11.}
\Emph{He adored the toppe of his rod}.
\MNote{3.

Falling proſtrate before the perſons or things adored.}
Thirdly, that the Scriptures alſo warrant vs (as the nature of the word
\Emph{adoration} giueth in al three tongues) to bowe downe our bodies,
to fal flat on the ground at the preſence of ſuch things, and at the
feete of holy perſons, ſpecially Angels, as Iohn doth here, theſe
examples proue.
\CNote{\XRef{Gen.~18.}}
Abraham adored the Angels that appeared to him.
\CNote{\XRef{Exod.~3.}}
Moyſes alſo the Angel that shewed himſelf out of the bush, who were
creatures, though they repreſented Gods Perſon, as this Angel here did,
that ſpake to S.~Iohn. Balaam adored the Angel that ſtood before him
with a ſword drawen
\XRef{Num.~22.}
\CNote{\XRef{Ioſue.~3.}}
Ioſue adored falling flat downe before the feet of the Angel, calling
him his Lord, knowing by the Angels owne teſtimonie, that it was but an
Angel. Who refuſed it not, but required yet more reuerence, cõmanding
him to plucke of his shoes, becauſe the ground was holy, no doubt ſo
made by the preſence only of the Angel.

Yea
\MNote{Adoring of Prophets and holy perſons.}
not only to Angels, but euen to great Prophets this deuotion was done,
\CNote{\XRef{Dan.~2.}}
as to Daniel by Nabuchodonoſor, who fel flat vpon his face before him,
and did other greate offices of religion, which the Prophet refuſed not,
becauſe they were done to God rather then to him, as
%%% !!! Cite?
S.~Hierom defendeth
the ſame againſt Porphyrie; who charged Daniel with intolerable pride
therin: and the ſaid holy Doctour alleageth the fact of Alexander the
great, that did the like to
\TNote{or Iaddus.}
Ioiadas the high Prieſt of the Iewes. Howſoeuer that be (for of the
Sacrifice there mentioned there may be ſome doubt, which the Church doth
alwaies immediately to God, and to no creature) the fact of the Prophets
\XRef{(4.~Reg.~2.)}
to Eliſæus is plaine: where they perceiuing that the double grace and
ſpirit of Elias was giuen to him, fel flat downe at his feet and
adored. So did
\CNote{\XRef{4.~Reg.~4.}}
the Sunamite: to omit that
\CNote{\XRef{Iudith.~13.}}
Achior adored Iudith, falling at her feet, as a woman bleſſed of God,
and infinit other places.

Al which things, by cõparing the Scriptures, our Aduerſaries should haue
found to be lawfully done to men, & Angels, & ſoueraigne holy
creatures. Whereby they might conuince themſelues, and perceiue, that
that thing could not be forbidden S.~Iohn to doe to the Angel, which
they pretend: though the Angel for cauſes might refuſe euen that which
S.~Iohn did lawfully vnto him, as S.~Peter did refuſe the honour giuen
him by Cornelius, according to S.~Chryſoſtom's opinion
\Cite{ho.~33. in c.~10. Act.}
Yea euen in the
\XRef{third chapter of this booke}
(if our Aduerſaries would looke no further) they might ſee where this
Angel prophecieth and promiſeth that the Iewes should fal downe before
the feet of the Angel of Philadelphia and adore. See the
\XRef{Annot. there.}}
And
\CNote{\XRef{Apoc.~22,~9.}}
I fel before his feete, to adore him. And he ſaith to me: See thou doe
not; I am thy fellow-ſeruant, and of thy Brethren that
%%% o-2835
haue the teſtimonie of \Sc{Iesvs}. Adore God. For the teſtimonie
of \Sc{Iesvs}, is the ſpirit of prophecie.

\V And I ſaw heauen opened, and behold a white horſe: and he that ſate
vpon him, was called Faithful and True, and with iuſtice he iudgeth &
fighteth. \V And his eyes as a flame of fire, and on his head many diadems,
hauing a name written, which no man knoweth but himſelf. \V
\CNote{\XRef{Eſa.~63,~1.}}
And he was clothed with a garment ſprinkled with bloud: and his name is
called,
\SNote{The ſecond Perſon in Trinitie, the Sõne or the Word of God, which
was made flesh.
\XRef{Io.~1.}}
\Sc{The Word of God}. \V And the hoſtes that are in heauen followed him
on white horſes clothed in white and pure ſilke. \V And out of his mouth
proceedeth a ſharp ſword: that in it he may ſtrike the Gentils. And
\CNote{\XRef{Apoc.~2,~27.}}
he ſhal rule them in a rod of yron: and he treadeth the wine preſſe of
the furie of the wrath of God omnipotent. \V And he hath in his garment
and in his thigh written,
\CNote{\XRef{Apoc.~17,~14.}}
\SNote{Euen according to his humanitie alſo.}
\Sc{King of Kings and Lord of Lords}.

\V And I ſaw one Angel ſtanding in the ſunne, & he cried with a loud
voice ſaying to al the birds that did flie by the middes of heauen: Come
and aſſemble together to the great ſupper of God: \V that you may eate
the flesh of Kings, and the flesh of Tribunes, & the fleſh of valiants,
and the fleſh of horſes and of them that ſit on them, and the fleſh of al
free-men and bond-men, and of litle and great.

\V And I ſaw the beaſt and the Kings of the earth, & their armies
gathered to make warre with him that ſate vpon the horſe and with his
armie. \V And the beaſt was apprehended, and with him the falſe-Prophet:
which wrought ſignes before him, wherewith he ſeduced them that tooke
the character of the beaſt, and that adored his image. Theſe two were
caſt aliue into the poole of fire burning alſo with brimſtone. \V And
the reſt were ſlaine by the ſword of him that ſitteth vpon the horſe,
which proceedeth out of his mouth: and al the birds were filled with
their fleſh.


\stopChapter


\stopcomponent


%%% Local Variables:
%%% mode: TeX
%%% eval: (long-s-mode)
%%% eval: (set-input-method "TeX")
%%% fill-column: 72
%%% eval: (auto-fill-mode)
%%% coding: utf-8-unix
%%% End:

