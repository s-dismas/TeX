%%%%%%%%%%%%%%%%%%%%%%%%%%%%%%%%%%%%%%%%%%%%%%%%%%%%%%%%%%%%%%%%%
%%%%
%%%% The (original) Douay Rheims Bible 
%%%%
%%%% New Testament
%%%% Apocalypse
%%%% Argument
%%%%
%%%%%%%%%%%%%%%%%%%%%%%%%%%%%%%%%%%%%%%%%%%%%%%%%%%%%%%%%%%%%%%%%




\startcomponent argument


\project douay-rheims


%%% 2934
%%% o-2796
\startArgument[
  title={\Sc{The Argvment of the
  \Fix{Apocalyspe}{Apocalypse}{obvious typo, fixed in other}
  of S.~Iohn.}},
  marking={argument of the Apocalypſe}
  ]

That which the old Teſtament foretold of Chriſt himſelf, the Apoſtles
could report the fulfilling thereof in the new Teſtament, by way of an
hiſtorie, euen from his Conception to his Glorification. But of his
Church, they could not doe the like: becauſe in their time it did but
begin: being to continue long after them, euen to the end of the world,
and then at length to be glorified, as Chriſt her Spouſe al-readie
is. Hereupon God would haue S.~Luke to report in the Actes of the
Apoſtles the ſtorie of the Churches beginning, and for the reſt of it to
the end, (that we might receiue this benefit alſo by the Apoſtles hands)
he would S.~Iohn to tel vs of it in this booke by way of a prophecie.

Of which booke S.~Hierome ſaith:
\CNote{\Cite{Hier. ad Paulin.}}
\Emph{The Apocalypſe of S.~Iohn hath as many ſacraments or myſteries, as
words.} Yea more then that, \Emph{In euery word there are hid manifold
and ſundrie ſenſes.} Therfore it is very litle that can here be noted,
in reſpect. Yet to giue the good Catholike (whoſe comfort is here) ſome
litle help, the booke may be deuided into fiue partes.

The
\CNote{\XRef{Ca.~1.~2.~3.}}
\MNote{1.~part.}
firſt (after the Proœme) conteineth ſeuen Epiſtles from Chriſt now in
glorie, to ſeuen Churches of Aſia, or (for, theſe he maketh al one) to
the ſeuen Bishops of thoſe Churches: meaning not to thoſe only, but to
al his Churches and Bishops through-out the world: ſaying therfore in
euery one of them, to al in general:
\Emph{He that hath an eare, let him heare what the Spirit ſaith to the
Churches.} As alſo in euery one he exhorteth vs to fight manfully (in
this ſpiritual warfare of ours againſt ſinne) for the victorie, and in
euery one accordingly promiſeth vs a reward in Heauen. But before this,
in the beginning of euery one, he partly commendeth, partly
reprehendeth, and exhorteth to penance. Where this is much to be noted
and feared, that among ſo many, he reproueth ſome-what in al, ſaue only
in two, which are the
\XRef{ſecond}
& the
\XRef{ſixt.}
In the beginning alſo of euery one, he taketh ſome peece out of the
apparition going before, to frame thereof his ſtyle agreably to the
matter of each Epiſtle.

After
\CNote{\XRef{Ca.~4. to the 8.}}
\MNote{2.}
this admonition to Paſtours and their flocks, the ſecond part followeth,
wherein the Church and whole courſe thereof from the beginning to the
end, is expreſſed in the opening of a booke in God's hand, and the ſeuen
ſeales thereof, by Chriſt. For the which, he ſeeth praiſe ſung now in
Heauen, and earth, not only to the Godhead, as before, but alſo (after
a new manner) to Chriſt according to his Manhood. And here, when he is
come to the opening of the laſt ſeale, ſignifying Domeſ-day, he letteth
that matter alone for a while,
\CNote{\XRef{Ca.~8. to the 12.}}
and to ſpeake more fully yet of the ſaid courſe of the Church, he
bringeth in another pagent (as it were) of ſeuen Angels with ſeuen
Trumpets. The effect of both the Seales and Trumpets, is this: That the
Church beginning and proceeding, there should be raiſed againſt it,
cruel perſecutions, and peſtilent hereſies: and at
%%% 2935
length after al hereſies, a certaine moſt blaſphemous Apoſtaſie, being
the next preparatiue to the
%%% o-2797
comming of Antichriſt: After al which, Antichriſt himſelf in perſon shal
appeare in the time of the ſixt ſeale, and ſixt trumpet, perſecuting and
ſeducing (for the short time of his reigne) more then al before him. The
Church notwithſtanding shal ſtil continue, and wade through al, becauſe
Chriſt her Spouſe is ſtronger then al theſe aduerſaries. Who alſo
ſtraight after the ſayd ſixt time, shal in the ſeuenth come in maieſtie
and iudge al.

Of
\CNote{\XRef{Ca.~12,~13,~14.}}
\MNote{3.}
the which iudgement, differing yet a while to ſpeake at large, he doth
firſt in the third part intreat more fully of the Diuels working by
Antichriſt and his companie againſt the Church, that the iuſtice of
Chriſt afterward in iudging may be more manifeſt.

At
\CNote{\XRef{C.~15. to the 21.}}
\MNote{4.}
length therefore in the fourth part he commeth to the ſeuen laſt
plagues, the ſeuenth of them conteining the final damnation of the whole
multitude, ſocietie or corps of the wicked, from the beginning of the
world to the end. Which multitude, in the
\XRef{Ghoſpel}
and
\XRef{firſt Epiſtle of this ſame S.~Iohn}
(as alſo in the other Scriptures commonly) is often called
\CNote{\XRef{1.~Io.~2.}
\XRef{Apoc.~17.}}
\L{Mundus}, \Emph{the world}. And here he calleth it partly,
\L{Meretricem}, \Emph{a whore or harlot}, becauſe with her concupiſcence
she entiſeth the carnal and earthly men away from God: partly,
\L{Ciuitatem Babylon}, \Emph{the Citie of Babylon}, becauſe it maketh
warre againſt Hieruſalem the Citie of God, and laboureth to hold God's
people captiue in ſinne, as it was shadowed in Nabuchodonoſor and his
Babylonians, leading and holding the Iewes with their Hieruſalem, in
captiuitie, vntil Cyrus (in figure of Chriſt) deliuered them. But
whether al theſe ſeuen plagues should be vnderſtood (as the ſeuenth) of
Domeſ-day it-ſelf, it is hard to define. More like it is, that the firſt
ſixe are to goe before Domeſ-day: but whether corporally and literally,
(ſo as Moyſes plagued Ægypt) or rather ſpiritually, it is more hard to
define. Yet it ſeemeth more eaſie, to vnderſtand them corporally, as
alſo the plagues wherewith Elias and his fellow shal in the time of
Antichriſt plague the wicked (which peraduanture shal be the ſame laſt
plagues) whereof we read in this booke
\XRef{c.~11. v.~6.}
But not content to haue deſcribed thus the damnation of the whole
adulterous and bloudy ſocietie, he doth alſo expreſly report of their
three grãd Captaines damnation, which are theſe, Antichriſt, and his
Falſe-prophet, and the Diuel himſelf the Authour of al this miſchiefe.

Finally,
\CNote{\XRef{C.~21.~22.}}
\MNote{5.}
on the other ſide, in the fifth part he reporteth the vnſpeakeable and
euerlaſting glorie, that the Church after al this ſuffering shal by
Chriſt her glorious Spouſe be aſſumpted vnto. And ſo concludeth the
booke.


\stopcomponent


%%% Local Variables:
%%% mode: TeX
%%% eval: (long-s-mode)
%%% eval: (set-input-method "TeX")
%%% fill-column: 72
%%% eval: (auto-fill-mode)
%%% coding: utf-8-unix
%%% End:
