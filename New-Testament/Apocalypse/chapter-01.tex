%%%%%%%%%%%%%%%%%%%%%%%%%%%%%%%%%%%%%%%%%%%%%%%%%%%%%%%%%%%%%%%%%
%%%%
%%%% The (original) Douay Rheims Bible 
%%%%
%%%% New Testament
%%%% Apocalypse
%%%% Chapter 01
%%%%
%%%%%%%%%%%%%%%%%%%%%%%%%%%%%%%%%%%%%%%%%%%%%%%%%%%%%%%%%%%%%%%%%




\startcomponent chapter-01


\project douay-rheims


%%% 2938
%%% o-2798
\startChapter[
  title={Chapter 1}
  ]

\Summary{9.~S.~Iohn
\MNote{\Sc{The 1.~Part.}

Seuen Epiſtles to the Churches.}
being banished in the Ile Patmos, is commanded to write to the ſeuen
Churches of Aſia (ſignified by the ſeuen candleſtickes) that which he
ſaw vpon a Sunday, round about the Sonne of man: 13.~whoſe manner of
apparition is deſcribed.}

The
\LNote{\Sc{Apocalypse.}}{Of
\MNote{An admonitiõ to the Reader concerning the difficultie of this
book.}
the Apocalypſe thus writeth the Ancient Father Denys, Bishop of Corinth,
as Euſebius alleageth him
\Cite{li.~7. c.~20. hiſt. Eccl.}
\Emph{Of this booke} (ſaith he) \Emph{this is my opinion, that the
matter thereof is farre more profound then my wit can reach vnto and I
doubt not but almoſt in euery ſentence of it there lieth hidden a
certaine ſenſe exceeding myſtical and maruelous, which though I
vnderſtand not, yet I conceiue that vnder the words there is a deep
meaning: and I meaſure not the matter by reaſon, but attribute al to
faith, taking it to be more high and diuine, then I can by cogitation
compriſe: not reprouing that which I vnderſtand not, but therfore I
admire with reuerence, becauſe my wit can not attaine to it.} Againe
S.~Auguſtin ſaith, \Emph{that in the Apocalypſe many things are
obſcurely ſpoken, to exerciſe the mind of the Reader: and yet ſome few
things left euident that through them a man may with labour ſearch out
the reſt. Specially for that the Authour ſo repeateth the ſame things
in diuers ſorts, that ſeeming to ſpeake of ſundry matters, indeed is found
to vtter the ſame things diuers waies.}
\Cite{li.~20. de Ciuit. Dei. c.~17.}

Which we ſet downe here in the beginning, to warne the good Chriſtian
Reader, to be humble and wiſe in the reading both of al other holy
Scriptures, & namely of this diuine and deep prophecie: giuing him
further to vnderſtand, that we wil in our Annotations, according to our
former trade and purpoſe, only or cheefely note vnto the ſtudious, ſuch
places as may be vſed by Catholikes, or abuſed by Heretikes, in the
controuerſies of this time, and ſome other alſo that haue ſpecial matter
of edification, and that as breefely as may be, for that the volume
groweth great.}
Apocalypſe of \Sc{Iesvs} Chriſt which God gaue him, to make manifeſt to
his ſeruants the things which muſt be done quickly: and ſignified,
ſending by his Angel to his ſeruant Iohn, \V who hath giuen teſtimonie
to the word of God, and the teſtimonie of \Sc{Iesvs} Chriſt, what things
ſoeuer he hath ſeen. \V Bleſſed is he that readeth and heareth the words
of this prophecie: and
\SNote{There be many (ſpecially now a-daies) that be great readers,
hearers & talkers of Scriptures. But that is not enough to make them
good or bleſſed before God, except they keep the things preſcribed and
taught therein according to our Sauiours ſaying
\XRef{Luc.~11.}
Bleſſed are they that heare the word of God, & keep it.}
keepeth thoſe things which be written in it. For the time is nigh.

\V Iohn
\LNote{To the 7.~Churches.}{That
\MNote{Numbers myſtical.}
certaine numbers may be obſerued as ſignificatiue and myſtical, it is
plaine by many places of holy Scripture, and by the ancient Doctours
ſpecial noting of the ſame to many purpoſes. Whereby we ſee the rashnes
of our Aduerſaries, in condemning generally al religious reſpect of
certaine numbers in praiers, faſts, or actions.
\MNote{The number of Seuẽ myſtical: ſpecially in this booke.}
Namely the number of
\Emph{Seuen}, is myſtical, and prophetical, perfect, and which (as
S.~Auguſtin ſaith) the Church knoweth by the Scriptures, to be ſpecially
dedicated to the Holy Ghoſt: and to appertaine to ſpiritual mundation,
as in the Prophets appointing of Naaman to wash ſeuen times in Iordan,
and the ſprinkling of the bloud ſeuen times againſt the tabernacle.
\Cite{li.~4. quæſt. in numer. q.~33.}
See
\Cite{li.~5. c.~5. de Gen. ad lit.}
&
\Cite{l.~5. quæſt. in Deuter. q.~42.}
Al theſe viſions ſtand vpon Seuens: Seuen Churches, ſeuen Angels, ſeuen
ſtarres, ſeuen ſpirits, ſeuen candleſticks, ſeuen lamps, ſeuen trumpets,
ſeuen vials, ſeuen hornes of the Lamb, ſeuen hilles, ſeuen thunders,
ſeuen heads of the Dragon, ſignifying the Diuel: ſeuen of the beaſt,
that is Antichriſt: ſeuen of the beaſt that the harlot rid vpon: finally
the number alſo of the viſions is ſpecially marked to be ſeuen, in this
booke. And euery time that this number is vſed in this prophecie, it
hath a myſterie & a more large meaning then the nature of that number is
preciſely and vulgarly taken for. As when he writeth to ſeuen Churches,
it is to be vnderſtood of al the Churches in the world, as the ſeuen
Angels for al the Angels or
\Fix{Gouernous}{Gouernours}{obvious typo, fixed in other}
of the whole Catholike Church,
and ſo-forth in the reſt; becauſe the number of \Emph{Seuen}, hath the
perfection of vniuerſalitie in it, as S.~Auguſtin ſaith
\Cite{li.~5. quæſt. in Deuter. q.~42.}}
to the ſeuen Churches which are in Aſia. Grace to you and peace from
\CNote{\XRef{Exo.~3,~14.}}
him that is, and that was, and that ſhal come, and
\LNote{From the 7.~ſpirits.}{The Holy Ghoſt may be here meant, and ſo
called for his ſeuen-fold guifts and graces, as ſome Expoſitours thinke.
\MNote{Grace & peace from God and the holy Angels.}
But it ſeemeth more probable that he ſpeaketh of the holy Angels, by
comparing this to the like in the
\XRef{5.~Chapter}
following: where he ſeemeth to cal theſe, the ſeuen Spirits ſent into al
the world, as S.~Paul to the Hebrewes
\XRef{(c.~1,~14.)}
ſpeaketh of Angels. And ſo the Proteſtants take it in their
commentaries. Which we note, becauſe thereupon they muſt needs confeſſe
that the Apoſtle here giueth or wisheth grace & peace not from God only,
but alſo from his Angels: though that benediction commeth one way of
God, and another way of his Angels or Saints, being but his
creatures. And ſo they may learne, that the faithful often ioyning in
one ſpeach,
\MNote{God and our Ladie ſaue vs, and the like.}
\Emph{God and our Lady, our Lord & any of his Saints}, to helpe vs or
bleſſe vs, is not ſuperſtitious, but an Apoſtolical ſpeach. And ſo the
Patriarch ſaid
\XRef{(Gen.~48. v.~16.)}
\Emph{The Angel that deliuereth me from al euils, bleſſe theſe
children.} See the
\XRef{Annot. Act.~15,~28.}}
from the ſeuen Spirits which are in the ſight of his throne, \V and from
\Sc{Iesvs} Chriſt who is the faithful witnes, the
\CNote{\XRef{Col.~1.}}
Firſt-borne of the dead, and the Prince of the Kings of the earth, who
hath loued vs, and
\CNote{\XRef{Heb.~9.}}
waſhed vs from our ſinnes in his bloud, \V and hath made vs
\CNote{\XRef{1.~Pet.~1.}
\XRef{2.~Pet.~2.}}
\LNote{A Kingdom and Prieſts.}{As
\MNote{How al Chriſtians be both Kings & Prieſts.}
al that truely ſerue God, and haue the dominion and ſuperioritie ouer
their concupiſcences and whatſoeuer would induce them to ſinne, be Kings;
ſo al that employ their workes and themſelues to ſerue God, & offer al
their actiõs as an acceptable Sacrifice to him, be
Prieſts. Neuertheleſſe, as if any man would therevpon affirme that there
ought to be no other earthly Powers or Kings to gouerne in worldly
affaires ouer Chriſtians, he were a ſeditious
\Fix{Heretikes:}{Heretike:}{obvious typo, fixed in other}
euen ſo are they that vpon this or the like places where al Chriſtians
be called Prieſts in a ſpiritual ſort, would therfore inferre, that
euery one is in proper ſignification a Prieſt, or that al be Prieſts
alike, or that there ought to be none but ſuch ſpiritual Prieſts. For it
is the ſeditious voice of Core, ſaying to Moyſes and Aaron, \Emph{Let it
ſuffice you, that al the multitude is of holy ones, and the Lord is in
them. Why are you extolled ouer the people of the Lord?
\XRef{Num.~16.}}}
a Kingdom and Prieſts to God and his Father, to him be glorie and empire
for euer and euer. Amen. \V Behold he commeth with the clouds, and euery
eie ſhal ſee him, and
\CNote{\XRef{Zac.~12.}}
they that pricked him. And al the Tribes of the earth ſhal bewaile
themſelues vpon him. Yea, Amen. \V
\CNote{\XRef{Eſa.~44.}
\XRef{Apoc.~21.}
\XRef{22,~13.}}
I am Alpha and Omega, the beginning and end, ſaith our Lord God,
which is, and which was, and which shal come, the Omnipotent.

%%% o-2799
\V I Iohn your brother and partaker in tribulation, and the Kingdom, and
patience in Chriſt \Sc{Iesvs}, was
\SNote{Banished thither for religiõ by Nero, or rather by Domitiã,
alſmoſt 60.~yeares after Chriſts Aſcenſion.}
in the Iland, which is called Patmos, for the word of God and the
teſtimonie of \Sc{Iesvs}. \V I was
%%% 2939
\SNote{I had a viſiõ, & not with my corporal eyes, but in ſpirit I
beheld the ſimilitudes of the things following.}
in ſpirit
\LNote{On the Dominical day.}{Many
\MNote{Difference of holy-daies and worke-daies.}
notable points may be marked here. Firſt, that euen in the Apoſtles time
there were daies deputed to the ſeruice of God, and ſo made holy and
different, though not by nature, yet by vſe and benediction, from other
profane or (as we cal them) worke-daies.

Secondly,
\MNote{Sunday made holy-day by the Apoſtles & the Churches authoritie.}
that the Apoſtles and faithful abrogated the Sabboth which was the
ſeuenth day, and made holy-day for it the next day following, being the
eight day in count from the creation: and that without al Scriptures, or
commandement of Chriſt that we read of, yea (which is more) not only
otherwiſe then was by the Law obſerued, but plainely otherwiſe then was
preſcribed by God himſelf in the ſecond commandement, yea and otherwiſe
then he ordained in the firſt creation, when he ſanctified preciſely the
Sabboth day, & not the day following. Such great power did Chriſt leaue
to his Church, and for ſuch cauſes gaue he the Holy Ghoſt to be reſident
in it, to guide it into al truths, euen ſuch as in the Scriptures are
not expreſſed.
\MNote{Other feaſts ordained by the Church.}
And if the Church had authoritie & inſpiration from God, to make Sunday
(being a worke-day before) an euerlaſting holy-day, and the Saturday,
that before was holy-day now a common worke-day: why may not the ſame
Church preſcribe & appoint the other holy feaſts of Eaſter, Whitſuntide,
Chriſtmas, and the reſt? For, the ſame warrant she hath for the one,
that she hath for the other.

Thirdly,
\MNote{As Saturday was in memorie of the creation ſo Sunday of Chriſt's
Reſurrection.}
it is to be noted that the cauſe of this change was, for that now we
Chriſtians eſteeming more our redemption, then our firſt creation, haue
the holy-day which was before for the remembrance of God's
accomplishment of the creation of things, now for the memorie of the
accomplishment of our redemption. Which therfore is kept vpon that day
on which our Lord roſe from death to life, which was the day after the
Sabboth, being called by the Iewes, \L{vna} or \L{prima
Sabbathi}, \Emph{the firſt of} or \Emph{after the Sabboth}.
\XRef{Mat.~28.}
\XRef{Act 20.}
\XRef{1.~Cor.~16.}
\MNote{The Church vſeth not the Heathenish names of daies, but, \L{Deis
Dominicus, feria, Sabatum}.}
Fourthly, it is to be marked that this holy-day by the Apoſtles
tradition alſo, was named \L{Dominus dies}, \Emph{our Lordes day}, or
\Emph{the Dominike}. Which is alſo an old Eccleſiaſtical word in our
language. For the name Sunday is a heathenish calling, as al other of
the week daies be in our language: ſome impoſed after the names of
planets, as in the Romans time: ſome by the name of certaine Idols that
the Saxons did worship, and to
\Fix{whith}{which}{obvious typo, fixed in other}
they dedicated theyr daies before they were Chriſtians. Which names the
Church vſeth not, but hath appointed to cal the firſt day, \Emph{the
Dominike}, after the Apoſtle here; the other by the name
of \Emph{Feries}, vntil the laſt of the weeke, which she calleth by the
old name, \Emph{Sabboth}, becauſe that was of God, and not by impoſition
of the Heathen. See the
\XRef{marginal Annotation Luc.~24,~1.}

Laſtly,
\MNote{God giueth greater grace at holy times of praier and faſting.}
obſerue, that God reuealeth ſuch great things to Prophets, rather vpon
holy-daies, & in times of contemplation, Sacrifice, and praier, then on
other profane daies. And therfore as S.~Peter
\XRef{(Act.~10.)}
had a reuelation at the ſixt houre of praier, and Zacharie
\XRef{(Luc.~1.)}
at the houre of incenſe, & Cornelius
\XRef{(Act.~10.)}
when he was at his praiers the ninth houre, ſo here, S.~Iohn noteth that
he had al the maruelous viſions vpon a Sunday.}
on the Dominical day, and heard behind me a great voice as it were of a
trompet \V ſaying: That which thou ſeeſt, write in a booke: and ſend to
the ſeuen Churches which are in Aſia, to Epheſus, and Smyrna, and
Pergamus, and Thiatira, and Sardis, and Philadelphia, and Laodicia. \V
\SNote{The 1.~General Viſiõ of the~7.
%%% !!! Cite?
according to S.~Ambroſe.}
And I turned, to ſee the voice that ſpake with me. And being turned I
ſaw ſeuen candleſticks of gold: \V and in the middes of the ſeuen
candeleſticks of gold, one
\SNote{It ſeemeth not to be Chriſt himſelf, but an Angel bearing
Chriſtes perſò; & vſing diuers ſpeaches proper to Chriſt.}
like to the Sonne of man,
\LNote{Veſted in a Prieſtly garment.}{He
\MNote{Prieſtly garments.}
appeared in a long garment or veſtement proper vnto Prieſts (for ſo the
word, \L{poderes}, doth ſignifie, as
\XRef{Sap.~18,~24.)}
and that was moſt agreable for him that repreſented the Perſon of Chriſt
the high Prieſt, and appeared to Iohn being a moſt holy Prieſt, and who
is ſpecially noted in the Eccleſiaſtical hiſtorie for his Prieſtly
garment called, \Emph{pecalon} or \Emph{lamina}.
\Cite{Euſeb. li.~3. hiſt. Eccl. cap.~25.}
&
\Cite{li.~5. c.~23.}}
veſted in a
\CNote{\XRef{Sap.~18,~24.}}
\TNote{\L{podere}}
prieſtly garment to the foot, & girded about neer to the paps with a
girdle of gold. \V And his head & haires were white, as white wool, and
as ſnow, and his eyes as the flame of fire. \V And his feet like to
latten, as in a burning fornace. And his voice as the voice of many
waters: \V and he had  in his right hand ſeuen ſtarres. And from his
mouth proceeded a ſharpe two-edged ſword: and his face, as the ſunne
ſhineth in his vertue. \V And when I had ſeen him, I fel at his feet as
dead. And he put his right hand vpon me, ſaying: Feare not.
\CNote{\XRef{Eſa.~41,~4.}
\XRef{44,~6.}}
I am the Firſt and the Laſt, \V and aliue, and was dead, and behold I am
liuing for euer and euer, and haue the keies of death and of hel. \V
Write therefore the things which thou haſt ſeen, and that are, and that
muſt be done after theſe: \V The Sacrament of the ſeuen ſtarres, which
thou haſt ſeen in my right hãd, and the ſeuen candleſticks of Gold.
\LNote{The ſeuen ſtarres.}{The
\MNote{The true religion manifeſt as the light on a candleſticke.}
Bishops are the ſtarres of the Church, as the Churches themſelues are
the golden candleſticks of the world: no doubt to ſignifie that Chriſt
preſerueth the truth only in and by the lawful Bishops and Catholike
Church, and that Chriſts truth is not to be ſought for in corners or
conuenticles of Heretikes, but at the Bishops hands, and
\CNote{\XRef{Mt.~5,~15.}}
vpon the candleſticke which shineth to al in the houſe.}
The ſeuen ſtarres, are
\LNote{The Angels of the Churches.}{The
\MNote{Angels Protectours.}
whole Church of Chriſt hath S.~Michael for her Keeper and Protectour,
and therfore keepeth his holy-day only by name, among al Angels. And as
earthly Kingdoms haue their ſpecial Angels Protectours, as we ſee in the
\XRef{10.~chapter of Daniel,}
ſo much more the particular Churches of Chriſtendom. See S.~Hierom
\Cite{in 34.~Ezech.}
But of thoſe Angels it is not here meant, as is manifeſt.
\MNote{Bishops and Prieſts are called Angels.}
And therfore Angels here muſt needs ſignifie the Prieſts or Bishops
ſpecially of the Churches here, & in them al the Gouernours of the whole
& of euery particular Church of Chriſtendom. They are called Angels, for
that they are God's meſſengers to vs, interpreters of his wil, our
keepers and directours in religion, our interceſſours, the cariers and
offerers of our praiers to him, and mediatours vnto him vnder
Chriſt. And for theſe cauſes and for their great dignitie they are here
and in
\CNote{\XRef{Mal.~2,~7.}}
other places of Scripture called Angels.}
the Angels of the ſeuen Churches. And
\SNote{S.~Irenæus alluding to this ſaith, \Emph{The Church euerywhere
preacheth the truth, & this is the ſeuen-fold candleſticke, bearing the
light of Chriſt &c.}
\Cite{Li.~5. aduerſ. hær.}}
the ſeuen candleſticks, are
\Fix{te}{the}{obvious typo, fixed in other}
ſeuen Churches.


\stopChapter


\stopcomponent


%%% Local Variables:
%%% mode: TeX
%%% eval: (long-s-mode)
%%% eval: (set-input-method "TeX")
%%% fill-column: 72
%%% eval: (auto-fill-mode)
%%% coding: utf-8-unix
%%% End:

