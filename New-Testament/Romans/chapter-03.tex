%%%%%%%%%%%%%%%%%%%%%%%%%%%%%%%%%%%%%%%%%%%%%%%%%%%%%%%%%%%%%%%%%
%%%%
%%%% The (original) Douay Rheims Bible 
%%%%
%%%% New Testament
%%%% Romans
%%%% Chapter 03
%%%%
%%%%%%%%%%%%%%%%%%%%%%%%%%%%%%%%%%%%%%%%%%%%%%%%%%%%%%%%%%%%%%%%%

%%% Latin checked by KK.



\startcomponent chapter-03


\project douay-rheims


%%% 2638
%%% o-2488
\startChapter[
  title={Chapter 3}
  ]

\Summary{He granteth that the Iewes did paſſe the Heathen Gentils, in
  Gods benefits, 9.~but not in their owne workes: concluding, that he
  hath shewed both Iew and Gentil to be ſinners: 18.~and therfore
  (inferring) that there muſt be ſome other way to Saluation indifferent
  to both; which is to beleeue in \Sc{Iesvs Christ}, that for his ſake
  their ſinnes may be forgiuen them.}

%%% !!! The CNotes in this chapter may be misplaced.
What preeminence then hath the Iew, or what is the profit of
circumciſion? \V Much by al meanes. Firſt, ſurely becauſe the wordes of
God were cõmitted to them. \V For what if certaine of them haue not
beleeued? Hath their incredulitie made the faith of God fruſtrate? \V
God forbid. But
\CNote{\XRef{Io.~3,~33.}}
God is true, &
\CNote{\XRef{Pſ.~115,~11.}}
\SNote{God only by nature is true, al mere men by nature may lie,
deceiue & be deceiued: yet God by his grace & ſpirit may & doth preſerue
the Apoſtles and principal Gouerners of his people & the Church and
Councels in al truth, though they were and are mere men.}
euery man a lyer, as it is written:
\CNote{\XRef{Pſ.~50,~6.}}
\Emph{That thou maieſt be iuſtified in thy wordes, and ouercome when
thou art iudged.} \V But
\LNote{If your iniquitie.}{No
\MNote{S.~Paules ſpeaches miſtaken of the wicked.}
maruel that many now-a-daies deduce falſe and deteſtable concluſions out
of this Apoſtles high and hard writings, ſeeing that
\CNote{\XRef{2.~Pet.~3.}}
S.~Peter noted it in his daies, and himſelf here confeſſeth that his
preaching & ſpeaches were then falſely miſconſtrued; as though he had
taught that the Iewes & Gentils il life & incredulity had been directly
the cauſe of Gods more mercie, & that therfore ſinne commeth of God to
the aduancement of his glorie, & conſequently that men might or ſhould
doe il, that good might enſue thereof. Which blaſphemous conſtructions
they tooke of theſe and the like wordes:
\CNote{\XRef{Ro.~5,~20.}}
\Emph{Where ſinne abounded, there did grace more abound}; and:
\CNote{\XRef{Ro.~3,~4.}}
\Emph{The Law entred in, that ſinne might abound}; and out of the
\XRef{Pſalme~50.}
\Emph{That thou maieſt be iuſtified in thy words, and ouercome when thou
art iudged.} As though he meant that men doe ſinne, to the end that God
may be iuſtified. And at al theſe & the like places of the Apoſtle
though forewarned by S.~Peter, and by the Apoſtles owne defence and
Proteſtation, that he neuer meant ſuch horrible things, yet the wicked
alſo of this time doe ſtumble and fal.
\MNote{The ſenſe of the places that ſoũd as if God cauſed ſinne.}
But the true meaning is in al ſuch places, that God can and doth, when
it pleaſeth him, conuert thoſe ſinnes which man committeth againſt him &
his commandments, to his glorie: though the ſinnes themſelues ſtand not
with his wil, intention, nor honour, but be directly againſt the ſame,
and therfore may not be committed that any good may fal. For, what good
ſoeuer accidẽtally falleth, it proceedeth not of the ſinne, but of God's
mercie that can pardon, and of his omnipotencie that can turne il to
good. And therfore againſt thoſe carnal interpretations, S.~Paul very
carefully & diligently giueth reaſon alſo in this place,
\XRef{v.~6.},
that it is impoſsible: becauſe God could not iuſtly puniſh any man, nor
ſit in iudgement at the later day for ſinne without plaine iniurie, if
either himſelf would haue ſinne committeth, or man might doe it to his
glorie. Therfore let al ſincere Readers of the Scriptures, and ſpecially
of S.~Paules writings, hold this for a certaintie, as the Apoſtles owne
defenſe (whatſoeuer he ſeeme to ſay hereafter ſounding in their ſenſe,
that ſinne commeth of God, or may therfore be comitteth that he may
worke good thereof) that the Apoſtle himſelf condemneth that ſenſe as
ſlanderous and blaſphemous.}
if our iniquitie commend the iuſtice of God, what ſhal we ſay? Is God
vniuſt that executeth wrath? (I ſpeake according to man) \V God forbid;
otherwiſe how ſhal God iudge this world? \V For if the veritie of God
hath abounded in my lie, vnto his glorie, why am I alſo yet iudged as a
ſinner, \V and not (as we are blaſphemed, and as ſome report vs to ſay)
let vs
%%% o-2489
doe euil, that there may come good? whoſe damnation is iuſt.

\V What then? doe we excel them? No, not ſo. For we haue argued the
Iewes and the Greeks, al to be vnder ſinne; \V as it is written:
\CNote{\XRef{Pſs.~13,~1.}}
\Emph{That
\LNote{Not any iuſt.}{Theſe
\MNote{How it is ſaid: \Emph{none iuſt}.}
general ſpeaches, that both Iew and Gentile be in ſinne, and none at al
iuſt, are not ſo to be taken, that none in neither ſort were euer
good: the Scriptures expreſly ſaying that
\CNote{\XRef{Iob.~1.}}
Iob,
\CNote{\XRef{Luc.~1.}}
Zacharie, Eliſabeth, and ſuch like, were iuſt before God; & it were
blaſphemie to ſay that theſe words alleaged out of the
\XRef{13.~Pſalme}
were meant in Chriſtes mother, in S.~Iohn the Baptiſt, in the Apoſtles
&c. For, this only is the ſenſe: that neither by the Law of nature, nor
Law of Moyſes, could any man be iuſt or auoid ſuch ſinnes as here be
rekened, but by faith and the grace of God, by which there were a
number in al Ages (ſpecially among the Iewes) that were iuſt and holy,
whom theſe words touch not, being ſpoken only to the multitude of the
wicked, which the Prophet maketh as it were a ſeueral body conſpiring
againſt Chriſt, and perſecuting the iuſt and godly of which il companie
he ſaith, that none was iuſt nor feared God.}
there is not any man iuſt, \V there is not that vnderſtandeth, there is
not that ſeeketh after God.} \V
\CNote{\XRef{Pſ.~52,~3.}}
\Emph{Al haue declined, they are become vnprofitable together: there is
not that doeth good, there is not ſo much as one.} \V
\CNote{\XRef{Pſ.~5,~11.}}
\Emph{Their throte is an open ſepulchre, with their tongues they
dealt deceitfully. The venim
\TNote{\L{Aſpidum}. A kind of litle ſerpents.}
of aſpes vnder their lippes.} \V 
\CNote{\XRef{Pſ.~139,~4.}}
\Emph{Whoſe mouth is ful of malediction and bitternes:} \V
\CNote{\XRef{Pſ.~9,~7.}}
\Emph{Their feet ſwift to ſheed bloud.}  \V
\CNote{\XRef{Eſ.~19,~7.}}
\Emph{Deſtruction & infelicitie in their waies:} \V
\CNote{\XRef{Pro.~1,~16.}}
\Emph{and the way of peace they haue not knowen.} \V
\CNote{\XRef{Pſ.~35,~2.}}
\Emph{There is no feare of God before their eyes.} \V And we know that
whatſoeuer the Law ſpeaketh, to them it ſpeaketh that are in the Law;
that euery mouth may be ſtopped, & al the world may be made ſubiect to
God: \V becauſe
\CNote{\XRef{Gal.~2,~16.}}
\LNote{By the workes of the Law.}{S.~Hierom
\MNote{No workes auaile without faith & grace.}
and S.~Chryſoſtom expound this of the ceremonial workes only: and in
that ſenſe the Apoſtle ſpecially proſecuteth this propoſition in his
\XRef{Epiſtle to the Galatians}.
But it is true alſo of al man's moral workes done without faith & the grace
of God; which can not be acceptable or auailable in God's ſight, to
iuſtifie any man. And ſo S.~Auguſtine taketh it
\Cite{De Sp. & lit. c.~8. to.~3.}}
by the workes of the Law no fleſh ſhal be iuſtified before him. For by
the Law is the knowledge of ſinne.

\V But now without the Law
\LNote{Iuſtice of God.}{Beware
\MNote{The Heretikes phantaſtical or imputatiue iuſtice.}
of the wicked and vaine commentarie of the Caluiniſtes, gloſsing, the
iuſtice of God to be that which is reſident in Chriſt, apprehended by
our faith; and ſo that imputed to vs which we indeed haue not. Wherein
at once they haue forged themſelues againſt God's manifeſt word, a new
no iuſtice, a phantaſtical apprehenſion of that which is not, a falſe
faith and vntrue imputation. Whereas the iuſtice of God here, is that
wherewith he indoweth a man at his firſt conuerſion, and is now in a
man, and therfore man's iuſtice: but yet God's iuſtice alſo, becauſe it
is of God. Of this iuſtice in vs, whereby we be truely iuſtified and
indeed made iuſt, S.~Auguſtine ſpeaketh thus:
\CNote{\Cite{De pre. mer. li.~1. c.~9,~10.}}
\MNote{True inhærent iuſtice.}
\Emph{The grace of Chriſt doth worke our illumination and iuſtification
inwardly alſo.} And againe: 
\Emph{He giueth to the faithful the moſt ſecret grace of his Spirit,
which ſecretly he powreth into infants alſo.} And againe:
\Emph{They are iuſtified in Chriſt that
beleeue in him through the ſecret communication and inſpiration of
ſpiritual grace, whereby euery one leaneth to our Lord.} And
againe: \Emph{He maketh iuſt renewing by the Spirit, and regeneration by
grace.}}
the iuſtice of God is manifeſted; teſtified by the Law and the
Prophets. \V And the iuſtice of God by faith of \Sc{Iesvs Christ}, vnto
al and vpon al that
\SNote{To beleeue in him, here compriſeth not only the act of faith, but
of hope & charitie, as the Apoſtle explicateth himſelf.
\XRef{Gal.~5,~6.}}
beleeue in him. For there is no diſtinction. \V For al haue ſinned; and
doe need the glorie of God. \V Iuſtified
\SNote{No man atteineth his firſt iuſtification by the merits either of
his faith or workes, but merely by Chriſtes grace and mercie: though his
faith & workes proceeding of grace be diſpoſitions and preparations
thereunto.}
gratis by his grace, by the redemption that is in
%%% 2639
\Sc{Christ Iesvs}, \V whom God hath propoſed a
\Var{propitiation,}{propitiatour.}
by faith in his bloud, to the ſhewing of his iuſtice, for the remiſſion
of former ſinnes, \V in the toleration of God, to the ſhewing of his
iuſtice in this time: that he may be iuſt, and iuſtifying him that is of
the faith of \Sc{Iesvs Christ}.

\V Where is then thy boaſting? it is excluded. By what law? of deeds?
No, but by the law of faith. \V For we account a man to be iuſtified
\LNote{By faith, without workes.}{This is the place whereupon the
Proteſtants gather falſly their only faith, and which they commonly
auouch, as though the Apoſtle ſaid, that only faith doth iuſtifie.
\MNote{What works are excluded from iuſtification.}
Where
he both in wordes and meaning excepteth only the workes of the Law done
without Chriſt before our conuerſion: neither excluding the Sacraments
of Baptiſme or Penance, nor hope and charitie, or other Chriſtian
vertues; al which be the iuſtice of faith. As the good workes proceeding
thereof, be likewiſe the law and iuſtice of faith. Al which the
Aduerſaries would exclude by foiſting in the terme, only. Of which kind
of men S.~Auguſtine vpon this place ſaith thus:
\CNote{\Cite{de grat. & lib. arb. c.~7.}}
\Emph{Men not vnderſtanding that which the Apoſtle ſaith, (we count a
man to be iuſtified by faith without the workes of the Law) did thinke
that he ſaid, faith would ſuffice a man though he liued il and had no
good workes. Which God forbid the veſſel of election should thinke: who
in a certaine place after he had ſaid,
\CNote{\XRef{Gal.~1.}}
In \Sc{Christ Iesvs} neither circumciſion nor prepuce auaileth any whit,
he ſtraight added, but faith which worketh by loue.}}
by faith without the works of the Law. \V Is he God of the Iewes only?
is he not alſo of the Gentils? Yes of the Gentils alſo. \V For it is one
God, that iuſtifieth circumciſion by faith, and prepuce by faith. \V Doe
we then deſtroy the Law by faith? God forbid, but we doe eſtabliſh the
Law.


\stopChapter


\stopcomponent


%%% Local Variables:
%%% mode: TeX
%%% eval: (long-s-mode)
%%% eval: (set-input-method "TeX")
%%% fill-column: 72
%%% eval: (auto-fill-mode)
%%% coding: utf-8-unix
%%% End:

