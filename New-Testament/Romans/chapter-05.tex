%%%%%%%%%%%%%%%%%%%%%%%%%%%%%%%%%%%%%%%%%%%%%%%%%%%%%%%%%%%%%%%%%
%%%%
%%%% The (original) Douay Rheims Bible 
%%%%
%%%% New Testament
%%%% Romans
%%%% Chapter 05
%%%%
%%%%%%%%%%%%%%%%%%%%%%%%%%%%%%%%%%%%%%%%%%%%%%%%%%%%%%%%%%%%%%%%%

%%% Latin checked by KK.



\startcomponent chapter-05


\project douay-rheims


%%% 2643
%%% o-2493
\startChapter[
  title={Chapter 5}
  ]

\Summary{Hauing therfore through faith in Chriſt obteined the beginning,
  he sheweth what great cauſe we haue to hope for the
  accomplishment. 12.~And then he proceedeth in his arguing, and sheweth
  that as by one, al were made ſinners, ſo by one, al muſt be made iuſt.}

Being iuſtified therfore by faith,
\LNote{Let vs haue.}{Whether
\MNote{Againſt the Heretikes ſpecial faith and ſecuritie.}
we read, 
\TNote{\G{ἔχομεν}}
\Emph{Let vs haue peace}, as diuerſe alſo of the Greeke Doctours
(Chryſoſt. Orig. Theodor. Oecum. Theophyl.) doe, or
\TNote{\G{ἔχομιν}}
\Emph{We haue peace}; it maketh nothing for the vaine ſecuritie and
infallible certaintie which our Aduerſaries ſay euery man ought to haue
vpon his preſumed iuſtification by faith, that himſelf is in God's
fauour, & ſure to be ſaued: \Emph{peace towards God}, being here nothing
els, but the ſincere reſt, tranquilitie, and comfort of mind and
conſcience, vpon the hope he hath, that he is reconciled to God. Sure it
is that the Catholike faith, by which and none other men be iuſtified,
neither teacheth nor breedeth any ſuch ſecuritie of ſaluation. And
therfore they haue made to themſelues another faith which they
cal \L{Fiduciam}, quite without the compaſſe of the Creed and
Scriptures.}
let vs haue peace toward God by our Lord \Sc{Iesvs Christ}; \V by whom
alſo we haue
\LNote{Acceſſe through faith.}{Iuſtification,
\MNote{Iuſtification attributed much to faith as to the fundation.}
implieth al grace and vertues receiued by Chriſt's merits; but the
entrance & acceſſe to this grace & happy ſtate is by faith: becauſe
faith is the ground and firſt foundation to build on, and port to enter
into the reſt. Which is the cauſe that our iuſtification is attributed
to faith namely in this Epiſtle, though faith itſelf be of grace alſo.}
acceſſe through faith into this grace wherin we ſtand, and glorie,
\SNote{Chriſtiã men doe not vaunt themſelues of the certaintie of their
ſaluation, but glorie in the hope thereof only, which hope is here
inſinuated to be giuen in our iuſtificatiõ, & afterward to be cõfirmed
by probatiõ in tribulation.}
in the hope of the glorie of the ſonnes of God. \V And not only this;
but alſo we glorie in tribulations, knowing that tribulation
worketh patience: \V and patience, probation; and
\LNote{Probation, hope.}{This
\MNote{Our hope is ſtrengthned by wel-doing.}
refelleth the errour alſo of the Proteſtants, that would haue our hope
to hold only on God's promiſes, and not a-whit on our doings. Where we
ſee that it ſtandeth (and is ſtrengthned alſo) vpon patience and
conſtancie, and good probation and trail of our ſelues in aduerſities:
and that ſo grounded vpon God's promiſes and our owne doings, it neuer
confoundeth.}
probation, hope; \V and hope confoundeth not: becauſe
\LNote{Charitie is powred.}{Charitie
\MNote{Charitie is a qualitie in vs.}
alſo is giuen vs in our firſt iuſtification, and not only imputed vnto
vs, but indeed inwardly powred into our harts by the Holy Ghoſt, who
with and in his guifts & graces is beſtowed vpon vs. For this Charitie
of God is not that which is in God, but that which he giueth vs, as
S.~Auguſtine expoundeth it.
\Cite{Li. de Sp. & lit. c.~32.}
Who referreth this place alſo to the grace of God giuen in the Sacrament
of Confirmation.
\Cite{de Bapt. cont. Donat. li.~3. c.~16.}}
the charitie of God is powred forth in our harts, by the Holy Ghoſt
which is giuen vs. \V For why did Chriſt, when we as yet were
\SNote{The Heretikes falſely trãſlate \Emph{of no ſtrength}, to take
away al free-wil.
\Cite{No. Teſt. 1580.}}
\TNote{\G{ἀσθενῶν}}
weake, according to the time die for the impious? \V For, ſcarſe for a
iuſt man doth any die: for perhaps for a good man durſt ſome man die. \V But
God commendeth his charitie in vs: becauſe, when as yet we were ſinners,
Chriſt died for vs. \V Much more therfore now being iuſtified in his
bloud, shal we be ſaued from wrath
%%% o-2494
by him. \V For if, when we were enemies, we were reconciled to God by
the death of his Sonne; much more being reconciled, shal we be ſaued in
the life of him. \V And not only this; but alſo we glorie in God through
our Lord \Sc{Iesvs Christ}, by whom now we haue receiued reconciliation.

\V Therfore, as
\LNote{By one man ſinne entred.}{By
\CNote{\Cite{Conc. Tri. feſſ. 5. decr. de pec. orig.}}
\MNote{Al by Adam borne in original ſinne.}
this place ſpecially the Church of God defendeth and proueth againſt the
old Heretikes the Pelagians, that denied children to haue any original
ſinne, or to be baptized for the remiſsion thereof; that in and by Adam
al be conceiued, borne, and conſtituted ſinners. Which no leſſe maketh
againſt the Caluiniſts alſo, that affirme Chriſtian mens children to be
holy from their mothers womb. And the ſame reaſon which S.~Auguſtine
deduceth
\Cite{(li.~1. c.~8.~9. de pec. meritis.)}
out of this text, to proue againſt the ſaid Pelagians, that the Apoſtle
meaneth not of the general imitation of Adam in actual ſinnes, ſerueth
againſt Eraſmus and others, inclining rather to that new expoſition,
then to the Churches and Fathers graue iudgement heerin.
\Cite{conc. Mileuitanum c.~2.}}
by one man ſinne entred into this world, and by ſinne, death; and ſo vnto
al men death did paſſe, in which al ſinned. \V For euen vnto the Law
ſinne was in the world: but ſinne was not imputed, when the Law was
not. \V But death reigned from Adam
\LNote{Vnto Moyſes.}{Euen in the time of the Law of nature, when men
knew not ſinne, and therfore it could not by man's iudgement be imputed;
and in the time of Moyſes Law, when the commandement taught them to know
it, but gaue them not ſtrength nor grace to auoid it, ſinne did reigne,
and thereupon death and damnation, euen til Moyſes \Emph{incluſiue}, that
is to ſay, euen til the end of his Law.
\MNote{Chriſt only not conceiued in ſinne, & (as it is thought) our
B.~Lady.}
And that not in them only which actually ſinned, as Adam did, but
infants which neuer did actually offend, but only were borne & conceiued
in ſinne, that is to ſay, hauing their natures defiled, deſtitute of
iuſtice, and auerted from God in Adam, and by their deſcent from him:
Chriſt only excepted, being conceiued without man's ſeed, and his Mother
for his honour and by his ſpecial protection (as many godly deuout men
iudge) preſerued from the ſame.}
vnto Moyſes, euen on them alſo that ſinned not after the ſimilitude of
the preuarication of Adam, who is a figure of him to come. \V But not as
the offence, ſo alſo the guift. For if by the offence of one, many died;
much more the grace of God and the guift, in the grace of one
man \Sc{Iesvs Christ}, hath abounded vpon many. \V And not as by one
\Var{ſinne,}{ſinner}
ſo alſo the guift. For iudgement indeed is of one, to condemnation: but
grace is of many offences, to iuſtification. \V For in the offence of
one, death reigned by one; much more they that receiue the aboundance of
grace and of donation & of iuſtice, shal reigne in life by
one, \Sc{Iesvs Christ}. \V Therfore as by the offence of one, vnto al
men to condemnation; ſo alſo by the iuſtice of one, vnto al men to
iuſtification of life. \V For as by the diſobedience of one man, many
were made ſinners; ſo alſo by the obedience of one, many
\SNote{Here we may ſee againſt the Heretikes, that they which be borne
of Chriſt, and iuſtified by him, be made & conſtituted iuſt indeed, & not
by imputation only: as al that be borne of Adam be vniuſt and ſinners in
truth, & not by imputation.}
shal be made iuſt. \V But the Law entred in,
\LNote{That ſinne might abound.}{That,
\MNote{The Law did not cauſe more ſinne, though that were the ſequele
therof.}
here hath not the ſignification of cauſalitie, as though the Law were
giuen for that cauſe to make ſinne more abound: but it noteth the ſequele,
becauſe that followed thereof, and ſo it came to paſſe that by the
prohibition of ſinne, ſinne increaſed: by occaſion wherof the force of
Chriſt's grace is more amply and aboundantly beſtowed in the new
Teſtament.}
that ſinne might abound. And where ſinne abounded, grace did more
abound. \V That as ſinne reigned to death; ſo alſo grace may reigne by
iuſtice to life euerlaſting, through \Sc{Iesvs Christ} our Lord.


\stopChapter


\stopcomponent


%%% Local Variables:
%%% mode: TeX
%%% eval: (long-s-mode)
%%% eval: (set-input-method "TeX")
%%% fill-column: 72
%%% eval: (auto-fill-mode)
%%% coding: utf-8-unix
%%% End:

