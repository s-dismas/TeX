%%%%%%%%%%%%%%%%%%%%%%%%%%%%%%%%%%%%%%%%%%%%%%%%%%%%%%%%%%%%%%%%%
%%%%
%%%% The (original) Douay Rheims Bible 
%%%%
%%%% New Testament
%%%% Epistles
%%%% Romans
%%%% Chapter 13
%%%%
%%%%%%%%%%%%%%%%%%%%%%%%%%%%%%%%%%%%%%%%%%%%%%%%%%%%%%%%%%%%%%%%%

%%% Latin checked by KK.




\startcomponent chapter-13


\project douay-rheims


%%% 2662
%%% o-2514
\startChapter[
  title={Chapter 13}
  ]

\Summary{To yeald obedience and al other duties vnto Poteſtates: 8.~to
  loue their neighbour which is the fulfilling of the Law: 11.~and
  ſpecially to conſider, that now being the time of grace we muſt doe
  nothing that may not beſeeme day-light.}

Let
\CNote{\XRef{Tit.~3,~1.}
\XRef{1.~Pet.~2,~13.}}
\LNote{Euery ſoule be ſubiect.}{Becauſe the Apoſtles preached libertie
by Chriſt from the yoke of the Law and ſeruitude of ſinne, and gaue al
the faithful both example and commandement to obey God more then men,
and withal euer charged them expreſly to be obedient and ſubiect to
their Prelates as to them which had cure of their ſoules and were by the
Holy Ghoſt placed ouer the Church of God: there were many in thoſe daies
newly conuerted that thought themſelues free from al temporal
Poteſtates, carnal Lords, and humane creatures or powers: wherupon the
bondman tooke himſelf to be looſe from his ſeruitude, the ſubiect from
his Soueraigne, were he Emperour, King, Duke, or what other ſecular
Magiſtrate ſoeuer; ſpecially the Princes of thoſe daies being Heathens
and perſecutours of the Apoſtles, and of Chriſtes religion.
For which cauſe and for that the Apoſtles were vntruly charged of their
Aduerſaries, that they withdrew men from order and obedience to Ciuil
lawes and Officers;
\MNote{Obedience to temporal Rulers, & in what caſes.}
S.~Paul here (as S.~Peter doth
\XRef{1.~Chap.~2.)}
cleereth himſelf, and expreſly chargeth euery man to be ſubiect to his
temporal Prince and Superiour: Not euery man to al that be in Office or
Superioritie, but euery one to him whom God hath put in authoritie ouer
him, by that he is his Maiſter, Lord, King, or ſuch like. Neither to
them in matters of religion or regiment of their ſoules (for moſt part
were Pagans, whom the Apoſtle could not wil men to obey in matters of
faith) but to them in ſuch things only as concerne the publike peace &
Policie, & what other cauſes ſoeuer conſiſt with God's holy wil and
ordinance. For
\CNote{\XRef{Act.~4,~19.}
\XRef{5,~29.}}
againſt God no power may be obeied.}
euery ſoul be ſubiect to higher powers, for there is
\LNote{No power but of God.}{S.~Chryſoſtome
\CNote{\Cite{Chryſ. in ep.~Ro. ho.~23.}}
\MNote{In what ſenſe, al power or ſuperioritie is of God.}
here noteth, that power, rule & Superioritie, is God's ordinance, but
not eſtſoones al Princes; becauſe many may vſurp, who reigne by his
permiſsion only, and not by his appointment: nor al actions that euery
one doeth in and by his ſoueraigne power; as Iulian's apoſtaſie and
affliction of Catholikes, Pharao's tyrannical oppreſsion of the
Iſraelites, Achab's perſecution of the Prophets, Nero's executing of the
Apoſtles, Herod's and Pilat's condemning of Chriſt: al which things God
permitted them, by the abuſe of their power to accomplish, and not being
the cauſe of their euil doings, turned and ordered the ſame to good
effects.
\Cite{S.~Auguſt. tract.~112 in Ioan.}
\Cite{S.~Tho. 1.~p. q.~19. a.~9.}}
no power but of God. And thoſe that are, of God are ordeined. \V
Therfore he that reſiſteth the power, reſiſteth the ordinance of
God. And
\LNote{They that reſiſt.}{Whoſoeuer
\MNote{In things lawfully commanded it is mortal ſinne not to obey our
Superiours.}
reſiſteth or obeieth not his lawful Superiour in thoſe cauſes wherin he
is ſubiect vnto him, withſtandeth God's appointment, & ſinneth deadly,
and is worthy to be punished both in this world by his Superiour, and by
God in the next life. For in temporal gouernement and cauſes, the
Chriſtians were bound in conſcience to obey their Heathen
\Fix{Emperous:}{Emperours:}{obvious typo, fixed in other}
though on the other ſide, they were bound vnder paine of damnation to
obey their Apoſtles and Prelates, and not to obey their Kings or
Emperours in matters of religion. Whereby it is cleere that when we be
commanded to obey our Superiours, it is menat alwaies and only in ſuch
things as they may lawfully command, and in reſpect of ſuch matters
wherein they be our Superiours.}
they that reſiſt, purchaſe to themſelues damnation. \V For Princes are
no feare to the good worke, but to the euil. But wilt thou not feare
the power? Doe good: and thou ſhalt haue praiſe of the ſame. \V For he
is God's Miniſter vnto thee for good. But if thou doe euil, feare; for
he
\LNote{Beareth not the ſword.}{That
\MNote{The Apoſtle ſpeaketh of tẽporal powers.}
the Apoſtle meaneth here ſpecially of temporal powers, we may ſee by the
ſword, tribute, & external compulſion, which he here attributeth to
them. And the Chriſtian men then had no doubt whether they ſhould obey
their Spiritual powers. But now the diſeaſe is cleane contrarie. For al
is giuen to the ſecular power, and nothing to the ſpiritual which
expreſly is ordained by Chriſt and the Holy Ghoſt: and al the faithful
are commanded to be ſubiect therunto, as to Chriſt's owne word and
wil.
\MNote{Hereſies againſt rule and Superioritie.}
There were Heretikes called \Emph{Begards}, that tooke away al rule and
Superioritie. The Wicklifiſts would obey not Prince nor Prelate, if he
were once in deadly ſinne. The Proteſtants of our time (as we may ſee in
al Countries where the ſecular ſword is drawen againſt their Sects) care
neither for the one nor for the other, though they extol only the
ſecular when it maketh for them.
\MNote{The obedience of Catholikes both to Spiritual & temporal
Superiours.}
The Catholikes only moſt humbly obey both, euen
according to God's ordinance, the one in temporal cauſes, and the other
in Spiritual: in which order both theſe States haue bleſſedly flouriſhed
in al Chriſtian countries euer ſince Chriſtes time, and it is the very
way to preſerue both, as one day al the world ſhal confeſſe with vs.}
beareth
%%% 2663
not the ſword without cauſe. For he is God's Miniſter: a reuenger vnto
wrath, to him that doeth euil. \V Therfore be ſubiect of neceſsitie, not
only for wrath, but alſo for conſcience ſake. \V For therfore
\LNote{You giue tributes.}{Though
\CNote{\Cite{Hiero. in Mat.~17.}}
euery man ought to be ready to ſerue his temporal Prince with his goods,
by tributes or what other lawful taxes and ſubſidies ſoeuer; yet they
may exempt by priuiledges whom they thinke good.
\MNote{The Clergie exempted frõ tribute.}
As in al countries Chriſtian: Prieſts for the honour of Chriſt, whoſe
Miniſters they be, haue by the grants & ancient charters of Kings been
excepted and exempted. Notwithſtanding they were neuer vnready to ſerue
voluntarily their Soueraigne, in al common cauſes, with whatſoeuer they
had. See
\XRef{Annot. in Mat.~17,~26.}}
you giue tributes alſo. For they are the Miniſters of God, ſeruing vnto
this purpoſe. \V Render therfore to al men their dew:
\CNote{\XRef{Mt.~22,~21.}}
to whom tribute, tribute: to whom cuſtom, cuſtom: to whom feare, feare:
to whom honour, honour. \V Owe
%%% o-2515
no man any thing: but that you loue one another. For he that loueth his
neighbour, hath
\SNote{Here we learne that the Law may be & is fulfilled by loue in this
life: againſt the Aduerſaries ſaying it is impoſsible to keep the
commandements.}
fulfilled the law. \V For,
\CNote{\XRef{Exo.~20,~13.}}
\Emph{Thou shalt not commit aduoutrie, Thou shalt not kil, Thou shalt
not ſteale, Thou shalt not beare falſe witnes, Thou shalt not couet,}
and if there be any other commandement, it is compriſed in this word,
\CNote{\XRef{Leu.~19,~18.}}
\Emph{Thou shalt loue thy neighbour as thy ſelf.} \V The loue of thy
neighbour, worketh no euil. Loue therfore is the fulneſſe of the Law. \V
And that knowing the ſeaſon, that it is now the houre for vs to riſe frõ
ſleep. For now our ſaluation is neerer then whẽ we beleeued. \V The night is
paſſed, and the day is at hand. Let vs therfore caſt off the workes of
\Fix{darneſſe,}{darkneſſe,}{obvious typo, fixed in other}
& doe on the armour of light. \V As in the day let vs walke
honeſtly
\LNote{Not in banketings.}{This
\MNote{S.~Auguſtines conuerſion.}
was the very place which S.~Auguſtine, that glorious Doctour, was by a
voice from Heauen directed vnto, at his firſt miraculous and happy
conuerſion, not only to the Catholike faith, but alſo to perpetual
continencie, by this voice comming from Heauen, \L{Tolle, lege: Tolle,
lege}, Take vp and read, take vp and read, as himſelf telleth.
\Cite{li.~8. Confeſ. c.~11.}}
not in banketings and drunkennes, not in chamberings and impudicities,
not in contention and emulation: \V but doe ye on our Lord \Sc{Iesvs
Christ}, and make not prouiſion for the fleſh in concupiſcences.


\stopChapter


\stopcomponent


%%% Local Variables:
%%% mode: TeX
%%% eval: (long-s-mode)
%%% eval: (set-input-method "TeX")
%%% fill-column: 72
%%% eval: (auto-fill-mode)
%%% coding: utf-8-unix
%%% End:

