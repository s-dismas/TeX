%%%%%%%%%%%%%%%%%%%%%%%%%%%%%%%%%%%%%%%%%%%%%%%%%%%%%%%%%%%%%%%%%
%%%%
%%%% The (original) Douay Rheims Bible 
%%%%
%%%% New Testament
%%%% Romans
%%%% Argument
%%%%
%%%%%%%%%%%%%%%%%%%%%%%%%%%%%%%%%%%%%%%%%%%%%%%%%%%%%%%%%%%%%%%%%




\startcomponent argument


\project douay-rheims


%%% 2628
%%% o-2479
\startArgument[
  title={\Sc{The Time When the Epistle to the Romanes was written, and
  the Argument Therof.}},
  marking={Argument of Romans}
  ]

The hiſtorie of S.~Paul, vntil he came to Rome, S.~Luke in the Actes of
the Apoſtles wrote exactly: and though without any mention of his
Epiſtles, yet certaine it is, that ſome of them he wrote before he came
there, to wit, the two vnto the Corinthians, and this to the Romanes: &
(
\CNote{\XRef{Gal.~2.}}
as it ſeemeth) before them al, the Epiſtle to the Galatians. Wherein yet
becauſe he maketh mention of the foureteenth yeare after his conuerſion,
it appeareth, that he preached ſo long without any writing.

And this order may thus briefely be gathered. Firſt he preached to the
Galatians
\XRef{Act.~16.}
\Emph{and paſsing through Phrygia and the countrey of Galatia}. Whereof
he maketh mention himſelfe alſo,
\XRef{Gal.~1.}
\Emph{We euangelized to you.}
\XRef{Gal.~4.}
\Emph{I euangelized to you heretofore.} After which the falſe Apoſtles
came and perſuaded
%%% 2629
them to receiue Circumciſion. Whereupon he ſaith
\XRef{Gal.~1.}
\Emph{I maruel that thus ſo ſoone you are transferred from him that called you
to the grace of Chriſt, vnto another Ghoſpel}: and wisheth therfore
%%% o-2480
\XRef{Gal.~4.}
ſaying: \Emph{And I would I were with you now.} And accordingly he came
vnto them afterward, as we read
\XRef{Act.~18.}
\Emph{Walking in order through the countrie of Galatia and Phrygia,
confirming al the Diſciples.} At which time alſo it ſeemeth, that he
tooke order with them about thoſe contributions to help the need of the
Chriſtians in Hieruſalem, whereof he ſpeaketh
\XRef{1.~Cor.~16}:
\Emph{And concerning the collections that are made for the Saints, as I
haue ordeined to the Churches of Galatia, ſo doe you alſo.} By which
words alſo it is euident, that the Corinthians had not as then made their
gathering. But when he wrote the Second to them (where in the
\XRef{11.~chapter}
he maketh mention of 14.~yeares, not only after his Conuerſion, as to
the Galatians, but alſo after his Rapte, which ſeemeth to haue been when
he was at Hieruſalem
\XRef{Act.~9.}
foure yeares after his conuerſion, \Emph{in a trance}, as he calleth it,
\XRef{Act.~22.~17.}) then were they redie. For ſo he ſaith
\XRef{2.~Cor.~8.}
\Emph{You haue begun from the yeare paſt;} and
\XRef{2.~Co.~9.}
\Emph{For the which I doe glorie of you to the Macedonians; that alſo
Achaia is ready from the yeare paſt}: Howbeit it followeth
there: \Emph{But I haue ſent the Brethren, that (as I haue ſaid) you may
be ready: leſt when the Macedonians come with me, and find you vnready,
we be aſhamed.} But when he wrote to the Romanes, then was he now come
to Corinth for the purpoſe, and had receiued their contribution, and was
readie to goe with it vnto Hieruſalem. For ſo he ſaieth
\XRef{Rom.~15.}
\Emph{Now therfore I wil goe vnto Hieruſalem to miniſter to the
Saints. For Macedonia and Achaia haue liked wel to make ſome
contribution vpon the poore Saints that are in Hieruſalem.}

So
\MNote{The argument of the Epiſtle to the Romanes.}
then, the Epiſtle to the Romanes was not the firſt that he wrote. But
yet it is
\CNote{\Cite{Epiph. Hær. 42. Marcioan.}
\Cite{Aug. in Expoſ. incho. Ep. ad Rom.}}
and alwaies was ſet firſt, becauſe of the primacie of that Church. For
which cauſe alſo he handleth in it ſuch matters as perteined not to them
alone, but to the vniuerſal Church, and ſpecially to al the Gentils: to
wit, the very frame (as it were) of the Church of Chriſt. \L{Tanquam
enim
\CNote{\XRef{2.~Cor.~5.}}
pro ipſo Domino legatione fungens, hoc eſt, pro
\CNote{\XRef{Epheſ.~2.}}
lapide angulari, vtrumque populum tam, ex Iudæis quam ex Gentibus
connectit in Chriſto per vinculum gratiæ.} So ſaith S.~Auguſtin, giuing
vs briefly the argument; in english thus: \Emph{As being a Legate for
our Lord himſelf, that is, for the corner-ſtone, he knitteth together in
Chriſt by the band of Grace, both peoples, as wel of the Iewes as of the
Gentils.} Shewing, that neither of them had in their Gentilitie or
Iudaiſme any workes to brag of, or to chalenge to themſelues
iuſtification or ſaluation thereby, but rather ſinnes they had to be
ſorie for, and to humble themſelues to the faith of Chriſt, that ſo they
might haue remiſsion of them, and ſtrength to doe meritorious workes
afterward. In which ſort becauſe the Gentils did humble themſelues,
therefor had they found mercy though they neuer wiſt of the Law of
Moyſes: but the Iewes, becauſe they ſtood vpon their owne workes, which
they did by their owne ſtrength, with the knowledge of the Law (being
therfore alſo called
\MNote{The workes of the Law.}
\Emph{the workes of the Law},) & ſo would not humble themſelues to
beleeue in Chriſt crucified, they miſſed of mercy, and became reprobate,
excepting a few \L{Reliquæ} that God of his goodnes had reſerued to
himſelf. Howbeit in the end, when the fulnes of the Gentils is come into
the Church, then shal the fulnes of the Iewes alſo open their eyes,
acknowledge their errour, and ſubmit themſelues to Chriſt and his
Church, in like manner. In the meane
%%% 2630
time, thoſe that haue found the grace to be Chriſtians, he exhorteth to
perſeuerance (as it was ſpecially needful in thoſe times of
perſecutions) and to lead their whole life now after Baptiſme in good
workes: and to be careful of vnitie, bearing therefore one with another,
both Iew and Gentil, al that they
%%% o-2481
may, and giuing no offence to them that are weake. Thus he diſputeth,
and thus he exhorteth through the whole Epiſtle: though, if we wil
diuide it by that which is principal in each part, we may ſay, that vnto
the
\XRef{12.~chapter}
is his diſputation: and from thence to the end, his exhortation.

Now in theſe points of faith, and in al others (as alſo in example of
life) the commendation that he giueth to the Church of Rome, is much to
be noted.
\CNote{\XRef{Rom.~1.}}
%%% !!! Where does this go? Are the others in the right places?
%%% \CNote{\XRef{Rom.~16.}}
\Emph{Your faith is renowmed in the whole world; and your obedience is
published into euery place. I reioyce therfore in you.} And againe:
\CNote{\XRef{Rom.~6.}}
\Emph{You haue obeied from the hart vnto that forme of doctrine, which
had been deliuered to you.} And thereupon againe:
\CNote{\XRef{Rom.~16.}}
\Emph{I deſire you, Brethren, to marke them that make diſſenſions and
ſcandals contrarie to the doctrine which you haue learned, and auoid
them. For ſuch doe not ſerue Chriſt our Lord, but their owne belly: and
by ſweet ſpeaches and benedictions ſeduce the harts of innocents.}
Therfore to shun Luther and Caluin, and al their crewes, we haue iuſt
reaſon and good warrant. They make diſſenſions and ſcandals againſt the
doctrine of the Romane Church. Let no man therefore be ſeduced by their
ſugred wordes.


\stopArgument


\stopcomponent


%%% Local Variables:
%%% mode: TeX
%%% eval: (long-s-mode)
%%% eval: (set-input-method "TeX")
%%% fill-column: 72
%%% eval: (auto-fill-mode)
%%% coding: utf-8-unix
%%% End:
