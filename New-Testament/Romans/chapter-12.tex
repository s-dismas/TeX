%%%%%%%%%%%%%%%%%%%%%%%%%%%%%%%%%%%%%%%%%%%%%%%%%%%%%%%%%%%%%%%%%
%%%%
%%%% The (original) Douay Rheims Bible 
%%%%
%%%% New Testament
%%%% Romans
%%%% Chapter 12
%%%%
%%%%%%%%%%%%%%%%%%%%%%%%%%%%%%%%%%%%%%%%%%%%%%%%%%%%%%%%%%%%%%%%%




\startcomponent chapter-12


\project douay-rheims


%%% 2661
%%% o-2512
\startChapter[
  title={Chapter 12}
  ]

\Summary{He exhorteth them to mortification of the body, 2.~to renouation
  of the mind, 3.~to keeping of vnitie by humilitie, 6.~to the right
  vſing of their guifts and functions, 9.~to many other good actions,
  17.~and ſpecially to louing of their enemies.}

I
%%% !!! This was under "A liuing Hoſt" before. But that had two LNotes
%%% in both versions. I move this part to here.
\LNote{I beſeech you.}{Leſt men should thinke by the former diſcourſe of
God's eternal predeſtination, that no reward were to be had of good life
and workes, the Apoſtle now earneſtly recommendeth to them holineſſe of
life.}
\MNote{The ſecond part of this Epiſtle, moral.}
beſeech you therfore, Brethren, by the mercie of God,
\CNote{\XRef{Phil.~4,~18.}}
that you exhibit your bodies
\LNote{A liuing Hoſt.}{Man
\MNote{The body chaſtiſed by penance is a grateful Sacrifice.}
maketh his body a Sacrifice to God by giuing it to ſuffer for him, by
chaſtiſing it with faſting, watching, and ſuch like, and by occupying it
in workes of charitie & vertue to God's honour. Whereby appeareth how
acceptable theſe workes are to God and grateful in his ſight, being
compared to a Sacrifice, which is an high ſeruice to him.}
a liuing Hoſt, holy, pleaſing God, your reaſonable ſeruice. \V And be not
conformed to this world; but be reformed in the newnes of your mind,
\CNote{\XRef{Eph.~5,~17.}
\XRef{1.~Theſ.~4,~3.}}
that you may proue what the good, & acceptable, & perfect wil of God
is. \V For I ſay by the grace that is giuen me, to al that are among
you,
\SNote{None muſt preſume to medle aboue the meaſure of God's guift, or
out of the cõpaſſe of his ſtate and vocation.}
not to be
%%% o-2513
more wiſe then behoueth to be wiſe, but to be wiſe vnto ſobrietie,
\CNote{\XRef{1.~Cor.~12,~11.}
\XRef{Eph.~4,~7.}}
to euery one as God hath deuided the meaſure of faith. \V For as in one
body we haue many members, but al the members haue not one action; \V ſo
we being many, are one body in Chriſt, & each one anothers members. \V
And hauing guifts, according to the grace that is giuen vs, different,
either
\SNote{Prophecie is interpretation of the Scriptures, which is according
to the rule of faith, when it is not againſt the right faith, or when it
is profitable to edifie charitie, as S.~Auguſtine ſpeaketh
\Cite{li.~3. Doct. Chr. c.~27.}
and
\Cite{li.~1. c.~36.}
and in effect he ſaith the ſame
\Cite{li.~12. Confeſs. c.~18. vnto c.~12.}}
prophecie,
\LNote{According to the rule of faith.}{By
\MNote{The Apoſtolical rule or Analogie of faith.}
this, and many places of holy writ, we may gather, that the Apoſtles by
the Holy Ghoſt, before they were ſundred into diuers Nations, ſet downe
among themſelues a certaine Rule and forme of faith and doctrine,
conteining not only the Articles of the Creed, but al other principles,
grounds, and the whole platforme of al the Chriſtiã religion. Which Rule
was before any of the Books of the new Teſtamẽt were writtẽ, & before
the faith was preached among the Gentils: by which not only euery other
inferiour Teacher's doctrine was tried, but al the Apoſtles, &
Euangeliſts preaching, writing, interpreting (which is here called
prophecying) were of God's Church approued and admitted, or diſproued
and reiected. This forme, by mouth and not by Scripture, euery Apoſtle
deliuered to the countrie by them conuerted. For keeping this forme,
\CNote{\XRef{c.~6,~17.}}
the Apoſtle before praiſed the Romanes, and afterward
\CNote{\XRef{c.~16,~17.}}
earneſtly warneth them by no man's plauſible ſpeach to be drawen from
the ſame. This he commendeth to Timothee,
\CNote{\XRef{1.~Tim.~6,~20.}}
calling it his \L{Depoſitum}. For not holding this faſt and ſure,
\CNote{\XRef{Gal.~1,~6.}}
he blameth the Galatians, further alſo
\CNote{\XRef{Gal.~2,~1.}}
denouncing to himſelf or an Angel that should write, teach, or expound
againſt that which they firſt receiued, Anathema, and commanding alwaies
to beware of them that taught otherwiſe. For feare of miſsing this line
of truth, himſelf notwithſtanding he had the Holy Ghoſt, yet leſt he
might haue preached in vaine and loſt his labour,
\CNote{\XRef{Act.~15,~4.}}
he went to conferre with Peter and the reſt. For the faſt keeping of
this Rule of truth, the Apoſtles held Councels, and their Succeſſours by
their example. For the holding of this Rule, and by the meaſure therof,
were al the holy Scriptures written. For and by the ſame, al the
glorious Doctours haue made their ſermons, commentaries, and
interpretations of God's word: al writings and interpretations no
otherwiſe admitted nor deemed to be of God, but as they be agreable to
this Rule.

And
\MNote{The Heretikes phantaſtical rule or rather rules of faith, many &
diuers one from another.}
this is the ſure Analogie and meaſure of faith, ſet downe and commended
to vs euery where for the Apoſtles tradition; and not the phantaſtical
rule or ſquare that euery Sect-maiſter pretendeth to gather out of the
Scriptures falſely vnderſtood and wreſted to his purpoſe, by which they
iudge of Doctour, Scripture, Church and al. Arius had by that meanes a
rule of his owne, Luther had his falſe weights, and Caluin his owne
alſo. According to which ſeueral meaſure of euery Sect, they haue their
expoſitions of God's word: and in England (as in other infected
Countries) they kept of late an apish imitation of this prophecying
which S.~Paul here and in other places ſpeaketh of, and which was an
exerciſe in the primitiue Church, meaſured not by euery man's peculiar
ſpirit, but by the former Rule of faith firſt ſet downe by the
Apoſtles. And therfore al this new phantaſtical Prophecying and al other
preaching in Caluin's ſchoole, is iuſtly by this note of the Apoſtle
condemned, for that it is not according to, but quite againſt the Rule
of faith.}
according to the rule of faith, \V or miniſterie in miniſtring, or he
that teacheth in doctrine, \V he that exhorteth in exhorting, he that
giueth in ſimplicitie, he that ruleth in carefulnes, he that ſheweth
mercie in cheerfulnes. \V
\TNote{\L{dilectio}}
Loue without ſimulation, Hating
\Fix{euel,}{euil,}{likely typo, fixed in other}
cleauing to good. \V Louing the charitie of the brotherhood one toward
another, with honour preuenting one another. \V In carefulnes not
ſlouthful. In ſpirit ſeruẽt. Seruing our Lord. \V Reioycing in
hope. Patient in tribulation. Inſtant in praier. \V Communicating to the
\Var{neceſsities}{memories}
of the Saints. Purſuing hoſpitalitie. \V Bleſſe them that perſecute you:
bleſſe, and
\SNote{Curſing is a vice wherunto the common people is much giuen, who
often curſe them on whom they can not otherwiſe be reuenged. They may
ſee here that it is a great fault.}
curſe not. \V To reioyce with them that reioyce, to weep with them that
weep. \V Being of one mind one toward another. Not minding high
things, but conſenting to the humble. \V Be not wiſe in your owne
conceit. \V To no man rendring euil for euil. Prouiding good things not
only before God, but alſo before al men. \V If it may be, as much as is
in you, hauing peace with al men. \V Not reuenging your ſelues, my
Deereſt, but giue place vnto wrath, for it is written:
\CNote{\XRef{Deu.~32,~35.}}
\Emph{Reuenge to me; I wil reward}, ſaith our Lord. \V But
\CNote{\XRef{Pro.~25,~21.}}
\Emph{if thine enemie hunger, giue him meat: if he thirſt, giue him
drinke. For, doing this, thou shalt heap coales of fire vpon his
head.} \V Be not ouercome of euil, but ouercome in good the euil.


\stopChapter


\stopcomponent


%%% Local Variables:
%%% mode: TeX
%%% eval: (long-s-mode)
%%% eval: (set-input-method "TeX")
%%% fill-column: 72
%%% eval: (auto-fill-mode)
%%% coding: utf-8-unix
%%% End:

