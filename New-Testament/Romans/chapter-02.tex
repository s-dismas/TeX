%%%%%%%%%%%%%%%%%%%%%%%%%%%%%%%%%%%%%%%%%%%%%%%%%%%%%%%%%%%%%%%%%
%%%%
%%%% The (original) Douay Rheims Bible 
%%%%
%%%% New Testament
%%%% Epistles
%%%% Romans
%%%% Chapter 02
%%%%
%%%%%%%%%%%%%%%%%%%%%%%%%%%%%%%%%%%%%%%%%%%%%%%%%%%%%%%%%%%%%%%%%




\startcomponent chapter-02


\project douay-rheims


%%% 2635
%%% o-2485
\startChapter[
  title={Chapter 2}
  ]

\Summary{Now alſo he sheweth that neither the Iewes could be ſaued by
  the knowledge of the Law, of the which they did ſo much brag againſt
  the Gentils, ſeeing they did notwithſtanding ſinne as the Gentils
  did. 14.~And therfore that the true Iew is the Chriſtian (though he be
  a Gentil) who by grace in his hart doeth the good workes that the Law
  commandeth.}

For the which cauſe thou art inexcuſable, ô man, whoſoeuer
\LNote{Thou that iudgeſt.}{Such
\MNote{Iudging other men.}
as by publike authoritie either ſpiritual or temporal haue to puniſh
offenders, be not forbidden to iudge or condemne any for their offenſes,
though themſelues be ſometimes guilty in their conſcience of the ſame
or greater: yet may it be matter of aggrauating ſinnes before God, when
they wil not repent of thoſe offenſes themſelues, for the which they
punish others. But if they be open offenders themſelues, in the ſame fort
for which they iudge other, they giue ſcandal, and thereby aggrauate
their ſinnes very much. Properly here he forbiddeth to charge another
falſely or truly with theſe crimes whereof himſelf is as farre guilty or
more then the other, as the Iewes ſpecially did the Gentils, to whom he
ſpeaketh here.}
thou be that iudgeſt. For wherein thou iudgeſt another, thou condemneſt
thyſelf. For thou doeſt the ſame things which thou iudgeſt. \V For we
know that the iudgement of God is according to veritie vpon them that
doe ſuch things. \V And doeſt thou ſuppoſe this, ô man, that iudgeſt
them which doe ſuch things, and doeſt the ſame, that thou ſhalt eſcape
the iudgement of God? \V Or
\LNote{Doeſt thou contemne?}{This
\MNote{God's lõg ſuffering is for our repẽtance.}
proueth that God offereth his grace & mercie to many, & by long patience
& ſufferance expected their repentance, differring their punishmẽt of
purpoſe that they may amend, and that he is not delighted in their
perdition, nor is the cauſe of their ſinne: but contrariewiſe that they
harden their owne harts, and of their owne free-wil reiect his grace and
contemne his benignitie.}
doeſt
%%% o-2486
thou contemne the riches of his goodnes, and patience, and longanimity,
not knowing that the benignity of God bringeth thee to penance? \V But
according to thy hardnes and impenitent hart, thou heapeſt to thy ſelf
wrath, in the day of wrath and of the reuelation of the iuſt iudgement
of God, \V who wil
\CNote{\XRef{Pſ.~63,~13.}}
render to euery man
\LNote{According to his workes.}{Though
\MNote{Good workes meritorious.}
the holy Apoſtles ſpecial purpoſe be in this Epiſtle, to commend vnto
the Gentils that truſted ſo much in their moral workes, the faith in
Chriſt; yet leſt any man ſhould thinke or gather vntruly of his wordes,
that Chriſtian mens workes were not meritorious or the cauſe of
Saluation, he expreſly writeth, that God giueth as wel euerlaſting life
and glorie to men, for and according to their good workes, as he giueth
damnation for the contrarie workes. And howſoeuer Heretikes fondly fly
from the euidence of theſe places, yet S.~Auguſtin ſaith,
\CNote{\Cite{Li. de grat. & lib. arb. c.~8.}}
Life euerlaſting to be rendred for good workes according to this
manifeſt Scripture: \Emph{God shal render to euery man according to his
workes.}}
according to his workes: \V
\SNote{Good mẽ alſo according to the merits of their good wil ſhal haue
their reward.
\Cite{Aug. ep.~47.}}
to them truely that according to patience in good worke, ſeeke glorie
and honour and incorruption, life eternal; \V but to them that are of
contention, and that obey not the truth, but giue credit to iniquitie,
wrath and indignation. \V Tribulation and anguiſh vpon euery ſoul of man
that worketh euil, of the Iew firſt and of the
\TNote{That is, \Emph{the Gentil}.}
Greek: \V but glorie and honour and peace to euery one that worketh
good, to the Iew
%%% 2636
and to the Greek. \V For
\CNote{\XRef{Deu.~10,~17.}
\XRef{Act.~10,~34.}}
there is no acception of perſons with God. \V For whoſoeuer haue ſinned
without the Law, without the Law ſhal periſh: and whoſoeuer haue ſinned
in the Law, by the Law ſhal be iudged. \V For
\CNote{\XRef{Mt.~7,~21.}
\XRef{Ia.~1,~21.}}
\LNote{Not the
\Fix{heares.}{hearers.}{obvious typo, fixed in other}}
{This ſame ſentence agreable alſo to Chriſtes wordes
\XRef{(Mat.~7,~21.)}
is the very ground of S.~Iames diſputatiõ, that not faith alone, but
good workes alſo doe iuſtifie. Therfore S.~Paul (howſoeuer ſome
peruerſly conſter his wordes in other places) meaneth the ſame that
S.~Iames. And here
\CNote{Aug. de Sp. & lit. c.~16. to.~3.}
he ſpeaketh not properly of the
\MNote{The firſt iuſtification without workes: the ſecond by workes.}
firſt iuſtification, when an Infidel or il man is made iuſt, who had no
acceptable workes before to be iuſtified by (of which kind he ſpecially
meaneth in other places of this Epiſtle) but he ſpeaketh of the ſecond
iuſtification or increaſe of former iuſtice, which he that is in Gods
grace, daily proceedeth in, by doing al kind of good workes, which be
iuſtices, and for doing of which, he is iuſt indeed before God.
\MNote{S.~Paul ſpeaketh of the firſt ſpecially, S.~Iames of the ſecond.}
And of this kind doth S.~Iames namely treate. Which is directly againſt
the Heretikes of this time, who not only attribute nothing to the workes
done in ſinne and infidelitie, but eſteeme nothing at al of a Chriſtian
mans workes toward iuſtificatiõ & ſaluation, condẽning thẽ as vncleane,
ſinful, hypocritical, Phariſaical, which is directly againſt theſe &
other Scriptures, and plaine blaſpheming of Chriſt and his grace, by
whoſe ſpirit and cooperation we doe them.}
not the hearers of the Law are iuſt with God: but the doers of the Law
\LNote{Shal be iuſtified.}{Of
\MNote{Againſt imputatiue iuſtice.}
al other Articles deceitfully handled by Heretikes, they vſe moſt guile
in this of Iuſtification; & ſpecially by the equiuocation of certaine
wordes; which is proper to al contentious wranglers, and namely in this
word, \Emph{Iuſtifie}. Which becauſe they find ſometime to ſignifie the
acquiting of a guilty man of ſome crime whereof he is indeed guilty, &
for which he ought to be condemned, (as by mans iudgement either of
ignorance or of purpoſe often a very malefactour is deemed or declared &
pronounced innocent) they falſly make it ſo ſignifie in this place & the
like, whereſoeuer man is ſaid to be iuſtified of God for his workes or
otherwiſe: as though it were ſaid, that God iuſtifieth man, that is to
ſay, imputeth to him the iuſtice of Chriſt though he be not indeed iuſt;
or of fauour reputeth him as iuſt, when indeed he is wicked, impious,
and vniuſt. Which is a moſt blaſphemous doctrine againſt God, making him
either ignorant who is iuſt, & ſo to erre in his iudgement; or not good,
that can loue and ſaue him whom he knoweth to be euil.
\MNote{True inherent iuſtice more for God's glorie, & for the
commẽdation of Chriſts merites.}
And a maruelous pittiful blindnes it is in the Churches Aduerſaries,
that they should thinke it more to God's glorie, and more to the
commendation of Chriſtes iuſtice, merites, and mercie, to cal and count
an il man ſo continuing, for iuſt; then by his grace and mercie to make
him of an one, iuſt indeed, and ſo truly to iuſtifie him, or as the word
doth here ſignifie, to eſteeme and approue for iuſt indeed, him that by
his grace
keepeth his law and commandements. For, that the keepers or doers of the
commandements be iuſt and ſo reputed, it is plaine by the correſpondence
to the former wordes: \Emph{Not the
\Fix{heares}{hearers}{obvious typo, fixed in other}
are iuſt, but the doers.} Whereupon
\Cite{S.~Auguſtin de Sp. & lit. c.~26. to.~3.}
hath theſe wordes: \Emph{When it be ſaid, The doers of the Law shal be
iuſtified, what other thing is ſaid, then, The iuſt shal be iuſtified?
for the doers of the Law verily are iuſt.}}
ſhal be iuſtified. \V For when the Gentils which haue not the Law,
naturally doe thoſe things that are of the Law; the ſame not hauing the
Law, themſelues are a law to themſelues: \V who ſhew the workes of the
Law written in their harts, their conſcience giuing teſtimonie to them,
and among themſelues mutually their thoughts accuſing, or alſo
defending, \V in the day when God ſhal iudge the ſecrets of men,
according to my Ghoſpel, by \Sc{Iesvs Christ}.

\V But if thou be ſurnamed a Iew, and reſteſt in the Law, and doeſt
glorie in God, \V and knoweſt his wil, and aproueſt the more profitable
things, inſtructed by the Law, \V preſumeſt that thy ſelf art a leader
of the blind, a light of them that are in darknes, \V a teacher of the
fooliſh, a maiſter of infants, hauing the forme of ſciẽce & of veritie
in the Law. \V Thou therfore
\SNote{It is a shameful and damnable thing for Preachers, Teachers, or
other guides of mens life, to cõmit the ſame things thẽſelues, which
they reproue in other.}
that teacheſt another, teacheſt not thy ſelf: that preacheſt, men ought
not to ſteale, thou ſtealeſt: \V that ſayeſt men ſhould not commit
aduoutrie, thou commiteſt aduoutrie: that abhorreſt idols, thou doeſt
ſacriledge: \V that doeſt glorie in the Law, thou by preuaricatiò of the
Law doeſt diſhonour God. (\V 
\CNote{\XRef{Eſ.~52,~5.}
\XRef{Ez.~36,~20.}}
\Emph{For}
\SNote{It is a great ſinne that by the il life of the faithful, our
Lords name ſhould be il ſpoken of amõg the miſbeleeuers, and many
withdrawen frõ the true religion thereby.}
\Emph{the name of God through you is blaſphemed among the Gentils}, as
it is written.) \V Circumciſion indeed profiteth, if thou obſerue the
Law: but if thou be a
%%% o-2487
preuaricatour of the Law, thy circumciſion is become
\SNote{Prepuce is the foreskin not circumciſed, & therfore ſignifieth
the Gentils, or the ſtate and condition of the Gentils: as circumciſion,
the Iewes and their ſtate.}
prepuce. \V If then the prepuce
\LNote{Keepe the iuſtices.}{If
\MNote{True iuſtice both in Iew and Gentile, is by keeping the Law.}
a Gentil either now ſince Chriſt, by his grace and faith, or any other
before Chriſt, not of the ſtocke of Abraham, through the Spirit of God
keep the iuſtices of the Law, he is iuſt no leſſe then if he had been
outwardly circumciſed, and shal condemne the circumciſed Iew not keeping
the Law, without which, his outward Sacrament cannot ſerue him, but shal
be much to his condemnation, that hauing the law and peculiar Sacraments
of God, he did not keepe the Law, nor inwardly exerciſe that in his hart
which the outward ſigne did import. And al this is no more but to
inſinuate that true iuſtice is not in faith only or knowledge of the
Law, or in the name either of Iew or Chriſtian, but in doing good workes
and keeping the Law by Gods grace.}
keepe the iuſtices of the Law; ſhal not his prepuce be reputed for
circumciſion? \V and ſhal not that which of nature is prepuce,
fulfilling the Law, iudge thee, that by the letter and circumciſion, art
a preuaricatour of the Law? \V For not he that is in open ſhew, is a
Iew, nor that which is in open ſhew in the fleſh, is circumciſiõ: \V
but he that is in ſecret a Iew; and the circumciſion of the hart,
\LNote{In Spirit, not letter.}{The
\MNote{The letter, and the ſpirit.}
outward ceremonies, Sacraments, threates, and commandements of God in
the Law, are called the \Emph{letter}; the inward working of God in mans
hart & indowing him with faith, hope, and charitie, and with loue,
liking, wil, & abilitie to keepe his commandements by the grace and
merites of Chriſt, are called the \Emph{ſpirit}.
\MNote{The carnal, & ſpiritual Iewe.}
In which ſenſe, the carnal Iew was a Iew according to the letter, and he
was circumciſed after the letter: but the true beleeuing Gentil
obſeruing by Gods grace in hart and in Gods ſight that which was meant
by that carnal ſigne, is a Iew according to the spirit, & iuſtified by
God. Of the ſpirit and letter S.~Auguſtin made a 
\CNote{de ſp. & lit. to.~3.}
famous worke, very neceſſarie for the vnderſtanding of this Epiſtle.}
in ſpirit, not in the letter: whoſe praiſe is not of men, but of God.


\stopChapter


\stopcomponent


%%% Local Variables:
%%% mode: TeX
%%% eval: (long-s-mode)
%%% eval: (set-input-method "TeX")
%%% fill-column: 72
%%% eval: (auto-fill-mode)
%%% coding: utf-8-unix
%%% End:

