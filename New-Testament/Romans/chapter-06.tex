%%%%%%%%%%%%%%%%%%%%%%%%%%%%%%%%%%%%%%%%%%%%%%%%%%%%%%%%%%%%%%%%%
%%%%
%%%% The (original) Douay Rheims Bible 
%%%%
%%%% New Testament
%%%% Epistles
%%%% Romans
%%%% Chapter 06
%%%%
%%%%%%%%%%%%%%%%%%%%%%%%%%%%%%%%%%%%%%%%%%%%%%%%%%%%%%%%%%%%%%%%%




\startcomponent chapter-06


\project douay-rheims


%%% 2645
%%% o-2495
\startChapter[
  title={Chapter 6}
  ]

\Summary{He exhorteth vs, now after Baptiſme, to liue no more in ſinne,
  but to walke in good workes: becauſe there we died to the one, and
  roſe againe to the other 14.~(grace alſo giuing vs ſufficient
  ſtrength) 16.~and were made free to the one, and ſeruants to the
  other; 21.~and ſpecially becauſe of the fruit here, and the end
  afterward, both of the one and of the other.}

What ſhal we ſay then? Shal we continue in ſinne that grace may
abound? \V God forbid. For we that are dead to ſinne, how ſhal we yet
liue therein? \V Are you ignorant that al 
%%% !!! Not marked in either text
\LNote{We that are baptized.}{That
\MNote{Not only faith.}
which before he chalenged from the Law of Moyſes, to faith, is now
attributed to Baptiſme, which is the firſt Sacrament of our faith and
the entrance to Chriſtian religion. Whereby it is plaine that he meaneth
not only faith to iuſtifie, but the
\Fix{Sacramens}{Sacraments}{obvious typo, fixed in other}
alſo, and al the Chriſtian religion, which he calleth the Law of ſpirit,
grace, and faith.}
we which are baptized in
Chriſt \Sc{Iesvs}, in his death we are baptized? \V For
\SNote{Remiſsion of ſinne, new life, ſanctification, and iuſtification,
are giuen by Baptiſme, becauſe it reſembleth in vs and applieth to vs
Chriſtes death and reſurrection, and engrafteth vs into him.}
we are buried together with him by Baptiſme into death: that as Chriſt
is riſen from the dead by the glorie of the Father, ſo we alſo may walke
in newneſſe of life. \V For if we become complanted to the ſimilitude of
his death
%%% o-2496
we ſhal be alſo of his reſurrection. \V Knowing this, that our
%%% !!! This and the next LNote are combined.
\LNote{Old man, body of ſinne.}{Our
\MNote{The old man, & the new.}
corrupt ſtate ſubiect to ſinne and concupiſcence, comming to vs from
Adam, is called the \Emph{Old man} as our perſon reformed in & by
Chriſt, is named the \Emph{New man}. And the lump and maſſe of ſinnes
which then ruled, is called the corps or body of ſinne.}
old man is crucified with him, that
%%% !!! part of above LNote{}
%%% \LNote{}{}
the body of ſinne may be
\SNote{Caſtalion noteth that Beza falſly tranſlateth \L{eneruetur},
for \L{deſtruatur}: weakned, for, deſtroied.}
deſtroied, to the end that we may ſerue ſinne no longer. \V For he that
is dead, is iuſtified from ſinne. \V And if we be dead with Chriſt, we
beleeue that we ſhal liue alſo together with Chriſt. \V Knowing that
Chriſt riſing againe from the dead, now dieth no more, death ſhal no
more haue dominion ouer him. \V For that he died,
\LNote{To ſinne he died.}{Chriſt
\MNote{Dying to ſinne, Liuing to God.}
died to ſinne, when by his death he deſtroied ſinne: We die to ſinne, in
that we be diſcharged of the power thereof, which before was as it were
the life of our perſons, and commanded al the parts and faculties of our
ſoule and body: as contrarie-wiſe we liue to God, when his grace ruleth
and worketh in vs, as the ſoule doth rule our mortal bodies.}
to ſinne he died once: but that he liueth, he liueth to God. \V So
thinke you alſo, that you are dead to ſinne, but aliue to God in
Chriſt \Sc{Iesvs} our Lord.

\V Let not
\LNote{Sinne reigne.}{Concupiſcence
\MNote{How concupiſcence is called ſinne.}
is here named ſinne, becauſe it is the effect, occaſion, and matter of
ſinne, and is as it were a diſeaſe or infirmitie in vs, inclining vs to
il, remaining alſo after Baptiſme according to the ſubſtance or matter
thereof: but it is not properly a ſinne, nor forbidden by commandement,
til it reigne in vs, and we obey and follow the deſires
\Fix{therrof.}{thereof.}{likely typo, fixed in other}
\Cite{Auguſt. li. de nupt. & concupiſc. c.~23.}
\Cite{Cont. 2.~epiſt. Pelag. li.~1. c.~13.}
\Cite{Conc. Trident. Seſſ.~5. decret. de pec. orig.}}
ſinne therfore reigne in your mortal body, that you obey the
concupiſcences thereof. \V But neither doe ye exhibit your members
inſtruments of iniquitie vnto ſinne: but exhibit your ſelues to God as
of dead men, aliue; and your members inſtruments of iuſtice to God. \V
For ſinne ſhal not haue dominion ouer you. For you are not vnder the
Law, but vnder grace.

\V What then? ſhal we ſinne, becauſe we are not vnder the Law, but vnder
grace? God forbid. \V 
\CNote{\XRef{Io.~8.~34.}
\XRef{2.~Pet.~2.~19.}}
Know you not that to whom you exhibit your ſelues ſeruants to obey, you
are the ſeruants of him whom you obey, whether it be of ſinne, to death,
or of obedience, to iuſtice. \V But thankes be to God, that you were the
ſeruants of ſinne, but
\SNote{Here againe is ſignified, that our diſcharge from the bondage of
ſinne, is by the Chriſtian faith, & by obedience to the whole doctrine
of Chriſt's religion: in that the Apoſtle attributeth this their
deliuerance from ſinne, to their humble receiuing of the Catholike faith.}
haue obeied from the hart, vnto that
\LNote{Forme of doctrine.}{At
\MNote{The doctrine of our firſt Apoſtles.}
the firſt conuerſion of euery Nation to the Catholike faith, there is a
forme & rule of beleefe ſet downe, vnto which when the people is once
put by their Apoſtles, they muſt neuer by any perſuaſion of men alter
the ſame, nor take of man or Angel, any new doctrine or Analogie of
faith, as the Proteſtants cal it.}
forme of doctrine, into the which you haue been deliuered. \V And being
made free from ſinne, you were made ſeruãts to iuſtice. \V I ſpeake an
humane thing, becauſe of the infirmitie of your fleſh. For as you haue
exhibited your members to ſerue vncleanneſſe and iniquitie, vnto
iniquitie; ſo now exhibit your members to ſerue iuſtice,
\SNote{He ſignifieth that as when they were ſubiect to ſinne by
continual & often working wickednes, they increaſed their iniquitie:
that ſo alſo now being iuſtified, they may & should by external works of
iuſtice, increaſe their iuſtice and ſanctification.}
vnto ſanctification. \V For when you were ſeruants of ſinne, you were
free to iuſtice. \V What fruit therfore had you then in thoſe things,
for which now you are aſhamed? For the end of them is death. \V But now
being made free from ſinne, and
%%% 2646
become ſeruants to God, you haue your fruit vnto ſanctification, but
the end, life euerlaſting. \V For the ſtipends of ſinne, death. But
\LNote{The grace of God, life euerlaſting.}{The
\MNote{Life euerlaſting a ſtipend, and yet grace.}
ſequele of ſpeach required, that as he ſaid, death or damnation is the
ſtipend of ſinne, ſo life euerlaſting is the ſtipend of iuſtice; and ſo
it is, and in the ſame ſenſe he ſpake in the laſt chapter:
\CNote{\XRef{v.~10.}}
\Emph{that as ſinne reigned to death, ſo grace may reigne by iuſtice to
life euerlaſting.} But here he changed the ſentence ſomwhat, calling
life euerlaſting \Emph{grace}, rather then \Emph{reward}: becauſe the
merits by which we attaine vnto life, be al God's guift and grace.
\Cite{Auguſt. Ep.~105. ad Sixtum.}}
the grace of God, life euerlaſting in Chriſt 
\Fix{\Sc{Iessvs}}{\Sc{Iesvs}}{obvious typo, fixed in other}
our Lord.


\stopChapter


\stopcomponent


%%% Local Variables:
%%% mode: TeX
%%% eval: (long-s-mode)
%%% eval: (set-input-method "TeX")
%%% fill-column: 72
%%% eval: (auto-fill-mode)
%%% coding: utf-8-unix
%%% End:

