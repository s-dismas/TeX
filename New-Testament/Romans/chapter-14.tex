%%%%%%%%%%%%%%%%%%%%%%%%%%%%%%%%%%%%%%%%%%%%%%%%%%%%%%%%%%%%%%%%%
%%%%
%%%% The (original) Douay Rheims Bible 
%%%%
%%%% New Testament
%%%% Romans
%%%% Chapter 14
%%%%
%%%%%%%%%%%%%%%%%%%%%%%%%%%%%%%%%%%%%%%%%%%%%%%%%%%%%%%%%%%%%%%%%




\startcomponent chapter-14


\project douay-rheims


%%% 2664
%%% o-2516
\startChapter[
  title={Chapter 14}
  ]

\Summary{Like a moderatour and peace-maker between the firme Chriſtians
  (who were the Gentils) and the infirme (who were the Chriſtian Iewes,
  hauing yet a ſcruple to ceaſe from keeping the ceremonial meats and daies
  of Moyſes Law) he exhorteth the Iew not to condemne the Gentil vſing
  his libertie; and the Gentil againe, not to condemne the ſcrupulous
  Iew: but rather to abſtaine from vſing his libertie, then offending
  the Iew, to be an occaſion vnto him of apoſtating.}

And him that is weak in faith, take vnto you; not in diſputations of
cogitations. \V For one beleeueth that he may
\LNote{Eate al things.}{By
\MNote{The Apoſtles meaning about eating or not eating certaine meats.}
ſimilitude of words the ſimple are ſoone deceiued, and Heretikes make
their vantage of any thing to ſeduce the vnlearned. There were diuers
meats forbidden in the Law of Moyſes, and for ſignification made and
counted vncleane, whereof the Iewes might not eate at al, as porke,
hare, conny, & ſuch like, both of fiſhes, foules, and beaſts, a great
number. Chriſt diſcharged al them that became Chriſtians, after his
Paſsion, of that obſeruance and al other ceremonies of the old
Law. Notwithſtanding, becauſe diuers that were brought-vp in the Law,
had a religion and conſcience, ſodenly to foreſake their former manner,
the Apoſtle here admoniſhed ſuch as be ſtronger and better inſtructed in
the caſe, to beare with the weaker ſort, that being Chriſtians could not
yet find in their harts to eate and vſe the meats forbidden by God in
the Law: as on the other ſide he warneth the weak that would not eate,
not to take offence or ſcandal at them that did eate without ſcruple,
any of the irregular or forbidden meats in the Law, nor in any wiſe to
iudge or condemne the eater, but to cõmit that to God, & finally that
they ſhould not condemne each other for eating or not eating.
\MNote{The Heretikes fondly abuſe this place againſt the faſts of the
Church.}
Now the Proteſtants fondly apply al this to the faſts of the Church, and
differences of meats in the ſame: as though the Church did forbid any
meat wholy neuer to be eaten or touched, or made any creatures vncleane,
or otherwiſe preſcribed any abſtinence, then for chaſtiſing of mens
bodies and ſeruice of God. It is a great blindnes that they can put no
difference betwixt Chriſtes faſt of fourtie daies,
\XRef{Mat.~4.}
Iohn's abſtaining from al delicate meats and drinkes,
\XRef{Mat.~3,~11.}
the widow Annes,
\XRef{Luc.~2,~37.}
the Nazareites,
\XRef{Num.~6.}
the Recabites,
\XRef{Ierem.~35,~14.}
the Niniuites,
\XRef{Ion.~3.}
S.~Paules,
\XRef{2.~Cor.~11,~27.}
S.~Timothees,
\XRef{1.~Tim.~5,~23.}
Iohn's Diſciples and Chriſt's Diſciples faſt
\XRef{Mat.~9,~14.~15.}
(which he ſaid they should keep after his departure from them:) and the
ceremonial diſtinction of creatures and meats, cleane and vncleane, in
the old Law. Of which it is euident the Apoſtle treateth in al this
chapter, & of none other at al. Therfore when the Proteſtants by the
words of this place would proue, that we be either made free from
faſting and from obeying the Churches commandement or following Chriſtes
example in that matter, or that the obſeruers of Chriſtian faſts be weak
in faith, & ought not in any wiſe condemne of ſinne the breakers of the
preſcribed faſts of the holy Church, they doe abuſe ignorantly or
wilfully the Apoſtles words and diſcourſe.}
eate al things: but he that is weak,
\Var{let him eate}{eateth}
herbs. \V Let not him that eateth, deſpiſe him that eateth not: and he
that eateth not, let him not iudge him that eateth. For God hath taken
him to him. \V Who art thou that iudgeſt another man's ſeruant? To his
owne Lord he
%%% 2665
ſtandeth or falleth. And he ſhal ſtand: for God is able to make him
ſtand. \V For one iudgeth
\LNote{Between day and day.}{By
\MNote{Diſtinction of daies.}
the like deceit they abuſe this place againſt the Holydies of Chriſt and
his B.~mother & Saints, which concerneth only the Iewes feſtiuities and
obſeruation of times, wherof in the 
\XRef{Epiſtle to the Galatians c.~4,~10.}}
between day and day; and another iudgeth euery day: let euery one abound
\LNote{Euery one in his owne ſenſe.}{The
\MNote{The text explicated concerning euery man's conſcience in Iudaical
meats and drinkes.}
Apoſtle doth not giue freedom, as the Churches enemies would haue it,
that euery man may doe or thinke what he liſt. But in this matter of
Iudaical obſeruation of daies and meats, & that for a time only, til the
Chriſtian religion ſhould be perfectly eſtablished, he would haue no
reſtraint made, but that euery one should be borne withal in his owne
ſenſe: yet ſo, that they should not condemne one another, nor make
neceſſitie of ſaluation in the obſeruation of the Iudaical rites of meats,
daies, &c.}
in his owne ſenſe. \V He that reſpecteth the day, reſpecteth to our
%%% o-2517
Lord. And he that eateth, eateth to our Lord: for he giueth thankes to
God. And he that eateth not, to our Lord he eateth not, and giueth
thankes to God. \V For none of vs liueth to himſelf: and no man dieth to
himſelf. \V For whether we liue we liue to our Lord; or whether we die,
we die to our Lord. Therfore whether we liue, or whether we die, we are
our Lord's. \V For to this end Chriſt died and roſe againe; that he may
haue dominion both of the dead and of the liuing. \V But thou, why
iudgeſt thou thy brother? or thou, why doeſt thou deſpiſe thy brother?
For
\CNote{\XRef{2.~Cor.~5,~10.}}
we ſhal al ſtand before the iudgemẽt ſeat of Chriſt. \V For it is
writtẽ:
\CNote{\XRef{Eſ.~45,~23.}}
\Emph{Liue I}, ſaith our Lord, \Emph{that euery knee shal bow to me; &
euery tõgue shal confeſſe to God.} \V Therfore euery one of vs for
himſelf ſhal render account to God. \V Let vs therfore no more iudge one
another. But this iudge ye rather, that you put not a ſtumbling block or
a ſcandal to your brother. \V I know and am perſuaded in our
Lord \Sc{Iesvs Christ}, that nothing is
\SNote{\Emph{Common}, that is, \Emph{vncleane}. See
\XRef{Annot. Marc.~7,~2.}

Though he wiſh the weake to be borne withal, yet he vttereth his mind
plainly, that indeed al the meats forbidden and vncleane in the Law, are
now through Chriſt cleãſed & lawful for euery man to vſe.}
common of it ſelf, but to him that ſuppoſeth any thing to be common, to
him it is common. \V For if becauſe of meat thy brother be greeued; not
thou walkeſt not according to charitie.
\CNote{\XRef{1.~Cor.~8.}}
Doe not with thy meat deſtroy him for whom Chriſt died. \V Let not then
our good be blaſphemed. \V For the Kingdom of God is
\LNote{Not meat and drinke.}{The
\MNote{Not eating, but diſobedience dãnable.}
ſubſtance of religion or the Kingdom of God ſtandeth not in meat or
drinke; and therfore the better might they vſe indifferencie &
toleration in that point for a time, for peace ſake and to auoid
ſcandal. But if the precept of Moyſes Law had bound ſtil as before, then
(not for the meats ſake, but for the diſobedience) it had been damnable
to haue eaten the vncleane meats.}
not meat and drinke; but iuſtice, and peace, and ioy in the Holy
Ghoſt. \V For he that in this ſerueth Chriſt, pleaſeth God, and is
acceptable to men. \V Therfore the things that are of peace let vs
purſue: and the things that are of edifying one toward another let vs
keep. \V Deſtroy not the worke of God for meat.
\CNote{\XRef{Tit.~1,~15.}}
Al things indeed are cleane: but it is il for the man that eateth by
giuing offence. \V It is good not to eate fleſh, and not to drinke wine,
nor that wherin thy brother is offended, or ſcandalized, or weakned. \V
Haſt thou faith?
\LNote{Haue it with thy ſelf.}{Thou that art perfect, and beleeueſt or
knoweſt certainly that thou art free from the Law concerning meats and
feſtiuities, yet to the trouble and hindrance of the feeble that can not
yet be brought ſo farre, be diſcrete and vtter not thy ſelf out of
ſeaſon.}
haue it with thy ſelf before God. Bleſſed is he that iudgeth not himſelf
in that which he approueth. \V But
\LNote{He that diſcerneth.}{If
\MNote{To doe againſt our cõſcience, is ſinne.}
the weak haue a conſcience, and ſhould be driuen to eate the things
which in his owne hart he thinketh he should not doe, he committeth
deadly ſinne, becauſe he doth againſt his conſcience, or againſt his
owne pretenſed knowledge.}
he that diſcerneth, if he eate, is damned; becauſe not of faith. For
\LNote{Al that is not of faith.}{The
\CNote{Chryſ. ho.~26. in ep.~Ro.}
\MNote{What actions of infidels are ſinne, & what are not.}
proper ſenſe of this ſpeach is, that euery thing that a man doeth
againſt his knowledge or conſcience, is a ſinne, for ſo by the
circumſtance of the letter, faith muſt here be taken, though S.~Auguſtin
ſometimes applieth it alſo to proue that al the actions of infidels
(meaning thoſe workes which directly proceed of their lacke of faith) be
ſinnes. But in any wiſe take heed of the 
\Fix{Haretikes}{Heretikes}{obvious typo, fixed in other}
commentarie, who
hereby would proue that the infidel ſinneth in honouring his parẽts,
fighting for his countrie, tilling his ground, and in al other
workes. And no maruel that they ſo hold of infidels, who maintaine
\CNote{Luther.}
that Chriſtian men alſo offend deadly in euery good deed.}
al that is not of faith, is ſinne.


\stopChapter


\stopcomponent


%%% Local Variables:
%%% mode: TeX
%%% eval: (long-s-mode)
%%% eval: (set-input-method "TeX")
%%% fill-column: 72
%%% eval: (auto-fill-mode)
%%% coding: utf-8-unix
%%% End:

