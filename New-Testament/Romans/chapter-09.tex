%%%%%%%%%%%%%%%%%%%%%%%%%%%%%%%%%%%%%%%%%%%%%%%%%%%%%%%%%%%%%%%%%
%%%%
%%%% The (original) Douay Rheims Bible 
%%%%
%%%% New Testament
%%%% Epistles
%%%% Romans
%%%% Chapter 09
%%%%
%%%%%%%%%%%%%%%%%%%%%%%%%%%%%%%%%%%%%%%%%%%%%%%%%%%%%%%%%%%%%%%%%




\startcomponent chapter-09


\project douay-rheims


%%% 2652
%%% o-2503
\startChapter[
  title={Chapter 9}
  ]

\Summary{With a proteſtation of his ſorrow for it (leſt they should
  thinke him to reioyce in their perdition) he inſinuateth the Iewes to
  be reprobate, although they come of Abrahãs flesh, 6.~ſaying, to be
  ſonnes of God, goeth not by that, but by God's grace: 19.~conſidering
  that al were one damned maſſe. 24.~By which grace the Gentils to be
  made his people: & ſo the Prophets to haue foretold of them
  both. 30.~And the cauſe hereof to be, that the Gentils ſubmit
  themſelues to the faith of Chriſt, which the Iewes wil not.}

I ſpeake the verity in Chriſt, I lie not, my conſcience bearing me
witnes in the Holy Ghoſt, \V that I haue great ſadneſſe & continual
ſorrow in my hart. \V For I wished my ſelf to be an
\LNote{Anathema.}{Anathema
\MNote{Anathema.}
by vſe of Scripture is either that which by
ſeparation from profane vſe, and by dedication to God, is holy,
dreadful, and not vulgarly to be touched; or  contrariewiſe, that which
is reiected, ſeuered, or abandoned from God, as curſed and deteſted, and
therfore is to be auoided. And in this later ſenſe (according as S.~Paul
taketh it
\XRef{1.~Cor.~16.}
\Emph{If any loue not our Lord \Sc{Iesvs Christ}, be he Anathema}, that
is to ſay, Away with him, Accurſed be he, Beware you company not with
him) the Church and holy Councels vſe the word for a curſe and
excommunication againſt Heretikes & other notorious offenders &
blaſphemers.
\MNote{In what ſenſe S.~Paul wiſheth to be anathema.}
Now how the Apoſtle, wiſhing himſelf to be Anathema from Chriſt to ſaue
his Countrie-mens ſoules, did take this word, it is a very hard thing to
determine. Some thinke, he deſired only to die for their
ſaluatiõ. Others, that being very loth to be kept from the fruitiõ of
Chriſt, yet he could be cõtent to be ſo ſtil for to ſaue their
ſoules. Others, that he wiſhed what malediction or ſeparation from
Chriſt ſoeuer that did not imply the disfauour of God towards him, nor
take away his loue toward God. This only is certaine, that it is a point
of vnſpeakable charitie in the Apoſtles breaſt, and a paterne to al
Bishops and Prieſts, how to loue the ſaluation of their flocke. As the
like was vttered by Moyſes when he ſaid:
\CNote{\XRef{Exo.~32,~32.}}
\Emph{Either forgiue the people, or blot me out of thy booke.}}
anathema from Chriſt for my brethren, who are my kinſmen according to
the fleſh, \V who are Iſraelites, whoſe is the adoption of ſonnes, and
the glorie, and the Teſtament, and the law-giuing, &
\TNote{\G{ἡ λατρεία}}
the ſeruice, and the promiſes: \V whoſe are the Fathers, & of whom
Chriſt is according to the fleſh, who is aboue al things God Bleſſed for
euer. Amen.

\V But not that the word of God is fruſtrate. For,
\LNote{Not al of Iſrael.}{Though
\MNote{God's promiſe not made to carnal Iſrael.}
the people of the Iewes were many waies honoured and priuiledged, and
namely by Chriſtes taking flesh of them, yet the promiſe of grace and
ſaluation was neither only made to them, nor to al them that carnally
came of them or their Fathers: God's election, and mercie depending vpon
his owne purpoſe, wil, and determination, and not tied to any Nation,
familie, or perſon.}
not al that are of Iſrael, they be Iſraelites: \V nor they that are the
ſeed of Abraham, al be children:
\CNote{\XRef{Gen.~21,~12.}}
\LNote{But in Iſaac.}{The
\MNote{Iſaac preferred before Iſmael.}
promiſe made to Abraham was not in Iſmael, who was a ſonne borne only by
flesh and nature; but in Iſaac, who was a ſonne obtained by promiſe,
faith, and miracle; and was a figure of the Churches children borne to
God in Baptiſme.}
\Emph{but in Iſaac shal the ſeed be called vnto thee}: \V that is to
ſay, not they that are the childrẽ of the fleſh they are the childrẽ of
God: but they that are the children of the promiſe, are eſteemed for the
ſeed. \V For the word of the
%%% o-2504
promiſe is this:
\CNote{\XRef{Gen.~18,~10.}}
\Emph{According to this time wil I come; & Sara shal haue a ſonne.} \V
And not only ſhe. But
\CNote{\XRef{Gen.~21,~21.}}
Rebecca alſo conceiuing
\LNote{Of one copulation.}{It
\MNote{Iacob before Eſau.}
is proued alſo by God's chooſing of Iacob before Eſau (who were not only
brethren by father and mother, but alſo twinnes, and Eſau the elder of
the two, which according to carnal count should haue had the
preeminence) that God in giuing graces followeth not the temporal or
carnal prerogatiues of men or families.}
of one copulation, of Iſaac our Father. \V For whẽ they were
\LNote{Not yet borne.}{By
\MNote{By the exãple of Iacob and Eſau, is ſhewed God's mere mercie in
the Elect, & iuſtice in the Reprobate.}
the ſame example of thoſe twinnes, it is euident alſo, that neither
Nations nor particular perſons be elected eternally, or called
temporally, or preferred to God's fauour before others, by their owne
merits: becauſe God, whẽ he made choiſe, and firſt loued Iacob, and
refuſed Eſau, reſpected them both as il, and the one no leſſe then the
other guilty of damnation for original ſinne, which was a-like in them
both. And therfore where iuſtly he might haue reprobated both, he ſaued
of mercie one. Which one therfore, being as il and as void of good as
the other, muſt hold of God's eternal purpoſe, mercie, and election,
that he was preferred before his brother which was elder then himſelf,
and no worſe then himſelf. And his brother Eſau on the other ſide hath no
cauſe to complaine. For that God neither did nor ſuffred any thing to be
done towards him, that his ſinne did not deſerue. For although God elect
eternally & giue his firſt grace without al merites, yet he doth not
reprobate or hate any man but for ſinne, or the foreſight thereof.}
not yet borne, nor had done any good or euil (that the purpoſe of God
according to election might ſtand) \V not of workes, but of the Caller
it was ſaid to her:
\CNote{\XRef{Gen.~25,~23.}}
\Emph{That the elder shal ſerue the yonger}, \V as it is written:
\CNote{\XRef{Mal.~1,~2.}}
\Emph{Iacob I loued, but Eſau I hated.}

\V
%%% !!! Make this an LNote?
\SNote{\Cite{S.~Hierom. q.~10. ad Hedibiam.}
Al the epiſtle ſurely to the Romanes needeth interpretation, and is
enwrapped with ſo great obſcurities that to vnderſtand it we need the
help of the Holy Ghoſt, who by the Apoſtle did dictate theſe ſame
things: but eſpecially this place. Howbeit nothing pleaſeth vs but that
which is Eccleſiaſtical, that is, the ſenſe of the Church.}
What ſhal we ſay then?
\LNote{Is there iniquitie?}{Vpon
\MNote{That God is not vniuſt, or an accepter of perſons, is declared by
familiar examples.}
the former diſcourſe, that of two perſons equal God calleth the one to
mercie, and leaueth the other in his ſinne, one might inferre that God
were vniuſt and an accepter of perſons. To which the Apoſtle anſwereth,
that God were not
\Fix{iuſt}{vniuſt}{possible typo, same in both}
nor indifferent indeed, ſo to vſe the matter where grace or ſaluation
were due. As if two men being Chriſtned, both beleeue wel, & liue wel:
if God ſhould giue Heauen to the one, and ſhould damne the other, then
were he vniuſt, partial, & forgetful of his promiſe: but reſpecting or
taking two, who both be worthy of damnatiõ (as al are before they be
firſt called to mercie) then the matter ſtandeth on mere mercie, and of
the giuers wil and liberalitie, in which caſe partialitie hath no
place. As for example
\MNote{S.~Auguſtines example is of two debters: the one forgiuen al, &
the other put to pay al, by the ſame creditour.
\Cite{li. de prædeſt. & gra. c.~4.}}
%%% !!! Two collumns ???
\starttabulate[|l|p|l|p|]
\NC 1.
\NC Two malefactours being condemned both for one crime, the Prince
pardoneth the one, & letteth the law proceed on the other.
\NC 1.
\NC So likewiſe, God ſeeing al mankind and euery one of the ſame in a
general condẽnation & maſſe of ſinne, in & by Adã, deliuereth ſome, and
not otherſome.
\NC \NR
\NC 2.
\NC The theefe that is pardoned, can not attribute his eſcape to his
owne deſeruings, but to the Princes mercie.
\NC 2.
\NC Al that be deliuered out of that common damnation, be deliuered by
grace and pardon, through the meanes and merits of Chriſt.
\NC \NR
\NC 3.
\NC The theefe that is executed, can not chalenge the Prince that he was
not pardoned alſo: but muſt aknowledge that he hath his deſeruing.
\NC 3.
\NC Such as be left in the common caſe of damnation, can not complaine,
becauſe they haue their deſeruing for ſinne.
\NC \NR
\NC 4.
\NC The ſtanders by muſt not ſay, that he was executed becauſe the
Prince would not pardon him. For that was not the cauſe, but his
offenſe.
\NC 4.
\NC We may not ſay that ſuch be damned, becauſe God did not pardon them,
but becauſe they did ſinne, and therfore deſerued it.
\NC \NR
\NC 5.
\NC If they aſke further, why the Prince pardoned not both, or executed
not both: the anſwer is, that as mercie is a goodly vertue, ſo iuſtice
is neceſſarie & commendable.
\NC 5.
\NC That ſome ſhould be damned, & not al pardoned, and otherſome
pardoned rather then al condemned, is agreable to God's iuſtice &
mercie: both which vertues in God's prouidence towards vs are
recommended.
\NC \NR
\NC 6.
\NC But if be further demanded why Iohn rather then Thomas was executed;
or Thomas rather then Iohn pardoned: anſwer, that (the parties being
otherwiſe equal) it hangeth merely and wholy vpon the Princes wil and
pleaſure.
\NC 6.
\NC That Saul should be rather pardoned then Caiphas (I meane where two
be equally euil & vnderſeruing) that is only God's holy wil and
appointement, by which many an vnworthy man getteth pardon, but no good
or iuſt or innocent perſon is euer damned.
\NC \NR
\stoptabulate

In
\MNote{Predeſtination & reprobation take not away free-wil neither muſt
any man be retchleſſe & deſperate.}
al this mercie of God towards ſome, and iuſtice towards otherſome, both
the pardoned worke by their owne free-wil, and thereby deſerue their
ſaluation, and the other no leſſe by their owne free-wil, without al
neceſsitie, worke wickednes, & themſelues and only of themſelues procure
their owne damnation. Therfore no man may without blaſphemie ſay, or can
truely ſay, that he hath nothing to doe towards his owne ſaluation, but
wil liue, and thinketh he may liue without care or cogitation of his end
the one way or the other, ſaying: If I be appointed to be ſaued, be it
ſo; if I be one deſigned to damnation, I can not help the matter: come
what come may. Theſe ſpeaches and cogitations are ſinful & come of the
enemie, and be rather ſignes of reprobation, then of election. Therfore
the good man muſt without ſearch of God's ſecrets, worke his owne
ſaluation, and (as S.~Peter ſaith)
\CNote{\XRef{2.~Pet.~1,~10.}}
\Emph{make his election ſure by good workes}, with continual hope of
God's mercie, being aſſured that if he beleeue wel & doe wel, he shal
haue wel. For example, if a husband-man should ſay: If God wil, I shal
haue corne enough; if not, I can make it; and ſo neglect to til his
ground: he may be ſure that he shal haue none, becauſe he wrought not
for it. Another man vſeth his diligence in tilling & ploughing, and
committeth the reſt to God: he findeth the fruit of his labours.}
Is there iniquitie with God? God forbid. \V For to Moyſes he ſaith:
\CNote{\XRef{Exo.~33,~19.}}
\Emph{I wil haue mercie on whom I haue mercie; and I wil shew mercie to
whom I wil shew mercie.} \V Therfore it is
\LNote{Not of the willer.}{If
\MNote{Our election or conuerſion is not of our ſelues, but of God's
grace and mercie.}
our election, calling, or firſt comming to God, lay wholy or principally
vpon our owne wil or workes; or if our willing or endeuouring to be
good, would ſerue without the help and grace of God, as the Pelagians
taught, then our election were wholy in our ſelues, which the Apoſtle
denieth. And then might Pharao and other indurate perſons (whom God hath
permitted to be obſtinate, to shew his power and iuſt iudgement vpon
them) be conuerted when themſelues liſt without God's help and
aſsiſtance: whereas we ſee the contrarie in al ſuch obſtinate offenders,
whom God for punishment of former ſinnes viſiteth not with his grace,
that by no threats, miracles, nor perſuaſion, they can be
conuerted. Whereupon we may not with Heretikes inferre, that man hath
not free-wil, or that our wil worketh nothing in our conuerſion or
comming to God: but this only, that our willing or working of any good
to our ſaluation, commeth of God's ſpecial motion, grace, and
aſsiſtance, that it is the ſecondary cauſe, not the principal.}
not of the
%%% 2653
willer, nor the runner, but of God that ſheweth mercie. \V For the
Scripture ſaith to Pharao:
\CNote{\XRef{Exo.~9,~16.}}
\Emph{That
\LNote{To this purpoſe haue I raiſed.}{He doth not ſay, that he hath of
purpoſe raiſed or ſet him vp to ſinne, or that he was the cauſe of the
ſame in Pharao, or that he intended his damnation directly or
abſolutely, or any otherwiſe but in reſpect of his demerits: but rather
(as the Apoſtle ſaith ſtraight after in this chapter of ſuch hardned and
obſtinate offenders) that he with long patience and toleration expected
his conuerſion, and (as S.~Chryſoſtome interpreteth this
word, \L{excitaui}) preſerued him aliue to repent, whom he might iuſtly
haue condemned before. In the
\XRef{9.~of Exodus,}
whence this allegation is, we read,
\CNote{Exo.~9,~16.}
\L{posui te}, \Emph{I haue put} or \Emph{ſet thee vp}, as here, \Emph{I
haue raiſed thee}.
\MNote{In what ſenſe, God raiſed vp Pharao.}
That is to ſay, I haue purpoſely aduanced thee to be ſo great a King,
and choſen thee out to be a notorious example both of the obdurate
obſtinacie that is in ſuch whom I haue for ſo great ſinnes forſaken, and
alſo to shew to the world, that no obſtinacie of neuer ſo mightie
offenders can reſiſt me, or doe any thing which shal not fal to my
glorie. Which is no more to ſay, but that God often for the punishment
of Nations, and to shew his iuſtice & glorie, giueth wicked Princes vnto
them, & by indowing them with power and proſperitie, and by taking his
grace from them vpon their deſerts, ſo hardneth their harts, as they
withſtand and contemne him, and afflict his people, in whoſe end and
fal, either temporal or eternal, at the length God wil euer be
glorified. Neither would he either raiſe or ſuffer any ſuch, or giue
them power and proſperitie in this life, wherupon he knoweth they wil be
worſe, but that he can worke al that to his honour and glorie. Mary,
that he vſeth not ſuch rigorous iuſtice on al the deſerue it, that is
his great grace and mercie. And that he exerciſeth his iuſtice vpon ſome
certaine perſons, rather then vpon otherſome of equal deſerts, that
lieth wholy vpon his wil, in whoſe iudgements there be many things
ſecret, but nothing vniuſt: as S.~Auguſtin teacheth.
\Cite{Ser.~88. de temp.}
Where (as alſo, 
\Cite{li. de prædeſt. & graite,~15.}
and in other places) he hath manie goodlie leſſons touching this high
point of doctrine. Of which we intend to recite ſome more vpon the 
%%% Point to proper commentary? Is it there?
\XRef{7.~or~9.~chapt. of Exodus};
if God wil giue vs meanes to ſet forth the old Teſtament in English.}
to this purpoſe haue I raiſed thee, that in thee I may shew my power;
and that my name may be renowmed in the whole earth.} \V Therfore on
whom he wil, he hath mercie; and whom he wil, he doth indurate.

\V Thou ſaiſt therfore vnto me: Why doth he yet complaine? for who
reſiſteth his wil? \V O man,
\LNote{Who art thou?}{Here
\MNote{Heretical bookes concerning predeſtination.}
the Apoſtle ſtaieth the rashnes and preſumption of ſuch poore wormes, as
take vpon them to queſtion with God of their election or reprobation, as
certaine impious Heretikes of our time haue done, ſetting out bookes
farſed with moſt blaſphemous and erroneous doctrine concerning this
high & hidden myſterie, and haue giuen occaſion to the ignorant which
alwaies be curious, to iangle, and perniciouſly to erre in theſe things,
that are impoſsible to be vnderſtood of any, or wel thought of, but of
the obedient and humble.}
who art thou that doeſt anſwer God? Doth the worke ſay to him that
wrought it: Why haſt thou made me thus? \V Or hath not
\LNote{The potter.}{This
\MNote{The example of the pot and the potter.}
example of the pot and potter reacheth no further but to declare, that
the creature may not reaſon with God his Maker, why he giueth not one ſo
great grace, as another, or why he pardoneth not one as wel as another:
no more then the chamber-pot may chalenge the Potter why he was not made
a drinking-pot, as wel as another. And therfore the Heretikes that
extend this ſimilitude to proue that man hath no free-wil no more then a
peece of clay, doe vntruely and deceitfully apply the example. Specially
when we may ſee expreſly in the booke of Exodus, that Pharao
notwithſtanding his indurate hart, had free-wil; where both it is
ſaid: \Emph{He would not diſmiſſe the people}; and: \Emph{He indurated 
his owne hart himſelf.}
\XRef{Exo. c.~8,~15.}
and (in the Hebrew)
\XRef{v.~32.}
and
\XRef{c.~9,~35.}
\XRef{1.~Reg.~6,~6.}
And this Apoſtle alſo writeth, that
\CNote{2.~Tim.~2,~21.}
a man may \Emph{cleanſe himſelf} from the filthy, and ſo become a
veſſel of honour in the houſe of God.}
the potter of the clay, power, of the ſame maſſe to make one veſſel vnto
honour, and another vnto contumelie? \V And if God willing to ſhew
wrath, & to make his might knowen,
\SNote{That God is not the cauſe of any mãs reprobation or damnation,
otherwiſe then for puniſhmẽt of his ſinnes, he ſheweth by
\Fix{that that}{that}{obvious typo, fixed in other}
he expecteth al mẽs amẽdemẽt with great patience, & conſequently that
they haue alſo free-wil.}
ſuſteined in much patience the veſſels of wrath
%%% Only in other
\Var{apt}{apted, fitted}
to deſtruction, \V that he might ſhew the riches of his glorie vpon the
veſſels of mercie which he prepared vnto glorie.

\V Whom alſo he hath called, vs, not only of the Iewes, but alſo of the
Gentils, \V as in Oſee he ſaith:
\CNote{\XRef{Oſ.~2,~23.}}
\Emph{I wil cal that which is not my people, my people; & her that was
not beloued, beloued: & her that hath not obteined mercie, hauing
obteined mercie.} \V
\CNote{\XRef{Oſ.~1,~10.}}
\Emph{And it shal be, in the place where it was ſaid to them, you are
not my people: there they shal be called the ſonnes of the liuing
God.} \V And Eſaie crieth for Iſrael:
\CNote{\XRef{Eſ.~10,~22.}}
\Emph{If the number of the children of Iſrael be as the ſand of the ſea,
the remaines shal be ſaued. \V For conſumnating a word, and abbridging
it in equitie: becauſe a word abbridged shal our Lord make vpon the
earth.} \V And as Eſay foretold:
\CNote{\XRef{Eſ.~1,~9.}}
\Emph{Vnles the Lord of Sabaoth had left vs ſeed, we had been made like
Sodom, and we had been like as Gomorrha.}

\V What ſhal we ſay then? That the Gentils which purſued not after
iuſtice, haue apprehended iuſtice, but the iuſtice that is of faith. \V
But Iſrael in purſuing the law of iuſtice, is not come vnto the law of
iuſtice. \V Why ſo?
%%% o-2505
Becauſe not of faith, but as it were of workes. For
\SNote{Here we ſee that they are the cauſe of their owne dãnation by
infidelity.}
they haue ſtumbled at the ſtone of ſtumbling, \V as it is written:
\CNote{\XRef{Eſ.~8,~14.}}
\Emph{Behold I put in Sion a ſtone of ſtumbling, and a rocke of
ſcandal}:
\CNote{\XRef{28,~16.}}
\Emph{and whoſoeuer beleeueth in him, shal not be confounded.}


\stopChapter


\stopcomponent


%%% Local Variables:
%%% mode: TeX
%%% eval: (long-s-mode)
%%% eval: (set-input-method "TeX")
%%% fill-column: 72
%%% eval: (auto-fill-mode)
%%% coding: utf-8-unix
%%% End:

