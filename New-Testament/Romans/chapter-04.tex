%%%%%%%%%%%%%%%%%%%%%%%%%%%%%%%%%%%%%%%%%%%%%%%%%%%%%%%%%%%%%%%%%
%%%%
%%%% The (original) Douay Rheims Bible 
%%%%
%%%% New Testament
%%%% Epistles
%%%% Romans
%%%% Chapter 04
%%%%
%%%%%%%%%%%%%%%%%%%%%%%%%%%%%%%%%%%%%%%%%%%%%%%%%%%%%%%%%%%%%%%%%




\startcomponent chapter-04


\project douay-rheims


%%% 2640
%%% o-2491
\startChapter[
  title={Chapter 4}
  ]

\Summary{That Abraham was not iuſtified by his owne power, but by God's
  grace, in whom he beleeued (6.~which is a way for the ſinner alſo to
  come to iuſtice.) 9.~And that, ſeeing he was not as then circumciſed;
  not only the circumciſed Iew, but alſo the vncircumciſed Gentil may by
  beleeuing the Chriſtian faith, come to iuſtice, as Abraham did:
  11.~ſpecially conſidering alſo, that Abraham was promiſed to be Father
  of the whole world, and not only of the Iewes, to whom only the Law
  was giuen: and that, not to fulfil the promiſe, but for another cauſe.}

What shal we ſay then that
\LNote{Abraham.}{The
\MNote{Abraham's works before faith.}
Apoſtle diſputing in this chapter, as before, againſt them that thought
they might be iuſtified by their works done without the grace of Chriſt
& faith in him, propoſeth Abraham for an example, and proueth that he
had no iuſtice nor eſtimation of iuſtice before God by any works done
before he had faith, or that proceeded not of faith & God's grace.}
Abraham did find, our Father according to the flesh? \V For if Abraham
were iuſtified
\LNote{By works.}{If
\MNote{Iuſtice before men, & iuſtice before God.}
Abraham did any commendable works before he beleeued Chriſt, as many
Philoſophers did, men might count him iuſt therfore; but in God's ſight
(who accepteth nothing without faith in him, or that proceedeth not from
his grace) he should neuer haue had the eſtimation of a iuſt
man. Therfore God in the Scriptures reputing him as a iuſt man, giueth
the cauſe thereof, ſaying: \Emph{Abraham beleeued God, and it was reputed
to him for iuſtice.}}
by works he hath glorie, but not with God. \V For what ſaith the
Scripture?
\CNote{\XRef{Gen.~5,~6.}
\XRef{Gal.~3,~6.}
\XRef{Ia.~2,~23.}}
\Emph{Abraham beleeued God, & it was reputed him to iuſtice.} \V But
\LNote{To him that worketh.}{That
\MNote{Not works, but mere grace is cauſe of our firſt iuſtification.}
is to ſay: He that preſumeth of his owne works as done of himſelf
without faith, God's help, and grace: and ſaying, that grace or
iuſtification were giuen to him for his works; this man doth chalenge
his iuſtification as debt, & not as of fauour & grace.}
to him that worketh, the reward is not imputed according to grace but
according to debt. \V But
\LNote{To him that worketh not.}{He worketh not (in this place) that
hath no works or alleageth not his works done in his infidelitie as
cauſe of his iuſtificatiõ, but faith in Chriſt, & that proceeding of mere
grace. Wherupon S.~Auguſtin ſaith: \Emph{Know thou that faith found thee
vniuſt. And if faith giuen to thee, made thee iuſt, it found thee a
wicked one whom it might make iuſt. If it found thee wicked, and of ſuch
an one made thee iuſt, what works hadſt thou being then wicked? None
couldeſt thou haue (nor canſt haue) before thou beleeuedſt. Beleeue then
in him that iuſtifieth the impious, that thy good works may be good
works indeed.}
\Cite{Auguſt. In Pſal.~31.}}
to him that worketh not, yet beleeueth in him that iuſtifieth the
impious, his faith is
\TNote{\G{λογίζεται}}
reputed to iuſtice according to the purpoſe of the grace of God. \V
\LNote{As Dauid termeth.}{The
\MNote{Heretical tranſlation.}
Proteſtants for,
\TNote{\G{λέγει}}
\Emph{termeth}, tranſlate, \Emph{deſcribeth}, for that they would haue
the ignorãt beleeue, the whole nature & definition of Iuſtificatiõ to be
nothing els but remiſſion of ſinnes, and no grace or inherent iuſtice
giuen from God at al. When the Apoſtle would ſay nothing els, but that
in the firſt iuſtification God findeth no good works or merits to
reward, but only ſinnes to forgiue vnto ſuch as haue faith in him.}
As Dauid alſo termeth the bleſſednes of a man, to whõ God reputeth
iuſtice without works: \V 
\CNote{Pſ.~31.~1.}
\Emph{Bleſſed are they,
%%% 2641
whoſe iniquities be forgiuen, and whoſe ſinnes be
%%% !!! This LNote is for the two marks in the text.
\LNote{Couered. 8.~Not imputed.}{You
\MNote{What is, \Emph{Sinnes couered} or \Emph{not imputed}.}
may not gather (as the Heretikes doe) of theſe termes, \Emph{couered},
and, \Emph{not imputed}, that the ſinnes of men be neuer truly forgiuen,
but hidden only. For that derogateth much to the force of Chriſts bloud
& to the grace of God, by which our offences be truly remitted. He is
the Lamb that
\CNote{\XRef{Io.~1,~29.}}
\Emph{taketh avvay} the ſinnes of the world, that
\CNote{\XRef{2.~Cor.~6,~11.}}
\Emph{waſheth}, and
\CNote{\XRef{Apoc.~1,~5.}}
\Emph{blotteth out} our ſinnes. Therfore to couer them, or, not to
impute them, is, not to charge vs with our ſinnes, becauſe by remiſſion
they be cleane taken away: otherwiſe it were but a feined
forgiueneſſe. See
\Cite{S.~Auguſtine in Pſal.~31. enarrat.~2.}}
couered. \V Bleſſed is the man to whom our Lord hath
%%% !!! This LNote is part of the above. How to mark this?
%%% \LNote{}{}
not imputed ſinne.}

\V This bleſſednes then doth it abide in the circumciſion, or in the
prepuce alſo? For we ſay that vnto Abraham faith was
\SNote{The word \Emph{Reputed}, doth not diminish the truth of the
iuſtice, as though it were reputed for iuſtice being not iuſtice indeed;
but ſignifieth, that as it was in itſelf, ſo God eſteemed & reputed it:
as the ſame greeke word muſt needs be taken
\XRef{v.~4.}
next going before, &
\XRef{1.~Cor.~4,~1.}
and elſwhere.}
reputed to iuſtice. \V How was it reputed? in circumciſion, or in
prepuce? Not in circumciſion, but in prepuce. \V And
\CNote{\XRef{Gen.~17,~10.}}
he receiued
\SNote{Our Sacraments of the new Law giue \L{ex opere operato}, the
grace and iuſtice of faith which here is commended: whereas circũciſion
was but a ſigne or marke of the ſame.}
the ſigne of circumciſion,
\LNote{A ſeale.}{The
\MNote{The Sacramẽts are not mere markes, but cauſes of iuſtification.}
Heretikes would proue hereby, that the
\Fix{Sacramens}{Sacraments}{obvious typo, fixed in other}
of the Church giue not grace or iuſtice of faith, but that they be
notes, markes, and badges only of our remiſſion of ſinnes had by faith
before, becauſe Abraham was iuſt before and took this Sacrament for a
ſeale therof only. To which muſt be anſwered, that it followeth not that
it is ſo in al, becauſe
\Fix{it it}{it}{obvious typo, fixed in other}
was ſo in the Patriarch, who was iuſt before,
and was therfore as it were the Founder of Circumciſion, or he in whom
God would firſt eſtabliſh the ſame: no more then it followeth that,
becauſe the Holy Sacrament of the Altar remitted not ſinnes to Chriſt
nor iuſtified him, therfore it hath that effect in none. Look
\Cite{S.~Auguſtine de Baptiſme contra Donatiſtas li.~4. c.~24.}
Where you shal ſee that (though not in Abraham) yet in Iſaac his ſonne,
and ſo conſequently in the reſt, the Sacrament went before, and iuſtice
followed.}
a ſeale of the iuſtice of faith that is in prepuce: that he might be the
Father of al that beleeue by the prepuce, that vnto them alſo it may be
reputed to iuſtice: \V and might be Father of circumciſion, not to them
only that are of the circumciſion, but to them alſo that follow the ſteps of
the faith that is in the prepuce of our Father Abraham. \V For not by
the Law was the promiſe to Abraham, or to his ſeed, that he ſhould be
heire of the world; but by the iuſtice of faith. \V For if they that are
of the Law, be heires; faith is made void, the promiſe is aboliſhed. \V
For the Law worketh wrath. For where is no Law, neither is there
preuarication. \V Therfore of faith: that according to grace the promiſe
may be firme to al the ſeed, not to that only which is of the Law, but
to that alſo which is of the faith of Abraham, who is 
%%% o-2492
the Father of vs al, (as it is written: \V For
\CNote{\XRef{Gen.~17,~4.}}
\Emph{a Father of many Nations haue I appointed thee}) before God, whom
\Var{thou didſt beleeue,}{he beleeued.}
who quickneth the dead; and calleth thoſe things that are not, as thoſe
things that are. \V Who contrarie to hope beleeued in hope; that he
might be made the Father of many Nations, according to that which was
ſaid to him:
\CNote{\XRef{Gen.~15,~5.}}
\Emph{So shal thy ſeede be}, as the ſtarres of Heauen, and the ſand of
the ſea. \V And he was not weakned in faith; neither did he conſider his
owne body now quite dead, whereas he was almoſt an hundred yeares old,
and the dead matrice of Sara. \V In the promiſe alſo of God he ſtaggered
not by diſtruſt; but was ſtrengthned in faith, giuing glorie to God: \V
moſt fully knowing that whatſoeuer he promiſed, he is able alſo to
doe. \V Therfore was it alſo reputed him to iuſtice.

\V And it is not written only for him, that it was reputed him to
iuſtice; \V but alſo
\LNote{For vs, to whom it ſhal be reputed.}{By this it is moſt plaine
againſt our Aduerſaries, that the faith which was reputed for iuſtice to
Abraham, was his beleefe of an Article reuealed to him by God, that is
to ſay, his aſſent & credit giuen to God's ſpeaches:
\MNote{By what faith we are iuſtified.}
as in vs his poſteritie according to the ſpirit, it is here plainly
ſaid, that iuſtice shal be reputed to vs by beleeuing the Articles of
Chriſtes death and Reſurrection, and not by any fond ſpecial
faith, \L{fiducia}, or confidence of each mans owne ſaluation. To
eſtablish the which fictiõ, they make no account of the faith Catholike,
that is, wherewith we beleeue the Articles of the faith, which only
iuſtifieth, but cal it by contempt, an hiſtorical faith: ſo as they may
terme Abraham's faith, & our Ladies faith, of which it was
\CNote{\XRef{Luc.~1,~45.}}
ſaid, \L{Beata quæ crediſtiſti}, \Emph{Bleſſed art thou that haſt
beleeued}. And ſo in truth they deny as wel the iuſtification by faith,
as by works.}
for vs, to whom it ſhal be reputed beleeuing in him, that raiſed
vp \Sc{Iesvs Christ} our Lord from the dead, \V who was deliuered vp for
our ſinnes, and roſe againe for our iuſtification.


\stopChapter


\stopcomponent


%%% Local Variables:
%%% mode: TeX
%%% eval: (long-s-mode)
%%% eval: (set-input-method "TeX")
%%% fill-column: 72
%%% eval: (auto-fill-mode)
%%% coding: utf-8-unix
%%% End:

