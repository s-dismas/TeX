%%%%%%%%%%%%%%%%%%%%%%%%%%%%%%%%%%%%%%%%%%%%%%%%%%%%%%%%%%%%%%%%%
%%%%
%%%% The (original) Douay Rheims Bible 
%%%%
%%%% New Testament
%%%% Epistles
%%%% Romans
%%%% Chapter 16
%%%%
%%%%%%%%%%%%%%%%%%%%%%%%%%%%%%%%%%%%%%%%%%%%%%%%%%%%%%%%%%%%%%%%%

%%% Latin checked by KK.




\startcomponent chapter-16


\project douay-rheims


%%% 2668
%%% o-2520
\startChapter[
  title={Chapter 16}
  ]

\Summary{He commendeth the bearer Phœbe to the Romanes, 3.~and himſelf
  to many there by name. 17.~He declareth the doctrine which the Romanes
  had learned, to be the touchſtone to know Seducers. 21.~He doth vnto
  them the commendations of al the Churches & of certaine perſons by
  name; 25.~and concludeth.}

%%% o-2521
And I commend to you Phœbe our Siſter, who is in the miniſterie of the
Church that is in Cenchris: \V that you receiue her in our Lord as it is
worthie for Saints: and that you aſſiſt her in whatſoeuer buſines ſhe
ſhal need you. For ſhe alſo hath aſſiſted many, and my ſelf.

\V
\SNote{The only ſalutation of ſo worthy a mã is ſufficient to fil him
with great grace that is ſo ſaluted.
\Cite{Chry. in 2.~Tim.~4.}}
Salute Priſca & Aquila my helpers in \Sc{Christ Iesvs}, \V who for my
life haue laid downe their neckes; to whom not I only giue thankes, but
alſo al the Churches of the Gentils, \V and their
\SNote{This domeſtical Church was either that faithful and Chriſtiã
houſhold, or rather the Chriſtians meeting together there & in ſuch good
houſes to heare diuine ſeruice & the Apoſtles preaching in thoſe times
of perſecution.}
domeſtical Church. Salute Epænetus my Beloued: who is the firſt fruit of
Aſia in Chriſt. \V Salute Marie who hath laboured much about vs. \V
Salute Andronicus and
\Var{Iulia}{Iunia}
my coſins and fellow captiues: who are noble among the Apoſtles, who
alſo before me were in Chriſt. \V Salute Ampliatus my beſt Beloued in
our Lord. \V Salute Vrbanus our helper in \Sc{Christ Iesvs}, and Stachys
my Beloued. \V Salute Apelles
\TNote{\G{τὸν δόκιμον}}
approued in Chriſt. Salute them that are of Ariſtobolus houſe. \V Salute
Herodion my kinſman. Salute them that are of Narciſſus houſe, that are
in our Lord. \V Salute Triphæna and Tryphoſa: who labour in our
Lord. Salute Perſis the Beloued, who hath much laboured in our Lord. \V
Salute Rufus the elect in our Lord and his mother and mine. \V Salute
Aſyncritus, Phlegon, Hermas, Patrobas, Hermes: and the Brethren that are
with them. \V Salute Philologus and
%%% 2669
Iulia, Nereus, and his ſiſter, and Olympias; and al the Saints that are
with them. \V
\LNote{Salute one another.}{Neuer Sect-maiſters made more foule or hard
ſhifts to proue or defend falſehood, then the
\Fix{Proteſtans:}{Proteſtants:}{obvious typo, fixed in other}
but in two points, about S.~Peter ſpecially, they paſſe euen them ſelues
in impudẽcie. The firſt is, that they hold he was not preferred before
the other Apoſtles, which is againſt the Scriptures moſt euidently.
\MNote{That S.~Peter was at Rome.}
The
ſecond is, that he was neuer at Rome, which is againſt al the
Eccleſiaſtical hiſtories, al the Fathers Greeke & Latine, againſt the
very ſenſe & ſight of the monuments of his Seat, Sepulcher, doctrine,
life, and death there. Greater euidence certes there is thereof and more
weighty teſtimonie, then of Romulus, Numas, Cæſar's or Cicero's being
there: yet were he a very brutiſh man that would deny this to the
diſcredit of ſo many Writers and the whole world. Much more monſtrous it
is, to heare any deny the other. Theodorete ſaith he was there, writing
\Cite{vpon this chapter.}
Proſper alſo
\Cite{carmine de ingratis in principio.}
S.~Leo
\Cite{de natali Petri.}
S.~Auguſtin
\Cite{to.~6. c.~4. cont. ep. fund.}
Oroſius
\Cite{li.~7. c.~6.}
S.~Chruſoſtome
\Cite{in Pſal.~48.}
S.~Epiphanius
\Cite{hær.~27.}
Prudentius
\Cite{in hymno.~2. S.~Laurentij}
&
\Cite{hymno.~11.}
Optatus
\Cite{li.~2. contra Donatiſtas.}
S.~Ambroſe
\Cite{li.~5. ep. de Baſilicis tradendis.}
S.~Hierome
\Cite{in Catalogo.}
Lactintius
\Cite{li.~4. c.~11. de vera ſapentia.}
Euſebius
\Cite{hiſt. Eccl. li.~2. c.~13,~15.}
S.~Athanaſius
\Cite{de fuga fina.}
S.~Cyprian
\Cite{ep.~53. nu.~6.}
Tertullian
\Cite{de præſcriptionibus nu.~14.}
and
\Cite{li.~4. contra Marcionem nu.~5.}
Origen
\Cite{in Geneſ. apud Enſeb. li.~3. c.~1.}
Irenæus
\Cite{li.~33. c.~3.}
Hegeſippus
\Cite{li.~3. c.~2. de excid. Hieroſolym.}
Caius & Papius the Apoſtles owne ſcholers, and Sionyſius the B.~of
Corinth, alleaged by Euſebius
\Cite{li.~2. c.~14. and 24.}
Ignatius
\Cite{ep.~ad Romanos.}
The
\CNote{Chalced. conc. act.~3.}
holy Councel of Chalcedon, and many others affirme it. Yea
\MNote{See the Annotations
\XRef{1.~Pet. c.~5,~13.}}
Peter himſelf
(according to the iudgement of the Ancient Fathers) confeſſeth he was at
Rome, calling it Babylon.
\XRef{1.~ep. c.~5.}
\Cite{Euſeb. li.~2. c.~14. hiſt. Ec.}
Some of theſe tel the time and cauſe of his firſt going thither: ſome,
how long he liued there: ſome, the manner of his death there: ſome, the
place of his burial: and al, that he was
\Fix{he}{the}{obvious typo, fixed in other}
firſt Biſhop there. How could
ſo many of ſuch wiſedom and ſpirit, ſo neere the Apoſtles time deceiue
or be deceiued? how could Caluin and his, after fifteen hundred yeares
know that which none of them could ſee?

Some great argument muſt they needs haue to controule the credit of the
whole world. This of truth is here their argument, neither haue they a
better in any place, to wit:
\MNote{The Proteſtãts great argumẽt, that Peter was neuer at Rome.}
If S.~Peter had been at Rome, S.~Paul would haue ſaluted him, as he did
others here in the end of his letter to the Romanes. Is not this a high
point to diſproue al antiquitie by? Any man of diſcretion may ſtraight
ſee, that S.~Peter might be knowen vnto S.~Paul to be out of the Citie,
either for perſecution or buſines, when this epiſtle was written (for he
went often out, as
\CNote{Epiph. her.~27.}
S.~Epiphanius declareth) & ſo the omitting to ſalute him, can proue no
more, but that then he was not in Rome: but it proueth not ſo much
neither; becauſe the Apoſtle might for reſpect of his dignitie & other
the Churches affaires write vnto him ſpecial letters, & ſo had no cauſe
to ſalute him in his common Epiſtle. Or how know they that this Epiſtle
was not ſent incloſed to S.~Peter, to be deliuered by his meanes to the
whole Church of the Romanes in ſome of their aſſemblies? It is very like
it was recommended to ſome one principal man or other that is not here
named: and twenty cauſes there may be vnknowen to vs, why he ſaluted him
not: but no cauſe why our Aduerſaries vpon ſuch friuolous reaſons ſhould
reproue an approued truth. For euen as wel might they ſay that S.~Iohn
was neuer at Epheſus becauſe S.~Paul in his Epiſtle to the Epheſiãs doth
not ſalute him. 
\MNote{The Heretikes hatred of the Romane See.}
And plaine it is, that it is the Romane ſeat and faith of Peter, which
they (as al Heretikes before them) doe feare & hate, and which wil be
their bane: and they know that there is no argument which conuinceth in
their conſcience, that Peter was neuer at Rome. Therfore to conclude we
ſay to them in S.~Auguſtines wordes:
\CNote{\Cite{li.~2. cont. lit. Petil. c.~51.}}
\Emph{Why cal you the Apoſtolike chaire, the chaire of peſtilence? what
hath the Church of Rome done againſt you, in which S.~Peter did ſit, and
from which by nefarious furie you haue ſeparated your ſelues?}}
%%% !!! Is this where this goes? Unclear in both.
\SNote{The Proteſtants, here reaſon thus: Peter is not here
ſaluted, therfore he was neuer at Rome. See the
\XRef{Annotation.}}
Salute one another in a
\LNote{Holy kiſſe.}{Hereof,
\CNote{Orig. in 16.~ad~Ro.}
\MNote{Kiſſing the Pax.}
and by the common vſage of the firſt Chriſtians, who had ſpecial regard
of vnitie and peace among themſelues, and for ſigne and proteſtation
thereof kiſſed one another, came our holy ceremonie of giuing
the \L{Pax}, or kiſſing one another in the Sacrifice of the bleſſed
Maſſe.}
holy kiſſe. Al the churches of Chriſt ſalute you.

\V And I deſire you, Brethren,
\LNote{To marke them.}{He
\MNote{Againſt Sect-maiſters how to examine our faith.}
carefully warneth them to take heed of ſeditious ſowers of Sects &
diſſenſion in religion, and this euer to be their marke, if they ſhould
teach or moue them to any thing which was not agreable to that which
they had learned at their conuerſion: not bidding them to examin the
caſe by the Scriptures, but by their firſt forme of faith and religion
deliuered to them before they had or did read any booke of the new
Teſtament.}
to marke them that make diſſenſions and ſcandals contrarie to the
doctrine which you haue
\SNote{Of the Prince of the Apoſtles, ſaith 
\Cite{Theodoret vpon this place.}}
learned, and auoid them. \V For ſuch doe not ſerue Chriſt our Lord,
\LNote{But their owne belly.}{Howſoeuer
\MNote{Heretikes giuẽ to voluptuouſnes.}
Heretikes pretend in wordes and external ſhew of their ſheep's coat,
indeed they ſeeke but after their owne profit and pleaſure, & by the
Apoſtles owne teſtimonie we be warranted ſo to iudge of them as of men
that indeed haue no religion nor conſcience.}
but their owne belly: and
\SNote{The ſpecial way that Heretikes haue euer had to beguile, was and
is by ſweet wordes & gay ſpeaches. Which their ſheeps coat ſee before
deſcribed particularly in the
\XRef{Annotations vpon S.~Matthew. c.~7,~15.}}
by ſweet ſpeaches and benedictions ſeduce the harts of innocents. \V For
\LNote{Your obedience.}{Againſt Heretikes and their illuſions, there is
no better way then in ſimplicitie to cleaue vnto that which hath been
taught before: for the which the Romane obedience is much commended. See
\XRef{Annot. vpon the firſt chap. verſ.~8.}}
your obedience is publiſhed into euery place. I reioyce therfore in
you. But I would haue you to be wiſe in good, and ſimple in euil. \V And
the God of peace cruſh Satan vnder your feet quickly. The grace of our
Lord \Sc{Iesvs Christ} be with you.

\V Timothee my Coadiutor ſaluteth you, and Lucius, and Iaſon, and
Soſipater, my kinſmen. \V I Tertius ſalute you, that wrote the epiſtle,
in our Lord. \V Caius mine hoſt, and
%%% o-2522
the whole Churches, ſaluteth you. Eraſtus the Cofferer of the citie
ſaluteth you, and Quartus, a Brother. \V The grace of our Lord \Sc{Iesvs
Christ} be with al you, Amen.

\V And to him that is able to confirme you according to my Ghoſpel and
preaching of \Sc{Iesvs Christ}, according to the reuelation of the
myſterie from eternal times kept ſecret, \V which now is opened by the
Scriptures of the Prophets according to the precept of the eternal God,
to the obedience of faith knowen in al Gentils, \V to God the only wiſe
through \Sc{Iesvs Christ}, to whom be honour & glorie for euer and
euer. Amen.


\stopChapter


\stopcomponent


%%% Local Variables:
%%% mode: TeX
%%% eval: (long-s-mode)
%%% eval: (set-input-method "TeX")
%%% fill-column: 72
%%% eval: (auto-fill-mode)
%%% coding: utf-8-unix
%%% End:

