%%%%%%%%%%%%%%%%%%%%%%%%%%%%%%%%%%%%%%%%%%%%%%%%%%%%%%%%%%%%%%%%%
%%%%
%%%% The (original) Douay Rheims Bible 
%%%%
%%%% New Testament
%%%% Epistles
%%%% Romans
%%%% Chapter 10
%%%%
%%%%%%%%%%%%%%%%%%%%%%%%%%%%%%%%%%%%%%%%%%%%%%%%%%%%%%%%%%%%%%%%%




\startcomponent chapter-10


\project douay-rheims


%%% 2656
%%% o-2507
\startChapter[
  title={Chapter 10}
  ]

\Summary{The Law was not (as the Iewes ignorant zeale ſuppoſed) for them
  to iuſtifie themſelues by it (conſidering that they could not fulfil
  it;) but to bring them to Chriſt, to beleeue in him, and ſo for his
  ſake to be iuſtified by the grace of God, 5.~according to Moyſes
  ſaying, and the Apoſtles preaching: 11.~that ſo the Gentils alſo
  (according to the Prophets) hearing and beleeuing might come to
  iuſtice; the Iewes in the meane time (though inexcuſably) remaining
  incredulous.}

Brethren, the wil of my hart ſurely and praier to God, is for them vnto
ſaluation. \V For I giue them teſtimonie that they haue zeale of God,
but not according to knowledge. \V For not knowing
\LNote{The iuſtice of God.}{The
\MNote{God's iuſtice, & the Iewes owne iuſtice.}
iuſtice of God, is that which God giueth vs through Chriſt. The Iewes
owne or proper iuſtice, is that which they had or chalenged to haue of
themſelues and by their owne ſtrength, holpen only by the knowledge of
the Law without the help or grace of Chriſt.}
the iuſtice of God, and ſeeking to eſtabliſh their owne, they haue not
been ſubiect to the iuſtice of God. \V For,
\SNote{The Law was not giuen to make a mã iuſt or perfect by it ſelf,
but to bring vs to Chriſt to be iuſtified by him.}
the end of the Law is Chriſt; vnto iuſtice to euery one that
%%% o-2508
beleeueth. \V For Moyſes wrote,
\SNote{The iuſtice of the Law of Moyſes went no further of itſelf, but
to ſaue a man frõ the temporal death and punishment preſcribed to the
tranſgreſſours of the ſame.}
that, the iuſtice which is of the Law,
\CNote{\XRef{Leu.~18,~5.}}
\Emph{the man that hath done it, shal liue in it.} \V But
\LNote{The iuſtice of faith.}{The
\MNote{Iuſtice of faith.}
iuſtice which is of faith, reacheth to the life to come, making man
aſſured of the truth of ſuch Articles as concerne the ſame: as, of
Chriſt's Aſcenſion to heauen, of his Deſcending to Hel, of his comming
downe to be Incarnate, and his Reſurrection and returne againe to be
glorified. By which his actions we be pardoned, iuſtified, and ſaued, as
by the Law we could neuer be.}
the iuſtice which is of faith, ſaith thus:
\CNote{\XRef{Deut.~30,~32.}}
\Emph{Say not in thy hart, Who shal aſcend into Heauen?} that is to
bring Chriſt downe. \V \Emph{Or who deſcendeth into the depth?}  that is
to cal Chriſt againe from the dead. \V But what ſaith the
Scripture? \Emph{The word is nigh, in thy mouth, and in thy hart.} This
is
\LNote{The word of faith.}{The
\MNote{Open confeſſion & proteſtation of our faith is ſomtime
neceſſarie.}
word of faith is the whole Law of Chriſt, concerning both life and
doctrine, grounded vpon this, that Chriſt is our Sauiour, & that he is
riſen againe. Which point, (as al other) muſt both be beleeued in hart,
and alſo be confeſſed by mouth. For though a man be iuſtified inwardly
when he hath the vertues of faith, hope, and charitie from God; yet if
occaſion be giuen, he is alſo bound to confeſſe with his mouth, and by
al his external actions, without shame or feare of the world, that which
he inwardly beleeueth: or els he cannot be ſaued. 
%%% ??? only in Heretikes
\MNote{Helcheſetæ}
Which is againſt certaine
\CNote{\Cite{Enſeb. li.~6. c.~31. hiſtor. Eccleſ.}}
old Heretikes, that taught a man might ſay or doe what he would, for
feare or danger, ſo that he kept his faith in hart.}
the word of faith which we preach. \V For if thou confeſſe with thy
mouth our Lord \Sc{Iesvs}, and in thy hart beleeue that God hath raiſed
him vp from the dead, thou ſhalt be ſaued. \V For with the hart we
beleeue vnto iuſtice; but with the mouth confeſsion is made to
ſaluation.

\V For the Scripture ſaith:
\CNote{\XRef{Eſ.~28,~16.}}
\Emph{Whoſoeuer beleeueth in him, shal not be confounded.} \V For there
is no diſtinction of the Iew and the Greeke: for one is Lord of al,
rich toward al that inuocate him. \V
\CNote{\XRef{Ioel.~2,~22.}}
\Emph{For euery one
\SNote{To beleeue in him & to inuocate him, is to ſerue him with al loue
& ſincere affection. Al that ſo doe, shal doubtleſſe be ſaued & shal
neuer be confounded.}
whoſoeuer
%%% 2657
shal inuocate the name of our Lord, shal be ſaued.} \V
\LNote{How shal they inuocate.}{This
\MNote{The place alleaged againſt inuocation of Saints anſwered.}
maketh not (as Heretikes pretend) againſt inuocation of Saints; the
Apoſtle ſaying nothing els, but that they can not inuocate Chriſt as
their Lord and Maiſter, in whom they doe not beleeue, and whom they
neuer heard of. For he ſpeaketh of Gentils or Pagans, who could not
inuocate him, vnleſſe they did firſt beleeue in him. To the due
inuocation of Chriſt, we muſt know him and our duties to him. And ſo it
is true alſo that we can not pray to our B.~Ladie nor any Saint in
Heauen, til we beleeue and know their perſons, dignitie, and grace, and
truſt that they can help vs. But if our Aduerſaries thinke that we can
not inuocate them, becauſe we can not beleeue in them; let them
vnderſtand that the Scripture vſeth alſo this ſpeach, to beleeue in men:
and it is the very Hebrew phraſe, which they should not be ignorant of
that brag therof ſo much.
\XRef{Exod.~14,~31.}
\Emph{They beleeued in God and in Moyſes.} and
\XRef{3.~Paral.~20,~20.}
in the Hebrew.
\XRef{Ep. ad Philem. v.~5.}
And the ancient Fathers did read in the Creed indifferently, \Emph{I
beleeue in the Catholike Church}; and, \Emph{I beleeue the Catholike
Church.}
\Cite{Conc. Nicen. apud Epiphan. in ſine Anceras}
\Cite{Hierom. contr. Lucif.}
\Cite{Cyril. Hieroſ. Cathec.~17.}}
How then shal they inuocate him in whom they haue not beleeued? Or how ſhal
they beleeue him whom they haue not heard? And how ſhal they heare
without a Preacher? \V But how shal they preach
\LNote{Vnleſſe they be ſent.}{This
\MNote{\Fix{Pearchers}{Preachers}{obvious typo, fixed in other}
not lawfully called nor ſent.}
place of the Apoſtle inuincibly condemneth al the preachings, writings,
ordinances, innouations, and vſurpations of Church, pulpit, & whatſoeuer
our new Euangeliſts haue intruted themſelues and entered into by the
window: shewing that they be euery one from the higheſt to the loweſt,
falſe Prophets, running and vſurping, being neuer lawfully called. Which
is ſo euident in the Heretikes of our daies, that 
\CNote{Confeſ. des Egliſe de France.}
the Caluiniſts
confeſſe it in thẽſelues, & ſay that there is an exception to be made in
them, becauſe they found the ſtate of the Church interrupted.}
vnles they be ſent? as it is written:
\CNote{\XRef{Eſ.~52,~7.}}
\Emph{How beautiful are the feet of them that euangelize peace, of them
that euangelize good things?} \V But al
\SNote{We ſee then that it is in a mans free-wil to beleeue or not to
beleeue, to obey or diſobey the Ghoſpel or truth preached.}
doe not obey the Ghoſpel. For Eſay ſaith, Lord,
\CNote{\XRef{Eſ.~53,~1.}}
\Emph{who hath beleeued the hearing of vs?} \V Faith then, is by
hearing: and hearing is by the word of Chriſt. \V But I ſay, haue they
not heard? and certes
\CNote{\XRef{Pſ.~18,~5.}}
\Emph{into al the earth hath the ſound of them gone forth: and vnto the
ends of the whole world the words of them.}

\V But I ſay, hath not Iſrael knowen? Moyſes firſt ſaith:
\CNote{\XRef{Deu.~32,~21.}}
\Emph{I wil bring you to emulation in that which is not a Nation: in a
foolish Nation I wil driue you into anger.} \V But Eſay is bold, and
ſaith:
\CNote{\XRef{Eſ.~65,~1.}}
\Emph{I was found of them that did not ſeeke me: openly I appeared to
them
\LNote{That asked not.}{That
\MNote{The firſt iuſtification of mere grace.}
Chriſt was found of thoſe that neuer asked after him, it proueth that
the firſt grace and our firſt iuſtification is without merits. That God
called ſo continually and earneſtly by his Prophets and by other ſignes,
and wonders, vpon the Iewes, and they withſtood it,
\MNote{Free-wil.}
free-wil is proued; and that God would haue men ſaued, and that they be
the cauſe of their owne damnation themſelues.}
that asked not of me.} \V But to Iſrael he ſaith:
\CNote{\XRef{Eſ.~63,~1.}}
\Emph{Al the day haue I ſpred my hands to a people that beleeueth not,
and contradicteth me.}


\stopChapter


\stopcomponent


%%% Local Variables:
%%% mode: TeX
%%% eval: (long-s-mode)
%%% eval: (set-input-method "TeX")
%%% fill-column: 72
%%% eval: (auto-fill-mode)
%%% coding: utf-8-unix
%%% End:

