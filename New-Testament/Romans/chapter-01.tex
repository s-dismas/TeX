%%%%%%%%%%%%%%%%%%%%%%%%%%%%%%%%%%%%%%%%%%%%%%%%%%%%%%%%%%%%%%%%%
%%%%
%%%% The (original) Douay Rheims Bible 
%%%%
%%%% New Testament
%%%% Epistles
%%%% Romans
%%%% Chapter 01
%%%%
%%%%%%%%%%%%%%%%%%%%%%%%%%%%%%%%%%%%%%%%%%%%%%%%%%%%%%%%%%%%%%%%%




\startcomponent chapter-01


\project douay-rheims


%%% 2632
%%% o-2482
\startChapter[
  title={Chapter 1}
  ]

\Summary{The foundation of his Apoſtleship being laid, 8.~he highly
  commendeth the Romanes, and proteſteth his affection towards them. And
  ſo comming to the matter, faith, our Chriſtian Catholike doctrine
  (that teacheth al to beleeue) to be the way to ſaluation: becauſe the
  Gentils (firſt of al) could not be ſaued by their Philoſophie, whereby
  they knew God, forſomuch as they did not ſerue him, but Idols; he
  therfore iuſtly permitting them to fal into al kind of moſt damnable
  ſinne.}

Paul the ſeruant of \Sc{Iesvs Christ}, called to be an Apoſtle,
\CNote{\XRef{Act.~13,~2.}}
ſeparated vnto the Ghoſpel of God, \V which before he had promiſed by
his Prophets in the holy Scriptures, \V of his Sonne, (who was made to
him of the ſeed of Dauid according to the fleſh, \V who was predeſtinate
the Sonne of God in power, according to the ſpirit of ſanctification, by
the reſurrection of our Lord \Sc{Iesvs Christ} from the dead, \V by whom
we receiued grace and Apoſtleſhip
\SNote{Faith muſt not be ſubiect to ſenſe, reaſõ, arguing or
vnderſtãding, but muſt commãd, & be obeid in humilitie and ſimplicitie.}
\TNote{\G{εἰς ὑπακοὴν πίστεως}}
for obedience to the faith
\SNote{S.~Auguſtin vſeth this place and the like againſt Heretikes,
which would draw the common Catholike faith of al Nations, to ſome
certaine countries or corners of the world.
\Cite{Aug. ep.~161.}}
in al Nations for the name of him, \V among whom are you alſo the called
of \Sc{Iesvs Christ}:) \V to al that are at Rome the beloued
\Fix{of of}{of}{obvious typo, fixed in other}
God, called to be Saints. 
%%% !!! Not marked in text.
\LNote{Grace to you & peace.}{It
\MNote{Apoſtolical ſalutatiõ or bleſſing.}
is a kind of bleſsing rather then a prophane ſalutatiõ, proper to the
Apoſtles, of greater vertue thẽ the benedictiõs of the Fathers in the
old Teſtament. The holy Fathers of the Church ſeemed to abſteine from it
for their reuerence to the Apoſtles.
\CNote{\Cite{Epiph. har.~66.}}
\MNote{The ſame vſed of Heretikes.}
The Manichees 
\Cite{(Auguſt. cont. ep. fundæ. c.~5.~6.)}
and other Heretikes (as alſo theſe of our time) becauſe they would be
counted Apoſtles, often vſe it.}
Grace to you and peace from God our Father,
and our Lord \Sc{Iesvs Christ}.

\V Firſt I giue thanks to my God through \Sc{Iesvs Christ} for al you,
becauſe
\LNote{Your faith renowmed.}{The holy Doctours vpon theſe words of
the Apoſtle, and ſpecially by our Maiſters promiſe
\CNote{\XRef{Luc.~22.}}
made to Peter, that his faith ſhould not faile, giue great teſtimonie
for the prouidence of God in the preſeruation of the Romane
faith. S.~Cyprian thus:
\Cite{ep.~51. nu.~6.}
\MNote{The Romane faith highly commended.}
\Emph{They are ſo bold to cary letters from prophane Schiſmatikes to the
chaire of Peter and the principal Church whence Prieſtly vnitie roſe:
not conſidering the Romanes to be them whoſe faith (the Apoſtle being
the commender) was praiſed, to whom miſbeleefe can not haue acceſſe.} So
\Cite{S.~Hierom Apolog. adu. Ruff. li.~3. c.~4. to.~2.}
\MNote{It can not faile nor be corrupted.}
\Emph{Know you that the Romane faith commended by the Apoſtles mouth,
wil receiue no ſuch deceites, nor can be poſsibly changed, though an
Angel taught otherwiſe, being fenſed by S.~Paules authoritie.} Againe
\Cite{ep.~63. ad Pammach}
&
\Cite{Oceanum c.~4. to.~2.}
\Emph{Whatſoeuer thou be that auoucheſt new ſectes, I pray thee haue
reſpect to the Romane eares, ſpare the faith which was praiſed by the
Apoſtles voice.} And in another place:
\CNote{\Cite{Proæm. l.~2. Com. in ep. ad Gal.}}
\Emph{Wil ye know, ô Paula, and Fuſtochium, how the Apoſtle hath noted
euery prouince with their proprieties? the faith of the people of Rome
is praiſed.
\MNote{The Romane ſtations, a tokẽ of greater faith and deuotion.}
Where is there ſo great concourſe to Churches and Martyrs
ſepulchers? Where ſoundeth, Amos, like thunder from heauen, or where are
the temples (void of Idols) ſo shaken as there? Not that the Romanes
haue another faith then the reſt of the Chriſtian Churches, but that there
is in them more deuotion and ſimplicitie of faith.} In another place the
ſame Holy Doctour ſignifieth that it is al one to ſay, the Romane faith,
and the Catholike.
\Cite{Apolog.~1. adu. Ruff. c.~1.}
So doth 
\Cite{S.~Cyprian. ep.~12. num.~1. ad Antonianum}:
and 
\Cite{S.~Ambroſe de obitu fratris, in med.}
\MNote{The Catholike and Romane faith al one.}
Whereupon, this word, \Emph{Romane}, is added to \Emph{Catholike}, in
many countries where Sectes doe abound, for the better diſtinction of
true beleeuers from Heretikes: which in al Ages did hate and abhorre the
Romane faith and Church, as al malefactours doe their Iudges and
correctours.}
your faith is renowmed in the whole world. \V For God is my witnes,
\TNote{\G{ᾧ λατρεύω }}
whom I ſerue
\LNote{Serue in ſpirit.}{Diuerſe Heretikes when they heare that God is a
ſpirit, and muſt be ſerued and adored in ſpirit, imagin that he muſt be
honoured only inwardly, without ceremonies & external workes: which you
ſee is otherwiſe, for that the Apoſtle ſerued God in ſpirit, by
preaching the Ghoſpel.
\MNote{How God is ſerued in ſpirit.}
To ſerue God then in ſpirit, is to ſerue him with faith, hope, and charitie,
and with al workes proceeding of them: as to ſerue him carnally, is,
with workes external, without the ſaid internal vertues.}
in my ſpirit in the Ghoſpel of his Sonne, that
\SNote{He praieth without intermiſsion that omitteth no day certaine
times of praier.
\Cite{Aug. hæreſ.~57.}}
without intermiſſion I make
\LNote{A memorie of you.}{A
\MNote{Praier for conuerſion of ſoules.}
great example of charitie for al men, ſpecially for Prelates & Paſtours,
not only to preach, but to pray continually for the conuerſion of people
to Chriſts faith: Which the Apoſtle did for them whom he neuer knew, in
reſpect of God's honour only and the zeale of ſoules.}
a memorie of you \V alwaies in my praiers, beſeeching, if by any meanes
I may ſometime at the length haue a proſperous iourney by the wil of
God, to come vnto you.
%%% o-2483
\V For I deſire to ſee you, that I may impart vnto you ſome ſpiritual
grace, to 
\SNote{The Romanes were conuerted & taught by S.~Peter before. Therefore
he vſeth that ſpeach, to cõfirme them in their faith.
\Cite{Authour Com apud Hier. Theodoret. in 16.~Rom. & Chryſ.}}
confirme you: \V that is to ſay, to be comforted together in you by that
which is common to vs both, your faith & mine. \V And I wil not haue you
ignorãt (Brethren) that I haue often purpoſed to come
%%% 2633
vnto you (and haue been ſtaied hitherto) that I may haue ſome fruit in
you, as alſo in the other Gentils. \V To the Greeks & the Barbarous, to
the wiſe and the vnwiſe I am debter. \V So (as much as is in me) I am
ready
\LNote{To euangelize.}{The
\MNote{The Ghoſpel is not only the written word.}
Ghoſpel is not only the life of our Sauiour written by the foure
Euangeliſts, nor only that which is written in the new Teſtament: but
their whole courſe of preaching & teaching the faith. Which faith
commeth ordinarily of preaching & hearing, and not of writing or
reading. And therfore S.~Paul thought not himſelf diſcharged by writing
to the Romanes, but his deſire was to preach vnto them: for that was the
proper comiſsion giuen to the Apoſtles,
\CNote{\XRef{Mat.~28.}}
to preach to al Nations.
\MNote{The Apoſtles writing, and preaching, whether more neceſſarie, and
how.}
The writing of the bookes of the Teſtament, is another part of God's
prouidence, neceſſarie for the Church in general, but not neceſſarie for
euery man in particular: as to be taught and preached vnto, is for euery
one of age and vnderſtanding.  And therfore S.~Peter (who was the cheefe
of the Commiſsion wrote litle; many of them wrote nothing at al: and
S.~Paul that wrote moſt, wrote but litle in compariſon of his preaching)
not to any but ſuch as were conuerted to the faith by preaching before.}
to euangelize to you alſo that are at Rome.

\V For I am not aſhamed of the Ghoſpel. For it is the power of God, vnto
ſaluation to euery one that beleeueth, to the Iewes firſt and to the
Greeke. \V For
\SNote{He meaneth not God's owne iuſtice in himſelf, but that iuſtice
wherwith God indoweth man when he iuſtifieth him.
\Cite{Aug. de Sp. & lit. c.~9.}
Whereby you may gather the vanitie of the Heretical imputatiue iuſtice.}
the iuſtice of God is reuealed therein by faith into faith; as it is
written:
\CNote{\XRef{Abac.~2,~4.}}
\Emph{And the iuſt
\LNote{Liueth by faith.}{In the
\XRef{10.~to the Hebrewes},
he ſheweth by this place of the Prophet
\XRef{(Abacuc.~2.)}
that the iuſt though he liue here in peregrination, and ſeeth not
preſently nor enioyeth the life euerlaſting promiſed to him, yet holdeth
faſt the hope therof by faith. In this place he applieth the Prophets
wordes further to this ſenſe,
\MNote{The Catholike or Chriſtian faith with good workes iuſtifieth, &
without this faith, no workes whatſoeuer.}
That it is our faith, that is to ſay, \Emph{the Catholike beleefe}
(ſaith 
\Cite{S.~Auguſtin li.~3. cont.~2. ep. Pelag.})
\Emph{which maketh a iuſt man, and diſtinguisheth between the iuſt and
vniuſt}; and that by the law of faith, and not by the law of
workes. Whereof it riſeth, that the Iew, the Heathen Philoſopher, and
the Heretike, though they excelled in al workes of moral vertues, could
not yet be iuſt: and a Catholike Chriſtian man liuing but an ordinarie
honeſt life, either not ſinning greatly, or ſupplying his faults by
penãce, is iuſt. And this difference riſeth by faith. Not that faith can
ſaue any man without workes, \Emph{For it is not a reprobate faith that
we ſpeake of}, (as the holy Doctour ſaith) \Emph{but that which worketh
by charitie}, and therfore remitteth ſinnes and maketh one iuſt. See
\Cite{S.~Auguſtines place}.}
liueth by faith.}

\V For the wrath of God from Heauen
\LNote{Is reuealed.}{By
\MNote{Not only by faith.}
al the paſſage following you may ſee, that the Ghoſpel and Chriſt's law
conſiſteth not only in preaching faith (though that be the ground, & is
firſt alwaies to be done) but to teach vertuous life and good workes,
and to denounce damnation to al them that commit deadly ſinnes & repent
not. And againe we ſee that not only lacke of faith is a ſinne, but al
other actes done againſt God's commandements.}
is reuealed, vpon al impietie and iniuſtice of thoſe men that deteine
the veritie of God in iniuſtice: \V becauſe, that of God which is
knowen, is manifeſt in them. For God hath manifeſted it vnto them. \V
For his inuiſible things, from the creation of the world are ſeen, being
vnderſtood by thoſe things that are made; his eternal power alſo and
Diuinitie: ſo that they are inexcuſable. \V Becauſe whereas they knew
God, they haue not glorified him as God, or giuen thanks: but are become
vaine in their cogitations, and their fooliſh hart hath been darkned. \V
For, ſaying themſelues to be wiſe, they became fooles. \V And they
changed the glorie of the incorruptible God, into a
\SNote{Loe theſe & the like are the Images or Idols ſo often condemned
in the ſcriptures, & not the holy Images of Chriſt and his Saints.}
ſimilitude of the image of a corruptible man, and of foules and
foure-footed beaſts and of them that creep. \V (For the which cauſe God
\SNote{\XRef{Eph.~4,~19.}
He ſaith, \Emph{They haue deliuered or giuen vp themſelues to al
vncleanneſſe.} By 
which cõferẽce of ſcriptures we learne that thẽſelues are the cauſe of
their owne ſinne and damnation, God of his iuſtice permitting & leauing
them to 
their owne wil, and ſo giuing them vp into paſsions &c.}
hath deliuered them vp vnto the deſires of their hart, into
vncleanneſſe, for to abuſe their owne bodies among themſelues
ignominiouſly.) \V Who haue changed the veritie of God into lying: and
haue worſhipped &
\TNote{\G{ἐλάτρευσαν}}
ſerued the creature rather then the Creatour, who is bleſſed for
euer. Amen. \V Therfore
\LNote{Hath deliuered them vp.}{As
\MNote{God is not the authour of ſinne.}
he ſaith here, God deliuered them vp, ſo to the Epheſians
\XRef{(c.~4,~19.)}
he ſaith of the ſame perſons and things: They deliuered themſelues vp to
al vncleanneſſe. So that it is not meant here that God doth driue,
force, or cauſe any man to ſinne, as diuers blaſphemous Heretikes doe
hold; but only that by his iuſt iudgement, for their owne deſeruing,
\MNote{God puniſheth ſinne by permitting men to fal further and
further.}
and for due puniſhment of their former grieuous offenſes, he withholdeth his
grace from them, and ſo ſuffreth them to fal further into other
ſinnes. As, for their crime of Idolatrie, to ſuffer them to fal into
vnnatural abominations: as now for hereſie, he taketh his grace and
mercie from many, and ſo they fal headlong into al kind of turpitude: as
contrariewiſe, for il life, he ſuffreth many to fal into hereſie. And
for Chriſt's ſake let euery one that is entãgled with the Idolatrie of
this time, that is to ſay, with theſe new Sectes, looke wel into his
owne conſcience, whether his forſaking the true God, may not come vnto
him for a puniſhment of his former or preſent il life which he liueth.}
God hath deliuered them into paſſiõs of ignominie. For their women haue
changed the natural vſe, into that vſe that is contrarie to nature. \V
And in like manner the men alſo, leauing the natural vſe of the woman, haue
burned in their deſires one toward another, men vpon men working
turpitude, & the reward of their errour (which they ſhould) receauing in
themſelues. \V And as
%%% o-2484
they liked not to haue God in knowledge; God deliuered them vp into a
reprobate ſenſe, to doe thoſe things that are not conuenient: \V
repleniſhed with al iniquitie, malice, fornication, auarice, wickednes,
ful of enuie, murder, contention, guile, malignitie, whiſperers, \V
detractours, odible to God, contumelious, proud, hawtie, inuentours of
euil things, diſobedient to parents, \V fooliſh, diſſolute, without
affection, without fidelitie, without mercie. \V Who whereas they knew
the iuſtice of God, did not vnderſtand that they which doe ſuch things,
are
\LNote{Worthie of death.}{Here
\MNote{Sinnes mortal and venial.}
you ſee why the Church taketh ſome ſinnes to be deadly, and calleth them
mortal: to wit, becauſe al that doe them, are worthy of damnation:
others be venial, that is to ſay, pardonable of their owne nature and
not worthie of eternal damnation.}
worthie of death: not only they that doe them, but they alſo that
conſent to the doers.


\stopChapter


\stopcomponent


%%% Local Variables:
%%% mode: TeX
%%% eval: (long-s-mode)
%%% eval: (set-input-method "TeX")
%%% fill-column: 72
%%% eval: (auto-fill-mode)
%%% coding: utf-8-unix
%%% End:

