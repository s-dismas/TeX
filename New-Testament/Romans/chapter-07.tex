%%%%%%%%%%%%%%%%%%%%%%%%%%%%%%%%%%%%%%%%%%%%%%%%%%%%%%%%%%%%%%%%%
%%%%
%%%% The (original) Douay Rheims Bible 
%%%%
%%%% New Testament
%%%% Epistles
%%%% Romans
%%%% Chapter 07
%%%%
%%%%%%%%%%%%%%%%%%%%%%%%%%%%%%%%%%%%%%%%%%%%%%%%%%%%%%%%%%%%%%%%%

%%% Latin checked by KK.




\startcomponent chapter-07


\project douay-rheims


%%% 2646
%%% o-2497
\startChapter[
  title={Chapter 7}
  ]

\Summary{Our former husband (ſinne) with his law, is dead in Baptiſme:
  and now we are maried to another husband (to Chriſt) to bring forth
  children to God, that is good workes. 7.~And how the Law being good,
  was yet to vs the law of ſinne and death, becauſe concupiſcence
  reigned in vs. 17.~But now by Baptiſme grace reigneth in vs, though
  alſo concupiſcence doth remaine and tempt vs ſtil.}

Are you ignorant, Brethren, (for I ſpeake to them that know the Law)
that the Law hath dominion ouer a man as long time as
%%% 'he' in text of other, hard to read in heretikes, but looks like a Var.
\Var{it}{he}
liueth? \V For
\CNote{\XRef{1.~Cor.~7,~39.}}
the womã that is vnder a husbãd,
\SNote{Nothing but death diſſolueth the band betwixt man & wife: though
for fornication one may depart from anothers companie. Therfore to marry
againe is aduoutrie, during the life of the partie ſeparated.}
her husbãd liuing is boũd to the law. But if her husband be dead she is
looſed frõ the law of her husbãd. Therfore her husbãd liuing, ſhe ſhal
be
%%% 2647
called an aduoutreſſe if ſhe be with another man: but if her 
\Fix{busband}{husband}{obvious typo, fixed in other}
be dead ſhe is deliuered from the law of her husband: ſo that ſhe is not
an aduoutreſſe if ſhe be
%%% o-2498
with another man. \V Therfore my Brethren
\SNote{Being now baptiſed and dead to ſinne, & engrafted in Chriſt's
myſtical body, you are diſcharged of the Law of Moyſes, and are free in
Chriſt.}
you alſo are made dead to the Law by the body of Chriſt: that you may be
another man's who is riſen againe from the dead, that we may fructifie
to God. \V For when we were in the flesh, the paſsions of ſinnes that
were by the Law, did worke in our members, to fructifie vnto death. \V
But now we are looſed from the law of death wherin we were deteined:
\TNote{\G{ὥστε δουλεύειν}}
in ſo much we ſerue in
\SNote{By Baptiſme we haue not Chriſtes iuſtice imputed to vs, but an
inward newneſſe of ſpirit giuen vs and reſident in vs.}
newneſſe of ſpirit, and not in the oldnes of the letter.

\V What ſhal we ſay then? Is the Law ſinne? God forbid. But ſinne I did
not know, but by the Law: for concupiſcence I knew not, vnleſſe the Law
did ſay:
\LNote{Thou shalt not couet.}{It
\MNote{Actual concupiſcence forbidden, not habitual.}
is not the habitual concupiſcence or infirmitie of our nature or ſenſual
deſire or inclination to euil, coueting againſt the ſpirit, that is
forbidden properly in this precept: but the conſent of our reaſon and
mind vnto it, to obey and follow the luſts therof, that is a ſinne and
prohibited.}
\CNote{\XRef{Exo.~20,~17.}
\XRef{Deu.~5,~21.}}
\Emph{Thou shalt not couet.} \V But
\SNote{Sinne or cõcupiſcence which was aſleep before, was wakened, by
prohibitiõ; the Law not being the cauſe therof, nor giuing occaſion
therunto, but occaſion being taken by our corrupt nature to reſiſt that
which was commanded.}
occaſion being taken, ſinne by the commandement wrought in me al
concupiſcence. For without the Law ſinne was dead. \V And I liued
without the Law ſometime. But when the commandement was come, ſinne
reuiued. \V And I was dead. And the commandement, that was vnto life,
the ſame to me was found to be vnto death. \V For ſinne taking occaſion
by the commandement, ſeduced me, and by it killed me. \V Therfore
\CNote{\XRef{1.~Tim.~1,~8.}}
the Law indeed is holy, and the commandement holy, and iuſt, and good.

\V That then which is good, to me was it made death? God forbid. But
ſinne, that it may apeare ſinne, by the good thing wrought me death:
that ſinne might become ſinning aboue meaſure by the commandement. \V
For we know that the Law is ſpiritual, but I am carnal, ſold vnder
ſinne. \V For
\LNote{That which I worke.}{This
\MNote{Sodain inuoluntarie motions are no ſinne.}
being vnderſtood of S.~Paul himſelf or any other iuſt perſon, the ſenſe
is, that the flesh and inferiour part ſtirreth vp diuerſe diſordered
motions and paſsions or pertubations againſt the mind, and vpon ſuch a
ſodain ſometimes inuadeth the ſame, that before it attendeth or reaſon
can gather itſelf to deliberate, man is in a ſort (though vnwittingly)
entangled. Which as ſoone as it is perceiued, being of the iuſt
condemned, reiected, and reſiſted, neuer maketh him a ſinner.}
that which I worke, I vnderſtand not. For
\LNote{Not that which I wil.}{He
\MNote{Concupiſcence taketh not away free-wil.}
meaneth not, that he can doe no good that he willeth or deſireth, or
that he is euer forced to doe that which his wil agreeth vnto: but that
by reaſon of the forcibleneſſe of concupiſcence, wherof he can not rid
himſelf during life, he can not accomplish al the deſires of his ſpirit
and mind, according as he ſaith to the Galations:
\CNote{\XRef{c.~5,~17.}}
\Emph{The flesh coueteth againſt the ſpirit, and the ſpirit againſt the
flesh, that not whatſoeuer you wil, you can doe.}}
not that which I wil, the ſame doe I, but which I hate, that I
doe. \V And if that which I wil not, the ſame I doe; I conſent to the
Law, that it is good.

\V But now, not I worke it any more, but the ſinne that dwelleth in
me. \V For I know that there dwelleth not in me, that is to ſay, in my
fleſh, good. For to wil, is preſent with me, but to accompliſh that
which is good, I find not. \V For
\LNote{Not the good which I wil.}{So
\MNote{Sinne is voluntarie, and, otherwiſe it is no ſinne.}
may the iuſt alſo be forced by the rage of concupiſcence or ſenſual
appetite, to doe or ſuffer many things in his inferiour part or external
members, which his wil conſenteth not vnto. And ſo long it is ſo
farre from ſinne, that (as 
\CNote{\Cite{Ep. ad Aſellicum~200.}}
S.~Auguſtine ſaith) he need neuer ſay to God, \Emph{forgiue vs our
ſinnes}, for the ſame. For, ſinne is voluntarie, and ſo be not theſe
paſsions.}
not the good which I wil, that doe I; but the euil
\LNote{Which I wil not.}{It maketh not any thing againſt free-wil that
the Apoſtle ſaith, that good men doe or ſuffer ſometimes in their
bodies, that which the wil agreeth not vnto; but it proueth plainely
free-wil: becauſe the proper act therof, that is, to wil or nil, to
conſent or diſſent, is euer (as you may ſee here) free in it ſelf: though
there may be internal or external force to ſtay the members of a man,
that they obey not in euery act, that which the wil commandeth or
preſcribeth. And therfore that is neuer imputed to man which he doth in
his external or internal faculties, when wil concurreth not. Yea
afterward
\XRef{(v.~20.)}
the Apoſtle ſaith, \L{Non ego operor}, man doeth not that which is not
done by his wil: which doth moſt euidently proue free-wil. Al which
S.~Auguſtin cleerly teacheth
\Cite{to.~4. in expoſition: quarundam propoſ. ad
Rom. propos.~43.~45. and~46.}
and in manie other places.}
which I wil not, that I doe. \V And if that which I wil not, the ſame I
doe: now not I worke it, but the ſinne that dwelleth in me. \V I find
therfore the Law, to me hauing a wil to doe good, that euil is preſent
with me. \V For I am delighted with the Law of God according to the
inward man: \V but I ſee another law in my members, repugning to the law
of my mind, and captiuing me in the law of ſinne that
%%% o-2499
is in my members. \V Vnhappie man that I am, who ſhal deliuer me from
the body of this death? \V The grace of God by \Sc{Iesvs Christ} our
Lord. Therfore I my ſelf
%%% !!! This LNote combined with following
\LNote{With the mind, with the flesh.}{Nothing
\MNote{Concupiſcence defileth not a iuſt man's actions as the Lutherans
ſay.}
done by concupiſcence (which the Apoſtle here calleth ſinne) whereunto
the ſpirit, reaſon, or mind of man conſenteth not, can make him guilty
before God. Neither can the motions of the flesh in a iuſt man euer any
whit defile the operations of his ſpirit, as the Lutherans doe hold: but
make them often more meritorious, for the continual combat that he hath
with them. For it is plaine that the operations of the flesh and of the
ſpirit doe not concurre together to make one act, as they imagine; the
Apoſtle concluding cleane contrarie. That in mind he ſerueth the Law of
God, in flesh the law of ſinne, that is to ſay, concupiſcence.}
with the mind
\TNote{\G{δουλεύω}}
ſerue the law of God, but
%%% !!! This LNote part of above.
%%% \LNote{}{}
with the fleſh, the law of ſinne.


\stopChapter


\stopcomponent


%%% Local Variables:
%%% mode: TeX
%%% eval: (long-s-mode)
%%% eval: (set-input-method "TeX")
%%% fill-column: 72
%%% eval: (auto-fill-mode)
%%% coding: utf-8-unix
%%% End:

