%%%%%%%%%%%%%%%%%%%%%%%%%%%%%%%%%%%%%%%%%%%%%%%%%%%%%%%%%%%%%%%%%
%%%%
%%%% The (original) Douay Rheims Bible 
%%%%
%%%% New Testament
%%%% Epistles
%%%% Romans
%%%% Chapter 08
%%%%
%%%%%%%%%%%%%%%%%%%%%%%%%%%%%%%%%%%%%%%%%%%%%%%%%%%%%%%%%%%%%%%%%




\startcomponent chapter-08


\project douay-rheims


%%% 2649
%%% o-2499
\startChapter[
  title={Chapter 8}
  ]

\Summary{That now after Baptiſme we are no more in ſtate of damnation,
  becauſe by the grace which we haue receiued, we are able to fulfil the
  Law; vnles we doe wilfully giue the dominion againe to
  concupiſcence. 18.~Then (becauſe of the perſecutions that then were)
  he comforteth and exhorteth them with many reaſons.}

%%% o-2500
There is now therfore no damnation to them that are in
Chriſt \Sc{Iesvs}; that walke not according to the flesh. \V For the Law
of the ſpirit of life in Chriſt \Sc{Iesvs}, hath deliuered me from the
law of ſinne and of death. \V For that which was impoſsible to the Law,
in that it was weakned by the flesh; God ſending his Sonne in the
ſimilitude of the flesh of ſinne, euen of ſinne damned ſinne in the
flesh, \V that
\SNote{This conuinceth againſt the Churches Aduerſaries, that the law,
that is, God's cõmandements may be kept, & that the keeping therof is
iuſtice, & that in 
\Fix{chriſtia}{chriſtian}{obvious typo, fixed in other}
men that is fulfilled by Chriſt's grace which by the force of the Law
could neuer be fulfilled.}
the iuſtification of the Law might be fulfilled in vs who walke not
according to the flesh, but according to the ſpirit. \V For they that
are according to the flesh, are affected to the things that are of the
flesh; but they that are according to the ſpirit are affected to the things
that are of the Spirit. \V For the wiſedom of the flesh is death; but
the wiſedom of the ſpirit, life and peace. \V Becauſe the wiſedom of the
flesh, is
\Var{an enemie}{enmitie}
to God: for to the Law of God it is not ſubiect, neither can it be. \V
And they that are in the flesh, can not pleaſe God. \V But you are not
in the flesh, but in the Spirit, yet if the Spirit of God dwel in
you. But if any man haue not the Spirit of Chriſt, the ſame is not
his. \V But if Chriſt be in you; the body indeed is dead becauſe of
ſinne, but the Spirit liueth becauſe of iuſtification. \V And if the
Spirit of him that raiſed vp \Sc{Iesvs} from the dead, dwel in you; he
that raiſed vp \Sc{Iesvs Christ} from the dead, shal quicken alſo your
mortal bodies, becauſe of his Spirit dwelling in you. \V Therfore
Brethren, we are debters, not to the flesh, to liue according to the
flesh. \V For if you liue according to the flesh, you shal die. But if
by the Spirit, you mortifie the deeds of the flesh, you shal liue. \V
For whoſoeuer
\SNote{He meaneth not that the Children of God be violẽtly compelled
againſt their wills, but that they be ſweetly drawen, moued, or induced
to doe good.
\Cite{Aug. Enchirid. c.~64.}
\Cite{De verb. Do. ſer. 41. c.~7.}
&
\Cite{de verb. Apoſt. ſer.~13. c.~11.~12.}}
are led by the Spirit of God they are the ſonnes of God. \V For
\CNote{\XRef{2.~Tim.~1,~7.}}
you haue not receiued the ſpirit of ſeruitude againe in feare; but
\CNote{\XRef{Gal.~4,~5.}}
you haue receiued the ſpirit of adoption of ſonnes, wherin we crie:
Abba, (Father). \V For
\LNote{The Spirit giueth teſtimonie.}{This
\MNote{The teſtimonie of the Spirit.}
place maketh not for the Heretikes ſpecial faith, or their preſumptuous
certainty that euery one of them is in grace; the teſtimonie of the
Spirit being nothing els but the inward good motions, cõfort, &
contentment, which the children of God doe daily feele more and more in
their harts by ſeruing him: by which they haue as it were an atteſtation
of his fauour towards them, whereby the hope of their iuſtification and
ſaluation is much corroborated and ſtrengthned.}
the Spirit himſelf, giueth teſtimonie to our ſpirit that we are the ſonnes
of God. \V And if ſonnes, heires alſo; heires truly of God, and coheires
of Chriſt:
\LNote{Yet if we ſuffer.}{Chriſtes
\MNote{Notwithſtanding Chriſt's ſatisfaction & Paſsion, yet ours alſo is
required.}
paines or paſsions haue not ſo ſatisfied for al, that Chriſtiã men be
diſcharged of their particular ſuffring or ſatisfying for each man's
owne part: neither be our paines nothing worthy to the attainement of
Heauen, becauſe Chriſt hath done enough; but quite contrarie: he was by
his Paſsion exalted to the glorie of Heauen; therfore we by compaſsion
or partaking with him in the like paſsions, shal attaine to be fellowes
with him in his Kingdom.}
yet if we ſuffer with him, that we may be alſo glorified with him.

\V For I thinke that the paſsions of this time are not
\LNote{Condigne.}{Our
\MNote{Al ſuffring in this life is nothing in compariſon of the heauenly
glorie, and yet it is meritorious and worthy of the ſame.}
Aduerſaries ground hereon, that the workes or ſufferances of this life
be not meritorious or worthy of life euerlaſting; where the Apoſtle
ſaith no ſuch thing, no more then he ſaith that Chriſt's Paſsions be not
meritorious of his glorie, which I thinke they dare not much auouch in
our Sauiour's actions. He expreſſeth only, that the very afflictions of
their owne nature, which we ſuffer with or for him, be but short,
momentanie, and of no account in compariſon of the recompence which we
shal haue in heauen. No more indeed were Chriſtes paines of their owne
nature, compared to his glorie, any whit comparable: yet they were
meritorious or worthy of Heauen; & ſo be ours. And therfore to expreſſe
the ſaid compariſon, here he ſaith, \Emph{They are not condigne
\TNote{\L{ad glorium,} \G{πρὸς τὴν δόξαν.}}
to the glorie.} He ſaith not, \Emph{of the glorie}, as the Heretikes
falſly tranſlate: though the Scripture ſpeaketh ſo alſo, when it
ſignifieth only a compariſon: as
\XRef{Prou.~3.}
in the Greeke, \L{Omne pretioſum non eſt
\TNote{\G{ἀξιον ἀυτῆς}}
illa dignum.} S.~Auguſtin, \L{illi dignum}. S.~Hierom, \L{non vales huic
comparari}: that is, No pretious thing is worthie of wiſedom, or to be
compared with it. See the like
\XRef{Eccle.~26,~20.}
\XRef{Tob.~9,~2.}
But when the Apoſtle wil expreſſe that they are condigne, worthy, or
meritorious of the glorie, he ſaith plainely: 
\CNote{2.~Cor.~4,~17.}
\Emph{That our tribulation which preſently is monentanie and light,
worketh aboue meaſure exceedingly an eternal weight of glorie in vs.}
\MNote{Whence the merit of workes riſeth.}
The valew of Chriſtes actions riſeth not of the length or greatnes of
them in themſelues, though ſo alſo they paſſed al mens doings: but of
the worthines of the Perſon. And ſo the value of ours alſo riſeth of the
grace of our adoption, which maketh thoſe actions that of their natures be
not meritorious nor anſwerable to the ioyes of Heauen in themſelues, to
be worthy of Heauen. And they might as wel proue that the workes of
ſinne doe not demerit damnation: for ſinne indeed for the quantity and
nature of the worke, is not anſwerable in pleaſure to the paine of Hel:
but becauſe it hath a departing or an auerſion from God, be it neuer ſo
short, it deſerueth damnation, becauſe it alwaies proceedeth from the
enemy of God, as good workes that be meritorious, proceed from the child
of God.}
\TNote{\L{condigna ad gloriam.}}
condigne to the glorie to come that ſhal be reuealed in vs. \V For the
expectation of the creature, expecteth the reuelation of
%%% o-2501
the ſonnes of God. \V For the creature is made ſubiect to vanitie, not
willing, but for him that made it ſubiect in hope: \V becauſe the
creature alſo itſelf ſhal be deliuered from the ſeruitude of corruption,
into the libertie of the glorie of the children of God. \V For we know
that euery creature groneth, & trauaileth euen til now. \V And not only
it, but we alſo our ſelues
%%% 2650
hauing the firſt fruits of the ſpirit, we alſo grone within our ſelues,
expecting the adoption of the ſonnes of God, the redemption of our
body. \V For
\LNote{By hope ſaued.}{That
\MNote{As ſometime faith only is named, ſo elſwhere only hope, & only
charitie, as the cauſe of our ſaluation.}
which in other places he attributeth to faith, is here attributed to
hope. For whenſoeuer there be many cauſes of one thing, the holy Writers
(as matter is miniſtred & occaſion giuen by the doctrine then handled)
ſometimes referre it to one of the cauſes, ſometime to another: not by
naming one alone, to exclude the other, as our Aduerſaries captiouſly &
ignorantly doe argue; but at diuers times and in ſundrie places to
expreſſe that, which in euery diſcourſe could not, nor needed not to be
vttered. In ſome diſcourſe, faith is to be recommended; in others,
charitie; in another, hope; ſometimes, almes, mercie; elſwhere, other
vertues. One while, \Emph{Euery one that beleeueth, is borne of God.}
\XRef{1.~Io.~5,~1.}
Another while, \Emph{Euery one that loueth, is borne of God.}
\XRef{1.~Io.~4,~7.}
Sometimes, \Emph{faith purifieth man's hart.}
\XRef{Act.~15,~9.}
And another time, \Emph{Charitie remitteth ſinnes.}
\XRef{1.~Pet.~4,~8.}
Of faith it is ſaid, \Emph{The iuſt liueth by faith.}
\XRef{Ro.~1,~17.}
Of charitie, \Emph{We know that we are transferred from death to life,
becauſe we loue &c.}
\XRef{1.~Io.~3,~14.}}
by hope we are ſaued. But hope that is ſeen, is not hope. For that which
a man ſeeth, wherfore doth he hope it? \V But if we hope for that which
we ſee not; we expect by patience. \V And in like manner alſo the Spirit
helpeth our infirmitie. For, what we ſhould pray as we ought, we know
not: but the Spirit himſelf requeſteth for vs with gronings
vnſpeakeable. \V And he that ſearcheth the harts, knoweth what
\LNote{The Spirit deſireth.}{Arius
\MNote{Scripture abuſed againſt the Godhead of the Holy Ghoſt.}
and Macedonius, old Heretikes, had their places to contend vpon againſt
the Churches ſenſe, as our new Maiſters now haue. They abuſed this text
to proue the Holy Ghoſt not to be God, becauſe he needed not to pray or
aske, but he might command if he were God. Therfore S.~Auguſtin
expoundeth it thus: \Emph{The Spirit prayeth}, that is, \Emph{cauſeth &
teacheth vs to pray, and when to pray, and what to pray, or aske.}
\Cite{Auguſt. de anima & cius orig. li.~4. c.~9.}
&
\Cite{ep.~121. c.~11.}}
the Spirit deſireth: becauſe according to God he requeſteth for the
Saints. \V And we know that to them that loue God, al things cooperate
vnto good, to ſuch as according to purpoſe are called to be Saints. \V
For whom he hath foreknowen, he hath alſo predeſtinated to be made
conformable to the image of his Sonne: that he might be the Firſt-borne
in many Brethren. \V And
\LNote{Whom he hath predeſtinated.}{God's
\MNote{The doctrine of predeſtination, how to be reuerenced, & what it
teacheth vs.}
eternal foreſight, loue, purpoſe, predeſtination, and election of his
deere children, & in time their calling, iuſtifying, glorifying by
Chriſt, as al other actes & intentions of his diuine wil and prouidence
towards their ſaluation, ought to be reuerenced of al men with dreadful
humilitie, & not to be ſought out or diſputed on with preſumptuous
boldnes and audacitie. For it is the gulfe that many proud perſons, both
in this Age and alwaies, haue by God's iuſt iudgement perished in,
founding theron moſt horrible blaſphemies againſt God's mercie, nature,
and goodnes, and diuers damnable errours againſt man's free-wil, &
againſt al good life & religion. This high concluſion is here ſet downe
for vs, that we may learne to know of whom we ought to depend in al our
life, by whom we expect our ſaluation, by whoſe prouidence al our
graces, guifts, and workes doe ſtand: by what an euerlaſting gratious
determination, our redemption, which is in Chriſt \Sc{Iesvs}, was
deſigned: and to giue God inceſſable thankes for our vocation and
preferment to the ſtate we be in, before the Iewes, who deſerued no
better then they, before the light of his mercie shining vpon vs
accepted vs, and reiected them.
\MNote{God's predeſtination taketh not away free-wil.}
But this ſaid eminent truth of God's eternal predeſtination ſtandeth (as
we are bound to beleeue vnder paine of damnation, whether we vnderſtand
how or no) & ſo S.~Auguſtin in al his diuine workes written of the ſame 
%%% !!! Theſe are probably wrong (or at least wrongly split up.
\Cite{(De gratia.}
&
\Cite{lib. arb. de corrept. & gratia.}
\Cite{Ad articulos ſalſio impoſitos.)}
defendeth, declareth, proueth, and conuinceth, that it doth ſtand (I
ſay) with man's free-wil and the true libertie of his actions, and
forceth no man to be either il or good, to ſinne or vertue, to ſaluation
or damnation, nor taketh away the meanes or nature of merits, and
cooperation with God to our owne and other mens ſaluation.}
whom he hath predeſtinated; them alſo he hath called. And whom he hath
called; them alſo he hath iuſtified. And whom he hath iuſtified; them
alſo hath he glorified. \V What ſhal we then ſay to theſe things? If God
be for vs, who is againſt vs? \V He that ſpared not alſo his owne Sonne,
but for vs al deliuered him; how hath he not alſo with him giuen vs al
things? \V Who ſhal accuſe againſt the elect of God? God that
iuſtifieth? \V Who is he that ſhal condemne? \Sc{Christ Iesvs} that
died, yea that is riſen alſo againe, who is on the right hand of God,
who alſo maketh interceſsion for vs. \V Who then ſhal ſeparate vs from
the charitie of Chriſt? tribulation? or diſtreſſe? or famine? or
nakednes? or danger? or perſecution? or the ſword? \V (as it is written:
\CNote{\XRef{Pſ.~43,~22.}}
\Emph{For we are killed for thy ſake al the day: we are eſteemed as
sheep of ſlaughter.}) \V But in al theſe things we ouercome becauſe of
him that hath loued vs. \V
\TNote{\G{πέπεισμαι γὰρ ὅτι}}
For
\LNote{I am ſure.}{This
\MNote{No man ordinarily is ſure of his ſaluatiõ, but only in hope.}
ſpeach is common in S.~Paul according to the latin tranſlation, when he
had no other aſſured knowledge but by hope: as
\XRef{Ro.~15,~14.}
\XRef{2.~Tim.~1,~5.}
\XRef{Heb.~4,~9.}
Where the 
\TNote{\G{πέπεισμαι}, \L{confido}.
\Cite{Hiero. q.~9. ad Algoſ.}}
Greeke word ſignifieth only a probable perſuaſion. And therfore except
he meanes of himſelf by ſpecial reuelation, or of the predeſtinate in
general, (in which two caſes it may ſtand for the certitude of faith or
infallible knowledge) otherwiſe that euery particular man should be
aſſured infallibly that himſelf should be iuſtified, and not that only,
but ſure alſo neuer to ſinne, or to haue the guift of perſeuerance, and
certaine knowledge of his predeſtination: that is a moſt damnable falſe
illuſion and preſumption, condemned by the Fathers of the holy Councel
of Trent.
\Cite{Seſſ.~6. c.~9. 12,~13.}}
I am ſure that neither death, nor life, nor Angels, nor Principalities,
nor Powers, neither things preſent, nor things to come, neither
might, \V nor height, nor depth, nor other creature, ſhal be able to
%%% o-2502
ſeparate vs from the charitie of God which is in Chriſt \Sc{Iesvs} our
Lord.


\stopChapter


\stopcomponent


%%% Local Variables:
%%% mode: TeX
%%% eval: (long-s-mode)
%%% eval: (set-input-method "TeX")
%%% fill-column: 72
%%% eval: (auto-fill-mode)
%%% coding: utf-8-unix
%%% End:

