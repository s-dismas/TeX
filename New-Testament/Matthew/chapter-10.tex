%%%%%%%%%%%%%%%%%%%%%%%%%%%%%%%%%%%%%%%%%%%%%%%%%%%%%%%%%%%%%%%%%
%%%%
%%%% The (original) Douay Rheims Bible 
%%%%
%%%% New Testament
%%%% Matthew
%%%% Chapter 10
%%%%
%%%%%%%%%%%%%%%%%%%%%%%%%%%%%%%%%%%%%%%%%%%%%%%%%%%%%%%%%%%%%%%%%




\startcomponent chapter-10


\project douay-rheims


%%% 2312
%%% o-2125
\startChapter[
  title={Chapter 10}
  ]

\Summary{He giueth to the Twelue the power of Miracles, and ſo ſendeth
  them to the loſt sheep of the Iewes, 5.~with inſtructions accordingly:
  10.~and by occaſion of the ſending, foretelleth of the perſecutions
  after his Aſcenſion, arming them and al other againſt the ſame,
  40.~and alſo exhorting the people to harbour his ſeruants in ſuch
  times of perſecution.}

And hauing called his twelue Diſciples togeather,
\CNote{\XRef{Mar.~3,~13.}
\XRef{6,~7.}
\XRef{Luc.~6,~13.}
\XRef{9,~1.}}
he gaue them
\LNote{Power}{Miracles were ſo neceſſarie to the confirmation of their
  doctrine beginning then to be preached, that not only Chriſt himſelfe
  did miracles, but alſo he gaue to his Apoſtles power to doe them.}
power ouer vncleane Spirits, that they ſhould caſt them out, & ſhould
cure al mãner of diſeaſe, & al manner of infirmitie.

\V And the names of the twelue Apoſtles be theſe: The
\LNote{Firſt Simon}{Peter 
\MNote{Peters Primacie.}
the firſt, not in calling, but in
preeminence. For (as S.~Ambroſe ſaith in
\XRef{2.~Cor.~12.}) 
\Emph{Andrew firſt folowed our Sauiour before Peter and yet the Primacie
Andrew receaued not, but Peter.} Which preeminence of S.~Peter aboue the
other Apoſtles is ſo plainly ſignified in this word, \Emph{firſt}, by
the iudgement euen of Heretikes, that
\CNote{\Cite{Beza in Annot. noui Teſt. 1556.}}
Beza, not withſtanding he
confeſſeth the conſent of al copies both Latin & Greeke, yet is not
aſhamed to ſay, that he ſuſpecteth that this word was thruſt into the
text by ſome fauourer of Peters Primacie. Wherby we haue alſo, that they
care no more for the Greek then for the Latin, when it maketh againſt
them, but at their pleaſure ſay that al is corrupted.}
firſt, Simon who is called Peter, and Andrew his brother, \V Iames of
Zebedee, and Iohn his brother, Philip and Barthlemew, Thomas and Matthew
the publican, and Iames of Alphaæus, and Thaddæus, \V Simõ Cananæus, &
Iudas Iſcariote, who alſo betrayed him. 

\V Theſe twelue did \Sc{Iesvs} ſend; commanding them, ſaying: Into the
way of the
\SNote{They haue here commiſſion to preach only in Iſrael: the time
  being not yet come to cal the Gentiles.}
Gentiles goe ye not, and into the cities of the Samaritans enter ye not:
\V but goe rather to the ſheep that are periſhed of the houſe of
Iſrael. \V And going preach, ſaying: That the Kingdom of Heauen is at
hand. \V Cure the ſick, raiſe the dead, cleanſe the lepers, caſt out
Diuels: gratis you haue receaued, gratis giue ye. \V 
\LNote{Do not poſſeſſe}{Preachers may not carefully ſeeke after the
  ſuperfluities of this life, or any thing which may be an impediment to
their function. And as for neceſſaries, they deſerue their temporal
liuing at their hands for whom they labour ſpiritually.}
Do not poſſeſſe gold, nor ſiluer, nor money in your purſes: \V not a ſkrip for
the way, neither two coates, neither ſhoes, neither rod. For the workman
is worthie of his meate. \V And into whatſoeuer citie or towne you ſhal
enter, inquire who in it is worthie: and there tarie til you goe
forth. \V And when ye enter into the houſe, ſalute it, ſaying:
\LNote{Peace be to this houſe}{As
\MNote{Biſhops bleſſing.}
Chriſt himſelf vſed theſe words, or this bleſſing often, \Emph{Peace be
  to you}, ſo here he biddeth his Apoſtles ſay the like to the houſe
where they come. And ſo hath it been alwaies a moſt godly vſe of
Biſhops
\CNote{\Cite{Aug. ciuit. li.~22. c.~8.}
\Cite{Leo Imp. in vit. S.~Shryſ.}
\Cite{Socrat. l.~6. c.~14.}}
to giue their bleſſing where they come; which bleſſing muſt
needs be of great grace & profit, when none but worthy Perſons (as here
we read) might take good therof; and when it is neuer loſt, but
returneth to the giuer, when the other partie is not worthy of it.
\MNote{It remitteth venial ſinnes.}
 Among other ſpiritual benefits it taketh away venial ſinnes. 
\Cite{Am. in 9.~Lu.}}
Peace be to this houſe.
\V And if ſo be that houſe be worthie, your peace ſhal come vpon it. But
if it be not worthie, your peace ſhal returne to you. \V And whoſoeuer
ſhal not receaue you, nor heare your wordes; going forth out of the
houſe or the citie
\LNote{Shake off the duſt}{To contemne the true Preachers, or not to
  receaue the truth preached, is a very damnable ſinne.}
ſhake of the duſt from your feet. \V Amen I ſay to you, it ſhal be
\LNote{More tolerable}{Hereby it is euident, that there be degrees &
  differences of damnation in Hel fire, according to mens deſerts.
\Cite{Aug. li.~4. de Baps. c.~19.}}
more tolerable for the land of the Sodomites and Gomorrheans in the day
of
\Fix{iugdement,}{iudgement,}{obvious typo, fixed in other}
then for that citie.

%%% o-2126
\V Behold I ſend you as ſheep in the middes of wolues. Be ye therfore
\SNote{Wiſedom and ſimplicitie both be neceſſarie in Preachers, Biſhops,
and Prieſts.}
wiſe as Serpents, and ſimple as Doues. \V And take heed of men. For they
wil deliuer you vp in Councels, and in their Synagogues they wil ſcourge
you. \V And to Preſidents and
\LNote{Kings}{In the beginning Kings and Emperours perſecuted the
 Church, that by the very death and bloud of Martyrs it ſhould grow
more miraculouſly. Afterward when the Emperours and Kings were
themſelues become Chriſtians, they vſed their power for the Church,
againſt Infidels and Heretikes.
\Cite{Aug. ep.~48.}}
to Kings ſhal you be led for my ſake, in teſtimonie to them and the
Gentiles. \V But when they ſhal deliuer you vp,
\CNote{\XRef{Mr.~13,~11.}
\XRef{Luc.~12,~11.}}
take no thought how or what to ſpeake: for
\LNote{It shal be giuen}{This is verified euen at this preſent alſo,
  when many good Catholikes, that haue no great learning, by their
  anſwers confound the Aduerſaries.}
it ſhal be giuen you in that houre what to ſpeake. \V For it is not you
that ſpeake, but the Spirit of your Father that ſpeaketh in you. \V
\CNote{\XRef{Luc.~21,~16.}}
The
brother alſo ſhal deliuer vp the brother to death, and the Father the
Sonne: and the children shal riſe vp againſt the parents, and ſhal worke
their death, \V and you ſhal be odious to al men for my name: but he
that shal perſeuer vnto the end, he shal be ſaued.

%%% 2313
\V And when they shal perſecute you in this citie, flee into an
other. Amen I ſay to you, you shal not finish al the cities of Iſrael,
til the Sonne of man come.

\V
\CNote{\XRef{Luc.~6,~40.}}
The Diſciple is not aboue the Maiſter, nor the Seruant aboue his
Lord. \V It ſufficeth the Diſciple that he be as his maiſter, and the
Seruant his Lord. If they haue called the Goodman of the houſe
Beelzebub, 
\LNote{How much more}{No maruel therfore if Heretikes cal Chriſts vicar
  Antichriſt, when their forefathers, the faithles Iewes, called Chriſt
  himſelfe Beelzebub.}
how much more them of his houſhold? \V Therfore feare ye not them. For
nothing is hid, that shal not be reuealed: and ſecret, that shal not be
known. \V That which I ſpeake to you in the dark, ſpeak ye in the light:
and that which you heare in the eare, preach ye vpon the houſe tops. \V
And 
\SNote{A goodly cõfort for Chriſtians and Catholiks and al good men, in
  the perſecutiõs of Turkes, of Heretikes, of al wicked men.}
feare ye not them that kil the body, and are not able to kil the ſoul:
but rather feare him that can deſtroy both ſoul and body into Hel.

\V Are not two ſparowes ſold for a farthing: and not one of them shal
fal vpon the ground without your Father? \V But your very haires of the
head are al numbered. \V Feare not therfore: better are you then many
ſparowes. \V
\CNote{\XRef{Mr.~8,~38.}
\XRef{Luc.~9,~26.}
\XRef{12,~8.}}
Euery one therfore that shal
\LNote{Confeſſe me}{See
\MNote{Confeſſing of Chriſt and his truth.}
how Chriſt eſteemed the open confeſſing of him, that is of his truth in
the Catholike Church. For as when Saul perſecuted the Church, he ſayd
\CNote{\XRef{Act.~9.}}
himſelf was perſecuted; ſo to confeſſe him, and his Church, is al
one. Contrariewiſe ſee how he abhoreth them that deny him before men,
which is not only to deny any one litle article of the Catholike faith,
commended to vs by the Church; but alſo to allow or conſent to hereſie
by any meanes, as by ſubſcribing, comming to their ſeruice and ſermons,
furthering them any way againſt Catholikes, and ſuch like.}
confeſſe me before men, I alſo wil confeſſe him before my Father which
is in Heauen. \V But he that shal denie me before men, I alſo wil denie
him before my Father which is in Heauen. \V Do not ye think
\CNote{\XRef{Luc.~12,~51.}}
that I came to ſend peace into the earth: I came 
\LNote{Not peace but ſword}{Chriſt came to breake the
  peace of
\Fix{wordlings}{worldlings}{obvious typo, fixed in other}
and ſinners; as when the ſonne beleeueth in him, and the father doth
not; the wife is a Catholike, and the husband is not. For to agree
togeather in infidelitie, hereſie, or any other ſinne, is a naughty
peace. This being the true meaning of Chriſts words, marke that the
Heretikes interpret this to maintaine their rebellions and troubles,
which their new Ghoſpel breedeth. 
\Cite{Beza in no. Teſt. an.~1565.}}
not to ſend peace, but the ſword. \V For I came to ſeparate
\CNote{\XRef{Mich.~7,~6.}}
man againſt
his father, and the daughter againſt her mother, and the daughter in law
againſt her mother in law. \V And a mans enemies, they of his owne
houshold. \V He that loueth father or 
%%% o-2127
mother
\LNote{More then}{No earthly thing, nor dutie to Parents, wife,
  children, countrie, or to a mans owne body & life, can be any iuſt
  excuſe why a man ſhould doe, or feine himſelf to doe or beleeue any
  thing, againſt Chriſt or the vnitie and faith of his Church.}
more then me, is not worthy of me: and he that loueth ſonne or daughter
aboue me, is not worthy of me. \V And he that taketh not his croſſe, and
foloweth me, is not worthy of me. \V He that hath found his life, shal
loſe it: and he that hath loſt his life for me, shal find it.

\V
\CNote{\XRef{Luc.~10,~16.}}
He that receaueth you, receaueth me: and he that receaueth me,
receaueth him that ſent me. \V He that receaueth a Prophet
\LNote{In the name}{Reward for 
\Fix{hoſpitally,}{hoſpitality,}{obvious typo, fixed in other}
and 
\Fix{ſpeciality}{ſpecially}{obvious typo, fixed in other}
for receauing an holy Perſon, as Prophet, Apoſtle, Bishop, or Prieſt
perſecuted for Chriſts ſake. For by receauing of him in that reſpect as
he is ſuch an one, he ſhal be partaker of his merits, and be rewarded as
for ſuch an one. Whereas on the contrarie ſide, he that receaueth an
Heretike in to his houſe and a falſe Preacher, doth communicate with his
wicked workes.
\XRef{Ep.~2.~Io.}}
in the name of a Prophet, shal receaue the reward of a Prophet, and he
that receaueth a
\SNote{The reward for harbouring & helping any iuſt perſon ſuffering for
his iuſtice & conſcience.}
iuſt man in the name of a iuſt man, shal receaue the reward of a iuſt
man. \V And
\CNote{\XRef{Mat.~9,~41.}}
whoſoeuer shal giue drinke to one of theſe litle ones a cup
of cold water, only in the name of a Diſciple, amen I ſay to you, he
shal not loſe his reward.

\stopChapter


\stopcomponent


%%% Local Variables:
%%% mode: TeX
%%% eval: (long-s-mode)
%%% eval: (set-input-method "TeX")
%%% fill-column: 72
%%% eval: (auto-fill-mode)
%%% coding: utf-8-unix
%%% End:
