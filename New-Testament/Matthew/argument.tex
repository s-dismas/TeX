%%%%%%%%%%%%%%%%%%%%%%%%%%%%%%%%%%%%%%%%%%%%%%%%%%%%%%%%%%%%%%%%%
%%%%
%%%% The (original) Douay Rheims Bible 
%%%%
%%%% New Testament
%%%% Matthew
%%%% Argument
%%%%
%%%%%%%%%%%%%%%%%%%%%%%%%%%%%%%%%%%%%%%%%%%%%%%%%%%%%%%%%%%%%%%%%




\startcomponent argument


\project douay-rheims


%%% 2290
%%% o-2102
\startArgument[
  title={\Sc{the argvment of s.~matthewes ghospel.}},
  marking={Argument of S.~Matthew's Gospel}
  ]

Matthewes Ghoſpel may be wel diuided into fiue partes. The firſt parte,
as touching the Infancie of our Lord Ieſus: Chap.~1.\ and~2.

The ſecond, of the preparation that was made to his manifeſtation:
chap.~3.\ and a piece of the 4.

The third, of his manifeſting of himſelfe by preaching and miracles,
and that in Galilee: the other piece of the 4.~chap.\ vnto the 19.

The fourth, of his comming into Iurie, toward of his Paſsion:
chap.~19.\ and 20.

The fifth, of the Holy weeke of his Paſsion in Hieruſalem:
chap.~21.\ vnto the end of the booke.

Of S.~Matthew we haue
\XRef{Mat.~9.}
\XRef{Mar.~2.}
\XRef{Luc.~5.}
How being before a Publican, he was called of our Lord, and made a
Diſciple. Then
\XRef{Luk.~6.}
\XRef{Mar.~3.}
\XRef{Mat.~10.}
How out of the whole number of the Diſciples he was choſen to be one of
the twelue Apoſtles. And out of them againe he was choſen (and none but
he and S.~Iohn) to be one of the foure Euangeliſtes. Among which foure
alſo, he was the firſt that wrote, about 8.\ or 10.\ yeares after
Chriſtes Aſcenſion.

\stopArgument


\stopcomponent


%%% Local Variables:
%%% mode: TeX
%%% eval: (long-s-mode)
%%% eval: (set-input-method "TeX")
%%% fill-column: 72
%%% eval: (auto-fill-mode)
%%% coding: utf-8-unix
%%% End:
