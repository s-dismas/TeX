%%%%%%%%%%%%%%%%%%%%%%%%%%%%%%%%%%%%%%%%%%%%%%%%%%%%%%%%%%%%%%%%%
%%%%
%%%% The (original) Douay Rheims Bible 
%%%%
%%%% New Testament
%%%% Matthew
%%%% Chapter 16
%%%%
%%%%%%%%%%%%%%%%%%%%%%%%%%%%%%%%%%%%%%%%%%%%%%%%%%%%%%%%%%%%%%%%%




\startcomponent chapter-16


\project douay-rheims


%%% 2328
%%% o-2143
\startChapter[
  title={Chapter 16}
  ]

\Summary{The obſtinate Phariſees and Sadducees, as though his foreſaid
  miracles were not ſufficient to proue him to be Chriſt, require to ſee
ſome one from Heauen. 5.~whervpon forſaking them, he warneth his
Diſciples to beware of the leauen of their doctrine: 13.~and Peter (the
time now approaching for him to goe into Iurie to his Paſsion) for
confeſsing him to be Chriſt, he maketh the Rock of his Church; giuing
fulnes of Eccleſiaſtical power accordingly. 21.~And after he ſo rebuketh
him for diſſuading his Croſſe and paſsion, that he alſo affirmeth the
like ſuffering in euerie one, to be neceſſarie to ſaluation.}

%%% o-2144
And
\CNote{\XRef{Mr.~8,~12.}
\XRef{Luc.~12,~54.}}
there came to him the Phariſees and Sadducees tempting: and they
demanded him to ſhew them a ſigne from Heauen. \V But he anſwered & ſaid
to them: When it is euening, you ſay: It wil be faire-weather, for the
element is red. \V And in the morning: This day there wil be a tempeſt,
for the element doth glow and lowre. The face therfore of the element
you haue ſkil to diſcerne: & the ſignes of times can you not? \V
\CNote{\XRef{Mat.~12,~39.}}
The naughtie and aduouterous Generation ſeeketh for a ſigne: and there
ſhal not a ſigne be giuen it, but the ſigne of Ionas the Prophet. And he
left them and went away.

\V And
\CNote{\XRef{Mr.~6,~14.}
\XRef{Luc.~12,~1.}}
when his Diſciples were come ouer the water, they forgot to take
bread \V Who ſaid to them: Looke wel and beware of the leauen of the
Phariſees & Sadducees. \V But they thought within them ſelues ſaying:
Becauſe we tooke not bread. \V And \Sc{Iesvs} knowing it, ſaid: why
%%% 2329
do you thinke within your ſelues, O ye of litle faith, for that you haue
not bread? \V Do you not yet vnderſtand, neither do you remember
\CNote{\XRef{Mt.~14,~17.}}
the
fiue loaues among fiue thouſand men, and how many baſkets you tooke vp?
\V
\CNote{\XRef{Mt.~15,~34.}}
neither the ſeauen loaues, among foure thouſand men, and how many
maundes you tooke vp? \V Why do you not vnderſtand that I ſaid not of
bread to you: Beware of the leauen of the Phariſees, & Sadducees? \V
Then they vnderſtood that he ſaid not they ſhould beware of the leauen
of bread, but of the doctrine of the Phariſees and Sadducees.

\V And
\CNote{\XRef{Mr.~8,~27.}
\XRef{Lu.~9,~18.}}
\Sc{Iesvs} came into the quarters of Cæſarea Philippi: and he asked his
Diſciples, ſaying:
\LNote{Whom ſay men}{Chriſt
\MNote{\Sc{Of Peters Primacie.}}
intending here to take order for the founding, regiment, & ſtabilitie of
his Church after his deceaſe, & to name the Perſon to whom he meant to
giue the general charge thereof, would before by interrogatories draw out
(& namely out of that one whom he thought to make the cheefe) the
profeſſion of that high and principal Article: that he was the Sonne of
the liuing God, which being the ground of the Churches faith, was a
neceſſarie qualitie and condition in him that was to be made Head of the
ſame Church, and the perpetual keeper of the ſaid faith, and al other
points thereon depending.}
whom ſay men that the Sonne of man is? \V 
\LNote{But they ſaid}{When Chriſt asked the Peoples opiniõ of him, the
  Apoſtles al indifferently made anſwer: but when he demanded what
  themſelues thought of him, then loe Peter the mouth and head of the
  whole felowſhip anſwered for al.
\Cite{Chryſoſtom. homil.~35. in Mat.}}
But they ſaid: Some Iohn the Baptiſt, & otherſome Elias, and others
Hieremie, or one of the Prophets. \V \Sc{Iesvs} ſaith to them: But whom
do you ſay that I am? \V Simon Peter anſwered & ſaid: \Emph{Thou art
  Chriſt the Sonne of the liuing God.} \V And \Sc{Iesvs} anſwering, ſaid
to him:
\LNote{Bleſſed art thou}{Though ſome other (as Nathanael
\XRef{Io.~1,~49.})
ſeemed to haue before beleeued and profeſſed the
ſame thing, for which Peter is here counted bleſſed, yet it may be
plainly gathered by this place, & ſo
\CNote{\Cite{Hilar. can.~6. in Mat.}
&
\Cite{li.~6. de Trinit.}
\Cite{Chryſ. ho.~55. in Mat.}}
S.~Hilarie and others thinke,
  that none before this did further vtter of him, then that he was the
  Sonne of God by adoption as other Saints be, though more excellent
  then other be. For it was of congruitie and Chriſtes ſpecial
  appointment, that he vpon whom he intended to found his new Church, &
  whoſe faith he would make infallible, ſhould haue the preeminence of
  this firſt profeſſion of Chriſtes natural diuinitie, or, that he was
  by nature the very Sonne of God; a thing ſo farre aboue the capacitie
  of nature, reaſon, fleſh, and bloud, and ſo repugnant to Peters ſenſe
  and ſight of Chriſtes humanitie, fleſh, and infirmities, that for the
  beleefe and publik profeſſion thereof he is counted bleſſed, as Abrahã
was for his faith; and hath great promiſes for himſelf and his
poſteritie, as the ſaid Patriarch had for him and his ſeed. According as
\CNote{\Cite{Baſil. li.~2. adu. Eunom.}}
S.~Baſil ſaith: Becauſe he excelled in faith, he receaued the building
of the Church committed to him.}
Bleſſed art thou Simon Bar-Iona: becauſe fleſh & bloud hath not reuealed
it to thee, but my Father which is in Heauen. \V 
\LNote{And I ſay to thee}{Our Lord recompenſeth Peter for his
  confeſſion, giuing him a great reward, in that vpon him be builded his
Church.
\Cite{Theophilactus. vpon this place.}}
And I ſay to thee:
 \Emph{That} 
\LNote{Thou art Peter}{Chriſt (in the
\XRef{firſt of Iohn v.~42.})
foretold and appointed that this man thẽ named Simon,
  should afterward be called \Emph{Cephas}, or \Emph{Petrus}, that is to
  ſay, a \Emph{Rock}; not then vttering the cauſe, but now expreſſing
  the ſame, \L{videlicet} (as
\CNote{\Cite{Cyr. l.~8. c.~12. Cõ. in Io.}}
S.~Cyril writeth) \Emph{For that vpon him
    as vpon a firme rock his Church should be builded.} Wherevnto
\CNote{\Cite{Hilar. in hunc locũ.}}
  S.~Hilarie agreeing ſaith: \Emph{O happie foundation of the Church in
    the impoſing of thy new name &c.} And yet Chriſt here doth not ſo
  much cal him by the name Peter or Rock, as he doth affirme him to be a
  rock; ſignifying by that Metaphore, both that he was deſigned for the
  foundation and groundwork of his houſe, which is the Church, & alſo
  that he should be of inuincible force, firmitie, durablenes, and
  ſtabilitie, to ſuſtaine al the windes, waues, and ſtormes that might
  fal or beate againſt the ſame. And the Aduerſaries obiecting againſt
  this, that Chriſt only is the Rock or fundation, wrangle againſt the
  very expreſſe Scriptures, & Chriſtes owne wordes, giuing both the name
  & the thing to this Apoſtle. And the ſimple may learne by S.~Baſils
  wordes, how the caſe ſtãdeth. \Emph{Though} (ſaith
\CNote{\Cite{Baſil li. de pœnit.}}
he) \Emph{Peter be
    a rock, yet he is not a rock as Chriſt is. For Chriſt is the true
    vnmouable rock of himſelf. Peter is vnmoueable by Chriſt the
    rock. For Ieſus doth communicate and impart his dignities, not
    voyding himſelf of them, but holding them to himſelf, beſtoweth them
  alſo vpon others. He is the light, and yet
\CNote{\XRef{Mt.~5,~14.}}
You are the light: he is the Prieſt, and yet he
\CNote{\XRef{Luc.~22,~19.}}
maketh Prieſts; he is the rock, and he made a
  rock.}}
\Emph{thou art}
\CNote{\XRef{Io.~1,~42.}}
\SNote{That is, a Rock.}
Peter; 
\LNote{And vpon this rock}{Vpon 
\MNote{Thou art \Emph{Cephas}, and vpon this \Emph{Cephas}.}
that which he ſaid Peter was, wil
  he build his Church; and therfore by moſt euidẽt ſequele he foundeth
  his Church vpõ Peter. And the Aduerſaries wrangling againſt this, doe
  againſt their owne conſcience & knowledge; ſpecially ſeeing they know
  and confeſſe that in Chriſtes wordes ſpeaking in the Syriake tõgue,
  there was no difference at al between \L{Petrus} and \L{Petra}; yea
  and that the 
%%% !!! Can I do this:
%%% petros -|
%%%         | rock
%%% petra  -|
\TNote{\G{πέτρος}, \G{πέτρα}, rock}
Greeke wordes alſo though differing in termination, yet ſignifie one
thing, to wit, a \Emph{rock}, or \Emph{ſtone}, as themſelues alſo
tranſlate it.
\XRef{Io.~1,~42.}
So that they which profeſſe to folow the
Hebrew, or Syriake, & the Greeke, & to tranſlate immediatly out of them
into Latin or English, should if they had dealt ſincerely, haue thus
turned Chriſtes wordes: \Emph{Thou art a rock, & vpon this rock}; or,
\Emph{Thou art Peter, and vpon this Peter wil I build my Church.} For ſo
Chriſt ſpake by their owne confeſſion without any differẽce. Which doth
expreſly ſtop them of al their vaine euaſiõs, that \L{Petrus}, the
former word is referred to the Apoſtles, and \L{Petra} the later word,
either to Chriſt only, or to Peters faith only; neither the ſaid
original tongues bearing it, nor the ſequele of the wordes, \Emph{vpon
  this}, ſuffering any relation in the world but to that which was
ſpoken of in the ſame ſentence next before; neither the wordes folowing
which are directly addreſſed to Peters Perſon, not Chriſtes intẽtion by
any meanes admitting it, which was not to make himſelf or to promiſſe
himſelf to be the head or foundation of the Church. For his Father gaue
him that dignitie, & he took not that honour to himſelf, nor ſent
himſelf, nor took the keies of Heauen of himſelf, but al of his
Father. He had his commiſſion the very houre of his incarnation. And
though
\CNote{\Cite{Aug. li.~1. retr. c.~21.}}
S.~Aug. ſometimes referre the word (\L{Petra}) to Chriſt in this
ſentence (which no doubt he did becauſe the terminations in Latin are
diuers, and becauſe he examined not the nature of the original wordes
which Chriſt ſpake, nor of the Greek, and therfore the Aduerſaries which
otherwiſe flee to the tõgues, should not in this caſe alleage him) yet
he neuer denieth but Peter alſo is the Rock & head of the Church, ſaying
that himſelf expounded it of Peter
\CNote{\Cite{In Pſal.~66. De. verb. Do. ſec. Io.}
\Cite{ſer.~49.}
%%% !!! ???
\Cite{ſer.~15, 16, 26, 29.}
\Cite{de Sanctis. Annot. in Iob. c.~30.}}
in many places, and alleageth alſo
S.~Amb. for the ſame in his Hymne which the Church ſingeth. And ſo do
we alleage the holy Councel of Chalcedon, 
\XRef{Act~3 pag.118.}
\Cite{Tertul. de præſcrip.}
\Cite{Origen, Ho.~5. in evo.}
\Cite{S.~Cyprian, De vnit. Ec.}
\Cite{S.~Hilarie, Can.~16. in Mat.}
\Cite{S.~Ambroſe, Ser.~47.~68. li.~6. in c.~9. Lucæ.}
\Cite{S.~Hierom, li.~1. in Iouin. & in c.~2. Eſſa. & in c.~16. Hier.}
\Cite{S.~Epiphanius, in Anchor.}
\Cite{S.~Chryſoſtum, Ho.~55. in Mat.}
\Cite{S.~Cyril, li.~2. c.~12. com in Io.}
\Cite{S.~Leo. ep.~89.}
\Cite{S.~Gregorie, Li.~4. ep.~42. ind.~13.}
\CNote{\Cite{Theod. li.~5. har. Fabul. c. de pœnit.}}
and others; euery one of them ſaying expreſly, that the Church was
founded and builded vpõ Peter. For though ſometimes they ſay the Church
to be builded on Peters faith, yet they meane not (as our Aduerſaries ſo
vnlearnedly take them) that it should be builded vpon faith either
ſeparated from the man, or in any other man; but vpon faith as in him
who here confeſſed that faith.}
\Emph{and vpon this rock}
\LNote{Rock}{The Aduerſaries hearing alſo the Fathers ſometimes
  ſay, that Peter had theſe promiſes and prerogatiues, as bearing the
  Perſon of al the Apoſtles or of the whole Church, deny abſurdly that
  himſelf in Perſon had theſe prerogatiues. As though Peter had been the
Proctour only of the Church or of the Apoſtles, confeſſing the faith and
receauing theſe things in other mens names. Where the holy Doctours
meane only, that theſe prerogatiues were not giuen to him for his owne
vſe, but for the good of the whole Church, and to be imparted to euery
vocation according to the meaſure of their callings; and that theſe
great priuileges giuen to Peter should not decay or die with his Perſon,
but be perpetual in the Church in his ſucceſſours. Therfore S.~Hierom
to Damaſus taketh this Rock not to be Peters Perſon only, but his
ſucceſſours and his Chaire. \Emph{I} (ſaith 
\CNote{\Cite{Hier. ep.~7. to.~2.}}
he) \Emph{folowing no cheefe
or principal but Chriſt, ioyne myſelf to the communion of Peters chaire,
vpõ that rock I know the Church was built.} And of that ſame Apoſtolike
Chaire S.~Auguſt. ſaith:
\CNote{\Cite{Pſa. cõt. part. Donat. to.~7.}}
\Emph{That ſame is the Rock which the proud gates of Hel do not
ouercome.} And S.~Leo,
\CNote{\Cite{Leo ep.~89.}}
\Emph{Our Lord would the
  Sacramẽt or myſterie of this guift ſo to pertaine vnto the office of
  al the Apoſtles, that he placed it principally in Bleſſed S.~Peter the
cheefe of al the Apoſtles, that from him as from a certaine head he
might poure out his guiftes, as it were through the whole body; that he
might vnderſtand himſelf to be an aliene from the diuine myſterie that
should preſume to reuolt from the ſoliditie or ſtedfaſtnes of Peter.}}
\Emph{rock wil I}
\LNote{Build my Church}{The Church or houſe of Chriſt was only
promiſed here to be builded vpon him (which was
fulfilled.
\XRef{Io.~21,~15.})
the foundation, ſtone, & other pillers
or matter being yet in preparing; and Chriſt himſelf being not only
the ſupereminent foundation but alſo the founder of the ſame; which is
an other more excellent qualitie then was in Peter, for which he
calleth it \Emph{my Church}: meaning ſpecially the Church of the new
Teſtament. Which was not perfectly formed and finished, and diſtincted
from the Synagogue til Whitſunday, though Chriſt gaue Peter and the
reſt their commiſſions actually before his Aſcenſion.}
\Emph{build my Church, and}
\LNote{Gates of Hel}{Becauſe the Church is reſſembled to a
houſe or a citie, the aduerſarie powers alſo be likened to a contrarie
houſe or towne, the gates wherof, that is to ſay, the fortitude, or
impugnations shal neuer preuaile againſt the citie of Chriſt. And ſo by
this promiſe we are aſſured that no hereſies nor other wicked attempts
cã preuail againſt the Church builded vpon Peter, which the Fathers cal
Peters See and the Romane Church.
\CNote{\Cite{Pſa. cõt. part. Donat.}}
\Emph{Count} (ſaith S.~Auguſtine)
\Emph{the Prieſts from the very See of Peter, and in that order of Fathers
  conſider who to whom hath ſucceeded: that ſame is the rock which the
  proud gates of Hel do not ouercome.} And in an other place,
\CNote{\Cite{De vtil. cred. c.~17.}}
\Emph{that is it which hath obtained the top of authoritie, Heretikes in
  vaine barking round about it.}}
\Emph{the gates of Hel shal not preuaile againſt it.} \V \Emph{And I}
\CNote{\XRef{Io.~21,~15.}}
\Emph{wil giue}
\LNote{To thee}{In ſaying, \Emph{to thee wil I giue}, it is
  plaine that as he gaue the keies to him, ſo he builded the Church vpon
  him. So ſaith S.~Cyprian:
\CNote{\Cite{Cypr. Epiſt.~73.}}
\Emph{To Peter firſt of al, vpon whom our
    Lord built the Church, and from whom he inſtituted and shewed the
    beginning of vnitie, did he giue this power, that that should be
    looſed in the Heauens, which he had looſed in earth.}
\CNote{\Cite{Greg. l.~4. ep.~32. ind.~13.}}
Wherby
  appeareth the vaine cauil of our Aduerſaries, which ſay the Church was
  built vpon Peters Confeſſion only, common to him and the reſt, and not
  vpon his Perſon, more then vpon the reſt.}
\Emph{to thee}
\LNote{The keies}{That is, 
\MNote{The dignities of the keies.}
the authoritie or Chaire, of doctrine,
knowledge, iudgement and diſcretion between true, and falſe doctrine:
the height of gouernement, the power of making lawes, of calling
Councels, of the principal voice in them, of confirming thẽ, of making
Canons, & holeſom decrees, of abrogating the contrarie, of ordaining
Biſhops and Paſtours, or depoſing and ſuſpending them: finally the power
to diſpenſe the goods of the Church both ſpiritual and temporal. Which
ſignification of preeminent power and authoritie by the word,
\Emph{keies}, the Scripture expreſſeth in many places: namely ſpeaking
of Chriſt:
\CNote{\XRef{Apoc.~1.}}
\Emph{I haue the keies of death and Hel, that is, the rule.}
And Againe:
\CNote{\XRef{Eſa.~22,~22.}}
\Emph{I wil giue the key of the houſe of Dauid vpon his
  shoulder.} Moreouer it ſignifieth that men cannot come into Heauen but
by him, the keies ſignifying alſo authoritie to open and ſhut, as it is
ſaid
\XRef{Apoc.~3.}
of Chriſt: \Emph{Who hath the key of Dauid,
he shutteth and no man openeth.} By which words we gather that Peters
authoritie is maruelous, to whom the keies, that is, the power to open
and shut Heauen, is giuen. And therfore by the name of keies is giuen
that ſupereminent power which is called, in compariſon of the power
granted to other Apoſtles, Biſhops, and Paſtours, \L{plenitudo
poteſtatis}, fulnes of power. 
\Cite{Bernard. lib.~2. de conſiderat. c.~8.}}
\Emph{the keyes of the Kingdom of Heauen. And}
\LNote{Whatſoeuer thou shalt bind}{Al kind of diſcipline and
puniſhment of offenders, either ſpiritual (which directly is here
meant) or corporal ſo farre as it tendeth to the execution of the
ſpiritual charge, is compriſed vnder the word, \Emph{bind}. Of which
ſort be Excommunications, Anathematiſmes, Suſpenſions, degradations,
and other cenſures, & penalties, or penãces enioyned either in the
Sacrament of Confeſſion, or in the exteriour Courts of the Church, for
puniſhment both of other crimes, and ſpecially of hereſie & rebellion
againſt the Church, and the cheefe paſtours therof.}
\Emph{what ſoeuer thou shalt bind vpon earth, it shal be bound alſo in
Heauen: and what ſoeuer thou shalt}
\LNote{Looſe}{To looſe, is as the cauſe and the offenders caſe
requireth; to looſe them of any former bandes, and to reſtore them to
the Churches Sacraments, and Communion of the Faithful, and execution
of their function; to pardon alſo either al, or part of the pennance
enioyned, or what debts ſoeuer man oweth to God, or the Church, for
the ſatisfaction of his ſinnes forgiuen. Which kind of releaſing or
looſing is called \Emph{Indulgence}: finally this, \Emph{whatſoeuer},
exepteth nothing that is puniſhable or pardonable by Chriſt in earth,
for he hath committed his power to Peter. And ſo the validitie of
Peters ſentence in binding or looſing whatſoeuer, ſhal by Chriſts
promiſe be ratified in Heauẽ. 
\Cite{Leo Ser. de Transfig. & }
\Cite{Ser.~2. in aniuerſ. aſſump. ad Pontif.}
\Cite{Hilar. can.~15. in Matt. Epiph. in Anchorato prope initium.}
If now any temporal power can ſhew their warrant out of Scripture for
ſuch ſoueraigne power, as is here giuen to Peter, & cõſequently to his
ſucceſſours, by theſe words, \Emph{whatſoeuer thou shalt bind}, and by
the very keies, wherby greateſt ſoueraigntie is ſignified in Gods Church
as in his familie and houſhold, and therfore principally attributed and
giuen to Chriſt
\CNote{\XRef{Eſa.~22.}
\XRef{Apoc.~3.}}
who in the Scripture is ſaid to haue the key of Dauid,
but here communicated alſo vnto Peter as the name of Rock: if I ſay any
temporal Poteſtate can ſhew authoritie for the like ſoueraigntie, let
thẽ chalenge hardly to be head, not only of one particular, but of the
whole vniuerſal Church.}
\Emph{looſe in earth, it shal be looſed alſo in Heauen.}

\V Then he commanded his Diſciples, that they ſhould tel 
%%% o-2145
no body that he was \Sc{Iesvs Christ}.

\V From that time \Sc{Iesvs} began to ſhew his Diſciples, that he muſt
goe to Hieruſalem, & ſuffer many things of the Ancients & Scribes and
Cheefe-Prieſts, and be killed, and the third day riſe againe. \V And
Peter taking him, began to rebuke him, ſaying: Lord, be it farre from
thee, this ſhal not be vnto thee. \V Who turning ſaid to Peter: Goe
after me
\SNote{This word in Hebrew ſignifieth an aduerſarie, as
\XRef{3.~Reg.~5,~4.}
and ſo it is taken here.}
Sathan, thou art a ſcandal vnto me: becauſe thou fauoureſt not the
things that are of God, but the things that are of men. \V Then
\Sc{Iesvs} ſaid to his Diſciples: If any man wil come after me, let him
denie himſelf, and take vp his croſſe, and follow me. \V For he that wil
ſaue his life, ſhal loſe it, and he that ſhal loſe his life for me, ſhal
find it. \V For what doth it profit a man, if he gaine the whole world,
and ſuſtaine the damage of his ſoule? Or what permutation ſhal a man
giue for his ſoule? \V For the Sonne of man ſhal come in the glorie of
his Father with his Angels: and then wil he render to euery man
according to his 
\LNote{Workes}{He
\MNote{Good workes.}
ſaith not, to giue euery man according to his mercie (or their faith)
but according to their workes. 
\Cite{Auguſt. de verb. Apoſt. Ser.~35.}
\MNote{\Fix{Frewil.}{Freewill.}{obvious typo, fixed in other}}
And againe; How ſhould our Sauiour reward euery one according to their
works, if there were no free wil
\Cite{Auguſt. lib.~2. cap.~4.~5.~8. de Act. cum Fælic. Manich.}}
workes.
\V Amen I ſay to you,
\CNote{\XRef{Mar.~9,~1.}
\XRef{Luc.~9,~27.}}
there be ſome of them that ſtand here, that ſhal
not taſte death, til they ſee the Sonne of man comming in his Kingdom.

\stopChapter


\stopcomponent


%%% Local Variables:
%%% mode: TeX
%%% eval: (long-s-mode)
%%% eval: (set-input-method "TeX")
%%% fill-column: 72
%%% eval: (auto-fill-mode)
%%% coding: utf-8-unix
%%% End:
