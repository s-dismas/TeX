%%%%%%%%%%%%%%%%%%%%%%%%%%%%%%%%%%%%%%%%%%%%%%%%%%%%%%%%%%%%%%%%%
%%%%
%%%% The (original) Douay Rheims Bible 
%%%%
%%%% New Testament
%%%% Matthew
%%%% Chapter 17
%%%%
%%%%%%%%%%%%%%%%%%%%%%%%%%%%%%%%%%%%%%%%%%%%%%%%%%%%%%%%%%%%%%%%%




\startcomponent chapter-17


\project douay-rheims


%%% 2332
%%% o-2148
\startChapter[
  title={Chapter 17}
  ]

\Summary{As he promiſed, he giueth them a ſight of the glorie, vnto
  which Suffering doth bring; 9.~and then againe doth inculcate his
  Paſsion. 14.~A Diuel alſo he caſteth out which his Diſciples could not
for their incredulitie, and lack of praying and faſting. 22.~Being yet
in Galilee, he reuealeth more about his Paſsion 24.~and the tribute that
the Collectours exacted for al, he payeth for himſelf and Peter;
declaring yet withal his freedom by word, and miracle.}

And
\MNote{The Transfiguration of our Lord.}
after ſix dayes
\CNote{\XRef{Mr.~9,~2.}
\XRef{Lu.~9,~28.}
\XRef{2.~Pet.~1,~17.}}
\Sc{Iesvs} taketh vnto him Peter, and Iames, & Iohn his
brother, & bringeth thẽ into a high mountaine apart: \V And he was
\LNote{Transfigured}{Mark
\MNote{Chriſt can exhibit his body vnder what forme he liſt.}
in this Transfiguration many maruelous points: as, that he made not
only his owne body, which then was mortal, but alſo the bodies of Moyſes
& Elias, the one dead, the other to die, for the time as it were
immortal; therby to repreſent the ſtate and glorie of his body and his
Saints in Heauen. By which maruelous transfiguring of his body, you may
the leſſe maruel that he can exhibit his body vnder the forme of bread
and wine, or otherwiſe as he liſt.}
transfigured before thẽ. And his face did ſhine
%%% 2333.pdf !!! Note the pdf extension
as the ſunne: & his garments became white as ſnow. \V And behold there
\LNote{Appeared Moyſes}{By
\MNote{Saints after their death deale with, and for the liuing.}
this that Moyſes perſonally appeared and was preſent with Chriſt, it is
plaine that the Saints departed may in Perſon be preſent at the affaires
of the liuing. 
\Cite{Auguſt. de cura pro mora. c.~15.~16.}
For euen as Angels
els where, ſo here the Saints alſo ſerued our Sauiour; and therfore as
Angels both in the old Teſtament & the new, were preſent often at the
affaires of men, ſo may Saints.}
appeared to them Moyſes and Elias talking with him. \V And Peter
anſwering, ſaid to \Sc{Iesvs}: Lord, it is good for vs to be here: if
thou wilt, let vs make here three tabernacles, one for thee, and one for
Moyſes, and one for Elias. \V And as he was yet ſpeaking, behold a
bright cloud ouerſhadowed them. And loe a voice out of the cloud,
ſaying: This is my welbeloued Sonne, in whom I am wel pleaſed: heare ye
him. \V And the Diſciples hearing it, fel vpon their face, and were ſore
afraid. \V And \Sc{Iesvs} came and touched them: and he ſaid to them:
Ariſe, and feare not. \V And they lifting vp their eyes, ſaw nobody, but
only \Sc{Iesvs}. \V And as they deſcended from the
\LNote{Mount}{This
\MNote{Holy places.}
mount (commonly eſteemed and named of the ancient Fathers Thabor)
S.~Peter calleth
\CNote{\XRef{2.~Pet.~1,~18.}}
\Emph{the holy Mount} becauſe of this wonderful viſion,
like as in the old Teſtament, where God appeared to Moyſes in the buſh,
and els where to others, he calleth the place of ſuch Apparitions,
\CNote{\XRef{Exo.~3,~5.}}
\Emph{holy ground}. 
\MNote{Deuotion and Pilgrimage to the ſame.}
Wherby it is euident that by ſuch Apparitions,
places are ſanctified, and thervpon groweth a religion and deuotion in
the Faithful toward ſuch places, and namely to this Mount Thabor (called
in S.~Hierom \Emph{Itabirium}
\Cite{Ep.~17.})
there was great
Pilgrimage in the Primitiue Church, as vnto al thoſe places which our
Sauiour had ſanctified with his preſence and miracles; 
\MNote{The holy land.}
and therfore to
the whole land of promiſe, for that cauſe called the holy Land. See
\Cite{S.~Hierom. in Epitap. Paulæ.} &
\Cite{ep.~17.} &
\Cite{18.~ad Marcellam.}}
mount, \Sc{Iesvs} commanded them, ſaying: Tel the viſion to no body, til
the Sonne of man be riſen from the dead.

\V And his Diſciples asked him, ſaying: what ſay the Scribes then, that
\CNote{\XRef{Ma.~4,~8.}}
Elias muſt come firſt? \V But he anſwering, ſaid to them:
\LNote{Elias shal come}{He
\MNote{Elias.}
diſtinguiſheth here plainly between Elias in Perſon, who is yet to come
before the iudgement; and Elias in name, to wit, 
\CNote{\XRef{Luc.~1,~17.}}
Iohn the Baptiſt,
who is come already in the ſpirit and vertue of Elias. So that it is not
Iohn Baptiſt only, nor principally of whom
\CNote{\XRef{Mal.~4,~5.}}
Malachie prophecieth (as our
Aduerſaries ſay) but Elias alſo himſelf in Perſon.}
Elias in deed ſhal come, and reſtore al things. \V And I ſay to you,
that Elias is already come, and they did not know him, but wrought on
him whatſoeuer they would. So alſo the Sonne of man ſhal ſuffer of
them. \V Then the Diſciples vnderſtood, that of Iohn the Baptiſt he had
ſpoken to them.

\V And
\CNote{\XRef{Mar.~9,~14.}
\XRef{Luc.~9,~37.}}
when he was come vnto the multitude, there came to him a man
falling downe vpon his knees before him, \V ſaying: Lord haue mercie
vpon my Sonne, for he is lunatike, and ſore vexed: for he falleth often
into the fire, and often into the water. \V And I offered him to thy
Diſciples, and they could not cure him. \V \Sc{Iesvs} anſwered and ſaid:
O faithles and peruerſe Generation, how long ſhal I be with
%%% o-2149
you? How long ſhal I ſuffer you? bring him hither to me. \V And
\Sc{Iesvs} rebuked him, and the Diuel went out of him, and the child was
cured from that houre. \V Then came the Diſciples to \Sc{Iesvs}
ſecretly, and ſaid:
\LNote{Why could not we}{No
\MNote{True miracles only in the Cath. Church.}
maruel if the Exorciſts of the Catholike Church which haue power to caſt
out Diuels, yet doe it not alwayes when they wil, and many times with
much a doe; wheras the Apoſtles hauing receaued this power
\CNote{\XRef{Mat.~10.}}
before ouer
vncleane Spirits, yet here cannot caſt them out. But as for Heretikes,
they can neuer doe it, nor any other true miracle, to confirme their
falſe faith.}
why could not we caſt him out? \V \Sc{Iesvs} ſaid to them, becauſe of
your incredulitie: For, Amen I ſay to you, if you haue
\LNote{Faith as a muſtard ſeed}{This is Catholike faith, by which only
  al miracles are wrought; yet not of euery one that hath the Catholike
  faith, but of ſuch as haue a great and forcible faith, and withal the
  gift of miracles. Theſe are able, as here we ſee by Chriſtes warrant,
  not only to doe other wonderful miracles here ſignified by this one,
  but alſo this very ſame, that is, to moue mountaines indeed, as
\CNote{\XRef{1.~Cor.~13.}}
  S.~Paul alſo preſuppoſeth, and
\CNote{\Cite{Hiero. in vita S.~Hilarionis. Niceph. li.~6. c.~17.}}
S.~Hierom. affirmeth, and
  Eccleſiaſtical hiſtories namely telleth of
\CNote{\Cite{Greg. Niſſ. de vit. Gregorij.}}
\MNote{Gregorius Thaumaturgus.}
Gregorius Neocæſarienſis, that he moued a mountaine to make roome for
the foundation of a Church; called therfore, and for other his wonderful
miracles, Thaumaturgus. And yet faithleſſe Heretikes laugh at al ſuch
things and beleeue them not.}
faith as a muſtard ſeed, you ſhal ſay to this mountaine, Remoue from
hence thither, and it ſhal remoue; and nothing ſhal be impoſſible to
you. \V But this kind is not caſt out but by
\LNote{Prayer and faſting}{The
\MNote{Prayer & Faſting.}
force of faſting and praying; wherby alſo we may ſee that the holy
Church in Exorciſmes doth according to the Scriptures, whẽ ſhe vſeth
beſide the name of \Sc{Iesvs}, many prayers, and much faſting, to driue
out Diuels, becauſe theſe alſo are here required beſide faith.}
prayer and faſting.

\V And
\CNote{\XRef{Mr.~9,~31.}
\XRef{Luc.~9,~44.}}
when they conuerſed in Galilee, \Sc{Iesvs} ſaid to them: The Sonne
of man is to be betraied into the hands of men: \V and they ſhal kil
him, and the third day he ſhal riſe againe. And they were ſtroken ſad
exceedingly. 

\V And when they were come to Capharnaum, there came they that receaued
the didrachmes, vnto Peter, and ſaid to him: Your maiſter doth he not
pay the 
\SNote{Theſe didrachmes were peeces of money which they payed for
  tribute.}
didrachmes? \V He ſaith, Yes. And when he was entered into the houſe,
\Sc{Iesvs} preuẽted him, ſaying: What is thy opinion Simon? The kings of
the earth, of whom receaue they tribute or cenſe? of their children, or
of ſtrangers? \V And he ſaid: Of ſtrangers. \Sc{Iesvs} ſaid to him: Then
\LNote{The Children free}{Though 
\MNote{The priuileges & exemptions of the Clergie.}
Chriſt to auoid ſcandal, payed
  tribute, yet indeed he ſheweth that both himſelf ought to be free from
ſuch payments (as being the Kings Sonne, aſwel by his eternal birth of
God the Father, as temporal of Dauid) and alſo his Apoſtles, as being
of his familie, and in them their ſucceſſours the whole Clergie, who are
called in Scripture the lot and portion of our Lord. Which exemption
and priuilege being grounded vpon the very law of nature itſelf, and
therfore practiſed euen among the Heathen
\XRef{(Gen.~42.~27.)}
good Chriſtian Princes haue confirmed and ratified by their lawes, in
the honour of Chriſt, whoſe miniſters they are, and as it were the Kings
Sonnes, as S.~Hierom declareth plainly in theſe words: \Emph{We for his
  honour pay not tributes, and as the Kings Sonnes, are free from ſuch
  payments.}
\Cite{Hiero. vpon this place.}}
the children are free. \V But that we may not ſcandalize them, goe thy waies
to the ſea, and caſt a hooke: and that fiſh which ſhal firſt come vp,
take: and when thou haſt opened his mouth, thou ſhalt find a 
\SNote{This ſtater was a double didrachme, & therfore was payed for two.}
ſtater: take that, and giue it them for
\LNote{Me and thee}{A 
\MNote{Peters preeminence.}
great myſterie in that he payed not only for
  himſelf, but for Peter bearing the Perſon of the Church, and in whom
  as the cheefe, the reſt were contained.
\Cite{Aug. q.~ex. no.~Teſt.}
\Cite{q.~75. Io.~4.}}
me and thee.

\stopChapter


\stopcomponent


%%% Local Variables:
%%% mode: TeX
%%% eval: (long-s-mode)
%%% eval: (set-input-method "TeX")
%%% fill-column: 72
%%% eval: (auto-fill-mode)
%%% coding: utf-8-unix
%%% End:
