%%%%%%%%%%%%%%%%%%%%%%%%%%%%%%%%%%%%%%%%%%%%%%%%%%%%%%%%%%%%%%%%%
%%%%
%%%% The (original) Douay Rheims Bible 
%%%%
%%%% New Testament
%%%% Matthew
%%%% Chapter 26
%%%%
%%%%%%%%%%%%%%%%%%%%%%%%%%%%%%%%%%%%%%%%%%%%%%%%%%%%%%%%%%%%%%%%%




\startcomponent chapter-26


\project douay-rheims


%%% 2356
%%% o-2174
\startChapter[
  title={Chapter 26}
  ]

\Summary{To the Councel of the Iewes, Iudas by occaſion of Marie
Magdalens ointment, doth ſel him for litle. 17.~After the Paſchal
lamb, 26.~he giueth them that bread of life (promiſed
\XRef{Io.~6.})
in a myſtical Sacrifice or Separation of his Body and Bloud. 31.~And
that night he is after his prayer 47.~taken of the Iewes men, Iudas
being their Captaine: and forſaken of the other eleuen for feare:
57.~is falſely accuſed, and impiouſly condemned of the Iewes Councel,
67.~and shamefully abuſed of them: 69.~and thriſe denied of Peter:
Al, euen as the Scriptures and himſelf had often foretold.}

And
\CNote{\XRef{Mr.~14,~1.}
\XRef{Luc.~22,~1.}}
\MNote{\Sc{Tenebre}-weneſday.}
it came to paſſe, when \Sc{Iesvs} had ended al theſe wordes, he ſaid to
his Diſciples: \V You know that after two dayes shal be Paſche, and the
Sonne of man ſhal be deliuered to be crucified. \V Then were gathered
togeather the cheefe Prieſts and Ancients of the People into the court of
the high Prieſt, who was called Caiphas: \V and they conſulted how they
might by ſome wile apprehend \Sc{Iesvs}, and kil him. \V But they ſaid:
Not on the feſtiual day, leſt perhaps there might be a tumult among the
People.

%%% o-2175
\V And
\CNote{\XRef{Mr.~14,~1.}}
when \Sc{Iesvs} was in Bethania in the houſe of Simon the
Leper \V
\CNote{\XRef{Io.~12,~3.}}
there came to him a woman hauing an alabaſter-boxe of pretious
ointment, and powred it out vpon his head as he ſate at the table. \V
And the Diſciples ſeeing it, had indignation ſaying: Whereto is
\LNote{This waſte}{Coſt
\MNote{Coſt vpon Churches, Altares &c.}
beſtowed vpon Chriſts body then aliue, being to the ſame not neceſſary,
ſeemed to the Diſciples loſt and fruitles: ſo the like beſtowed vpon
the ſame body in the Sacrament, vpon Altars, or Churches, ſeemeth to
the ſimple loſt, or leſſe meritorious, then if the ſame were beſtowed
vpon the poore.}
this waiſte? \V For this might haue been ſold for much, and giuen to the
poore. \V And \Sc{Iesvs} knowing it, ſaid to them: Why doe you moleſt
this woman? for ſhe hath wrought a 
\LNote{Good worke}{Coſt
\MNote{Releefe of the poore.}
beſtowed for religion, deuotion, & ſignification, is a meritorious
worke, and often more meritorious then to giue to the poore; though
both be very good, and in ſome caſe the poore are to be preferred: yea
\CNote{\Cite{Ambr. l.~2. c.~28.}}
in certaine caſes of neceſsity, the Church wil breake the very
conſecrated veſſels & iewels of ſiluer, and gold, and beſtow them in
works of mercy. But we may remember very wel, and our Fathers knew it
much better, that the poore were then beſt releeued, when moſt was
beſtowed vpon the Church.}
good worke vpon me. \V For the poore you haue alwayes with you: but me
\LNote{Haue not}{We
\MNote{Chriſt alwaies with vs in the B.~Sacrament.}
haue him not in viſible manner as he conuerſed on the earth with his
Diſciples, needing releeſe like other poore men; but we haue him after
an other ſort in the B.~Sacrament, and yet haue him truly and really
the ſelf ſame body. Therfore he ſaith, they ſhould not haue him,
becauſe they ſhould not ſo haue him, but after an other manner. As when
he ſaid
\XRef{Luc.~24.}
\Emph{When I was with you}; as though he were not then with them.}
you haue not alwayes. \V For ſhe in powring this ointment vpon my body
hath done it to burie me. \V Amen I ſay to you, whereſoeuer this Ghoſpel
ſhal be preached in the whole world, that alſo which ſhe hath done,
%%% 2357
\SNote{Hereby we learne that the good works of Saints are to be recorded
and ſet forth to their honour in the Church after their death. Whereof
riſe their holy daies & Commemorations.}
ſhal be reported for a memorie of her. \V
\CNote{\XRef{Mr.~14,~10.}
\XRef{Luc.~22,~3.}}
Then wẽt one of the Twelue,
which was called Iudas Iſcarioth, to the cheefe Prieſts, & ſaid to them:
What wil you giue me, and I wil deliuer him vnto you? But they appointed
vnto him thirtie peeces of ſiluer. \V And from thenceforth he ſought
opportunitie to betray him.

\V
\MNote{\Sc{Mavndy}-thurſday.}
And
\CNote{\XRef{Mr.~14,~12.}
\XRef{Lu.~22,~7.}}
the firſt day of the Azymes the Diſciples came to \Sc{Iesvs} ſaying:
Where wilt thou that we prepare for thee to eate the Paſche? \V
But \Sc{Iesvs} ſaid: Goe ye into the citie to a certaine man, and ſay to
him: The Maiſter ſaith, my time is at hand, with thee doe I make the
Paſche with my Diſciples. \V And the Diſciples did as \Sc{Iesvs}
appointed thẽ, and they prepared the Paſche. \V But when it was Euen, he
ſate downe with his
\LNote{Twelue}{It
\MNote{A wonderful myſterie in the inſtitution of the B.~Sacrament.}
muſt needs be a great myſterie that he was to worke in the inſtitution
of the new Sacrifice by the maruelous tranſmutation of bread and wine
into his body and bloud: whereas he admitted none (although many
preſent in the citie) but the twelue Apoſtles, which were to haue the
adminiſtration and conſecration thereof by the Order of 
\Fix{Priſtood,}{Prieſthood,}{obvious typo, fixed in other}
which alſo was there giuen them to that purpoſe. Whereas at the
eating of the Paſchal lamb al the familie was wont to be preſent.}
twelue Diſciples. \V And while they were eating, he ſaid: Amen I ſay to
you, that one of you ſhal betray me. \V And they being very ſad, began
euery one to ſay: Is it I Lord? \V But he anſwering ſaid:
\CNote{\XRef{Pſ.~40.~10.}}
He that
dippeth his hand with me in the diſh, he ſhal betray me. \V The Sonne of
man indeed goeth as it is written of him: but woe be to that man, by
whom the Sonne of man ſhal be betrayed. It were good for him, if that
man had not been borne. \V And Iudas that betrayed him, anſwering ſaid:
Is it I Rabbi? He ſaith to him: Thou haſt ſaid.

\V And
\CNote{\XRef{1.~Cor.~11,~14.}}
whiles they were at ſupper, \Sc{Iesvs}
\LNote{He tooke bread}{Here
\MNote{The holy Euchariſt is both a Sacrifice and a Sacrament.}
at once is inſtituted, for the continuance of the external office of
 Chriſtes eternal Prieſthood, according to the order of Melchiſedech,
 both a Sacrifice, and a Sacrament, though the Scriptures giue neither
 of theſe names to this action, and our Aduerſaries without al reaſon or
 religion accept in a ſort the one, and vtterly deny the other. A
 Sacrifice, in that it is ordained to continue the memory of Chriſtes
 death and oblation vpon the Croſſe, and the application of the general
 vertue thereof to our particular neceſsities, by conſecrating the
 ſeueral elements, not into Chriſtes whole Perſon as it was borne of the
 Virgin, or now is in Heauen, but the bread into his body apart, as 
 \Fix{bettrayed,}{betrayed,}{obvious typo, fixed in other}
 broken, and giuen for vs, the wine into his bloud apart, as
 shed out of his body, for remiſsion of ſinnes, and dedication of the
 new Teſtament; which be conditions of his Perſon as he was in Sacrifice
 and Oblation. In which myſtical and vnſpeakable manner, he would haue
 the Church to offer and Sacrifice him daily, and he in myſterie and
 Sacrament dyeth, though now not only in Heauen, but alſo in the
 Sacrament, he be indeed \L{per Concomitantiam} (as the Church calleth
 it; that is, by ſequele of al his partes to each other) whole, aliue,
 and immortal. Which point becauſe our aduerſaries vnderſtand
 not,
\CNote{\XRef{Mt.~22,~29.}}
\Emph{not knowing the Scriptures nor the power of God}, they
 blaſpheme, and abuſe the People to their damnation. It is alſo a
 Sacrament, in that it is ordained to be receaued into our bodies, and
 to feede the ſame to reſurrection and immortality, & to giue grace and
 ſaluation to our ſoules, if we worthily receaue it.}
tooke bread, and
\LNote{Bleſſed}{Our
\MNote{The bleſſing of Chriſt referred to the creatures and working an
effect in them.}
Aduerſaries for the two wordes that are in Greeke and
Latin, \L{benedixit}, and \L{gratias egit}, \Emph{he bleſſed, he gaue
thanks}, vſe only the later, of purpoſe, to ſignifie that Chriſt
bleſſed not nor conſecrated the bread and the wine, & ſo by that
bleſſing wrought any effect vpon them, but gaue thankes only to his
Father, as we doe in ſaying grace. But the truth is that the
word, \G{εὐλογειν}, ſignifieth properly to bleſſe, and is referred to
the thing that is bleſſed, as
\XRef{Luc.~9.}
of the fiſhes,
\G{εὐλόγησεν αὐτοὺς}, \L{benedixit eis}, \Emph{he bleſſed them}: and
thereby wrought in them that wonderful multiplication. So the bleſſing
of God is alwayes affectual,
\MNote{Conſecration.}
and therfore here alſo he bleſſed the bread, and by that bleſſing with
the wordes folowing, made it his body.
\Cite{Ambro. li. de his qui initi. myſt. c.~9.}
\Cite{Aug. ep.~55. ad Paulinum.}
Now whereas taking the cup it
is ſaid: \Emph{he gaue thankes}. We ſay that it is al one with
bleſſing, and that he bleſſed the cup,
as before the bread: as it is euident by theſe wordes of S.~Paul,
\CNote{\XRef{1.~Cor.10,~16.}}
\L{Calix cui benedicimus}, the cup which we bleſſe: and therfore he
calleth it,
\L{Calicem benedicimus}, the cup of bleſsing, vſing the ſame Greeke
word that is ſpoken of the bread. But why is it then ſaid here, he gaue
thankes? becauſe we tranſlate the wordes faithfully as in the Greeke
and the Latin, and becauſe the ſenſe is al one, as we are taught by
S.~Paul before alleaged, and by the Fathers, which cal this giuing of
thankes ouer the cup or ouer the bread, the bleſsing therof. 
\Cite{S.~Iuſtin. In fin.~2. Apol.}
\TNote{\G{τὸν άρτον ευχαριϛηθέντα}}
\L{Panem Euchariſtiſatum.}
 \Cite{S.~Irenee li.~4. c.~34.}
 \L{Panem in quo gratia acta
 ſunt.}
 \Cite{S.~Cyprian de cœn. Do.}
 \L{Calix ſolemni benedictione ſacratus.} that is, \Emph{The bread
 bleſſed by giuing thãkes vpon it,} \Emph{The cup conſecrated by ſolemne
 bleſsing.}} 
bleſſed, and brake: and he gaue to his Diſciples, and ſaid: Take ye, and
eate:
\LNote{This is}{The
\MNote{Tranſubſtantiation.}
bread and the wine be turned into the body and bloud of Chriſt by the
ſame omnipotent power by which the world was made, and the Word was
incarnate in the wombe of the Virgin.
 \Cite{Damaſc. li.~4. c.~14.}
 \Cite{Cypr. de cœn. Domini.}
 \Cite{Amb. li. de myſt. init. c.~9.}}
\Sc{This is}
\LNote{My body}{He
\MNote{No figurative but a real preſence.}
ſaid not: \Emph{This bread is a figure of my body}; or, \Emph{This wine,
 is a figure of my bloud}, but, \Emph{This is my body}, and, \Emph{This
 is my bloud.} 
\Cite{Damaſc. li.~4. c.~14.}
\Cite{Theophyl. in hunc locum. Conc.~2.}
\Cite{Nic. act.~6, to.~4. eiuſdem actionis in fine.}
When ſome Fathers cal it a figure or ſigne, they meane the outward
 formes of bread and wine.}
\Sc{my body.} \V And taking the chalice,
%%% !!! Missing note, extra note mark ???
%%% \LNote{}{}
he gaue thankes: and gaue to them, ſaying: Drinke
\SNote{See the margẽt note
\XRef{Mar.~12,~23.}}
ye al of this. \V \Sc{For this is}
%%% o-2176
\LNote{Bloud of the new Teſtament}{As
\TNote{
%%% !!! Three separate words, each on own line:
ὲκχυνόμενον

κλώμενον

διδόμενον}
the old Teſtament was
dedicated with bloud in theſe words: \Emph{This
is the bloud of the Teſtament &c.}
\XRef{Heb.~9.}
ſo here is the inſtitution of the new Teſtament in
Chriſts bloud, by theſe wordes: \Emph{This is the bloud of the new
Teſtament &c} Which is here myſtically shed, and not only afterward vpon
the Croſſe: for the Greeke is the preſent tenſe in al the Euangeliſtes,
and S.~Paul: and likewiſe ſpeaking of the body
\XRef{1.~Cor.~11.}
it is in the Greeke the preſent tenſe, and
\XRef{Luc.~22.}
and there alſo in the Latin. And the Heretikes
them ſelues ſo put it in their tranſlations.}
\Sc{my blovd of the new Testament which shal be shed for many vnto
remission of sinnes.} \V And I ſay to you, I wil not drinke from
henceforth of this
\LNote{Fruit of the vine}{S.~Luke
\MNote{The elements after conſecration called bread & wine.}
putteth theſe words before he come to the conſecration, wherby it
ſeemeth that he ſpeaketh of the wine of the Paſchal lamb; and therfore
nameth it, the fruit of the vine. But if he ſpeake of the wine which was
now his bloud, he nameth it notwithſtanding wine, as S.~Paul nameth the
other bread, for three cauſes. Firſt becauſe it was ſo before: as Eue
is called
\CNote{\XRef{Gen.~2.}}
Adams bone, and
\CNote{\XRef{Exo.~7.}}
\Emph{Aarons rod deuoured their rods.} Wheras
they were not now rods, but ſerpents. And:
\CNote{\XRef{Io.~2.}}
\Emph{He taſted the water
turned into wine.} Wheras it was now wine & not water; and ſuch
like. Secondly, becauſe it keepeth the formes of bread & wine, and
things are called as they appeare, as when Raphael is called a yong man
\XRef{Tob.~5.}
and, \Emph{Three men appeared to Abraham}
\XRef{Gen.~18.}
Whereas they were three Angels. Thirdly, becauſe
Chriſt in this Sacrament is very true and principal bread and wine,
feeding & refreshing vs in body & ſoule to euerlaſting life.}
fruit of the vine, vntil that day when I ſhal drinke it with you new in
the Kingdom of my Father. \V And an hymne being ſaid, they went forth
vnto Mount-oliuet.

\V
\MNote{\Sc{Thvrsday} night}
Then \Sc{Iesvs} ſaith to them: Al you shal be ſcandalized in me in this
night. For it is written:
\CNote{\XRef{Zac.~13,~7.}}
\Emph{I wil ſtrike the Paſtor, and the sheep
of the flock shal be diſperſed.} \V But after I shal be riſen againe, I
wil goe before you into Galilee. \V And Peter anſwering, ſaid to him:
Although al shal be ſcandalized in thee, I wil neuer be
ſcandalized. \V \Sc{Iesvs} ſaid to him: Amen I ſay to thee, that in this
night before the cock crow, thou ſhalt denie me thriſe. \V
\CNote{\XRef{Io.~13,~38.}}
Peter ſaith
to him: Yea though I should die with thee, I wil not denie thee. Likewiſe
alſo ſaid al the Diſciples.

\V Then \Sc{Iesvs} commeth with them into a village called Gethſemani:
and he ſaid to his Diſciples: Sit you here til I goe yonder, and
pray. \V And taking to him Peter and the two ſonnes of Zebedee, he began
to waxe ſorowful and to be ſad. \V The he ſaith to them: My ſoul is
ſorowful euen vnto death: ſtay here, and watch with me. \V And being
gone forward a litle, he fel vpon his face, praying, and ſaying: My
Father, if it be poſſible, let this chalice paſſe from me. Neuertheleſſe
\LNote{Not as I wil}{A perfect example of obedience & ſubmitting
our ſelf and our willes to Gods wil and ordinance in al aduerſity; and
 that we should deſire nothing temporal, but vnder the condition of his
 holy pleaſure and appointment.}
not as I wil, but as thou. \V And he commeth to his Diſciples, and
findeth them ſleeping, and he ſaith to Peter: Euen ſo? Could you not
watch one houre with me? \V
\LNote{Watch and pray}{Hereof
\MNote{Vigils and Nocturnes.}
came Vigils and Nocturnes, that is, watching and praying in the night,
commonly vſed in the Primitiue Church of al Chriſtians, as is plaine by
\CNote{\Cite{De orat. Do. nu.~15.}}
S.~Cyprian and
\CNote{\Cite{Adu. Vigilent. ep.~53.}}
S.~Hierom; but afterward & vntil this day, ſpecially of Religious
Perſons.} watch ye, & pray that ye enter not
%%% 2358
into tentation. The Spirit in deed is prompt, but the fleſh weak. \V
Againe the ſecond time he went, and prayed, ſaying: My Father, if this
chalice may not paſſe, but I muſt drinke it, thy wil be done. \V And he
commeth againe, and findeth them ſleeping, for their eyes were become
heauy. \V And leauing them, he went againe: and he prayed the third
time, ſaying the ſelfſame word. \V Then he commeth to his Diſciples, and
ſaith to them: Sleepe ye now and take reſt. Behold the houre approcheth,
and the Sonne of man ſhal be betrayed into the hands of ſinners. \V
Riſe, let vs goe: behold he approcheth that ſhal betray me.

\V
\CNote{\XRef{Io.~18,~3.}}
As he yet ſpake, behold Iudas one of the Twelue came, and with him a
great multitude with ſwordes and clubs, ſent from the cheefe Prieſts and
the Ancients of the People.
%%% o-2177
\V And he that betrayed him, gaue them a ſigne, ſaying: Whomſoeuer I
ſhal kiſſe, that is he, hold him. \V And forthwith comming
to \Sc{Iesvs}, he ſaid: Haile Rabbi. And he kiſſed him. \V
And \Sc{Iesvs}, ſaid to him: Freind, wherto art thou come? Then they
drew nere, and laid hands on \Sc{Iesvs}, and held him. \V And behold one
of them that were with \Sc{Iesvs}, ſtretching forth his hand, drew out
his ſword; and ſtriking the ſeruant of the high Prieſt, cut of his
eare. \V Then \Sc{Iesvs} ſaith to him: Returne thy ſword into his place:
for al that take the ſword ſhal periſh with the ſword. \V Thinkeſt thou
that I can not aske my Father, and he wil giue me preſently more then
twelue legions of Angels? \V How then ſhal the ſcriptures be fulfilled,
that ſo it muſt be done? \V In that houre \Sc{Iesvs} ſaid to the
multitudes: You are come out as it were to a theefe with ſwords and
clubs to apprehend me. I ſate daily with you teaching in the temple, and
you laid no hands on me. \V And al this was done, that the ſcriptures of
the Prophets might be fulfilled. Then the Diſciples al leauing him,
fled.

\V But they taking hold of \Sc{Iesvs}, led him to Caiphas the high
Prieſt, where the Scribes and Ancients were aſſembled. \V And Peter
folowed him a farre off, euen to the court of the high Prieſt. And going
in he ſate with the ſeruants, that he might ſee the end. \V And the
cheefe Prieſts and the whole Councel ſought falſe witnes
againſt \Sc{Iesvs}, that they might put him to death: \V and they found
not, whereas many falſe witneſſes had come in. And laſt of al there came
two falſe witneſſes; \V and they ſaid:
\CNote{\XRef{Io.~2,~19.}}
This man ſaid, I am able to
deſtroy the temple of God, and after three dayes to reedifie it. \V And
the high Prieſt riſing vp, ſaid to him: Anſwereſt thou nothing to the
things which theſe doe teſtifie againſt thee? \V But \Sc{Iesvs} held his
peace. And the high Prieſt ſaid to him: I adiure thee by the liuing God,
that thou tel vs if thou be Chriſt the Sonne of God. \V \Sc{Iesvs} ſaith
to him: Thou haſt ſaid. Neuertheles I ſay to you, hereafter you ſhal ſee
\CNote{\XRef{Dan.~7,~13.}}
the Sonne of man ſitting on the right hand of the power of God, and
comming in the clouds of Heauen. \V Then the high Prieſt rent his
garments, ſaying: He hath blaſphemed, what need we witneſſes any
further? Behold, now you haue heard the blaſphemie; \V how thinke you?
But they anſwering ſaid: He is guilty of death. \V Then did they ſpit on
his face, and buffeted him, & other ſmote his
%%% o-2178
face with the palmes of their hands, \V ſaying: Prophecie vnto vs 
\Fix{ô}{O}{Possible typo, fixed in other}
Chriſt; who is he that ſtrook thee?

%%% 2359
\V But Peter ſate without in the court; and there came to him one
\LNote{Wench}{S.~Gregorie
\MNote{The vertue of the holy Ghoſt.}
declaring the difference of the Apoſtles before the receauing of the
 Holy Ghoſt and after, ſaith thus: \Emph{Euen this very Paſtour of the
 Church himſelf, at whoſe moſt ſacred body we ſit, how weake he was, the
 wench can tel you, but how ſtrong he was after, his anſwer to the high
 %%% !!! The citation of Acts below is part of the quote from
 %%% S. Gregory? Should this be marked somehow?
 Prieſt declareth, Act.~5.~29. We muſt obey God rather then men.}
 \Cite{Greg. ho.~20. Io. Euang.}}
wench, ſaying: Thou alſo waſt with \Sc{Iesvs} the Galilean. \V But he
denied before them al, ſaying: I wot not what thou ſayeſt. \V And as he
went out of the gate, an other wench ſaw him, and ſhe ſaith to them that
were there: And this felow alſo was with \Sc{Iesvs} the Nazarite. \V And
againe he denied with an oth: That I know not the man. \V And after a
litle they came that ſtood by, and ſaid to Peter: Surely thou alſo art
of them: for euen thy ſpeach doth bewray thee. \V Then he began
\LNote{To curſe}{A
\MNote{Mans infirmitie.}
goodly example and warning to mans infirmity, to take heed of
 preſumption, and to hang only vpon God in tentations.}
to curſe and to ſweare that he knew not the man. And incontinent the
cock crew. \V And Peter remembred the word of \Sc{Iesvs} which he had
ſaid: Before the cock crow, thou ſhalt deny me thriſe. And going forth,
\LNote{Wept bitterly}{S.~Ambroſe
\MNote{Peters teares & repentance.}
in his Hymne that the Church vſeth at Laudes, ſpeaking of this, ſaith
\L{Hoc ipſa Petra eccleſia canente, culpam diluit.} When the Cock crew,
the Rock of the Church him ſelf waſhed away his fault.
\Cite{S.~Auguſt. 1.~Retract. c.~21.}}
he wept bitterly.


\stopChapter


\stopcomponent


%%% Local Variables:
%%% mode: TeX
%%% eval: (long-s-mode)
%%% eval: (set-input-method "TeX")
%%% fill-column: 72
%%% eval: (auto-fill-mode)
%%% coding: utf-8-unix
%%% End:
