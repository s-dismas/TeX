%%%%%%%%%%%%%%%%%%%%%%%%%%%%%%%%%%%%%%%%%%%%%%%%%%%%%%%%%%%%%%%%%
%%%%
%%%% The (original) Douay Rheims Bible 
%%%%
%%%% New Testament
%%%% Matthew
%%%% Chapter 04
%%%%
%%%%%%%%%%%%%%%%%%%%%%%%%%%%%%%%%%%%%%%%%%%%%%%%%%%%%%%%%%%%%%%%%




\startcomponent chapter-04


\project douay-rheims


%%% 2297
%%% o-2109
\startChapter[
  title={Chapter 4}
  ]

\Summary{Chriſt going into the deſert, to prepare himſelf before his
  Manifeſtation, ouercometh the Diuels tentations. 12.~Beginning in
  Galilee, as the Prophet ſaid he should; 18.~he calleth foure
  Diſciples; and with his preaching and miracles draweth vnto him
  innumerable folowers.}

Then
\CNote{\XRef{Mr.~1,~12.}
\XRef{Lu.~4,~21.}}
\Sc{Iesvs} was led of the Spirit into the
\LNote{Deſert}{As
\MNote{Eremites.}
Iohn the Baptiſt, ſo our Sauiour by going into the deſert, and there
liuing in contemplation euen among brute beaſts, and ſubiect to the
aſſaults of the Diuel for our ſinnes, giueth a warrant and example to
ſuch holy men as haue liued in wilderneſſe for penance and
contemplation, called Eremites.}
deſert, to be tempted of the Diuel. \V And when he had 
\LNote{Faſted fourtie daies}{Elias
\MNote{The Lent-faſt.}
and Moyſes (ſaith S.~Hierom) by the faſt of 40.\ daies, were filled with
the familiaritie of God, and our Lord himſelf in the wilderneſſe faſted
as many to leaue vnto vs the ſolemne daies of faſt (that is, Lent) 
\Cite{Hierom. in c.~58. Eſa.}
S.~Auguſtine alſo hath the very
like words
\Cite{ep.~119.}
And generally al the ancient Fathers that by
occaſion, or of purpoſe ſpeake of the Lent-faſt, make it not only an
imitation of our Sauiours faſt, but alſo an Apoſtolical tradition, and
of neceſſitie to be kept.
\CNote{\Cite{Igna. ep.~5.}}
\Emph{Contẽne not Lent} (ſaith S.~Ignatius)
\Emph{for it containeth the imitation of our Lords conuerſation.} And
S.~Ambroſe ſaith plainely, that
\CNote{\Cite{Ambr. de Quadrag. ſer.~36.~34.}}
\Emph{it was not ordained by men but
  conſecrated by God: nor inuented by any earthly cogitation but
  commaunded by the heauenly Maieſtie.} And againe, that it is ſinne not
to faſt al the Lent. S.~Hieroms words alſo be moſt plaine:
\CNote{\Cite{Hier. ep.~54. ad Marcel. adu. Mõtanũ.}}
\Emph{we}
(ſaith he) \Emph{faſt fourtie daies, or, make one Lent in a yeare,
  according to the tradition of the Apoſtles, in time conuenient.} This
time moſt conuenient is (as S.~Auguſtine ſaith
\Cite{ep.~219.})
immediatly before Eaſter, thereby to communicate with our Sauiours
Paſſion: and (as other writers do adde) thereby to come the better
prepared and more worthily, to the great ſolemnitie of Chriſts
Reſurrection: beſide many other goodly reaſons in the ancient Fathers
which for breuitie we omit. See (good Chriſtian Reader) 12.\ notable
Sermons of S.~Leo the Great
\Cite{de Quadrageſima},
of Lent: namely
\Cite{Ser.~6. and 9.}
where he calleth it the Apoſtles ordinance by the doctrine of the
Holy-Ghoſt. See S.~Ambroſe from the 
\Cite{23.~Sermon}
forward; in S.~Bernard 
\Cite{7.~Sermons},
and in many other Fathers the like. Laſt of
al, note wel the ſaying of
\CNote{\Cite{Aug. Ser. 69. de temp.}}
S.~Auguſtine, who affirmeth that by due
obſeruation thereof, the wicked be ſeparated from the good, Infidels
from Chriſtians, Heretikes from faithful Catholikes.}
faſted fourtie daies and fourtie nights, afterward he was hungrie. \V
And the tempter 
approched and ſaid to him: If thou be Sonne of God, commaund that theſe
ſtones be made bread. \V Who anſwered and ſaid: It is written,
\CNote{\XRef{Deu.~8,~3.}}
\Emph{Not
in bread alone doth man liue, but in euery word that proceedeth from the
mouth of God.}

\V Then the Diuel tooke him vp into the holy citie, and ſet him vpon the
pinnacle of the Temple, \V and ſaid to him: If thou be the Sonne of God,
caſt thy ſelf downe, for 
\LNote{It is written}{Heretikes alleage ſcriptures, as here the Deuil
  doth in the falſe ſenſe; the Church vſeth them, as Chriſt doth in the
  true ſenſe, and to confute their falſehood. 
\Cite{Aug. cont. lit. Petil. lib.~2. c.~58. to 5.}}
it is written:
\CNote{\XRef{Ps.~90,~12.}}
\Emph{That he wil giue his
  Angels charge of thee, & in their hands shal they hold thee vp, leaſt
  perhaps thou knock thy foote againſt a ſtone.} \V \Sc{Iesvs} ſayd to
him againe: It is written,
\CNote{\XRef{Deu.~6,~16.}}
\Emph{Thou shalt not tempt the Lord thy God.}

\V Againe the Diuel tooke him vp into a very high mountaine: and he
ſhewed him the Kingdoms of the world, and the glorie of them, \V and
ſayd to him: Al theſe wil I giue thee, if faling downe thou wilt adore
me. \V Then \Sc{Iesvs} ſaith to him: Auant Satan; for it is written,
\CNote{\XRef{Deu.~6,~13.}}
\Emph{The Lord thy God shalt thou adore, and}
\LNote{Him only ſerue}{It was not ſayd, ſaith
  S.~Auguſtine: The Lord thy God only ſhalt thou adore, as it was ſaid:
  Him only \Emph{shalt thou ſerue}; in Greeke, \G{λατρεύσεις} 
\Cite{Aug. ſup. Gen. q.~16.}
Whervpon the Catholike Church hath
alwayes vſed this moſt true & neceſſarie diſtinction, that there is an
honour dew to God only, which to giue vnto any creature, were idolatrie;
and there is an honour dew to creatures alſo according to their
dignitie, as to Saints, holy things, and holy places. See 
\Cite{Euſeb. Hiſt. Ec. li.~4. c.~14.}
\Cite{S.~Hierom cont. Vigil. ep.~53.}
\Cite{Aug. lib.~10. Ciuit. c.~2.}
\Cite{Li.~1. Trin. c.~6.}
\Cite{Cont. Nic.~2.}
\Cite{Damaſc. li.~1}
\Cite{de Imag. Led. in 4.~Luc.}}
\Emph{him onely shalt thou ſerue.} \V Then the Diuel left him; and
behold Angels came, and miniſtred 
%%% 2298
to him. 

\V And
\CNote{\XRef{Mr.~1,~14.}
\XRef{Luc.~4,~14.}}
\MNote{The third part of the Ghoſpel, of Chriſts manifeſting himſelf by
  preaching, & that in Galilee.}
when \Sc{Iesvs} had heard that Iohn was deliuered vp, he retired into
Galilee: \V and leauing the citie Nazareth, came and dwelt in Capharnaum
a ſea towne, in the borders of Zabulon of Nephthali, \V that it might be
fulfilled which was ſaid by Eſay the Prophet. \V
\CNote{\XRef{Eſa.~9,~1.}}
\Emph{Land of Zabulon &
land of Nephthali, the way of the ſea beyond Iordan of Galilee, of the
Gentils:} \V \Emph{the people that ſate in darkneſſe, hath ſeen great
  light: & to them that ſate in a countrie of the shadow of death, light
is riſen to them.} \V From that time \Sc{Iesvs} began to
%%% o-2110
preach, and to ſay:
\CNote{\XRef{Mr.~1,~15.}}
\LNote{Doe pennance}{That penance is neceſſarie alſo before Baptiſme,
  for ſuch as be of age; as Iohns, ſo our Sauiours preaching declareth,
  both beginning with penance.}
Doe pennance, for the Kingdom of Heauen is at hand.

\V And \Sc{Iesvs}
\CNote{\XRef{Lu.~5,~1.}}
walking by the ſea of Galilee, ſaw two brethren, Simon
who is called Peter, and Andrew his brother, caſting a net into the ſea
(for they were fiſhers) \V and he ſayth to them: Come ye after me, and I
wil make you to be fiſhers of men. \V But they incontinent leauing the
nets, folowed him. \V And going forward from thence, he ſaw
\CNote{\XRef{Mar.~1,~19.}
\XRef{Luc.~5,~10.}}
other two
brethren, Iames of Zebedee, and Iohn his brother, in a ſhip with Zebedee
their father, reparing their nets: and he called them. \V And they
forthwith left their nets and father and folowed him.

\V And \Sc{Iesvs} went round about al Galilee, teaching in their
Synagogues, & preaching the Ghoſpel of the Kingdom: and
\LNote{Healing euery maladie}{Chriſt (ſaith S.~Auguſtine) by miracles
  gat authoritie, by authoritie found credit, by credit drew together a
  multitude, by a multitude obtained antiquitie, by antiquitie fortified
  a Religion, which not only the moſt fond new riſing of Heretikes vſing
  deceitful wiles, but neither the drowſie old errours of the very
  Heathen with violence ſetting againſt it, might in anie part ſhake and
  caſt downe.
\Cite{Aug. de vtil. cred. c.~14.}}
healing euery maladie, and euery infirmitie, in the people. \V And the
bruit of him went into al Syria, and they preſented to him al that were
il at eaſe, diuerſly taken with diſeaſes and torments, and ſuch as were
poſſeſt, and Lunatikes, and ſick of the palſey, and he cured them: \V
And much people folowed him from Galilee, and Decapolis, and
Hieruſalem, and from Iurie and from beyond Iordan.

\stopChapter


\stopcomponent


%%% Local Variables:
%%% mode: TeX
%%% eval: (long-s-mode)
%%% eval: (set-input-method "TeX")
%%% fill-column: 72
%%% eval: (auto-fill-mode)
%%% coding: utf-8-unix
%%% End:
