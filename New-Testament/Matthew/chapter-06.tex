%%%%%%%%%%%%%%%%%%%%%%%%%%%%%%%%%%%%%%%%%%%%%%%%%%%%%%%%%%%%%%%%%
%%%%
%%%% The (original) Douay Rheims Bible 
%%%%
%%%% New Testament
%%%% Matthew
%%%% Chapter 06
%%%%
%%%%%%%%%%%%%%%%%%%%%%%%%%%%%%%%%%%%%%%%%%%%%%%%%%%%%%%%%%%%%%%%%




\startcomponent chapter-06


\project douay-rheims


%%% 2302
%%% o-2114
\startChapter[
  title={Chapter 6}
  ]

\Summary{In this ſecond chapter of his Sermon, he controwleth the
  Phariſees iuſtice (that is, their almes, prayer, and faſting) for the
  ſcope and intention thereof, which was vaine glorie 19.~Their end alſo
was to be rich, but ours muſt not be ſo much as in neceſſaries.}

Take good heed that you doe not your
\LNote{Iuſtice}{Hereby
\MNote{Good workes iuſtice.}
it is plaine that good workes be iuſtice, and that man doing them doth
iuſtice, and is thereby iuſt & iuſtified, & not by faith only. Al which
iuſtice of a Chriſtian man, our Sauiour here compriſeth in theſe three
workes, in Almes, faſting, and prayers. 
\Cite{Aug. li. perf. iuſt. c.~8.}
So that to giue almes, is to do iuſtice, and the workes of mercie are
iuſtice.
\Cite{Aug. in Pſal.~49. v.~5.}}
iuſtice before men, to be ſeen of them: otherwiſe reward you ſhal not
haue with your father which is in heauen.

\V Therfore when thou
\MNote{The firſt worke of iuſtice.}
doeſt an almes-deed, ſound not a trompet before thee, as the Hypocrites
do in the Synagogues and in the ſtreetes, that they may be honoured of
men: Amen I ſay to you, they haue receiued their reward. \V But when
thou doeſt an almes-deed, let not thy left hand know what thy right hand
doeth: \V that thy almes-deed may be in ſecret, and thy father
%%% 2303
which ſeeth in ſecret, wil
\LNote{Repay}{This 
\MNote{Merites.}
repaying and rewarding of good workes in heauen, often mentioned here by
our Sauiour, declareth that the ſayd workes are meritorious, and that we
may do them in reſpect of that reward.}
repay \Fix{the.}{thee.}{Likely typo, same in other}

%%% o-2115
\V And when ye
\MNote{The ſecond worke of iuſtice.}
pray, you ſhal not be as the
\LNote{Hypocrites}{Hypocriſie
\MNote{Hypocriſie.}
is forbidden in al theſe three workes of iuſtice, and not the doing of
them openly to the glorie of God, and the profite of our neighbour, &
our owne ſaluation: for Chriſt before
\XRef{(c.~5.)}
biddeth, ſaying:
\Emph{Let your light ſo shine before men &c.} And in al ſuch workes
S.~Gregories rule is to be followed. \Emph{The worke ſo to be in
  publike, that the intention remayne in ſecret,} 
\Cite{Ho.~11. in Euang. c.~10.}}
Hypocrites, that loue to ſtand and pray in the Synagogues & corners of
the ſtreetes, that they may be ſeen of men: Amen I ſay to you, they haue
receiued their reward. \V But thou when thou ſhalt pray, enter into thy
chamber, and hauing ſhut the doore, pray to thy father in ſecret: and
thy father which ſeeth in ſecret, wil repay thee. \V And when you are
praying, ſpeake not much, as the Heathen. For they thinke that in their
\LNote{Much ſpeaking}{Long prayer is not forbid, for Chriſt
\CNote{\XRef{Luc.~6,~12.}
\XRef{18,~1.}
\XRef{21,~36.}}
himſelf
  ſpent whole nights in prayer; and he ſayth, we muſt pray alwayes; and
\CNote{\XRef{1.~Theſ.~5,~17.}}
  the Apoſtle exhorteth to pray without intermiſſion; and the holy
  Church
\CNote{\XRef{Cypr. de orat. De. in fine.}}
from the beginning hath had her Canonical houres of prayer: but
idle and voluntary babling, either of the Heathens to their gods, or of
Heretikes, that by long Rhetorical prayers thinke to perſuade God:
wheras the Collects of the Church are moſt breefe & moſt effectual. See 
\Cite{S.~Auguſtine ep.~121, c.~8. 9.~10.}}
much-ſpeaking they may be heard. \V Be not you therefore like to them,
for your father knoweth what is needeful for you, before you aske him.

\V Thus therefore ſhal you pray.
\CNote{\XRef{Luc.~11,~2.}}
\MNote{The \Sc{Pater Noster}}
\Sc{Ovr Father} \Emph{which art in heauen, ſanctified be thy name.} \V
\Emph{Let thy Kingdom come. Thy wil be done, as in heauen, in earth
  alſo.} \V \Emph{Giue vs today our}
\SNote{In S.~Luke, the Latin is \L{Panem quotidianum}, \Emph{dayly
    bread}, the Greeke being indifferent to both \G{τὸν ἐπιούσιον}.}
\LNote{Superſubſtantial bread}{By this bread ſo called here
  according to the Latin word, & the 
\TNote{\G{τὸν ἐπιούσιον}}
Greeke, we aske not only al neceſſarie ſuſtenance for the bodie, but
much more al ſpiritual food, namely the
\MNote{The B.~Sacrament.}
bleſſed Sacrament itſelf, which is Chriſt the true bread that came from
Heauen, & the bread of life to vs that eate his bodie. 
\Cite{Cypr. de orat. Do.}
\Cite{Aug. ep.~121. c.~11.}
And therfore it is called here Superſubſtantial, that is, the bread that
paſſeth and excelleth al creatures 
\Cite{Hiero. in 2.~Titus}
\Cite{In 6.~Mat.}
\Cite{Amb. li.~5. de Sacr. c.~4.}
\Cite{Aug. ſer.~18. de Verb. Do. ſec. Mat.}
\Cite{S.~Germanus in Theoria.}}
\Emph{ſuperſubſtãtial bread.} \V \Emph{And forgiue vs our}
\LNote{Debts}{Theſe
\MNote{Venial ſinnes.}
debts do ſignifie not only mortal ſinnes, but alſo venial, as
S.~Auguſtine often teacheth: and therfore euery man be he neuer ſo iuſt,
yet becauſe he can not liue without venial ſinnes, may very truly and
ought to ſay this prayer.
\Cite{Aug. cont. duas ep. Pelag. li.~1. c.~14.}
\Cite{li.~21. de Ciuit. c.~27.}}
\Emph{debtes, as we alſo forgiue our debtors,} \V \Emph{And}
\LNote{Leade vs not}{S.~Cypr.
\CNote{\Cite{In Expoſ. orat. Do.}}
readeth,
\MNote{God is not author of euil.}
\L{Ne patiaris nos induci} Suffer vs not to be led, as S.~Auguſtine
noteth
\Cite{li. de bo. perſeu. c.~6.}
and ſo the holy Church
vnderſtandeth it, becauſe God (as
\CNote{\XRef{Iac.~1.}}
S.~Iames ſayth) tempteth no man:
though for our ſinnes, or for our probation and crowne, he permit vs to
be tempted. Beware then of Beza's expoſition vpon this place, who
(according to the Caluiniſts opinion) ſaith, that God leadeth them into
tentation, into whom himſelf bringeth in Satan for to fil their harts:
ſo making God the authour of ſinne.}
\Emph{leade vs not into tentatiõ. But deliuer vs from euil. Amen.} \V
For
\LNote{If you forgiue}{This poynt, of forgiuing our Brother, when we
aske forgiuenes of God, our Sauiour repeateth agayne, as a thing much to
be conſidered: and therfore commended in the parable alſo of the ſeruant
that would not forgiue his felow ſeruant,
\XRef{Mat.~18.}}
if you wil
\CNote{\XRef{Mr.~11,~25.}}
forgiue men their offences, your heauenly father wil forgiue
you alſo your offences. \V But if you wil not forgiue men, neither wil
your father forgiue you your offences.

\V And when you
\MNote{The third worke of iuſtice.}
\LNote{Faſt}{He
\MNote{Publike faſt.}
forbiddeth not open and publike faſts, which in the
\CNote{\XRef{Iud.~20,~26.}
\XRef{2.~Eſd.~9.}
\XRef{Ioel.~2,~15.}
\XRef{Ion.~3.}}
Scriptures were
commanded and proclaimed to the people of God; and the Niniuites by ſuch
faſting appeaſed Gods wrath: but to faſt for vaine glorie and praiſe of
men, and to be deſirous by the very face and look to be taken for a
faſter, that is forbidden, & that is hypocriſie.}
faſt, be not as the hypocrites, ſad. For they disfigure their faces,
that they may appeare vnto men to faſt. Amen I ſay to you, that they
haue receiued their reward.

\V But thou when thou doeſt faſt, anoynt thy head, and waſh thy face: \V
that thou appeare not to men to faſt, but to thy father which is in
ſecret: and thy father which ſeeth in ſecret, wil repay thee.
\CNote{\XRef{Luc.~12,~33.}}
\V Heape not vp to your ſelues treaſures on the earth: where the ruſt &
mothe do corrupt, & where theeues digge through and ſteale. \V But
heape vp to your ſelues 
\LNote{Treaſures in Heauen}{Treaſures
\MNote{Meritorious workes.}
layd vp in Heauen, muſt needs ſignifie, not faith only, but plentiful
almes, and deeds of mercie, and other good workes, which God keeping, as
in a booke, wil reward them accordingly: as of the contrarie the Apoſtle
ſaith: \Emph{He that ſoweth ſparingly, shal reape ſparingly.}
\XRef{2.~cor.~9.}}
treaſures in heauẽ: where neither the ruſt nor
mothe doth corrupt, and where theeues do not digge through nor
ſteale. \V For where thy treaſure is, there is thy hart alſo. \V
\CNote{\XRef{Luc.~11,~34.}}
The
candel of thy body is thine eye. If thine eye be ſimple, thy whole body
ſhal be lightſome. \V But if thine eye be naught: thy whole body ſhal be
darkeſome. If then the light that is in thee, be darkenes: the darkenes
it ſelf how great ſhal it be?

\V No man can
\CNote{\XRef{Luc.~16,~13.}}
ſerue
\LNote{Two Maſters}{Two religions, God and Baal, Chriſt and Caluin,
  Maſſe and Communion, the Catholike Church and Heretical
  Conuenticles. Let them marke this leſſon of our Sauiour, that thinke
  they may ſerue al maſters, al times, al religions. Agayne, theſe two
  maſters do ſignifie, God and the world, the fleſh and the ſpirit,
  iuſtice and ſinne.}
two maſters. For either he wil hate the one, and loue the other: or he
wil ſuſtayne the one, and contemne the other. You cannot ſerue God and
Mammon. 

\V Therfore I ſay to you,
\CNote{\XRef{Luc.~12,22.}}
be not
\LNote{Careful}{Prudent prouiſion is not prohibited, but too much
  doubtfulnes and feare of Gods prouiſion for vs: to whom we ought with
  patience to commit the reſt, when we haue done ſufficiently for our
  part.}
careful for your life, what you ſhal eate, neither for your body what
rayment you ſhal 
%%% o-2116
put on. Is not the life more then the meate: and the
body more then the rayment? \V Behold the foules of the ayre, that they
ſow not, neither reape, nor gather into barnes: and your heauenly father
feedeth thẽ. Are not you much more of price then they? \V And which of
you by caring, can added to his ſtature one cubite? \V And for rayment
why are you careful? Conſider the lilies of the field how they grow:
they labour not, neither do they ſpinne. \V But I ſay to you, that
neither Salomon in al his glorie was arayed as one of theſe. \V And if
the graſſe of the field, which to day is, and to morow is caſt into the
%%% 2304
ouen, God doth ſo clothe: how much more you, O ye of very ſmal faith? \V
Be not careful therefore, ſaying, what ſhal we eate, or what ſhal we
drinke, or wherewith ſhal we be couered? \V for al theſe things the
\SNote{They ſeeke tẽporal things only, and that not of the true God, but
of their Idols, or by their owne induſtrie.}
Heathen do ſeeke after. For your father knoweth that you neede al theſe
things. \V Seeke therefore firſt the Kingdom of God, and the iuſtice of
him, and al theſe things shal be giuen you beſides. \V Be not careful
therfore for the morrow; for the morrow day wil be careful for
itſelf. Sufficient for the day is the euil thereof.

\stopChapter


\stopcomponent


%%% Local Variables:
%%% mode: TeX
%%% eval: (long-s-mode)
%%% eval: (set-input-method "TeX")
%%% fill-column: 72
%%% eval: (auto-fill-mode)
%%% coding: utf-8-unix
%%% End:
