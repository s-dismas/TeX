%%%%%%%%%%%%%%%%%%%%%%%%%%%%%%%%%%%%%%%%%%%%%%%%%%%%%%%%%%%%%%%%%
%%%%
%%%% The (original) Douay Rheims Bible 
%%%%
%%%% New Testament
%%%% Matthew
%%%% Chapter 09
%%%%
%%%%%%%%%%%%%%%%%%%%%%%%%%%%%%%%%%%%%%%%%%%%%%%%%%%%%%%%%%%%%%%%%




\startcomponent chapter-09


\project douay-rheims


%%% 2309
%%% o-2122
\startChapter[
  title={Chapter 9}
  ]

\Summary{The Maiſters of the Iewes he confuteth both with reaſons and
  miracles: defending his remitting of ſinnes, 9.~his eating with
  ſinners, 14.~and his condeſcending to his weake Diſciples, vntil he
  haue made them ſtronger. 18.~shewing alſo in two miracles, the order
  of his prouidence, about the Iewes and Gentils, leauing the one, when
  he called the other: 27.~he cureth two blind men, and one
  poſſeſſed. 35.~And hauing with ſo many miracles togeather, confuted
  his enemies, and yet they worſe and worſe, vpon pitie toward the
  people, he thinketh of ſending true paſtours vnto them.}

And entring into a boat, he paſſed ouer the water, and came into his
owne citie. \V And
\CNote{\XRef{Mr.~2,~3.}
\XRef{Luc.~5,~18.}}
behold they brought to him one ſick of the palſey
lying in bed. And \Sc{Iesvs}
\SNote{We ſee that the faith of one helpeth to obtaine for an other.}
ſeeing their faith, ſaid to the ſick of the palſey: Haue a good hart
Sonne, thy ſinnes are forgiuen thee. \V And behold certaine of the
Scribes ſaid 
\Fix{withing}{within}{obvious typo, fixed in other}
themſelues: 
\LNote{He blaſphemeth}{When the Iewes heard Chriſt remit ſinnes, they
  charged him with blaſphemie, as Heretikes now charge his Prieſts of
  the new Teſtament, for that they remit ſinnes; to whom he ſaid:
  \Emph{Whoſe ſinnes you shal forgiue, they are forgiuen &c.}
\XRef{Io.~20.}}
He blaſphemeth. \V And \Sc{Iesvs} ſeeing their thoughtes, ſaid: Wherfore
think you euil in your harts? \V
\LNote{Whether is eaſier}{The 
\MNote{Men haue power to forgiue ſinnes.}
faithleſſe Iewes thought (as Heretikes
  now adaies) that to forgiue ſinnes was ſo proper to God, that it could
not be communicated vnto man; but Chriſt ſheweth, that as to worke
miracles is otherwiſe proper to God only, and yet this power is
communicated to men, ſo alſo to forgiue ſinnes.}
Whether is eaſier, to ſay, thy ſinnes are forgiuen thee: or to ſay,
ariſe and walk. \V But that you may know that
\LNote{The Sonne of man in earth}{Chriſt had power to remit
ſinnes, and often executed the ſame, not only as he was God, but alſo
as he was a man, becauſe he was head of the Church, and our cheefe
Biſhop & Prieſt according to his manhood, in reſpect wherof al power
was giuen him in Heauen and earth.
\XRef{Mat.~28. v.~18.}}
the Sonne of man hath power in earth to forgiue ſinnes, (then ſaid he to the
ſick of palſey) Ariſe, take vp thy bed, and goe into thy houſe. \V
And he aroſe, and went into his houſe. \V And the multitudes ſeeing it,
were afrayd, and
\LNote{Glorified}{The faithful people did glorifie God, that gaue ſuch
  power to men, for to remit ſinnes, & to doe miracles, knowing that 
\Fix{wihch}{which}{obvious typo, fixed in other}
God committeth to men, is not to his derogation, but to his glorie,
himſelf only being ſtil the principal worker of that effect, men being
only his miniſters, and ſubſtitutes working vnder him, and by his
commiſſion and authoritie.}
glorified God that gaue ſuch power 
\LNote{To men}{Not only Chriſt as he was man, had this power to forgiue
ſinnes, but by him and from him the Apoſtles, and conſequently
Prieſts.
\XRef{Mat.~28.}
\Emph{Al power is giuen me.}
\XRef{Mat.~18.}
\Emph{Whatſoeuer you shal looſe in earth, shal be looſed in Heauen.}
\XRef{Ioan.~20.}
\Emph{whoſe ſinnes you shal forgiue, they are forgiuen.}}
to men.

\V And
\CNote{\XRef{Mr.~2,~14.}
\XRef{Luc.~5,~27.}}
when \Sc{Iesvs} paſſed forth from thence, he ſaw a man ſitting in
the cuſtome-houſe, named Matthew; And he ſaith to him: Folow me. And he
aroſe vp, and folowed him. \V And it came to paſſe as he was ſitting at
meate in the houſe, behold many Publicans and ſinners
%%% 2310
came, and ſate downe with \Sc{Iesvs} and his Diſciples. \V And the
Phariſees ſeeing it, ſaid to his Diſciples: why doth your Maiſter eate
with Publicans and ſinners? \V But \Sc{Iesvs} hearing it, ſaid: They
that are in health, need not a phyſicion, but they that are il at
eaſe. \V But go your wayes & learne what it is, 
\CNote{\XRef{Oſe.~6,~6. }}
\Emph{I wil mercie, &}
\LNote{Not ſacrifice}{Theſe 
\MNote{External Sacrifice.}
are the wordes of the Prophet, who ſpake them euen then when ſacrifices
were offered by Gods commandment; ſo that it maketh not againſt
ſacrifice: But he ſaith that ſacrifice only without mercie, and
charitie, and generally with mortal ſinne, is not acceptable. The Iewes
offered their ſacrifices dewly, but in the meane time they had no pitie
nor mercie on their brethren; that is it, which God miſliketh.}
\Emph{not ſacrifice.} For I am not come to cal the iuſt, but ſinners.

\V Then
\CNote{\XRef{Mar.~2,~18.}
\XRef{Luc.~5,~33. }}
came to him the Diſciples of Iohn, ſaying: Why do we and the
Phariſees
\LNote{Faſt often}{By the often faſting of S.~Iohns Diſciples, we may
gather that he appointed them a preſcript manner of faſting: as it is
certaine he taught them a forme of prayer.
\XRef{Lu.~5. &~11.}}
faſt often, but thy Diſciples do not 
%%% o-2123
faſt? \V And \Sc{Iesvs} ſaid to
them: Can the children of the Bridegroome mourne, as long as the
Bridegroome is with thẽ? But the dayes wil come when the Bridegroome
ſhal be taken away from them, and
\SNote{Chriſt ſignifieth that the Church shal vſe faſting-daies after
  his Aſcenſion.
\Cite{Epiph. in Comp. fid. Cath. Aug. ep.~80.}}
then they ſhal faſt. \V And no body putteth a peece of raw cloth to an
old garment. For he taketh away the peecing therof frõ the garment, and
there is made a greater rent. \V Neither do they put
\LNote{New wine}{By this new wine, he doth plainly here ſignifie
  faſting, and the ſtrait kind of life: by the old bottels, them that can
  not away therewith.}
new wine into old bottels. Otherwiſe the bottels breake, and the wine
runneth out, and the bottels periſh. But new wine they put into new
bottels: and both are preſerued togeather.

\V
\MNote{Mr.~5,~22. Lu.~8,~41.}
As he was ſpeaking this vnto them, behold a certaine Gouernour
approched, and adored him, ſaying: Lord, my daughter is euen now dead;
but come, lay thy hand vpon her, and ſhe shal liue. \V And \Sc{Iesvs}
ryſing vp folowed him, and his Diſciples. \V And behold a woman which
was troubled with an iſſue of bloud
\LNote{Twelue yeares}{This woman a Gentil, had her diſeaſe twelue
yeares, and the Gouerners daughter a Iewe (which is here rayſed to life)
was twelue yeares old.
\XRef{Luc.~8.}
Marke then the Allegorie hereof
in the Iewes and Gentils. As that woman fel ſick when the wench was
borne, ſo the Gentils went their owne wayes into idolatrie, when the
Iewes in Abraham beleeued. Againe, as Chriſt here went to raiſe the
wench, and by the way the woman was firſt healed, and then the wench
reuiued: ſo Chriſt came to the Iewes, but the Gentils beleeued firſt,
and were ſaued; and in the end the Iewes ſhal beleeue alſo.
\Cite{Hiero. in Mat.}}
twelue yeares, came behind him, and touched the hemme of his garment. \V
For she ſaid within herſelf: If I shal
\LNote{Touch only}{Not only Chriſtes wordes, but his garment and touch
  thereof, or any thing to him belonging, might doe, & did miracles,
  force proceeding from his holy Perſon to them.  
\MNote{Relikes and Images.}
Yea this woman returning home
\CNote{\Cite{Euſeb. li.~7. c.~14. hiſt.}}
  ſet vp an Image of Chriſt, for memorie of this benefit,
and the hemme of the ſame Image did alſo miracles. This Image Iulian the
Apoſtate threw downe, and ſet vp his owne in ſteed thereof, which was
immediatly deſtroyed by fire from Heauen. But the image of Chriſt broken
in peeces by the Heathen, the Chriſtians afterward gathering the peeces
togeather placed it in the Church: where it was, as
\CNote{\Cite{li.~5. c.~20.}}
Sozomenus writeth, vnto his time.}
touch only his garment, I shal be ſafe. \V But \Sc{Iesvs} turning and
ſeeing her, ſaid: Haue a good hart daughter, 
\SNote{Loe, her deuotion to the hemme of his garment, was not
  ſuperſtitiõ, but a token of greater faith; ſo is the deuout touching
  of holy relikes.}
thy faith hath made thee ſafe. And the woman became whole from that
houre. \V And when \Sc{Iesvs} was come into the houſe of the Gouernour,
& ſaw minſtrels and the multitude keeping a ſturre, \V he ſaid: Depart,
for the wench is not dead, but ſleepeth. And they laughed him to
ſkorne. \V And when the multitude was put forth, he entred in, and held
her hand. And the maid aroſe. \V And this bruit went forth into al that
countrie. 

\V And as \Sc{Iesvs} paſſed forth from thence, there folowed him two
blind men crying and ſaying: Haue mercie on vs, O Sonne of Dauid. \V And
when he was come to the houſe, the blind came to him. And \Sc{Iesvs}
ſaith to them:
\LNote{Do you beleeue that I can?}{We ſee here that to the corporal
healing of theſe men he requireth only this faith, that he is able;
which faith is not ſufficient to iuſtifie them. How then doe the
Heretikes by this and the like places plead for their only iuſtifying
faith? See the
\XRef{Annot. Mar.~5,~36.}}
Do you beleeue, that I can doe this vnto you? They ſay to him: Yea
Lord. \V Then he touched their eyes, ſaying: According to your faith, be
it donne to you. \V And their eyes were opened, and \Sc{Iesvs} threatned
them, ſaying: See that no man know it. \V But they went forth, and
bruited him in al that countrie.

\V And when they were gone forth,
\CNote{\XRef{Mat.~12,~22.}}
behold they brought him a dumme man,
poſſeſſed with a Diuel. \V And after the Diuel was caſt out, the dumme
man ſpake, and the multitudes marueled ſaying: Neuer was the like ſeene
in Iſrael. 
%%% o-2124
\V But
\CNote{\XRef{Mt.~12,~24.}}
the Phariſees ſaid:
\SNote{In like manner ſay the Heretikes, calling al miracles done in the
Catholike Church, the lying ſignes of Antichriſt.}
In the Prince of Diuels he caſteth out Diuels.

\V And \Sc{Iesvs} went about al the cities, and townes, teaching in
their Synagogues, and preaching the Ghoſpel of the Kingdom, and curing
euery diſeaſe, and euery infirmitie. \V And ſeing the multitudes,
%%% 2311
he pitied them; becauſe they were vexed, and lay like ſheep that haue
not a ſhepeard. \V Then he ſaith to his Diſciples: The harueſt ſurely
is great, but the workmen are few. \V
\LNote{Pray therfore}{Therfore doth the Church pray and faſt in the
Imber dayes, when holy Orders are giuen, that is, when workmen are
prepared to be ſent into the harueſt. See
\XRef{Act.~13.}}
Pray therfore the Lord of the harueſt, that he ſend forth workmen into
his harueſt.

\stopChapter


\stopcomponent


%%% Local Variables:
%%% mode: TeX
%%% eval: (long-s-mode)
%%% eval: (set-input-method "TeX")
%%% fill-column: 72
%%% eval: (auto-fill-mode)
%%% coding: utf-8-unix
%%% End:
