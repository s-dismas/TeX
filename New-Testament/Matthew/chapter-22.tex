%%%%%%%%%%%%%%%%%%%%%%%%%%%%%%%%%%%%%%%%%%%%%%%%%%%%%%%%%%%%%%%%%
%%%%
%%%% The (original) Douay Rheims Bible 
%%%%
%%%% New Testament
%%%% Matthew
%%%% Chapter 22
%%%%
%%%%%%%%%%%%%%%%%%%%%%%%%%%%%%%%%%%%%%%%%%%%%%%%%%%%%%%%%%%%%%%%%




\startcomponent chapter-22


\project douay-rheims


%%% 2345
%%% o-2161
\startChapter[
  title={Chapter 22}
  ]

\Summary{Yet by one other parable he foresheweth the moſt deſerued
  reprobation of the earthly & perſecuting Iewes, and the gratious
  vocation of the Gentils in their place. 15.~Then he defeateth the
  ſnare of the Phariſees and Herodians about paying tribute to
  Cæſar. 23.~He anſwereth alſo the inuention of the Sadducees againſt
  the Reſurrection: 34.~and a queſtion that the Phariſees aſke to poſe
  him: turning and poſing them againe, becauſe they imagined that Chriſt
  should be no more then a man: 46.~and ſo he putteth al the buſy ſects to
  ſilence.}

%%% o-2162
And \Sc{Iesvs} anſwering, ſpake againe in parables to them, ſaying: \V
The Kingdom of Heauen is likened to a man being a King, which made a
\LNote{Mariage}{Then did God the Father make this mariage, when by the
  myſterie of the Incarnation he ioyned to his Sonne our Lord, the holy
  Church for his ſpouſe. 
\Cite{Greg. hom.~38.}}
mariage to his ſonne. \V And he ſent his
\LNote{Seruants}{The firſt ſeruants here ſent to inuite, were the
  Prophets, the ſecond, were the Apoſtles, and al that afterward conuerted
  Countries, or that haue and doe reconcile men to the Church.}
ſeruants to cal them that were inuited to the mariage: and they would
not come. \V Againe he ſent other ſeruants, ſaying: Tel them that were
inuited, Behold I haue prepared my dinner; my beeues & fatlings are
killed, and al things are ready: come ye to the mariage. \V But they
neglected, and went their waies,
\LNote{One to his farme}{Such 
\MNote{Worldly excuſe againſt reconciliation.}
as refuſe to be reconciled to Chriſtes
  Church, alleage often vaine impediments, and worldly excuſes, which at
the day of iudgement wil not ſerue them.}
one to his farme, and an other to his merchãdiſe: \V and the reſt laid
hands vpon his ſeruants, and ſpitefully intreating them, murdered
them. \V But when the King had heard of it, he was wroth, and ſending
his hoſts, deſtroied thoſe murderers, and burnt their citie. \V Then he
ſaith to his ſeruants: The mariage indeed is ready: but they that were
inuited, were not worthie. \V Goe ye therfore into the high wayes; and
whoſoeuer you shal find, cal to the mariage. \V And his ſeruants going
forth into the wayes, gathered togeather al that they found,
\SNote{Not only good men be within the Church, but alſo euil men:
  againſt the Heretikes of theſe daies.}
bad and good: and the mariage was filled with gheſts. \V And the King
went in to ſee the gheſts: and he ſaw there
\LNote{A man not attyred}{It profiteth not much to be within the Church
  and to be a Catholike, except a man be of good life, for ſuch an one
  ſhal be dãned, becauſe with faith he hath not good workes, as is
  euident by the example of this man, who was within, & at the feaſt as
  the reſt, but lacked the garment of charitie & good workes. 
\MNote{The Church conſiſteth of good and bad.}
And by
  this man are repreſented al the bad that are called. And therfore they
alſo are in the Church as this man was at the feaſt: but becauſe he was
called, and yet none of the elect, it is euident that the Church doth
not conſiſt of the elect only, contrarie to our Aduerſaries.}
a man not attired in a wedding garment. \V And he ſaith to him: Freind,
how cameſt thou in hither not hauing a wedding garment? But he was
dumme. \V Then the King ſaid to the waiters: Bind his hands and feet,
and caſt him into the vtterdarkenes: there shal be weeping & gnashing of
teeth. \V For many be called, but few elect.

\V
\CNote{\XRef{Mr.~12,~13.}
\XRef{Lu.~20,~20.}}
Then the Phariſees departing, conſulted among them ſelues for to
entrap him in his talke. \V And they ſend to him their Diſciples with
the Herodians, ſaying: Maiſter, we know that thou art a true ſpeaker,
and teacheſt the way of God in truth, neither careſt thou for any
man. For thou doſt not reſpect the perſon of men: \V Tel vs therfore
what is thy opinion, is it lawful to giue tribute to Cæſar, or not? \V
But \Sc{Iesvs} knowing their naughtines, ſaid: What do you tempt me
Hypocrites? \V Shew me the tribute coine. And they offred him a
penie. \V And \Sc{Iesvs} ſaith to them: Whoſe is this image and
ſuperſcription? \V They ſay to him: Cæſars. Then he ſaith to them:
Render therfore the things that are Cæſars,
\LNote{To Cæſar}{Temporal
\MNote{Neither muſt tẽporal Princes exact, nor their Subiects giue vnto
  thẽ, Eccleſiaſtical iuriſdiction.}
duties and payments exacted by worldly Princes muſt be payd, ſo that God
be not defrauded of his more ſoueraigne dutie. And therfore Princes haue
to take heed how they exact, and others how they giue to Cæſar, that is,
to their Prince, the things that are due to God, that is, to his
Eccleſiaſtical miniſters. Wherevpon S.~Athanaſius reciteth theſe goodly
wordes out of an epiſtle of the ancient & famous Confeſſour Hoſius
Cordubenſis to Conſtantius the Arian Emperour: Ceaſe I beſeech thee and
remember that thou art mortal, feare the day of iudgement, intermedle
not with Eccleſiaſtical matters, neither doe thou command vs in this
kind but rather learne them of vs. To thee God hath committed the
Empire, to vs he hath committed the things that belong to the
Church. And as he that with malicious eyes carpeth thine Empire,
gaineſaieth the ordinance of God: ſo doe thou alſo beware, leſt in
drawing vnto thee Eccleſiaſtical matters, thou be made guilty of a great
crime. It is writtẽ: Giue ye the things that are Cæſars, to Cæſar, and
the things that are Gods, to God. Therfore neither is it lawful for vs
in earth to hold the Empire, neither haſt thou (O Emperour) power ouer
incenſe and ſacred things. 
\Cite{Athan. Ep. ad Solit. vitam agentes.}
And S.~Ambroſe to
Valentinian the Emperour (who by the il counſel of his mother Iuſtina
an Arian, required of S.~Ambroſe to haue one Church in Millan deputed
to the Arian Heretikes) ſaith: We pay 
\Fix{thay}{that}{obvious typo, fixed in other}
which is Cæſars, to Cæſar: and that which is Gods, to God. Tribute is
Cæſars, it is not denied: the Church is Gods, it may not verily be
yealded to Cæſar: becauſe the Temple of God can not be Cæſars
right. Which no man can denie but it is ſpoken with the honour of the
Emperour, for what is more honorable then that the Emperour be ſaid to
be the ſõne of the Church? For a good Emperour is within the Church, not
aboue the Church.
\Cite{Ambr. l.~5. Epiſt. Orat. de Baſil. trad.}}
to Cæſar: and the things that are Gods, to God. \V And hearing it they
marueled, and leauing him went their wayes.

\V
\CNote{\XRef{Mr.~12,~18.}
\XRef{Luc.~29,~27.}
\XRef{Act.~23,~6.}}
That day there came to him the Sadducees, that ſay 
%%% o-2163
there is no
Reſurrection, and asked him, \V ſaying: Maiſter, Moyſes ſaid,
\CNote{\XRef{Deu.~25,~5.}}
\Emph{If a
man die not hauing a child, that his brother marie his wife, and raiſe
vp ſeed to his brother.} \V And there were with vs ſeauen brethren: and
the firſt hauing maried a wife, died; and not hauing iſſue, left his
wife to his brother. \V In like manner the ſecond and the third euen to
the ſeauenth. \V And laſt of al the woman died alſo. \V In the
Reſurrection therfore whoſe wife of the ſeauen ſhal ſhe be? for they al
had her. \V And \Sc{Iesvs} anſwering, ſaid to them: You doe erre, not
knowing the Scriptures, not the power of God. \V For in the Reſurrection
neither ſhal they marie nor be maried: but are
\LNote{As Angels}{As
\MNote{The Saints heare our prayers.}
Chriſt proueth here, that in Heauen they neither marie nor are maried,
becauſe there they ſhal be as Angels; by the very ſame reaſon, is
proued, that Saints may heare our prayers and help vs, be they neer or
farre of; becauſe the Angels do ſo, and in euery monent are preſent
where they liſt, and need not to be neer vs, when they heare, or help
vs.}
\LNote{As Angels}{Not
\MNote{Religious ſingle life, Angelical.}
to marie nor be maried, is to be like to Angels: therfore is the
ſtate of Religious men, and women, and Prieſts, for not marying,
worthily called of the Fathers an Angelical life.
\Cite{Cyp. lib.~2. de deſcipl. & hab. Virg. ſub finem.}}
as the Angels of God in Heauen. \V And concerning the Reſurrection of
the dead, haue you not read that which was ſpoken of God ſaying to
you. \V
\CNote{\XRef{Exo.~3,~6.}}
\Emph{I am the God of Abraham, and the God of Iſaac, and the God
of Iacob}? He is not God
\LNote{Of the dead}{S.~Hierom by this place diſproueth the Heretike
  Vigilantius, and in him theſe of our time, which to diminiſh the
  honour of Saints, cal them of purpoſe, dead men.}
of the dead, but of the liuing. \V And the multitudes hearing it,
marueled at his doctrine.

\V
\CNote{\XRef{Mr.~12,~28.}}
But the Phariſees hearing that he had put the Sadducees to ſilẽce,
came togeather: \V and one of them a Doctour of law asked of him,
tempting him: \V Maiſter, which is the great commandement in the law? \V
\Sc{Iesvs} ſaid to him:
\CNote{\XRef{Dut.~6,~5.}}
\Emph{Thou shalt loue the Lord thy God from thy
  whole hart, and with thy whole ſoul, and with thy whole mind.} \V This
is the greateſt & the firſt commandement. \V And the ſecond is like to
this:
\CNote{\XRef{Lu.~19,~18.}}
\Emph{Thou shalt loue thy neighbour as thy ſelf.} \V
\LNote{On theſe two}{Hereby
\MNote{Not only faith.}
it is euident that al dependeth not vpon faith only, but much more vpõ
charitie (though faith be the firſt) which is the loue of God, and of
our neighbour, which is the ſumme of al the law and the Prophets,
becauſe he that hath this double charitie expreſſed here by theſe two
principal commandements, fulfilleth and accompliſheth al that is
commanded in the Law and the Prophets.}
On theſe two commandements dependeth the whole Law and the Prophets.

\V And
\CNote{\XRef{Mr.~12,~35.}
\XRef{Luc.~20,~41.}}
the Phariſees being aſſembled, \Sc{Iesvs} asked them \V ſaying:
What is your opinion of Chriſt? whoſe ſonne is he? They ſay to him,
Dauids. \V He ſaith to them: How then doth Dauid in ſpirit cal him Lord,
ſaying: \V
\CNote{\XRef{Pſ.~109,~1.}}
\Emph{The Lord ſaid to my Lord, ſit on my right hand, vntil I put
thine enemies the foot-ſtole of thy feet}? \V If Dauid therfore cal him
Lord, how is he his ſonne? \V And no man could anſwer him a word:
neither durſt any man from that day ask him any more.

\stopChapter


\stopcomponent


%%% Local Variables:
%%% mode: TeX
%%% eval: (long-s-mode)
%%% eval: (set-input-method "TeX")
%%% fill-column: 72
%%% eval: (auto-fill-mode)
%%% coding: utf-8-unix
%%% End:
