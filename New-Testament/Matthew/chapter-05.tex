%%%%%%%%%%%%%%%%%%%%%%%%%%%%%%%%%%%%%%%%%%%%%%%%%%%%%%%%%%%%%%%%%
%%%%
%%%% The (original) Douay Rheims Bible 
%%%%
%%%% New Testament
%%%% Matthew
%%%% Chapter 05
%%%%
%%%%%%%%%%%%%%%%%%%%%%%%%%%%%%%%%%%%%%%%%%%%%%%%%%%%%%%%%%%%%%%%%

%%% Latin checked by KK.



\startcomponent chapter-05


\project douay-rheims


%%% 2299
%%% o-2111
\startChapter[
  title={Chapter 5}
  ]

\Summary{Firſt, 3.~he promiſeth rewardes, 13.~and he layeth before the
  Apoſtles their offices. 17.~Secondly, he proteſteth vnto vs that we
  muſt keep the commandements, and that more exactly then the Scribes &
  Phariſees, whoſe iuſtice was counted moſt perfit; but yet that it was
  vnſufficient, he sheweth in the precepts of 21.~Murder, 27.~Aduoutrie,
31.~Diuorce, 33.~Swearing, 38.~Reuenge, 42.~Vſurie, 43.~Enemies.}

And
\MNote{The Sermon of Chriſt vpon the Mount; containing the paterne of a Chriſtiã
life, in theſe three chapters folowing wherof S.~Auguſtine hath two
goodly bookes
\XRef{Io.~4.}}
ſeeing the multitudes, he
\CNote{\XRef{Luc.~6,~20.}}
went vp into a mountaine, and when he was ſet,
his Diſciples came vnto him \V and opening his mouth he taught them,
ſaying. 

\V Bleſſed
\MNote{The eight Beatitudes; which are a part of the Catechiſme.}
are the poore in Spirit: for theirs is the Kingdom of Heauen. \V Bleſſed
are the meek: for they ſhal poſſeſſe the land. \V Bleſſed are they that
mourne: for they ſhal be comforted. \V Bleſſed are they that hunger and
thirſt after iuſtice: for they ſhal haue their fil. \V Bleſſed are the
merciful: for they ſhal obtayne mercie. \V Bleſſed are the cleane of
hart: for they ſhal ſee God. \V Bleſſed are the peace-makers: for they
ſhal be called the children of
%%% 2300
God. \V Bleſſed are they that ſuffer perſecution
\LNote{For iuſtice}{Heretikes
\MNote{Falſe Martyrs.}
and other malefactours ſometime ſuffer willingly and ſtoutly: but they
are not bleſſed, becauſe they ſuffer not for iuſtice. For (ſayth
S.~Aug.) they can not ſuffer for iuſtice, that haue deuided the Church,
and, where ſound faith or charitie is not, there cannot be iuſtice.  
\Cite{Cont. ep. Parm. li.~1. c.~9.}
\Cite{Ep.~50. Pſal.~4. Cont.~2.}
And ſo by this ſcripture are excluded al falſe Martyrs, as S.~Auguſtine
often declareth, and
\Cite{S.~Cypr. de Vnit. Eccl. nu.~8.}}
for iuſtice: for theirs is the kingdom of Heauen. \V Bleſſed are ye when
they ſhal reuile you, and perſecute you, & ſpeake al that naught is
againſt you, vntruly, for my ſake: \V be glad & reioyce, for your
\LNote{Reward}{In 
\TNote{\L{Merces} \G{Μισθὸς}}
Latin and Greeke the word ſignifieth very wages, and hire, due for
workes, and ſo preſuppoſeth a meritorious deede.}
reward is very great in Heauen. For ſo they perſecuted the Prophets,
that were before you.

\V You are the
\CNote{\XRef{Mr.~9,~50.}
\XRef{Luc.~14,~34.}}
ſalt of the earth. \V But if the ſalt leefe his vertue,
wherewith ſhal it be ſalted? It is good for nothing any more but to be
caſt forth, and to be troden of men. \V You are 
\LNote{The light}{This 
\MNote{The Church viſible}
light of the world, and citie on a mountayne,
  and candle vpon a candleſticke, ſignifie the Clergie, and the whole
  Church, ſo built vpon Chriſt the mountayne, that it muſt needes be viſible,
and cannot be hid nor vnknowen. 
\Cite{Aug. cont. Fulg. Dona. c.~18.}
\Cite{Lib.~16. cont. Fauſt. c.~17.}
And therfore, the Church being a candle not vnder a buſhel, but ſhining
to al in the houſe (that is) in the world, what ſhal I ſay more (ſayth
S.~Auguſtine) then that they are blind which ſhut their eyes againſt the
candle that is ſet on the candleſticke? 
\Cite{Tract.~2. in ep.~Io.}}
the light of the world. A citie cannot be hid, ſituated on a mountaine. \V
Neither do men light a 
%%% o-2112
\CNote{\XRef{Mr.~4,~21.}
\XRef{Lu.~8,~16.}
\XRef{11,~33.}}
candel and put it vnder a buſhel, but vpon a
candleſtike, that it may ſhine to al that are in the houſe. \V So let 
\SNote{The good life of the Clergie edifieth much, and is
  Gods great honor: whereas the contrarie diſhonoureth him.}
your light shine before men, that they may ſee your good workes, and
glorifie your Father which is in Heauen.

\V Doe not thinke that I am come to breake the Law, or the Prophets. I
am not come to breake, but to fulfil. \V For aſſuredly I ſay vnto you,
\CNote{\XRef{Luc.~16,~17.}}
til Heauen and earth paſſe, one iot, or one tittle shal not paſſe of the
Law, til al be fulfilled. \V He therfore that shal
\CNote{\XRef{Ia.~2,~10.}}
breake
\LNote{One of theſe}{Behold
\MNote{True inherent iuſtice.}
how neceſſarie it is, not only to beleeue, but to keep al the
commaundements, euen the very leaſt.}
one of theſe leaſt commandements, and shal ſo teach men, shal be caled
the leaſt in the Kingdom of Heauen. But he that shal doe and teach, he
shal be called great in 
\Fix{}{the }{obvious typo, fixed in other}% to prevent a paragraph break
Kingdom of heauen. \V For I tel you, that vnles
\LNote{Your Iuſtice}{It is our iuſtice, when it is giuen vs of God.
\Cite{Aug. in Ps.~30. Conc.~1. De Sp. & lit. C.~9.}
So that Chriſtians are truly iuſt, & haue in themſelues inherent iuſtice,
by doing Gods commaundements, without which iuſtice of workes no man of
age can be ſaued.
\Cite{Aug. de fid. & oper. C.~16.}
Whereby we ſee ſaluation, iuſtice, & iuſtification, not to come of only
faith, or imputation of Chriſtes iuſtice.}
your iuſtice abound more then that of the Scribes and Phariſees, you
shal not enter into the Kingdom of Heauen.

\V You haue heard that it was ſaid to them of old:
\CNote{\XRef{Exo.~20,~13.}
\XRef{Deut.~5,~17.}}
Thou shalt not
kil. And whoſo killeth, shal be in danger of iudgement. \V But I ſay to
you, that whoſoeuer is angrie with his brother, shal be in danger of
iudgment. And whoſoeuer shal ſay to his brother, Raca, shal be in danger
of a councel. And whoſoeuer shal ſay, Thou foole, shal be guilty of the
\LNote{Hel of fyre}{Here
\MNote{Venial ſinnes.}
is a playne difference of ſinnes, ſome mortal, that bring to Hel,
ſome leſſe, and leſſe puniſhed, called venial.}
Hel of fire. \V If therfore thou offer thy
\LNote{Guift at the altar}{Beware of coming to the holy altar or any
  Sacrament out of charitie. But be firſt reconciled to thy brother, and
much more to the Catholike Church, which is the whole brotherhood of
Chriſtian men,
\XRef{Heb.~13,~1.}}
guift at the Altar, and there thou remember that thy brother hath ought
againſt thee; \V leaue there thy offering before the Altar, and goe
firſt to be reconciled to thy brother: and then coming thou shalt offer
thy guift. \V
\CNote{\XRef{Luc.~12,~58.}}
Be at agreement with thy aduerſarie betimes, whiles thou
art in the way with him; leſt perhaps the aduerſarie deliuer thee to the
iudge, and the iudge deliuer thee to the officer, and thou be caſt into
\SNote{This Priſon is takẽ of very anciẽt Fathers, for Purgatorie:
  namely
\Cite{S.~Cypr. ep.~12. ad Anton. nu.~6.}}
priſon. \V Amen I ſay to thee, thou shalt not goe out from thence til
thou repay the laſt farthing.

\V You haue heard that it was ſaid to them of old:
\CNote{\XRef{Exo.~20,~14.}}
Thou shalt not commit
aduoutrie. \V But I ſay to you, that whoſoeuer shal ſee a woman to luſt
after her, hath already committed aduoutrie with her in his hart. \V And
if thy right eye ſcandalize thee, pluck it out, & caſt it from thee. For
it is expedient for thee that one of thy limmes periſh, rather then thy
whole body be caſt into Hel. \V And if thy right hand ſcandalize thee,
cut it of, and caſt it from thee: for it is expedient for thee that one
of thy limmes periſh rather then that thy whole body goe into Hel.

\V It was ſaid alſo,
\CNote{\XRef{Deu.~24,~1.}
\XRef{Mt.~19,~6.}}
whoſoeuer ſhal diſmiſſe his wife, let 
%%% o-2113
him giue
%%% 2301
her a bil of diuorcemẽt. \V But I ſay to you, whoſoeuer ſhal diſmiſſe
his wife, 
\LNote{Excepting the cauſe of fornication}{This exception is only to
  ſhew, that for this one cauſe a man may put away his wife for euer: but
not that 
\Fix{be}{he}{obvious typo, fixed in other}
may marrie an other as it is moſt plaine in S.~Marke and S.~Luke, who
leaue out this exception, ſaying: 
\MNote{Mariage a Sacrament and is not diſſolued by diuorce.}
\CNote{\XRef{Mr.~10,~11.}
\XRef{Lu.~16,~18.}}
\Emph{Whoſoeuer diſmiſſeth his wife and marieth an other, committeth
  aduoutrie.} See the
\XRef{Annot. Luc.~19,~9.}
But if both parties be in one and the ſame fault, then can neither of
them not ſo much as deuorce or put away the other.}
excepting the cauſe of fornication, maketh her to commit
aduoutrie: And he that ſhal marie her that is diſmiſſed; 
\LNote{Committeth aduoutrie}{The knot of Mariage is a thing of ſo great
a Sacrament, that not by ſeparation itſelf of the parties it can be
looſed, being not lawful neither for the one part nor the other, to
marie againe vpon deuorce.
\Cite{Aug. de bo. Coniug. c.~7.}}
committeth aduoutrie.

\V Againe you haue heard that it was ſayd to them of old,
\CNote{\XRef{Exo.~20,~7.}
\XRef{Leu.~19,~11.}}
Thou ſhalt not
commit periurie: but thou ſhalt performe thy othes to our Lord. \V But I
ſay to you
\LNote{Not to ſweare}{The Anabaptiſts here not folowing the Churches
  iudgement, but the bare letter (as other Heretikes in other caſes)
  hold that there is no oath lawful, no not before a iudge, whereas
  Chriſt ſpeaketh againſt raſh and vſual ſwearing in common talke, when
  there is no cauſe.}
not to ſweare at al: neither by heauen, becauſe it is the throne of God:
neither by the earth, becauſe it is the foote-ſtole of his feete:
neither by Hieruſalem, becauſe it is the citie of the great King. \V
Neither ſhalt thou ſweare by thy head, becauſe thou canſt not make one
heare white or blacke. \V Let your talke be yea, yea: no, no: and that
which is ouer & aboue theſe, is of euil.

\V You haue heard that it was ſayd,
\CNote{\XRef{Exo.~21,~24.}}
An eye for an eye, and a tooth for a
tooth. \V But I ſay to you
\LNote{Not to Reſiſt euil}{Here alſo the Anabaptiſts gather of the
  letter, that it is not lawful to go to law for our right; as Luther
  alſo vpon this place held, that Chriſtians might not reſiſt the
  Turke. Whereas by this, as by that which foloweth, patience only is
  ſignified, & a wil to ſuffer more, rather then to reuenge. For neither
did Chriſt nor S.~Paul folow the letter, by turning the other cheeke.
\XRef{Io.~18.}
\XRef{Act.~23.}}
not to reſiſt euil: but if one ſtrike thee on thy right cheeke, turne to
him alſo the other: \V and to him that wil cõtend with thee in
iudgement, and take away thy coate, let goe thy cloke alſo vnto him. \V
and whoſoeuer wil force thee one mile, goe with him other twayne. \V He
that asketh of thee, giue to him: and
\CNote{\XRef{Deu.~15,~7.}}
to him that would borow of thee, turne not away.

\V You haue heard that it was ſayd,
\CNote{\XRef{Leu.~19,~18.}}
Thou ſhalt loue thy neighbour, &
\SNote{So taught the Phariſees, not the Law.}
hate thine enemie. \V But I ſay to you loue your enemies, doe good to
thẽ that hate you: and pray for thẽ that perſecute and abuſe you: \V
that you may be the children of your father which is in heauen, who
maketh his ſunne to riſe vpon good & bad, and rayneth vpon iuſt and 
\SNote{We ſee then that the tẽporal proſperitie of perſons and countries
is no ſigne of better men of truer religion.}
vniuſt. \V For if you loue them that loue you, what reward ſhal you
haue, do not alſo the Publicans this? \V And if you ſalute your brethren
only, what do you more, do not alſo the Heathen this? \V Be you perfect
therfore, as alſo your heauenly Father is perfect.

\stopChapter


\stopcomponent


%%% Local Variables:
%%% mode: TeX
%%% eval: (long-s-mode)
%%% eval: (set-input-method "TeX")
%%% fill-column: 72
%%% eval: (auto-fill-mode)
%%% coding: utf-8-unix
%%% End:
