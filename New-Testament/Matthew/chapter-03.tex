%%%%%%%%%%%%%%%%%%%%%%%%%%%%%%%%%%%%%%%%%%%%%%%%%%%%%%%%%%%%%%%%%
%%%%
%%%% The (original) Douay Rheims Bible 
%%%%
%%%% New Testament
%%%% Matthew
%%%% Chapter 03
%%%%
%%%%%%%%%%%%%%%%%%%%%%%%%%%%%%%%%%%%%%%%%%%%%%%%%%%%%%%%%%%%%%%%%




\startcomponent chapter-03


\project douay-rheims


%%% 2295
%%% o-2107
\startChapter[
  title={Chapter 3}
  ]

\Summary{Iohn Baptiſt by his Eremites life, by his preaching and
  Baptiſme, calleth al vnto pennance, to prepare them to Chriſt. 10.~He
  preacheth to the Phariſees and Saducees, threatning to them (vnles
  they truly doe pennance) reprobation here, and damnation hereafter;
  and for ſaluation ſendeth them to Chriſt and his Baptiſme. Which being
far more excellent then Iohns, yet Chriſt himſelf among thoſe penitents
vouchſafeth to come vnto Iohns Baptiſme. Where he hath teſtimonie from
Heauen alſo.}

And
\MNote{The ſecond part of this Ghoſpel, Of the Preparation that was made
to the manifeſtation of Chriſt.}
in thoſe dayes
\CNote{\XRef{Mr.~1,~4.}
\XRef{Luc.~3.}}
cometh Iohn the Baptiſt preaching in the
\LNote{Deſert}{Of 
\MNote{Eremites.}
this word \Emph{deſert} (in Greeke \Emph{eremus})
  commeth the name \Emph{Eremitages} & \Emph{Eremites}, that liue a
  religious and auſtere life in deſerts and ſolitarie places, by the
  example of 
\Fix{S.~Sohn}{S.~Iohn}{obvious typo, fixed in other}
Baptiſt; whom the holy Doctours therfore cal the
  Prince and as it were the authour of ſuch profeſſion. 
\Cite{S.~Chrys. ho.~1. in Marcum, & ho. de Io Baptiſta.}
\Cite{Hiero. ad Euſtach. de cuſtod. virg. Iſid. li.~2 c.~15. de diui. off.}
\Cite{Bernardus de excel. Io. Baptiſte.}
Wherewith the Proteſtants are ſo offended that they ſay, S.~Chryſoſtom
ſpake raſhly, and vntruely. 
And no maruel, for whereas the Euangeliſt
himſelf in this place maketh him a perfect paterne of pennance, and
Eremitical life, for deſert or wildernes, for his rough and rude
apparel, for abſtayning from al delicate meates (according to our
Sauiours teſtimonie alſo of him
\XRef{Mt.~11,~8.}
\XRef{Luc.~7,~33})
they are not aſhamed to peruert al with this ſtrange commentarie, that
it was a deſert
\CNote{\Cite{Magdeb. Cent.~5. c.~6. Pag.~711.}
\Cite{Cent.~1. li.~1. c.~10.}}
ful of townes and villages, his garment was
\CNote{\Cite{Cythræus in 3.~c. Mat.}}
chamlet, his meate
\CNote{\Cite{Buſerus ibid.}}
ſuch as the countrey gaue, and the people there vſed: to make him
thereby but a common man like to the reſt, in his manner of life: cleane
againſt Scriptures, Fathers, & reaſon.}
deſert of Iewrie, \V & ſaying:
\LNote{Due pennance}{So
\MNote{Pennance.}
is the Latin, word for word, ſo readeth al antiquitie, namely 
\Cite{S.~Cyprian ep.~52.} often, and 
\Cite{S.~Auguſtin li.~13. Confes. c.~12.}
and it is a very vſual
ſpeach in the New Teſtament, ſpecially in the preaching of S.~Iohn
Baptiſt,
\CNote{\XRef{Mt.~4,~17.}
\XRef{Lu.~13,~3.~5.}
\XRef{Lu.~24,~47.}}
Chriſt himſelf, and
\CNote{\XRef{Act.~2,38.}
\XRef{26,~20.}}
the Apoſtles; to ſignifie perfect
repentance, which hath not only confeſſion and amendment, but
contrition, or ſorow for the offence, and paineful ſatisfaction: ſuch as
S.~Cyprian ſpeaketh of in al the foreſaid epiſtle. But the Aduerſaries
of purpoſe (as
\TNote{Annot. in hunc locum.}
namely Beza proteſteth) miſlike that interpretation,
becauſe it fauoureth Satisfaction for ſinne, which they cannot
abide. Where if they pretend the
\TNote{\G{Μεταvοεῖν}. \G{Μετάνοια}.}
Greeke word, we ſend them to theſe places
\XRef{Mat.~12,~21.}
\XRef{Luc.~10,~23.}
\XRef{2.~Cor.~7,~9.}
Where it muſt needes ſignifie
ſorowful, payneful, and ſatisfactorie repentance. We tel them alſo that
\CNote{\Cite{Serm. in fam. & ſiccitat.}}
S.~Baſil a Greeke Doctour calleth the Niniuites repentance with faſting,
and hairecloth, and aſhes, by the ſame Greeke word \G{μετάνοιαν}. And more we wil
tel them in other places.}
Doe pennance: for the Kingdom of Heauen is at hand. \V For this is he
that was ſpoken of by Eſay the Prophet, ſaying:
\CNote{\XRef{Es.~40,~3.}}
\Emph{A voyce of one
  crying in the deſert, prepare ye the way of our Lord, make ſtraight
  his pathes.} \V And the ſaid Iohn had his garment of Camels heare, and
a girdle of a skinne about his loynes: and his meate was locuſtes &
wilde honie.

\V Then, went forth to him Hieruſalem and al Iewrie, and al the countrey
about Iordan: \V & were baptized of him in Iordan,
\LNote{Confeſsing their ſinnes}{Iohn did prepare the way to Chriſt and
  his Sacraments, not only by his Baptiſme, but by inducing the people
  to Cõfeſſion of their ſinnes. Which is not to acknowledge themſelues
  in general to be ſinners, but alſo to vtter euery man his ſinnes.}
confeſſing their ſinnes. \V And ſeeing many of the Phariſees and
Sadducees coming to his Baptiſme, he ſaid to them.

Ye vipers brood, who hath ſhewed you to flee from the wrath to come? \V
Yeald therfore
\LNote{Fruit worthie}{He preacheth Satisfaction by doing worthy fruits
or workes of penance, which are (as S.~Hierom.\ ſaith in
\XRef{2.~Ioel})
faſting, praying, almes, and the like.}
fruit worthie of pennance. \V And delight not to ſay within your ſelues,
we haue Abraham to our father. For I tel you that God is able of theſe
ſtones to raiſe vp children to Abraham. \V For now 
\LNote{The axe}{Here Preachers are taught to dehort from doing euil for
  feare of Hel, and to exhort to do good in hope of Heauen: which kind
  of preaching our Aduer.\ do condemne.}
the axe is put to the roote of the trees. Euery tree therfore that doth
\SNote{It is not only damnable, to doe il, but alſo, not to do good.
\Cite{Aug. Ser.~6. de temp.}}
not yeald good fruit, ſhal be cut downe, & caſt into the fire. \V
\CNote{\XRef{Mr.~1,~8.}
\XRef{Lu.~3,~16.}
\XRef{Io.~1,~26.}
\XRef{Act.~11,~16.}
\XRef{19,~4.}}
I indeed baptize you
\LNote{In water}{Iohns
\MNote{Iohns baptiſme and Chriſts.}
Baptiſme did not remit ſinnes, nor was comparable to Chriſts Baptiſme,
as here it is playne, & in manie other places. 
\Cite{Hiero. adu. Lucifer.}
\Cite{Aug. de Bap. cont. Donat. li.~5. c.~9. 10. 11.}
Yet it is an article of our Adu.\ that
\Fix{th' one}{the one}{possible typo, same in other}
is no better then the other which they ſay not to extol Iohns, but to
derogate from Chriſts baptiſme, ſo farre, that they make it of no more
valure or efficacie for remiſſion of ſinnes, & grace and iuſtification,
then was Iohns: thereby to maintaine their manifold hereſies, that
Baptiſme taketh not away ſinnes; that a mã is no cleaner nor iuſter by
the Sacramẽt of Baptiſme then before; that it is not neceſſarie for
children vnto ſaluation, but it is enough to be borne of Chriſtian
parents; & ſuch like erroneous poſitions wel knowen among the
Caluiniſts.}
in water vnto pennance, but he that ſhal come after me, is ſtronger then
I, whoſe ſhoes I am not worthy to beare, he ſhal baptize you in the Holy
Ghoſt
%%% 2296
%%% o-2108
& fire. \V Whoſe fanne is in his hand, and he ſhal cleane purge his
\LNote{Floore}{This floore is his Church militant here in earth,
  wherein are both good and bad (here ſignified by corne and chaffe) til
the ſeparation be made in the day of iudgement: contrarie to the
doctrine of the Heretikes, that hold, the Church to conſiſt only of the
good.}
floore: and he wil gather his wheate into the barne, but the chaffe he
wil burne with vnquenchable fire.

\V Then cometh \Sc{Iesvs} from Galilee to Iordan, vnto Iohn, to be
baptized of him. \V But Iohn ſtayed him, ſaying: I ought to be baptized
of thee, and comeſt thou to me? \V And \Sc{Iesvs} anſwering, ſayd to
him: Suffer me for this time. For ſo it becommeth vs to fulfil al
iuſtice. Then he ſuffered him. \V And \Sc{Iesvs} being baptized,
forthwith came out of the water: and loe the Heauens were
\LNote{Opened}{To ſignifie that Heauen was ſhut in the old Law, til
  Chriſt by his Paſſion opened it, and ſo by his Aſcenſion was the firſt
that entered into it; contrarie to the doctrine of the Heretikes. See
\XRef{Hebr.~9,~8. and 11,~40.}}
opened to him: & he ſaw the Spirit of God deſcending as a doue, & coming vpon
him. \V And behold a voice from Heauen ſaying: This is my beloued Sonne,
in whom I am wel pleaſed.

\stopChapter


\stopcomponent


%%% Local Variables:
%%% mode: TeX
%%% eval: (long-s-mode)
%%% eval: (set-input-method "TeX")
%%% fill-column: 72
%%% eval: (auto-fill-mode)
%%% coding: utf-8-unix
%%% End:
