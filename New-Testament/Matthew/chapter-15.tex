%%%%%%%%%%%%%%%%%%%%%%%%%%%%%%%%%%%%%%%%%%%%%%%%%%%%%%%%%%%%%%%%%
%%%%
%%%% The (original) Douay Rheims Bible 
%%%%
%%%% New Testament
%%%% Matthew
%%%% Chapter 15
%%%%
%%%%%%%%%%%%%%%%%%%%%%%%%%%%%%%%%%%%%%%%%%%%%%%%%%%%%%%%%%%%%%%%%




\startcomponent chapter-15


\project douay-rheims


%%% 2325
%%% o-2141
\startChapter[
  title={Chapter 15}
  ]

\Summary{The Phariſees of Hieruſalem coming ſo farre to carp him, he
  chargeth with a tradition contrarie to Gods commandement. 10.~And to
  the People he yealdeth the reaſon of that which they reproued: 15.~&
  againe to his Diſciples, shewing the ground of the Phariſaical washing
(to wit, that meates otherwiſe defile the ſoule) to be falſe. 21.~then
  he goeth aſide to hide himſelf among the Gentils, where, in a woman he
findeth
%%% 2326
ſuch faith, that he is faine, leſt the Gentils should before the time
extort the whole bread, as she had a crumme, to returne to the
Iewes. 34.~where (al contrarie to thoſe Phariſees) the common People
ſeeke wonderfully vnto him: and he after he hath cured their diſeaſed,
feedeth 4000.\ of them with ſeauen loaues.}

Then
\CNote{\XRef{Mr.~7,~1.}}
came to him from Hieruſalem Scribes and Phariſees, ſaying: \V Why
do thy Diſciples tranſgreſſe the tradition of the Ancients? For they waſh
not their hands when they eate bread. \V But he anſwering ſaid to them:
Why do you alſo tranſgreſſe the commandement of God for your tradition?
For God ſaid: \V
\CNote{\XRef{Exo.~20,~12.}}
\Emph{Honour father and mother.} And:
\CNote{\XRef{Leu.~20,~9.}}
\Emph{He that
  shal curſe father or mother, dying let him dye.} \V But you ſay:
whoſoeuer ſhal ſay to father or mother, the guift whatſoeuer proceedeth
from me, ſhal profit thee: \V And ſhal not honour his father or his
mother: & you haue made fruſtrate the commandement of God for your own
tradition. \V Hypocrits, wel hath Eſay Prophecied of you, ſaying: \V
\CNote{\XRef{Eſa.~29.~13.}}
\Emph{This People honoureth me}
\LNote{With their lips}{This 
\Fix{in}{is}{obvious typo, fixed in other}
to be vnderſtood properly of ſuch as haue euer God in their mouth, the
Word of our Lord, the Scriptures, the Ghoſpel, but in their hart and al
their life be in deed Godles. It may be applied alſo to ſuch as ſay
their prayers without attention or eleuation of mind to God, whether he
vnderſtand the prayers or no, that ſaith them. For many a poore
Chriſtian man that vnderſtandeth not the wordes he ſpeaketh, hath his
hart neerer Heauen, more feruor & deuotion, more edification to himſelf,
more profit in ſpirit (as the
\CNote{\XRef{1.~Cor.~14.}}
Apoſtle ſpeaketh) & leſſe diſtractions,
then not only al Heretikes which haue no true feeling of ſuch things,
but then many learned Catholikes. And therfore it is not to be
vnderſtood of praying in vnknown tongues, as Heretikes ſometime expound
it, farre wide from the circumſtance of the place and Chriſtes
intention, ſpeaking of the hypocritical Iewes.}
\Emph{with their lips: but their hart is farre from me.} \V \Emph{And
  in vaine do they worship me, teaching doctrines and}
\LNote{Commandements of men}{Such only are here called
  traditiõs, doctrines, or commandements of men, which be either
  repugnant to Gods lawes, as this of defrauding their parents vnder
  pretenſe of religion: or which at the leaſt be friuolous,
  vnprofitable, and impertinent to pietie or true worſhip, as that
  other ſort of ſo often waſhing hands, and veſſels, without regard of
  inward puritie of hart and mind. 
\MNote{The difference between the Iewiſh traditions here reprehended,
  and the Churches Apoſtolical traditions.}
Let no man therfore be abuſed with
  the Proteſtants peruerſe application of this place againſt the holy
  lawes, canons, and precepts of the Church, and our ſpiritual
  Gouernours, concerning faſtes, feſtiuities, and other rules of
  diſcipline, and due order in life, and in the ſeruice of God. For ſuch
are not repugnant but conſonant to Gods Word & al pietie, & our Lord is
truly honoured, worſhiped, and ſerued both by the making and alſo by the
obſeruing of them.
\CNote{\XRef{2.~Theſ.~2,~15.}
\XRef{1.~Cor.~11.}}
S.~Paul gaue commandement both by his epiſtles, and
by word of mouth, euen in ſuch matters wherin Chriſt had preſcribed
nothing at al, & he chargeth the Faithful to obſerue the ſame.
\CNote{\XRef{Act.~15.}}
The Apoſtles & Prieſts at Hieruſalem made lawes, and the Chriſtiãs were
bound to obey them.
\CNote{\Cite{Aug. ſer. de tep.~251.}
See
\XRef{1.~Cor.~16,~2.}}
The keeping of Sunday in ſteed of the Sabboth is the
tradition of the Apoſtles: and dare the Heretikes deny the due
obſeruation therof to be an acceptable worship of God?
\CNote{\Cite{Epiph. har.~75.}}
They preſcribed
the Feaſtes of Eaſter, and whitſontide, and other Solemnities of Chriſt,
and his Saints, which the Proteſtants them ſelues obſerue.
\CNote{\Cite{Jiero. ep.~54. ad Marcel. contra Mont.}}
They
appointed the Lent & Imber faſtes and other, as wel to chaſtiſe the
concupiſcence of man, as to ſerue and pleaſe God therby, as is plaine in
the faſting of
\CNote{\XRef{Lu.~2,~37.}
\XRef{Tob.~12.}
\XRef{Iud. c.~8.}
\XRef{Eſt.~4.}}
Anna, Tobie, Iudith, Eſther; who ſerued and pleaſed God
therby. Therfore neither theſe, nor other ſuch Apoſtolike Ordinances,
nor any precepts of the holy Church, or of our lawful Paſtours, are
implied in theſe Phariſaical traditions here reprehended; nor to be
counted or called the doctrines and commandements of men, becauſe they
are not made by mere humane power, but by Chriſtes warrant and
authoritie, and by ſuch as he hath placed to rule his Church, of whõ he
ſaith:
\CNote{\XRef{Lu.~10,~16.}}
\Emph{He that heareth you, heareth me: he that diſpiſeth you
  deſpiſeth me.} They are made by the Holy Ghoſt, ioyning with our
Paſtours in the regimẽt of the Faithful. They are made by our Mother the
Church, which whoſoeuer obeieth not,
\CNote{\XRef{Mat.~18,~17.}}
we are warned to take him as an
Heathẽ. But on the other ſide, al lawes, doctrines, ſeruices, and
iniunctions of Heretikes, how ſoeuer pretended to be conſonant to the
Scriptures, be commandements of men: becauſe both the things by them
preſcribed are impious, and the Authours haue neither ſending nor
commiſſion from God.}
\Emph{commandements of men.}

\V And hauing called togeather the multitudes vnto him, he ſaid to them:
Heare ye and vnderſtand. \V 
\LNote{Not that which entereth}{The
\MNote{Difference of meates.}
Catholikes doe not abſtaine from certaine meates, for that they eſteeme
any meate vncleane, either by creation, or by Iudaical obſeruation: they
abſtaine, for chaſtiſement of their concupiſcences
\Cite{Aug. li. de mor. Ec. Cath. c.~33.}}
Not that which entreth into the mouth, defileth a man: but that which
proceedeth out of the mouth, that defileth a man. \V Then came his
Diſciples, and ſaid to him: Doſt thou know that the Phariſees, when they
heard this word, were ſcandalized? \V But he anſwering ſaid: Al planting
which my Heauenly Father hath not planted, ſhal be rooted vp. \V Let
them alone: blind they are, guides of the blind. And if the blind be
guide to the blind, both fal into the ditch. \V And Peter anſwering ſaid
to him: Expound vs this parable. \V But he ſaid: Are you alſo as yet
without vnderſtanding? \V Doe you not vnderſtand, that al that entreth
into the mouth, goeth into the belly, and is caſt forth into the priuy?
\V But the things that proceed out of the mouth, come forth from the
hart, and thoſe things
\LNote{Defile a man}{It 
\MNote{Catholike abſtinence.}
is ſinne only, which properly defileth man, and
meates of them ſelues or of their owne nature doe not defile, but ſo
farre as by accidẽt they make a man to ſinne, as the diſobedience of
Gods commandement, or of our Superiours, who forbid ſome meates for
certaine times, and cauſes, is a ſinne. As
\CNote{\XRef{Gen.~3.}}
the apple which our firſt
parẽts did eate of, though of itſelf it did not defile them, yet being
eaten againſt the precept, it did defile. So neither flesh nor fish of
itſelf doth defile, but the breach of the Churches precept defileth.}
defile a man. \V For from the hart come forth euil cogitations, murders,
aduoutries, fornications, thefts, falſe teſtimonies, blaſphemies. \V
Theſe are the things that
%%% o-2142
defile a man. But to eate with vnwaſhen hands, doth not defile a man.

\V And \Sc{Iesvs} went forth from thence and retired into the quarters
of Tyre and Sidon. \V And behold
\CNote{\XRef{Mr.~7,~25.}}
a woman of Chanaan came forth out of
thoſe coaſts, & crying out, ſaid to him: Haue mercie vpon me, O Lord the
Sonne of Dauid: my daughter is ſore vexed of a Diuel. \V Who anſwered
her not a word. And his Diſciples came and beſought him ſaying: Dimiſſe
her, becauſe ſhe crieth out after vs. \V And he anſwering ſaid: I was
not ſent but to the ſheep that are loſt of the houſe of Iſrael. \V But
ſhe came and adored him, ſaying: Lord, help me. \V Who anſwering, ſaid:
It is not good to take the bread of the Children, and to caſt it to the
dogs. \V But ſhe ſaid: Yea Lord; for the whelps alſo eate of the crummes
that fal from the table of their maiſters. \V Then \Sc{Iesvs} 
%%% 2327
anſwering ſaid to her: O woman,
\SNote{It were a ſtrãge caſe that Chriſt ſhould commend in this woman a
  ſole faith without good workes, that is to ſay, a dead faith ſuch as
  could not worke by loue, and which S.~Iames doubted not to cal the
  faith not of Chriſtians but of Diuels.
\Cite{Aug. de Fid. & Op. c.~16.}}
great is thy faith: be it done to thee as thou wilt: And her daughter
was made hole from that houre.

\V And when \Sc{Iesvs} was paſſed from thence, he came beſide the ſea of
Galilee: & aſcending into the mountaine, ſate there. \V And there came
to him great multitudes, hauing with thẽ dumme perſons, blind, lame,
feeble, and many others: and they caſt them downe at his feete, and he
cured them: \V ſo that the multitudes marueled ſeeing the dumme ſpeake,
the lame walke, the blind ſee: and they magnified the God of Iſrael. \V
And
\CNote{\XRef{Mar.~8,~1.}}
\Sc{Iesvs} called togeather his Diſciples, and ſaid: I pitie the
multitude becauſe three dayes now they continue with me, & haue not what
to eate: and dimiſſe them faſting I wil not, leſt they faint in the
way. \V And the Diſciples ſay vnto him: whence then may we get ſo many
loaues in the deſert as to fil ſo great a multitude? \V And \Sc{Iesvs}
ſaid to them: How many loaues haue you? But they ſaid: Seauen, & a few
litle fiſhes. \V And he commanded the multitude to ſit downe vpon the
ground. \V And taking the Seauen loaues & the fiſhes, and giuing
thankes, he brake, & gaue to his Diſciples, and
\SNote{Here we ſee againe that the People muſt not be their owne
  caruers, nor receaue the Sacraments or other ſpiritual ſuſtenance
  immediatly of Chriſt, or at their owne hand, but of their ſpiritual
  gouerners.}
the Diſciples gaue to the People. \V And they did al eat, and had their
fill. And that which was left of the fragments they tooke vp, ſeauen
baſkets ful. \V And there were that did eate, foure thouſand men, beſide
children & women.

\V And hauing dimiſſed the multitude, he went vp into a boate, and came
into the coaſtes of Magedan.

\stopChapter


\stopcomponent


%%% Local Variables:
%%% mode: TeX
%%% eval: (long-s-mode)
%%% eval: (set-input-method "TeX")
%%% fill-column: 72
%%% eval: (auto-fill-mode)
%%% coding: utf-8-unix
%%% End:
