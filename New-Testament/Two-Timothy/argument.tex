%%%%%%%%%%%%%%%%%%%%%%%%%%%%%%%%%%%%%%%%%%%%%%%%%%%%%%%%%%%%%%%%%
%%%%
%%%% The (original) Douay Rheims Bible 
%%%%
%%%% New Testament
%%%% Epistles
%%%% Two Timothy
%%%% Argument
%%%%
%%%%%%%%%%%%%%%%%%%%%%%%%%%%%%%%%%%%%%%%%%%%%%%%%%%%%%%%%%%%%%%%%




\startcomponent argument


\project douay-rheims


%%% 2825
%%% o-2685
\startArgument[
  title={\Sc{The Argvment of the Second Epistle of S.~Pavl to Timothee.}},
  marking={Argument of Two Timothee}
  ]

The cheefe ſcope of this ſecond to Timothee, is, to open vnto him that
his martyrdom is at hand. Which yet he doth not plainly before the end:
preparing firſt his mind with much circumſtance, becauſe he knew it would
greiue him ſore, and alſo might be a tentation vnto him. Therfore he
talketh of the cauſe of his trouble, and of the reward: that the one is
honourable, and the other moſt glorious: and exhorteth him to be
conſtant in the faith, to be ready alwaies to ſuffer for it, to fulfil
his miniſterie to the end, as himſelf now had done his.

Whereby it is certaine, that it was written at Rome, in his laſt
apprehenſion and empriſonment there: as he ſignifieth by theſe words
\XRef{Chap.~1:}
\Emph{Oneſiphorus was not aſhamed of my chaine, but when he was come to
Rome, carefully ſought me, &c.} And of his martyrdom, thus: \Emph{For I
am now ready to be offered, and the time of my reſolution} (or death)
\Emph{is at hand.}
\XRef{Cap.~4.}


\stopArgument


\stopcomponent


%%% Local Variables:
%%% mode: TeX
%%% eval: (long-s-mode)
%%% eval: (set-input-method "TeX")
%%% fill-column: 72
%%% eval: (auto-fill-mode)
%%% coding: utf-8-unix
%%% End:
