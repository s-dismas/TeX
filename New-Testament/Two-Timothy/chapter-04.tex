%%%%%%%%%%%%%%%%%%%%%%%%%%%%%%%%%%%%%%%%%%%%%%%%%%%%%%%%%%%%%%%%%
%%%%
%%%% The (original) Douay Rheims Bible 
%%%%
%%%% New Testament
%%%% Epistles
%%%% Two Timothy
%%%% Chapter 04
%%%%
%%%%%%%%%%%%%%%%%%%%%%%%%%%%%%%%%%%%%%%%%%%%%%%%%%%%%%%%%%%%%%%%%

%%% Latin checked by KK.




\startcomponent chapter-04


\project douay-rheims


%%% 2832
%%% o-2692
\startChapter[
  title={Chapter 4}
  ]

\Summary{He requireth him to be earneſt while he may, becauſe the time
  wil come when they wil not abide Catholike preaching, 5.~and to fulfil
  his courſe, as himſelf now hath done: 9.~and to come vnto him with
  ſpeed, becauſe the reſt of his traine are diſperſed, and he draweth
  now to heauen.}

I teſtifie before God and \Sc{Iesvs} Chriſt who ſhal iudge the liuing
and the dead, and by his aduent, and his Kingdom: \V Preach the
word. Vrge in ſeaſon, out of ſeaſon, reproue, beſeech, rebuke in al
patience and doctrine. \V For
\LNote{There shal be a time.}{If
\MNote{The Apoſtle prophecied of our new delicate Preachers.}
euer this time come (as needs it muſt that the Apoſtle fore-ſaw and
fore-told) now it is vndoubtedly. For the properties fal ſo iuſt in
euery point vpon our new Maiſters and their Diſciples, that they may
ſeem to be pourtered out, rather then prophecied of. Neuer were there
ſuch delicate Doctours that could ſo pleaſantly claw and ſo ſweetly
rubbe the itching eares of their hearers, as theſe, which haue a
doctrine framed for euery mans phanſie, luſt, liking, and deſire: the
people not ſo faſt crying,
\CNote{\XRef{Eſa.~30. v.~10.}}
\Emph{ſpeake placentia, things that pleaſe}:
but the Maiſters as faſt warranting them to doe \Emph{placentia}.}
there ſhal be a time when they wil not beare ſound doctrine: but
according to their owne deſires they wil heape to themſelues Maiſters,
hauing itching eares, \V and from the truth certes they wil auert their
hearing, and to fables they wil be conuerted. \V But be thou vigilant,
labour in al things, doe the worke of an Euangeliſt, fulfil thy
miniſterie. Be ſober. \V For I am euen now
\SNote{The martyrdom of Saints is ſo acceptable to God, that it is
counted as it were a Sacrifice in his ſight, and therfore hath many
effects both in the partie that ſuffereth it, and in others that are
partakers of the merit as of a Sacrifice: which name it hath by a
Metaphore.}
to be ſacrificed: and the time of my reſolution is at hand. \V I haue
fought a good fight, I haue conſummate my courſe, I haue kept the
faith. \V Concerning the reſt, there is laid vp for me
\LNote{A crowne of iuſtice.}{This
\MNote{Workes meritorious.}
place conuinceth for the Catholikes, that al good workes done by God's
grace after the firſt iuſtification be truly and properly meritorious,
and fully worthy of euerlaſting life: and that thereupon heauen is the
due and iuſt ſtipend, crowne, or recompenſe which God by his iuſtice
oweth to the perſons ſo working by his grace.
\MNote{How heauen is due both of iuſtice and mercie.}
For he rendreth or repaieth heauen as a iuſt iudge, & not only as a
merciful giuer. And the crowne which he paieth, is not only of mercie or
fauour or grace, but alſo of iuſtice. It is his merciful fauour and
grace, that we worke wel and merit heauen: it is his iuſtice, for thoſe
merits to giue vs a crowne correſpondent in heauen. S.~Auguſtin vpon
theſe words of the Apoſtle, expreſſeth both briefely thus, \Emph{How
should he repay as a iuſt iudge, vnles he had firſt giuen as a merciful
father?}
\Cite{Li. de great. & lib. arbit. c.~6.}

And when you heare or read any thing in the Scriptures, that may ſeeme
to derogate from mans workes in this caſe, it is alwaies meant of workes
conſidered in their owne nature and valure, not implying the grace of
Chriſt, by which grace it commeth, and not of the worke in it-ſelf that
we haue a right to heauen and deſerue it worthily; which the Apoſtle in
the
\XRef{6.~to the Hebrewes}
more then inſinuateth, ſaying theſe words,
\MNote{It is not of vs, but of God's grace, that workes be meritorious.}
\Emph{God is not vniuſt, to
forget your worke and loue which you haue shewed in his name, &c.} As
though he would ſay, that he were vniuſt if he did forget to recompenſe
their workes.
\CNote{\XRef{Mat.~20.}}
The parable alſo of the men ſent into the vineyard, proueth that heauen
is our owne right, bargained for and wrought for, and accordingly paid
vnto vs as our hire at the day of iudgement for that is \L{merces} &
\G{μισθός} whereby the Scripture ſo often calleth it. It is the goale,
the marke, the price, the hire of al ſtriuing, running, labouring, due
both by promiſe & by couenant & right debt. See a notable place in
S.~Auguſtin
\Cite{in Pſal.~83. in fine:}
and
\Cite{100.~in initio.}
&
\Cite{ho.~14. c.~2. li.~50. hom.}
S.~Cyprian alſo, and namely the later end of his booke
\Cite{de opere & eleomoſyna:}
\MNote{To ſuch good workes heauen is due: to ſay the contrarie, is to
derogate from Gods grace.}
and thou ſhalt eaſily contemne the contrarie falſhood, which doth not ſo
much derogate from mans workes, as from Gods grace which is the cauſe
and ground of al worthines in mans merits. S.~Auguſtines words be theſe,
\CNote{\Cite{In Pſ.~100.}}
\Emph{Marke that he to whom our Lord gaue grace, hath our Lord alſo his
debter. He found him a giuer, in the time of mercie: he hath him his
debter in the time of iudgement.} See the place and the reſt here coted,
where he examineth and explicated the matter at large.}
a crowne of iuſtice, which our Lord wil render to me in that day, a iuſt
%%% o-2693
iudge: and not only to me, but to them alſo that loue his comming.

\V Make haſt to come to me quickly. \V For Demas hath left me, louing
this world, and is gone to Theſſalonica: Creſcens into Galatia, Titus
into Dalmatia. \V
\CNote{\XRef{Col.~4,~14.}}
Luke only is with me. Take Marke, and bring him with thee: for he is
profitable to me for the miniſterie. \V But Tychicus I haue ſent to
Epheſus. \V The cloke that I left at Troas with Carpus, comming bring
with thee, and the books, eſpecially the parchment. \V Alexander the
Copperſmith hath ſhewed me much euil: our Lord wil reward him according
to his workes: \V whom doe thou alſo auoid, for he hath greatly reſiſted
our words. \V In my firſt anſwer no man was with me, but al did forſake
me: be it not imputed to them. \V But our Lord ſtood to me, and
ſtrengthned me, that by me the preaching may be accompliſhed, and al
Gentils may heare: and I was deliuered from the mouth of the lion. \V
Our Lord
\Var{hath deliuered}{wil deliuer}
me from al euil worke: and wil ſaue me vnto his heauenly Kingdom. To
whom be glorie for euer and euer. Amen.

\V Salute Priſca and Aquila, and
\CNote{\XRef{2.~Timo.~3,~16.}}
the houſe of Oneſiphorus. \V Eraſtus remained at Corinth. And Trophimus
I left ſicke at Miletum. \V Make haſt to come before winter. Eubulus and
Pudens and
\SNote{This Liuns was Coadiutour with and vnder S.~Peter, and ſo counted
ſecond in the number of Popes.}
Linus and Claudia, and al the Brethren, ſalute thee. \V Our
Lord \Sc{Iesvs} Chriſt be with thy ſpirit. Grace be with you. Amen.


\stopChapter


\stopcomponent


%%% Local Variables:
%%% mode: TeX
%%% eval: (long-s-mode)
%%% eval: (set-input-method "TeX")
%%% fill-column: 72
%%% eval: (auto-fill-mode)
%%% coding: utf-8-unix
%%% End:

