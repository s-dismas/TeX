%%%%%%%%%%%%%%%%%%%%%%%%%%%%%%%%%%%%%%%%%%%%%%%%%%%%%%%%%%%%%%%%%
%%%%
%%%% The (original) Douay Rheims Bible 
%%%%
%%%% New Testament
%%%% Epistles
%%%% Two Timothy
%%%% Chapter 03
%%%%
%%%%%%%%%%%%%%%%%%%%%%%%%%%%%%%%%%%%%%%%%%%%%%%%%%%%%%%%%%%%%%%%%




\startcomponent chapter-03


\project douay-rheims


%%% 2830
%%% o-2690
\startChapter[
  title={Chapter 3}
  ]

\Summary{He prophecieth of Heretikes to come, 6.~and noteth certaine
  then alſo for ſuch, bidding him to auoid them, 10.~and (whatſoeuer
  perſecution befal for it) to continue conſtant in the Catholike
  doctrine, both becauſe of his Maiſter (S.~Paul himſelf) 15.~and alſo
  becauſe of his owne knowledge in the Scriptures.}

And this know thou, that
\CNote{\XRef{1.~Tim.~4,~1.}}
in the laſt daies shal approch perilous times. \V And
\LNote{Men shal be.}{Al theſe words S.~Cyprian expoundeth of ſuch as by
pride and diſobedience reſiſt Gods Prieſts.
%%% !!! Should this be \Emph{...}?
Let no faithful man, ſaith
he, that keepeth in mind our Lordes and the Apoſtles admonition, maruel
if he ſee in the later times ſome proud and ſtubburne fellowes and the
enemies of God's Prieſts, goe out of the Church or impugne the ſame:
when both our Lord and the Apoſtle foretold vs that ſuch ſhould be.
\Cite{Cypr. ep.~55. nu.~3.}}
men shal be louers of themſelues, couetous, hautie, proud, blaſphemous,
not obediẽt to their
\Fix{parens,}{parents,}{obvious typo, fixed in other}
vnkind, wicked, \V without affection, without peace, accuſers,
incontinent, vnmerciful, without benignitie, \V traitours, ſtubburne,
puffed vp, and louers of voluptuouſnes more then of God: \V hauing an
appearance indeed of pietie, but denying the vertue thereof. And theſe
auoid. \V For of theſe be they that craftily enter into houſes; & lead
captiue ſeely
\LNote{Women loden.}{Women
\MNote{Women eaſily ſeduced by hereſie.}
loden with ſinnes, are for ſuch their deſeruings, and through the
frailtie of their ſexe, more ſubiect to the heretikes deceits, then men:
the enemie attempting (as he did in the fal of our firſt parents) by
them to ouerthrow men. See
\Cite{S.~Hierom vpon the 3.~chapter of Ieremie,}
where he addeth that euery hereſie is firſt broched \L{propter gulam &
ventrem}, for gluttonie and belly-cheere.}
women loden with
%%% o-2691
ſinnes, which are led with diuers deſires: \V alwaies
learning, and neuer attaining to the knowledge of the truth. \V But as
\SNote{That thoſe Magicians which reſiſted Moyſes, were thus called, it
is not written in al the old Teſtament: therfore it came to the Apoſtles
knowledge by tradition, as the Church now hath the names of the
3.~Kings, of the penitent theefe, of the ſouldiar that pearced Chriſts
ſide on the Croſſe, and of
\Fix{they}{the}{obvious typo, fixed in other}
like.}
Iannes and Mambres
\CNote{\XRef{Exo.~7.}}
reſiſted Moyſes, ſo theſe alſo reſiſt the truth, men corrupted in mind,
reprobate concerning the faith. \V But they shal proſper no further: for
their
\LNote{Folly manifeſt.}{Al
\MNote{The folly of Heretikes in time appeareth.}
heretikes in the beginning ſeeme to haue ſome ſhew of truth, God for
iuſt puniſhment of mens ſinnes permitting them for ſome while in ſome
perſons and places to preuaile: but in ſhort time God detecteth them,
and openeth the eyes of men to ſee their deceits: in ſo much that after
the firſt brunt they be mainteined by force only, al wiſe men in a
manner ſeeing their falſhood, though for troubling the ſtate of ſuch
common-weales where vnluckily they haue been receiued, they can not be
ſo ſodenly extirped.}
folly shal be manifeſt to al, as theirs alſo was.

\V But thou haſt attained to my doctrine, inſtitution, purpoſe, faith,
longanimitie, loue, patience, \V perſecutions, paſſions: what manner of
things were done to me at Antioche, at Iconium, at Lyſtra: what manner
of perſecutions I ſuſtained. And out of al our Lord deliuered me. \V And
\LNote{Al that wil liue.}{Al
\MNote{Perſecution.}
holy men ſuffer one kind of perſecution or other, being greeued &
moleſted by the wicked, one way or another: but not al that ſuffer
perſecution be holy, as al malefactours. The Church and Catholike
Princes perſecute heretikes, and be perſecuted of them againe, as
S.~Auguſtin often declareth. See
\Cite{ep.~48.}}
al that wil liue godly in Chriſt \Sc{Iesvs}, shal ſuffer perſecutiõ. \V
But euil men & ſeducers shal
%%% LNote not marked in either
\LNote{Proſper.}{Though hereſies and the Authours of them be after a
while diſcouered and by litle and litle forſaken generally of the
honeſt, diſcret, and men careful of their owne ſaluation; yet their
Authours and other great ſinners proceed from one errour and hereſie to
another, and finally to plaine Atheiſme and al diueliſh diſorder.}
proſper to the worſe: erring, and driuing
into errour. \V But thou,
\SNote{In al danger and diuerſitie of falſe Sects, S.~Paules admonition
is, euer to abide in that was firſt taught and deliuered, neuer to giue
ouer our old faith for a new fanſie. This is it which before
\Fix{be}{he}{obvious typo, fixed in other}
calleth \L{depoſitum}.
\XRef{1.~Tim.~6.}
and
\XRef{2.~Tim.~1.}}
continue in thoſe things which thou haſt learned, & are committed to
thee: knowing of whom thou haſt learned; \V & becauſe from thine
infancie thou haſt knowen
%%% 2831
the holy Scriptures, which can inſtruct thee to ſaluation, by the faith
that is in Chriſt \Sc{Iesvs}.

\V
\CNote{\XRef{1.~Pet.~1,~21.}}
\LNote{Al Scripture.}{Beſides
\MNote{The great profit of reading the Scriptures.}
the Apoſtles teaching and tradition, the reading of holy Scriptures is a
great defenſe and help of the faithful, and ſpecially of a Biſhop, not
only to auoid and condemne al hereſies, but to the guiding of a man in
al iuſtice, good life, and workes. Which commendation is not here giuen
to the books of the new Teſtament only (whereof he here ſpeaketh not,
as being yet for a great part not written) but to the Scripture of the
old Teſtament alſo, yea and to euery booke of it. For there is not one
of them, nor any part of them, but it is profitable to the end
aforeſaid, if it be read and vnderſtood according to the ſame Spirit
wherewith it was written.

The
\MNote{The Heretikes fooliſh argumẽt: Al Scripture is profitable, ergo
only Scripture is neceſſarie & ſufficient.}
Heretikes vpon this commendation of holy Scriptures, pretend (very
ſimply in good ſooth) that therfore nothing is neceſſarie to iuſtice and
ſaluation but Scriptures. As though euery thing that is profitable or
neceſſarie to any effect, excluded al other help, and were only enough
to attaine the ſame. By which reaſon a man might as wel proue that the
old Teſtament were enough, and ſo exclude the new: or any one peece of
al the old, and thereby exclude the reſt. For he affirmeth euery
Scripture to haue the foreſaid vtilities. And they might ſee in the very
next line before, that he requireth his conſtant perſeuerance in the
doctrine which he had taught him ouer and aboue that he had learned out
of the Scriptures of the old Teſtament, which he had read from his
infancie, but could not thereby learne al the myſteries of Chriſtian
religion therein. Neither doth the Apoſtle affirme here that he had his
knowledge of Scriptures, by reading only, without the help of Maiſters
and Teachers, as the Aduerſaries hereupon (to commit the holy Scriptures
to euery mans preſumption) doe gather: but affirmeth only that Timothee
knew the Scriptures and therfore had ſtudied them by hearing good
Readers and Teachers, as S.~Paul himſelf did of Gamaliel and the like,
and as al Chriſtian ſtudents doe, that be trained vp from their youth in
Catholike vniuerſities in the ſtudie of Diuinitie.}
Al Scripture inſpired of God, is profitable to teach, to argue, to
correct, to inſtruct in iuſtice: that the man of God may be perfect,
inſtructed to euery good worke.


\stopChapter


\stopcomponent


%%% Local Variables:
%%% mode: TeX
%%% eval: (long-s-mode)
%%% eval: (set-input-method "TeX")
%%% fill-column: 72
%%% eval: (auto-fill-mode)
%%% coding: utf-8-unix
%%% End:

