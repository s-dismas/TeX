%%%%%%%%%%%%%%%%%%%%%%%%%%%%%%%%%%%%%%%%%%%%%%%%%%%%%%%%%%%%%%%%%
%%%%
%%%% The (original) Douay Rheims Bible 
%%%%
%%%% New Testament
%%%% Ephesians
%%%% Argument
%%%%
%%%%%%%%%%%%%%%%%%%%%%%%%%%%%%%%%%%%%%%%%%%%%%%%%%%%%%%%%%%%%%%%%




\startcomponent argument


\project douay-rheims


%%% 2755
%%% o-2612
\startArgument[
  title={\Sc{The Argvment of the Epistle of S.~Pavl to the Ephesians.}},
  marking={Argument of Ephesians}
  ]

Of S.~Paules firſt comming to Epheſus, and short abode there, we read
\XRef{Act.~18.}
And immediately
\XRef{Act.~19.}
of his returning thither according to his promiſe, what time he abode
there \Emph{three moneths, ſpeaking to the Iewes in the Synagogue.}
\XRef{Act.~19. v.~8.}
and afterward apart from them (becauſe they were obſtinate)
\Emph{two yeares} in a certaine ſchoole, \Emph{ſo that al that dwelt in
Aſia, heard the word of our Lord, Iewes and Gentils}.
\XRef{Act.~19. v.~10.}
The whole time himſelf calleth \Emph{three yeares}, in his exhortation
at Miletum to the Clergie of Epheſus.
\XRef{Act.~20. v.~31.}

After al this he writeth this Epiſtle vnto them from Rome (as it is
ſaid) being then
\CNote{\XRef{Eph.~3. v.~1.}
&
\XRef{4. v.~1.}
%%% !!! Where does this go?
\XRef{Eph.~6. v.~20.}}
\Emph{priſoner} and \Emph{in chaines}: and that as it ſeemeth, not the
firſt time of his being in bonds there, wherof we read
\XRef{Act.~28.}
but the ſecond time, wherof we read in the Eccleſiaſtical Stories
afterward: becauſe he ſaith in
\XRef{this Epiſtle c.~6. v.~21.}
\Emph{Tychicus wil certifie you of al things, whom I haue ſent to you.}
Of whom againe in the
\XRef{2.~to Tim. c.~4. v.~12.}
he ſaith: \Emph{Tychicus I haue ſent to Epheſus.} And the ſaid
\XRef{2.~Epiſtle to Timothee}
(no doubt) was written very litle before his death; for in it thus he
ſaith: \Emph{I am euen now to be ſacrificed, & the time of my reſolution
is at hand.}
\XRef{2.~Tim.~4,~6.}

In the three firſt chapters, he commendeth vnto them the grace of God,
in calling of the Gentils no leſſe then the Iewes, and making one
bleſſed Church of both. Wherin his intention is to moue them to
perſeuer (for otherwiſe they should be paſsing vngrateful) and ſpecially
not to be moued with his trouble, who was their Apoſtle knowing (belike)
that
\CNote{See
\XRef{Act.~20. v.~25,~32.}}
it would be a great tentation vnto them, if they should heare ſoone
after, that he were executed: therfore alſo arming them in the end of
the Epiſtle, as it were in complete harneſſe.

In the other three chapters he exhorteth them to good life, in al
points, and al ſtates, as it becommeth Chriſtians: and afore al other
things that they be moſt ſtudious to continue in the vnitie of the
Church, and obedience of the Paſtours therof, whom Chriſt hath giuen to
continue and to be our ſtay againſt al Heretikes, from his Aſcenſion,
euen to the ful building vp of his Church in the end of the world.


\stopArgument


\stopcomponent


%%% Local Variables:
%%% mode: TeX
%%% eval: (long-s-mode)
%%% eval: (set-input-method "TeX")
%%% fill-column: 72
%%% eval: (auto-fill-mode)
%%% coding: utf-8-unix
%%% End:
