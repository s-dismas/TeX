%%%%%%%%%%%%%%%%%%%%%%%%%%%%%%%%%%%%%%%%%%%%%%%%%%%%%%%%%%%%%%%%%
%%%%
%%%% The (original) Douay Rheims Bible 
%%%%
%%%% New Testament
%%%% Epistles
%%%% Ephesians
%%%% Chapter 04
%%%%
%%%%%%%%%%%%%%%%%%%%%%%%%%%%%%%%%%%%%%%%%%%%%%%%%%%%%%%%%%%%%%%%%

%%% Latin checked by KK.




\startcomponent chapter-04


\project douay-rheims


%%% 2760
%%% o-2618
\startChapter[
  title={Chapter 4}
  ]

\Summary{He exhorteth them to keep the vnitie of the Church moſt
  carefully with al humilitie, bringing them many motiues therunto: 7.~&
  anſwering that euen the diuerſitie it-ſelf of offices is not for
  diuiſion, as being the guift of Chriſt himſelf, but to build vp the
  Church, and to hold al in the vnitie therof againſt the ſuttle
  circumuentions of Heretikes: that vnder Chriſt the Head, in the Church
  being the body, euery member may proſper. 17.~Neither (as touching
  life) muſt we liue like the Heathen, but as it becommeth Chriſtiãs,
  laying off al our old corrupt manners, & increaſing daily in al
  goodnes.}

I therfore priſoner in our Lord, beſeech you, that you walke worthy of
the vocation in which you are called, \V with al humilitie and mildnes,
with patience, ſupporting one another in charitie, \V careful to keep
the vnitie of the ſpirit in the bond of peace. \V One body & one ſpirit:
as you are called in one hope of your vocation. \V One Lord,
\LNote{One faith.}{As
\MNote{Vnitie of the Cat. Church.}
rebellion is the bane of ciuil Common-wealths and Kingdoms, and peace,
and concord, the preſeruation of the ſame: ſo is Schiſme, diuiſion, and
diuerſitie of faith or fellowship in the ſeruice of God, the calamitie
of the Church: and peace, vnitie, vniformitie, the ſpecial bleſſing of God
therein: and in the Church aboue al Common-wealths, becauſe it is in al
points a Monarchie tending euery way to vnitie. But one God, but one
Chriſt, but one Church, but one hope, one faith, one baptiſme, one head,
one body. Wherof S.~Cyprian
\Cite{lib. de vnit. Eccleſ. nu.~3.}
ſaith thus: \Emph{One Church the Holy Ghoſt in the perſon of our Lord
deſigneth & ſaith, One is my doue. This vnitie of the Church he that
holdeth not, doth he thinke he holdeth the faith? He that withſtandeth
and reſiſteth the Church, he that forſaketh Peters chaire vpon which the
Church was built, doth he truſt that he is in the Church? When the
bleſſed Apoſtle S.~Paul alſo sheweth this Sacrament of vnitie, ſaying:
One body & one Spirit &c. Which vnitie we Bishops ſpecially that rule in
the Church, ought to hold faſt and maintaine, that we may proue the
Bishops function alſo it-ſelf to be one and vndiuided, &c.} And againe,
\CNote{\Cite{Ep.~40.}}
\MNote{Schiſme deteſtable.}
\Emph{There is one God, and one Chriſt, and one Church, and one Chaire,
by our Lordes voice founded vpon Peter. Another altar to be ſet-vp, or a
new Prieſthood to be made, beſides one altar & one Prieſthood, is
impoſſible. Whoſoeuer gathereth els-where, ſcattereth. It is adulterous,
it is impious, it is ſacrilegious, whatſoeuer is inſtituted by man's
furie to the breach of Gods diuine diſpoſition. Get ye farre from the
contagion of ſuch men, & fly from their ſpeaches as a canker and
peſtilence, our Lord hauing præmonished and warned before-hand. They are
blind, leaders of the blind &c.} Whereby we learne that this vnitie of
the Church commended ſo much vnto vs, conſiſteth in the mutual
fellowship of al Bishops with the See of Peter. S.~Hilarie alſo
\Cite{(li. ad Conſant. Auguſt.)}
thus applieth this ſame place of the Apoſtle againſt the Arians, as we
may doe againſt the Caluiniſts.
\MNote{Among heretikes as many faiths as wils.}
\Emph{Perilous and miſerable it is}, ſaith he, \Emph{that there are now
ſo many faiths as wils, and ſo many doctrines as manners; whiles either
faiths are ſo written as we wil, or as we wil, ſo are vnderſtood: and
wheras according to one God, and one Lord, and one Baptiſme, there is
alſo one faith, we fal away from that which is the only faith, and
whiles moe faiths be made, they begin to come to that, that there is
none at al.}}
one faith, one Baptiſme. \V
\CNote{\XRef{Malac.~2,~10.}}
One God and Father of al, which is ouer al, and by al, & in al vs. \V
But
\CNote{\XRef{Ro.~12,~4.}
\XRef{1.~Cor.~12,~4.}}
to euery one of vs is giuen grace according to the meaſure of the
donation of Chriſt. \V For the which he ſaith:
\CNote{\XRef{Pſ.~67,~19.}}
\Emph{Aſcending on high he lead captiuitie captiue: he haue guiftes to
men.} (\V And that he aſcended, what is it, but becauſe he deſcẽded alſo
firſt into the
\SNote{He meaneth ſpecially of his deſcending to Hel.}
the inferiour parts of the earth? \V He that deſcẽded, the ſame is alſo he
that is aſcended aboue al the Heauens, that he might fil al things.) \V
And
\CNote{\XRef{1.~Cor.~12,~28.}}
he gaue,
\LNote{Some Apoſtles.}{Many
\MNote{The Heretikes foolish negatiue argumẽt againſt the Pope
anſwered.}
functions that were euen in the Apoſtles time, are not here named: which
muſt be noted againſt the Aduerſaries that cal here for Popes. As though
the names of Bishops, Prieſts, or Deacons were not as wel left out as
Popes: whom yet they can not deny to haue been in vſe in S.~Paules
daies. And therfore they haue no more reaſõ out of this place to diſpute
againſt the Pope, thẽ againſt the reſt of the Eccleſiaſtical
functions. Neither is it neceſſarie to reduce ſuch as be not ſpecified
here, to theſe here named: though indeed both other Bishops and Prelates
and ſpecially Popes may be conteined vnder the names of Apoſtles,
Doctours, and Paſtours.
\MNote{The Popes office is called an Apoſtleship.}
Certes the room and dignitie of the Pope is a
very continual Apoſtleship, and S.~Bernard calleth it \L{Apoſtolatum}. 
\Cite{Bern. ad Euang. lib.~4. c.~4. &~c.~6, in fine.}}
ſome Apoſtles, & ſome Prophets, & other-ſome Euãgeliſts & other-ſome
Paſtours & Doctours, \V to the conſummation of the Saints, vnto the
worke of the Miniſterie, vnto the edifying of the body of Chriſt:
%%% 2761
\V
%%% !!! only marked in other
\LNote{Vntil we meet.}{The
\MNote{Continual ſucceſſion of Bishops, an euident argument of the true
viſible Church.}
Church of God shal neuer lack theſe ſpiritual functions, or ſuch as be
anſwerable to them, according to the time and ſtate of the Church, til
the worlds end. Whereby you may proue, the Catholike Church, that is to
ſay, that viſible companie of Chriſtians which hath euer had, and by
good recordes can proue they haue had, a continual ordinarie ſucceſſion
of Bishops, Paſtours, and Doctours, to be the only true Church: and
theſe other good fellowes that for many worlds or Ages together can not
shew that they had any one Bishop, or ordinarie yea or extraordinarie
officer for them and their Sect, to be an adulterous Heretical
Generation.
\MNote{The Fathers refuted Heretikes by the ſucceſſion of the Bishops of
Rome.}
And this place of the Apoſtle aſſuring to the true Church a perpetual
viſible continuance of Paſtours and Apoſtles or their Succeſſours,
warranted the holy Fathers to trie al Heretikes by the moſt famous
ſucceſſion of the Popes of Rome. So did 
\Cite{S.~Irenæus li.~3. c.~3.}
\Cite{Tertullian, in præſcript Optatus li.~2. cont. Parmen.}
\Cite{S.~Auguſtin, in pſ. cont. part. Donat.}
&
\Cite{cont. ep. Manic. c.~4.}
&
\Cite{Ep.~65.}
\Cite{Epip. hæreſ.~27.}
and others.}
vntil we meet al into the vnitie of faith and knowledge of the Sonne
of God into a perfect man, into the meaſure of the age of the fulnes of
Chriſt: \V that now we be not children wauering, and caried about
\LNote{With euery wind.}{The
\MNote{Heretical blaſts carie away the inconſtant only.}
ſpecial vſe of the ſpiritual Gouernours is, to keep vs in vnitie and
conſtancie of the Catholike faith, that we be not caried away with the
blaſt or wind of euery hereſie. Which is a very proper note of Sects and
new doctrines that trouble the infirme weaklings of the Church, by
certaine ſeaſons of diuers Ages: as ſometime the Arians, then the
Manichees, another time the Neſtorians, then the Lutherans, Caluiniſts,
and ſuch like: who at diuers times in diuers places, haue blowen diuers
blaſts of falſe doctrine.}
with euery wind of doctrine in the wickednes of men, in craftines to the
circumuention of errour. \V But doing the truth in charitie, let vs in
al things grow in him which is the Head,
%%% o-2619
Chriſt: \V of whõ the whole body being compact and knit together by al
iuncture of ſubminiſtratiõ, according to the operation in the meaſure of
euery member, maketh the increaſe of the body vnto the edifying of
it-ſelf in charitie.

\V This therfore I ſay and teſtifie in our Lord: that now you walke not
as alſo the
\CNote{\XRef{1.~Pet.~4,~3.}
\XRef{Ro.~1,~21.}}
Gentils walking in the vanitie of their ſenſe, \V hauing their
vnderſtanding obſcured with darkenes, alienated from the life of God by
the ignorance that is in them, becauſe of the blindnes of their hart, \V
who deſpairing,
\CNote{\XRef{Ro.~1,~14.}}
haue giuen vp themſelues to impudicitie, vnto the operation of al
vncleannes, vnto auarice. \V But you haue not ſo learned Chriſt: \V if
yet you haue heard him, & haue been taught in him, (as the truth is in
\Sc{Iesvs}.) \V
\CNote{\XRef{Coloſ.~3,~8.}
\XRef{Heb.~12,~1.}}
Lay you away according to the old conuerſatiõ the old man which
is corrupted according to the deſires of errour. \V And
\SNote{The Apoſtle teacheth vs not to apprehend Chriſt's iuſtice by
faith only, but to be renewed in our ſelues truly, & to put on vs the
man formed & created in iuſtice and holines of truth. By the which, free
wil alſo is proued to be in vs, to worke with God, or to conſent vnto
him in our ſanctification.}
be renewed in the ſpirit of your mind: \V and put on the new man which
according to God is created in iuſtice, and holineſſe of the truth. \V
For the which cauſe laying away lying,
\CNote{\XRef{Zach.~8,~16.}}
ſpeake ye truth euery one with
\Fix{is}{his}{obvious typo, fixed in other}
neighbour, becauſe we are members one of another.

\V
\CNote{\XRef{Pſ.~4,~5.}}
Be angrie and ſinne not. Let not the ſunne goe downe vpon your anger. \V
Giue not place to the Diuel. \V He that ſtole, let him now not ſteale:
but rather let him labour in working with his hands that which is good,
that he may haue whence to giue vnto him that ſuffereth neceſſitie. \V
Al naughtie ſpeach let it not proceed out of your mouth: but if there be
any good to the edifying of the faith, that it may giue grace to the
hearers. \V And contriſtate not the holy Spirit of God: in which you are
ſigned vnto the day of redemption. \V Let al bitternes and anger, and
indignation, and clamour, and blaſphemie be taken away from you with al
malice. \V And be gentle one to another, merciful, pardoning one
another, as alſo God in Chriſt hath pardoned you.



\stopChapter


\stopcomponent


%%% Local Variables:
%%% mode: TeX
%%% eval: (long-s-mode)
%%% eval: (set-input-method "TeX")
%%% fill-column: 72
%%% eval: (auto-fill-mode)
%%% coding: utf-8-unix
%%% End:

