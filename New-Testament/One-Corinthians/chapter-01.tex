%%%%%%%%%%%%%%%%%%%%%%%%%%%%%%%%%%%%%%%%%%%%%%%%%%%%%%%%%%%%%%%%%
%%%%
%%%% The (original) Douay Rheims Bible 
%%%%
%%%% New Testament
%%%% Epistles
%%%% One Corinthians
%%%% Chapter 01
%%%%
%%%%%%%%%%%%%%%%%%%%%%%%%%%%%%%%%%%%%%%%%%%%%%%%%%%%%%%%%%%%%%%%%




\startcomponent chapter-01


\project douay-rheims


%%% 2672
%%% o-2525
\startChapter[
  title={Chapter 1}
  ]

\Summary{After
\MNote{The 1.~part.

Of Schiſmes that were about their Baptizers & Preachers.}
ſalutation, 4.~hauing acknowledged the graces of their Church, 10.~he
dehorteth them from their Schiſmatical boaſting againſt one another in
their Baptizers (telling them that they muſt boaſt only in Chriſt for
their Baptiſme) 17.~and in their Preachers, who had the wiſedom of
words: telling them that it is the preaching of the Croſſe, whereby God
ſaueth the world, and wherin only Chriſtians should boaſt: 26.~ſeeing
God of purpoſe choſe the contemptible, that ſo himſelf might haue the
glorie.}

Paul called to be an Apoſtle of \Sc{Iesvs} Chriſt, by the wil of God,
and Softhenes a Brother, \V to the Church of God that is at Corinth, to
the ſanctified in Chriſt \Sc{Iesvs}, called to be Saints, with al that
inuocate the name of our Lord \Sc{Iesvs} Chriſt in euery place of theirs
and ours. \V Grace to you and peace from God our Father and our Lord
\Sc{Iesvs} Chriſt.

\V I giue thanks to my God alwaies for you for the grace of God that is
giuen you in Chriſt \Sc{Iesvs}, \V that in al things you be made rich in
him, in al vtterance, and
\LNote{In al knowledge.}{Obſerue
\MNote{Faith commeth by hearing rather then reading.}
that the Apoſtles neuer wrote their letters but to ſuch as were
conuerted to Chriſtes faith before. For men can not lightly learne the
Chriſtian religion by reading Scriptures, but by hearing and by the
preſence of their Teachers, which may inſtruct them at large and
particularly of euery Article, as clerely and breefly by letters they
could not doe. Neither doth now any man learne his faith firſt but by
hearing of his parents and Maiſters. For if we should when we come to
yeares of diſcretion, be ſet to picke our faith out of the Scriptures,
there would be a mad worke and many faiths among vs.}
in al knowledge, (\V as the teſtimonie of Chriſt is confirmed in
you,) \V ſo that nothing is wanting to you in any grace, expecting the
reuelation of our Lord \Sc{Iesvs} Chriſt, \V who alſo wil confirme you
vnto the end without crime, in the day of the comming of our
Lord \Sc{Iesvs} Chriſt. \V God is faithful; by whom you are called into
the ſocietie of his Sonne \Sc{Iesvs} Chriſt our Lord.

\V And I beſeech you, Brethren, by the name of our Lord \Sc{Iesvs}
Chriſt, that you al ſay one thing, and that there be no ſchiſmes among
you: but that you be perfect in one ſenſe, &
%%% o-2526
in one knowledge. \V For it is ſignified vnto me (my Brethren) of you,
by them that are of Chloe, that there be contentions among you. \V And
I meane this, for that euery one of you ſaith:
\SNote{The beginning of al Schiſmes is ouermuch admiring & addicting
mens ſelues to their owne particular Maiſters.}
I certes am Paules, & I Apollo's, but I Cepha's, and I Chriſt's. \V Is
Chriſt deuided? Why, was Paul crucified for you? or in the name of Paul
were you baptized? \V I giue God
%%% 2673
thanks, that I baptized none of you, but
\CNote{\XRef{Act.~18,~8.}}
Criſpus and Caius: \V leſt any man ſay that in my name you were
baptized. \V And I baptized alſo the houſe of Stephanas. But I know not
if I haue baptized any other.

\V For Chriſt ſent me not to baptize, but to euangelize: not in wiſedom
of ſpeach, that the croſſe of Chriſt be not made void. \V For the word
of the croſſe, to them indeed that periſh, is fooliſhnes; but to them
that are ſaued, that is, to vs, it is the power of God. \V For it is
written:
\CNote{\XRef{Eſ.~33,~18.}}
\Emph{I wil deſtroy the wiſedom of the wiſe; and the prudence of the
prudent I wil reiect. \V Where is the wiſe? where is the Scribe? where
is the diſputer of this world?} Hath not God made the wiſdom of this
world fooliſh? \V For becauſe in the wiſedom of God the world did not by
wiſedom know God; it pleaſed God by the fooliſhnes of the preaching to
ſaue them that beleeue. \V For both the Iewes aske ſignes, and the
Greeks ſeeke wiſedom: \V but we preach Chriſt crucified, to the Iewes
certes a ſcandal, and to the Gentils, fooliſhnes: \V but to the called
Iewes & Greeks, Chriſt the power of God and the wiſedom of God. \V For
that which is the fooliſh of God, is wiſer then men; and that which is
the infirme of God, is ſtronger then men. \V For ſee your vocation,
Brethren, that not many wiſe according to the fleſh, not many mightie,
not many noble: \V but the fooliſh things of the world hath God choſen,
that he may confound the wiſe; and the weak things of the world hath God
choſen, that he may confound the ſtrong: \V and the baſe things of the
world and the contemptible hath God choſen, and thoſe things which are
not, that he might deſtroy thoſe things which are; \V that no fleſh may
glorie in his ſight. \V And of him you are in Chriſt \Sc{Iesvs},
\LNote{Who is made.}{He
\MNote{Chriſt is made our iuſtice, becauſe he is the Authour of the
iuſtice in vs.}
meaneth not, as our Aduerſaries captiouſly take it, that we haue no
iuſtice, ſapience, nor ſanctity of our owne, other then Chriſtes imputed
to vs: but the ſenſe is, that he is made the Authour, giuer, and
meritorious cauſe of al theſe vertues in vs. For ſo the Apoſtle
interpreteth himſelf plainly in the
\XRef{6.~chapter}
following, when he writeth thus: \Emph{You be washed, you be iuſtified,
you be ſanctified in the name of our Lord} \Sc{Iesvs Christ} \Emph{and
in the Spirit of our God.}}
who is made vnto vs wiſedom from God, & iuſtice, ſanctification, and
redemption: \V that as it is written:
\CNote{\XRef{Ier.~9,~23.}}
\Emph{He that doth glorie, may glorie in our Lord.}


\stopChapter


\stopcomponent


%%% Local Variables:
%%% mode: TeX
%%% eval: (long-s-mode)
%%% eval: (set-input-method "TeX")
%%% fill-column: 72
%%% eval: (auto-fill-mode)
%%% coding: utf-8-unix
%%% End:

