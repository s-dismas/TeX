%%%%%%%%%%%%%%%%%%%%%%%%%%%%%%%%%%%%%%%%%%%%%%%%%%%%%%%%%%%%%%%%%
%%%%
%%%% The (original) Douay Rheims Bible 
%%%%
%%%% New Testament
%%%% Epistles
%%%% One Corinthians
%%%% Chapter 02
%%%%
%%%%%%%%%%%%%%%%%%%%%%%%%%%%%%%%%%%%%%%%%%%%%%%%%%%%%%%%%%%%%%%%%




\startcomponent chapter-02


\project douay-rheims


%%% 2674
%%% o-2527
\startChapter[
  title={Chapter 2}
  ]

\Summary{That his owne preaching among them, was in humble manner in the
  ſight of man. 5.~Howbeit it is moſt profound wiſedom (as they should
  and would perceiue, if they were not carnal) which is taught in the
  Church of Chriſt.}

And I (Brethren) when I came to you, I came not in loftineſſe of ſpeach
or of wiſedom, preaching to you the teſtimonie of Chriſt. \V For I
iudged not my ſelf to know any thing among you but \Sc{Iesvs} Chriſt,
and him crucified. \V And
\CNote{\XRef{Act.~18.}}
I was with you in infirmitie, and feare and much trembling: \V and my
ſpeach and my preaching was not in the perſuaſible words of humane
wiſedom, but in ſhewing of ſpirit and power; \V that your faith might
not be in the wiſedom of men, but in the power of God.

But we ſpeake wiſedom among the perfect. \V But the wiſedom not of this
world, neither of the Princes of this world, that come to naught: \V but
we ſpeake the wiſedom of God in a myſterie, which is hid, which God did
predeſtinate before the worlds, vnto our glorie: \V which none of the
Princes of this world did know: for if they had knowen, they would neuer
haue crucified the Lord of glorie. \V But as it is written:
\CNote{\XRef{Eſ.~64,~4.}}
\Emph{That which eye hath not ſeen, nor eare hath heard, neither hath it
aſcended into the hart of man, what things God hath prepared for them
that loue him.} \V But to vs God hath reuealed by his Spirit. For the
%%% o-2528
Spirit ſearcheth al things, yea the profundities of God. \V For what man
knoweth the things of a man, but
\LNote{But the ſpirit of man.}{One
\MNote{How Angels and Saints & mortal men know our cogitations.}
man can not know another's cogitations naturally: but God giueth to
Prophets and other, euen in this world oftentimes, by extraordinary
grace to know mens ſecrets. As he did to 
\CNote{\XRef{Act.~5,~4.}}
S.~Peter, to know the fraud of Ananias and Saphira: and to
\CNote{\XRef{4.~Reg.~5.}
&
\XRef{6.}}
Eliſeus, his ſeruant's bribery in his
abſence, and what was done in the King of Syria his chamber. And he
giueth to al
\CNote{\XRef{Luc.~15,~7.}}
Angels and Saints (ſo farre as is conuenient to our neceſsities and
their heauenly glorie) to vnderſtand not only our vocal praiers, but our
inward repentance and deſires.}
the ſpirit of a man that is in him? ſo the things alſo that are of God
no man knoweth, but the ſpirit of God.

\V And we haue receiued not the ſpirit of this world; but the ſpirit
that is of God:
\LNote{That we may know.}{The
\MNote{The Heretikes allegation for their vaine ſecuritie, anſwered.}
Proteſtants that chalenge a particular ſpirit reuealing to each one his
owne predeſtination, iuſtification, and ſaluation, would draw this text
to that purpoſe. Which importeth nothing els (as is plaine by the
Apoſtles diſcourſe) but that the Holy Ghoſt hath giuen to the Apoſtles,
& by them to other Chriſtian men, to know God's ineffable guifts
beſtowed vpon the beleeuers in this time of grace: that is, Chriſtes
Incarnation, Paſsion, preſence in the Sacrament, & the incomprehenſible
ioyes of Heauen, which Pagans, Iewes and Heretikes deride.}
that we may know the things that of God are giuen to vs. \V Which alſo
we ſpeake not in learned words of humane wiſedom; but in the doctrine of
the Spirit, comparing ſpiritual things to the ſpiritual. \V But
\LNote{The ſenſual man.}{The
\MNote{The ſenſual man.}
ſenſual man is he ſpecially, that meaſureth theſe heauenly myſteries by
natural reaſon, humane prudence, external ſenſe, and worldly affection,
as the Iew, Pagane, and Heretike doe: and ſometime both here and
elſwhere the more infirme and ignorant ſort of Chriſtian men be called
ſenſual or carnal alſo, who being occupied in ſecular affaires, and
giuen to ſenſual ioy and worldlines, haue no ſuch ſenſe nor feeling of
theſe great guifts of God, as the perfecter ſort of the faithful
haue. Who trying theſe high points of religion, not by reaſon and ſenſe,
but by grace, faith, and Spirit, be therfore called ſpiritual.
\MNote{The ſpiritual man.}
The ſpiritual then is he, that iudgeth and diſcerneth the truth of ſuch
things as the carnal can not attaine vnto: 
\MNote{How the ſpiritual man iudgeth al, & is iudged of none.}
that doth by the ſpirit of the Church, wherof he is partaker in the
vnitie of the ſame, not only ſee the errours of the carnal, but
condemneth them and iudgeth euery power reſiſting God's ſpirit and word:
the carnal Iew, Heathen, or Heretike, hauing no meanes nor right to
iudge of the ſaid ſpiritual man. For when the ſpiritual is ſaid to be
iudged of none, the meaning is not that he should not be ſubiect or
obedient to his Paſtours and ſpiritual Powers and to the whole Church,
ſpecially for the trail or examination of al his life, doctrine, and faith:
but that a Catholike man and namely a Teacher of Catholike doctrine in
the Church, should not be any whit ſubiect to the iudgement of the
Heathen or the Heretike, nor care what of ignorance or infidelitie they
ſay againſt him. For ſuch carnal men haue no iudgement in ſuch things,
nor can attaine to the Churches wiſedom in any ceremonie, myſterie, or
matter which they condemne.

Therfore S.~Irenæus excellently declaring that the Church and euery
ſpiritual child therof, iudgeth and condemneth al falſe Prophets and
Heretikes of what ſort ſoeuer, at length concludeth with theſe notable
words:
\CNote{Iren. li.~4. c.~62.}
\Emph{The ſpiritual shal iudge alſo al that make ſchiſmes, which be
cruel, not hauing the loue of God, and reſpecting their owne priuate,
more then the vnitie of the Church; mangle, deuide, and (as much as in
them liteth) kil for ſmal cauſes the great and glorious body of Chriſt,
ſpeaking peace, and ſeeking battaile. He shal iudge alſo them that be
out of the truth, that is to ſay, out of the Church:
\MNote{The Church is vnder no man's iudgement.}
which Church shal be vnder no man's iudgement for to the Church are al
things knowen, in which is perfect faith of the Father, and of al the
diſpenſation of Chriſt, and firme knowledge of the Holy Ghoſt that
teacheth al truth.}}
the ſenſual man perceiueth not thoſe things that are of the Spirit of
God. For it is fooliſhnes to him, and he can not vnderſtand; becauſe he
is ſpiritually examined. \V But the ſpiritual man iudgeth al things: and
himſelf is iudged of no man. \V For
\CNote{\XRef{Eſa.~40,~14.}}
who hath knowen the ſenſe of our Lord that may inſtruct him? But we haue
the ſenſe of Chriſt.


\stopChapter


\stopcomponent


%%% Local Variables:
%%% mode: TeX
%%% eval: (long-s-mode)
%%% eval: (set-input-method "TeX")
%%% fill-column: 72
%%% eval: (auto-fill-mode)
%%% coding: utf-8-unix
%%% End:

