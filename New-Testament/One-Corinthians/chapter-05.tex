%%%%%%%%%%%%%%%%%%%%%%%%%%%%%%%%%%%%%%%%%%%%%%%%%%%%%%%%%%%%%%%%%
%%%%
%%%% The (original) Douay Rheims Bible 
%%%%
%%%% New Testament
%%%% Epistles
%%%% One Corinthians
%%%% Chapter 05
%%%%
%%%%%%%%%%%%%%%%%%%%%%%%%%%%%%%%%%%%%%%%%%%%%%%%%%%%%%%%%%%%%%%%%




\startcomponent chapter-05


\project douay-rheims


%%% 2679
%%% o-2533
\startChapter[
  title={Chapter 5}
  ]

\Summary{
\Fix{Sharpyrebuking}{Sharply rebuking}{obvious typo, fixed in other}
\MNote{The ſecond part of the Epiſtle: of the inceſtuous fornicatour; &
  lawing before Infidels.}
their Chergies negligence, 3.~himſelf abſent excommunicateth that
publike inceſtuous perſon: 6.~commanding that hereafter no Chriſtian be
ſo tolerated in any open crime, but excommunicated.}

There is plainely heard fornication among you, and ſuch fornication, as
the like is not among the Heathen, ſo that one hath his
\CNote{\XRef{Leu.~18,~8,}
\XRef{20,~11.}}
fathers wife. \V And you are puffed vp; and
\SNote{Chriſtian men should be ſorrowful to ſee greuious offences borne
withal, and ought zelouſly to ſeeke the offenders punishment by
excommunication.}
haue not mourned rather, that he might be taken away from among you,
that hath done this deed. \V
\LNote{I abſent.}{S.~Paul here vſeth his Apoſtolike power, of binding
this inceſtuous perſon, excommunicating him by his letters
and \L{Mandatum}, though abſent.}
I indeed abſent in body, but preſent in ſpirit, haue already iudged, as
preſent, him that hath ſo done, \V in the name of our Lord \Sc{Iesvs}
Chriſt,
\LNote{You being gathered.}{Though
\MNote{The authoritie of Eccleſiaſtical cenſures is in the Clergie only,
& is executed in the name of Chriſt.}
he commanded the acte ſhould be done in the face of the Church, as ſuch
ſentences and cenſures be at this day executed alſo, yet the iudgement
and authoritie of giuing ſentence was in himſelf, and not in the whole
multitude, as the Proteſtants and the popular Sectaries affirme. For the
power of binding & looſing was not giuen to the whole Church, but as in
the perſons of the Prelates, & to them for the benefit of the
whole. Whervpon
\Cite{S.~Chryſoſtome vpon theſe words,}
\L{Dic Eccleſia}, \Emph{Tel the Church}, Mat.~18. \Emph{Complaine to the
Church, that is,} ſaith he, \Emph{to the Prelates and Preſidents therof.}}
you being gathered together and my ſpirit,
\LNote{With the vertue.}{Al ſuch great power ouer ſinners, is holden and
exerciſed in the name & vertue of \Sc{Christ Iesvs} and whoſoeuer
ſetteth light by it, deſpiſeth our Lord's name and power.}
with the vertue of our Lord \Sc{Iesvs}; \V to deliuer ſuch an one
\LNote{To Satan.}{To aſſure vs that al excommunicate perſons be in the
power & poſſeſſion of the Diuel, & quite out of Chriſt's protection as
ſoone as they be ſeparated by the Churches ſentence, from her body and
the Sacraments and fellowſhip of Chriſtian Catholike men; 
\MNote{The terrible ſentence of excommunication.}
it pleaſed God
to giue power to the Apoſtles and Prelates in the primitiue Church, to
cauſe the Diuel ſtraight vpon their ſentẽce of excõmunication, to inuade
the body of the excommunicate, & to torment him corporally. So Chriſt
excommunicated Iudas, and the Diuel entred into him, and he went forth
of the happie fellowship of the Apoſtles.
\XRef{Io.~13,~27.}
So this Apoſtle excommunicated Alexander and Hymenæus, and Satan
ſtraight tooke them:
\XRef{1.~Tim.~1.}
Yea it is thought that
\CNote{\XRef{Act.~5.}}
S.~Peter excommunicated Ananias & Sapphira, and for ſigne of his power
and terrour of the ſentence ſtrook them both ſtarke dead.
\Cite{De mirabil. S.~Scripturæ li.~3. c.~16. apud D.~Aug.}
Which miraculous power though it be not ioyned not to that ſentence, yet
as farre as concerneth the puniſhment ſpiritual, which it ſpecially
appartaineth vnto, it is as before, and is by the iudgement of the holy
Doctours
\Cite{(Cyp. ep.~16. nu.~3.}
\Cite{Chryſ. in 1.~Tim.~1. ho.~5.}
\Cite{Ambroſ. ref. in 1.~Tim.~1.}
\Cite{Hiero. ep. ad Heliod. c.~7.}
\Cite{Aug. de cor. & gra. c.~11.})
the terribleſt and greateſt puniſhment in the world; yea farre paſsing al
earthly paine and torment of this life, and being a very reſemblance of
damnation, and ſo often called by the Fathers, namely S.~Auguſtine.
\Emph{And by this ſpiritual ſword}
\CNote{\Cite{locis citatis.}}
(Saith S.~Cyprian) \Emph{al muſt die in their ſoules, that obey not the
Prieſts of Chriſt in the new law, as they that were diſobedient to the
Iudges of the old law, were ſlaine with the corporal ſword.} Would God
the world knew what a maruelous puniſhment Chriſt hath appointed the
Prieſts to execute vpon the offenders of his lawes, and ſpecially vpon
the diſobedient, as Heretikes namely.}
to Satan for the deſtruction of the flesh, that the ſpirit may be ſaued
in the day of our Lord \Sc{Iesvs} Chriſt. \V Your glorying is not
good. Know you not that a litle leauen corrupteth the whole paſte? \V
Purge the old leauen, that you may be a new paſte, as you are
azymes. For our Paſche, Chriſt, is immolated. \V Therfore
\LNote{Let vs feaſt.}{The
\MNote{Puritie in receiuing the B.~Sacrament.}
Paſchal lamb, which was the moſt expreſſe figure of Chriſt euery way,
\CNote{\XRef{Exo.~12.}}
was firſt ſacrificed and afterward eaten with azymes or vnleauened
bread. So Chriſt our Paſchal, being then newly ſacrificed on the Croſſe,
is recommended to them as to be eaten with al puritie and ſinceritie, in
the Holy Sacrament. Which myſterie the holy Church in theſe words
cõmendeth to the faithful euery yeare at the feaſt of Eaſter.}
let vs feaſt, not in the old leauen, nor in the leauen of malice and
wickednes, but in the azymes of ſinceritie and veritie.

\V I wrote to you in
\SNote{Either this Epiſtle in the words before, or ſome other.}
an epiſtle, not to keep companie with fornicatours. \V I meane not the
fornicatours of this
%%% o-2534
world, or the couetous or the extorſioners, or ſeruers of
\Fix{Idolds:}{Idols:}{obvious typo, fixed in other}
otherwiſe you ſhould haue gone out of this world. \V But now I wrote to
you, not to keep companie, if he that is named a Brother, be a
fornicatour, or a couetous perſon, or
\SNote{A notorious wilful corruption in the
\Cite{bible~1562:}
tranſlating in the verſe before, \Emph{Idolaters}; and
here, \Emph{worshipper of images}: the Apoſtles word being one,
\G{εἰδωλολάτρης}, \Emph{Idolater}.}
a ſeruer of Idols, or a railer, or a drunkard, or an
%%% 2680
extorſioner: with ſuch an one
\LNote{Not to take meat.}{It
\MNote{We are bound to auoid, not al ſinners, but the excommunicate
only, & them, except in certaine caſes.}
is not meant that we ſhould ſeparate our ſelues corporally frõ al
ſinners, or that we might refuſe to liue in one Church or fellowship of
Sacraments with them, which was the errour & occaſion of the Donatiſtes
great ſchiſme: nor that euery man is ſtraight after he hath committed
any deadly ſinne, excommunicated, as ſome Lutherãs hold: but that we
ſhould auoid thẽ when the Church hath excõmunicated them for ſuch:
though in mind,
and condemnation of their faults, euery one ought to be alwaies farre
from them. As for the Heathen & Pagans, which be not vnder the Churches
diſcipline, and at that time in external worldly affaires dealt with
Chriſtians and liued among them whether they would or no, the Apoſtle
did not forbid Chriſtians their companie.}
not ſo much as to take meat. \V For what is it to me to iudge of thẽ
that are without? Doe not you iudge of them that are within? \V for them
that are without, God wil iudge. Take away
\LNote{The euil one.}{He concludeth that though they can not, nor
himſelf neither, cut off the Heathen that be publike offenders, yet the
il perſon by him excommunicated being one of their owne body, they may
cut off, as is aforeſaid, and auoid his company. Vpon which commandement
of the Apoſtle, we ſee that we are bound by God's word to auoid al
companie and conuerſation with the excommunicate, except in caſes of
neceſsitie, and the ſpiritual profit of the perſon excommunicated.}
the euil one from among your ſelues.


\stopChapter


\stopcomponent


%%% Local Variables:
%%% mode: TeX
%%% eval: (long-s-mode)
%%% eval: (set-input-method "TeX")
%%% fill-column: 72
%%% eval: (auto-fill-mode)
%%% coding: utf-8-unix
%%% End:

