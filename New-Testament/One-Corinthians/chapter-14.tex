%%%%%%%%%%%%%%%%%%%%%%%%%%%%%%%%%%%%%%%%%%%%%%%%%%%%%%%%%%%%%%%%%
%%%%
%%%% The (original) Douay Rheims Bible 
%%%%
%%%% New Testament
%%%% One Corinthians
%%%% Chapter 14
%%%%
%%%%%%%%%%%%%%%%%%%%%%%%%%%%%%%%%%%%%%%%%%%%%%%%%%%%%%%%%%%%%%%%%

%%% Latin checked by KK.



\startcomponent chapter-14


\project douay-rheims


%%% 2702
%%% o-2557
\startChapter[
  title={Chapter 14}
  ]

\Summary{Againſt their vaine childishnes, that thought it a goodly
  matter to be able to ſpeake (by miracle) ſtrange languages in the
  Church, 
  %%% !!! Present in both, but marked only in other
\SNote{Much like to ſome fond Linguiſts of our time, who thinke
  themſelues better thẽ a Doctour of Diuinitie that is not a Linguiſt.}
  preferring their languages before prophecying, that is opening
  of myſteries: he declareth that this guift of languages is inferiour
  to the guift of prophecy. 26.~Giuing order alſo how both guifts are to
  be vſed; to wit, the Prophet to ſubmit himſelf to other Prophets: &
  the Speaker of Languages not to publish his inſpiration, vnles there
  be an Interpreter. 34.~Prouided alwaies, that women ſpeake not at al in
  the Church.}

%%% o-2558
Follow Charitie, earneſtly purſue ſpiritual things: but
\LNote{Rather prophecie.}{The
\MNote{A paraphraſtical expoſition of this Chapter concerning vnknowen
tongues.}
guift of prophecying, that is, of expounding the hard points of our
religion, is better then the guift of ſtrange tongues, though both be
good.}
rather that you may prophecy. \V For he that ſpeaketh with tongues
ſpeaketh
%%% !!! marked only in other
\LNote{Not to men.}{To talke in a ſtrange language, vnknowen alſo to
himſelf, profiteth not the hearers, though in reſpect of God who
vnderſtandeth al tongues and things, and for the myſteries which he
vttereth in his ſpirit, and for his owne edification in ſpirit and
affection, there be no difference: but the Prophet or Expoſitour
treating of the ſame matters to the vnderſtanding of the whole
aſſemblie, edifieth not himſelf alone but al his hearers.}
not to men, but to God: for no man heareth. But in ſpirit he
ſpeaketh myſteries. \V For he that prophecieth, ſpeaketh to men vnto
edification, & exhortatiõ, & cõſolation.

\V He that ſpeaketh with tongues, edifieth himſelf; but he that
prophecieth, edifieth the Church. \V And I would haue you al to ſpeake
with tongues, but rather to prophecy. For greater is he that
prophecieth, then he that ſpeaketh with tongues: vnleſſe perhaps he
interpret, that the Church may take edification. \V But now, Brethren;
\LNote{If I come.}{That is, if I your Apoſtle, and Doctour should preach
to you in an vnknowen tongue, and neuer vſe any kind of expoſition,
interpretation, or explication of my ſtrange words, what profit could
you take thereby?}
if I come to you ſpeaking with tongues, what ſhal I profit you, vnleſſe
I ſpeake to you either in reuelation, or in knowledge, or in prophecie,
or in doctrine? \V Yet the things without life that giue a ſound, be it
pipe or
%%% 2703
harpe, vnleſſe they giue a diſtinction of ſounds, how ſhal that be
knowen which is piped, or which is harped? \V For 
%%% !!! marked only in other
\LNote{If the trumpet.}{As the Trumpeter can not giue warning to or from
the fight, vnles he vſe a diſtinct & intelligible ſound or ſtroke knowen
to the ſouldiars: euen ſo the Preacher that exhorteth to good life, or
dehorteth from ſinne, except he doe it in a ſpeach which his hearers
vnderſtand, can not attaine to his purpoſe, nor doe the people any
good.}
if the trumpet giue an
vncertaine voice, who ſhal prepare himſelf to battel? \V So you alſo by
a tongue vnleſſe you vtter manifeſt ſpeach, how ſhal that be knowen that
is ſaid? for you ſhal be ſpeaking into the aire. \V There are (for
example) ſo many kinds of tongues in this world, & none is without
voice. \V If then I know not the vertue of the voice, I ſhal be to him
to whom I ſpeake, barbarous; and he that ſpeaketh barbarous to me. \V So
you alſo, becauſe you be emulatours of ſpirits: ſeek to abound vnto the
edifying of the Church. \V And therfore he that ſpeaketh with the
tongue,
\LNote{Let him pray that.}{He that hath only the guift of ſtrange
tongues, let him pray to God for the guift of interpretation; that the
one may be more profitable by the other. For, to exhort or preach in a
ſtrange tongue was not vnlawful nor vnprofitable, but glorious to God,
ſo that the ſpeach had been either by himſelf, or by another, afterward
expounded.}
let him pray that he may interpret. \V For if I pray with the tongue,
\LNote{My ſpirit praieth.}{Alſo when a man praieth in a ſtrange tongue
which himſelf vnderſtandeth not, it is not ſo fruitful for inſtruction
to him, as if he knew particularly what he praied. Neuertheles the
Apoſtle forbiddeth not ſuch praying neither, confeſſing that his ſpirit,
hart, and affection praieth wel towards God, though his mind &
vnderſtanding be not profited to inſtruction, as otherwiſe it might haue
been if he vnderſtood the words. Neither yet doth he appoint ſuch an one
to get his ſtrange praier tranſlated into the vulgar tongue, to obteine
thereby the foreſaid inſtruction. See the
\XRef{Declaration following of this Chapter.}}
my ſpirit praieth, but my vnderſtanding is without fruit.

\V What is it then? I wil pray in the ſpirit, I wil pray alſo in the
vnderſtanding: I wil ſing in the ſpirit, I wil ſing alſo in the
vnderſtanding. \V But if thou bleſſe in the ſpirit, he that ſupplieth
the place
\SNote{By this word are meant al rude vnlearned men, but ſpecially the
ſimple which were yet vnchriſtned, as the Catechumens, which came in to
thoſe ſpiritual exerciſes, as alſo infidels did at their pleaſures.}
\TNote{\L{idiotæ}.}
of the vulgar how ſhal he ſay, Amen, vpon thy bleſſing? becauſe he
knoweth not what thou ſaieſt. \V For thou indeed giueſt thankes wel, but
the other is not edified. \V I giue my God thankes, that I ſpeake
\Var{with the tongue of you al.}{with tongues more thẽ you al.}
\V But in the Church I wil ſpeake fiue words with my vnderſtanding that
I may
%%% o-2559
inſtruct others alſo; rather then ten thouſand words in a tongue. \V
Brethren, be not made children in ſenſe, but in malice be children, and
in ſenſe be perfect. \V In the Law it is written:
\CNote{\XRef{Eſ.~28,~11.}}
\Emph{That in other tongues and other lippes I wil ſpeake to this
people: and neither ſo wil they heare me, ſaith our Lord.} \V Therfore
languages are for
\LNote{A ſigne.}{The extraordinarie guift of tongues was a miraculous
ſigne in the primitiue Church, to be vſed ſpecially in the Nations of
the Heathen for their conuerſion.}
a ſigne not to the faithful, but to infidels: but prophecies, not to
infidels, but to the faithful. \V If therfore the whole Church come
together in one, and al ſpeake with tongues, and there enter in vulgar
perſons or
\LNote{Infidels.}{In the primitiue Church, when Infidels dwelt neer or
among Chriſtians, and oftentimes came vnto their publike preaching &
exerciſes of exhortation and expoſition of Scriptures and the like: it
was both vnprofitable and ridiculous to heare a number talking,
teaching, ſinging Pſalmes, & the like, one in this language, & another
in that, al at once like a black-faunts, and one often not vnderſtood of
another; ſometime not to themſelues, and to ſtrangers or the ſimple
ſtanders by, not at al. Where otherwiſe if they had ſpoken either in
knowen tongues, or had done it in order, hauing an expoſitour or
interpreter withal, the Infidels might haue been conuinced.}
infidels, wil they not ſay that you be mad? \V But if al prophecie, and
there enter in any infidel or vulgar perſon, he is conuinced of al, he
is iudged of al. \V The ſecrets of his hart are made manifeſt, and ſo
falling on his face he wil adore God, pronouncing that God is in you
indeed.

\V What is it then, Brethrẽ? when you come together, euery one of you
hath 
%%% !!! marked only in other
\LNote{A Pſalme.}{We
\MNote{Of what ſpiritual exerciſe the Apoſtle ſpeaketh.}
ſee here that thoſe ſpiritual exerciſes conſiſted ſpecially, firſt, in
ſinging or giuing forth new Pſalmes or praiers and lauds: ſecondly, in
Doctrine, teaching, or reading lectures: thirdly, in Reuelations of
ſecret things either preſent or to come: fourthly, in ſpeaking tongues
of ſtrange Nations: laſtly, in tranſlating or interpreting that which
was ſaid, into ſome common knowen language, as into Greek, Latin, &c. Al
which guifts they had among them by miracle from the Holy Ghoſt.}
a pſalme, hath a doctrine, hath a reuelation, hath a tongue, hath
an interpretation: let al things be done to edification. \V Whether a
man ſpeake with tongue, by two, or at the moſt by three, and
\LNote{In courſe.}{Al
\MNote{The diſorders in the ſame.}
theſe things they did without order, of pride and contention, they
preached, they prophecied, they praied, they bleſſed, without any ſeemly
reſpect one of another, or obſeruing of turnes and entercourſe of
vttering their guifts. Yea women without couer or veile, and without
regard of their ſexe or the Angels, or Prieſts or their owne husbands,
malapertly ſpake tongues, taught or prophecied with the reſt. This was
then the diſorder among the Corinthians, which the Apoſtle in this whole
chapter reprehendeth and ſought to redreſſe, by forbidding women vtterly
that publike exerciſe, and teaching men, in what order and courſe as wel
for ſpeaking in tongues, as interpreting and prophecying it should be
kept.}
in courſe, and let one interpret. \V But if there be not an interpreter,
let him hold his peace in the Church, and ſpeake to himſelf and to
God. \V And let Prophets ſpeake two or three, and let the reſt iudge. \V
But if it be reuealed to another ſitting, let the firſt hold his
peace. \V For you may al prophecie one by one: that al may learne, and
al may be exhorted: \V and the ſpirits of prophets are ſubiect to
prophets. \V For God is not the God of diſſenſion, but of peace: as alſo
in al the Churches of the Saints I teach.

\V
%%% !!! A More Ample Declaration of the Sense of this 14.~Chapter.
%%% !!! actually appears before this annotation.
\LNote{Let women hold their peace.}{There
\MNote{Women may haue any temporal Soueraigntie, but no Eccleſiaſtical
function.}
be, or were, certaine Heretikes in our Countrie (for ſuch euer take the
Scriptures diuerſely for the aduantage of time) that denied women to
hold lawfully any kingdom or temporal Soueraignty: but that is falſe and
againſt both reaſon and the Scriptures. 
\CNote{\XRef{1.~Cor.~11,~16.}}
This only in that ſexe is true,
that it is not capable of holy orders, ſpiritual Regiment or Cure of
ſoules: and therfore can not doe any function proper to Prieſts and
Bishops: nor ſpeake in the Church, and ſo not preach, nor diſpute, nor
haue or giue voice deliberatiue or definitiue in Councels and publike
Aſſemblies, concerning matters of Religion, nor make Eccleſiaſtical
lawes concerning the ſame, nor bind, nor looſe, nor excommunicate, nor
ſuſpend, nor degrade, nor abſolue, nor miniſter Sacraments, other then
Baptiſme in the caſe of mere neceſsitie, when neither Prieſt nor other
mã cã be had: much leſſe preſcribe any thing to the Clergie, how to
miniſter thẽ, or giue any man right to rule, preach, or execute any
ſpiritual function as vnder her & by her authoritie: no creature being
able to impart that wherof itſelf is incapable both by nature &
Scriptures. This Regiment is expreſly giuen to the Apoſtles, Bishopes,
and Prelates: they only haue authoritie to bind and looſe,
\XRef{Mat.~18.}:
they only are ſet by the Holy Ghoſt to gouerne the Church,
\XRef{Act.~20}:
they only haue cure of our ſoules directly, and muſt make account to God
for the ſame,
\XRef{Hebr.~13.}}
Let
\CNote{\XRef{1.~Tim.~2,~12.}}
women hold their peace in the Churches: for it is not permitted thẽ to
ſpeake, but to be ſubiect, as alſo
\CNote{\XRef{Gen.~3,~16.}}
the Law ſaith. \V But if they liſt learne any thing, let them aske their
owne husbands at home. For it is a foule thing for a woman to ſpeake in
the Church. \V Or did the word of God proceed from you? came it vnto you
only? \V If any man ſeeme to be a Prophet, or ſpiritual, let him know
the things that I write
%%% 2704
to you, that they are the commandements of our Lord. \V But if any man
know not, he ſhal not be knowen. \V Therfore, Brethren, be earneſt to
prophecie: and to ſpeake with tongues prohibit not. \V But let al things
be done honeſtly and according to order among you.


\stopChapter


\stopcomponent


%%% Local Variables:
%%% mode: TeX
%%% eval: (long-s-mode)
%%% eval: (set-input-method "TeX")
%%% fill-column: 72
%%% eval: (auto-fill-mode)
%%% coding: utf-8-unix
%%% End:

