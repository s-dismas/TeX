%%%%%%%%%%%%%%%%%%%%%%%%%%%%%%%%%%%%%%%%%%%%%%%%%%%%%%%%%%%%%%%%%
%%%%
%%%% The (original) Douay Rheims Bible 
%%%%
%%%% New Testament
%%%% Epistles
%%%% One Corinthians
%%%% Declaration
%%%%
%%%%%%%%%%%%%%%%%%%%%%%%%%%%%%%%%%%%%%%%%%%%%%%%%%%%%%%%%%%%%%%%%




\startcomponent declaration


\project douay-rheims

%%% !!! This actually appears in the middle of the annotations. After
%%% !!! 'in courſe.' and before 'Let women hold their peace.'


%%% 2705
%%% o-2560
\startArgument[
  title={\Sc{A More Ample Declaration of the Sense of this 14.~Chapter.}},
  marking={Declaration of C.~14.}
  ]

This
\MNote{That S.~Paul's place maketh nothing agaĩſt the ſeruice in the
  latin tõgue.}
then being the ſcope and direct drift of the Apoſtle, as is moſt cleere
by his whole diſcourſe, & by the record of al antiquitie: let the godly,
graue, & diſcret Reader take a taſt in this one point, of the
Proteſtants deceitful dealing, abuſing the ſimplicitie of the popular,
by peruerſe applicatiõ of God's holy word, vpõ ſome ſmal ſimilitude &
equiuicatiõ of certaine termes
%%% o-2561
againſt the approued godly vſe & truth of
the vniuerſal Church, for the ſeruice in the Latin or Greek tongue:
which they ignorantly, or rather wilfully, pretend to be againſt this
diſcourſe of S.~Paul touching ſtrange tongues.
\MNote{By ſtrange tongues the Apoſtle meaneth not the Latin Greek or
Hebrew.}
Know therfore, firſt, that here
\Fix{his}{is}{obvious typo, fixed in other}
no word written or meant of any other tongues but ſuch as men ſpake in
the Primitiue Church by miracle: & that nothing is meant of thoſe
tongues which were the common languages of the world or of the Faithful,
vnderſtood of the learned & ciuil people in euery great citie, & in
which the Scriptures of the Old or new Teſtament were written, as, the
Hebrew, Greek, and Latin. For though theſe alſo, might be giuen by
miracle & without ſtudy, yet being knowen to the Iewes, Romans, or
Greeks in euery place, they be not counted among the differences of
barbarous & ſtrange tongues here ſpoken of, which could not be
interpreted commonly, but by the miraculous guift alſo of
interpretation. And therfore this Apoſtle (as the Euangeliſts alſo and
others did their books) wrote his Epiſtles in Greek to the Romanes & to
al other Churches. Which when he wrote, though he penned them not in the
vulgar language peculiar to euery people, yet he wrote them not in
Tongue, that is, in any ſtrange tongue not intelligible without the
guift of interpretation, wherof he ſpeaketh here: but in a notable,
knowen, & learned ſpeach, interpretable of thouſands in euery countrie.
\MNote{S.~Auguſtin our Apoſtle brought in the Seruice in the Latin
tongue.}
No more did S.~Auguſtin our Apoſtle ſpeaking in Latin, & bringing in the
Scriptures & Seruice in Latin, preach & pray in Tongues according to the
Apoſtles meaning here. For the Latin was not, nor is not, in any part of
the Weſt, either miraculous or ſtrange, though it be not the National
tongue of any one countrie this day. And therfore S.~Bede ſaith,
\Cite{(li.~1. hiſt. Ang. c.~1.)}
that being then foure diuers vulgar languages in our countrie, the Latin
was made common to them al.
\MNote{The Latin ſeruice one and the ſame in al countries and ſtrãge to
none.}
And indeed of the two (though in truth
neither ſort be forbidden by this paſſage of S.~Paul) the barbarous
languages of euery ſeueral prouince in reſpect of the whole Church of
Chriſt, are rather the ſtrange tongues here ſpoken of, then the common
Latin tongue, which is vniuerſally of al the Weſt Church more or leſſe
learned, and pertaineth much more to vnitie and orderly coniunction of
al Nations in one faith, Seruice, and worship of God, then if it were in
the ſundry barbarous ſpeaches of euery Prouince. Wherin al Chriſtians
that trauel about this part of the world or the Indes either,
whereſoeuer they come, shal find the ſelf-ſame Maſſe, Mattins, &
Seruice, as they had at home.
\MNote{The ſeruice in vulgar tongues ſtrange & barbarous to euery
ſtranger.}
Where now if we goe to Germanie, or the Germans or Geneuians come to vs,
each others Seruice shal be thought ſtrange and barbarous. Yea and the
Seruice of our owne language within a few hundreth yeares (or rather
euery Age) shal wholy become barbarous and vnknowen to ourſelues; our
tongue (as al vulgar) doth ſo often change.

And
\MNote{Whether the ſeruice in vulgar tõgues doe more edifie.}
for edification, that is, for increaſe of faith, true knowledge, and
good life, the experience of a few yeares hath giuen al the world a ful
demonſtration whether our Forefathers were not as wiſe, as faithful, as
deuout, as fearful to breake God's lawes, & as likely to be ſaued, as we
are in al our tongues, tranſlations, & English praiers. Much vanitie,
curioſitie, contempt of Superiours, diſputes, emulations, contentions,
Schiſmes, horrible errours, profanatiõ & diuulgation of the ſecret
Myſteries of the dreadful Sacraments,
\CNote{See
\XRef{Annot. 1.~Cor.~10,~15.}}
which of purpoſe were hidden from the vulgar (as S.~Denys
\Cite{Eccl. Hier. c.~1.}
and S.~Baſil
\Cite{de Sp. Sancti. c.~27.}
teſtifie) are fallen by the ſame; but vertue or ſound knowledge none at
al.

Wherin
\MNote{The vertue of the Sacramẽts & Seruice conſiſteth not in the
peoples vnderſtanding.}
this alſo is a groſſe illuſion and vntruth, that the force and efficacie
of the
%%% 2706
Sacraments, Sacrifice, and common praier, dependeth vpon the peoples
vnderſtanding, hearing, or knowledge: the principal efficacie of ſuch
things & of the whole miniſterie of the Church, conſiſting ſpecially of
the very vertue of the worke, & the publike office of the Prieſts, who
be appointed in Chriſtes behalfe to diſpoſe the Myſteries to our moſt
good: the infant, innocent, idiote & vnlearned, taking no leſſe fruit of
Baptiſme & al other diuine offices, meet for euery ones condition, then
the learnedſt Clerke in the Realme: and more, if they be more humble,
charitable, deuout, and obedient, then the other, hauing leſſe of theſe
qualities and more learning.

Which
\MNote{The people is to be taught the meaning of Sacramẽts and
ceremonies, ãd are taught in al Catholike countries.}
we ſay not as though it were inconuenient for the people to be wel
inſtructed in the meaning of the Sacraments and holy ceremonies and
ſeruice of the Church (for that to their comfort and neceſſarie
knowledge, both by preaching, Catechizing, and reading of good Catholike
books, Chriſtian people doe learne in al Nations, much more in thoſe
countries were the Seruice is in Latin then in our Nation, God knoweth.)
But we 
ſay that there be other waies to inſtruct them, & the ſame leſſe ſubiect
to danger & diſorder, then to turne it into vulgar tongues. We ſay, the
ſimple people and many one that thinke themſelues ſome body, vnderſtand
as litle of the ſenſe of diuers Pſalmes, Leſſons, & Orations in the
vulgar tongue, as if they were in Latin, yea & often take them in a
wrong, peruerſe, & pernicious ſenſe, which lightly they could not haue
done in 
Latin. We ſay, that ſuch as would learne in deuotion and humilitie, may,
and muſt rather with diligence learne the tongue that ſuch Diuine things
be written in, or vſe other diligence in hearing ſermons & inſtructions,
then for a few mens not neceſſarie knowledge, the holy vniuerſal order
of God's Church should be altered. For if in the Kingdom of England only
it be not conuenient, neceſſarie, nor almoſt poſsible, to accomodate
their Seruice book to euery prouince & people of diuers tongues: how
much leſſe should the whole Church ſo doe conſiſting of ſo many
differences? Neither doth the Apoſtle in al this Chapter appoint any
ſuch
%%% o-2562
thing to be done, but admonisheth them to pray and labour for the grace
of vnderſtanding and interpretation, or to get others to interpret or
expound vnto them.
\MNote{Catholike people in euery coũtrie vnderſtandeth euery ceremonie,
and can behaue themſelues accordingly.}
And that much more may we doe concerning the Seruice in Latin, which is
no ſtrange nor miraculouſly gotten or vnderſtood tongue, but common to
the moſt & cheefe Churches of the world, and hath been, ſince the
Apoſtles time, daily with al diligence throughout al thoſe parts of
Chriſtendom, expounded in euery houſe, ſchoole, church, and pulpit: and
is ſo wel knowen for euery neceſſarie part of the diuine Seruice, that
by the diligence of parents, Maiſters, and Curates, euery Catholike of
age almoſt, can tel the ſenſe of euery ceremonie of the Maſſe, what to
anſwer, when to ſay \Emph{Amen} at the Prieſts benediction, when to
confeſſe, when to adore, when to ſtand, when to kneel, when to receiue,
what to receiue, when to come, when to depart, and al other dueties of
praying and ſeruing, ſufficient to ſaluation. And thus is it euident
that S.~Paul ſpeaketh not of the common tongues, of the Churches
Seruice.

Secondly,
\MNote{That he ſpeaketh not of the Churches ſeruice, is proued by
inuincible arguments.}
it is as certaine, that he meaneth not nor writeth any word in this
place of the Churches publike Seruice, praier, or miniſtration of the
holy Sacrament, wherin the office of the Church ſpecially conſiſteth:
but only of a certaine exerciſe of mutual conference, wherin one did
open to another and to the aſſemblie, miraculous guifts and graces of
the Holy Ghoſt, and ſuch Canticles, Pſalmes, ſecret Myſteries, ſorts of
languages, and other Reuelations, as it pleaſed God to giue vnto
certaine both men and women in that firſt beginning of his Church. In
doing of this, the Corinthians committed many diſorders, turning Gods
guifts to pride and vanitie, and namely that guift of tongues: which
being indeed the leaſt of al guifts, yet moſt puffed vp the hauers, and
now alſo doth commõly puffe vp the Profeſſours of ſuch knowledge,
according as
\CNote{\Cite{Aug. doct. Chr. li.~2. c.~13.}}
S.~Auguſtin writeth therof. This exerciſe and the diſorder therof was
not in the Church (for any thing we can read in antiquitie) theſe
fourteen hundreth yeares: and therfore neither the vſe nor abuſe, nor
S.~Paules reprehenſion or redreſsing therof, can concerne any whit the
Seruice of the Church. Furthermore this is euident, that the Corinthians
had their Seruice in Greek at this ſame time, and it was not done in
theſe miraculous tongues. Nothing is meant then of the Church
Seruice. Againe the publike Seruice had but one language: in this
exerciſe they ſpake in many tongues. In the publike Seruice euery man
had not his owne ſpecial tongue, his ſpecial Interpretation, ſpecial
Reuelation, proper Pſalmes: but in this they had. Againe the publike
Seruice had in it the miniſtration of the Holy Sacrament principally:
which was not done in this time of conference. For into this exerciſe
were admitted Cathechumẽs, and Infidels, & whoſoeuer would: in this womẽ
%%% 2707
before S.~Paules order, did ſpeake and prophecie: ſo did they neuer in
the Miniſtration of the Sacrament: with many other plaine
differences; that by no meanes the Apoſtles words can be rightly &
truely applied to the Corinthians Seruice then, or ours now. Therfore
it is either great ignorance of the Proteſtants, or great guilfulnes, ſo
vntruely and peruerſly to apply them.

Neither
\MNote{The Apoſtle ſpeaketh not of the peoples priuate praiers in latin,
as vpõ primars, beades, or otherwiſe.}
is here any thing meãt of the priuate praiers which deuout perſõs of al
ſorts & ſexes haue euer vſed, ſpecially in Latin, as wel vpõ their
primars as Beads. For, the priuate praiers here ſpoken of, were pſalmes
or hymns and ſonnets newly inſpired to them by God, & in this conference
or prophecying, vttered to anothers comfort, or to thẽſelues and God
only. But the praiers, pſalmes, and holy words of the Chriſtian people
vſed priuately, are not compoſed by them, nor diuerſly inſpired to
themſelues, nor now to be approued or examined in the aſſemblies: but
they are ſuch as were giuen and written by the Holy Ghoſt, and
preſcribed by Chriſt and his Church for the faithful to vſe, namely the
\Emph{Pater noſter}, the \Emph{Ave Maria}, and the \Emph{Creed}, our
\Emph{Ladies Mattins}, the \Emph{Litanies}, & the like. Therfore the
Apoſtle preſcribeth nothing here therof, condemneth nothing therin,
toucheth the ſame nothing at al. But the deuout people in their ancient
right may and ought ſtil vſe their Latin primars, beades, and praiers,
as euer before.
\MNote{Latin praiers tranſlated, or the people taught the contents
therof.}
Which the wiſedom of the Church for great cauſes hath better liked and
allowed of then that they ſhould be in vulgar tongues, though ſhe
wholy forbiddeth not, but ſometimes granteth to haue them tranſlated; and
would gladly haue al faithful people in order and humilitie learne, as
they may, the contents of their praiers: and hath commanded alſo in ſome
Councels, that ſuch as can not learne diſtinctly in Latin (ſpecially the
\Emph{Pater noſter} and the \Emph{Creed}) ſhould be taught them in the
vulgar tongue. And therfore as we doubt not but it is acceptable to God,
and auailable in al neceſſities, and more agreable to the vſe of al
Chriſtian people euer ſince their conuerſion, to pray in Latin, then in
the vulgar, though euery one in particular, vnderſtand not what he
ſaith:
\MNote{The peoples deuotion nothing the leſſe for praying in Latin.}
ſo it is plaine that ſuch pray with as great conſolation of
ſpirit, with as litle tediouſnes, with as great deuotion and affection,
and oftentimes more, then the other: and alwaies more then any
Schiſmatike or Heretike in his knowen language. Such holy Oraiſons be in
manner
%%% o-2563
conſecrated & ſanctified in and by the Holy Ghoſt that firſt
inſpired them; and there is a reuerence & Maieſtie in the Churches
tongue dedicated in our Sauiours Croſſe, & giueth more force & value to
them ſaid in the Churches obedience, then to others. The children cried
\CNote{\XRef{Mat.~24.}}
\Emph{Hoſanna} to our Sauiour, and were allowed, though they knew not
what they ſaid.
\MNote{The ſeruice alwaies in Latin throughout the weſt Church.}
It is wel neer a thouſand yeares that
\CNote{\Cite{Greg. li.~27. Moral c.~6.}}
our people which could nothing els but \L{barbarum frendere}, did ſing
\Emph{Alleluya}, & not, \Emph{Praiſe ye the Lord}; & longer agoe ſince
the poore husband-men ſang the ſame at the plough in other countries.
\Cite{Hiero. to.~1, ep.~5.}
And \L{Surſum corda}, and \GG{Kyrie eleiſon}, and the Pſalmes of Dauid
ſung in Latin in the Seruice of the Primitiue Church, haue the ancient &
flat teſtimonies of S.~Cyprian, S.~Auguſtin, S.~Hierom and other
Fathers.
\Cite{Grego. li.~7. ep.~63.}
\Cite{Cyp. exp. orat. do. nu.~13.}
\Cite{Aug. c.~13. de bono perſeuer. & de bono vid. c.~16.}
and
\Cite{ep.~178.}
\Cite{Hiero. præfat. in Pſal. ad Sophron.}
\Cite{Aug. de Catechiz. rud. c.~9.}
\Cite{de Doct. Chr. li.~2. c.~13.}
See
\Cite{ep.~10. of Auguſt of S.~Hieroms Latin tranſlation}
read in the Churches of Africa. Praiers are not made to teach, make
learned, or increaſe knowledge, though by occaſion they ſometimes
inſtruct vs: but their ſpecial vſe is, to offer our harts, deſires, and
wants to God, and to ſhew that we hang of him in al things: and this
euery Catholike doth for his condition, whether he vnderſtand the words
of his praier of not. The ſimple ſort can not vnderſtand al Pſalmes, nor
ſcarſe the learned, no though they
\Fix{be be}{be}{obvious typo, fixed in other}
tranſlated or read in knowen tongues: men muſt not ceaſe to vſe them for
al that, when they are knowen to containe God's holy praiſes.
\MNote{It is not neceſſary to vnderſtand our praiers.}
The ſimple people when they deſire any thing ſpecially at Gods hand, are
not bound to know, neither can they tel, to what petition or part of the
\Emph{Pater noſter} their demand pertaineth, though it be in Engliſh
neuer ſo much. They can not tel no more what is, \Emph{Thy kingdõ come},
then \L{Adveniat regnum tuum}; nor whether their petition for their
ſicke children or any other neceſſitie pertaine to this part or to \L{Fiat
volutas tua}, or \L{Ne nos inducas}, or to what other part els. It is
enough that they can tel, this holy Oraiſon to be appointed to vs, to
cal vpon God in al our deſires: more then this, is not neceſſarie.
\MNote{How farre is ſufficẽt for the people to vnderſtand.}
And the tranſlation of ſuch holy things often breedeth manifold danger
and irreuerence in the vulgar (as to thinke God is authour of ſinne,
when they read \Emph{Lead vs not into tentation}) and ſeldom any
edification at al. For though when the praiers be turned and read in
Engliſh, the people knoweth the words, yet they are not edified to the
inſtruction of their mind and vnderſtanding, except they knew the ſenſe
of
%%% 2708
the words alſo & meaning of the Holy Ghoſt.
\MNote{How the mind or vnderſtanding is edified.}
For if any mã thinke that S.~Paul ſpeaking of edification of man's mind
or vnderſtanding, meaneth the vnderſtanding of the words only, he is
fouly deceiued. For, what is a child of fiue or ſixe yeares old edified
or increaſed in knowledge by his \Emph{Pater noſter} in English? It is
the ſenſe therfore, which euery man can not haue, neither in English nor
Latin, the knowledge wherof properly and rightly edifieth to
inſtruction: and the knowledge of the words only, often edifieth neuer a
whit, and ſometimes buildeth to errour and deſtruction: as it is plaine
in al Heretikes and many curious perſons beſides. Finally both the one
and the other without charitie and humilitie maketh the Heretikes and
Schiſmatikes with al their English and what other tongues and
intelligence ſoeuer, to be \L{æs sonans & cymbalum tinniens}, ſounding
braſſe and a tinkling cymbal.

To conclude, for praying either publikly or priuately in Latin which is
the common ſacred tongue of the greateſt part of the Chriſtian world,
this is thought by the wiſeſt & godlieſt to be moſt expedient, and is
certainely ſeen to be nothing repugnant to S.~Paul. If any yet wil be
contentious in the matter, we muſt anſwer them with this ſame Apoſtle:
\CNote{1.~Cor.~13.}
\Emph{The Church of God hath no ſuch cuſtome}; and with this notable
ſaying of S.~Auguſtin,
\Cite{ep.~118. c.~1.}
\MNote{A notable rule of S.~Auguſtin.}
\Emph{Any thing that the whole Church doth practiſe and obſerue
throughout the world, to diſpute therof as though it were not to be
done, is moſt inſolent madneſſe.}


\stopArgument


\stopcomponent


%%% Local Variables:
%%% mode: TeX
%%% eval: (long-s-mode)
%%% eval: (set-input-method "TeX")
%%% fill-column: 72
%%% eval: (auto-fill-mode)
%%% coding: utf-8-unix
%%% End:

