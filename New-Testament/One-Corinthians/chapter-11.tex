%%%%%%%%%%%%%%%%%%%%%%%%%%%%%%%%%%%%%%%%%%%%%%%%%%%%%%%%%%%%%%%%%
%%%%
%%%% The (original) Douay Rheims Bible 
%%%%
%%%% New Testament
%%%% One Corinthians
%%%% Chapter 11
%%%%
%%%%%%%%%%%%%%%%%%%%%%%%%%%%%%%%%%%%%%%%%%%%%%%%%%%%%%%%%%%%%%%%%




\startcomponent chapter-11


\project douay-rheims


%%% 2693
%%% o-2548
\startChapter[
  title={Chapter 11}
  ]

\Summary{He
\MNote{The 5.~part.

Of his Traditions.}
commendeth them for keeping his traditions generally. 3.~And in
particular for this that a man praied and prophecied bare-headed, a
woman veiled, he bringeth many reaſons. 17.~About another, he
reprehendeth the rich, that at the charitable ſupper ſupped
vncharitably: 23.~telling them that they receiued therfore vnworthily
the B.~Sacrament, and shewing them what an heinous ſinne that is, ſeeing
it is our Lordes body and the repreſentation of his death, as he by
tradition had taught them.}

Be ye followers of me, as I alſo of Chriſt. \V And I praiſe you
Brethren, that in al things you be mindful of me: and as I haue
deliuered vnto you, you keep
\LNote{My precepts.}{Our Paſtours and Prelates haue authoritie to
command, and we are bound to obey. And the Gouerners of the Church may
take order and preſcribe that which is comely in euery ſtate, as time
and place require, though the things be not of the ſubſtance of
religion.}
my
\TNote{In the Greek, \Emph{Traditions}, \G{παραδόσεις}}
precepts.

\V And I wil haue you know, that the head of euery man, is Chriſt: and
the head of the woman, is the man: and the head of Chriſt, is God.
%%% o-2549
\V Euery man praying or prophecying with his
%%% 2694
head couered, diſhoneſteth his head. \V But
\LNote{Euery woman.}{What guifts of God ſoeuer women haue, though
ſupernatural, as ſome had in the primitiue Church, yet they may not
forget their womanly shamefaſtnes, but shew themſelues ſubiect and
modeſt, and couer their heads with a veile.}
euery woman praying or prophecying with her head not couered,
diſhoneſteth her head: for it is al one as if ſhe were made bald. \V For
if a woman be not couered, let her be polled. But if it be a foule thing
for a woman to be polled or made bald, let her couer her head. \V The man
truely ought not to couer his head, becauſe he is the image and glorie
of God; but the woman is the glorie of the man. \V For the man is not of
the woman, but the woman of the man. \V For
\CNote{\XRef{Gen.~1,~21.}}
the man was not created for the woman, but the woman for the man. \V
(Therfore ought the woman to haue power vpon her head for the Angels.) \V
But yet neither the man without the woman; nor the woman without the
man, in our Lord. \V For as the woman is of the man, ſo alſo the man by
the woman: but al things of God. \V Your ſelues iudge: doth it become a
woman not couered to pray vnto God? \V Neither doth nature itſelf teach
you, that a man indeed if he nouriſh his haire, it is an ignominie for
him: \V But if a woman nouriſh her haire, it is a
\Fix{gIorie}{glorie}{obvious typo, fixed in other}
for her, becauſe haire is giuen her for a veile? \V But if any man ſeeme
to be contentious, we haue no ſuch
\LNote{Cuſtome.}{If
\MNote{The Cuſtome of the Church, is a good anſwer againſt al
wranglers.}
women or other, to defend their diſorder & malapertnes, diſpute or
alleage Scriptures and reaſons, or require cauſes of their Preachers why
and by what authoritie they ſhould be thus reſtrained in things
indifferent, make them no other anſwer but this: This is the cuſtome of
the Church, this is our cuſtome. Which is a goodly rule to repreſſe the
ſaucineſſe of contentious ianglers, which being out of al modeſtie and
reaſon, neuer want wordes and replies againſt the Church. Which Church
if it could then by preſcription of twenty or thirty yeares, and by the
authoritie of one or two of their firſt Preachers, ſtop the mouthes of
the ſeditious: what ſhould not the cuſtome of fifteen hundred yeares,
and the decrees of many hundred Paſtours, gaine of reaſonable, modeſt,
and humble men.}
cuſtome, nor the
%%% !!! Var only in other, both have Small Caps, but that seems wrong.
\Var{Church}{Churches}
of God.

\V And this I command: not praiſing it, that you come together not to
better, but to worſe. \V Firſt indeed when you come together into the
Church, I heare that there are ſchiſmes among you, and in part I beleeue
it. \V For
\LNote{There muſt be hereſies.}{When
\MNote{That hereſies shal come, and wherfore.}
the Apoſtle ſaith: \Emph{Hereſies muſt be}, he sheweth the euent, and
not that God hath directly ſo appointed it as neceſſarie. For, that they
be, it commeth of man's malice & free-wil; but that they be conuerted to
the manifeſtation of the good and conſtant in faith & the Churches
vnitie, that is God's ſpecial worke of prouidence that worketh good of
euil. And for that there ſhould fal Hereſies and Schiſmes, ſpecially
concerning the Article and vſe of the B.~Sacrament if the Altar, whereof
he now beginneth to treat, it may make vs maruel the leſſe, to ſee ſo
great diſſenſions, Hereſies, and Schiſmes of the wicked and weake in
faith concerning the ſame. Such things wil be, but woe to him by hwom
ſcandals or Sectes doe come.
\MNote{What commoditie we may make of hereſies.}
\Emph{Let vs vſe Heretikes}, ſaith S.~Auguſtin, \Emph{not to that end to
approue their errours, but that be defending the Catholike doctrine
againſt their deceits, we may be more watchful and wary: becauſe it is
moſt truely written, There muſt be hereſies that the tried & approued
may be manifeſted or diſcouered from the holow harts among you. Let vs
vſe this benefit of God's prouidence. For Heretikes be made of ſuch as
would erre or be naught, though they were in the Church: but being out,
they profit vs exceedingly, not by teaching the truth which they know
not, but by ſtirring vp the carnal in the Church to deeke truth, and the
ſpiritual Catholike, to cleere the truth. For there be innumerable holy
approued men in the Church, but they be not diſcerned from other among
vs, nor manifeſt, ſo long as we had rather ſleep in darknes of
ignorance, then behold the light of truth. Therfore many are raiſed out
of their ſleep by Heretikes to ſee the day of God, and are glad therof.
\Cite{Auguſt. c.~8. de vera relig.}}}
there muſt be hereſies alſo: that they alſo which are approued, may be
made manifeſt among you. \V When you come therfore together in one, it
is not now to eate
\LNote{Our Lordes ſupper.}{The
\MNote{Agapæ or ſuppers of charitie.}
Chriſtians at or about the time of the Churches only Sacrifice & their
communicating therof, kept great feaſts, which continued long, for that
the reliefe of the poore vpon the common charges of the richer ſort, and
the charitie and vnitie of al ſorts were much preſerued thereby, 
\CNote{\Cite{Conc. Gang. can.~11.}
\Cite{Con. Laodic. can.~27,~28.}}
for which cauſe they were called \G{ἀγάπαις}, that is, \Emph{Charities},
of the ancient Fathers, and were kept commonly in Church-houſes or
porches adioyning, or in the body of the Church (wherof ſee Tertullian,
\Cite{Apolog. c.~39.}
Clemens Alexand, S.~Iuſtine, S.~Auguſtin
\Cite{cont Fauſt. li.~20. c.~20.})
after the Sacrifice and Communion was ended, as S.~Chryſoſtom
\Cite{ho.~17. in 1.~Cor. in initio}
iudgeth. Thoſe feaſts S.~Paul here calleth \L{cœnas Dominicas}, becauſe
they were made in the Churches which then were called \L{Dominicæ}, that
is, \Emph{Our Lordes houſes}. The diſorder therfore kept among the
Corinthians in theſe Church-feaſts of Charitie, the Apoſtle ſeeketh here
to redreſſe, from the foul abuſes expreſſed here in the text.
\MNote{Whether the Apoſtle meane by \Emph{our Lord's ſupper}, the
B.~Sacrament.}
And as S.~Ambroſe
\Cite{in hunc locum,}
and moſt good Authours now thinke, this which he calleth \L{Dominicam
cœnam}, is not meant of the B.~Sacrament, as the circumſtances alſo of
the text doe giue, namely, the reiecting of the poore, the rich mens
priuate deuouring of al, not expecting one another, gluttony and
drunkeneſſe in the ſame, which can not agree to the Holy Sacrament. And
therfore the Heretikes haue ſmal reaſon, vpon this place, to name the
ſaid Holy Sacrament, rather, \Emph{the Supper of the Lord}, then after
the manner of the primitiue Church, the \Emph{Euchariſt}, \Sc{Masse}, or
\Emph{Lyturgie}. But by like they would bring it to the ſupper againe or
Euening ſeruice, when men be not faſting, the rather to take away the
old eſtimation of the holines therof.}
our Lordes ſupper. \V For euery one taketh his owne ſupper before to
eate. And one certes is an hungred, and another is drunke. \V Why, haue
you not houſes to eate and drinke in? or contemne ye the Church of God:
and confound them that haue not? What ſhal I ſay to you? praiſe I you in
this? I doe not praiſe you.

\V 
\CNote{\XRef{Mt.~26,~26.}
\XRef{Mr.~14,~22.}
\XRef{Lu.~22,~19.}}
For I receiued of our Lord that which alſo
\LNote{I haue deliuered.}{As
\MNote{Traditiõ without writing.}
al other parts of religiõ were firſt deliuered by preaching & word of
mouth to euery Nation conuerted, ſo this holy order and vſe of the
B.~Sacrament was by S.~Paul firſt giuen vnto the Corinthians by
tradition. Vnto which as receiued of our Lord he reuoketh them by this
Epiſtle, not putting in writing particularly al things pertaining to the
order, vſe, and inſtitution, as he afterward ſaith: but repeating the
ſumme and ſubſtance therof, and leauing the reſidue to his returne.
\MNote{Wether the Catholikes or Proteſtãts doe more imitate Chriſts
inſtitution of the B.~Sacrament.}
But his words and narration here written we wil particularly proſecute,
becauſe the Heretikes make profeſſion to follow the ſame in their
pretended reformation of the Maſſe.}
I haue deliuered vnto you,
\SNote{The Apoſtles drift in al that he ſaith here of the Sacrament, is
againſt vnworthy receiuing (as S.~Auguſtine noteth
\Cite{Ep.~118. c.~3.})
and not to ſet out the whole order of miniſtratiõ, as the heretikes doe
ignorãtly imagine.}
that our Lord \Sc{Iesvs}
\LNote{In the night.}{Firſt
\MNote{Al circumſtances in our Sauiour's action about the B.~Sacrament
need not be imitated.}
the Aduerſaries may be here conuinced that al the circumſtances of time,
perſon, & place which in Chriſtes action are noted, need not to be
imitated. As that the Sacrament ſhould be iminſtred at night, to men
only, to only twelue, after of at ſupper, & ſuch like: becauſe (as
S.~Cyprian
\Cite{ep.~43. nu.~7.}
& S.~Auguſtin
\Cite{ep.~118. c.~6.}
note) there were cauſes of thoſe accidents in Chriſt that are not now to
be alleaged for vs. He inſtituted then this holy act: we doe not. He
made his Apoſtles Prieſts, that is to ſay, gaue them cõmiſſion to doe &
miniſter the ſame: we doe not. He would haue this the laſt act of his
life & within the bounds of his Paſſion: it is not ſo with vs. He would
eate & make an end of the Paſchal to accõpliſh the old Law: that can not
be in our action. Therfore he muſt needs doe it after ſupper and at
night: we may not doe ſo. He excluded al women, al the reſt of his
Diſciples, al lay men: we inuite al faithful, men & women. In many
circumſtances then, neither we may imitate Chriſtes firſt action, nor
the Heretikes as yet doe: though they ſeem to encline by abandoning
other names ſauing this (calling it Supper) to haue it at night & after
meate: though (as is before noted) they haue no iuſt cauſe to cal it ſo
vpon Chriſts fact, seeing the Eunageliſts doe plainely ſhew
\CNote{\XRef{Io.~13,~2.}}
that the Sacrament was inſtituted after Supper, as the Apoſtle himſelf
here recordeth of the later part in expreſſe ſpeach. And moſt men
thinke, a long ſermon and the washing of the Apoſtles feet came between;
yea and that the ſupper was quite finiſhed & grace ſaid. But in al theſe
and ſuch like things, the Catholike Church only, by Chriſtes ſpirit can
tel, which things are imitable, which not, in al is actions.}
in the night that he was betraied,
\LNote{Tooke.}{Chriſt
\MNote{The Proteſtãts imitate not Chriſt in bleſſing the bread and wine.}
took bread into his hands, applying this ceremonie, action, and
benediction to it, & did bleſſe the very element, vſe dpower & actiue
words vpon it
\CNote{\XRef{Luc.~9,~26.}}
as he did ouer the bread & fiſhes which he multiplied: and ſo doth the
Church of God: and ſo doe not the Proteſtants, if they follow their owne
book & doctrine; but they let the bread & cup ſtand aloofe, & occupie
Chriſtes wordes by way of report & narration, applying them not al al to
the matter propoſed to be occupied: and therfore, howſoeuer the ſimple
people be deluded by the reherſel of the ſame wordes which Chriſt vſed,
yet conſecration, benediction, or ſanctification of bread an wine they
profeſſe they make none at al. At the firſt alteration of religion,
there was a figure of the Croſſe at this word, \Emph{He bleſſed}; and at
the word, \Emph{He tooke}, there was a gloſſe or rubrike that appointed
the Miniſter to imitate Chriſt's action, & to take the bread into his
hands: afterward that was reformed and Chriſt's action aboliſhed, and
his bleſſing of bread turned to thankes-giuing to God.}
tooke
\LNote{Bread.}{Chriſt
\MNote{They imitate his not in vnleauened bread, and mingling water with
wine.}
made the holy Sacrament of vnleauened bread, & al the Latin Church
imitateth him in the ſame as a thing much more agreable to the
ſignificatiõ both in itſelf & in our liues, then the leauen. Yet our
Aduerſaries neither follow Chriſt, S.~Paul, nor the VVeſt Church in the
ſame: but rather purpoſely make cholſe of that kind that is in itſelf
more vnſeemly, & to the firſt inſtitution leſſe agreable. In the other
part of the Sacrament they contemne Chriſt and his Church much more
impudently and damnably. For Chriſt and al the Apoſtles & al Catholike
Churches in the world haue euer mixed their wine with water, for great
myſterie & ſignification, ſpecially for that water guſhed together with
bloud out of
\Fix{or}{our}{obvious typo, fixed in other}
Lordes ſide. \Emph{This our Lord did} (ſaith S.~Cyp.
\Cite{Ep.~63. ad Cecil. nu.~47.})
\Emph{and none rightly offereth, that followeth not him therin.} Thus
Irenæus
\Cite{(ho.~1. c.~1.)}
Iuſtine
\Cite{(Apolog.~2. in fine.)}
& al the Fathers teſtifie the Primitiue Church did; and in this ſort it
is done in al the \Sc{Masses} of the Greeks. S.~Iames, S.~Baſil,
S.~Chryſoſtom's. And yet our Proteſtants pretending to reduce al to
Chriſt, wil not die as he did, and al the Apoſtles and Churches that
euer were.}
bread: \V and giuing thankes brake, and ſaid:
%%% !!! This appears out of order in both, after 'This doe'. Does this
%%% matter? Do we care?
\LNote{Take and eate.}{This pertaineth to the receiuing of thoſe things
which by the conſecration are preſent and ſacrificed before: as when the
people or Prieſts in the old Law did eate the Hoſts offered or part
therof, they were made partakers of the Sacrifice done to God before.
\MNote{The Sacramẽt conſiſteth not in the receiuing.}
And this is not the ſubſtance, or being, or making of the Sacrament or
Sacrifice of Chriſtes body and bloud: but it is the vſe and application
to the receiuer of the things that were made and offered to God
before. There is a difference betwixt the making of a medicine or the
ſubſtãce and ingredients of it, and the taking of it. Now the receiuing
being but a conſequence or one of the ends why the Sacrament was made,
and the meane to apply it vnto vs:
\MNote{Why the Proteſtants cal it the Communion.}
the Aduerſaries vnlearnedly make it al & ſome, & therfore improperly
name the whole Sacrament & miniſtration therof, by calling it
Communioin. Which name they giue alſo rather then any other, to make the
ignorant beleeue that many muſt communicate together: as though it were
ſo called for that it is common to many. By which colluſion they take
away the receiuing of the Prieſt alone, of the ſicke alone, of reſeruing
the conſecrated Hoſt and the whole Sacrament.
\MNote{Communion which is a part of the \Sc{Masse}, what it ſignifieth.}
Againſt which deceit, know that this part of the \Sc{Masse} is not
called communion, for that many should concurre together alwaies in the
external Sacrament: but for that we doe communicate or ioyne in vnitie
and perfect fellowship of one body, with al Chriſtian men in the world,
with al (we ſay) that eate it through the whole Church and not with them
only which eate with vs at one time. And this fellowship riſeth of that,
that we be, euery time we receiue either alone or with companie,
partakers of that one body which is receiued throughout al the world.
\Emph{It is al called Communion} (ſaith
\CNote{\Cite{li.~4. c.~14. de orrhod fide.}}
S.~Damaſcene) \Emph{& ſo indeed it is, for that by it we communicate
with Chriſt, & be partakers of his flesh & diuinitie, & by it doe
communicate and are vnited one with another. Only let vs take heed that
we doe not participate with heretikes.} And when the Apoſtle ſaith, that
al be one bread and one body that are partakers of one bread, he meaneth
not of them only that communicate at one time and place: but that al be
ſo, that communicate in vnitie through the whole Church. Then the name
Communion is as ignorantly vſed of them as the name of Supper.}
Take ye & eate,
\LNote{This is.}{Theſe
\MNote{The wordes of conſecration, to be ſaid ouer the bread and wine,
the which the Proteſtants doe not.}
words being ſet downe, not in the perſon of the Euangeliſtes or
Apoſtles, but expreſſed as in Chriſtes owne perſon, to be ſaid ouer the
bread, and the like ouer the wine, are the formes of the Sacrament and
words of conſecration: neither is it a Sacrament but (as
\CNote{\Cite{Tract.~80. in Io.}}
S.~Auguſtin
ſaith) when the words come, that is to ſay, actiuely and preſently be
applied to the elements of the ſame. Therfore the Proteſtants neuer
applying theſe words more then the whole whole narration of the
inſtitution, nor reciting the whole (as is ſaid) otherwiſe then in
hiſtorical manner, (as if
\Fix{owne}{one}{obvious typo, fixed in other}
would miniſter Baptiſme and neuer apply the words of the Sacrament to
the child, but only read Chriſtes ſpeaches of the ſame) make no
Sacrament at al. And that theſe proper words be the only forme of this
Sacrament, and ſo to be ſpoken ouer or vpon the bread and wine,
S.~Ambroſe plainly and preciſely writeth, recording how farre the
Euangeliſts narratiue words doe goe, and where Chriſtes wone peculiar
myſtical words of conſecration begin: and ſo the reſt of the Fathers.
\Cite{Ambroſ. li.~4. de Sacram. c.~4.}
&
\Cite{c.~9. de init. Myſter.}
\Cite{Iuſti. Apolog.~2. in fine.}
\Cite{Cyprian. de Cæn Dom. num.~1.~2.}
\Cite{Auguſt. Serm.~28. de verb Dom. ſec. Mat.}
\Cite{Tertull. li.~4. cont. Marc.}
\Cite{Chryſoſt. ho.~2. in 2.~ad Tim. in fine.}
&
\Cite{hom. de prodis. Iudæ. to.~3.}
\Cite{Gregor. Nyſſ. in orat. Catech. Damaſc. li.~4. c.~14.}}
\Sc{This is
\LNote{My body.}{When
\MNote{The Proteſtants haue taken away the B.~Sacrament altogether.}
the words of Conſecration be by the ſaid impietie of the Proteſtants,
thus remoued from the element, no maruel if Chriſtes holy body and bloud
be not there, or that it is now no more a Sacrament, but common bread
and wine. So they that vniuſtly charge the Catho. Church with defrauding
the people of one peece of the Sacramẽt, haue in very deed left no part
nor ſpice of Sacramẽt, niether following Chriſt as they pretend, nor
S.~Paul, nor any Euangeliſt, but their owne deteſtable Sect, hauing
boldly defaced the whole inſtitution, not in any  accidental indifferent
circumſtances, but in the very ſubſtance and al. The right name is gone,
the due elements both gone, no bleſsing or conſecration, or other action
ouer them, the formes be gone: and conſequently the body and bloud, the
Sacrament and the Sacrafice.}
my Body which shal be delivered for yov.}
\LNote{This doe.}{By
\MNote{The power to conſecrate giuen to Prieſts only.}
theſe words, authoritie and power is giuen to the Apoſtles, and by the
like in the Sacrament of Orders, to al lawful Prieſts only. No maruel
then that the new heretical Miniſters being Lay-men, giue the people
nothing but bare bread and wine, profane, naked, and natural elements
void of Sacrament and al grace. See the
\XRef{Annotation vpon S.~Luke chap.~22,~19.}}
This doe ye for
%%% o-2550
the commemoration of me. \V In like manner alſo the
chalice after he had ſupped, ſaying: \Sc{This Chalice is the new
Testament in my Blovd.} This doe ye, as often as you ſhal drinke, for
the commemoration of me. \V For as often as you ſhal eate this bread,
and drinke the chalice,
\LNote{You shal shew.}{Vpon
\MNote{How Chriſts death is ſhewed by the B.~Sacrament itſelf, without
ſermon or otherwiſe.}
this word the Heretikes fondly ground their falſe ſuppoſition, that this
Sacrament can not rightly be miniſtred or made without a ſermon of the
death of Chriſt: and that this and other Sacraments in the Church be not
profitable, when they be miniſtred in a ſtrange language. As though the
grace, forcce, operation, & actiuitie, together with the inſtruction &
repreſentation or the things which they ſignifie, were not in the very
ſubſtance, matter, forme, vſe, and worke itſelf of euery of the
Sacraments: and as though preaching were not one way to shew Chriſtes
paſſion, and the Sacraments another way: namely this Sacrament,
conteining in the very kinds of the elements and the action, a moſt
liuely repreſentation of Chriſtes death. As wiſely might they ſay that
neither Abel's Sacrifice, nor the Paſchal lamb could ſignifie Chriſtes
death without a Sermon.}
you ſhal ſhew the death of our Lord, vntil he come. \V Therfore
whoſoeuer ſhal eate this bread, or drinke the chalice of our Lord
vnworthily, he ſhal be
\LNote{Guilty of the body.}{Firſt
\MNote{The wicked receiue the body & bloud.}
herupon marke wel, that il men receiue the body and bloud of Chriſt, be
they infidels or il liuers. For in this caſe they could not be guilty of
that which they receiue not. Secondly, that it could not be ſo heinous
an offenſe for any man to receiue a peece of bread or a cup of wine,
though they were a true Sacrament. For it is a deadly ſinne to receiue
any Sacrament with wil & intention to continue in ſinne, or without
repentance of former ſinnes:
\MNote{The real preſence is proued by the heinous offẽſe of vnworthy
receiuing.}
but yet by the vnworthy receiuing of no other Sacrament is man guilty of
Chriſtes body and bloud, but here where the vnworthy (as S.~Chryſoſtom
ſaith) doth vilany to Chriſtes owne perſon, as the Iewes or Gentils did,
that crucified it.
\Cite{Chryſ. ho. de non contemn. Ec.}
&
\Cite{Ho.~60. &~61. ad po Antioch.}
Which inuincibly proueth againſt the Heretikes that Chriſt is really
preſent.}
guilty of the body and of the bloud of our Lord. \V But
\LNote{Let him proue.}{A
\MNote{Confeſsion before receiuing the B.~Sacrament.}
man muſt examine his life diligently whether he be in any mortal ſinne,
and muſt confeſſe himſelf of euery offenſe which he knoweth or feareth
to be deadly, before he preſume to come the Holy Sacrament. For ſo the
Apoſtles doctrine here with the continual cuſtom of the Cath. Church and
the Fathers example, bind him to doe.
\Cite{Cyp. de lapſ. nu.~7.}
\Cite{Aug. Eccl. dog. c.~53.}}
let a man proue himſelf: and ſo, let him eate of that bread, and drinke
of the chalice. \V For he that eateth and drinketh vnworthily, eateth
and drinketh iudgement to himſelf,
\LNote{Not diſcerning the body.}{That
\MNote{Adoration of the B.~Sacrament.}
is, becauſe he putteth no difference nor diſtinction betwixt this high
meate and others: and therfore S.~Auguſtin ſaith
\Cite{ep.~118. c.~3.}
\Emph{That it is he that the Apoſtle ſaith shal be damned, that doth not
by ſingular veneratiõ or adoratiõ make a differẽce between this meate
and al others.} And againe
\Cite{in Pſal.~98.}
\Emph{No man eateth it before he adore it.} And
\Cite{li.~3. c.~12. de Sp. San.}
\Emph{We adore the flesh of Chriſt in the Myſterie.} S.~Chryſoſt.
\Cite{ho.~24. in 1.~Cor.}
\Emph{We adore him on the altar, as the Sages did in the manger.}
S.~Nazianzene
\Cite{in Epitaph Gorgoniæ.}
\Emph{My ſiſter called on him which is worshipped vpon the altar.}
Theodorete
\Cite{Dial.~2. Inconſ.}
\Emph{The myſtical tokens be adored.} S.~Denys, this Apoſtles ſcholer,
made ſolemne inuocation of the Sacrament after Conſecration.
\Cite{Eccleſiaſt. Hier. c.~3. part.~3. in princep.} and before the
receiuing, the whole Church of God crieth vpon it,
%%% !!! Marked in both, used only in other
\CNote{See the
\XRef{Annot. Mat.~8,~8.}}
\L{Domine non ſum dignus, Deus propitius eſto mihi peccatori},
%%% !!! Wrong translation. Do we care?
\Emph{Lamb of God that takeſt away the ſinnes of the world, haue mercie
on vs.}
\MNote{The manifeſt honour and diſcerning of Chriſtes body in the
Cath. Church.}
And for better diſcerning of this diuine meate, we are called from
common prophane howſes to God's Church: for this we are forbidden to make
it in vulgar apparel, and are appointed ſacred ſolemne veſtiments.
\Cite{Hiero. in Epitaph Næpot.}
&
\Cite{li.~2. adu. Pela. c.~9.}
\Cite{Paulinus ep.~12. ad Scuer.}
\Cite{Io. Diac. in vit. D.~Greg. li.~3.~59.}
For this, is the halowing of Corporals and Chalices,
\Cite{Ambr. 2.~Oſſ. c.~28.}
\Cite{Nazianz. Orat. ad Arianus. Optatus li.~6. in initio.}
%%% !!! Marked only in this, but Cite belongs to above CNote.
%%% \CNote{See the
%%% \XRef{Annot. Mat.~8,~8.}}
For this, profane tables are remoued & altars conſecrated.
\Cite{Aug. ſerm. de temp.~255.}
For this, the very Prieſts themſelues are honourable, chaſt, ſacred,
\Cite{Hiero. ep.~1. ad Heliodorum c.~7.}
\Cite{Li.~1. adu. Iouin c.~19.}
\Cite{Amb. in 1.~Tim.~1.}
For this, the people is forbidden to touch it with common hands.
\Cite{Nazia. Orat. ad Arianus in initio.}
For this, great care and ſolicitude is taken that no part of either kind
fal to the ground.
\Cite{Cyril.}
\Cite{Hiero. Myſtag.~5. in fine.}
\Cite{Orig. ho.~13. in c.~25. Exod.}
For this, ſacred prouiſion is made that if any hoſts or parts of the
Sacrament doe remaine vnredeiued, they be moſt religiouſly reſerued with
al honour and diligence poſsible: and for this, examination of
conſciences, confeſsion, continencie, & (as 
\CNote{ep.~118. c.~6.}
S.~Auguſtin ſaith) receiuing it faſting. Thus doe we Catholikes & the
Church of God diſcerne the holy body & bloud by S.~Paules rule, not only
from 
\MNote{The profane bread of the Proteſtants.}
your prophane bread and wine (which not by any ſecret abuſe of your
Curates or Clerkes, but by the very order of your booke, the Miniſter,
if any remaine after your Communion, may take home with him to his owne
vſe, and therfore is no more holy by your owne iudgement then the reſt
of his meates) but from al other either vulgar or ſanctified meates, as
\CNote{\Cite{Aug. de pec. merit, li.~2. c.~24. Ep. Iuda.}}
\MNote{Holy bread.}
the Catechumens bread, and our vſual holy bread. If al this be plaine
and true, and you haue nothing agreable to the Apoſtles nor Chriſtes
Inſtitution but al cleane contrarie, then \L{imperet vobis Deus}, and
cocfound you for not diſcerning his holy Body, and for conculcating the
bloud of the new Teſtament.}
not diſcerning the body of our Lord. \V Therfore are there among you
many weake and feeble, and
\LNote{Many ſleep.}{We
\MNote{Vnworthy receiuing.}
ſee here by this, it is a fearful caſe and crime to defile by ſinne (as
much as in vs lieth) the body of Chriſt in the Sacrament, ſeeing God
ſtrook many to death for it in the Primitiue Church, & puniſhed others
by greeiuous ſicknes. No maruel that ſo many ſtrange diſeaſed and deaths
fal vpon vs now in the world.}
many ſleep. \V But if we did
\LNote{Iudge your-ſelues.}{We
\MNote{Penance and ſatisfaction.}
may note here that is is not enough, only to ſinne no more, or to repent
lightly of that which is paſt: but that we ſhould puniſh ourſelues
according to the weight of the faults paſt and forgiuen: and alſo that
God wil puniſh vs by temporal ſcourges in this life or the next, if we
doe not make our-ſelues very cleane before we come to receiue his holy
Sacrament. Whoſe heauy hands we may eſcape by puniſhing our-ſelues bby
faſting and other penance.}
iudge our ſelues, we ſhould not be iudged. \V But whiles we are iudged,
of our Lord we are chaſtiſed; that with this world we be not damned. \V
Therfore, my Brethren, when you come together to
%%% 2695
eate,
\LNote{Expect one another.}{Returning now to their former fault and
diſorder for the which he tooke this occaſion to talke of the Holy
Sacrament, and how great a fault it is to come vnworthily to it; he
exhorteth them to keep their ſaid ſuppers or feaſts in vnitie, peace,
and ſobrietie, the rich expecting the poore, &c.}
expect one another. \V If any man be an hungred, let him eate at home;
that you come not together vnto iudgement. And the reſt
\LNote{I wil diſpoſe.}{Many
\MNote{The Maſſe is agreable to the Apoſtles vſe and tradition: the
communion is not.}
particular orders & decrees, moe then be here or in any other book of
the new Teſtament expreſly written, did the Apoſtles, as we ſee here,
and namely S.~Paul to Corinthians, ſet downe by tradition, which our
whole miniſtration of the \Sc{Masse} is agreable vnto, as the ſubſtance
of the Sacrifice and Sacrament is by the premiſſes proued to be moſt
conſonant: Caluin's ſupper and Communion in al points wholy repugnant to
the ſame. And that it agreeth not to theſe other not written traditions,
they eaſily confeſſe. The
\CNote{\XRef{Aug. ep.~118. c.~6.}}
Apoſtles deliuered vnto the Church to take it only faſting: they care
not for it. The Apoſtles taught the Church to conſecrate by the words
and the ſigne of the Croſſe, without which (ſaith S.~Auguſtin
\Cite{tract. in lo. 118.}
\Cite{Serm.~75. in append.}
\Cite{Chryſoſt. hom.~55. in 16.~Matth.})
no Sacrament is rightly perfitted: the Proteſtants haue taken it
away. The Apoſtles taught the Church to keep
\CNote{\Cite{Aug. tract.~34. in Io.}
&
\Cite{Chry. ho.~21. in Act.}}
a Memorie or inuocation of Saints in this Sacrifice: the Caluiniſts haue
none. The Apoſtles decreed that in this Sacrifice there should be
ſpecial praiers for the dead.
\Cite{Chryſ. hom.~3. in epiſt. ad Philip.}
\Cite{Auguſt. de cur. pro mort. c.~1.}
they haue none. Likewiſe that water should be mixed with the wine, and
ſo forth. See
\XRef{Annot. in c.~11. v.~13. Bread.}
Therfore if Caluin had made his new adminiſtration according to al the
Apoſtles written words, yet not knowing how many things beſide, the
Apoſtle had to preſcribe in theſe words, \L{Cetera cum venero disponam}
(the reſt I wil diſpoſe, when I come) he could not haue ſatisfied any
wiſe man in his new change. But now ſeeing they are fallen to ſo
palpable blindnes, that their doing is directly oppoſit to the very
Scripture alſo, which they pretend to follow only, and haue quite
deſtroied both the name, ſubſtance, and al good accidents of Chriſtes
principal Sacrament, we truſt al the world wil ſee their folly and
impudencie.}
I wil diſpoſe, when I come.


\stopChapter


\stopcomponent


%%% Local Variables:
%%% mode: TeX
%%% eval: (long-s-mode)
%%% eval: (set-input-method "TeX")
%%% fill-column: 72
%%% eval: (auto-fill-mode)
%%% coding: utf-8-unix
%%% End:

