%%%%%%%%%%%%%%%%%%%%%%%%%%%%%%%%%%%%%%%%%%%%%%%%%%%%%%%%%%%%%%%%%
%%%%
%%%% The (original) Douay Rheims Bible 
%%%%
%%%% New Testament
%%%% Epistles
%%%% One Corinthians
%%%% Chapter 10
%%%%
%%%%%%%%%%%%%%%%%%%%%%%%%%%%%%%%%%%%%%%%%%%%%%%%%%%%%%%%%%%%%%%%%




\startcomponent chapter-10


\project douay-rheims


%%% 2690
%%% o-2544
\startChapter[
  title={Chapter 10}
  ]

%%% !!! See the summary for chap. 9.
\Summary{See the argument of the 9.~Chapter, which comprehendeth the
  contents of this alſo.}

%%% o-2545
%%% !!! The XRefs in this paragraph are all confused.
For I wil not haue you ignorant, Brethren, that our Fathers were al
\CNote{\XRef{Exo.~13,~22.}
\XRef{Nu.~8,~8.}}
vnder the cloud, & al
\CNote{\XRef{Exo.~14,~22.}}
paſſed through the ſea, \V and al in Moyſes were baptized in the cloud
and in the ſea: \V and
\CNote{\XRef{Exo.~16,~15.}}
al did eate
\LNote{The ſame.}{The
\MNote{The old figures of our Sacraments.}
red ſea and the cloud, a figure of our Baptiſme: the Manna from Heauen
and water miraculouſly drawen out of the rock, a figure of the holy
Sacrament of Chriſtes body and bloud: our Sacrament containing the
things and graces in truth, which theirs only ſignified.
\MNote{We receiue greater benefits by our Sacraments
\Fix{the}{then}{obvious typo, same in both}
the Iewes did by theirs.}
And it is an impudent forgerie of the
\CNote{\Cite{Calu. in hunc loc.}}
Caluiniſts, to write vpon this place, that the Iewes receiued no leſſe
the truth and ſubſtance of Chriſt and his benefits in their figures or
Sacraments, then we doe in ours: and that they and we al eate and drinke
of the ſelf ſame meate and drinke: the Apoſtle ſaying only, that they
among themſelues did al feed of one bread, & drinke of one rock: which
was a figure of Chriſt, therin eſpecially, that out of Chriſtes ſide
pearced vpon the Croſſe gushed out bloud and water for the matter of our
Sacraments.}
the ſame ſpiritual food, \V and al
\CNote{\XRef{Exo.~17,~6.}}
drunke the ſame ſpiritual drinke (and they
\CNote{\XRef{Nu.~20,~10.}}
drunke of the ſpiritual rock that followed them, and the rock was
Chriſt.) \V But in the more part of them God was not wel pleaſed. For
they
\CNote{\XRef{Nu.~26,~63.}}
were ouerthrowen in the deſert. \V And theſe things were done in a
figure of vs, that we be not coueting euil things, as
\CNote{\XRef{Nu.~11,~4.}}
they alſo coueted. \V Neither become ye Idolaters, as certaine of them:
as is written:
\CNote{\XRef{Exo.~32,~6.}}
\Emph{The people ſate downe to eate and drinke, and roſe vp to play.} \V
Neither let vs fornicate,
\CNote{\XRef{Nu.~25,~1.}}
as certaine of them did fornicate, and there fel in one day three and
twentie thouſand. \V Neither let vs tempt Chriſt, as certaine of them
tempted, and
\CNote{\XRef{Nu.~21,~5.}}
periſhed by the ſerpents. \V Neither doe you murmure, as
\CNote{\XRef{Nu.~11,~23.}
\XRef{14,~37.}}
certaine of them murmured, and periſhed by the deſtroyer. \V And al
theſe things chanced to them in figure: but they are written to our
correption, vpon whom the ends of the world are come. \V Therfore he
that thinketh himſelf to ſtand, let him take heed
\SNote{It is profitable to al, or in a manner to al, for to keep them in
humilitie, not to know what they shal be, ſaith S.~Auguſtin. Which
maketh againſt the vaine ſecuritie of the Proteſtants.}
leſt he fal. \V
\Var{Let not tentation apprehend}{Tentatiõ hath not apprehended}
you, but humane. And God is faithful, who wil not ſuffer you to be
tempted aboue that which you are able: but wil make alſo with tentation
\TNote{\G{ἔκβασιν}}
iſſue, that you may be able to ſuſteine.

\V For the which cauſe, my Deareſt, fly from the ſeruing of Idols. \V I
ſpeake
\LNote{As to wiſe men.}{To
\MNote{The Apoſtle and ancient fathers ſpeake couertly of the
B.~Sacrament.}
cauſe them to leaue the Sacrifices and meats or drinkes offered to
Idols, he putteth them in mind of the only true Sacrifice and meate and
drinke of Chriſtes body and bloud: of which, and the Sacrifice of Idols
alſo, they might not be in any caſe partakers. Vſing this terme, \L{ut
prudentibus loquor}, in the ſame ſenſe (as it is thought) as the Fathers
of the primitiue Church did giue a watch-word of keeping ſecret from the
Infidels and vnbaptized, the myſterie of this diuine Sacrifice, by theſe
wordes, \L{Norunt fideles, norunt qui initiati ſunt.}
\Cite{Auguſt, in Pſ.~39.}
&
\Cite{33. Conc. 1.~2.}
&
\Cite{Pſ.~109.}
\Cite{Ho.~41. c.~4. in lib.~50. hom.}
\Cite{Orig. in Leuit. ho.~9.}
\Cite{Chryſ. ho.~27. in Gen. in fine.}
\Cite{ho.~51. ad po. Antioch.}
\Cite{ho.~3. in 1.~Tim.}
S.~Paul ſaith: I ſpeake to you boldly of this myſterie as to the wiſer
and better inſtructed in the ſame.}
as to wiſe men: your ſelues iudge what I ſay. \V The chalice of
benediction
\LNote{Which we bleſſe.}{That
\MNote{The Apoſtles bleſſed the Chalice, & ſo conſecrated.}
is to ſay, the 
\TNote{\L{Calix cui benedicimus}
\G{ὃ εὐλογοῦμεν}}
Chalice of Conſecration which we Apoſtles and Prieſts by
Chriſtes commiſsion doe conſecrate: by which ſpeach as wel the
Caluiniſts (that vſe no cõſecration of the cup at al, blaſphemouſly
calling it magical murmuration, and peruerſely referring the
benediction, to thankes-giuing to God) as alſo the Lutherans be refuted,
who affirme Chriſtes body & bloud to be made preſent by receiuing and in
the receiuing only. For the Apoſtle expreſly referreth the benediction
to the chalice, and not to God, making the holy bloud and the
communicating therof the effect of the benediction.}
which we doe bleſſe, is it not the communication of the bloud of Chriſt?
and the bread which we break, is it not
\LNote{The participation of the body.}{The
\MNote{Our vniting to Chriſt by the B.~Sacrament.}
holy Sacrament and Sacrifice of Chriſt's body and bloud being receiued
of vs, ioyneth vs in ſoul & body and engrafteth vs into Chriſt himſelf,
making vs partakers, and as a peece of his body & bloud. \Emph{For not
by loue or ſpirit only} (ſaith S.~Chryſoſtom) \Emph{but in very deed we
are vnited in his flesh, made one body with him, members of his flesh
and bones.}
\Cite{Chryſ. ho.~45. in Io. ſub finem.}
And S.~Cyril, \Emph{Such is the force of myſtical benediction that it
maketh Chriſt corporally by communicating of his flesh to dwel in vs.}
\Cite{Cyril. li.~10. in Io. c.~13.}}
the participation of the body of our Lord? \V For being many, we are
\LNote{One bread, one body.}{As
\MNote{Our vnion among our ſelues by the B.~Sacrament.}
we be firſt made one with Chriſt by eating his body and drinking his
bloud, ſo ſecondly are we conioyned by this one bread which is his body,
& cup which is his bloud, in the perfect vnion and fellowship of al
Catholike men, in one Church which is his body Myſtical. Which name of
Body myſtical is ſpecially attributed and appropriated to this one
Common-wealth and Societie of faithful men, by reaſon that al the true
perſons and true members of the ſame, be maruelouſly knit together by
Chriſtes owne one body, and by the ſelf-ſame bloud in this diuine
Sacrament. See
\Cite{S.~Aug. li.~21. c.~25. de ciu. Dei.}
\Cite{Hilar. li.~8. de Trin. circa med.}}
one bread, one body, al that participate of one bread. \V Behold Iſrael
according to the fleſh:
\LNote{They that eate the Hoſts.}{It
\MNote{Participation in Sacrament or ſacrifice, sheweth of what
ſocietie we are.}
is plaine alſo by the example of the Iewes in their Sacrifices, that he
that eateth any of the Hoſt immolated, is partaker of the Sacrifice, and
ioyned by office and obligation to God, of whoſe Sacrifice he eateth.}
they that eate the Hoſts, are they not partakers of the altar? \V What
then? doe I ſay that that which is immolated to Idols, is any thing? or
that the Idol is any thing? \V But the things that the Heathen doe
immolate, to Diuels they doe immolate, and not to God. And
\LNote{I wil not haue you.}{I conclude then (ſaith the Apoſtle) thus:
that as the Chriſtian which eateth and drinketh of the Sacrifice or
Sacrament of the altar, by his eating is participãt of Chriſtes body,
and is ioyned in fellowship to al Chriſtian people that eate & drinke of
the ſame, being the Hoſt of the new Law: and as al that did eate of
the Hoſts of the Sacrifices of Moyſes Law were belonging & aſſociated to
that ſtate and to God to whom the Sacrifice was done; euen ſo whoſoeuer
eateth of the meates offered to Idols, he sheweth & profeſſeth himſelfe
to be of the Communion and Societie of the ſame Idols.}
I wil not haue you become fellowes of Diuels. \V
\LNote{You cannot drinke.}{Vpon the premiſſes he warneth them plainely,
that they muſt either forſake the Sacrifice & fellowship of the Idols &
Idolaters, or els refuſe the Sacrifice of Chriſt's body and bloud in the
Church.
\MNote{The ſacrifice of the altar is proued by the Apoſtles compariſon
with the ſacrifices of Iewes and Gentils.}
In al which diſcourſe we may obſerue that our bread and chalice,
our table and altar, the participation of our Hoſt and oblation, be
compared or reſembled point by point, in al effects, conditions, and
proprieties, to the altars, Hoſts, Sacrifices and Immolations of the
Iewes and Gentils. Which the Apoſtle would not, nor could not haue done
in this Sacrament of the Altar, rather then in other Sacraments or
ſeruice of our religion, if it only had not been a Sacrifice and the
proper worship of God among the Chriſtians, as the other were among the
Iewes and Heathen.
\MNote{It is proued to be a ſacrifice, out of the fathers.}
And ſo doe al the Fathers acknowledge, calling it only, & continually
almoſt, by ſuch termes as they doe no other Sacrament or ceremonie of
Chriſtes religion:
\Emph{The lamb of God laid vpon the table}:
\Cite{Conc. Nic.}
\Emph{the vnbloudy ſeruice of the Sacrifice},
\Cite{In Conc. Epheſ. ep. ad Neſtor. pag.~60.}
\Emph{the Sacrifice of Sacrifices},
\Cite{Dionyſ. Ec. Hier. c.~3.}
\Emph{the quickning holy Sacrifice}:
\Emph{the vnbloudy Hoſt and Victime}:
\Cite{Cyril. Alex. in Conc. Epheſ. Anath.~11.}
\Emph{the propitiatorie Sacrifice both for the liuing and the dead}:
\Cite{Tertul. de cor. Milit.}
\Cite{Chryſ. ho.~41. in 1.~Cor.}
\Cite{Ho.~3. ad Philip}
\Cite{Ho.~66. ad po Antioch.}
\Cite{Cypr. ep.~66.}
&
\Cite{de cœn. Do. nu.~1.}
\Cite{Auguſt. Ench.~109.}
\Cite{Quæſt.~2. ad Dulcit. to.~4.}
\Cite{Ser.~34. de verb. Apoſt.}
\Emph{the Sacrifice of our Mediatour}:
\Emph{the Sacrifice of our price};
\Emph{the Sacrifice of the new Teſtament}:
\Emph{the Sacrifice of the Church}:
\Cite{Auguſt. li.~9. c.~13.}
&
\Cite{li.~3. de bapt. c.~19.}
\Emph{the one only inconſumptible Victime without which there is no
religion}:
\Cite{Cyprian de cœn. Do nu.~2.}
\Cite{Chryſ. ho.~17. ad Hebr.}
%%% !!! Really three Cites, or maybe XRefs?
\CNote{\Cite{Cypr. Iuſtin. Irenæ. infra.}}
\Emph{The pure Oblation},
\Emph{the new Offering of the new Law}:
\Emph{the vital and impolluted Hoſt}:
\Emph{the honourable and dreadful Sacrifice}:
\Emph{the Sacrifice of thankes-giuing or Euchariſtical}: and
\Emph{the Sacrifice of Melchiſedech}. Which Melchiſedech by his Oblation
in bread and wine did properly and moſt ſingularly prefigure this office
of Chriſtes eternal Prieſthood & ſacrificing himſelf vnder the formes of
bread and wine: which shal continue in the Church throughout al
Chriſtian Nations inſteed of al the Offerings of Aarons Prieſthood, as
the
\CNote{\XRef{Malac.~1,~11.}}
Prophet Malachie did foretel; as S.~Cyprian, S.~Iuſtine, S.~Irenæus and
other moſt ancient Doctours and Martyrs doe teſtifie.
\Cite{Cypr. ep.~63. nu.~2.}
\Cite{Iuſtin. Dial cum Trypho poſt med.}
\Cite{Irenæ. li.~4. c.~32.}
And
\Cite{S.~Auguſtin li.~17. c.~20. de ciu. Die.}
&
\Cite{li.~1. cont. adu. log. & proph. c.~12.}
&
\Cite{li.~3. de bapt. c.~19.}
\Cite{S.~Leo ſer.~8. de Paſsione:}
and others doe expreſly auouch that this one Sacrifice hath ſucceeded al
other & fulfilled al other differences of Sacrifices: that it hath the
force and vertue of al other, to be offered for al perſons and cauſes
that the others, for the liuing and the dead, for the ſinnes and for
thankes-giuing, and for what other neceſsitie ſoeuer of body or ſoule.
\CNote{\Cite{Amb. ep.~33.}}
\MNote{The Fathers called this ſacrifice, the \Sc{Masse}.}
Which holy action of Sacrifice they alſo cal the \Sc{Masse} in plaine
words.
%%% !!! These are probably wrong.
\Cite{Auguſt. ſer.~251.~91. Con. Cartha.~2. c.~3.~4. c.~84.}
\Cite{Mileuit.~12.}
\Cite{Leo. ep.~88.~81. c.~2.}
\Cite{Greg. li.~2. ep.~9.~91.~&c.}
This is the Apoſtles and Fathers doctrine. God grant the Aduerſaries may
find mercie to ſee ſo euident and inuincible a truth.}
You can not drinke the chalice of our Lord, and the chalice of Diuels:
%%% o-2546
you can not be 
%%% !!! Only marked in other
\LNote{Partakers of the table.}{Though
\MNote{The diſtinctiò of Chriſtian Catholikes frõ the reſt, is by not
cõmunicating with thẽ ſpecially in their Sacrifices, and at the
Communion table.}
the faithful people be many waies knowen to be God's peculiar, and be
ioyned both to him & among themſelues, & alſo ſeuered & diſtinguished
from al others that pertaine not to him, as wel Iewes and Pagans, as
Heretikes and Schiſmatikes, by ſundry other external ſignes of
Sacraments, doctrine, and gouernement: yet the moſt proper & ſubſtantial
vnion or difference conſiſteth in the Sacrifice and altar: by which God
ſo ſpecially bindeth his Church vnto him, & himſelf vnto his Church,
that he acknowledgeth none to be his, that is not partaker of his one
only Table and Sacrifice in his Church: and acquitteth himſelf of al
ſuch as ioyne in fellowship with any of the Heathen at their Idolatrie,
or with the Iewes at their Sacrifices, or with Heretikes and
Schiſmatikes at their prophane and deteſtable table.
\MNote{The heretikes Communion is the very table and cup of Diuels.}
Which becauſe it is the proper badge of their ſeparation from Chriſt and
his Church; and an altar purpoſely erected againſt Chriſtes Altar,
Prieſthood, and Sacrifice, is indeed a very Sacrifice, or (as the Apoſtle
here ſpeaketh) a table and cup of Diuels, that is to ſay, wherin the
Diuel is properly ſerued, and Chriſtes honour (no leſſe then
\CNote{\XRef{3.~Reg.~12.}}
by the altars of Ieroboam or any prophane ſuperſtitious rites of
Gentilitie) defiled. And therfore al Catholike men, if they look to haue
fellowship with Chriſt and his members in his body and bloud, muſt deeme
of it as of Idolatrie or ſacrilegious ſuperſtition, and abſtaine from
it and from al ſocietie of the ſame, as good 
\CNote{\XRef{Tob.~1.}}
Tobie did from Ieroboams calues and the altars in Dan and Bethel: and as
the good
\CNote{\XRef{3.~Reg.~12.}}
faithful did from the Excelſes and from the Temple and Sacrifices of
Samaria. Now in the Chriſtian times we haue no other Idols but hereſies,
nor Idolathytes, but their falſe ſeruices shifted into our Churches
inſteed of God's true, and only worship.
\Cite{Cyp. de vnit. Ec. nu.~2.}
\Cite{Hiero. in 11.~Oſee.}
&
\Cite{8.~Amos.}
&
\Cite{in 2.~Habac.}
\Cite{Aug. in pſ.~80. v.~10.}
\Cite{De Ciu. Dei. li.~18. c.~51.}}
partakers of the table of our Lord, and of the table of Diuels. \V Or
doe we emulate our Lord? Why, are we ſtronger then he?

Al
%%% !!! LNote really should be before 'Al'
\LNote{Al things are lawful.}{Hitherto
\MNote{How by participation with Idolaters, Idolatrie is committed.}
the Apoſtles arguments and examples whereby he would auert them from the
meates offered to Idols, ſeeme plainly to condemne their fact as
Idololatrical, or as participant and acceſſory to Idolatrie, and not
only as of ſcandal giuen to the weake Brethren: and ſo no doubt it was
in that they went into the very Temple of the Idols, and did with the
reſt that ſerued the Idols eate and drinke of the flesh and libaments
directly offered to the Idol, yea and feaſted together in the ſame
bankets made to the honour of the ſame Idols: which could not but defile
them and entangle them with Idolatrie: not for that the meate itſelf was
iuſtly belonging to any other but to God, or could be defiled, made
noiſome or vnlawful to be eaten; but for and in reſpect of the abuſe of
the ſame and deteſtable dedicating of that to the diuel, which belonged
not to him, but to God alone. Of which ſacrilegious act they ought not
to be partakers, as needs they muſt entring & eating with them in their
ſolemnities.
To this end hath S.~Paul hitherto admonished the Corinthians.
\MNote{How to auoid ſcandal in things indifferent.}
Now he
declareth that otherwiſe in prophane feaſts it is lawful to eate without
curious doubting or asking whether this or that were offered meates, and
in markets to buy whatſoeuer is there ſold, without ſcruple and without
taking knowledge whether it be of the Idolathytes or not: with this
exception, firſt, that if one should inuite him to eate, or buy this or
that as ſacred and offered meates, that then he should not eate it, leſt
he should ſeeme to approue the offering of it to the Idol, or to like it
the better for the ſame. Secondly, when the weake Brother may take
offence by the ſame. For though it be lawful in itſelf to eate any of
theſe meates without care of the Idol; yet al lawful things be not in
euery time and place expedient to be done.}
things are lawful for me, but al things are not expedient. \V Al things
are lawful for me, but al things doe not edifie. \V Let no man ſeeke his
owne, but another man's. \V Al that is ſold in the ſhambles eate: asking
no queſtion for conſcience. \V
\CNote{\XRef{Pſ.~23,~1.}}
\Emph{The earth is our Lordes, and the
%%% 2691
fulnes therof.} \V If any inuite you of the infidels, and you wil goe;
eate of al that is ſet before you, asking no queſtion for conſcience. \V
But if any man ſay: This is immolated to Idols; doe not eate for his
ſake that ſhewed it, and for conſcience: \V conſcience I ſay not thine but
the other's. For why is my libertie iudged of another man's
conſcience? \V If I participate with thankes; why am I blaſphemed for
that which I giue thankes for? \V Therfore whether you eate, or drinke,
or doe any other thing; doe al things vnto the glorie of God. \V Be
without offenſe to the Iewes & to the Gentils, & to the Church of
God: \V as I alſo in al things doe pleaſe al men, not ſeeking that which
is profitable to my ſelf, but which is to many; that they may be ſaued.


\stopChapter


\stopcomponent


%%% Local Variables:
%%% mode: TeX
%%% eval: (long-s-mode)
%%% eval: (set-input-method "TeX")
%%% fill-column: 72
%%% eval: (auto-fill-mode)
%%% coding: utf-8-unix
%%% End:

