%%%%%%%%%%%%%%%%%%%%%%%%%%%%%%%%%%%%%%%%%%%%%%%%%%%%%%%%%%%%%%%%%
%%%%
%%%% The (original) Douay Rheims Bible 
%%%%
%%%% New Testament
%%%% Epistles
%%%% One Corinthians
%%%% Chapter 09
%%%%
%%%%%%%%%%%%%%%%%%%%%%%%%%%%%%%%%%%%%%%%%%%%%%%%%%%%%%%%%%%%%%%%%




\startcomponent chapter-09


\project douay-rheims


%%% 2687
%%% o-2542
\startChapter[
  title={Chapter 9}
  ]

%%% !!! These verse numbers make little sense. Are they just doing
%%% things out of order? But, this summary also applies to chap. 10,
%%% which see. What do we do?
\Summary{To them that ſo vaunted their libertie about \L{Idolothyta},
  he bringeth his owne example, to wit, that he alſo had libertie to
  liue by the Ghoſpel, but yet that he vſed it not, ſo to auoid ſcandal
  of the infirme, and becauſe it was more meritorious. 24.~Declaiming
  againſt their ſecuritie, and shewing them by ſimilitudes and examples,
  24.~both of himſelf, 1.~And of the Iſraelites, that ſaluation is not
  ſo lightly come by: 14.~and ſo concludeth againe againſt eating
  of \L{Idolothyta}, becauſe it is alſo to commit idolatrie, 21.~and not
  only to giue il example to the infirme.}

Am I not free? Am I not an Apoſtle? Haue I not ſeen Chriſt \Sc{Iesvs}
our Lord? Are not you
\LNote{My worke.}{As
\MNote{The Heretikes fond pretenſe of God's honour.}
he called himſelf before God's Coadiutour, ſo here he boldly alſo
chalengeth the Corinthians conuerſion to be his handy-worke in our Lord:
nothing derogating thereby from Chriſt, as the Proteſtants rudely charge
the Fathers & Catholike men (vnder pretenſe of God's honour) for vſing
ſuch phraſes or ſpeaches in the Apoſtles ſenſe, of the Saints or
Sacraments.}
my worke in our Lord? \V And if to others I be not an Apoſtle, but
yet to you I am. For you are the ſeale of my Apoſtleſhip in our Lord. \V
My defenſe to them that examine me is this: \V Haue not we power to eate
and drinke? \V Haue we not power to lead about
\LNote{A woman a Siſter.}{The
\MNote{Heretical tranſlation.}
Heretikes peruerſely (as they doe al other places for the aduantage of
their Sect) expound this of the Apoſtles wiues, and for, \Emph{woman},
\CNote{\Cite{new Teſt.~1580.}}
tranſlate, \Emph{wife}, al belles ſounding wedding to them. Where the
Apoſtle meaneth plainely the deuout women that after the manner of
Iewrie did ſerue the Preacher of neceſſaries,
\CNote{Mt.~25,~55.}
of which ſort many
followed Chriſt, and ſuſtained him and his of their ſubſtance. So doth
S.~Chryſoſtome, Theodorete, and al the Greeks
\Cite{(Oecu, in collect ſuper hunc lo.)}
take it. So doth S.~Auguſtin
\Cite{De op. Monach. c.~4.}
and S.~Hierom
\Cite{li.~1. adu Iouinianum c.~14.}
both diſputing and prouing it by the very words of the text. S.~Ambroſe
alſo
\Cite{vpon this place.}
And the thing is moſt plaine. For to what end should he talke of
burdening the Corinthians with finding his wife, when himſelf
\XRef{c.~7,7.~8.}
cleerly ſaith that he was ſingle?}
a woman a Siſter, as alſo the reſt of the Apoſtles, and our Lord's
Brethren, and
\SNote{He nameth Cephas (that is Peter) to proue his purpoſe by the
example of the cheefe and Prince of the Apoſtles.
\Cite{S.~Ambro.}
\Cite{S.~Chryſ.}
\Cite{Oecum vpon this place.}}
Cephas? \V Or I only and Barnabas haue not we power to doe this? \V
\LNote{Who plaieth the ſouldiar?}{He
\MNote{Paſtours and Preachers due.}
proueth by the Scriptures and natural reaſons that Preachers and
Paſtours may chalenge their finding of their flocks, though himſelf for
cauſes had not, nor intended not to vſe his right and libertie therin.}
Who euer plaieth the ſouldiar at his owne charges? who planteth a vine,
and eateth not of the fruit therof? who feedeth a flock, and eateth not
of the milke of the flock? \V Speake
%%% 2688
I theſe things according to man? Or doth not the Law alſo ſay theſe
things? \V For it is written in the Law of Moyſes:
\CNote{\XRef{Deu.~25,~4.}}
\Emph{Thou shalt not mooſel the mouth of the oxe that
\SNote{In that countrie they did tread out their corne with oxen, as we
doe thresh it out.}
treadeth out the corne.} Why, hath God care of oxen? \V Or for vs certes
doth he ſay it? For they are written for vs. Becauſe he that eareth,
%%% o-2543
ought to eare in hope: and he that treadeth, in hope to receiue fruit. \V
If we haue ſowen vnto you ſpiritual things, is it a great matter if we
reape your carnal things? \V If other be partakers of your power; why
not we rather? Howbeit we haue not vſed, this power: but we beare al
things, leſt we ſhould giue any offence to the Ghoſpel of Chriſt. \V
Know you not
\CNote{\XRef{Deu.~18,~1.}}
that they which worke in the holy place, eate the things that are of the
holy place: and they that ſerue
\SNote{The English Bible
\Cite{(1562)}
here and in the next chapter, ſaith thriſe
for \Emph{altar}, \Emph{temple}: moſt falſely &
\Fix{hertically,}{heretically,}{obvious typo, fixed in other}
againſt holy altars, which about the time of that tranſlation, were
digged downe in England.}
\TNote{\G{τῷ θυσιαστηρίῳ}}
the altar, participate with the altar? \V So alſo our Lord ordained for
them that preach the Ghoſpel, to liue of the Ghoſpel.

\V But I haue vſed none of theſe. Neither haue I written theſe things,
that they ſhould be ſo done in me; for it is good for me to die rather,
then that any mã ſhould make my glorie void. \V For &
\LNote{If I euangelize.}{If
\MNote{Works of ſupererogation.}
I should preach either of compulſion and ſeruil feare, or mere
neceſsitie, not hauing otherwiſe to liue and ſuſtaine my ſelf in this
world, I could not looke for reward in Heauen. But now doing it, not
only as enioyned me, but alſo as of loue and charitie, and freely
without putting any man to coſt, and that voluntarily and of very deſire
to ſaue my hearers, I shal haue my reward of God, yea and a reward of
Supererogation, which is giuen to them that of aboundant charitie doe
more in the ſeruice of God then they be commanded, as S.~Auguſtin
expoundeth it.
\Cite{De op. Mon. c.~5.}}
if I euangelize, it is no glorie to me: for neceſſitie lieth vpon me:
for woe is to me if I euangelize not. \V For if I doe this willingly, I
haue reward: but if againſt my wil, a charge is committed to me. \V What
is my reward then? That preaching the Ghoſpel, I yeald the Ghoſpel
without coſt, that I abuſe not my power in the Ghoſpel. \V For whereas I
was free of al, I made my ſelf the ſeruant of al: that I might gaine the
moe. \V And I became to the Iewes as a Iew, that I might gaine the
Iewes. \V To them that are vnder the Law, as though I were vnder the Law
(whereas my ſelf was not vnder the Law) that I might gaine them that
were vnder the Law. To them that were without the Law, as though I were
without the Law (whereas I was not without the law of God, but was in the
law of Chriſt) that I might gaine them that were without the Law. \V To
the weake I became weake, that I might gaine the weake. To al men
\SNote{Not by fiction or ſimulation, but by compaſſion of the
infirmities of al ſorts.
\Cite{Aug. ep.~9.}}
I became al things, that I might ſaue al. \V And I doe al things for the
Ghoſpel,
\LNote{That I may be partaker.}{A
\MNote{Doing wel in reſpect of reward.}
ſingular place to conuince the Proteſtants, that wil not haue men worke
wel in reſpect of reward at God's hand: the Apoſtle confeſsing expreſly
that al this that he doth either of duety or of Supererogation aboue
duety (as to preach of free-coſt, and to worke with his owne hands to
get his owne meate and his fellowes, and to abſtaine from many lawful
things) al is, the rather to attaine the reward of Heauen.}
that I may be made partaker therof.

\V Know you not that they that run in the race, al run indeed, but one
receiueth the price?
\LNote{So run.}{If
\MNote{Running for the game.}
ſuch as run for a prize, to make themſelues more ſwift, and to win the
game, abſtaine from many meats and pleaſures; what should not we doe or
ſuffer to winne the crowne of glorie, propoſed and promiſed to none but
ſuch as run, trauel, and endeauour for it?}
So run that you may obteine. \V And euery one that ſtriueth for the
maiſtrie, refraineth himſelf from al things: and they certes that they
may receiue a corruptible crowne: but we an incorruptible. \V I therfore
ſo run, not as it were at an vncertaine thing: ſo I fight, not as it
were beating the aire: \V but
\LNote{I chaſtiſe.}{The
\MNote{Penance meritorious.}
goale of euerlaſting glorie is not promiſed nor ſet forth for only-faith
men; for ſuch run at random: but it is the prize of them that chaſtiſe
and ſubdue their bodies and fleshly deſires by faſting, watching,
voluntary pouertie, and other afflictions. Lord, how farre is the carnal
doctrine of the Sectaries and the manners of theſe daies from the
Apoſtles ſpirit! Wherein euen we that be Catholikes, though we doe not
condemne with the Proteſtants theſe voluntarie afflictions as
ſuperfluous (much leſſe as ſuperſtitious or iniurious to Chriſt's
death,) but much commend them, yet we vſe nothing the zeale and
diligence of our firſt Chriſtian Anceſtours herein, and therfore are
like to be more ſubiect to God's temporal chaſtiſements, at the leaſt in
the next life, then they were.}
I chaſtiſe my body, and bring it into ſeruitude,
\LNote{Leſt perhaps.}{Here
\MNote{S.~Paul had not the Proteſtants ſecuritie of ſaluation.}
may we lambs tremble
\CNote{Aug. apud Pet. Lomb. in hunc locum.}
(ſaith a holy Father) when the ramme, the guide of the flock, muſt ſo
labour and punish himſelf (beſides al his other miſeries adioyned to the
preaching of the Ghoſpel) leſt perhaps he miſſe the marke. A man might
thinke S.~Paul should be as ſure and as confident of God's grace &
ſaluation as we poore wretched caitiues but the Heretikes vnhappy
ſecuritie, preſumption, and faithles perſuaſion of their ſaluation is
not \L{fides Apoſtolorum}, but \L{fides Dæmonorum}, not the \Emph{faith
of the Apoſtles}, but the \Emph{faith of Diuels.}}
leſt perhaps when I haue preached to others, my ſelf become reprobate.



\stopChapter


\stopcomponent


%%% Local Variables:
%%% mode: TeX
%%% eval: (long-s-mode)
%%% eval: (set-input-method "TeX")
%%% fill-column: 72
%%% eval: (auto-fill-mode)
%%% coding: utf-8-unix
%%% End:

