%%%%%%%%%%%%%%%%%%%%%%%%%%%%%%%%%%%%%%%%%%%%%%%%%%%%%%%%%%%%%%%%%
%%%%
%%%% The (original) Douay Rheims Bible 
%%%%
%%%% New Testament
%%%% One Corinthians
%%%% Chapter 07
%%%%
%%%%%%%%%%%%%%%%%%%%%%%%%%%%%%%%%%%%%%%%%%%%%%%%%%%%%%%%%%%%%%%%%




\startcomponent chapter-07


\project douay-rheims


%%% 2682
%%% o-2536
\startChapter[
  title={Chapter 7}
  ]

\Summary{That
\MNote{The 3.~part.

Of Mariage and continencie.}
married folke may aske their debt, and muſt pay it, though
  it be better for them to conteine, 8.~as alſo for the vnmarried and
  widowes to continue ſingle, though they may marrie. 10.~That the
  married may not depart from one another (nor in any caſe marrie
  another, during the life of the former) 12.~vnles it be from one that
  is vnbaptized, which yet he diſſuadeth: 17.~counſeling alſo euery one
  to be content with his ſtate wherin he was Chriſtned. 25.~Virginitie
  is not commanded, but counſeled as the better and more meritorious
  then Marriage, 39.~as alſo widowhood.}

%%% o-2537
And concerning the things wherof you wrote to me: It is good for a mã
not to touch a woman. \V But becauſe of fornicatiõ let euery mã haue
\LNote{His owne wife.}{He
\MNote{The Apoſtle biddeth not al to marrie, but to keep their wiues
before married.}
ſaith not, as the Proteſtants here pretend to excuſe the vnlawful
coniunction of Votaries, \Emph{Let euery one marry}: but,
\TNote{\G{τὴν ἑαυτοῦ}}
let euery one haue, keep, or vſe his owne wife to whom he was married
before his conuerſion. For the Apoſtle anſwereth here to the firſt
queſtion of the Corinthians, which was not, whether it were lawful to
marry, but whether they were not bound vpon their cõuerſion, to abſtaine
from the company of their wiues married before in their infidelitie, as
ſome did perſuade them that they ought to doe.
\Cite{Hiero. li.~1. cont. Iouin. c.~4.}
\Cite{Chryſ. in locum ho.~19.}}
his owne wife, & let euery woman haue her owne husbãd. \V
\LNote{Let the husband render.}{Theſe words open the Apoſtles intention
and talke to be onely of ſuch as are already married, and to inſtruct
them of the bond and obligation that is between the married couple for
rendring of the debt of carnal copulation one to another: declaring that
the married perſons haue yealded their bodies ſo one to another that
they can not without mutual conſent, neither perpetually, nor for a
time, defraud one the other.}
Let the husbãd
\TNote{\L{debitũ reddat}}
render his debt to the wife: and the wife alſo in like manner to her
huſband. \V The woman hath not the power of her owne body: but her
huſband. And in like manner the man alſo hath not power of his owne
body; but the woman. \V Defraud not one another, except perhaps by
conſent for a time,
\SNote{If the layman can not pray, vnles he abſtain from his wife: the
Prieſt that alwaies muſt offer Sacrifices and alwaies pray, muſt
therfore alwaies be free frõ matrimonie.
\Cite{Hiero. li.~1. c.~19. aduerſ. Iouin.}}
that you may
\LNote{Giue your ſelf to praier.}{This
\MNote{Continẽcie in married folkes for praier ſake.}
time, & the Heretikes doctrine, and high eſtimation of matrimonial
actes, are farre from the puritie of the Apoſtolike and primitiue
Church, when the Chriſtians to make their praiers & faſtings more
acceptable to God, abſtained by mutual conſent euen from their lawful
wiues: our new Maiſters not much abſteining (as it may be thought) from
their wiues for any ſuch matter. And yet S.~Auguſtine ſaith, the Prelate
ſhould paſſe other in this caſe, and think that not to be lawful for
him, that may be borne in others, becauſe he muſt daily ſupply Chriſtes
roome, offer, baptize, and pray for the people. So ſaith he,
\Cite{li.~1. q.~ex vtroque teſt. q.~127. in fine.}
See
\Cite{S.~Hiero. li.~1. c.~19. aduer. Iouin.}
\Cite{S.~Ambr. li.~1. Oſſe. c.~vlt.}
But alas for the people, whoſe married Paſtours are in this point farre
worſe then the vulgar folke, neither teaching continencie, nor giuing
good example.}
giue your ſelf
\Var{to praier:}{to faſting & praier:}
and returne againe together, leſt Satan tempt you for your
incontinencie. \V But I ſay this
\LNote{By indulgence not commandement.}{Leſt
\MNote{Perpetual continencie, euen in married folkes, better then carnal
copulation.}
ſome might miſconſtrue his former words, as though he had preciſely
commanded married perſons not to abſtaine perpetually from carnal
copulation, or not to giue their conſent one to another of continencie
but for a time onely: he declareth plainely that he gaue no rule or
precept abſolutely therin, but that he ſpake al the foreſaid,
condeſcending to their infirmities onely, inſinuating that continencie
from carnal copulation is much better, & that himſelf kept it
continually.
\Cite{Aug. de bon. coniug. c.~10.}
\Cite{Enchirid. c.~78.}}
by indulgence, not by commandement. \V For I would al men to be as my
ſelf: but euery one hath
\LNote{A proper guift.}{To ſuch as may lawfully marry, or be already
married God giueth not alwaies that more high and ſpecial guift or grace
of cõtinencie, though euery one of them al that duely aſke & labour for
it, might haue it: but ſuch are not bound to endeauour or ſeeke for it
alwaies, & therfore can not be commanded to abſtaine further then they
like.
\MNote{Who are boũd to liue continently: & that God giueth this guift to
al that aske it.}
But whẽſoeuer a mã is bound to abſtaine, either by vow or any
other neceſſarie occaſion (as if one of the parties be in priſon, warre,
baniſhment, ſicknes, or abſent perpetually by lawful diuorce) the other
muſt needs in paine of damnation abſtaine, and can not excuſe the want
of the guift of chaſtitie. For
\CNote{\Cite{See S.~Aug. li.~2. c.~19.~20. de Adult. Coniug. to.~6.}}
he is bound to aske it & to ſeeke for it of God by faſting, praying, &
chaſtiſing his body: & ſo labouring duely for it, God wil giue the grace
of chaſtitie. So had S.~Paul it, & ſo had al the holy men that euer
liued chaſt. Therfore deteſt the doctrine of the Proteſtants in this
point, that when they liſt not faſt nor pray for it, ſay they haue not
the guift. And it were a great maruel why ſo few of the new Sects or
rather none now a-daies haue that guift, but that we ſee it is obtained
by thoſe meanes which our Forefathers vſed, & they vſe not at al. To
liue in marriage continently without the breach of coniugal fidelitie,
is a guift of God alſo; but men muſt not breake their faith of wedlocke
for wãt of it, but muſt know that God giueth that guift to ſuch as
humbly aske it of him.
\Cite{Aug. de grat.}
&
\Cite{li. arbitrio c.~4.}
\Cite{De continent. c.~1.}}
a proper guift of God; one ſo, and another ſo. \V
\SNote{Before he treated of the continencie of ſuch as were married, now
he giueth leſſons for the vnmarried alſo.}
But I ſay to the vnmarried and to widowes: It is good for them if they
ſo abide euen as I alſo. \V But
\LNote{If they conteine not.}{He
\MNote{The Apoſtle permitteth marriage to them that be free, not to
vowed perſons.}
meaneth of ſuch as be free: for if they marry after
\Fix{thy}{they}{obvious typo, fixed in other}
haue made vow or promiſe to God of chaſtitie, they are worthily damned;
ſuch being bound to conteine, and ſo may conteine if they liſt.
\Cite{Aug. de bono vidius. c.~8.~9.}
\Cite{de adult. coniug. li.~1. c.~15.}
&
\Cite{de fide ad Petram c.~3. in fine.}
\Cite{Ambroſ. ad virg. lapſam. c.~5.}}
if they doe not conteine themſelues, let them marrie. For it is 
%%% !!! Only marked in other
\LNote{Better to marry.}{It is better to marry for the ſaid perſons that
be free, then to be ouerthrowen and fal into fornication. For, \Emph{to
burne}, or, \Emph{to be burnt}, is not to be tempted onely (as the
Proteſtants thinke that picke quarels eaſily to marry) but it ſignifieth
\CNote{\Cite{Theodoret in hunc locum.}}
to yeald to concupiſcence either in mind or external worke. We ſay alſo,
for ſuch as be free. For concerning others lawfully made Prieſts, and
ſuch as otherwiſe haue made vow of chaſtitie, they can not marry at al,
and therfore there is no compariſon in them betwixt marriage &
fornication or burning. For their marriage is but pretenſed, and is the
worſt ſort of incontinencie and fornication or burning.}
better to marrie then
\TNote{\G{κρεῖττον γάρ ἐστιν}}
to be burnt.

%%% 2683
\V But to them that be ioyned in matrimonie, not I giue commandement,
but our Lord,
\CNote{\XRef{Mt.~5,~32.}
\XRef{19,~9.}
\XRef{Mr.~10,~9.}
\XRef{Lu.~16,~18.}}
that the wife depart not from her husband: \V and if ſhe depart,
\LNote{To remaine vnmarried.}{Neither
\MNote{After diuorce not to marrie.}
partie may dimiſſe the other and marry another for any cauſe. For
though they be ſeparated for fornication, yet neither may marry againe. 
\Cite{Aug. de adult. coniug. li.~1. c.~8.~9.}
and
\Cite{li.~2. c.~3.~19.}
See
\XRef{Annot. Mat.~19.}
And S.~Auguſtine in his whole books.
\Cite{de adulter. coniugijs. to.~6.}}
to remaine vnmarried, or to be reconciled to her husband. And let not
the huſband put away his wife.

\V For the reſt,
\LNote{I ſay, not our Lord.}{By
\MNote{The Apoſtles precepts.}
this we learne, that there were many matters ouer and aboue the things
that Chriſt taught or preſcribed, left to the Apoſtles order and
interpretation: wherin they might, as the caſe required, either command
or counſel; & we bound to obey accordingly.}
I ſay, not our Lord: If any Brother haue a wife an infidel, and ſhe
conſent to dwel with him; let him not put her away. \V And if any woman
haue a husband an infidel, and he conſent to dwel with her; let her not
put away her husband. \V For the man an infidel is ſanctified by the
faithful woman; and the woman an infidel
\LNote{Sanctified.}{When
\MNote{How the infidel, or infidel's child, are ſanctified by the
Chriſtian.}
the infidel partie is ſaid to be cleane or ſanctified by the faithful,
or the children of their marriage to be cleane, we may not thinke that
they be in grace or ſtate of ſaluation thereby, but onely that the
marriage is
\CNote{\Cite{Hiero. li.~1. c.~5. aduer. Iouin.}}
an occaſion of ſanctification to the infidel partie and to the
children. For S.~Auguſtine
\Cite{(li.~3. de pec. mer. & remiſ. c.~12.)}
concludeth againſt the Pelagians, as we may doe againſt the Caluiniſts,
holding Chriſtian mens children to be holy from their mothers womb and
not to need Baptiſme, that what other ſanctification ſoeuer it be that
is here meant, it can not be enough to ſaluation without faith,
Baptiſme, &c.}
is ſanctified by the faithful husband: otherwiſe your children ſhould be
vncleane; but now they are holy. \V But if the infidel depart, let him
depart. For the Brother or Siſter is not ſubiect to ſeruitude in
ſuch. But in peace hath God called vs. \V For how knoweſt thou woman, if
thou ſhalt ſaue thy husband? or how knoweſt thou man, if thou ſhalt ſaue
the woman? \V But to euery one as our Lord hath deuided, as God hath
called euery one, ſo let him walke, and as in al Churches I teach. \V Is
any man called being circumciſed? let him not procure prepuce. Is any
man called in prepuce? let him not be circumciſed.
%%% o-2538
\V Circumciſion is nothing, and prepuce is nothing:
%%% !!! Unmarked in both
\LNote{But the obſeruation.}{Neither to be Iew nor Gentil, bõd or free,
married or ſingle, nor the faith it ſelf which is proper to Chriſtian
men, wil ſerue to ſaluation, without good works & keeping the
commandements. 
\Cite{S.~Hiero. adu. Iouin li.~1. c.~16.}}
but the obſeruation of the commandments of God. \V Euery one in the
vocation that he was called, in it let him abide. \V Waſt thou called
being a bondman? care not for it: but if thou  canſt be made free, vſe
it rather. \V For he that in our Lord is called, being a bondman, is the
\TNote{\L{libertus}}
franchiſed of our Lord. Likewiſe he that is called, being free, is the
bondman of Chriſt. \V You were bought with price, be not made the
\SNote{You muſt not ſerue men ſo that you obey & pleaſe them more thẽ God.}
bondmen of men. \V Euery
\Var{Brother}{one, Brethrẽ,}
wherin he was called, in that let him abide before God.

\V And as concerning virgins, a commandement of our Lord I haue not: but
\LNote{Counſel I giue.}{A
\MNote{The difference of counſels and precepts.}
counſel is one thing, a commandement is another. To doe that which is
counſeled, is not neceſſarie, becauſe one may be ſaued
notwithſtanding. But he that wil doe that which he is coulſeled vnto,
ſhal haue a higher degree of glorie. He that fulfilleth not a
commandement, except he doe penance, can not eſcape puniſhment.
\Cite{Aug. li. de virg. c.~11.}
&
\Cite{14.}}
counſel I giue, as hauing obteined mercie of our Lord to be faithful. \V
I thinke therfore that this is good for the preſent neceſſitie, becauſe
it is good for a man ſo to be. \V Art thou tied to a wife? ſeeke not to
be looſed. Art thou looſe from a wife? ſeeke not a wife. \V But if thou
take a wife,
\SNote{Virginitie counſeled as the better: Marriage not forbidden,
becauſe it is no ſinne.}
thou haſt not ſinned. And
\LNote{If a virgin marrie.}{He
\MNote{A profeſſed virgin may not marrie.}
ſpeaketh not of that virgin which hath dedicated her ſelf to God. (For
if any ſuch marry ſhe ſhal be damned for breaking her firſt vow) but onely
of yong maides vnmarried in the world.
\Cite{Hiero. adu. Iouin. li.~1. c.~7.}
\Cite{Chryſ. ho.~20.}
Theodorete, Photius, and the other Greek Doctours vpon this place
\Cite{apud Oecum Epiph. hæreſ. 61.}}
if a virgin marrie, ſhe hath not ſinned. Neuertheleſſe
\LNote{Tribulation of the flesh.}{They
\MNote{Virginitie counſeled as more meritorious.}
are maruelouſly deceiued (ſaith S.~Auguſtine
\Cite{li. de virg. c.~13.)}
that thinke the Apoſtle counſeleth virginitie rather then marriage,
onely for that marriage hath many miſeries and moleſtations ioyned vnto
it, which by virginitie shal be auoided, & not in reſpect of the greater
reward in Heauen. For the Apoſtles prouident counſeling to virginitie,
is for the next life, and he alleageth theſe troubles of marriage in
that ſenſe ſpecially as they be a hindrance from the ſeruice of God
here, & therfore an impediment to vs toward the next life and the more
ample ioyes thereof.}
tribulation of the fleſh ſhal ſuch haue. But I ſpare you. \V This
therfore I ſay, Brethren: The time is ſhort, it remaineth, that they
alſo which haue wiues, be
\LNote{As though they had not.}{He
\MNote{The continencie of married folke.}
exhorteth that ſuch as haue wiues, should not wholy beſtow themſelues in
the vaine tranſitorie pleaſure and voluptuouſnes of their flesh, but
liue in ſuch moderation, that their marriage hinder them as litle as may
be, from ſpiritual cogitations.
\MNote{Their perpetual continencie, beſt.}
Which is beſt fulfilled of them that by mutual conſent doe wholy
conteine, whether they haue had children or none, contemning carnal
iſſue for the ioyes of Heauen. And theſe marriages be more bleſſed then
any other, ſaith S.~Auguſtine.
\Cite{de Ser. Do. in monte li.~1. cap.~14.}}
as though they had not; \V and they that weep, as though they wept not;
and they that reioyce, as though they reioyced not; and they that buy,
as though they poſſeſſed not; \V and they that vſe this world, as though
they vſed it not. For the figure of this world paſſeth away. \V But I
would haue you to be without carefulnes. He that is without a wife, is
\LNote{Careful for the things of our Lord.}{The
\MNote{Virginitie preferred, and why.}
Proteſtãts might here learne if they liſt, firſt that virginitie is not
onely preferred before marriage, for that it is a more quiet ſtate of
life in this world, but for that it is more conuenient for the ſeruice
of God. Secondly that virginitie hath a grateful puritie and ſanctitie
both of body & ſoule, which marriage hath not.
Thirdly, they may learne the cauſe why the Church of God requireth
chaſtitie in the Clergie, and forbiddeth not onely fornication, but al
carnal copulation euen in lawful wedlocke.
\MNote{Why cõtinencie is required in the Clergie.}
Which is not onely to the end
that God's Prieſts be not diuided from him by the clogges of marriage,
but alſo that they be cleane and pure from the fleshly actes of
copulation.}
careful for the things that pertaine to our Lord, how he may pleaſe
God. \V But he that is with a wife, is careful for the things that
pertaine to the world, how he may pleaſe his wife: and he is deuided. \V
And the womã vnmarried & the virgin, thinketh on the things that
pertaine to our Lord: that ſhe may be holy both in body and in
ſpirit. \V But ſhe that is married, thinketh on the things that pertaine to
the world, how she may pleaſe her husband. \V And this I ſpeake to your
profit: not to caſt a ſnare vpon you, but to that which is honeſt, &
that may giue
%%% 2684
you power without impediment to attend vpon our Lord. \V But if any man
thinke that he ſeemeth deſhonoured vpon his virgin, for that ſhe is paſt
age, and if it muſt ſo be, let him doe that he wil. He ſinneth not if
ſhe marrie. \V For he that hath determined in his hart being ſettled,
not hauing neceſſitie, but
%%% o-2539
hauing power of his owne wil, and hath iudged this in his hart, to keep
his virgin, doeth wel. \V Therfore both he that ioyneth his virgin in
matrimonie, doeth wel: and he that ioyneth not, doeth better.

\V
\CNote{\XRef{Ro.~7,~1.}}
A woman is bound to the law ſo long time as her husband liueth: but if
her husband ſleep, ſhe is at libertie: let her marrie to whom ſhe wil:
only in our Lord. \V But
\SNote{The ſtate of widowhood more bleſſed, thẽ the ſtate of
matrimonie.}
more bleſſed ſhal ſhe be, if ſhe ſo remaine, according to my
counſel. And I thinke that I alſo haue the Spirit of God.


\stopChapter


\stopcomponent


%%% Local Variables:
%%% mode: TeX
%%% eval: (long-s-mode)
%%% eval: (set-input-method "TeX")
%%% fill-column: 72
%%% eval: (auto-fill-mode)
%%% coding: utf-8-unix
%%% End:

