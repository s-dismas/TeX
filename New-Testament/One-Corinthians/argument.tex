%%%%%%%%%%%%%%%%%%%%%%%%%%%%%%%%%%%%%%%%%%%%%%%%%%%%%%%%%%%%%%%%%
%%%%
%%%% The (original) Douay Rheims Bible 
%%%%
%%%% New Testament
%%%% Epistles
%%%% One Corinthians
%%%% Argument
%%%%
%%%%%%%%%%%%%%%%%%%%%%%%%%%%%%%%%%%%%%%%%%%%%%%%%%%%%%%%%%%%%%%%%




\startcomponent argument


\project douay-rheims


%%% 2671
%%% o-2524
\startArgument[
  title={\Sc{The Argvment of the First Epistle to the Corinthians.}},
  marking={Argument of One Corinthians}
  ]

How S.~Paul planted the Church at Corinth, continuing there a yeare and
an halfe together, we read
\XRef{Act.~18.}
After that, when he was at Epheſus
\XRef{Act.~19.}
about the end of the three yeares that he abode there, he wrote this
firſt Epiſtle to the Corinthians. For euen as S.~Luke there writteth:
\CNote{\XRef{Act.~19,~21.}}
\Emph{When theſe things were ended, Paul purpoſed in the Spirit, when he
had gone ouer Macedonia and Achaia, to goe to Hieruſalem}: ſo likewiſe
doth S.~Paul himſelfe write here:
\CNote{\XRef{1.~Cor.~15,~5.}}
\Emph{I wil come to you} in Achaia \Emph{when I ſhal haue gone ouer
Macedonia, for I wil goe ouer Macedonia: but I wil tarie at Epheſus
vntil Pentacoſt.}

The matter that he writeth of, is not one, as is the Epiſtle to the
Romanes, but diuers. Partly ſuch faults of theirs, as were ſignified
vnto him \Emph{by them that were of Chloe}.
\XRef{1.~Cor.~1,~11.}
Partly ſuch queſtions as themſelues wrote to him of:
\Emph{And concerning the things that you wrote to me}.
\XRef{1.~Cor.~7,~1.}
For ſo we may (as it ſeemeth) diuide the Epiſtle into theſe two
parts. Or, to put al together, he writeth of eight things: 1.~Of certaine
Schiſmes beginning among them, by occaſion of certaine Preachers, whom
in the Second Epiſtle he toucheth more plainely, as being
Falſe-apoſtles.
\XRef{chap.~1.~2.~3.~4.}
2.~Of an inceſtuous fornicatour, and ſome that went to law before
infidel iudges.
\XRef{chap.~5.~6.}
3.~Of Matrimonie and Continencie.
\XRef{chap.~7.}
4.~Of meats ſacrificed to Idols.
\XRef{chap.~8.~9.~10.}
5.~Of his Traditions.
\XRef{chap.~11.}
6.~Of the Guifts of the Holy Ghoſt.
\XRef{chap.~12.~13.~14.}
7.~Of the Reſurrection.
\XRef{chap.~15.}
8.~Of the Contributions that he gathered of the Gentils, to ſuccour the
Chriſtian Iewes at Hieruſalem.
\XRef{chap.~16.}


\stopArgument


\stopcomponent


%%% Local Variables:
%%% mode: TeX
%%% eval: (long-s-mode)
%%% eval: (set-input-method "TeX")
%%% fill-column: 72
%%% eval: (auto-fill-mode)
%%% coding: utf-8-unix
%%% End:
