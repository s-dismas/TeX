%%%%%%%%%%%%%%%%%%%%%%%%%%%%%%%%%%%%%%%%%%%%%%%%%%%%%%%%%%%%%%%%%
%%%%
%%%% The (original) Douay Rheims Bible 
%%%%
%%%% New Testament
%%%% One Corinthians
%%%% Chapter 03
%%%%
%%%%%%%%%%%%%%%%%%%%%%%%%%%%%%%%%%%%%%%%%%%%%%%%%%%%%%%%%%%%%%%%%




\startcomponent chapter-03


\project douay-rheims


%%% 2675
%%% o-2529
\startChapter[
  title={Chapter 3}
  ]

\Summary{If they wil not be carnal ſtil, they muſt boaſt in God only,
  and not in their Preachers, which are but his Miniſters, 10.~and need
  to looke wel how they preach: 12.~becauſe not al preaching, though it
  be Catholike, is meritorious: but rather it buildeth matter to be
  purged by fire, when it is vaine and vnfruitful (as alſo any other
  like workes of other Catholikes.) Marie if it be heretical, deſtroying
  the Temple of God, then it worketh damnation. 18.~The remedie is, to
  humble themſelues and referre al to God.}

And I, Brethren, could not ſpeake to you as to ſpiritual, but as to
carnal. As it were to litle ones in Chriſt, \V I gaue you
\SNote{The Church only hath truth both in her milke and in her bread:
that is, whether she inſtruct the perfect, or the imperfect who are
called carnal.
\Cite{Aug. li.~15. c.~3. cont. Fauſt.}}
milke to drinke, not meate: for you could not as yet. But neither can
you now verily, for yet you are carnal. \V For whereas there is among
you emulation and contention are you not
%%% 2676
carnal, and walke according to man? \V When one ſaith: I certes am
Paules, & another: I Apollo's; are you not
\Var{men}{carnal}?
What is Apollo then? and what is Paul? \V The Miniſters of him whom you
haue beleeued, and to euery one as our Lord hath giuen. \V I planted,
Apollo watered; but God gaue the increaſe. \V Therfore neither he that
planteth is any thing, nor he that watereth; but he that giueth the
increaſe, God. \V And he that planteth and he that watereth are one. And
\LNote{Euery man shal receiue according.}{A
\MNote{Good works meritorious, and the rewards in Heauen are different
according to the ſame.}
moſt plaine text for proofe that men by their labours, and by the
diuerſities thereof, shal be diuerſly rewarded in Heauen: and therfore
that by their works proceeding of grace, they doe deſerue or merit
Heauen, and the more or leſſe ioy in the ſame. For though the holy
Scripture commonly vſe not this word merit, yet in places innumerable of
the old and new Teſtament, the very true ſenſe of merit is conteined,
and ſo often as the word, \L{merces}, and the like be vſed, they be euer
vnderſtood as correlatiues or correſpondent vnto it. For if the ioy of
Heauen be retribution, repaiment, hire, wages for works (as in infinite
places of holy Scripture,) then the works can be none other but the
valure, deſert, price, werth, and merit of the ſame. And indeed this
word, \Emph{reward}, which in our English tongue may ſignifie a
voluntary or bountiful guift, doth not ſo wel expreſſe the nature of the
\TNote{\L{Merces.}}
Latin word, or the
\TNote{\G{μισθὸς}}
Greeke, which are rather the very ſtipend that the hired worke-man or
iournie-man couenanteth to haue of him whoſe worke he doth, and is a
thing equally and iuſtly anſwering to the time and weight of his trauels
and works (in which ſenſe the Scripture ſaith:
\CNote{\XRef{1.~Tim.5,~18.}}
\L{Dignus eſt operarius mercede ſua}, the worke-man is worthy of his
hire) rather then a free guift: though, becauſe faithful men muſt
acknowledge that their merits be the guifts and graces of God, they
rather vſe the word reward, then hire, ſtipend, or repaiment: though
indeed it be al one, as you may ſee by diuers places of holy writ, as,
%%% !!! All three of these go here?
\CNote{\XRef{Apoc.~22,~13.}
\XRef{Mat.~16,~28.}
\XRef{Ro.~2,~6.}}
\Emph{My merces} (reward) \Emph{is with me to render to euery one
%%% This looks like some greek from Romans and Eccleſiasticus ---
%%% (See below TNote)
\TNote{\G{κατὰ τὰ ἔργα}}
according to his works.} And, \Emph{Our Lord wil render vnto me
according to my iuſtice.}
\XRef{Pſ.~17.}
And the very worde \Emph{merit} (equiualent to the Greek) is vſed thus:
\Emph{Mercie shal make a place to euery one
%%% !!! This TNote is the same as the above
%%% \TNote{}
according to the merit of works.}
\XRef{Eccle.~16,~15.}
And, \Emph{If you doe your iuſtice before men, you shal not haue reward
in Heauen.}
\XRef{Mat.~6,~1.}
Where you ſee that the reward of Heauen is recompenſe of iuſtice. And
the euaſion of the Heretikes is friuolous and euidently falſe, as the
former and like words doe conuince: for they ſay Heauen is
our \L{Merces} or reward, not becauſe it is due to our works, but to the
promiſe of God; where the words be plaine, \Emph{According to euery
man's works} or \Emph{labours}: vpon which works, and for which works
conditionally, the promiſe of Heauen was made.}
euery one shal receiue his owne reward according to his owne labour. \V
For we are God's
\SNote{A maruelous dignitie of ſpiritual Paſtours, that they be not only
the inſtruments or Miniſters of Chriſt (as Caſtal, noteth Beza falſly
tranſlateth \L{Adminiſtri}, for \L{Coadiutores}) but indeed God's
Coadiutours in the worke of Saluation.}
\TNote{\G{συνεργοί}}
Coadiutours; you are God's huſbandrie, you are God's building. \V
According to the grace that is giuen me, as a wiſe Worke-maſter haue I
laid the foundation: and another buildeth thereupon. But let euery one
look how he buildeth thereon. \V For other foundation no man can lay,
beſide that which is laid; which is \Sc{Christ Iesvs}. \V And if any man
build
\LNote{Vpon this foundation.}{The
\MNote{Building of gold, or ſtubble.}
foundation is Chriſt, and faith in him working by charitie. The vpper
building may be either pure and perfect matter of gold, ſiluer, and
pretious ſtone, which (according to the moſt authentical and probable
expoſition) be good workes of charitie, and al Chriſtian iuſtice done by
God's grace: or els, wood, hay, ſtubble, which ſignifie the manifold
actes of man's infirmitie and his venial ſinnes; which more or leſſe
mixed & medled with the better matter aforeſaid, require more or leſſe
punishment or purgation at the day of our death. At which day, if by
penance or other meanes in the Church, the ſaid venial ſinnes be
before-hand cleanſed, there shal need no purging at al, but they
shal ſtraight receiue the reward due to them.}
vpon this foundation, gold, ſiluer, pretious ſtones, wood, hay,
ſtubble, \V the worke of euery one
\LNote{Shal be manifeſt.}{Whether
\MNote{Our works shal be manifeſted by fire.}
our life and works be pure and need no cleanſing, now in this world is
hard to iudge: but the day of our Lord, which is at our death, wil make
it plaine in what termes euery man's life is towards God. For then
Purgatorie fire shal reueale and proue it. For, whoſoeuer hath any
impure matter of venial ſinnes or ſuch other debts, to God's iuſtice
paiable and purgable, muſt into that fire, and after due paiment and
cleanſing, be ſaued through the ſame. Where the works of the perfect men
& ſuch as died with al debts paied, cleanſed, or forgiuen, are quitted
from the fire, and neuer incurre damage, paine, or loſſe thereby. The
places of Fathers expounding this for Purgatorie, be very many moſt
euident, which are cited in the
\XRef{laſt Annotation following}.}
ſhal be manifeſt: for
\LNote{The day of our Lord shal declare.}{That this purgation rather
ſignifieth the place of God's iuſtice after our death, then any
affliction in this life, the Apoſtles preciſe ſpecifying of fire
declareth, and of reuealing and notifying the difference of mens works
by the ſame: which is not done euidently euer in this life: and namely
the word, \Emph{day of our Lord};
\MNote{What is ſignified by, \Emph{the day of our Lord}.}
which commonly and properly ſignifieth in Scripture & namely in this
Apoſtle 
\XRef{(1.~Cor.~5,~5.}
\XRef{2.~Cor.~1,~13.}
\XRef{Philip.~1,~10.}
\XRef{1.~Theſ.~5,~2.}
\XRef{2.~Theſ.~2,~2.})
either the particular, or the general iudgement: and therfore that the
trail ſpoken of, is not properly nor literally meãt any afflictiõ or
aduerſitie of this life, as 
\CNote{\Cite{Caluin in hunc locum.}}
Caluin alſo cõfeſſeth, coyning a foolish new conſtruction of his
owne. Where you may note alſo in that man's Cõmẽtarie, that this word,
\L{dies Domini}, was ſo preiudicial againſt him and al other expoſitions
of the trial to be made in this world, that he would gladly haue
(\L{Domini}) out, reading thus, \Emph{A day shal come which shal open
&c.} Where vnderſtand, that if it were only \L{Dies} (as
\TNote{\G{ἡ ἡμέρα}}
in the Greeke) yet thereby alſo the Scripture is wont to ſignifie the
ſelfe ſame thing: as,
\XRef{2.~Tim.~1,~22.~28.}
and
\XRef{2.~Tim.~4,~8.}
and
\XRef{Heb.~10,~25.}
\Emph{the day}, as in this place, with the Greeke article only, which is
al one with \L{Dies illa}, or \L{Dies Domini}.}
the day of our Lord wil declare, becauſe it ſhal be reuealed in fire:
and the worke of euery one of what
%%% o-2530
kind it is, the fire ſhal trie. \V If
any man's worke abide, which he built therupon; he ſhal receiue
reward. \V If any man's worke burne, he ſhal ſuffer detriment: but
himſelf ſhal be ſaued: yet ſo
\LNote{As by fire.}{
\Cite{S.~Auguſtin vpon theſe words of the Pſalme.~37.}
\Emph{Lord rebuke me not in thine indignation, nor amend me in thy
wrath.} For it shal \Emph{come to paſſe} (ſaith he) \Emph{that ſome be
amended in the wrath of God and be rebuked in his indignatiõ.
\MNote{Two fires after this life: one eternal, the other temporal, that
is, the purging or amending fire.}
And not al
perhaps that are rebuked, shal be amended, but yet ſome there shal be
ſaued by amending. It shal be ſo ſurely, becauſe amending is named: yet
ſo as by fire. But ſome there shal be rebuked, and not amended; to
whom he shal ſay: Goe ye into euerlaſting fire. Fearing therfore theſe
more greiuous paines, he deſireth that he may neither be rebuked in
indignation by eternal fire, nor amended in his wrath; that is to ſay:
Purge me in this life, and make me ſuch an one as shal not need the
amending fire; being for them which shal be ſaued, yet ſo as by
fire. Wherfore? but becauſe here they build vpon the foundation, wood,
hay, ſtubble? For if they did build gold, ſiluer, and pretious ſtones,
they should be ſecure from both fires, not only from that eternal which
shal torment the impious eternally; but alſo from that which shal amend
them that shal be ſaued by fire. For it is ſaid:} He shal be ſafe, yet
ſo as by fire. \Emph{And becauſe it is  ſaid, he shal be ſafe, that fire
is contemned. Yea verily though ſafe by fire,
\MNote{Purgatorie fire paſſeth al the paines of this life.}
yet that fire shal be more
grieuous, then whatſoeuer a man can ſuffer in this life. And you know
how great euils the wicked haue ſuffred, and may ſuffer: yet they haue
ſuffred ſuch as the good alſo might ſuffer. For what hath any
malefactour ſuffred by the lawes, that a Martyr hath not ſuffred in the
confeſſion of Chriſt? Theſe euils therfore that are here, be much more
eaſie: and yet ſee how men, not to ſuffer them, doe whatſoeuer thou
commandeſt. How much better doe they that which God commandeth, that
they may not ſuffer thoſe greater paines?} Thus farre S.~Auguſtin.  See
\Cite{S.~Ambbr. vpon this place. 1.~Cor.~3.}
&
\Cite{Ser.~20. in Pſal.~118.}
\Cite{Hiero. li.~2. c.~13. adu Iouinianum.}
\Cite{Greg. li.~4. Dialog. c.~19}
&
\Cite{in Pſal.~3. Pænit. in principio.},
\Cite{Origen ho.~6. in c.~15. Exod.}
and
\Cite{ho.~14. in c.~24. Leuit.}}
as by fire. \V Know you not that you are the Temple of God; and the
Spirit of God dwelleth in you? \V But if any violate the Temple of God,
God wil deſtroy him. For the Temple of God is holy: which you are. \V
Let no man ſeduce himſelf: if any man ſeeme to be wiſe among you in this
world, let him become a foole that he may be wiſe. \V For the wiſedom of
this world is fooliſhnes with God. For it is written:
\CNote{\XRef{Io.~5,~13.}}
\Emph{I wil compaſſe the wiſe in their ſubteltie.} \V And againe:
\CNote{\XRef{Pſ.~93,~11.}}
\Emph{Our Lord knoweth the cogitations of the wiſe that they be vaine.}
\V Let no mã therfore glorie in men. For al things are yours: \V whether
it be Paul, or Apollo, or Cephas, or the world, or life, or death, or
things preſent, or things to come; for al are yours: \V and you are
Chriſt's, and Chriſt is God's.


\stopChapter


\stopcomponent


%%% Local Variables:
%%% mode: TeX
%%% eval: (long-s-mode)
%%% eval: (set-input-method "TeX")
%%% fill-column: 72
%%% eval: (auto-fill-mode)
%%% coding: utf-8-unix
%%% End:

