%%%%%%%%%%%%%%%%%%%%%%%%%%%%%%%%%%%%%%%%%%%%%%%%%%%%%%%%%%%%%%%%%
%%%%
%%%% The (original) Douay Rheims Bible 
%%%%
%%%% New Testament
%%%% Epistles
%%%% Two Corinthians
%%%% Argument
%%%%
%%%%%%%%%%%%%%%%%%%%%%%%%%%%%%%%%%%%%%%%%%%%%%%%%%%%%%%%%%%%%%%%%




\startcomponent argument


\project douay-rheims


%%% 2712
%%% o-2568
\startArgument[
  title={\Sc{The Argvment of the Second Epistle to the Corinthians.}},
  marking={Argument of Two Corinthians}
  ]

For the time when this Epiſtle was written, looke the
\XRef{Argument of the epiſtle to the Romanes}:
to wit, about the eighteenth yeare after his conuerſion, & our Lordes
paſsion, becauſe
\CNote{\XRef{2.~Cor.~11,~1.}}
in the
\XRef{11.~chapter}
he maketh mention of 14.~yeares, not only after his conuerſion, as
\CNote{\XRef{Gal.~2,~1.}}
to the Galatians, but alſo after his rapt, which ſeemeth to haue been
when he was at Hieruſalem
\XRef{Act.~9,~26.}
foure yeares after his conuerſion
\XRef{(Gal.~1,~18.)}
in a trance or exceſſe of mind, as he calleth it,
\XRef{Act.~22,~17.}
It was written at Troas (it is thought) and ſent by Titus, as we read
\XRef{chap.~8.}

It is for the moſt part againſt thoſe falſe Apoſtles whom in the firſt
part of the firſt to the Corinthians, he noted, or rather ſpared, but
now is conſtrained to deale openly againſt them, & to defend both his
owne perſon which they ſought to bring into contempt, making way thereby
to the correption of the Corinthians, and withal to mainteine the
excellencie of the Miniſterie and Miniſters of the new Teſtament, aboe
which they did magnifie the Miniſterie of the old Teſtament: bearing
themſelues very high becauſe they were Iewes.

Againſt theſe therfore S.~Paul auoucheth the preeminent power of his
Miniſterie, by which power alſo he giueth a pardon to the inceſtuous
fornicatour whom he excommunicated in the laſt epiſtle, ſeeing now his
penance, and againe threatneth to come & excommunicate thoſe that had
grieuouſly ſinned and remained impenitent. Two chapters alſo he
interpoſeth of the contributions to the Church of Hieruſalem, mentioned
in his laſt, exhorting them to doe liberally, and alſo to haue al in a
readines againſt his comming.


\stopArgument


\stopcomponent


%%% Local Variables:
%%% mode: TeX
%%% eval: (long-s-mode)
%%% eval: (set-input-method "TeX")
%%% fill-column: 72
%%% eval: (auto-fill-mode)
%%% coding: utf-8-unix
%%% End:
