%%%%%%%%%%%%%%%%%%%%%%%%%%%%%%%%%%%%%%%%%%%%%%%%%%%%%%%%%%%%%%%%%
%%%%
%%%% The (original) Douay Rheims Bible 
%%%%
%%%% New Testament
%%%% Epistles
%%%% Two Corinthians
%%%% Chapter 01
%%%%
%%%%%%%%%%%%%%%%%%%%%%%%%%%%%%%%%%%%%%%%%%%%%%%%%%%%%%%%%%%%%%%%%




\startcomponent chapter-01


\project douay-rheims


%%% 2713
%%% o-2569
\startChapter[
  title={Chapter 1}
  ]

\Summary{By his troubles in Aſia he comforteth them: and (againſt his
  aduerſaries the falſe Apoſtles of the Iewes) alleageth to them the
  teſtimonie of his owne and alſo of their conſcience, 17.~anſwering
  them that obiected lightnes againſt him, for not comming to Corinth
  according to his promiſe.}

Paul an Apoſtle of \Sc{Iesvs} Chriſt by the wil of God, and Timothee our
Brother: to the Church of God that is at Corinth, with al the Saints
that are in al Achaia. \V Grace vnto you and peace from God our Father,
& from our Lord \Sc{Iesvs} Chriſt. \V Bleſſed be the God and Father of
our Lord \Sc{Iesvs} Chriſt, the Father of mercies and God of al
comfort, \V who comforteth vs in al our tribulations; that we alſo may
be able to comfort them that are in al diſtreſſe,
\TNote{\G{διὰ τῆς παρακλήσεως ἧς παρακαλούμεθα}}
by the exhortatiõ wherwith we alſo are exhorted of God. \V For as the
\LNote{Paſsions of Chriſt.}{Al
\MNote{Al our afflictions, be Chriſt's afflictions, for the coniunction
between the head & body.}
the afflictions of the faithful be called Chriſtes owne paſſions not
only becauſe they be ſuffered for him, but for that there is ſo ſtrait
coniunction & communion betwixt him being the head, & euery of the
liuing members of his body, which is the Church, that whatſoeuer is
ſuffered by any one of the ſame, is counted as a peece of his owne
Paſſion. As likewiſe whatſoeuer good workes be done to any of them, or
by any of them be accepted as done to, or by Chriſt himſelf.
\MNote{Merit and ſatisfaction.}
Which thing if the Proteſtants wel weighed, they would not maruel that
the Catholike Church attributeth ſuch force of merit & ſatisfaction to
the worke of holy men.}
paſſiõs of Chriſt abound in vs: ſo alſo by Chriſt doth our
\LNote{The comfort abound.}{Worldly
\MNote{Worldly men feel not the comfort of afflicted Catholikes.}
men that ſee only the exteriour miſeries and afflictions that Catholikes
doe ſuffer being perſecuted by the Heathen or Heretikes, deeme them
exceeding miſerable. But if they felt or could conceiue the aboundance
of conſolation which Chriſt euer giueth according to the meaſure of
their afflictions, they would neuer wonder at the voluntary toleration
of what torments ſoeuer for Chriſtes ſake, but would wiſh rather
themſelues to be in any dungeon in England with the comfort that ſuch
haue from God, then to liue out of the Church in al the wealth of the
world.}
comfort abound. \V And whether we be in tribulation, for your
exhortation & ſaluatiõ: whether we be exhorted, for your exhortation &
ſaluation, which worketh the toleratiõ of the ſame paſſions which we
alſo doe ſuffer: \V and our hope is firme for you: knowing that as you are
partakers of the paſſions, ſo ſhal you be of the conſolation alſo.

\V For we wil not haue you ignorant, Brethren, cõcerning our tribulation
which hapned in Aſia, that we were preſſed aboue meaſure aboue our
power, ſo that it was tedious vnto vs euen to liue. \V But we in our
ſelues, had the anſwer of death, that we be not truſting in our ſelues,
but in
%%% o-2570
God who raiſeth vp the dead, \V who hath deliuered and doth deliuer vs
out of ſo great dangers: in whom we hope that he wil yet alſo deliuer
vs, \V you
\LNote{You helping in praier.}{S.~Paul
\MNote{Interceſſion of Saints or holy men for vs, no derogation to
Chriſt.}
knew that the help of other mens praiers was nothing derogatorie to the
office of Chriſtes mediation or interceſsion for him, nor to the hope
that he had in God: and therfore he craueth the Corinthians aid herin as
a ſupport and ſuccours for himſelf in the ſight of God. With what reaſon
or Scripture then can the Proteſtants ſay, that the praiers of Saints be
iniurious to Chriſt, or not to ſtand with the confidence we haue in him?
As though it were more diſhonour to God that we ſhould vſe the aid of
Saintes in heauen then of ſinners in earth: or
\CNote{Hierom. cont. Vigil.}
that the interceſsion of theſe our fellowes beneath, were more auailable
then the praiers of thoſe that be in the glorious ſight of God aboue.}
helping withal in praier for vs, that
\LNote{By many mens.}{He
\MNote{Publike prayers & faſts.}
meaneth, that as the praiers of many ioyned together for him, shal be
rather heard, then of any alone; ſo their common thankes-giuing to God
for granting their requeſt, ſhal be more acceptable & glorious to God,
then any one man's thankes alone. Which thing doth much commend the holy
Churches publike praiers, proceſsions, ſtations, and pilgrimages, where
ſo many meet and vniformly ioyne their praiers and lauds together vnto
God.}
by many mens perſons, thankes for that guift which is in vs, may be
giuen by many in our behalfe. \V For our glorie is this, the teſtimonie
of our
%%% 2714
conſcience, that in ſimplicity and ſincerity of God, and not in carnal
wiſedom, but in the grace of God we haue conuerted in this world: and
more aboundantly towards you. \V For we write no other things to you
then that you haue read and know. And I hope that you ſhal know vnto the
end: \V as alſo you haue knowen vs in part, that we are
\LNote{Your glorie.}{The
\MNote{Their glorie in heauen that conuert others.}
Apoſtles, Teachers, and Preachers, that conuert countries or particular
perſons to Chriſt, & the peoples or parties by them conuerted, ſhal in
the day of iugement haue much mutual ioy and glorie of and for each
other; one giuing to the other great matter of merit in this life, and
afterward ioy in the next. See
\XRef{1.~Theſſ.~2. v.~19.}}
your glorie, as you alſo ours in the day of our Lord \Sc{Iesvs}
Chriſt. \V And in this confidence I would firſt haue come to you, that
you might haue a ſecond grace: \V and by you paſſe into Macedonia, and
againe from Macedonia come to you, and of you be brought on my way into
Iewrie.

\V Wheras then I was thus minded, did I vſe lightnes? Or the things that
I mind, doe I mind according to the fleſh, that there be with me,
\Emph{It is} and \Emph{It is not}? \V But God is faithful, becauſe our
preaching which was to you, there is not in it,
\LNote{It is, it is not.}{As
\MNote{The Proteſtãts inconſtancie in changing their writings,
tranſlations, ſeruice books &c.}
he diſchargeth himſelf of al other leuitie touching his promiſe or
purpoſe of comming to them, ſo much more of al inconſtancie in preaching
Chriſtes doctrine and faith; wherin one day to affirme, another day to
deny, to diſſent from his fellowes or from himſelf, to change euery yeare
or in euery epiſtle the forme of his former teaching, to come daily with
new deuiſes repugnant to his owne rules, were not agreable to an Apoſtle
and true Teacher of Chriſt, but proper to
\Fix{falſſe-Prophets}{falſe-Prophets}{obvious typo, fixed in other}
& Heretikes. Wherof we haue notorious examples in the Proteſtants: who
being deſtitute of the Spirit of peace, concord, conſtancie, vnitie, &
veritie, as they varie from their owne writings which they retract,
reforme, or deforme continually, ſo both in their preaching & forme or
Seruice, they are ſo reſtles, changeable, and repugnant to themſelues,
that if they were not kept in awe with much adoe, by temporal lawes, or
by the shame and rebuke of the world, they would coine vs euery yeare or
euery Parliament new Communions, new faithes, and new Chriſtes, as you
ſee by the manifold endeauours of the Puritans. And this to be the
proper note of falſe Apoſtles and Heretikes, ſee in
\Cite{S.~Irenæus li.~1. c.~18.}
and
\Cite{Tertull. de præſcrips. S.~Baſil. ep.~12.}}
\Emph{It is}, and \Emph{It is not}. \V For the Sonne of God \Sc{Iesvs}
Chriſt, who by vs was preached among you, by me and Syluanus and
Timothee, was not, \Emph{It is}, and \Emph{It is not}, but \Emph{It is},
was in him. \V For al the promiſes of God that are, in him \Emph{It is}:
therfore alſo by him, Amen to God, vnto our glorie. \V And he that
confirmeth vs with you in Chriſt, and that hath anointed vs, God: \V who
alſo
\LNote{Hath ſealed.}{The
\MNote{The indeleble Character of Baptiſme, Cõfirmation, Holy Orders.}
learned Diuines proue by \Fix{his}{this}{obvious typo, fixed in other}
place & by
\CNote{\XRef{Eph.~4,~30.}}
the like in the fourth to the Epheſians, that the Sacrament of Baptiſme
doth not only giue grace, but imprinteth & ſealeth the ſoule of the
Baptized, with a ſpiritual ſigne, marke, badge, or token, which can
neuer be blotted out, neither by ſinne, hereſie, apoſtaſie, nor other
waies, but remaineth for euer in man for the cogniſance of his
Chriſtendome, & for diſtinction from other which were neuer of Chriſtes
fold. By which alſo he is as it were conſecrated and deputed to God,
made capable and partaker of the rightes of the Church, and ſubiect to
her lawes and diſcipline. See
\Cite{S.~Hierom. in 4.~Epheſ.}
\Cite{S.~Ambroſe li.~1. de Sp. Sancto cap.~6.}
\Cite{S.~Cyril. Hieroſol. Cathecheſi.~17. at the end,}
and
\Cite{S.~Dionyſius Areopag. c.~2. Eccle. Hierarch.}
The which Fathers expreſſe that ſpiritual ſigne by diuers agreable
names, which the Church and moſt Diuines, after S.~Auguſtin, cal the
\Emph{Character} of Baptiſme. By the truth and force of which ſpiritual
note or marke of the ſoul, he ſpecially conuinceth the Donatiſtes, that
the ſaid Sacrament though giuen and miniſtred by Heretikes or
Schiſmatikes or who els ſoeuer, can neuer be reiterated. See
\Cite{ep.~57.}
&
\Cite{l.~6. cont. Donat. c.~8.}
&
\Cite{li.~2. cont. Parmenian c.~13.}
As the like indeleble Characters giuen alſo by the Sacrament of
Confirmation and Orders, doe make thoſe alſo irreiterable and neuer to
be receiued but once. Wheras al other Sacraments ſauing theſe three, may
be often receiued of the ſelf-ſame perſon. And that holy Orders can not
be iterated, ſee S.~Auguſtin in
\Cite{li.~2. cont. Parmen. c.~11.}
\Cite{li. de bono coniug. c.~24.}
&
\Cite{S.~Gregorie li.~2. Regiſt. ep.~32.}
The like of Confirmation is decreed in the moſt ancient Councel 
\CNote{See \Cite{conc. Tarrac. to.~2. concil.}}
\Cite{Tarracon cap.~6.}
Finally that this Character is giuen only by theſe ſaid three
Sacraments, & is the cauſe that none of them can be in any man repeated
or reiterated, ſee the decrees of the Councels
\Cite{Florentine}
&
\Cite{Trent.}
Which yet is no new deuiſe of them, as the Heretikes falſely affirme,
but agreable (as you ſee) both to the Scriptures, and alſo to the
ancient Fathers & Councels.}
hath ſealed vs, and giuen the pledge of the Spirit in our harts. \V And
I cal God to witneſſe vpon my ſoul, that ſparing you, I came not any
more to Corinth, \V 
%%% !!! marked only in other
\LNote{Not becauſe we ouer-rule.}{Caluin
\MNote{The Caluiniſts wil be ſubiect to no tribunal in earth for trial
of their religion.}
and his ſeditious Sectaries with other like \Emph{which deſpiſe dominion}, as
S.~Iude deſcribeth ſuch, would by this place deliuer themſelues from al
yoke of ſpiritual Magiſtrates and Rulers: namely that they be ſubiect to
no man touching their faith, or for the examination and trial of their
doctrine, but to God and his word only. And no maruel that the
malefactours and rebelles of the Church would come to no tribunal but
God's, that ſo they may remaine vnpuniſhed at leaſt during this
life. For though the Scriptures plainely condemne their hereſies, yet
they could writh themſelues out by falſe gloſſes, conſtructions,
corruptions, and denials of the books to be Canonical, if there were no
lawes or iudicial ſentence of men to rule and repreſſe them.

Notwithſtanding
\MNote{Tyrannical dominiõ is forbid in Prelates, not Eccleſiaſtical
Soueraigntie for examinatiõ of faith or manners.}
then theſe wordes of S.~Paul, whereby only tyrannical,
inſolent, and proud behauiour & indiſcrete rigour of Prelates or
Apoſtles towards their flocks is noted, as alſo in the
\XRef{firſt of S.~Peter cap.~5.}
(the
\TNote{\G{κατακυριεύουσιν}}
Greek word in theſe places, and in the Goſpel
\XRef{Mt.~20,~25.}
\XRef{Mr.~10,~42.}
ſignifying lordly & inſolent dominion:) yet he had & exerciſed iuſt
rule, preeminence, & prelacie ouer them, not only for their life, but
alſo & principally touching their faith. For he might and did cal them
to account for the ſame, and excommunicated heretikes for foreſaking
their faith
\XRef{1.~Cor.~4,~5.}
\XRef{2.~Cor.~10,~4.}
\XRef{13,~10.}
\XRef{1.~Tim.~1,~20.}
\XRef{Tit.~1,~11.}
And al Chriſtian men are bound to obey their lawful Prelates in matters
of faith and doctrine ſpecially, and muſt not vnder that ridiculous
pretence of obeying God's word only (which is the shift of al other
Heretikes, as Anabaptiſts, Arians, and the like, as wel as the
Proteſtants) diſobey God's Church, Councels, and their owne Paſtours and
Bishops, who by the Scriptures haue the regiment of their ſoules, and
may examine and punish as wel Iohn Caluin as Simon Magus, for falling
from the Catholike faith. For though God alone be the Lord author and giuer
of faith, yet they are his
\TNote{\G{συνεργοὶ}}
cooperatours and coadiutours by whom the faithful doe beleeue & be
preſerued in the true faith, and be defended from wolues, which be
Heretikes, ſeeking to corrupt them in the ſame. And this ſame Apoſtle
\CNote{\XRef{1.~Cor.~3,~9.}
\XRef{\XRef{1.~Cor.~4,~15.}}}
chalengeth to be their father, as he that begat and formed them by his
preaching in Chriſt.}
not becauſe we ouer-rule your faith: but, we are
helpers of your ioy. For in the faith you ſtand.


\stopChapter


\stopcomponent


%%% Local Variables:
%%% mode: TeX
%%% eval: (long-s-mode)
%%% eval: (set-input-method "TeX")
%%% fill-column: 72
%%% eval: (auto-fill-mode)
%%% coding: utf-8-unix
%%% End:

