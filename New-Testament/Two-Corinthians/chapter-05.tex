%%%%%%%%%%%%%%%%%%%%%%%%%%%%%%%%%%%%%%%%%%%%%%%%%%%%%%%%%%%%%%%%%
%%%%
%%%% The (original) Douay Rheims Bible 
%%%%
%%%% New Testament
%%%% Two Corinthians
%%%% Chapter 05
%%%%
%%%%%%%%%%%%%%%%%%%%%%%%%%%%%%%%%%%%%%%%%%%%%%%%%%%%%%%%%%%%%%%%%




\startcomponent chapter-05


\project douay-rheims


%%% 2723
%%% o-2579
\startChapter[
  title={Chapter 5}
  ]

\Summary{That after death of the body the ſoule may goe to heauẽ:
  therfore, although naturally we abhorre death, by grace he deſireth it
  rather: 9.~in conſideratiõ of Chriſtes iuſt iudgement, liuing as in
  the ſight of God, yea and of their conſciences. 12.~Which he ſpeaketh
  not to praiſe himſelf, but becauſe of his Aduerſaries who did glorie
  in carnal reſpects: but he and the other Apoſtles regard nothing but
  their reconciliation vnto God by Chriſt, and to reconcile others alſo,
  as being his Legates for that purpoſe.}

For we know that if our earthly houſe of this habitation be diſſolued,
that we haue a building of God, a houſe not made with hand, eternal in
Heauen. \V For in this alſo doe we grone, deſirous to be ouer-clothed
with our habitation that is from Heauen: \V yet ſo, if we be found
clothed, not naked. \V For we alſo that are in this tabernacle, grone
being burdned: becauſe we would not be ſpoiled, but ouer-clothed, that
that which is mortal might be ſwallowed vp of life. \V And he that
maketh vs to this ſame, is God, who hath giuen vs the pledge of the
Spirit. \V Being bold therfore alwaies, and knowing that while we are in
the body, we are pilgrimes from God, (\V for we walke by faith and not
by ſight) \V but we are bold, and haue a good wil to be pilgrimes rather
from the body, &
\SNote{This place proueth that the Saints departed now ſince Chriſt,
ſleep not til the day of iudgement, and that they be not holden in any
ſeueral place of reſt from the fruition of God til the reſurrection of
their bodies, but that they be preſent with God in their ſoules.}
to be preſent with our Lord. \V And therfore we endeauour, whether
abſent or preſent, to pleaſe him. \V For
\CNote{\XRef{Ro.~14,~10.}}
we muſt al be manifeſted before the iudgement ſeat of Chriſt, that euery
one may receiue
\LNote{The proper things of his body.}{S.~Auguſtin
\MNote{The obiection againſt praiers for the dead, anſwered by
S.~Auguſtin.}
\Cite{(Enchirid. c.~110.)}
obiecteth this ſpeach of the Apoſtle, as in the perſon of ſuch as deny
the praiers, almes, and Sacrifices of the liuing to be auailable for the
dead, and he anſwereth as followeth: \Emph{This practiſe} (ſaith he)
\Emph{of God's Church in the cõmendation of the dead is nothing
repugnant to the ſentence of the Apoſtle, where he ſaith, that we shal
al ſtand before the iudgement ſeat of Chriſt, that euery one may receiue
according to his deſerts in the body, either good or euil. For, in his
life and before death he deſerued this, that theſe workes after his
death might be profitable vnto him. For indeed they be not profitable
for al men. And why ſo? but becauſe of the difference and diuerſitie of
mens liues whiles they were in flesh.} The like he hath in diuers other
places. 
\Cite{Auguſt. li. de Præd. Sanct. c.~12.}
&
\Cite{ad Dulcit. q.~2.}
And ſo hath S.~Denys
\Cite{c.~7. Ec. Hierarch.}}
the proper things of the body, according as he hath done
\LNote{Either good or euil.}{Heauen
\MNote{Workes meritorious and demeritorious.}
is as wel the reward of good workes, as Hel is the ſtipend of il
workes. Neither is faith alone ſufficient to procure ſaluation, nor
lacke of faith the only cauſe of damnation: by good deeds men merit the
one, and by il deeds they deſerue the other. This is the Apoſtles
doctrine here and in other places, howſoeuer the Aduerſaries of good
life and workes teach otherwiſe.}
either good or euil. \V Knowing therfore the feare of our Lord we vſe
perſuaſion to men: but to God we are manifeſt. \V And I hope alſo that
in your conſciences we are manifeſt. \V We
%%% o-2580
commend not our ſelues againe to you, but giue you occaſion to glorie
for vs: that you may haue againſt them that glorie in face, and not in
hart. \V For whether we exceed in mind, to God: or whether we be ſober,
to you. \V For the charitie of Chriſt vrgeth vs; iudging this, that if
one died for al, then al
\Fix{weare}{were}{likely typo, fixed in other}
dead. \V And Chriſt died for al: that they alſo which liue, may not now
liue to themſelues, but to him that died for them and roſe againe. \V
Therfore we from hence-forth know no man according to the flesh. And if
we haue knowen Chriſt according to the fleſh: but now we know him no
more.

\V If then any be in Chriſt a new creature: the old are paſſed, behold
\CNote{\XRef{Eſa.~34, 19.}
\XRef{Apoc.~21,~5.}}
al things are made new. \V But al of God, who hath reconciled vs to
himſelf by Chriſt: and hath giuen
\LNote{The miniſterie of reconciliation.}{Chriſt
\MNote{Bishops and Prieſts, vnder Chriſt Miniſters of our
reconciliation.}
is the cheefe Miniſter, according to his manhood, of al our recõcilemẽt
to God: and for him, as his Miniſters the Apoſtles and their
Succeſſours, the Bishops and Prieſts of his Church, in whom the word of
reconcilement, as wel by miniſtring of the Sacrifice and Sacraments for
remiſsion of ſinnes, as by preaching and gouernement of the world to
ſaluation, is placed. And therfore their preaching muſt be to vs, as if
Chriſt himſelf did preach: their abſolution and remiſsion of ſinnes, as
Chriſtes owne pardon: their whole office being nothing els (as we ſee by
this paſſage) but the Vicarship of Chriſt.}
vs the miniſterie of reconciliation. \V For God indeed was in Chriſt
reconciling the world to himſelf, not imputing to them their ſinnes, and
hath put in vs the word of reconciliation. \V For Chriſt therfore we are
Legates, God as it were exhorting by vs. For Chriſt we beſeech you, be
reconciled to God.
%%% 2724
%%% !!! '\V Him' only on previous page, and other
\V Him that knew no ſinne, for vs he made
\SNote{That is to ſay, a Sacrifice and an Hoſt for ſinne. See the 
\XRef{laſt annot. of this chapter.}}
ſinne: that we might be made
\LNote{The iuſtice of God.}{\Emph{Euen as}
\MNote{God's iuſtice, wherwith he maketh vs iuſt.}
(ſaith S.~Auguſtin) \Emph{when we read, Saluation is our Lordes, it is
not meant that ſaluation whereby our Lord is ſaued, but whereby they are
ſaued whom he ſaueth: ſo when it is ſaid, God's iuſtice, that is not to
be vnderſtood wherwith God is iuſt, but that wherwith men are iuſt whom
by his grace he iuſtifieth.}  See S.~Auguſtin
\Cite{de Sp. & lit. c.~12.}
&
\Cite{ep.~120.}
and abhorre Caluin's wicked and vnlearned gloſſes on this place, that
teacheth iuſtice no otherwiſe to be in man, then ſinne in Chriſt. Whereas
the Scriptures cal men iuſt, becauſe
\CNote{\XRef{1.~Io.~3,~7.}}
\Emph{he doth} iuſtice: but not ſo cal they Chriſt ſinne, becauſe he
doth ſinne, but becauſe he taketh away ſinne, and is a ſacrifice for
ſinne, as the Heretikes know very wel, that know the vſe and
ſignification of the
\TNote{\H{חֲטָאָ֗ה}}
Hebrew word in al the old Teſtament, namely
\XRef{Pſal.~39,~8.}
and in the booke of Leuiticus very often
\XRef{c.~5.}
\XRef{6.}
\XRef{9.}
\XRef{12.}
\XRef{14.}
\XRef{16.}
and
\XRef{Numer. c.~29.}}
the iuſtice of God in him.


\stopChapter


\stopcomponent


%%% Local Variables:
%%% mode: TeX
%%% eval: (long-s-mode)
%%% eval: (set-input-method "TeX")
%%% fill-column: 72
%%% eval: (auto-fill-mode)
%%% coding: utf-8-unix
%%% End:

