%%%%%%%%%%%%%%%%%%%%%%%%%%%%%%%%%%%%%%%%%%%%%%%%%%%%%%%%%%%%%%%%%
%%%%
%%%% The (original) Douay Rheims Bible 
%%%%
%%%% New Testament
%%%% Two Corinthians
%%%% Chapter 03
%%%%
%%%%%%%%%%%%%%%%%%%%%%%%%%%%%%%%%%%%%%%%%%%%%%%%%%%%%%%%%%%%%%%%%

%%% Latin checked by KK.



\startcomponent chapter-03


\project douay-rheims


%%% 2719
%%% o-2575
\startChapter[
  title={Chapter 3}
  ]

\Summary{Leſt the Iudaical falſe Apoſtles should obiect againe that he
  praiſeth himſelf, he ſaith that the Corinthians are his commendation:
  and they in their harts being iuſtified by his miniſterie, he therof
  inferreth that the Miniſters of the new Teſtament are farre more
  glorious then they of the old, 12.~and our people more lightened then
  theirs.}

Begin we againe to commend our-ſelues? or doe we need (as certaine)
epiſtles of cõmendation to you, or from you? \V Our epiſtle you are,
writtẽ in our harts, which is knowẽ & read of al men: \V being
manifeſted that you are
\LNote{The Epiſtle of Chriſt.}{S.~Paul
\MNote{The Apoſtles wrote the Ghoſpel in mens harts much more then in
paper.}
and other holy Writers of Scriptures did ſet downe many things in
writing, by penne, inke, and paper, al which be of the Holy Ghoſt: but
the ſpecial and proper book of Chriſtes truth and Ghoſpel, is not the
external writing in thoſe dead creatures, but in the harts of the
faithful, being the proper ſubiect of theſe truths and graces preached
in the new Teſtament, and the habitacle of the  Holy Ghoſt. In the which
book of faithful mens harts S.~Paul wrote diuers things not vttered in
any Epiſtle: as ſundrie of the Apoſtles wrote the Chriſtian religion in
the harts of their hearers only, and in other material books not at
al. Wherof S.~Irenæus
\Cite{li.~3. c.~4.}
ſaith:
\MNote{Scripture written, and Tradition vnwrittẽ.}
\Emph{What and if the Apoſtles alſo had left no Scriptures, ought we not
to follow the order of the tradition, which they deliuered vnto them to whom
they committed the Churches? To the which ordinance many Nations of
thoſe barbarous people that haue beleeued in Chriſt, doe conſent,
without letter or inke, hauing ſaluation written in their harts, and
keeping diligently the tradition of the Elders.} And S.~Hierom
\Cite{(cont. Io. Hieroſ. c.~9. ad Pam.)}
\Emph{In the Creed of our faith and hope, which being deliuered by
tradition from the Apoſtles, is not written in paper and inke, but in
tables carnal of the hart.} And this is the Churches book alſo, whereby
and wherin ſhe keepeth faithfully al truth written in the harts of thoſe
to whom the Apoſtles did preach, with the like diligence as she keepeth
& preſerueth the other book which is of holy Scriptures, from al
corruption of Heretikes and other iniuries.}
the epiſtle of Chriſt, miniſtred by vs, & written not with inke, but
with the Spirit
%%% o-2576
of the liuing God: not in the
tables of ſtone, but in the tables carnal of the hart. \V And ſuch
cõfidẽce we haue by Chriſt to God: \V not that we be ſufficient to
thinke any thing
\LNote{Of our ſelues.}{This
\MNote{God's grace & fre-wil both muſt concurre.}
maketh firſt againſt the Heretikes called Pelagians, that hold our
meritorious actions or cogitations to be of free-wil only, and not of
God's ſpecial grace. Secondly againſt the Proteſtants, who on the
contrarie ſide referre al to God, and take away man's freedom and proper
motion in his thoughts and doings: the Apoſtle confeſsing our good
cogitations to be our owne, but not as comming of our-ſelues, but of
God.}
of our-ſelues, as of our-ſelues: but our ſufficiẽcie is of God. \V Who
alſo hath made vs meet Miniſters of the new Teſtament not in the letter,
but in the Spirit. For
\LNote{The letter killeth.}{As
\MNote{The letter killeth both Iew and Heretike.}
the letter of the old Law not truely vnderſtood, nor referred to Chriſt,
commanding and not giuing grace and ſpirit to fulfil that which was
commanded, did by occaſion kil the carnal Iew: ſo the letter of the new
Teſtament not truely taken nor expounded by the Spirit of Chriſt (which
is only in his Church) killeth the Heretike: who alſo being carnal and
void of ſpirit, gaineth nothing by the external precepts or good leſſons
of the Scriptures, but rather taketh hurt by the ſame. See S.~Auguſtin
\Cite{to.~10. Ser.~70.}
&
\Cite{100. de tempore.}
&
\Cite{li. de Sp. & lit. c.~5.~6.~&~ſeq.}}
the letter killeth: but the Spirit quickneth. \V And if the miniſtration
of death with letters figured in ſtones, was in glorie, ſo that the
children of Iſrael could not behold the face of Moyſes for the glorie of
his countenãce, that is made void; \V how ſhal not the miniſtration of
the Spirit be more in glorie?
%%% 2720
\V For if the miniſtration of damnation be in glorie,
\LNote{Much more.}{The
\MNote{The preeminence of the new Teſtamẽt, Sacramẽts, &c.}
preeminence of the new Teſtament and of the prieſthood or Miniſterie
therof before the old, is, that the new, by al her Sacraments and
Prieſts as Miniſters immediate of grace and remiſsion of ſinnes, doth ſo
\L{ex opere operato} giue the ſpirit of life and charitie into the
harts of the faithful, as the old did giue the letter or external act of
the Law.}
much more the miniſterie of iuſtice aboundeth in glorie. \V For neither
was it glorified, which in this part was glorious, by reaſon of the
excelling glorie. \V For if that which is made void, is by glorie: much
more that which abideth, is in glorie.

\V Hauing therfore ſuch hope, we vſe much confidence: \V and not
\CNote{\XRef{Exo.~34,~33.}}
as Moyſes put a veile vpon his face, that the children of Iſrael might
not behold his face, which is made void, \V but their ſenſes were
dulled. For vntil this preſent day,
\LNote{The ſelf-ſame veile.}{As
\MNote{The Heretikes more blind in not ſeeing the Church, then the Iewes
in not ſeeing Chriſt.}
the Iewes reading the old Teſtament, by reaſon of their blindnes (which
God for the punishment of their incredulitie ſuffereth to remaine as a
couer vpon their eyes and harts) can not ſee Chriſt in the Scriptures
which they daily heare read in their Synagogues, but ſhal, when they
beleeue in him and haue the couer remoued, perceiue al to be moſt
plainely done and ſpoken of him in their law & Scriptures: euen ſo
Heretikes hauing (as S.~Auguſtin
\CNote{Aug. in Pſal.~30. Conc.~2.}
noteth) a farre greater couer of blindnes and incredulitie ouer their
harts in reſpect of the Catholike Church which they impugne, then the
Iewes haue concerning Chriſt, can not ſee, though they read or heare the
Scriptures read neuer ſo much, the maruelous euidence of the Catholike
Church & truth in al points: but when they ſhal returne againe to the
obedience of the ſame Church, they ſhal find the Scriptures moſt cleare
for her & her doctrine, and ſhal wonder at their former blindnes.}
the ſelf-ſame veile in the lecture of the old Teſtament remaineth
vnreuealed (becauſe in Chriſt it is made void) \V but vntil this preſent
day, when Moyſes is read, a veile is put vpon their hart. \V But when he
ſhal be conuerted to our Lord, the veile ſhal be taken away. \V And
\CNote{\XRef{Io.~4,~24.}}
our Lord is a Spirit. And where the Spirit of our Lord is, there is
\LNote{Libertie.}{The
\MNote{The Chriſtian libertie.}
Spirit and grace of God in the new Teſtament diſchargeth vs of the
bondage of the Law and ſinne, but is not a warrant to vs of fleshly
licence, as S.~Peter
\CNote{\XRef{1.~Pet.~2. 16.}}
writeth or diſchargeth Chriſtians of their obedience to order, law, and
power of Magiſtrates ſpiritual or temporal, as ſome Heretikes of theſe
daies doe ſeditiouſly teach.}
libertie. \V But we al, beholding the glorie of our Lord with face
reuealed, are transformed into the ſame image from glorie vnto glorie, as
of our Lordes Spirit.


\stopChapter


\stopcomponent


%%% Local Variables:
%%% mode: TeX
%%% eval: (long-s-mode)
%%% eval: (set-input-method "TeX")
%%% fill-column: 72
%%% eval: (auto-fill-mode)
%%% coding: utf-8-unix
%%% End:

