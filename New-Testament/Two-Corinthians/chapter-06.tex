%%%%%%%%%%%%%%%%%%%%%%%%%%%%%%%%%%%%%%%%%%%%%%%%%%%%%%%%%%%%%%%%%
%%%%
%%%% The (original) Douay Rheims Bible 
%%%%
%%%% New Testament
%%%% Epistles
%%%% Two Corinthians
%%%% Chapter 06
%%%%
%%%%%%%%%%%%%%%%%%%%%%%%%%%%%%%%%%%%%%%%%%%%%%%%%%%%%%%%%%%%%%%%%




\startcomponent chapter-06


\project douay-rheims


%%% 2725
%%% o-2581
\startChapter[
  title={Chapter 6}
  ]

\Summary{That he helpeth with his exhortations, and in al things
  behaueth himſelf as becommeth a Miniſter of God. 11.~Which he ſpeaketh
  ſo openly, becauſe his hart is open vnto them: exhorting them to be
  likewiſe open-harted towards him, 14.~and to auoid thoſe infidels.}

And we
\LNote{Helping.}{For
\MNote{God's Miniſters are Coadiutours.}
that he declared before the Miniſters of the new Teſtament to be
Chriſtes Deputies, and that when they preach or doe any function, God as
it were ſpeaketh or doeth it by them, he boldly now ſaith, \Emph{Helping
therfore}: that is to ſay, ioyning or working together with God, we doe
exhort.}
helping doe exhort, that you receiue not the
\LNote{Grace in vaine.}{The
\MNote{God's grace forceth no man againſt his wil.}
\TNote{\G{συνεργοῦντες}}
grace of God worketh not in man againſt his wil, nor forceth any thing
without his acceptation and conſent: and therfore it lieth in man's wil to
fruſtrate or to follow the motion of God, as this text plainely
proueth.}
grace of God in vaine. (\V For he ſaith:
\CNote{\XRef{Iſ.~49,~8.}}
\Emph{in time accepted haue I heard thee; and in the day of ſaluation
haue I holpen thee.} Behold, now is the time acceptable: behold now the
day of ſaluation.) \V To no man giuing offence, that our miniſterie be
not blamed: \V but in al things let vs exhibit our ſelues as the
Miniſters of God, in much patience, in tribulations, in neceſsities, in
diſtreſſes, \V in ſtripes, in priſons, in ſeditions, in labours,
\LNote{In watching.}{When
\MNote{Voluntarie penance.}
in the middes of many miſeries and perſecutions, the Apoſtles yet of
their owne accord added and required voluntarie vigils, faſtings, and
chaſtitie, we may wel perceiue theſe workes to be wonderful grateful to
God, and ſpecially needful in the Clergie.}
in watchings, in faſtings, \V in chaſtitie, in knowledge, in
longanimitie, in ſweetnes, in the Holy Ghoſt, in charitie not feined, \V
in the word of truth, in the vertue of God; by the armour of iuſtice on
the right hand and on the left, \V by honour and diſhonour, by infamie
and good fame: as ſeducers, and true: as they that are vnknowen, and
knowen: \V as dying, and behold we liue: as chaſtened, & not killed: \V
as ſorrowful, but alwaies reioycing: as needie, but enriching many: as
\SNote{S.~Auguſtin
\Cite{(in Pſ.~113.)}
gathereth hereby, that the Apoſtles did vow pouertie.}
hauing nothing, and poſſeſsing al things.

\V Our mouth is open to you, ô Corinthians, our hart is dilated. \V You
are not ſtraitned in vs: but in your owne bowels you are ſtraitned. \V
But hauing the ſame reward (I ſpeake as to my owne children) be you alſo
dilated. \V
\SNote{It is not lawful for Catholikes to marrie with Heretikes or
Infidels. See 
\Cite{S.~Hierom. cont Iouinian. li.~1.}
\Cite{Conc. Laod. c.~10.}
and
\Cite{31.}}
Beare not the yoke with infidels. For what participation hath iuſtice
with iniquitie? or
\LNote{What ſocietie.}{Generally
\MNote{Not to communicate with Heretikes in any actes of religion.}
here is forbidden conuerſation and dealing with al Infidels, and
conſequently with Heretikes; but ſpecially in praiers, or meeting at
their Schiſmatical Seruice, preaching, or other diuine office
whatſoeuer. Which the Apoſtle here vttereth in more particular and
different termes, that Chriſtian folke may take the better heed of
it. No ſocietie (ſaith he) nor fellowship, no participation nor
agreement, no conſent between light and darknes, Chriſt and Baal, the
Temple of God and the Temple of Idols: al ſeruice, as pretended worship
of God ſet vp by Heretikes or Schiſmatikes, being nothing els but
Seruice of Baal and plaine Idolatrie, and their conuenticles nothing but
conſpirations againſt Chriſt. From ſuch therfore ſpecially we muſt ſeuer
our ſelues alwaies in hart and mind, and, touching any act of religion,
in body alſo, according as the children of Iſrael were commanded by God
to ſeparate themſelues from the Schiſmatikes Core, Dathan, & Abiron, and
their tabernacles, by theſe words:
\CNote{\XRef{Num.~16,~26.}}
\Emph{Depart from the tabernacles of the impious men, and touch ye not
thoſe things which pertaine to them, leſt you be enwrapped in their
ſinnes.}}
what ſocietie is there between light and darkenes? \V And what agreement
with Chriſt and Belial? or what part hath the faithful with the infidel?
\V And what agreement hath the Temple of God with the Idols? For
%%% o-2582
you are the Temple of the liuing God: as God ſaith: 
\CNote{\XRef{Leu.~26,~11.}}
\Emph{That I wil dwel, and walke in them, and wil be their God: and they
shal be my people.} \V For the which cauſe,
\CNote{\XRef{Eſ.~52,~11.}}
\Emph{Goe out of the middes of them, and ſeparate your ſelues}, ſaith our
Lord,
\CNote{\XRef{Hier.~31,~1.}}
\Emph{And touch not the vncleane: and I wil receiue you. \V And I wil be
a Father to you: and you shal be my ſonnes & daughters, ſaith our Lord
omnipotent.}


\stopChapter


\stopcomponent


%%% Local Variables:
%%% mode: TeX
%%% eval: (long-s-mode)
%%% eval: (set-input-method "TeX")
%%% fill-column: 72
%%% eval: (auto-fill-mode)
%%% coding: utf-8-unix
%%% End:

