%%%%%%%%%%%%%%%%%%%%%%%%%%%%%%%%%%%%%%%%%%%%%%%%%%%%%%%%%%%%%%%%%
%%%%
%%%% The (original) Douay Rheims Bible 
%%%%
%%%% New Testament
%%%% Epistles
%%%% Two Corinthians
%%%% Chapter 02
%%%%
%%%%%%%%%%%%%%%%%%%%%%%%%%%%%%%%%%%%%%%%%%%%%%%%%%%%%%%%%%%%%%%%%




\startcomponent chapter-02


\project douay-rheims


%%% 2716
%%% o-2572
\startChapter[
  title={Chapter 2}
  ]

\Summary{Proſecuting the true cauſe which in the laſt chapter he gaue of
  his not comming, 6.~he pardoneth now after ſome part of penance, him
  that for inceſt he excommunicated in the laſt epiſtle, requiring them
  obediently to conſent therunto. 12.~Then of his going from Troas in
  to Macedonia; God euery where giuing him the triumph.}

And I haue determined with my-ſelf this ſame thing, not to come to you
againe in ſorrow. \V For if I make you ſorie; & who is it that cã make
me glad, but he that is made ſorie by me? \V And this ſame I wrote to
you; that I may not, when I come, haue ſorrow vpon ſorrow, of the which I
ought to reioyce: truſting in you al, that my ioy is the ioy of you
al. \V For of much tribulation and anguiſh of hart I wrote to you by
many teares: not that you ſhould be made ſorie; but that you may know
what charitie I haue more aboundantly toward you. \V And if any man hath
made ſorrowful, not me hath he made ſorrowful, but in part, that I
burden not al you. \V To him that is ſuch a one
\LNote{This rebuke ſufficeth.}{This
\MNote{The Apoſtle excommunicateth, enioyneth penance, & afterward
pardoneth and abſolueth.}
Corinthian for inceſt was excommunicated & put to penance by the
Apoſtle, as appeareth in the
\XRef{former Epiſtle c.~5.}
And here order is giuen for his abſolution & pardoning. Wherin firſt we
haue a plaine exãple & proofe of the Apoſtolike power, there of binding,
& here of looſing: there, of puniſhing, here of pardoning: there of
retaining ſinnes, here of remiſſion. Secondly we may hereby  proue that
not only amendment, ceaſing to ſinne, or repentãce in hart & before God
alone, is alwaies enough to obteine ful reconcilement, wheras we ſee
here his ſeparation alſo from the faithful, and the Sacraments, and from
al companie or dealing with other Chriſtian men, beſides other bodily
affliction: al which,
\CNote{\XRef{1.~Cor.~3,~5.}}
called of the Apoſtle before \L{interitum carnis} \Emph{the deſtruction
of the flesh}, and named here \Emph{Rebuke}, (or as the
\TNote{\G{ἐπιτιμία}}
Greek word alſo importeth) mulct, penaltie, correction, chaſtiſement,
were enioyned him by the Apoſtles commandement in the face of the
Church, and by the offender patiently ſuſtained ſo long.
\MNote{Pardon or remiſſion of penãce enioyned.}
Thirdly, we ſee that it lieth in the hands of the Apoſtles, Biſhops, &
ſpiritual Magiſtrates, to meaſure the time of ſuch penance or
diſcipline, not only according to the weight of the offence cõmitted,
but alſo according to the weaknes of the perſons punished, and other
reſpects, of time and place as to their wiſedom shal be thought moſt
agreable to the parties good, and the Churches edification.
\MNote{Penance & ſatisfaction euidently proued agaĩſt the Proteſtants.}
Laſtly by this whole handling of the offenders caſe, we may refute the
wicked hereſie of the Proteſtants, that would make the ſimple beleeue,
no punishment of a man's owne perſon for ſinnes committed, nor penance
enioyned by the Church, nor any paines temporal or ſatisfaction for our
life paſt, to be neceſſarie, but al ſuch things to be ſuperfluous,
becauſe Chriſt hath ſatisfied enough for al. Which Epicurian doctrine is
refelled, not only hereby, but alſo by the 
\CNote{\XRef{Ioel.~1,~12.}}
Prophets,
\CNote{\XRef{Mt.~3.}}
Iohn the Baptiſtes,
\CNote{\XRef{&~4.}}
Chriſtes, & the
\CNote{\XRef{Act.~2.}
&
\XRef{26.}}
the Apoſtles preaching of penance and condigne workes or fruits of
repentance, to euery man in his owne perſon, & not in Chriſtes perſon
only: and by the whole life and moſt plaine ſpeaches and penitential
canons of the holy Doctours and Councels preſcribing times of penance,
commanding penance, enioyned penance, and continually vſing the word
ſatisfaction in this caſe throughout al their workes, as our Aduerſaries
themſelues can not but confeſſe.}
this rebuke ſufficeth that is giuen of many: \V ſo that contrariewiſe
you ſhould rather pardon and comfort him, leſt perhaps ſuch an one be
ſwallowed vp with ouer great ſorrow. \V For the which cauſe
\LNote{I beſeech you.}{They
\MNote{Zeale againſt the excommunicate.}
which at the beginning did beare too much with the offender and ſeemed
loth to haue him excommunicated in ſo auſtere manner, yet through their
obedience to the Apoſtle became on the other ſide ſo rigorous, and ſo
farre deteſted the malefactour after he was excommunicated, that the
Apoſtle now meaning to abſolue him, was glad to intreat, and command
them alſo, to accept him to their companie and grace againe.}
I beſeech you that you
%%% o-2573
confirme charitie toward him. \V For therfore alſo haue I written that I
may know the experiment of you, whether in al things you be
\LNote{Obedient.}{Though
\MNote{The Apoſtle chalengeth their obediẽce to his Eccleſiaſtical
authoritie.}
in the laſt chapter he diſcharged himſelf of tyrannical dominion ouer
them, yet he chalengeth their obediẽce in al things as their Paſtour and
Superiour, and conſequently in this point of receiuing to mercie the
penitent Corinthian. Wherby we ſee, that as the power and authoritie of
excommunicating, ſo of abſoluing alſo was in S.~Paules perſon, though
both were to be done in the face of the Church: els he would not haue
commanded or required their obedience.}
obedient. \V
And whom you haue pardoned any thing,
\LNote{I alſo.}{The Heretikes and others not wel founded in the
Scriptures and antiquitie, maruel
\Fix{at}{that}{obvious typo, fixed in other}
the Popes pardons, counting them either fruitles or vnlawful or no older
then S.~Gregorie.
\MNote{The authoritie of indulgences whervpon it is grounded.}
But indeed the authoritie, power, and right of them is
of Chriſtes owne word and commiſſion, principally giuen to Peter, and ſo
afterward to al the Apoſtles, and in their perſons to al the cheefe
Paſtours of the Church, when it was ſaid:
\CNote{\XRef{Mat.~18,~18.}}
\Emph{Whatſoeuer you looſe in earth, shal be looſed in heauen}. By which
commiſſion the holy Biſhops of old did cut-off large peeces of penance
enioyned to offenders, and gaue peace, grace, or indulgence,
\CNote{\Cite{Cypria. ep.~13.~14.~15.}}
before they had accompliſhed the meaſure of their appointed or deſerued
puniſhment. And that is to giue pardon. And ſo S.~Paul here did towards
the Corinthian, whom he aſſoileth of mere grace and mercie, as the word
\TNote{\G{κεχάρισμαι}}
\L{donare} or \L{condonare} doth ſignifie, when he might longer haue
kept him in 
penance and temporal affliction for his offenſe. Wherof though he had
already before God inwardly repented, yet was he iuſtly holden vnder
this correction for ſome ſatisfaction of his fault paſt, during the
Apoſtles pleaſure.
\MNote{What is a pardon or indulgence.}
To remit then the temporal puniſhment or chaſtiſement
due to ſinners after the offenſe it-ſelf & the guilt therof be forgiuen
of God, is an indulgence or pardon. Which the principal Magiſtrates of
God's Church by Chriſtes warrant and the Apoſtles example, haue euer
done, being no leſſe authorized to pardon then to puniſh; and by
imitation of our Maiſter (who forgaue
\CNote{\XRef{Io.~8,~11.}}
the aſuoutereſſe and diuers other offenders, not only their ſinnes, but
alſo often the temporal puniſhments due for the ſame) are as much giuen
to mercie as to iuſtice.}
I alſo. For, my-ſelf alſo that which
\SNote{Though he did great penance (ſaith Theodorete) yet he calleth
this pardoning, \G{χάριν}, \Emph{a grace}, becauſe his ſinne was greater
then his penance.}
\TNote{\G{κεχάρισμαι}}
I pardoned, if I pardoned any thing,
\LNote{For you.}{Theodorete
\CNote{\Cite{Theodor. in hunc locum.}}
\MNote{Indulgences or pardons in the primitiue Church.}
vpon this place ſaith that the Apoſtle gaue this pardon to the
Corinthian at the interceſſion of the bleſſed men Timotheus and
Titus. And we may read in ſundrie places, of 
\CNote{\Cite{Cypr. locis citatis.}}
S.~Cyprian namely, that indulgences or remiſsions were giuen in the
primitiue Church by the mediation of holy Confeſſours or Martyrs, and by
communicating the ſatisfactorie workes of one to another: to which end
they gaue their letters to Bishops in the behalfe of diuers their
Chriſtian Brethren: a thing moſt agreable to the mutual entercourſe that
is between the members of Chriſtes myſtical body, and very anſwerable to
God's iuſtice,
\CNote{\XRef{2.~Cor.~8.}}
which by ſupply of the one ſort that aboundeth, ſtandeth entire in
reſpect of the other ſort alſo that wanteth. In which kind the 
\CNote{\XRef{Col.~1,~24.}}
Apoſtle confeſſeth that himſelf by his ſuffering and tribulations
ſupplieth the wants of ſuch paſsions as Chriſt had to ſuffer, not in his
owne perſon, but in his body, which is his Church. Wherupon we inferre
moſt aſſuredly, that the ſatisfactorie and penal workes of holy Saints
ſuffered in this life, be communicable and applicable to the vſe of
other faithful men their fellow-members in our Lord, and to be diſpenſed
according to euery ones neceſſitie and deſeruing, by them whom Chriſt
hath conſtituted ouer his familie, and hath made the diſpenſers of his
treaſures.}
for you
\LNote{In the perſon of Chriſt.}{For
\MNote{Al pardon and remiſsion is in the vertue and name of Chriſt.}
that many might of ignorance or pride reproue the practiſe of Gods
Church and her Officers, or deny the Apoſtles authoritie to be ſo great
ouer mens ſoules as to puniſh and pardon in this ſort, S.~Paul doth
purpoſely and preciſely tel them that he doth giue pardon as Chriſtes
Vicar, or as bearing his perſon in this caſe: and therfore that no man
may maruel of his power herein, except he thinke that Chriſtes power,
authoritie, and commiſsion is not ſufficient to releaſe temporal
punishment due to ſinners.
\MNote{Heretical trãſlation.}
And this to be the proper meaning of theſe words, 
\TNote{\G{ἐν προσώπῳ Χριστοῦ}}
\Emph{In the perſon of Chriſt}, and not as the Proteſtants would haue it
(the better to auoid the former concluſion of the Apoſtles giuing
indulgence) \Emph{In the face or ſight of Chriſt}, you may eaſily
vnderſtand by the Apoſtles like inſinuation of Chriſtes power, when he
committeth this offender to Satan, affirming that he gaue that ſentẽce
in the name and with the \Emph{vertue or power of our Lord} \Sc{Iesvs
Christ}. In al which caſes the Proteſtants blindnes is exceeding great,
who can not ſee that this is not the way to extol Chriſtes power, to
deny it to his Prieſts, ſeing the
\CNote{\XRef{1.~Cor.~5,~4.}}
Apoſtle chalengeth it by that that Chriſt hath ſuch power, & that
himſelf doth it in
\Fix{is}{his}{obvious typo, fixed in other}
name, vertue, and perſon. So now in this and in no other name giue Popes
and Bishops their pardons. Which pertaining properly to releaſing only
of temporal punishment due after the ſinne and the eternal punishment be
forgiuen, is not ſo great a matter as the remiſsion of the ſinne it
ſelf: which yet the Prieſts
\CNote{\XRef{Io.~20,~23.}}
by expreſſe comiſsion doe alſo remit.}
in the perſon of Chriſt, \V that we be not
\LNote{Circumuented of Satan.}{We
\MNote{Al binding & looſing muſt be vſed to the parties ſaluation.}
may ſee hereby, that the diſpenſation of ſuch diſcipline and the
releaſing of the ſame, be put into the power and hands of Gods
Miniſters, to deale more or leſſe rigorouſly, to pardon ſooner or later,
puniſh longer or ſhorter while, as ſhal be thought beſt to their
wiſedom. For the end of al ſuch correction or pardoning, muſt be the
ſaluation of the parties ſoul, as the Apoſtle noted
\XRef{1.~Cor.~5,~5.}
Which to ſome, and ſome certaine times, may be better procured by rigour
of diſcipline then by indulgẽce, to ſome others, by lenitie & humane
dealing (ſo pardoning of penance is called in old 
\CNote{\Cite{Con. Ni. can.~12.}
\Cite{Ancyra. can.~2. &~5.}}
Councels) rather then by ouer-much chaſtiſement.
\MNote{The great penance of the primitiue Church.}
For conſideration wherof, in ſome Ages of the Church, much diſcipline,
great penance & ſatisfaction was both enioyned and alſo willingly
ſuſteined, and then was the leſſe pardoning and fewer indulgences;
becauſe in that voluntary vſe and acceptation of puniſhment, and great
zeale and feruour of ſpirit, euery man fulfilled his penance, and few
asked pardon.
\MNote{Why more
\Fix{and pardons}{pardons and}{obvious typo, fixed in other}
Indulgences now then in old time.}
Now in the fal of deuotion and lothſomnes that men commonly haue to doe
great penance, though the ſinnes be farre greater then euer before, yet
our holy mother the Church knowing with the Apoſtle the cogitations of
Satan, how he would in this delicate time, driue men either to
deſperation, or to forſake Chriſt & his Church & al hope of ſaluation,
rather then they would enter into the courſe of canonical diſcipline,
enioyneth ſmal penance, and ſeldom vſeth extremitie with offenders as
the holy Biſhops of the primitiue Church did, but condeſcending to the
weaknes of her children, pardoneth exceeding often and much, not only al
enioyned penance but alſo al or great parts of what puniſhment temporal
ſoeuer due or deſerued, either in this world or in the next. As for the
Heretikes which neither like the Churches lenitie and pardoning in theſe
daies, nor the old rigour of the primitiue Church, they be like to the
Iewes
\CNote{\XRef{Mat.~11,~18.}}
that condemned Iohn the Baptiſt of auſteritie, & Chriſt of too much
freedom and libertie: not knowing nor liking indeed either Chriſtes
ordinance and commiſſion in binding or looſing, or his prouidence in the
gouernement of the Church.}
circumuented of Satan. For we are not ignorant of his cogitations.

\V And when I was come to Troas for the Ghoſpel of Chriſt, and a doore
was opened vnto me in our Lord, \V I had no reſt in my ſpirit, for that
I found not Titus my Brother, but bidding them fare
%%% 2717
wel, I went forth into Macedonia. \V And thankes be to God, who alwaies
triumpheth vs in Chriſt \Sc{Iesvs}, and manifeſteth the odour of his
knowledge by vs in euery place. \V For we are the good odour of Chriſt
vnto God in thẽ that are ſaued, & in them that periſh. \V To ſome indeed
the odour of death vnto death: but to others the odour of life vnto
life. And to theſe things who is ſo ſufficient? \V For we are not as
very many
\LNote{Adulterating.}{The
\MNote{The Heretikes corrupting of the Scripture.}
\TNote{\G{καπηλεύοντες}}
Greek word ſignifieth to make commoditie of the word of God as vulgar
Vintners doe of their wine. Whereby is expreſſed the peculiar trade of
al Heretikes, and exceeding proper to the Proteſtants, that ſo corrupt
Scriptures by mixture of their owne phantaſies, by falſe tranſlatiõs,
gloſſes, colourable & pleaſant commentaries, to deceiue the taſt of the
ſimple, as tauerners and tapſters doe, to make their wines ſalable by
manifold artificial deceits. The Apoſtles contrariewiſe, as al
Catholikes, deliuer the Scriptures and vtter the word of God ſincerely
and entirely, in the ſame ſenſe and ſort as the Fathers left them to the
Church, interpreting them by the ſame Spirit by which they were written
or ſpoken.}
adulterating the word of God, but of ſinceritie, and as of God, before
God, in Chriſt we ſpeak.


\stopChapter


\stopcomponent


%%% Local Variables:
%%% mode: TeX
%%% eval: (long-s-mode)
%%% eval: (set-input-method "TeX")
%%% fill-column: 72
%%% eval: (auto-fill-mode)
%%% coding: utf-8-unix
%%% End:

