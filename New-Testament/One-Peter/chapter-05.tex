%%%%%%%%%%%%%%%%%%%%%%%%%%%%%%%%%%%%%%%%%%%%%%%%%%%%%%%%%%%%%%%%%
%%%%
%%%% The (original) Douay Rheims Bible 
%%%%
%%%% New Testament
%%%% Epistles
%%%% One Peter
%%%% Chapter 05
%%%%
%%%%%%%%%%%%%%%%%%%%%%%%%%%%%%%%%%%%%%%%%%%%%%%%%%%%%%%%%%%%%%%%%




\startcomponent chapter-05


\project douay-rheims


%%% 2902
%%% o-2763
\startChapter[
  title={Chapter 05}
  ]

\Summary{He exhorteth Prieſts to feed their flockes, only for Gods ſake
  and reward of heauen, without al lordlines: 5.~the laie to obey: al to
  be humble one towards another: 8.~to be conſtant in the Catho. faith,
  conſidering it is not man, but that lion the Diuel that perſecuteth
  them, 9.~as he doth the whole Church alſo, & that God wil after a
  while make them ſecure in heauen.}

%%% o-2764
The
\LNote{Seniours.}{Though
\MNote{\Emph{Senior} in the vulgar tranſlation is often Prieſt or
Bishop. See
\XRef{Act.~15.}}
the Latin, \L{Senior}, be not appropriated to holy Order by vſe of
ſpeach, neither in the Latin nor in our language: yet it is plaine that
the Greek word \GG{Presbyter}, which the Apoſtle here vſeth, is here
alſo (as commonly in other places of the new Teſtament) a word for
Eccleſiaſtical office, and not of age, and is as much to ſay as
\Emph{Prieſt} or \Emph{Bishop}. For the Apoſtle himſelf being of that
Order, ſpeaketh (as by his words it is plaine) to ſuch as had charge of
ſoules, ſaying, \Emph{Feed the flock of God which is among you}. Becauſe
we follow the vulgar tranſlation, we ſay \Emph{Seniours}, and
\Fix{Seniours:}{Seniour:}{obvious typo, fixed in other}
whereas otherwiſe we might and should ſay according to the Greek,
\Emph{The Prieſts therfore I beſeech, my ſelf a fellow Prieſt with
them.} So doth S.~Hierom read (\GG{Preſbyteros compreſbyter}) and
expound
\Cite{ep.~85.}
So tranſlateth Eraſmus, and Beza himſelf.}
Seniours therfore that are among you, I beſeech, myſelf a fellow-Senior
with them & a witneſſe of the Paſſions of Chriſt, who am alſo partaker
of that glorie which is to be reuealed in time to come: \V feed the
flock of God which is among you
\TNote{\G{ἐπισκοποῦντες}}
prouiding not by conſtraint, but willingly according to God:
\SNote{Deſire of lucre, or to exerciſe holy functions for gaine, is a
filthy fault in the Clergie, and therfore much to be auoided.}
neither for filthie lucre ſake, but voluntarily: \V neither as
\LNote{Ouer-ruling.}{Not
\MNote{Not Superioritie but tyrãnie and lordlines is forbiddẽ in the
Clergie.}
ſuperiority, preeminence, ſoueraignty, or rule on the one ſide, nor
\Fix{abedience,}{obedience,}{probable typo, fixed in other}
ſubiection, and inferiority on the other ſide, be forbidden in the
Clergie: but tyrannie, pride, and ambitious domination be forbidden; and
humility, meeknes, moderation, are commended in Eccleſiaſtical Officers:
The
\TNote{\G{κατακυριεύοντες}}
Greek word here of rule or ouer-ruling, being the ſame that our Sauiour
vſeth in the Ghoſpel of the tyrannical rule of ſecular Heathen Princes,
\CNote{\XRef{Mat.~20. v.~25.}}
ſaying to his Apoſtles, that it shal not be ſo among them: according as
here the Prince of the Apoſtles teacheth his Brethren the Eccleſiaſtical
Rulers.}
ouer-ruling
\LNote{The Clergie.}{Some
\MNote{Heret. tranſlation.}
of the English new tranſlations turne it corruptely, \Emph{Parishes}:
others, \Emph{heritages}: both to auoid the moſt knowen, true, and
common word in al Chriſtian languages, to wit,
\MNote{The name of Clergie and Clerke.}
\Emph{Clergie}, a word,
by vſe of al antiquity, & agreably to the holy Scriptures, made
proper to the Spiritualty or Clergie. Though in another more vulgar
acception it may agree to al Chriſt's choſen heritage, as wel of lay
people as Prieſts. Which the Proteſtants had rather follow; becauſe they
wil haue no difference between the laity and the Clergie. But the holy
Fathers farre otherwiſe euen from the beginning. Whereof ſee S.~Cyprian
\Cite{ep.~4.~5.~6. &c.}
And S.~Hierom
\Cite{ep.~2. to Nepitianus c.~5.}
where he interpreteth this word. \Emph{Therfore} (ſaith he) \L{Clericus}
that is \Emph{a Clergie man, which ſerueth the Church of Chriſt, let him
firſt interpret his name, & the ſignification of the name being declared,
let him endeauour to be that which he is called. If} \G{κλῆρος}
(\GG{Clerus}) \Emph{in Greek, be called in
Latin,} \L{Sors}, \Emph{therfore are they
called} \L{Clerici}, \Emph{that is, Clergie men, becauſe they are of the
lot of our Lord, or becauſe our Lord himſelf is the lot or portion of
Clergie men, &c.}

Which calling no doubt was taken out of the holy Scriptures,
\XRef{Numer.~18.}
&
\XRef{Deuter.~18.}
where God is called
\TNote{\G{κληρονομία}
\G{κληρονομήσεις}}
the inheritance, lot, and portion of the Prieſts and Leuites: and now
when mẽ be made of the Clergie, they ſay, \L{Dominus
pars hæreditatis mea}, that is, \Emph{Our Lord is the portion of mine
inheritance}: but ſpecially out of the new Teſtament,
\XRef{Act.~1,~21.~25.}
and
\XRef{8,~21.}
Where the lot or office of the Eccleſiaſtical miniſterie is called by
this word \G{κλῆρος}, \L{Clerus}.
\MNote{Prieſts crownes.}
See in Venerable Bede the cauſes why this holy ſtate being ſeuered by
name from the Laity, doth weare alſo a crowne on their head for
diſtinction
\Cite{Lib.~5. hiſt. Angl. cap.~22.}}
the Clergie, but made examples of the flocke from the hart. \V And when
the Prince of paſtours ſhal appeare, you ſhal receiue the incorruptible
\LNote{Crowne of glorie.}{As
\MNote{The heauenly crowne of Doctours, and Preachers.}
life euerlaſting shal be the reward of al the iuſt, ſo the Preachers and
Paſtours that doe wel, for their doing shal haue that reward in a more
excellent degree, expreſſed here by theſe words, \Emph{Crowne of
glorie}, according to the ſaying of
\XRef{Daniel c.~12.}
\Emph{They that ſleep in the duſt of the earth, shal awake, one ſort to
life euerlaſting, others to euerlaſting rebuke. But ſuch as be learned
shal shine as the brightnes of the firmament: and ſuch as inſtruct many
to iuſtice, shal be as ſtarres, during al eternitie.}}
crowne of glorie.

\V In like manner ye yong men be ſubiect to the Seniours. And doe ye al
inſinuate humilitie one to another, becauſe
\CNote{\XRef{Prou.~3.}}
\Emph{God reſiſteth the proud, and to the humble he giueth grace.} \V
\CNote{\XRef{Iac.~4.}}
Be ye humbled therfore vnder the mightie hand of God, that he may exalt
you in the time of viſitation: \V
\CNote{\XRef{Iac.~4.}
\XRef{Pſa.~54.}
\XRef{Mat.~6,~25.}}
caſting al your carefulnes vpon him, becauſe he hath care of you. \V Be
ſober and watch: becauſe your aduerſarie the Diuel as a roaring lion
goeth about, ſeeking whom he may deuoure. \V Whom reſiſt ye, ſtrong in
faith: knowing that the ſelf-ſame affliction is made to that your
Fraternitie which is in the world. \V But the God of al grace, which
hath called vs vnto his eternal glorie in Chriſt \Sc{Iesvs}, he wil
perfit you hauing ſuffered a litle, and confirme, and ſtabliſh you. \V
To him be glorie and empire for euer and euer. Amen.

\V By Syluanus, a faithful Brother to you, as I thinke, I haue briefely
written: beſeeching & teſtifying that this is the true grace of God,
wherein you ſtand. \V The Church ſaluteth you,
\LNote{That is in Babylon.}{The
\MNote{S.~Peter writeth from Babylon, that is Rome.}
Proteſtants shew themſelues here (as in al places where any controuerſie
is, or that maketh againſt them) to be moſt vnhoneſt and partial
handlers of God's word. The ancient Fathers, namely S.~Herom
\Cite{in Catalogo de Scriptoribus Eccleſtiaſticie, verbo Marcus:}
Euſebius
\Cite{li.~2, c.~14. hiſt.}
Oecumenius
\Cite{vpon this place,}
and many moe agree, that Rome is meant by the word Babylon, here alſo,
as in the
\XRef{16.}
and
\XRef{17.~of the Apocalypſe:}
ſaying plainely, that S.~Peter wrote this Epiſtle at Rome,
\MNote{Why Rome was called Babylon.}
which is called Babylon for the reſemblance it had to Babylon that great
citie in Chaldea (where the Iewes were captiues) for magnificence,
Monarchie, reſort and confuſion of al peoples and tongues, and for that
it was before Chriſt and long after, the ſeat of al Ethnike ſuperſtition
& Idolatrie, & the ſlaughter-houſe of the Apoſtles & other Chriſtian
men, the Heathen Emperours then keeping their cheefe reſidence there.
See S.~Leo
\Cite{Ser.~1. in nat. Petri & Pauli.}

This being moſt plaine, and conſonant to that which followeth of
S.~Marke, whom al the Eccleſiaſtical hiſtories agree to haue been Peters
ſcholer at Rome, and that he there wrote
\Fix{is}{his}{obvious typo, fixed in other}
Ghoſpel:
\MNote{The Proteſtãts wil haue Babylon to ſignifie Rome in other places
but not here.}
yet our Aduerſaries fearing hereby the ſequele of Peters or the Popes
ſupremacie at Rome, deny that euer he was there, or that this
Epiſtle was written there, or that Babylon doth here ſignifie Rome: but
they ſay that Peter wrote his Epiſtle at Babylon in Chaldea, though
they neuer read either in Scriptures or other holy or profane hiſtorie,
that this Apoſtle was euer in that towne. But ſee their shameles
partiality. Here Babylon (they ſay) is not takẽ for Rome, becauſe it
would follow that Peter was at Rome &c. but in the Apocalypſe where al
euil is ſpokẽ of Babylõ, there they wil haue it ſignifie nothing els but
Rome, & the Romane Church alſo, not (as the Fathers interpret it) the
temporal ſtate of the Heathen Empire there. So doe they follow, in euery
word no other thing but the aduantage of their owne hereſie. See the
\XRef{Annotation vpon the laſt of the Romanes v.~16.}
and
\XRef{17.~of the Apocalypſe v.~5.}

And
\MNote{The Proteſtãts wrangle about the time of Peters being at Rome.}
as for their wrangling vpon the ſupputation of the time of his going
thither, and the number of yeares that he was there, & the diuerſitie
that ſeemeth to be in the Eccleſiaſtical Writers concerning the ſame,
read B.~Fisher & others that ſubſtantially anſwer al ſuch cauils. And if
ſuch contentious reaſoning might take place, we should hardly beleeue
the principal things recorded either in Eccleſiaſtical hiſtories, or in
the Scriptures themſelues.
\MNote{Many things moſt true (euen in the Scriptures) are not agreed
vpon concerning the time.}
Concerning the time of Chriſts flying into Ægypt, of the
comming of the Sages to adore him, yea of the yeares of his age, &
time of his death, al ancient Writers doe not agree. And concerning the
day of his laſt ſupper and inſtitution of the Holy Sacrament, there is
diuerſitie of opinions. Shal we therfore inferre that he neuer died,
and that the other things neuer were? Can the Heretikes accord al the
hiſtories that ſeeme euen in holy Scripture to haue contradiction? Can
they tel vs certainly, when Dauid firſt came to Saul and the like? Doubt
they whether the world was euer created, becauſe the count of the yeares
is diuers? Doe they not beleeue that Paradiſe euer was, becauſe no man
knoweth where it is: and ſuch other things infinit to rehearſe? Which
when they were done, were plaine and knowen things in the world: and now
for vs to cal them to an account after ſo many yeares, Ages, and worlds,
is but ſophiſtication and plaine infidelitie. And this Sect of the
Proteſtants ſtanding only vpon deſtruction, and negatiues, & dealing
with our religion euen as Iulian, Porphyrie, and Lucian did, it is an
eaſie thing for them to beſtow their time in picking of quarels.}
that is in Babylon, coelect: and Marke my ſonne. \V
\CNote{\XRef{Ro.~16,~16.}
\XRef{1.~Cor.~16,~20.}
\XRef{2.~Cor.~13,~12.}}
Salute one another in a holy kiſſe. Grace be to al you which are in
Chriſt \Sc{Iesvs}. Amen.


\stopChapter


\stopcomponent


%%% Local Variables:
%%% mode: TeX
%%% eval: (long-s-mode)
%%% eval: (set-input-method "TeX")
%%% fill-column: 72
%%% eval: (auto-fill-mode)
%%% coding: utf-8-unix
%%% End:

