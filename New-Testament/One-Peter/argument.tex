%%%%%%%%%%%%%%%%%%%%%%%%%%%%%%%%%%%%%%%%%%%%%%%%%%%%%%%%%%%%%%%%%
%%%%
%%%% The (original) Douay Rheims Bible 
%%%%
%%%% New Testament
%%%% One Peter
%%%% Argument
%%%%
%%%%%%%%%%%%%%%%%%%%%%%%%%%%%%%%%%%%%%%%%%%%%%%%%%%%%%%%%%%%%%%%%

%%% Latin checked by KK.



\startcomponent argument


\project douay-rheims


%%% 2893
%%% o-2754
\startArgument[
  title={\Sc{The Argvment of Both the Epistles of S.~Peter, the First,
  and the Second.}},
  marking={Argument of One and Two Peter}
  ]

Of S.~Peter we read at large both in the Ghoſpels, and in the Actes of
the Apoſtles: and namely, that Chriſt deſigned him, and alſo made him
his Vicar (as S.~Matthew for that cauſe in the Catalogue of the Apoſtles
\CNote{c.~10. v.~2.}
calleth him \L{Primus}, \Emph{the firſt}, and al antiquitie, \L{Princeps
Apoſtolorum}, \Emph{the Prince of the Apoſtles}) and that he accordingly
executed that office after Chriſtes departure, planting the Church firſt
among the Iewes in Hieruſalem and in al that country and coaſts about,
as Chriſt alſo himſelf before had preached to the Iewes alone.

But preaching at length to the Gentils alſo, according to Chriſtes
commiſsion
\XRef{(Mat.~28. v.~19.)}
and being now come to Rome, the head citie of the Gentils, from thence he
writeth this Epiſtle to his Chriſtian Iewes, hauing care of them in his
abſence, no leſſe then when he was preſent: and not to the Iewes that
were at home (belike becauſe they had S.~Iames, or his Succeſſour
S.~Simon Cleophæ, reſident with them) but
\CNote{\XRef{1.~Pet.~1.}}
to them that were diſperſed in Pontus, Galatia, Cappadocia, and
Bythnia.

And that he writeth it from Rome, himſelf ſignifieth ſaying:
\CNote{\XRef{1.~Pet.~5.}}
\Emph{The Church that is in Babylon ſaluteth you.}
\SNote{See the
\XRef{Annotation 1.~Pet.~5. v.~13.}}
Where by Babylon he meaneth Rome, as al antiquitie doth interpret him:
not that he ſo calleth the Church of Rome, but the Heathen ſtate of the
Romane Empire, which then, and 300.~yeares after, vnto the conuerſion of
Conſtantinus the Emperour, did perſecute the elect Church of Rome, in
ſo-much that the firſt 33.~Bishops thereof vnto S.~Silueſter, were al
Martyrs.

For the matter whereof he writeth, himſelf doth ſignifie it in theſe
words:
\CNote{\XRef{2.~Pet.~3.}}
\Emph{This loe the ſecond Epiſtle I write to you, my Deareſt, in which
(Epiſtles) I ſtirre vp by admonition, your ſincere mind that you may be
mindful of thoſe words &c.} So he ſaith there of both together. And
againe of the firſt to the ſame purpoſe, in another place:
\CNote{\XRef{1.~Pet.~5.}}
\Emph{I haue breefly written, beſeeching and teſtifying that this is the
true grace of God, wherein you ſtand.} For there were at that time
certaine Seducers (as
\SNote{See the
\XRef{Annotation vpon S.~Iames epiſtle c.~2. v.~21.}}
S.~Auguſt. alſo hath told vs) who went about to teach \Emph{Only faith},
as though good workes were not neceſſarie, nor meritorious. There were
alſo great perſecutions, to compel them with terrour to denie Chriſt &
al his religion. He therfore exhorteth them accordingly, neither for
perſecution, neither by ſeduction to forſake it: though in the firſt,
his exhortation is more principally againſt perſecution: and in the
ſecond more principally againſt ſeduction. The firſt Epiſtle is noted to
be very like to S.~Paules epiſtle to the Epheſians, in words alſo, and
ſo thicke of Scriptures, as though he ſpake nothing els.

The time when the firſt was written, is vncertaine: the ſecond was
written a litle before his death, as is gathered by his words in the
ſame.
\XRef{c.~1. v.~14.}


\stopArgument


\stopcomponent


%%% Local Variables:
%%% mode: TeX
%%% eval: (long-s-mode)
%%% eval: (set-input-method "TeX")
%%% fill-column: 72
%%% eval: (auto-fill-mode)
%%% coding: utf-8-unix
%%% End:
