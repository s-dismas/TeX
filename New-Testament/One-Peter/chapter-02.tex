%%%%%%%%%%%%%%%%%%%%%%%%%%%%%%%%%%%%%%%%%%%%%%%%%%%%%%%%%%%%%%%%%
%%%%
%%%% The (original) Douay Rheims Bible 
%%%%
%%%% New Testament
%%%% One Peter
%%%% Chapter 02
%%%%
%%%%%%%%%%%%%%%%%%%%%%%%%%%%%%%%%%%%%%%%%%%%%%%%%%%%%%%%%%%%%%%%%




\startcomponent chapter-02


\project douay-rheims


%%% 2895
%%% o-2757
\startChapter[
  title={Chapter 02}
  ]

\Summary{Now after their Baptiſme, what muſt be their meat: 4.~and being
  come to Chriſt, how happie they be aboue their incredulous Brethren,
  according to the Scriptures alſo. 11.~Whereupon he beſeecheth them to
  shine in good life among the Heathen, ſo to procure their conuerſion:
  13.~to be obedient ſubiects to higher Powers (howſoeuer ſome
  miſconſter Chriſtian libertie) 14.~and ſeruants to obey their
  Maiſters. 19.~And ſo, doing wel, though they ſuffer for it, it is very
  meritorious. 21.~Whereas Chriſt alſo not only gaue them example,
  24.~but alſo by his death hath made them able to liue iuſtly.}

Laying away therfore al malice, and al guile, and ſimulations, and
enuies, and al detractions, \V as infants euen now borne, reaſonable,
milke without guile deſire ye, that in it you may grow vnto
ſaluation. \V If yet you haue taſted that our Lord is ſweet. \V Vnto
whom  approching, a liuing ſtone, of men indeed reprobated, but of God
elect and made honorable: \V be ye alſo your ſelues ſuperedified
%%% 2896
as it were liuing ſtones,
\Var{ſpiritual houſes}{a ſpiritual houſe}
a holy prieſthood to offer
\LNote{Spiritual hoſts.}{Here
\MNote{Spiritual hoſts and Prieſts.}
\Fix{were}{we}{obvious typo, fixed in other}
ſee, that as he ſpeaketh of ſpiritual hoſts, which euery Chriſtian man
offereth, ſo he ſpeaketh not properly of prieſthood, when he maketh al
Prieſts, but of a ſpiritual prieſthood. Which ſpiritual prieſthood was
alſo in al the Iewes: but the prieſthood (properly ſo called) was only
in the ſonnes of Aaron, and they offered the Sacrifices (properly ſo
called) which none beſides might offer.}
ſpiritual hoſts, acceptable to God by \Sc{Iesvs} Chriſt. \V For the
which cauſe the Scripture conteineth,
\CNote{\XRef{Eſ.~28,~26.}}
\Emph{Behold I put in Sion a principal corner-ſtone elect, pretious. And
he that shal beleeue in him, shal not be confounded.} \V To you therfore
that beleeue, honour: but to them that beleeue not,
\CNote{\XRef{Pſ.~117.}}
\Emph{the ſtone which the builders reiected, the ſame is made into the
head of the corner}: \V and
\CNote{\XRef{Mat.~21.}}
a ſtone of offenſe, & a rocke of ſcandal, to them that ſtumble at the
word, neither doe beleeue
\Var{wherin alſo they are put.}{whereto alſo they are ordeined.}
\V But you are an
\CNote{\XRef{Act.~4.}
\XRef{Eſ.~8.}
\XRef{Ro.~9,~33.}
\XRef{Exo.~19.}}
elect Generation, a
\SNote{The Proteſtants can no more gather of this, that al Chriſtians be
Prieſts, then that al be Kings as is moſt plaine.
\XRef{Apoc.~1,~6.}
&
\XRef{5,~10.}
\Emph{Thou haſt made vs a Kingdõ (or Kings) & Prieſts.}}
kingly Prieſthood, a holy Nation, a people of purchace: that you may
declare his vertues, which from darkeneſſe hath called you into his
maruelous light. \V
\CNote{\XRef{Apoc.~1,}}
\Emph{Which ſometime not a people: but now the people of God. Which not
hauing obtained mercie: but now hauing obtained mercie.}

\V My Deareſt I beſeech you as ſtrangers & pilgrimes,
\CNote{\XRef{Oſe.~2.}
\XRef{Ro,~9.}
\XRef{Gal.~5,~16.}}
to refraine your ſelues from carnal deſires which warre againſt the
ſoule, \V hauing your conuerſation good among the Gentils: that in that
wherein they miſreport of you as of malefactours, by the good workes
conſidering you,
\CNote{Mt.~5,~16.}
they may glorifie God in the day of viſitation. \V
\CNote{Ro.~13,~1.}
\LNote{Be ſubiect.}{Not
\MNote{Obedience to temporal Princes.}
only our Maiſter Chriſt, but the Apoſtles and al Chriſtians were euer
charged by ſuch as thought to bring them in hatred with Princes, with
diſobedience to Kings and temporal Magiſtrates. Therfore both
\CNote{\XRef{Ro.~15.}}
S.~Paul and this Apoſtle doe ſpecially warne the faithful, that they
giue no occaſion by their il demeanure to ſecular Princes, that the
Heathen should count them diſobedient or ſeditious workers againſt the
States of the world.}
Be ſubiect therfore
%%% !!! LNote not marked in either
\LNote{To euery humane creature.}{So
\MNote{God inſtituted the Spiritual gouernement in more excellent manner
then the temporal.}
he calleth the temporal Magiſtrate elected by the people, or holding
their Soueraignty by birth & carnal propagation, ordained for the
worldly wealth, peace, and proſperitie of the ſubiects: to put a
difference betwixt that humane Superiority, and the ſpiritual Rulers and
regiment, guiding & gouerning the people to an higher end, and
inſtituted by God himſelf immediately. For Chriſt did expreſly
conſtitute the forme of regiment vſed euer ſince in the Church. He made
one the cheefe, placing Peter in the Supremacie: he called the Apoſtles
and Diſciples, giuing them their ſeueral authorities. Afterward
\CNote{\XRef{Act.~1.}}
God guided the lot for choice of S.~Matthias in Iudas place: and the
Holy Ghoſt expreſly and namely ſeuered & choſe Paul and Barnabas vnto
their Apoſtolical function: and generally the Apoſtle ſaith of al
ſpiritual Rulers,
\CNote{\XRef{Act.~20.}}
\Emph{The holy Ghoſt hath placed you to rule the Church of God.}

And although al power be of God, and Kings rule by him, yet that is no
otherwiſe, but by his ordinarie concurrence, and prouidence, whereby he
procureth the earthly commodity or wealth of men, by maintaining of due
ſuperiority and ſubiection one towards another, and by giuing power to
the people and Common-wealth to chooſe to themſelues ſome kind or forme
of Regiment, vnder which they be content to liue for their preſeruation
in peace and tranquility. But ſpiritual ſuperiority is farre more
excellent, as in more excellent ſort depending, not of man's ordinance,
election, or (as this Apoſtle ſpeaketh) creation, but of the Holy Ghoſt,
who is alwaies reſident in the Church (which is Chriſt's body myſtical,
and therfore another manner of Common-wealth then the earthly)
concurring in ſingular ſort to the creation of al neceſſarie Officers in
the ſaid Church, euen to the worlds end, as
\CNote{\XRef{Eph.~4.}}
S.~Paul writeth to the Epheſians.

Leſt therfore the people, being then in ſo preciſe ſort alwaies warned
of the excellencie of their Spiritual Gouernours
\CNote{\XRef{Hebr.~13.}}
and of their obedience toward them, might neglect their dueties to
Temporal Magiſtrates, ſpecially being infidels, and many times tyrants
and perſecutours of the faith, as Nero and other were then: therfore
S.~Peter here warneth them to be ſubiect, for their bodies and goods and
other temporal things, euen to the worldly Princes both infidels and
Chriſtians, whom he calleth humane creatures.}
to euery
\SNote{So is the Greek, but the Proteſt. in fauour of temporal lawes
made againſt the Cat. religion, trãſlate it very falſely thus,
\Emph{to al mãner ordinãce of man}: themſelues boldly reiecting
Eccleſiaſtical decrees as mens ordinances.}
humane
\TNote{\G{κτίσει}}
creature for God: whether it be
\LNote{To the King as excelling.}{Some
\MNote{Heret. tranſlation.}
ſimple heretikes, & others alſo not vnlearned, at the beginning, for
lacke of better places, would haue proued by this, that the King was
Head of the Church, and aboue al Spiritual Rulers: and to make it ſound
better that way, they falſely tranſlated it, \Emph{To the King as to the
cheefe Head.} In the
\Cite{Bible of the yeare 1562.}
\MNote{The Kings excellencie of power is in reſpect of the nobilitie and
lay Magiſtrates vnder him.}
But it is euident that he calleth the King, the precellent or more
excellent, in reſpect of his
\Fix{Vicegerents}{Viceregents}{likely typo, same in both}
which he calleth Dukes or Gouernours that be at his
appointment; and not in reſpect of Popes, Bishops, or Prieſts, as they
haue the rule of mens ſoules: who could not in that charge be vnder ſuch
Kings or Emperours as the Apoſtle ſpeaketh of; no more then the Kings or
Emperours then, could be Heads of the Church, being Heathen men and no
members thereof, much leſſe the cheefe members. See a notable place in
\Cite{S.~Ignatius ep.
\Fix{and}{ad}{obvious typo, fixed in other}
Smyrnenſes,}
where he exhorteth them firſt to honour God, next the Bishop, & then the
King.

This
\MNote{Chriſtiã Princes haue no more right to be ſupreme Heads in
ſpiritual cauſes then the Heathen.}
is an inuincible demonſtration, that this text maketh not for any
ſpiritual claime of earthly Kings, becauſe it giueth no more to any
Prince then may and ought to be done & granted to a Heathen
Magiſtrate. Neither is there any thing in al the new Teſtament that
proueth the Prince to be Head or cheefe Gouernour of the Church in
ſpiritual or Eccleſiaſtical cauſes, more then it proueth any heathen
Emperour of Rome to haue been. For they were bound in temporal things to
obey the Heathen being lawful Kings, to be ſubiect to them euen for
conſcience, to keep their temporal lawes, to pay them tribute, to pray
for them, and to doe al other natural duties: and more no Scriptures
bind vs to doe to Chriſtian Kings.}
to King, as excelling: \V or to Rulers as ſent by him to the reuenge of
malefactours, but to the praiſe of the good: \V for ſo is the wil of
God, that doing wel you may make the ignorance of vnwiſe men to be
dumme: \V as free, and
\LNote{Not as hauing.}{There
\MNote{Libertines.}
were ſome Libertines in thoſe daies, as there be now, that vnder
pretence of libertie of the Ghoſpel, ſought to be free from ſubiection
and lawes of men, as now vnder the like wicked pretence, Heretikes
refuſe to obey their ſpiritual Rulers and to obſerue their lawes.}
not
%%% o-2758
as hauing the freedom for a cloke of malice, but as the ſeruants of
God. \V Honour al men.
\SNote{In this ſpeach is often commẽded the vnitie of al Chriſtians
among themſelues.}
Loue the fraternitie. Feare God. Honour the King.

\V Seruants be ſubiect in al feare to your Maiſters, not only to the
good & modeſt,
\LNote{But alſo the wayward.}{The
\MNote{Deadly ſinnes of Princes or Superiours exempt not the ſubiects
from obedience, as Wicleffe held.}
Wiclefiſtes and their followers in theſe daies, ſometimes to moue the
people vnto ſedition, hold and teach that Maiſters, and Magiſtrates loſe
their authoritie ouer their ſeruants and ſubiects, if they be once in
deadly ſinne, & that the people in that caſe need not in conſcience obey
them. Which is a pernicious and falſe doctrine, as is plaine by this
place, where we be expreſly commanded to obey euen the il
conditioned. Which muſt be alwaies vnderſtood, if they command nothing
againſt God. For then this rule is euer to be followed: \Emph{We muſt
obey God rather then men.}
\XRef{Act.~5,~29.}}
but alſo to the waiward. \V For this is thankes, if for conſcience of
God a man ſuſtaine ſorrowes, ſuffering vniuſtly. \V For what glorie is
it, if ſinning, and buffeted you ſuffer? But if doing wel you ſuſtaine
patiently, this is thanke before God. \V For vnto this are you called:
becauſe Chriſt alſo ſuffred for
\Var{vs}{you}
leauing
\Var{you}{vs}
an example that you may follow his ſteps. \V
\CNote{\XRef{Eſ.~53,~9.}}
\Emph{Who did no ſinne, neither was guile found in his mouth.} \V Who
when he was reuiled,
\TNote{\G{οὐκ ἀντελοιδόρει}}
did not reuile: when he ſuffred he threatned not: but deliuered himſelf
to him that iudged him vniuſtly. \V Who himſelf
\CNote{\XRef{Eſ.~53,~9.}
\XRef{Mt.~8.~17.}}
bare our ſinnes in his body vpon the tree: that dead to ſinnes, we may
liue to iuſtice. By whoſe ſtripes you are healed. \V For you were as
ſheep ſtraying: but you be conuerted now to the Paſtour & Bishop of your
ſoules.


\stopChapter


\stopcomponent


%%% Local Variables:
%%% mode: TeX
%%% eval: (long-s-mode)
%%% eval: (set-input-method "TeX")
%%% fill-column: 72
%%% eval: (auto-fill-mode)
%%% coding: utf-8-unix
%%% End:

