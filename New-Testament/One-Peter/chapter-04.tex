%%%%%%%%%%%%%%%%%%%%%%%%%%%%%%%%%%%%%%%%%%%%%%%%%%%%%%%%%%%%%%%%%
%%%%
%%%% The (original) Douay Rheims Bible 
%%%%
%%%% New Testament
%%%% One Peter
%%%% Chapter 04
%%%%
%%%%%%%%%%%%%%%%%%%%%%%%%%%%%%%%%%%%%%%%%%%%%%%%%%%%%%%%%%%%%%%%%




\startcomponent chapter-04


\project douay-rheims


%%% 2900
%%% o-2762
\startChapter[
  title={Chapter 04}
  ]

\Summary{That they arme themſelues, to ſinne no more after Baptiſme,
  againſt the tentations of the Heathen, conſidering that the general
  end now approcheth: 8.~ſpecially toward their euen-Chriſtians to shew
  their charitie, hoſpitalitie, and grace, doing al to the glorie of
  God. 12.~And as for being perſecuted becauſe they are Chriſtians, to
  reioyce, conſidering the reward that they shal haue with Chriſt, and
  damnation that they auoid hereby.}

Chriſt therfore hauing ſuffered in the fleſh, be you alſo armed with the
ſame cogitation. Becauſe he that hath ſuffered in the fleſh, hath ceaſed
from ſinnes: \V that now not after the deſires of men, but according to
the wil of God he liue the reſt of his time in the fleſh. \V For the
time paſt ſufficeth (to accompliſh the wil of the Gentils) them that
haue walked in riotouſnes, deſires, exceſſe of wine, banketings,
potations, and vnlawful ſeruices of Idols. \V Wherein they maruel
blaſpheming, you not concurring into the ſame confuſion of
riotouſnes. \V Who shal render account to him, which is ready to iudge
the liuing and the dead. \V For, for this cauſe alſo was
\SNote{It hath the ſame difficulty and ſenſe that the other like words
haue before,
\XRef{chap.~3.}
See the
\XRef{annotation there v.~19.}
and
\Cite{S.~Aug. ep.~69.}
&
\Cite{Oecumenius vpon this place.}}
it euangelized to the dead: that they may be iudged indeed according to
men, in the fleſh: but may liue according to God in the Spirit. \V And
the end of al
\Var{ſhal approch.}{is at hand.}

%%% 2901
Be wiſe therfore, and watch in praiers. \V But before al things, hauing
mutual charitie continual among your ſelues: becauſe
\CNote{\XRef{Pro.~10.}
\XRef{Ro.~12.}}
\LNote{Charitie couereth.}{Faith
\MNote{Not only faith.}
only cannot iuſtifie, ſeeing that charitie alſo doth cauſe remiſſion of
ſinnes.
\MNote{Workes of mercie.}
And ſaying charitie, he meaneth loue and charitable workes toward our
neighbours, vnto which workes of mercie the Scriptures doe ſpecially
attribute the force to extinguish al ſinnes. See S.~Auguſtin
\Cite{c.~69. Enchiridij}
and
\Cite{tract.~1. in ep.~1. Io. c.~1.}
& Venerable Bede
\Cite{vpon this place.}
And in the like ſenſe the holy Scriptures commonly commend vnto vs almes
and deeds of mercie for redemption of our ſinnes.
\XRef{Prouerb. c.~10.}
\XRef{Eccleſiaſtici~12. v.~2.}
\XRef{Danielis c.~4. v.~24.}}
charitie couereth the multitude of ſinnes. \V
\CNote{\XRef{Heb.~13.}}
Vſing hoſpitalitie one toward another without murmuring. \V 
\CNote{\XRef{Ro.~12,~6.}}
Euery one as he hath receiued grace, miniſtring the ſame one toward
another: as good diſpenſers of the manifold grace of God. \V If any man
ſpeake, as the words of God. If any man miniſter, as of the power, which
God adminiſtreth. That in al things God may be honoured by \Sc{Iesvs}
Chriſt: to whom is glorie & empire for euer and euer. Amen.

\V My deareſt, thinke it not ſtrange in the feruour which is to you for
a tentation, as though ſome new thing hapned to you: \V But
communicating with the paſſions of Chriſt, be glad, that in the
reuelation alſo of his glorie you may be glad reioycing. \V
\CNote{\XRef{Mt.~5,~13.}}
If you be reuiled in the name of Chriſt, you ſhal be bleſſed: becauſe
that which is of the honour,
%%% o-2763
glorie, and vertue of God, and the Spirit which is his, ſhal reſt vpon
you. \V But let none of you ſuffer as a murderer, or a theefe, or a
railer, or a coueter of other mens things. \V But if as a Chriſtian, let
him not be aſhamed, but let him glorifie God in this name. \V For
\CNote{\XRef{Hier.~25,~19.}}
the time is
\LNote{That iudgement begin.}{In
\MNote{The better mẽ moſt afflicted in this life.}
this time of the new Teſtament, the faithful and al thoſe that meane to
liue godly (ſpecially of the Clergie) muſt firſt and principally be
ſubiect to God's chaſtiſement and temporal afflictions, which are here
called iudgement. Which the Apoſtle recordeth for the comfort and
confirmation of the Catholike Chriſtians, who were at the time of the
writing hereof, exceedingly perſecuted by the heathen Princes and
people.}
that iudgement begin of the houſe of God. And if firſt of vs, what ſhal
be the end of them that beleeue not the Ghoſpel of God? \V And
\CNote{\XRef{Pro.~11,~31.}}
\LNote{If the iuſt.}{Not that a man dying iuſt & in the fauour of God,
can afterward be in doubt of his ſaluation, or may be reiected of God:
\MNote{The iuſt man himſelf is hardly ſaued.}
but that the iuſt being both in this life ſubiect to aſſaults,
tentations, troubles, and dangers of falling from God and looſing their
ſtate of iuſtice & alſo oftentimes to make a ſtrait count, and to be
temporally chaſtiſed in the next life, cannot be ſaued without great
watch, feare, and trembling, and much labouring and chaſtiſement.
\MNote{Againſt the vaine ſecuritie of only faith.}
And
this is farre contrarie to the Proteſtants doctrine, that putteth no
iuſtice but in faith alone, maketh none iuſt indeed and in truth,
teacheth men be ſo ſecure and aſſured of their ſaluatiõ, that he that
hath liued wickedly al his life, if he only haue their faith at his
death, that is, if
\Fix{the}{he}{obvious typo, fixed in other}
beleeue ſtedfaſtly that he is one of the elect, he shal be as ſure of
his ſaluation immediately after his departure, as the beſt liuer in the
world.}
if the iuſt man ſhal ſcarce be ſaued, where ſhal the impious & ſinner
appeare? \V Therfore they alſo that ſuffer according to the wil of God,
let them commend their ſoules to the faithful Creatour, in good deeds.


\stopChapter


\stopcomponent


%%% Local Variables:
%%% mode: TeX
%%% eval: (long-s-mode)
%%% eval: (set-input-method "TeX")
%%% fill-column: 72
%%% eval: (auto-fill-mode)
%%% coding: utf-8-unix
%%% End:

