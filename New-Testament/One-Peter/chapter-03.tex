%%%%%%%%%%%%%%%%%%%%%%%%%%%%%%%%%%%%%%%%%%%%%%%%%%%%%%%%%%%%%%%%%
%%%%
%%%% The (original) Douay Rheims Bible 
%%%%
%%%% New Testament
%%%% Epistles
%%%% One Peter
%%%% Chapter 03
%%%%
%%%%%%%%%%%%%%%%%%%%%%%%%%%%%%%%%%%%%%%%%%%%%%%%%%%%%%%%%%%%%%%%%




\startcomponent chapter-03


\project douay-rheims


%%% 2898
%%% o-2759
\startChapter[
  title={Chapter 03}
  ]

\Summary{The dutie of wiues & husbands to each other. 9.~None to doe or
  ſpeake euil by their perſecutours, 15.~but to anſwer them alwaies with
  modeſtie, and ſpecially with innocencie, after the example of Chriſt
  moſt innocent: whoſe body though they killed, yet his ſoule liued and
  preached afterward to the ſoules in Hel (namely to thoſe in the time
  of Noes floud being a figure of our Baptiſme) roſe againe, and
  aſcended.}

In
\SNote{How wiues should behaue themſelues toward their huſbands.}
like manner alſo
\CNote{\XRef{Eph.~5,~28.}
\XRef{Col.~3,~18.}}
let the women be ſubiect to their husbands: that if any beleeue not the
word, by the conuerſation of the women without the word they may be
wonne, \V conſidering your chaſt conuerſation in feare. \V
\CNote{\XRef{1.~Tim.~2,~9.}}
\SNote{Againſt the proud, curious and coſtly attire of women, wherin
this il time of ours exceedeth.}
Whoſe
trimming let it not be outwardly the plaiting of haire, or laying on
gold round about, or of putting on veſtures: \V but the man of the hart
that is hidden, in the
%%% o-2760
incorruptibilitie of a quiet and a modeſt ſpirit,
which is rich in the ſight of God. \V For ſo ſometime the holy women
alſo that truſted in God, adorned themſelues, ſubiect to their owne
husbands. \V As
\CNote{\XRef{Gen.~18.}}
Sara obeied Abraham, calling him Lord:
\TNote{\G{ὡς}}
whoſe daughters you are, doing wel, and not fearing any pertubation. \V
\SNote{How husbands should behaue themſelues toward their wiues.}
Husbands likewiſe, dwelling with them according to knowledge, as vnto
the weaker feminine veſſel imparting honour, as it were to the coheires
alſo of the grace of life: that your praiers be not hindred.

\V And
\Var{in fine}{in faith}
al of one mind, hauing compaſsion, louers of the Fraternitie, merciful,
modeſt, humble. \V
\CNote{\XRef{Pro.~17,~13.}
\XRef{Mat.~5,~44.}}
Not rendring euil for euil, nor curſe for curſe: but contrariewiſe,
bleſſing: for vnto this are you called, that you may by inheritance
poſſeſſe a benediction. \V
\CNote{\XRef{Pſ.~33,~13.}}
\Emph{For he that wil loue life, and ſee good daies, let him refraine
his tongue from euil, & his lippes that they ſpeake not guile. \V Let
him decline from euil, and doe good: let him enquire peace, & follow it:
\V becauſe the eyes of our Lord are vpon the iuſt, and his eares vnto
their praiers: but the countenance of our Lord vpon them that doe euil
things.} \V And who is he that can hurt you, if you be emulatours of
good? \V But
\CNote{\XRef{Mat.~5,~10.}}
& if you ſuffer ought for iuſtice, bleſſed are ye. And the feare of them
feare ye not, & be not troubled. \V But ſanctifie our Lord Chriſt
in your harts, ready alwaies to ſatisfie euery one that asketh you a
reaſon of that hope which is in you: \V but with modeſtie and feare,
hauing a good conſcience: that in that which they ſpeake il of you, they
may be confounded with calumniate
%%% 2899
your good conuerſation in Chriſt. \V For it is better to ſuffer as doing
wel (if the wil of God wil haue it ſo) then doing il.

\V Becauſe Chriſt alſo died once for our ſinnes, the iuſt for the
vniuſt: that he might offer vs to God, mortified certes in flesh, but
quickned in ſpirit. \V In the which ſpirit comming he preached
\LNote{To them that were in priſon.}{Auguſtin
\MNote{Chriſt in ſoule deſcended vnto hel, whiles his body lay in the
graue.}
in his
\Cite{99.~Epiſtle in principio,}
confeſſeth this place to be exceeding hard to vnderſtand, & to haue many
difficulties which he could neuer explicate to his owne
ſatisfaction. Yet vnto Heretikes this and al other texts be eaſie, not
doubting but that is the ſenſe which themſelues imagin, whatſoeuer other
men deeme thereof. S.~Auguſtin only findeth himſelf ſure of this, that
Chriſt's deſcending into Hel in ſoule after his death, is plainely
proued hereby. Which thing he declareth there, to be conformable to
diuers other expreſſe words of holy Writ, and namely to this ſame
Apoſtles ſermon
\XRef{Act.~2.}
And at length he concludeth thus, \L{Quis ergo niſi infidelis negauerit
ſuiſſe apud inferas Chriſtam?} that is, \Emph{Therfore who but an
infidel, wil deny that Chriſt was in hel?}
\MNote{The Caluiniſts denying the ſame, are by S.~Auguſtins iudgement
infidels.}
Caluin then (you ſee) with al his followers are infidels, who inſteed of
this deſcending of Chriſt in ſoule after his death, haue inuented
another deſperate kind of Chriſt's being in Hel, when he was yet aliue
on the Croſſe. S.~Athanaſius alſo in his epiſtle cited by S.~Epiphanius
\Cite{hær.~77. in principio.}
and in his booke
\Cite{de incartatione Verbi propius initio.}
S.~Cyril
\Cite{de recta fide ad Theodoſium,}
Occumenius, and diuers others vpon this place, proue Chriſt's deſcending
to Hel. As they likewiſe declare vpon the words following, that he
preached to the Spirits or ſoules of men deteined in Hel or in priſon.

But
\MNote{Certaine difficulties whereof S.~Auguſtin doubteth.}
whether this word \Emph{Priſon} or \Emph{Hel} be meant of the inferiour
place of the damned, or of \L{Limbus patrum} called Abraham's boſome, or
ſome other place of temporal chaſtiſemẽt; and, to whom he preached
there, and who by his preaching or preſence there were deliuered, and
who they were that are called \Emph{Incredulous in the daies of Noe}; al
theſe things S.~Auguſtin calleth great profundities, confeſſing himſelf
to be vnable to reach vnto it: only holding faſt and aſſured this
article of our faith, that he deliuered none deputed to damnation in
the loweſt Hel, and yet not doubting but that he releaſed diuers out of
places of paines there.
\MNote{Purgatorie.}
Which cã not be out of any other place thẽ Purgatorie. See the
\Cite{ſaid Epiſtle,}
where alſo he inſinuateth other expoſitions for explication of the
manifold difficulties of this hard text, which were too long to reherſe,
our ſpecial purpoſe being only to note briefely the things that touch
the controuerſies of this time.}
to
\Var{them}{thoſe ſpirits}
alſo that were in priſon: \V which had been
\LNote{Incredulous ſometime.}{They
\MNote{What were the incredulous perſons of whõ the Apoſtle here
ſpeaketh.}
that take the former words, of Chriſt's deſcending to Hel, and
deliuering certaine there deteined, doe expound this, not of ſuch as
died in their infidelitie or without al faith in God, for ſuch were not
deliuered: but either of ſome that once were incredulous, and afterward
repented before their death: or rather & ſpecially of ſuch as otherwiſe
were faithful, but yet truſted not Noes preaching by his worke and word,
that God would deſtroy the world by water. Who yet being otherwiſe good
men, when the matter came to paſſe, were ſorie for their errour, and
died by the floud corporally, but yet in ſtate of ſaluation, & being
chaſtiſed for their fault in the next life, were deliuered by Chriſt's
deſcending thither. And not they only, but al others in  the like
condition. For the Apoſtle giueth theſe of Noes time but for an
example.}
incredulous ſometime,
\CNote{\XRef{Gen.~6.}
\XRef{Mt.~24.}}
when they expected the patience of God in the daies of Noe, when the
arke was a building: in the which, few, that is,
\CNote{\XRef{Gen.~7,~7.}}
eight ſoules were ſaued by water. \V Whereunto Baptiſme being
\LNote{Of the like forme.}{The
\MNote{Noes Arke & the water, a figure of Chriſt's Croſſe & Baptiſme.}
water bearing vp the Arke from ſinking, and the perſons in it from
drowning, was a figure of baptiſme, that likewiſe ſaueth the worthie
receiuers
\Fix{and from}{from}{obvious typo, fixed in other}
euerlaſting perishing. \Emph{As Noe} (ſaith S.~Auguſtin) \Emph{with his,
was deliuered by the water
\Fix{the}{and the}{obvious typo, fixed in other}
word, ſo the familie of Chriſt by Baptiſme ſigned with Chriſts Paſsion
on the Croſſe.}
\Cite{Li.~2. Cont. Fauſtum c.~14.}
Againe he ſaith,
\CNote{\Cite{Ibid. c.~17.}}
that as the water ſaued none out of the Arke, but was rather their
deſtruction; ſo the Sacrament of Baptiſme
\Fix{reciued}{receiued}{obvious typo, fixed in other}
out of the Catholike Church at Heretikes or Schiſmatikes hands, though
it be the ſame water & Sacrament that the Catholike Church hath, yet
profiteth none to ſaluation, but rather worketh their perdition.
\MNote{Baptiſme receiued of Heretikes or Schiſmatikes, when damnable,
when not.}
Which
yet is not meant in caſe of extreme neceſſitie, when the partie should
die without the ſaid Sacrament, except he tooke it at an Heretikes or
Schiſmatikes hand. Neither is it meant in the caſe of infants, to whom
the Sacrament is cauſe of ſaluation, they being in no fault for
receiuing it at the hands of the vnfaithful, though their parents and
freinds that offer them vnto ſuch to be baptized, be in no ſmal
fault. S.~Hierom to Damaſus Pope of Rome
\CNote{\Cite{Ep.~57.}}
compareth that See to the Arke, & them that communicate with it, to them
that were ſaued in the Arke: al other Schiſmatikes and Heretikes, to the
reſt that
\Fix{where}{were}{obvious typo, fixed in other}
drowned.}
of the like forme now ſaueth
\Var{you}{vs}
alſo: not the laying away of the filth of the flesh, but
\LNote{The examination of a good conſcience.}{The
\MNote{The ceremonies of Baptiſme, namely \L{abrenuntio} &c.}
Apoſtle ſeemeth to allude here to the very forme of Catholike Baptiſme,
conteining certaine interrogatories and ſolemne promiſes made of the
articles of the Chriſtian faith, and of good life, and of renouncing
Satan & al his pomps and workes. Which (no doubt) howſoeuer the
Caluiniſts eſteeme of them, are the very Apoſtolike ceremonies vſed in
the miniſtration of this Sacrament. See
\Cite{S.~Denys in fine Ec. hierarchiæ.}
\Cite{S.~Cyril li.~12. in Io. c.~64.}
\Cite{S.~Auguſtin ep.~23.}
\Cite{S.~Baſil de Sp. ſancto. c.~12. and 15.}
\Cite{S.~Ambroſe de ijs qui myſterijs initiantur. c.~2.~3.~4.}}
the examination of a good conſcience toward God by the reſurrection of
\Sc{Iesvs} Chriſt. \V Who is on the right hand of God, ſwallowing death,
that we might be made heires of life euerlaſting: being gone into
Heauen, Angels and Potentates and Powers ſubiected to him.


\stopChapter


\stopcomponent


%%% Local Variables:
%%% mode: TeX
%%% eval: (long-s-mode)
%%% eval: (set-input-method "TeX")
%%% fill-column: 72
%%% eval: (auto-fill-mode)
%%% coding: utf-8-unix
%%% End:

