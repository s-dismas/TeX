%%%%%%%%%%%%%%%%%%%%%%%%%%%%%%%%%%%%%%%%%%%%%%%%%%%%%%%%%%%%%%%%%
%%%%
%%%% The (original) Douay Rheims Bible 
%%%%
%%%% New Testament
%%%% Two Peter
%%%% Chapter 03
%%%%
%%%%%%%%%%%%%%%%%%%%%%%%%%%%%%%%%%%%%%%%%%%%%%%%%%%%%%%%%%%%%%%%%

%%% Latin checked by KK.



\startcomponent chapter-03


\project douay-rheims


%%% 2909
%%% o-2771
\startChapter[
  title={Chapter 03}
  ]

\Summary{Theſe two Epiſtles he writeth to confirme them in the Apoſtles
  doctrine, and warneth them of ſcorners that shal come, and denie
  Domes-day. 5.~Whoſe vaine argument he anſwereth, and giueth the
  reaſon of God's ſo long patience, 10.~exhorting to al holines of life
  in reſpect of that terrible day: 16.~Finally giuing warning of ſuch as
  miſinterpret S.~Paules Epiſtles & the other Scriptures, and that we
  muſt not for any thing fal from the true faith.}

This loe the ſecond epiſtle I write to you, my Deareſt,
\TNote{\L{in quibus}}
in which I ſtirre vp by admonition your ſincere mind: \V that you may be
mindful of thoſe words which I told you before from the holy Prophets,
and of your Apoſtles, of the precepts of our Lord and Sauiour. \V
Knowing this firſt, that
\CNote{\XRef{2.~Tim.~3.}
\XRef{1.~Iude.~18.}}
in the laſt daies ſhal come mockers in deceit, walking according to
their owne cõcupiſcẽces, \V ſaying, Where is his promiſe or his cõming? For
ſince the time that the Fathers ſlept, al things doe ſo perſeuere frõ
the beginning of creature. \V For they are wilfully ignorant of this,
that the Heauens were before, and the earth, out of water, and through
water, conſiſting by the word of God: \V by the which, that world then,
being ouerflowed with water perished. \V But the Heauens which now are,
and the earth, are by the ſame word kept in ſtore, reſerued to fire vnto
the day of iudgement and of the perdition of the impious men. \V But
this one thing be not ignorant of, my Deareſt, that
\CNote{\XRef{Pſal.~89.}}
one day with our Lord is as a thouſand yeares, & a thouſand yeares as
one day. \V Our Lord ſlacketh not his promiſe, as ſome doe eſteeme it:
but he doth patiently for you,
\CNote{\XRef{Ezec.~33.}
\XRef{1.~Tim.~2.}}
not willing that any periſh, but that al returne to
%%% 2910
penance. \V And
\CNote{\XRef{1.~Theſ.~5.}
\XRef{Apoc.~3.}}
the day of our Lord ſhal come as a theefe, in the which the Heauens ſhal
paſſe with great violence, but the elements ſhal be reſolued with heat,
and the earth and the workes which are in it, ſhal be burnt.

\V Therfore whereas al theſe things are to be diſſolued, what manner of
men ought you to be in holy cõuerſations and godlineſſes, \V expecting
and haſting vnto the comming of the day of our Lord, by which the
Heauens burning ſhal be reſolued, and the elements ſhal melt with the
heat of fire? \V But we expect
\CNote{\XRef{Eſa.~65,~17.}
\XRef{Apo.~21,~1.}}
new Heauens and a new earth
%%% o-2772
according to his promiſes, in which iuſtice inhabiteth.

\V For the which cauſe, my Deareſt, expecting theſe things, labour
earneſtly to be found immaculate and vnſpotted to him in peace: \V and
\CNote{\XRef{Ro.~2,~4.}}
the longanimitie of our Lord, doe ye account ſaluation, as alſo our moſt
deare Brother Paul according to the wiſdom giuen him hath written to
you: \V as alſo in al epiſtles ſpeaking in them of theſe things; in the
which are
\LNote{Certaine things hard.}{This
\MNote{The heretical proud ſpirit of priuate interpretation of
Scriptures.}
is a plaine text to conuince the Proteſtants, who (as al heretikes
lightly doe and did from the beginning) ſay the Scriptures be eaſie to
vnderſtand, and therfore may be not only read ſafely, but alſo expounded
boldly of al the people, as wel vnlearned as learned: and conſequently
euery one by himſelf and his priuate ſpirit, without reſpect of the
expoſitions of the learned Fathers, or expectation of the Churches,
their Paſtours and Prelates iudgement, may determine and make choice of
ſuch ſenſe as himſelf liketh or thinketh agreable. For this is partly
their ſaying, partly the neceſſarie ſequele of their foolish opinion,
which admitteth nothing but the bare Scriptures. And Luther ſaid that
the Scriptures were more plaine then al the Fathers commentaries: and ſo
al to be ſuperfluous but the Bible.
\Cite{Prefat. aſſert. art. damnat.}

Againſt
\MNote{The Scriptures be hard, namely S.~Paules epiſtles, ſpecially
where he ſpeaketh of iuſtification by faith.}
al which Diuelish and ſeditious arrogancie, tending to make the people
eſteem themſelues learned or ſufficient without their Paſtours and
ſpiritual Rulers help, to guide themſelues in al matters of doctrine and
doubts in religion: the holy Apoſtle here telleth and fore-warneth the
faithful, that the Scriptures be ful of difficultie, and ſpecially
S.~Paules epiſtles of al other parts of holy writ, and that ignorant men
and vnſtable or phantaſtical fellowes puffed to and fro with euery blaſt
of doctrine and hereſie, abuſe, peruert, and miſconſter them to their
owne damnation. And
\CNote{\Cite{De fid. & op. c.~14.}}
S.~Auguſtin ſaith, that the ſpecial difficulty in S.~Paules epiſtles,
which ignorant and euil men doe ſo peruert, and which S.~Peter meaneth,
is his hard ſpeach and much commendation of that faith which he ſaith
doth iuſtifie. Which the ignorant euen from the Apoſtles time, and much
more now, haue and doe ſo miſconſter, as though he had meant that only
faith without good workes could iuſtifie or ſaue a man. Againſt which
wicked collection and abuſe of S.~Paules words, the ſaid Father ſaith al
theſe Canonical or Catholike epiſtles were writtẽ.

But
\MNote{The Proteſtãts idle diſtinctiõ between difficultie in the
Epiſtles and difficultie in the things.}
the Heretikes here to shift of the matter, and to creep out after their
fashion, anſwer, that S.~Peter ſaith not, S.~Paules epiſtles be hard,
but that many things in them are hard. Which may be to the Catholikes an
example of their ſophiſtical euaſions from the euidence of God's
word.
\MNote{The Greek copies haue both, ſome \G{ἐν οἷς}, \Emph{in which
things}: ſome \G{ἐν αἷς}, \Emph{in which epiſtles}.}
As though it were not alone to ſay, \Emph{Such an Authour or
Writer is hard}: and, \Emph{There be many things in that Writer hard to
be vnderſtood.} For, whether it be that the argument and matter be high
and paſt vulgar capacitie, as that of predeſtination, reprobatiõ,
vocatiõ of the Gentils, & iuſtifying faith: or whether his mãner of
ſtile and writing be obſcure: al proue that his epiſtles be hard and
other Scriptures alſo: becauſe S.~Peter here affirmeth that by reaſon of
the difficulties in them, whether in the ſtyle, or in the depth of the
matter, the ignorant and vnſtable (ſuch as Heretikes be) doe peruert his
writings, as alſo other Scriptures, to theyr owne damnation. Whereby it
is plaine that it is a very dangerous thing for ſuch as be ignorant, or
for wild witted fellowes, to read the Scriptures. For ſuch conditioned
men be they that become Heretikes, and through ignorance, pride, &
priuate phantaſie, meeting with hard places of S.~Paules epiſtles or
other Scriptures, breed Hereſies.

And
\MNote{Not only the matter, but the ſtyle of the Scriptures is hard.}
that not only the things treated of in the holy Scriptures, but alſo
that the very manner of writing and enditing thereof, is high and hard,
and purpoſely by God's prouidence oppointed to be written in ſuch ſort,
ſee S.~Auguſtin
\Cite{li.~2. de doct. Chriſt. c.~6.}
&
\Cite{ep.~119.}
S.~Ambroſe
\Cite{ep.~34. in principio.}
S.~Hierom to Palinus
\Cite{ep.~103. c.~5.~6.~7.}
who alſo
\Cite{(ep.~65. c.~1.)}
ſaith that in his old age, when he should rather haue taught then be
taught, he went as farre as Alexandria, only to heare Didymus, and to
haue his help for the vnderſtanding of the Scriptures, & confeſſeth with
great thankes to the ſaid Didymus, that he learned of him that which
before he knew not. Dauid ſaith,
\CNote{\XRef{Pſ.~118.}}
\Emph{Giue me vnderſtanding and I wil ſearch thy law.} The Eunuch in the
Actes, ſaid,
\CNote{\XRef{Act.~8.}}
How can I vnderſtand without an interpreter? The Apoſtles,
\CNote{\XRef{Luc.~24. v.~45.}}
til Chriſt opened their ſenſe to vnderſtand the Scriptures, could not
vnderſtand them. The holy Doctours by continual ſtudie, watching, and
praying, had much a-doe to vnderſtand them: that great Clerke
S.~Auguſtin confeſſing in the foreſaid
\Cite{epiſtle~119. c.~21.}
that there were many moe things that he vnderſtood not, then that he
vnderſtood. The Heretikes ſay the Fathers did commonly erre, and how
could ſuch great wiſe learned men be deceiued in reading and expounding
the Scriptures, if they were not hard? And if they were hard to thẽ, how
are they eaſie to theſe new Maiſters the Heretikes? Finally, why doe
they write ſo many new gloſſes, ſcholies, commentaries, as a cart can
not carrie? Why doe Luther, Zuinglius, Caluin, and their Companions
agree no better vpon the interpretation of the Scriptures, if they be
not hard? Whereat ſtumbled al the old heretikes & the new, Arius,
Macedonius, Vigilantius, Neſtorius, Berengarius, Wicleffe, Proteſtants,
Puritanes, Anabaptiſts, and the reſt, but at the hardnes of the
Scriptures? They be hard then to vnderſtand, and Heretikes peruert them
to their owne damnation.}
certaine things hard to be vnderſtood, which the vnlearned and vnſtable
depraue, as alſo the reſt of the Scriptures, to their owne perdition. \V
You therfore, Brethren, fore-knowing, take heed leſt led aſide by the
errour of the vnwiſe you fal away from your owne ſtedfaſtnes. \V But
grow in grace and in knowledge of our Lord and Sauiour \Sc{Iesvs}
Chriſt. To him be glorie both now and vnto the day of eternitie. Amen.


\stopChapter


\stopcomponent


%%% Local Variables:
%%% mode: TeX
%%% eval: (long-s-mode)
%%% eval: (set-input-method "TeX")
%%% fill-column: 72
%%% eval: (auto-fill-mode)
%%% coding: utf-8-unix
%%% End:

