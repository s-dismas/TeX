%%%%%%%%%%%%%%%%%%%%%%%%%%%%%%%%%%%%%%%%%%%%%%%%%%%%%%%%%%%%%%%%%
%%%%
%%%% The (original) Douay Rheims Bible 
%%%%
%%%% New Testament
%%%% Two Peter
%%%% Chapter 01
%%%%
%%%%%%%%%%%%%%%%%%%%%%%%%%%%%%%%%%%%%%%%%%%%%%%%%%%%%%%%%%%%%%%%%

%%% Latin checked by KK.



\startcomponent chapter-01


\project douay-rheims


%%% 2905
%%% o-2766
\startChapter[
  title={Chapter 01}
  ]

\Summary{How much God hath done for them, making them Chriſtians: 5.~and
  that they again, muſt doe their part, not hauing only faith, but al
  other vertues alſo and good workes, that ſo they may haue the more
  aſſurance to enter into the Kingdom of Heauen. 13.~And that he is ſo
  careful to admonish them, knowing that his death is at hand, knowing
  alſo moſt certainely the comming of Chriſt by the witnes of the Father
  himſelf, as alſo by the Prophets. Concerning whom he warneth them that
  they follow not priuate ſpirits, but the Holy Ghoſt (ſpeaking now in
  the Church.)}

Simon Peter ſeruant and Apoſtle of \Sc{Iesvs} Chriſt, to them that haue
obtained equal faith with vs in the iuſtice of our God and Sauiour
\Sc{Iesvs} Chriſt. \V Grace to you & peace be accompliſhed in the
knowledge of God & Chriſt \Sc{Iesvs} our Lord:

\V As al things of his diuine power which pertaine to life & godlines,
are giuen vs by the knowledge of him which hath called vs by his owne
proper glorie and vertue, \V by whom he hath giuen vs moſt great
pretious promiſes: that by theſe you may be
%%% o-2767
made partakers of the diuine nature, flying the corruption of that
concupiſcence which is in the world. \V And you employing al care
miniſter ye in your faith, vertue: and in vertue, knowledge: \V and in
knowledge, abſtinence: and in abſtinence, patience: and in patience,
pietie: \V and in pietie, loue of the Fraternitie: and in the loue of
the Fraternitie, charitie. \V For if theſe things be preſent with you,
and abound, they ſhal make you not vacant nor without fruit in the
knowledge of our Lord \Sc{Iesvs} Chriſt. \V For, he that hath not theſe
things ready, is blind, and groping with his hand, hauing forgotten the
purging of his old ſinnes.

\V Wherfore, Brethren, labour the more that
\LNote{By good workes.}{Here
\MNote{Good workes muſt concurre with God's predeſtination to the effect
thereof.}
we ſee, that Gods eternal predeſtination and election conſiſteth with
good workes: yea that the certainty and effect thereof is procured by
man's free wil and good workes, and that our wel doing is a meane for vs
to attaine to the effect of Gods predeſtination, that is, to life
euerlaſting. And therfore it is a deſperate folly and a great ſigne of
reprobate perſons, to ſay, If I be predeſtinate, doe what I wil, I shal
be ſaued. Nay, the Apoſtle ſaith, if thou hope to be one of the
predeſtinate (for know it thou canſt not) doe wel, that thou maiſt be
the more aſſured to attaine to that thou hopeſt: or, make it ſure by
good workes. The Proteſtants in ſuch caſes not much liking theſe words,
\TNote{\G{διὰ τῶν καλῶν ἔργων}}
\Emph{by good workes}, though the latin haue it vniuerſally, and ſome
Greek copies alſo, as Beza confeſſeth, leaue them out in their
tranſlations, by their wonted policie.}
by good workes you may make ſure your vocation and election. For doing
theſe things,
%%% 2906
you ſhal not ſinne at any time. \V For ſo there ſhal be miniſtred to you
aboundantly an entrance into the euerlaſting Kingdom of our Lord and
Sauiour \Sc{Iesvs} Chriſt. \V For the which cauſe I wil begin to
admoniſh you alwaies of theſe things: and you indeed knowing and being
confirmed in the preſent truth. \V But I thinke it meet as long as I am
in this tabernacle, to ſtirre you vp by admonition: \V being certaine
that the laying away of my tabernacle is at hand, according as our Lord
\Sc{Iesvs} Chriſt alſo ſignified to me. \V And I wil doe my diligence,
you to haue often
\LNote{After my deceaſe alſo.}{Theſe
\MNote{The heretikes (according to their cuſtom) exclude this ſenſe
altogether by their falſe trãſlatiõ.}
words though they may be eaſily altered by conſtruction into diuers
ſenſes not vntrue, yet the correſpondence of the parts of the ſentence
going before and following, giue moſt plaine this meaning, that as
during his life he would not omit to put them in memorie of the things
he taught them, ſo after his death (which he knew should be shortly) he
would not faile to endeauour that they might be mindful of the
ſame. Signifying that his care ouer them should not ceaſe by death, &
that by his interceſſion before God after his departure, he would doe
the ſame thing for them that he did before in his life by teaching and
preaching. This is the ſenſe that the
\CNote{\Cite{Oecum. in hunc loc. Gagn.}}
Greek Scholics ſpeake of, and this is moſt proper to the text and
conſonant to the old vſe of this Apoſtle and other Apoſtolike Saints &
Fathers of the primitiue Church.

S.~Clement in his
\Cite{Epiſtle to S.~Iames our Lords Brother,}
witneſſeth, that
\MNote{S.~Peters Paſtoral care and protection of the Church after his
death.}
S.~Peter encouraging him to take after his deceaſe the charge of the
Apoſtolike Romane See, promiſed that after his departure he would not
ceaſe to pray for him & his flock, thereby to eaſe him of his Paſtoral
burden.
\Cite{To.~1. Concil. ep.~1. S.~Clem. in initio.}
And S.~Leo the Great, one of his Succeſſours in the ſaid See, often
attributeth the good adminiſtration and gouernment thereof to S.~Peters
praiers & aſſiſtance: namely in theſe goodly words
\Cite{Ser.~3. in Anniuerſ. die aſſumpt. ad Pontif.}
\Emph{We are much bound} (ſaith he) \Emph{to giue thankes to our Lord
and Redeemer Ieſus Chriſt, that hath giuen ſo great power to him whom he
made the Prince of the whole Church; that if in our time alſo any thing
be done wel & be rightly ordered by vs, it is to be imputed to his
workes and his gouernment, to whom it was ſaid},
\CNote{\XRef{Luc.~22.}
\XRef{Io.~21.}}
And thou being conuerted confirme thy Brethren: \Emph{to whom our Lord
after his reſurrection ſaid thriſe}, Feed my sheep. \Emph{Which now alſo
without doubt the godly Paſtour doth execute, confirming vs with
\Fix{is}{his}{obvious typo, fixed in other}
exhortations, and not ceaſing to pray for vs, that we be ouercome
with no tentation, &c.}

Yea
\MNote{The Saints in heauen pray for the liuing.}
it was a commong thing in the Primitiue Church among the ancient
Chriſtians, and alwaies ſince among the faithful to make couenant in
their life time, that whether of them went to heauen before the other,
he should pray for his freind & fellow yet aliue. See the Eccleſiaſtical
hiſtorie of the holy Virgin & Martyr Potamiæna, promiſing at the houre
of her Martyrdom, that after her death she would procure mercie of God
to Baſilides one of the ſouldiars that led her to execution, and ſo she
did
\Cite{Euſeb. lib.~6. c.~4.}
Alſo S.~Cyprian
\Cite{ep.~57. in fine.}
\Emph{Let vs} (ſaith he) \Emph{pray mutually one for another, and
whether of vs two shal by God's clemencie be firſt called for, let his
loue continue, and his praier not ceaſe for his Brethren and Siſters in
the world.} So ſaid this holy Martyr at that time when Chriſtians were
ſo farre from Caluiniſme (which abhorreth the praiers of Saints &
praying to them) that to be ſure, they bargained before-hand to haue the
Martyrs & other Saints to pray for them. The ſame S.~Cyprian alſo in his
booke
\Cite{De diſciplina & habitu virginum, in fine,}
after a godly exhortation made to the holy Virgins or Nonnes in his
time, ſpeaketh thus vnto them: \L{Tantum tunc memento noſtri cum incipiet
in vobis virginitas honerari}: that is, \Emph{Only then haue vs in
remembrance, when your virginitie shal begin to be honoured}: that is,
after their departure.
\MNote{Feaſts of holy Virgins.}
Where he inſinuateth the vſe of the Catholike
Church in keeping the feſtiual daies and other duties toward the holy
Virgins in heauen. S.~Hierom alſo in the ſame manner ſpeaketh to
Heliodorus, ſaying, that when he is once in heauen, then he wil pray for
him that exhorted and incited him to the bleſſed ſtate of the Monaſtical
life,
\Cite{Ep.~1. c.~3.}

And
\MNote{Inuocation of Saints.}
ſo doth he ſpeake to the vertuous matrone Paula after her death,
deſiring her to pray for him in his old age, affirming that she shal the
more eaſily obtaine, the neerer she is now ioyned to Chriſt in heauen.
\Cite{in Epitaph. Paula in fine.}
It were too long to report, how S.~Auguſtin deſireth to be holpen by
S.~Cyprians praiers (then, and long before a Saint in heauen) to the
vnderſtanding of the truth concerning the peace and regiment of the
Church.
\Cite{li.~5. de Bapt. cont. Donatiſtas. c.~17.}
And in another place the ſame holy Doctour alleageth the ſaid Cyprian
ſaying, that great numbers of our parents, brethren, children, freinds,
& other, expect vs in great ſolicitude and carefulnes of our ſaluation,
being ſure of their owne.
\Cite{li.~1. de prædeſt. Sanctorum. c.~14.}
S.~Gregorie Nazianzen in his orations of the praiſe of S.~Cyprian
\Cite{in fine,}
and of S.~Baſil alſo
\Cite{in fine,}
declareth how they pray for the people. Which two Saints he there
inuocateth, as al the ancient Fathers did, both generally al Saints, and
(as occaſion ſerued) particularly their ſpecial Patrones. Among the reſt
ſee how holy Ephrem
\Cite{(in orat. de laud. S.~Deipara)}
praied to our B.~Ladie with the ſame termes of \Emph{Aduocate, Hope,
Reconciliatrix}, that the faithful yet vſe, and the Proteſtants can not
abide. S.~Baſil
\Cite{ho. de 40.~Martyribus in fine.}
S.~Athanaſius
\Cite{Ser. in Euang. de S.~Deipara in fine.}
S.~Hilarie
\Cite{in Pſal.~124.}
S.~Chryſoſtom
\Cite{ho.~46. ad po. Antiochenum in fine.}
Theodoret
\Cite{de curat. Græcorum affectuum li.~8. in fine.}
Finally al the Fathers are ful of theſe things: who better knew the
meaning of the Scripture and the ſenſe of the Holy Ghoſt, then theſe new
interpreters doe.}
after my deceaſe alſo, that you may keep a memorie of theſe things.

\V For, not hauing followed vnlearned fables, haue we made the power and
\Var{preſence}{preſcience}
of our Lord \Sc{Iesvs} Chriſt knowen to you: but
\SNote{By this it is plaine, that either Iohn, Iames, or Peter muſt be
the Authour of this epiſtle. For theſe three only were preſent at the
Transfiguration.
\XRef{Mat.~17,~1.}}
made beholders of his greatneſſe. \V For,
\CNote{\XRef{Mt.~17,~5.}}
he receiuing from God his Father honour and glorie, this manner of voice
comming downe to him from the magnifical glorie, \Emph{This
\Fix{is my}{my}{obvious typo, fixed in other}
beloued
Sonne in whom I haue pleaſed my ſelf, heare him.} \V And this voice we
heard brought from heauen, when
\Fix{the}{we}{obvious typo, fixed in other}
were with him in the
\SNote{You ſee that places are made holy by Chriſt's preſence, & that al
places be not alike holy. See
\XRef{Annot. Act.~7,~33.}}
holy mount. \V And we haue the Prophetical word more ſure: which you doe
wel attending vnto, as to a candel ſhining in a darke place, vntil the
day dawne, & the day-ſtarre ariſe in your harts: \V vnderſtanding this
firſt, that no prophecie of Scripture is made by
\LNote{Priuate.}{The
\MNote{Priuate phantaſtical interpretations.}
Scriptures can not be rightly expounded of euery priuate ſpirit or
phantaſie of the vulgar reader: but by the ſame ſpirit wherewith they
were writtẽ, which is reſident in the Church.}
priuate interpretation. \V For,
\CNote{\XRef{2.~Tim.~3,~17.}}
not by man's wil was prophecie brought at any time: but the holy men of
God ſpake, inſpired with the Holy Ghoſt.


\stopChapter


\stopcomponent


%%% Local Variables:
%%% mode: TeX
%%% eval: (long-s-mode)
%%% eval: (set-input-method "TeX")
%%% fill-column: 72
%%% eval: (auto-fill-mode)
%%% coding: utf-8-unix
%%% End:

