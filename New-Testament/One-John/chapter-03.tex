%%%%%%%%%%%%%%%%%%%%%%%%%%%%%%%%%%%%%%%%%%%%%%%%%%%%%%%%%%%%%%%%%
%%%%
%%%% The (original) Douay Rheims Bible 
%%%%
%%%% New Testament
%%%% One John
%%%% Chapter 03
%%%%
%%%%%%%%%%%%%%%%%%%%%%%%%%%%%%%%%%%%%%%%%%%%%%%%%%%%%%%%%%%%%%%%%




\startcomponent chapter-03


\project douay-rheims


%%% 2918
%%% o-2780
\startChapter[
  title={Chapter 03}
  ]

\Summary{It is not for the ſonnes of God, to ſinne mortally, but for the
  ſonnes of the Diuel, wherby they are knowen one from another, & not by
  only faith. 11.~True faith is, that we alſo loue our Brethren, giuing
  both our life and ſubſtance for them. 19.~Such vnfeined loue may haue
  great confidence before God. 23.~Becauſe the keeping of his
  commandements doth much pleaſe him, which conſiſt in faith and
  charitie.}

See what manner of charitie the Father hath giuen vs, that we ſhould be
named and be
\SNote{Not by nature, as Chriſt is: but by grace and adoption.}
the ſonnes of God. For this cauſe the world doth not know vs, becauſe it
hath not knowẽ him. \V My Deareſt, now we are the ſonnes of God; & it
hath not yet appeared what we ſhal be. We know that when he shal appeare, we
ſhal be like to him: becauſe we ſhal
\SNote{How we shal ſee God & be like vnto him in the next life, ſee
S.~Auguſtin,
\Cite{ep.~111.}
\Cite{112}
&
\Cite{li.~12. de ciuit. Dei. c.~29.}}
ſee him as he is. \V And euery one that hath this hope in him,
\SNote{This teacheth vs that man ſanctifieth himſelf by his free-wil
working together with Gods grace. S.~Auguſtin
\Cite{vpon this place.}}
ſanctifieth himſelf, as he alſo is holy. \V Euery one that committeth
ſinne, committeth alſo iniquitie: and
\LNote{Sinne is iniquitie.}{Iniquitie
\MNote{Concupiſcẽce remaining after Baptiſme is no ſinne, without
conſent.}
is not taken here for wickednes, as it is commonly vſed both in Latin
and in our language, as is plaine by the Greek word \G{ἀνομία},
ſignifying nothing els but a ſwaruing or declining from the ſtraight
line of the law of God or nature. So that the Apoſtle meaneth, that
euery ſinne is an obliquitie or defect from the rule of the law: but not
contrarie, that euery ſuch ſwaruing from the law, should be properly a
ſinne, as the Heretikes doe vntruely gather, to proue that concupiſcence
remaining after Baptiſme is a very ſinne, though we neuer giue our
conſent vnto it. And though in the
\XRef{5.~chapter following verſ.~17.}
the Apoſtle turne the ſpeach, affirming euery iniquitie to be a ſinne,
yet there the Greek word is not the ſame as before, \G{ἀνομία}, but
\G{ἀδικία}. By which it is plaine that there he meaneth
by \Emph{iniquitie}, mans actual and proper tranſgreſſion which muſt
needs be a ſinne. See S.~Auguſtin
\Cite{cont. Iulian. li.~5. c.~5.}
S.~Ambr.
\Cite{li. de Apologia Dauid. c.~13.}}
ſinne is iniquitie. \V And
\CNote{\XRef{Eſ.~53,~4.}}
you know that he appeared to take away our ſinnes:
\CNote{\XRef{1.~Pet.~2,~24.}}
and ſinne in him there is none. \V Euery one that abideth in him,
\LNote{Sinneth not.}{Iouinian
\MNote{Heretical expoſition of Scriptures.}
& Pelagius falſely (as Heretikes vſe to doe) argued vpon theſe words and
thoſe that follow vers.~9: the one, that the baptized could ſinne no
more; the other, that no man being or remaining iuſt could ſinne. But
among many good ſenſes giuen of this place, this ſeemeth moſt agreable,
that the Apoſtle should ſay,
\MNote{No man in grace ſinneth mortally.}
that mortal ſinne doth not conſiſt together with the grace of God, &
therfore can not be committed by a man continuing the ſonne of God. And
ſo is the like ſpeach in the 9.~verſe following to be taken. See
S.~Hierom
\Cite{li.~2. cont Iouinianum c.~1.}}
ſinneth not: and euery one that ſinneth, hath not ſeen him, nor knowen
him. \V Litle children, let no man ſeduce you.
\LNote{He that doeth iuſtice.}{He
\MNote{True iuſtice.}
doeth inculcate this often, that man's true iuſtice or righteouſnes
conſiſteth in doing or working iuſtice, and that ſo he is iuſt, and
biddeth them not to be ſeduced by Heretikes, in this point.}
He that doeth iuſtice, is iuſt: euen as he alſo is iuſt. \V
\CNote{\XRef{Io.~8,~44.}}
He that committeth ſinne,
%%% o-2781
is of the diuel:
%%% 2919
becauſe the diuel
\LNote{Sinneth from the beginning.}{The Diuel was created holy and in
grace, and not in ſinne: but he fel of his owne free wil from
God. Therfore theſe words \Emph{from the beginning}, may be interpreted
thus, from the beginning of ſinne, and ſo the Apoſtle wel ſay, The Diuel
committed the firſt ſinne. So S.~Auguſtin
\Cite{li.~11. de cie. Dei c.~15.}
expoundeth it.
\MNote{How the Diuel ſinned frõ the beginning.}
The moſt ſimple meaning ſeemeth to be, that he ſinned from the beginning
of the world, not taking the beginning preciſely for the firſt inſtant
or moment of the creation, but ſtraight vpon the beginning, as it muſt
needs alſo be taken in
\XRef{S.~Iohn's Ghoſpel c.~8,~44.}}
ſinneth from the beginning. For this, appeared the Sonne of God, that he
might diſſolue the workes of the diuel. \V Euery one that is borne of
God, committeth not ſinne: becauſe his ſeed abideth in him, and he can
not ſinne becauſe he is borne of God. \V In this are the children of God
manifeſt, and the children of the diuel. Euery one that is not iuſt, is
not of God, and he that loueth not his brother. \V Becauſe this is the
annuntiation, which you haue heard from the beginning,
\CNote{\XRef{Io.~13,~15.}}
That you loue one another. \V Not as
\CNote{\XRef{Gen.~4,~8.}}
Cain, who was of the wicked, and killed his brother. And for what cauſe
killed he him? Becauſe his workes were wicked: but his brothers, iuſt.

\V Maruel not, Brethren, if the world hate you. \V We know that we are
tranſlated from death to life, becauſe we loue the Brethren. He that
loueth not, abideth in death. \V Whoſoeuer hateth his brother, is a
murderer. And you know that no murderer hath life euerlaſting abiding in
himſelf. \V
\CNote{\XRef{Io.~15,~13.}}
In this we haue knowen the charitie of God, becauſe he hath yealded his
life for vs: and we ought to yeald our liues for the Brethren. \V
\CNote{\XRef{Ia.~2,~15.}}
He that ſhal haue the ſubſtance of the world, and ſhal
\SNote{Euery man is bound to giue almes according to his abilitie, when
he ſeeth his brother in great neceſſitie.}
ſee his brother haue need, and ſhal ſhut his bowels from him: how doth
the charitie of God abide him?

\V My litle children, let vs not loue in word, nor in tongue but in deed
and truth. \V In this we know that we are of the truth: and in his ſight
we ſhal perſuade our harts. \V For if our hart doe reprehend vs, God is
greater then our hart, and knoweth al things. \V My Deareſt, if our hart
doe not reprehend vs, we haue confidence toward God. \V And
\CNote{\XRef{Mt.~21.}
\XRef{Io.~14.}
\XRef{1.~Io.~5.}}
whatſoeuer we ſhal aske,
\LNote{We shal receiue, becauſe.}{Let
\MNote{Not only faith.}
the Proteſtants be ashamed to ſay, that we obtaine al of God by only
faith, the Apoſtle here attributing it to the keeping of God's
commandements. Note here alſo that God's commandements are not impoſſible
to be kept, but were then, and are now obſerued of good men.}
we ſhal receiue of him: becauſe we keep his commandements, and doe thoſe
things which are pleaſing before him. \V And
\CNote{\XRef{Io.~17,~3.}
\XRef{13,~34.}}
this is his commandement, that we beleeue in the name of his Sonne
\Sc{Iesvs} Chriſt: and
\SNote{Leſt any man should thinke by the words next before, only faith
in Chriſt to be commanded or to pleaſe God, he addeth to faith, the
commandement of charitie or loue of our neighbour.}
loue one another, as he hath giuen commandement vnto vs. \V And
\CNote{\XRef{Io.~14,~23.}}
he that keepeth his commandements, abideth in him, and he in him. And in
this we know that he abideth in vs, by the Spirit which he hath giuen vs.


\stopChapter


\stopcomponent


%%% Local Variables:
%%% mode: TeX
%%% eval: (long-s-mode)
%%% eval: (set-input-method "TeX")
%%% fill-column: 72
%%% eval: (auto-fill-mode)
%%% coding: utf-8-unix
%%% End:

