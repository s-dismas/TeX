%%%%%%%%%%%%%%%%%%%%%%%%%%%%%%%%%%%%%%%%%%%%%%%%%%%%%%%%%%%%%%%%%
%%%%
%%%% The (original) Douay Rheims Bible 
%%%%
%%%% New Testament
%%%% One John
%%%% Chapter 04
%%%%
%%%%%%%%%%%%%%%%%%%%%%%%%%%%%%%%%%%%%%%%%%%%%%%%%%%%%%%%%%%%%%%%%




\startcomponent chapter-04


\project douay-rheims


%%% 2920
%%% o-2782
\startChapter[
  title={Chapter 04}
  ]

\Summary{We may not beleeue al that boaſt of the ſpirit, but trie them,
  whether they teach Catholike articles of the faith (namely the
  incarnation of Chriſt:) whether their doctrine be not worldly, and
  themſelues diſobedient hearers of the Apoſtles. 7.~We muſt loue one
  another, conſidering the exceeding loue of God in ſending his Sonne to
  ſaue vs. 17.~An argument of perfect charitie is, if we haue nothing in
  our conſcience to feare in the day of iudgement. 19.~And an argument
  that we loue God is, if we loue our Brethren.}

My Deareſt,
\LNote{Beleeue not euery ſpirit.}{That
\MNote{Heretical boaſting of the ſpirit.}
is, Receiue not euery doctrine of ſuch as boaſt themſelues to haue the
ſpirit. For there be many falſe Prophets, that is to ſay, Heretikes,
which shal goe out of the Church, and chalenge the ſpirit, and vant of
God's word, Scripture, and Ghoſpel, which indeed be ſeducers.}
beleeue not euery ſpirit, but
\LNote{Proue the ſpirits.}{It
\MNote{The Church only, not euery priuate man, hath to proue & diſcerne
ſpirits.}
is not meant by this place, as the Proteſtants would haue it, that
euery particular perſon should of himſelf examine, trie, or iudge who is
a true or falſe Doctour, and which is true or falſe doctrine. But the
Apoſtle here would euery one to diſcerne theſe diuerſities of ſpirits,
by taking knowledge of them to whom God hath giuen the guift of
diſcerning ſpirits and doctrines (which S.~Paul expreſly ſaith is giuen
but to ſome, and not to euery one,
\XRef{1.~Cor.~12.)}
& by obeying the Church of God, to whom Chriſt hath giuen
\CNote{\XRef{Io.~14,~16.}}
the Spirit of truth.  And this is only the ſure way to proue the ſpirits
and doctrines of theſe daies. And al they that would bring vs from our
Paſtours and the Churches iudgement, to our owne priuate trial, ſeeke
nothing els but to driue vs to miſerable vncertainty in al our beleefe:
\MNote{Caluin.}
As Caluin doth, who
\Cite{vpon this place}
ſaith, that priuate men may examine the general Councels doctrines.}
proue the ſpirits if they be of God: becauſe many falſe Prophets are
gone out into the world. \V In this is the ſpirit of God knowen.
\LNote{Euery ſpirit that confeſſeth.}{The
\MNote{To confeſſe or deny any article which the Cath. Church teacheth,
is at al times a certaine marke of Catholike or Heretike.}
Apoſtle ſpeaketh according to that time, and for that part of Chriſtian
doctrine which then was ſpecially to be cõfeſſed, taught, & mainteined
againſt certaine wicked Heretikes, Corinthus, Ebion, & the like, that
taught wickedly againſt the Perſon and both natures of
Chriſt \Sc{Iesvs}. The Apoſtle therfore giueth the faithful people this
tokẽ to know the true Teachers of thoſe daies frõ the falſe. Not that
this marke would ſerue for al times, or in caſe of al other falſe
doctrines, but that it was then a neceſſarie note. As if a good
Catholike Writer, Paſtour, or parents would warne al theirs, now in
theſe daies, to giue eare only to ſuch Teachers as acknowledge Chriſt
our Sauiour to be really preſent, and ſacrificed in the B.~Maſſe, & that
al ſuch are true Preachers and of God, the reſt to be of the Diuel,
or to be counted the ſpirit of Antichriſt. Which ſpirit of Antichriſt
(he ſaith) was come euen then, and is no doubt much more now in al
Heretikes, al being precurſours of that great Antichriſt which shal come
towards the later end.}
Euery ſpirit that confeſſeth \Sc{Iesvs} Chriſt to haue come in fleſh, is
of God: \V and euery ſpirit
\LNote{That diſſolueth.}{To
\MNote{Many old hereſies that diſſolued Chriſt.}
diſſolue, looſe, or ſeparate \Sc{Iesvs} a-ſunder, was proper to al thoſe
old Heretikes that taught either againſt his Diuinitie, or Humanitie, or
the Vnitie of his Perſon, being of two natures, as Cerinthus, Ebion,
Neſtorius, Eutyches, Manes, or Manichæus, Cerdon, Apelles, Apollinaris
and the like.
\MNote{The Greek text corrupted by old heretikes.}
And this is one place by which we may ſee that the cõmon
Greek copies be not euer authentical, & that our old approued
tranſlation may not alwaies be examined by the Greek that now is, which
the Proteſtants only follow: but that it is to be preſuppoſed, when our
old Latin text differeth plainely from the Greek, that in old time
either al or the more 
approued Greek reading was otherwiſe, & that often the ſaid Greek was
corrupted then or ſince by Heretikes or otherwiſe. For of the Greeks,
S.~Irenæus
\Cite{li.~3. c.~18:}
among the Latin Fathers, S.~Auguſtin
\Cite{tract.~6. in fine,}
S.~Leo
\Cite{ep.~10. c.~5.}
and Venerable Bede did read as we doe. And this reading maketh more
againſt the ſaid Heretikes, then that which the common Greek now hath,
to wit, \Emph{Euery ſpirit that confeſſeth not Chriſt to haue come in
flesh, is not of God.} Which is alſo in effect ſaid before vers.~2. And
that therfore it was corrupted and altered by Heretikes, ſee the words
of Socrates alſo a Greek Writer, very agreable to this purpoſe.
\CNote{\Cite{li.~7. c.~32.}}
\Emph{Neſtorius} (ſaith he) \Emph{being eloquent by nature}, which is
often in Heretikes, \Emph{accounted himſelf therfore learned, &
diſdained to ſtudy the old Interpreters, counting himſelf better then
them al: being ignorant that in S.~Iohns Catholike epiſtle the old}
(Greek) \Emph{copies had}: \Sc{Every One that Dissolveth Iesvs, Is Not of
God}. So ſaith he, adding moreouer that ſuch as would ſeparate the
diuinitie from the diſpenſation of Chriſt's humanitie, took out of the
old copies this ſenſe. For which the old Expoſitours noted that theſe
which would looſe \Sc{Iesvs}, had corrupted this Epiſtle. See alſo the
\Cite{Tripartite li.~12. c.~4.}}
that diſſolueth \Sc{Iesvs}, is not of God: and this is
\TNote{\G{τὸ τοῦ ἀντιχρίστου}}
Antichriſt, of whom you haue heard that he commeth, and now he is
%%% o-2783
in the world. \V You are of God, litle children, and haue ouercome
him. Becauſe greater is he that is in you, then he that is in the world.
\V They are of the world: therfore of the world they ſpeake, and the
world heareth them. \V We are of God.
\CNote{\XRef{Io.~8,~47.}
\XRef{10,~17.}}
He that knoweth God, heareth vs. He that is not of God, heareth vs not.
\LNote{In this we know.}{This
\MNote{A ſure marke of true or falſe Teachers.}
is the moſt ſure & general marke to know the true ſpirits and Prophets
from the falſe: that thoſe which be of God, wil heare and obey their
Apoſtles & lawful Paſtours ſucceeding the Apoſtles, & ſubmit themſelues
to the Church of God: the other, that be not of God, wil not heare
either Apoſtle, Paſtour, or Church, but be their owne Iudges.}
In this we know the ſpirit of truth, and the ſpirit of errour.

\V My Deareſt, let vs loue one another: becauſe charitie is of God. And
euery one that loueth is borne of God, & knoweth God. \V He that loueth
not, knoweth not God: becauſe God is charitie. \V
\CNote{\XRef{Io.~3,~16.}}
In this hath the charitie of God appeared in vs, becauſe God hath ſent
his only-begotten Sonne into the world that we may liue by him. \V In
this is
%%% 2921
charitie: not as though we haue loued him, but becauſe he hath loued vs,
and ſent his Sonne a propitiation for our ſinnes.

\V My Deareſt, if God hath ſo loued vs, we alſo ought to loue one
another. \V
\CNote{\XRef{Io.~1.~18.}
\XRef{1.~Tim.~6,~16.}}
God
\SNote{No man in this life, nor with corporal eyes, cã ſee the proper
eſſence or ſubſtance of the Deitie. See S.~Auguſt.
\Cite{ad Paulin. de vidẽdo Deo. ep.~112.}}
no man hath ſeen at any time. If we loue one another, God abideth in vs,
and his charitie in vs is perfited. \V In this we know that we abide in
him, and he in vs: becauſe he of his Spirit hath giuen to vs. \V And we
haue ſeen, and doe teſtifie, that the Father hath ſent his Sonne the
Sauiour of the world. \V Whoſoeuer ſhal confeſſe that \Sc{Iesvs} is the
Sonne of God, God abideth in him, and he in God. \V And we haue knowen
and haue beleeued the charitie, which God hath in vs. God is charitie:
and he that abideth in charitie, abideth in God, and God in him. \V In
this is charitie perfited with vs,
\LNote{That we may haue confidence.}{Confidence
\MNote{Againſt the Proteſtãts ſpecial faith and preſumptuous ſecuritie
of ſaluation.}
called in Latin \L{Fiducia}, is neither, al one with faith, nor a
perſuaſion infallible that maketh a man no leſſe ſecure and certaine of
his ſaluation, then of the things that we are bound to beleeue, as the
Proteſtants falſely teach: but it is only a hope wel corroborated,
confirmed, and ſtrengthned vpon the promiſes and grace of God, and the
parties merits. And the words both following and going before, proue alſo
euidently againſt the Proteſtants, that our confidence and hope in the
day of iudgement dependeth not only vpon our apprehẽſion of Chriſt's
merits by faith, or vpon his grace and mercie, but alſo vpon our
conformitie to Chriſt in this life, in charitie and good workes. And
that is the doctrine of S.~Peter when he ſaid,
\CNote{\XRef{2.~Pet.~1,~10.}}
\Emph{Labour, that by good workes you may make ſure your vocation and
election}: and S.~Paules meaning, when he ſaid,
\CNote{\XRef{2.~Tim.~4,~7.}}
\Emph{I haue fought a good fight, there is laid vp for me a crowne of
iuſtice, which our Lord wil render to me in that day a iuſt iudge.}}
that we may haue confidence in the day of iudgement: becauſe as he is,
we alſo are in the world. \V
\LNote{Feare is not in Charitie.}{The
\MNote{The feare of God in iuſt men, conſiſteth with charitie.}
Heretikes very falſly vnderſtand this place ſo, that Chriſtian godly men
ought to haue no doubt, miſtruſt, or feare of hel and damnation. Which
is moſt euidently againſt the Scriptures, commending euery-where vnto vs
the awe and feare of God and his iudgements. \Emph{Feare him} (ſaith our
Sauiour
\XRef{Mat.~10.)}
\Emph{that can caſt body and ſoul into hel.} And
\XRef{Pſal.~118.}
\Emph{Pearſe my flesh with thy feare.} Which feare of God's iudgements
cauſed
\CNote{\XRef{1.~Cor.~9.}}
S.~Paul and al good men to chaſtiſe their bodies, leſt they should be
reprobate and damned. And the wiſe man for this cauſe affirmeth him to
be happie,
\CNote{\XRef{Prou.~28.}}
\Emph{that is euer fearful}. And holy Iob ſaith,
\CNote{\XRef{Iob.~c.~9.}}
\Emph{I feared al my workes}. And the Apoſtle,
\CNote{\XRef{Phil.~2.}}
\Emph{With feare and trembling worke your ſaluation}. Which kind of
feare is euen in the iuſteſt men and moſt ful of charitie, conſiſting
wel with the ſame vertue, and is calleth \L{Filialis timor}, becauſe it
is ſuch as the good child ought to haue toward his Father.

But
\MNote{What feare agreeth not with charitie.}
there is a kind of feare which ſtandeth not with charitie, and is cleane
againſt hope alſo, that which bringeth ſuch perplexitie and anxietie of
conſcience, that it induceth a mã to miſtruſt or deſpaire of God's
mercies. That ſeruile feare alſo which maketh a man often to leaue
ſinning & to doe the external workes of iuſtice, not for any loue or
delight he hath in God or his lawes, but only for feare of damnation,
\MNote{Seruile feare is not il.}
though it be not il in it-ſelf, but very profitable, as that which
helpeth toward the loue of God, yet it ſtãdeth not with charitie
neither, but is daily more & more leſſened, & at length quite driuen out
by charitie. Of theſe kind of feares then the Apoſtle ſpeaketh, and (as
ſome expound) of the feare of men alſo, of which our Sauiour ſaith,
\CNote{\XRef{Mat.~10.}}
\Emph{Feare not them that kil the body.}}
Feare is not in charitie: but perfect charitie caſteth out feare,
becauſe feare hath painefulnes. And he that feareth, is not perfect in
charitie. \V Let vs therfore loue God, becauſe God firſt hath loued
vs. \V If any man ſhal ſay, that I loue God; and hateth his brother, he
is a lier. For he that loueth not his brother whom he ſeeth, God whom he
ſeeth not, how can he loue? \V
\CNote{\XRef{Io.~13,~34.}
\XRef{15,~12.}}
And this commandement we haue from God: that he which loueth God, loue
alſo his brother.


\stopChapter


\stopcomponent


%%% Local Variables:
%%% mode: TeX
%%% eval: (long-s-mode)
%%% eval: (set-input-method "TeX")
%%% fill-column: 72
%%% eval: (auto-fill-mode)
%%% coding: utf-8-unix
%%% End:

