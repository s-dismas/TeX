%%%%%%%%%%%%%%%%%%%%%%%%%%%%%%%%%%%%%%%%%%%%%%%%%%%%%%%%%%%%%%%%%
%%%%
%%%% The (original) Douay Rheims Bible 
%%%%
%%%% New Testament
%%%% One John
%%%% Chapter 05
%%%%
%%%%%%%%%%%%%%%%%%%%%%%%%%%%%%%%%%%%%%%%%%%%%%%%%%%%%%%%%%%%%%%%%




\startcomponent chapter-05


\project douay-rheims


%%% 2923
%%% o-2785
\startChapter[
  title={Chapter 05}
  ]

\Summary{They that loue God, muſt loue his natural Sonne \Sc{Iesvs}, and
  his ſonnes by adoption, & keep his commandements, which to the
  regenerate are light. 4.~But not, vnles they continue in the Catholike
  faith, namely of this article, that \Sc{Iesvs} is the Sonne of God,
  and therfore able to giue vs life euerlaſting, 14.~and al our
  petitions 16.~and our praiers for al our Brethren that ſinne not vnto
  death, dying in their mortal ſinnes by impenitence. Laſt of al, he
  warneth them not to communicate with Idols.}

Whoſoeuer beleeueth that \Sc{Iesvs} is Chriſt, is borne of God. And
euery one that loueth him which begat, loueth him alſo which was borne
of him. \V In this we know that we loue the children of God: when as we
loue God, and keep his commandements. \V For this is the charitie of
God, that we keep his commandements:
\CNote{\XRef{Mat.~11,~30.}}
and
\LNote{His commandements are not heauie.}{How
\MNote{The commandements poſſible to be kept.}
can the Proteſtants ſay that Gods commandemẽts cã not poſſibly be
fulfilled or kept in this life, ſeeing that the Apoſtle ſaith,
\Emph{they be not heauie}: and Chriſt ſaith,
\CNote{\XRef{Mat.~11,~30.}}
\Emph{his yoke is ſweete, and his burden light}? See for the ful
vnderſtanding of this place, S.~Aug.
\Cite{de perfectione iuſtitiæ c.~10.}
\MNote{Heret. tranſlation.}
The heretikes in fauour of their foreſaid errour, rather tranſlate,
\Emph{His commandements are not
\TNote{\G{βαρεῖαι}}
grieuous}, then, \Emph{are not heauie}.}
his commandements are not heauy. \V Becauſe al that is borne of God,
ouercommeth the world: And this is the victorie which ouercommeth the
world, our faith. \V Who is he
\CNote{\XRef{1.~Cor.~15,~57.}}
that ouercommeth the world, but he that beleeueth that \Sc{Iesvs} is the
Sonne of God? \V This is he that came by water & bloud \Sc{Iesvs}
Chriſt: not in water only, but in water and bloud. And it is
%%% o-2786
the Spirit which teſtifieth, that Chriſt is the truth.

\V For there be
\LNote{Three which giue teſtimonie.}{An
\MNote{Three perſons & one ſubſtance in the B.~Trinitie.}
expreſſe place for the diſtinction of three Perſons, and the vnitie of
nature and eſſence in the B.~Trinitie; againſt the Arians and other like
Heretikes, who haue in diuers Ages found themſelues ſo preſſed with
theſe plaine Scriptures, that
\MNote{The Arians corrupt the text of Scripture.}
they haue (as it is thought) altered and
corrupted the text both in Greek and Latin many waies: euen as the
Proteſtants handle thoſe textes that make againſt them. But becauſe we
are not now troubled with Arianiſme ſo much as with Caluiniſme, we need
not ſtand vpon the varietie of reading or expoſition of this
paſſage. See S.~Hierom, in his
\Cite{epiſtle put before the 7.~Canonical or Catholike Epiſtles.}}
three which giue teſtimonie in heauen, the Father, the Word, and the
Holy Ghoſt. And theſe three be one. \V And there be three which giue
teſtimonie in earth: the ſpirit, water, and bloud and theſe three be
one. \V If we receiue the teſtimonie of men, the teſtimonie of God is
greater. Becauſe this is the teſtimonie of God which is greater, that he
hath teſtified of his Sonne. \V
\CNote{\XRef{Io.~3,~36.}}
He that beleeueth in the Sonne of God, hath the teſtimonie of God in
himſelf. He that beleeueth not the Sonne, maketh him a lier: becauſe he
beleeueth not in the teſtimonie which God hath teſtified of his Sonne.
\V And this is the teſtimonie, that God hath giuen vs life
euerlaſting. And this life is in his Sonne. \V He that hath the Sonne,
hath life. He that hath not the Sonne of God, hath not life.

\V Theſe things I write to you, that you may know that you haue eternal
life which beleeue in the name of the Sonne of God. \V And this is the
confidence which we haue toward him: that,
\CNote {\XRef{Mt.~7,~7.
\XRef{21,~22.}}
\XRef{1.~Io.~3,~22.}}
whatſoeuer we ſhal aske according to his wil, he heareth vs. \V And
\Var{we know}{if we know}
that he heareth vs whatſoeuer we ſhal aske: we know that we haue the
petitions which we requeſt of him.

\V He that knoweth his brother to ſinne a ſinne not to death, let
%%% 2924
him aske, and life ſhal be giuen him, ſinning not to death. There is
\LNote{A ſinne to death.}{A
\MNote{What is a ſinne to death.}
ſinne to death is another thing then a mortal ſinne. For it is that
mortal ſinne only, whereof a man is neuer penitent before his death, or
in which he continueth til death, and dieth in it. \Emph{I affirme}
(ſaith S.~Auguſtin
\Cite{de correp. & grat. c.~12.)}
\Emph{that a ſinne to death is to leaue faith working by charitie euen
til death.} So likewiſe in the words before, \Emph{a ſinne not to
death}, is not that which we cal a venial ſinne, but any that a man
committeth and continueth not therin til death.}
a ſinne to death:
\LNote{For that I ſay not.}{If the ſinne to death whereof he ſpeaketh,
be the ſinne wherin a man dieth without repentance, according to
S.~Auguſtines wordes before rehearſed: then the praier which he ſpeaketh
of, muſt needs be praier for the dead.
\MNote{Praier for the dead.}
Becauſe he ſpeaketh of praying, or not praying, for them that died in
deadly ſinne, exhorting vs to pray, and encouraging vs to doe it with
confidence to be heard, if we pray for them that departed this life not
in deadly ſinne: and contrariwiſe in a mãner diſſuading & diſcouraging
vs from praying for ſuch as continued in wickednes euen til their liues
end. And S.~Auguſtin ſetteth downe the Churches practiſe agreable to the
Apoſtles meaning,
\Cite{li.~21. c.~24. de Ciuit. Dei.}
\MNote{Some of the dead may not be praied for.}
\Emph{If there be any} (ſaith he) \Emph{that perſiſt til death in
impenitencie of hart, doth the Church now pray for them, that is, for
the ſoules of them that ſo are departed?} So ſaith he. And this is the
cauſe, that
\Cite{Concilium Bracharenſe primum cap.~34.}
forbideth to pray for ſuch as die in deſperation, or kil themſelues: and
the reaſon, why the Church forbeareth to pray for Heretikes that die in
their hereſie, or mainteine hereſie vnto death and by their death.

And
\MNote{It is proued that the Apoſtle ſpeaketh of praying for the dead.}
that the place is moſt properly or only meant of praying for the
departed, this conuinceth, that neither the Church nor any man is
dehorted here from praying for any ſinner yet liuing, nor for the
remiſſion of any ſinne in this life: al ſinnes (of what ſort ſoeuer)
being pardonable, ſo long as the committers of them be in caſe and ſtate
to repent: as they be ſo long as they be in this world. And we ſee that
the Church praieth, and is often heard, for Heretikes, Iewes, Turkes,
Apoſtataes, and what other infidels or il mẽ ſoeuer, during their liues.
\MNote{The Caluiniſts blaſphemie, to auoid this ſenſe of the Apoſtle.}
And it is great blaſphemie that the Caluiniſtes vtter vpon this place: to
wit, that Apoſtaſie & certaine other ſinnes of the reprobate, can not be
forgiuen at al in this life. Which they hold, only to auoid the ſequele
of praying for the dead vpon theſe words of S.~Iohn. Beſides that they
muſt take vpon them preſumptuouſly, to know and diſcerne of God's
ſecrets, who be reprobate, and who be not, and according to that, pray
for ſome, and not for other-ſome: al which is moſt wicked and abſurd
preſumption.

As for their allegation, that S.~Ieremie the Prophet was forbidden to
pray for the Iewes, & warned that he should not be heard,
\Cite{Chap.~4.~11.~14.}
there is great difference. Firſt he had a reuelation by the words of
God, that they would continue in their wickednes, as we haue not of any
certaine perſon, whereof S.~Iohn here ſpeaketh. Secondly, Ieremie was
not forbidden to pray for the remiſsion of their ſinnes, nor had denial
to be heard therein for any man's particular caſe, whereof the Apoſtle
here ſpeaketh: but he was told that they should not eſcape the temporal
punishment & affliction which he had deſigned for them, and that he
would not heare him therin.}
for that I ſay not that any man aske. \V Al
\TNote{\G{ἀδικία}}
iniquitie, is ſinne. And there is a ſinne
\Var{to death.}{not to death.}
\V We know that euery one which is borne of God, ſinneth not: but the
generation of God preſerueth him, and the wicked one toucheth him
not. \V We know that we are of God, and the whole world is ſet in
wickedneſſe. \V And we know that the Sonne of God commeth: and he
\CNote{\XRef{Luc.~24,~45.}}
hath giuen vs vnderſtanding, that we may know the true God, & may be in
his true Sonne. This is the true God, & life euerlaſting. \V My litle
children, keep your ſelues
\LNote{From idols.}{It
\MNote{Heret. tranſlation againſt ſacred images.}
is ſo knowen a treacherie of Heretikes to tranſlate \L{idola} images (as
here and in a number of places, ſpecially of the English Bible printed
the yeare~1562) that we need not much to ſtand vpon it. As this alſo is
ſeen to al the world, that they doe it of purpoſe to ſeduce the poore
ignorant people, and to make them thinke, that whatſoeuer in the
Scriptures is ſpoken againſt the idols of the Gentils (which the Prophet
\CNote{\XRef{Pſal.~113.}}
calleth \L{Simulacra Gentium}) is meant of pictures, ſacred images, &
holy memories of Chriſt and his Saints. Againſt ſuch ſeducers the ſecond
ſacred Councel of Nice, called the ſeuenth Synod, decreeth thus
\CNote{Edit. Colon. an.~1567.}
\Cite{Act.~4. pag.~122.}
\MNote{The 2.~Councel of Nice pronounceth anathema, that is a curſe
againſt the Caluiniſts.}
\L{Quicumque ſententias ſacre ſcripturæ de Idolis, contra venerandas
imagines addueunt, anathema. Qui venerandas imagines idola appellant,
anathema. Qui dicunt quod Chriſtiani adorant imagines vt Deos,
anathema.} that is, \Emph{Anathema to al them that bring the ſentences of holy
Scripture touching Idols, againſt the venerable images. Anathema to them
that cal the venerable images, Idols. Anathema to them that ſay,
Chriſtians adore images as Gods.}

Now in their later tranſlation the Heretikes perceiuing that the world
ſeeth their vnhoneſt dealing, corrected themſelues in ſome places, and
in this place haue put, \Emph{idols}, in the text, but to giue the
people a watch-word that the Churches images are to be compriſed in the
word, \Emph{idols},
\CNote{The Bible of the yeare~1577.}
they haue put, \Emph{images}, in the margent.
\MNote{The great difference of idol and image.}
But concerning this matter, it is moſt euident that neither euery Idol
is an image, nor euery image an idol: and that, howſoeuer the origine or
etymologie of the word, \Emph{idol}, may be taken in the Greek, yet both
the words & the things be in truth and by the vſe of al tongues, farre
differing. The great dragon that the Babylonians adored
\XRef{(Dan.~14.)}
was an idol, but not an image: the Cherubins in Salomons Temple were
images, but not idols: and the face of the Queene in her coine or
els-where, as Cæſar's face vpon the coine that Chriſt called for, is an
image, but not an idol: and the Heretikes dare not tranſlate that text
of Scripture thus, \Emph{whoſe idol is this ſuperſcription?} nor cal the
Queenes image, the idol of the Queene: nor Chriſt, the idol of his
Father: nor woman, the idol of the man: nor man, the idol of God. Al
which in Scripture be named images for al that, and be ſo indeed, and
not idols. Which conuinceth, that the Heretikes be falſe and corrupt
tranſlatours in this place and other the like, confounding theſe two
words as if they were al one.

But
\MNote{Sacred images in Churches, by God's owne warrant.}
as for the hauing of images or purtraites of holy things, not only in
priuate houſes, but alſo in Churches, God himſelf doth warrant vs, who
\CNote{\XRef{Exod.~25.}}
commanded euen the Ieres themſelues (a people moſt prone to idolatrie,
and that after he had giuen them a ſpecial precept of not hauing,
making, or worshipping of idols) to make the images of Angels (the
Cherubins) and that in the ſoueraigne holieſt place of adoration that
was in the Temple, and about the Arke. Yea and in reſpect of which
ſacred images partly, they did (as S.~Hierom ſaith
\Cite{ep.~17. c.~3.)}
ſo great reuerence to the holy place called \L{Sancta ſanctorum}. If
they then were warranted & commanded to make and haue in ſo great
reuerence the images of mere ſpirits or Angels, whoſe natural shape
could not be expreſſed: how much more may we Chriſtians haue and
reuerence the images of Chriſt, his B.~mother, the Apoſtles and other
Saints, being men, whoſe shape may be expreſſed?
So doth the ſaid Nicene Councel argue againſt the Heretikes which at
that time were the Aduerſaries of images.

And note here, that eight hundred yeares agoe, they were ſtraight counted
Heretikes, that began to ſpeake againſt images,
\MNote{The 2.~Councel of Nice was gathered againſt image breakers.}
& that Councel was called purpoſely for them, and condemned them for
Heretikes, & confirmed the former ancient reuerence and vſe of ſacred
images.
\MNote{The antiquitie of holy images.}
Which began euen in our Sauiours time or litle after, when good
religious folke for loue and reuerence made his image, namely the woman
that he healed of the bloudy fluxe. Which image was alſo approued by
miracles, as the Eccleſiaſtical hiſtorie telleth, and namely Euſebius
\Cite{Eccl. hiſt. li.~7. c.~14.}
\CNote{\Cite{loco citato}}
who alſo witneſſeth that the images of Peter and Paul were in his
daies. As you may ſee alſo in S.~Aug.
\Cite{(li. de conſenſ. Euangeliſt. c.~10.)}
that their pictures commonly ſtood together in Rome, euen as at this
day. Of our Ladies image ſee S.~Gregorie
\Cite{li.~7. ep.~5. indict.~2. ad lannar}
&
\Cite{ep.~53.}
in whom alſo
\Cite{(li.~7. ep.~109.)}
you may ſee
\MNote{The vſe and fruite of holy images.}
the true vſe of images, and that they are the books of the vnlearned,
and that the people ought to be inſtructed and taught the right vſe of
them, euen as at this day good Catholike folke doe vſe them to help and
increaſe their deuotion in al Catholike Churches: yea the Lutherans
themſelues reteine them ſtil. S.~Damaſcene wrote three books in defenſe
of ſacred images againſt the foreſaid Heretikes.}
\TNote{\G{ἀπὸ τῶν εἰδώλων}}
from Idols. Amen.


\stopChapter


\stopcomponent


%%% Local Variables:
%%% mode: TeX
%%% eval: (long-s-mode)
%%% eval: (set-input-method "TeX")
%%% fill-column: 72
%%% eval: (auto-fill-mode)
%%% coding: utf-8-unix
%%% End:

