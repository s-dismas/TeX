%%%%%%%%%%%%%%%%%%%%%%%%%%%%%%%%%%%%%%%%%%%%%%%%%%%%%%%%%%%%%%%%%
%%%%
%%%% The (original) Douay Rheims Bible 
%%%%
%%%% New Testament
%%%% One John
%%%% Argument
%%%%
%%%%%%%%%%%%%%%%%%%%%%%%%%%%%%%%%%%%%%%%%%%%%%%%%%%%%%%%%%%%%%%%%




\startcomponent argument


\project douay-rheims


%%% 2912
%%% o-2774
\startArgument[
  title={\Sc{The Argvment of S.~Iohns Three Epistles.}},
  marking={Argument of S.~Iohns Epiſtles.}
  ]

Of S.~Iohn was ſaid in the
\XRef{Argument before his Ghoſpel.}
Now here follow
his three Epiſtles: one to al Catholikes (though
\CNote{\Cite{Higinus ep.~1. to.~1. Concil.}
\Cite{Auguſt. li.~2. Euang. quæſt. q.~39.}}
ſome ancient doe cal it, \Emph{Ad Parthos}:) the other two being very
short, vnto a certaine Ladie, & to one Gaius. The effect of al is, to
witnes vnto them the certaintie of the Catholike faith, & to exhort them
to continue ſtil in it: alſo to loue the Catholike Church, and ſo,
neither to become heretikes, nor Schiſmatikes: but rather to auoid al
ſuch, as the fore-runners of Antichriſt, and to remember, that
Catholikes need not to goe to ſchoole to any ſuch Maiſters, hauing at
home in the Catholike Church, the doctrine of the Holy Ghoſt himſelf,
who was giuen to the Church viſibly in the beginning, to lead her into
al truth, and to continue with her for euer. Therfore he ſaith:
\CNote{\XRef{1.~Iohn.~2. v.~42.}}
\Emph{That which you haue heard from the beginning, let it abide in
you.} Likewiſe a litle after,
\XRef{v.~27.}
and
\XRef{ep.~2. v.~6.}
\Emph{This is the commandement, that as you haue heard from the
beginning, you walke in the ſame, becauſe many ſeducers are gone out
into the world.} and
\XRef{v.~8. &~9.}

And not only thus in general, but alſo in particular he expreſſeth the
points which the heretikes did then cal in queſtion. Some were about
Chriſt himſelf. For they denied that \Sc{Iesvs} is Chriſt, that he is
the very Sonne of God, that he is incarnate.
\XRef{Ep.~1. c.~2. v.~22.}
and
\XRef{Ep.~2. v.~7.}
And againſt ſuch it was that he wrote his Ghoſpel alſo, as he there
ſignifieth
\XRef{Iohn.~20. v.~31.}
Other points are about our iuſtification, againſt only faith, and for
good workes, as alſo S.~Aug.
\CNote{\Cite{De fid. & op. c.~14.}}
noted, whoſe words were cited before. Hereupon he ſaith: \Emph{If we ſay
we haue ſocietie with God, and walke in darkenes, we lie.}
\XRef{Ep.~1. c.~1.}
Againe,
\CNote{\XRef{1.~Iohn.~2. &~5.}}
\Emph{He that ſaith he knoweth God, and keepeth not his commandements,
is a lier.} Againe, \Emph{This is the charitie of God, that we keep his
commandements, and his commandements are not heauie.} Finally,
\Emph{Children let no man ſeduce you. He that doth iuſtice, is iuſt,
euen as he is iuſt.}
\XRef{Ep.~1. c.~3. v.~7.~8.~9.}
Likewiſe
\XRef{c.~2. v.~29.}
and indeed in al the three Epiſtles throughout, he doth inculcate good
workes & keeping the commandements, againſt the hereſie of only faith.


\stopArgument


\stopcomponent


%%% Local Variables:
%%% mode: TeX
%%% eval: (long-s-mode)
%%% eval: (set-input-method "TeX")
%%% fill-column: 72
%%% eval: (auto-fill-mode)
%%% coding: utf-8-unix
%%% End:
