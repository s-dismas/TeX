%%%%%%%%%%%%%%%%%%%%%%%%%%%%%%%%%%%%%%%%%%%%%%%%%%%%%%%%%%%%%%%%%
%%%%
%%%% The (original) Douay Rheims Bible 
%%%%
%%%% New Testament
%%%% One John
%%%% Chapter 02
%%%%
%%%%%%%%%%%%%%%%%%%%%%%%%%%%%%%%%%%%%%%%%%%%%%%%%%%%%%%%%%%%%%%%%




\startcomponent chapter-02


\project douay-rheims


%%% 2915
%%% o-2776
\startChapter[
  title={Chapter 02}
  ]

\Summary{If any ſinne mortally, he muſt not deſpaire. 3.~To know God
  rightly, is not to beleeue only, but to keep his commandements: 7.~and
  that this is no new doctrine, but the very primitiue, though a new
  life it is. 9.~Therfore he that beleeueth muſt alſo loue his Brethren:
  12.~and that men muſt not loue the world but doe that which God
  willeth. 18.~Many are gone out of the Church and become Seducers, al
  the Miniſters of Antichriſt: but true Chriſtians muſt continue in
  their old faith, conſidering the reward, & that they need not goe to
  ſchole to any Heretike, the Holy Ghoſt himſelf being the
  Schole-maiſter of the Church. 29.~He doth earneſtly inculcate iuſtice
  and good workes.}

%%% o-2777
My litle children, theſe things I write to you,
\LNote{That you ſinne not.}{S.~Iohn (ſaith V.~Bede
\Cite{vpon this place)}
is not contrarie to himſelf, in that he ſeeketh here to make them
without ſinne, whom he ſaid in the laſt chapter could not be without al
ſinnes: but in the former place he warned vs only of our frailety, that
we should not arrogate to our ſelues perfect innocencie; here he
prouoketh vs to watchfulnes and diligence in reſiſting and auoiding
ſinnes, ſpecially the greater, which by God's grace may more eaſily be
repelled.}
that you ſinne not. But and if any man ſhal ſinne, we haue
\LNote{An aduocate.}{The
\MNote{How Chriſt is our only Aduocate.}
calling and office of an Aduocate, is in many things proper to Chriſt,
and in euery condition more ſingularly and excellently agreeing to him
then to any Angel, Saint, or creature liuing: though theſe alſo be
rightly and truely ſo called, and that not only without al derogation,
but much to the honour of Chriſt's aduocation. To him ſoly and only it
agreeth to procure vs mercie before God's face, by the general ranſom,
price, & paiment of his bloud for our deliuerie, as is ſaid in the
ſentence following, \Emph{And he is the propitiation for our ſinnes, and
not for ours only, but for the whole worlds.} In which ſort he is our
only Aduocate, becauſe he is our only Redeemer. And hereupon he alone
immediately, by and through himſelf, and without the aid or aſſiſtance
of any other, man or Angel, in his owne name, right, and merits,
confidently dealeth in our cauſes before God our iudge, & ſo procureth
our pardon, which is the higheſt degree of aduocation that can be.

Al
\MNote{How Angels, Saints, & men aliue are our Aduocates.}
which notwithſtanding, yet the Angels, and Saints, & our fellowes aliue,
may and doe pray for vs, and in that they deale with God by interceſſion
to procure mercie for vs, may iuſtly be called our Aduocates: not ſo as
Chriſt is, who demandeth al things immediately by his owne merits, but
as ſecondary Interceſſours, who neuer aske nor obtaine any thing for vs,
but \L{per Chriſtum Dominum noſtrum}, by and through Chriſt our common
Lord, Aduocate, and Redeemer of mankind. And behold how S.~Auguſtin
\Cite{(Tract.~1. in ep.~Io. vpon theſe very words)}
preuented the Heretikes cauillations.
\MNote{Saints in heauen pray for vs.}
\L{Sed dicit aliquis, &c.}
\Emph{But ſome man wil ſay, Doe not the Saints then pray for vs? doe not
Bishops then or Prelates and Paſtours pray for the people? Yes}, ſaith
he: \Emph{Marke the Scriptures, and you shal find that the Apoſtles
praied for the people, & againe deſired the people to pray for them, and
ſo the head praieth for al, and the members one for another.} And
likewiſe (leſt the Heretikes should ſay, there is a difference betwixt
the liuing and the dead in this caſe) thus the ſame holy Father writeth
\Cite{vpon the 85.~Pſalme in fine.}
\Emph{Our Lord Ieſus Chriſt doth yet make interceſsion for vs, al the
Martyrs that be with him, pray for vs: neither wil their interceſsion
ceaſe, til we ceaſe our groanings.}

In this ſenſe therfore whoſoeuer praieth for vs, either aliue or dead,
is our Aduocate: as S.~Auguſtin
\Cite{(ep.~59. to Paulinus circa med.)}
calleth Bishops, the peoples Aduocates, when they giue them their
benediction or bleſſing.
\MNote{The B.~virgin is our Aduocate.}
So doth the holy Church cal our B.~Lady our Aduocate, by the very words
of S.~Irenæus, that you may ſee ſuch ſpeaches be no new inuentions of
the later Ages, but Apoſtolical. 
\CNote{\Cite{Iren. li.~3. c.~31.}
&
\Cite{li.~5. poſt med.}}
\Emph{The obedient Virgin} \Sc{Marie} (ſaith he) \Emph{is made the
Aduocate of the diſobedient virgin Eue.} And to confound the Proteſtants
plainely, in that they thinke or pretend that the aduocation or
patronage of Saints should be iniurious to Chriſt, remember that
\CNote{\Cite{D.~Hiero. in Mat. c.~18.}}
\MNote{Angels are our Protectours.}
our Sauiour acknowledgeth Angels to be deputed for the protection (which
is nothing els but aduocation) of infants before the face of God,
beſides the plaine examples in the old Teſtament.
\XRef{Gen.~48. v.~16.}
\XRef{Tob.~5,~27.}
&
\XRef{c.~12. v.~12.}
\XRef{Dan.~10.}
And this not only the Catholike Church, but the very English Proteſtants
themſelues in their ſeruice booke and in the Collect of Michel-mas day,
profeſſe, and pray for the ſame protection or aduocation of Angels, and
defend the ſame againſt their yonger brethren the Puritanes.}
an
\TNote{\G{παράκλητον}}
Aduocate with the Father, \Sc{Iesvs} Chriſt the iuſt: \V and he is the
propitiation for our ſinnes: and not for ours only, but alſo
\LNote{For the whole worlds.}{S.~Auguſtin
\MNote{The Catholike Church is the only true Church.}
gathereth hereof againſt the Donatiſts, and al other Heretikes, that
would driue the Church into corners or ſome certaine countries, from the
vniuerſalitie of al Nations (whereof it was named by the Apoſtles,
Catholike) that the true religion, and Church, and conſequently the
effects of Chriſts propitiation, death, and aduocation, pertaineth not
to one Age, nation, or people, but to the whole world. S.~Auguſtin
\Cite{vpon this place}
\Cite{to.~9. tract.~1. in ep.~Io.}}
for the whole worldes. \V And in this we know we haue knowen him, if we
obſerue his commandements. \V
\LNote{He that ſaith he knoweth.}{To
\MNote{Not only faith.}
know God here, ſignifieth (as it doth often in the Scriptures) to loue,
that is, as in the laſt chapter, to be in ſocietie with him, and to haue
familiar and experimental knowledge of his graces. If any vant himſelfe
thus to know God, and yet keepeth not his commandements, he is a lier,
as al Caluiniſtes and Lutherans, that profeſſe themſelues to be in the
fauour of God by only faith: affirming, that they neither keep, nor
poſſibly can keep his commandements.}
He that ſaith he knoweth him, and keepeth not his commandements, is a
lier, and the truth is not in him: \V But he that keepeth his word, in
him in very deed the charitie of God is perfited: in this we know that
we be in him. \V He that ſaith he abideth in him, ought euen as he
walked, himſelf alſo to walke.

\V My Deareſt, I write not a new commandement to you, but an old
commandement which you had from the beginning. The old commandement is
the word which you haue heard. \V Againe
\CNote{\XRef{Io.~13,~34.}
\XRef{15,~12.}}
a new commandement write I to you, which thing is true
\Fix{hoth}{both}{obvious typo, fixed in other}
in him and in you: becauſe the darkeneſſe is paſſed, and the true light
now ſhineth. \V He that ſaith he is in the light, and hateth his
brother, is in the darkeneſſe euen vntil now. \V
\CNote{\XRef{1.~Io.~3,~14.}}
He that loueth his brother, abideth in the light, and ſcandal is not in
him. \V But he that hateth his brother, is in the darkeneſſe, and
walketh in the darkeneſſe, and knoweth not whither he goeth, becauſe the
darkenes hath blinded his eyes.

\V I write vnto you litle children, becauſe your ſinnes are forgiuen you
for his name. \V I write vnto you fathers, becauſe you haue knowen him
which is from the beginning. I write vnto you yong men, becauſe you haue
ouercome the wicked one. \V I write to you infants, becauſe you haue
knowen the Father. I write vnto you yong men, becauſe you are ſtrong,
and the word of God abideth in you, and you haue ouercome the wicked
one. \V Loue not the world, nor thoſe things which are in the world. If
any man loue the world, the charitie of the Father is not in him. \V
Becauſe
\SNote{How al ſinne & tentation proceed of theſe three, ſee S.~Thomas in
\Cite{Summe.~1.~2. quæſt.~77. art.~5.}}
al that is in the world, is the concupiſcence of the flesh, and the
concupiſcence of the eyes, and the pride of life, which is not of the
Father, but is of the world. \V And the world paſſeth,
%%% o-2778
and the
concupiſcence thereof. But he that doeth the wil of God, abideth for
euer.

%%% 2916
\V Litle children, it is the laſt houre, & as you haue heard, that
%%% !!! should have a final 'τος' ligature
\TNote{\G{ὁ ἀντίχριστος}}
Antichriſt commeth: now there are become
\LNote{Many Antichriſts.}{\Emph{The
\MNote{Al Heretikes are Antichriſts, the fore-runners of the great
Antichriſt.}
holy Apoſtle S.~Iohn} (ſaith S.~Cyprian) \Emph{did not put a difference
betwixt one hereſie or ſchiſme and another, nor meant any ſort that
ſpecially ſeparated themſelues, but generally called al without
exception}, Antichriſtes, \Emph{that were aduerſaries to the Church, or
were gone out from the ſame.} And a litle after, \Emph{It is euident
that al be here called Antichriſtes, that haue ſeuered themſelues from
the charitie and vnitie of the Catholike Church.} So writeth he
\Cite{ep.~76. nu.~1. ad Magnum}
Whereby we may learne, that al Heretikes, or rather Arch-heretikes be
properly the precurſours of that one and ſpecial Antichriſt, which is to
come at the laſt end of the world, & which is called here immediately
before, \G{ὁ ἀντίχριστος}, \Emph{that peculiar and ſingular Antichriſt}.}
many Antichriſts, whereby we know, that it is the laſt houre. \V
\LNote{They went out from vs.}{An
\MNote{The marke of al heretikes is, their going out of the Catholike
ſocietie.}
euident note and marke, whereby to conuince al Heretikes and falſe
Teachers, to wit, that being once of the common Catholike Chriſtian
fellowship, they forſooke it, and went out from the ſame. Simon Magus,
Nicolas the Deacon, Hymenæus, Alexander, Philetius, Arius, Macedonius,
Pelagius, Neſtorius, Eutyches, Luther, Caluin, and the like, were of the
common ſocietie of al vs that be Chriſtian Catholikes, they went out
from vs whom they ſaw to liue in vnitie of faith & religion together, &
made themſelues new Couenticles, therfore they were (as the Apoſtle here
sheweth) Antichriſtes, and we and al that abide in the ancient
fellowship of Chriſtian religion, that went not out of their fellowship,
in which we neuer were, nor out of any other ſocietie of knowen
Chriſtians, can not be Schiſmatikes or Heretikes, but muſt needs be true
Chriſtian Catholike men.
\MNote{The Catholikes can not be proued to haue gone out.}
Let our Aduerſaries tel vs, out of what Church
we euer departed, when, and where, and vnder what perſons it was that we
reuolted, as we can tel them the yeare, the places, the Ringleaders of
their reuolt.}
They went out from vs; but
\LNote{They were not of vs.}{He
\MNote{How Heretikes are of the Church, before they fal.}
meaneth not, that Heretikes were not, or could not be in or of the
Church, before they went out or fel into their hereſie or ſchiſme: but
partly that many of them which afterward fal out, though they were
before with the reſt, and partakers of al the Sacraments with other
their fellowes, yet indeed were of naughtie life & conſcience when they
were within, and ſo being rather as il humours and ſuperfluous
excrements, then true and liuely parts of the body, after a ſort may be
ſaid not to haue been of the body at al. So S.~Auguſtin expoundeth
theſe words in his
\Cite{commentarie vpon this place, tract.~3.}
but els-where, more agreably as it ſeemeth, that the Apoſtle meaneth,
that ſuch as wil not tarie in the Church, but finally forſake it to the
end, in the preſcience of God, and in reſpect of the ſmal benefit they
shal haue by their temporal ſmal abode there, be not of or in the
Church, though according to this preſent ſtate, they are truely members
thereof. 
\Cite{Li. de corrept. & gr. c.~9.}
&
\Cite{de dono perſeuer. c.~8.}}
they
\SNote{They were of vs for the time, that is, of and in the Church:
otherwiſe they could not haue gone out. But they were not of the
conſtant ſort, or of the elect & predeſtinat: for then they had taried
within, or returned before their death.}
were not of vs. For if they had been of vs, they would ſurely haue
remained with vs: but
\LNote{That they may be manifeſt.}{God
\MNote{By hereſies conſtant Catholikes are knowen.}
permitteth hereſie to be, that ſuch as be permanent, conſtant, and
choſen members and children of the Catholike Church, only knowen to God
before, may now alſo be made manifeſt to the world, by their conſtant
remaining in the \Sc{Chvrch}, when the wind and blaſt of euery hereſie
or tentation driueth out the other light & vnſtable perſons.}
that they may be manifeſt that they are not al of vs. \V But you haue
\TNote{\G{χρῖσμα} \GG{Chriſma}, whereof \Emph{Chriſt}
& \Emph{Chriſtians.}}
the vnction from the Holy one, and
\LNote{Know al things.}{They
\MNote{Euery good Catholike is ſufficiently taught by the Church to
ſaluation.}
that abide in the vnitie of Chriſtes Church, haue \Emph{the vnction},
that is, the Holy Ghoſt, who teacheth al truth. Not that euery member or
man thereof hath al knowledge in himſelf perſonally, but that euery one
which is of that happie ſocietie to which Chriſt promiſed and gaue the
Holy Ghoſt, is partaker of al other mens guifts and graces in the ſame
Holy Spirit, to his ſaluation. Neither need any to ſeeke truth at
Heretikes hands or others that be gone out, when it is within
themſelues, and only within themſelues in God's Church. \Emph{If thou
loue vnitie} (ſaith S.~Auguſtin) \Emph{for thee alſo hath he, whoſoeuer
hath any thing in it. Take away enuie, it is thine which I haue, it is
mine which thou haſt, &c.}
\Cite{Tract.~32. in Euang. Ioan.}}
know al things. \V I haue not written to you as to them that know not
the truth, but as to them that know it: and that no lie is of the
truth. \V Who is a lier, but he which denieth that \Sc{Iesvs} is Chriſt?
This is Antichriſt which denieth the Father and the Sonne. \V Euery one
that denieth the Sonne, neither hath he the Father. He that confeſſeth
the Sonne, hath the Father alſo. \V You, that which you haue
\SNote{Keep that firmely & conſtantly which you haue heard euen from the
beginning, by the mouth of the Apoſtles; & not that only which you haue
receiued by writing.}
heard from the beginning, let it abide in you. If that abide in you
which you haue heard from the beginning, you alſo ſhal abide in the
Sonne & the Father. \V And this is the promiſe which he promiſed vs,
life euerlaſting.

\V Theſe things haue I written to you concerning them that ſeduce
you. \V And you, the vnction which you haue receiued from him, let it
abide in you. And you haue no need that any man teach you: but as his
vnction teacheth you of al things, and it is true, and it is no lie. And
as it hath taught you, abide
\Var{in him.}{in it.}
\V And now litle children abide in him: that when he ſhal appeare, we
may haue confidence, and not be confounded of him in his comming. \V If
you know that he is iuſt, know ye that euery one alſo
\SNote{We ſee it is Apoſtolical doctrine, that men may doe or worke
iuſtice, and that ſo doing they be iuſt by their workes proceeding of
God's grace, & not by faith or imputation only.}
which doeth iuſtice, is borne of him.


\stopChapter


\stopcomponent


%%% Local Variables:
%%% mode: TeX
%%% eval: (long-s-mode)
%%% eval: (set-input-method "TeX")
%%% fill-column: 72
%%% eval: (auto-fill-mode)
%%% coding: utf-8-unix
%%% End:

