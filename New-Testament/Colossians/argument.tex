%%%%%%%%%%%%%%%%%%%%%%%%%%%%%%%%%%%%%%%%%%%%%%%%%%%%%%%%%%%%%%%%%
%%%%
%%%% The (original) Douay Rheims Bible 
%%%%
%%%% New Testament
%%%% Colossians
%%%% Argument
%%%%
%%%%%%%%%%%%%%%%%%%%%%%%%%%%%%%%%%%%%%%%%%%%%%%%%%%%%%%%%%%%%%%%%




\startcomponent argument


\project douay-rheims


%%% 2776
%%% o-2635
\startArgument[
  title={\Sc{The Argvment of the Epistle of S.~Pavl to the Colossians.}},
  marking={Argument of Colossians}
  ]

The Epiſtle to the Coloſsians is not only in ſenſe, but almoſt in words
alſo, al one with the Epiſtle to the Epheſians, and was ſent alſo by the
ſame meſſenger Tychicus:
\XRef{c.~4. v.~7.}
And in it he maketh like mention of his bands and ſufferings.
\XRef{c.~1. v.~24.}
and
%%% !!! Two XRefs?
\XRef{c.~4. v.~3,~18.}
And therfore no doubt it was written at Rome at the ſame time, to wit,
in his laſt apprehenſion, yet before he knew of his martyrdom.

This difference there is, that he had himſelf preached to the Epheſians,
but with the Coloſsians he had neuer been, as he ſignifieth,
\XRef{c.~2. v.~1.}
Therfore although in matters of exhortation he be here briefer then to
the Epheſians, yet in matters of doctrine he is longer. And generally he
aſſureth them that to be the truth, which their Apoſtle Epaphras had
taught them, but namely he giueth them warning both of the Iudaical
Falſe-apoſtles, who ſought to corrupt thẽ with ſome ceremonies of Moyſes
law; & alſo of the Platonike Philoſophers, who reiected Chriſt (who is
indeed the Head of the Church and the Mediatour to bring vs to God) and
inſtead of him, brought in certaine Angels as more excellent then he, whom
they termed, \L{Minores Dij}, teaching the people to ſacrifice vnto them
(calling that, humilitie) that they might bring them to the great
God. With which falſehood the hereſie of Simon Magus a long time
deceiued many, as we read in
\Cite{Epiphan. Hæreſ.~21.}

Againſt ſuch therfore S.~Paul telleth the Coloſsians, that Chriſt is the
Creatour of al the Angels, God in perſon, the Head of the Church, the
principal in al reſpects: that he is the Redeemer, Mediatour, and
pacifier between God and men, and therfore by him we muſt goe to God, ſo
that whether we pray our ſelues, or deſire any other in earth or in
Heauen to pray for vs, al muſt be done (as the Cath. Church in euery
Collect doth) \L{Per Chriſtum Dominum noſtrum}, that is, \Emph{through
Chriſt our Lord}, or \L{per Do. noſtrum Ieſum Chriſtum filium tuum, qui
tecum viuit & regnat, &c.} Whereby the Church profeſſeth continually
againſt ſuch ſeduction, both the Mediatourship, and the Godhead of
Chriſt.


\stopArgument


\stopcomponent


%%% Local Variables:
%%% mode: TeX
%%% eval: (long-s-mode)
%%% eval: (set-input-method "TeX")
%%% fill-column: 72
%%% eval: (auto-fill-mode)
%%% coding: utf-8-unix
%%% End:
