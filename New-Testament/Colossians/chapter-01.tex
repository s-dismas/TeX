%%%%%%%%%%%%%%%%%%%%%%%%%%%%%%%%%%%%%%%%%%%%%%%%%%%%%%%%%%%%%%%%%
%%%%
%%%% The (original) Douay Rheims Bible 
%%%%
%%%% New Testament
%%%% Epistles
%%%% Colossians
%%%% Chapter 01
%%%%
%%%%%%%%%%%%%%%%%%%%%%%%%%%%%%%%%%%%%%%%%%%%%%%%%%%%%%%%%%%%%%%%%




\startcomponent chapter-01


\project douay-rheims


%%% 2777
%%% o-2636
\startChapter[
  title={Chapter 1}
  ]

\Summary{Saying, that he thanketh God for their excellent faith and
  charitie, and continually praieth for their encreaſe, he doeth withal
  giue witnes to the preaching of their Apoſtle Epaphras, and extolleth
  the grace of God in bringing them to Chriſt, who is cheefe aboue al &
  peace-maker by his bloud. This is the Ghoſpel not of Epaphras alone,
  but of the vniuerſal Church, and of Paul himſelf who alſo ſuffereth
  for it.}

Paul an Apoſtle of \Sc{Iesvs} Chriſt by the wil of God, and Brother
Timothee: \V to them that are at Coloſſa Saints and faithful Brethren in
Chriſt \Sc{Iesvs}.

\V Grace to you and peace from God our Father and our Lord \Sc{Iesvs}
Chriſt.

We giue thankes to God and the Father of our Lord \Sc{Iesvs} Chriſt
alwaies for you, praying: \V hearing your faith in Chriſt \Sc{Iesvs},
and the loue which you haue toward al the Saints, \V for the hope that
is laid vp for you in Heauen, which you haue heard in the word of the
truth of the Ghoſpel, \V that is come to you, as alſo
\SNote{He sheweth that the Church and Chriſtes Ghoſpel should daily grow
and be ſpred at length through the whole world. Which can not ſtand with
the heretikes opinion of the decay therof ſo quickely after Chriſtes
time, nor agree by any meanes to their obſcure Conuenticles. See
S.~Auguſtin
\Cite{ep.~80. in fine.}}
in the whole world it is, and fruictifieth, and groweth, euen as in you
ſince that day that you heard and knew the grace of God in truth, \V as
you learned of Epaphras our deareſt fellow-ſeruant, who is a faithful
Miniſter of \Sc{Iesvs} Chriſt for you, \V who alſo hath manifeſted to vs
your loue in ſpirit. \V Therfore we alſo from the day that we heard it,
ceaſe not praying for you and deſiring, that you may be filled with the
knowledge of his wil, in al wiſedom, and ſpiritual vnderſtanding: \V
that you may walke
\SNote{So S.~Ambr. & the Gr. Doctours, or thus \Emph{worthily pleaſing
God, &c.}}
\TNote{\G{ἀξίως τοῦ κυρίου}}
worthie of God, in al things pleaſing: Fructifying
%%% o-2637
in
\SNote{Many things requiſit, and diuers things acceptable to God beſide
faith.}
al good worke, & increaſing in the knowledge of God: \V in al power
ſtrengthned according to the might of his glorie, in al patience and
longanimitie with ioy \V giuing thankes to God and the Father, who hath
made vs
\SNote{We are not only by acceptation or imputation partakers of
Chriſtes benefits, but are by his grace made worthie therof & deſerue
our ſaluation condignely.}
worthy vnto the part of the lot of the Saints in the light: \V Who hath
deliuered vs from the power of darkenes, and hath tranſlated vs into the
Kingdom of the Sonne of his loue, \V in whom we haue
%%% 2778
redemption, the remiſſion of ſinnes: \V who is the
\CNote{\XRef{Heb.~1,~3.}}
Image of the inuiſible God, the firſt-borne of al creature: \V becauſe
\CNote{\XRef{Ioa.~1,~3.}}
in him were created al things in Heauen, and in earth, viſible, and
inuiſible, whether Thrones or
\Fix{Dominiations,}{Dominations,}{obvious typo, fixed in other}
or Principalities, or Poteſtates: al by him & in him were created: \V
and he is before al, and al conſiſt in him. \V And he is the Head of the
body, the \Sc{Chvrch}, who is the beginning, Firſt-borne of the dead:
that he may be in al things holding the primacie: \V becauſe in him it
hath wel pleaſed, al fulnes to inhabit: \V and by him to reconcile al
things vnto himſelf, pacifying by the bloud of his croſſe, whether the
things in earth, or the things that are in Heauen. \V And you, wheras
you were ſometime alienated and enemies in ſenſe, in euil workes: \V yet
now he hath reconciled in the body of his fleſh by death, to preſent you
holy & immaculate, and blameles before him: \V if yet ye continue in the
faith, grounded and ſtable, and vnmoueable from the hope of the Ghoſpel
which you haue heard, which is preached among al creatures that are
vnder Heauen, wherof I Paul am made a Miniſter. \V Who now reioyce in
ſuffering for you, and
\LNote{Doe accomplish that wanteth.}{As
\MNote{There is no want in Chriſtes paſſions, which he ſuffred in himſelf
as Head: but there is want in thoſe paſſiõs of Chriſt which he daily
ſuffereth in his body the Church & the members
\Fix{therfore.}{therof.}{likely typo, fixed in other}}
Chriſt the Head and his body make one perſon myſtical & one ful Chriſt,
the Church being therfore his plenitude, fulnes, or complement
\XRef{Epheſ.~1.}
ſo the paſſions of the Head and the afflictions of the body & members
make on complete maſſe of paſſions. With ſuch difference for al that,
between the one ſort and the other, as the preeminence of the Head (and
ſpecially ſuch a Head) aboue the body, requireth and giueth. And not
only thoſe paſſions which he ſuffered in himſelf, which were fully ended
in his death, & were in themſelues fully ſufficient for the redemption
of the world & remiſſion of al ſinnes, but al thoſe which his body and
members ſuffer, are his alſo, and of him they receiue the condition,
qualitie, and force to be meritorious and ſatisfactorie. For though
there be no inſufficiencie in the actions or paſſions of Chriſt the
Head, yet his wiſedom, wil, and iuſtice requireth and ordaineth,
\CNote{\XRef{Ro.~8,~17.}
\Cite{Leo ſer.~19. de paſsione.}}
that his body and members should be fellowes of his paſſions, as they
looke to be fellowes of his glorie: that ſo ſuffering with him & by his
exãple, they may applie to thẽſelues and others the general medicine of
Chriſtes merits and ſatisfactiõs, as it is effectually alſo applied to vs
by Sacramẽts, Sacrifice, and other waies alſo: the one ſort being no
more iniurious to Chriſtes death then the other, notwithſtanding the
vaine clamours of the Proteſtants, that would vnder pretence of Chriſtes
paſſion take away the valure of al good deeds.
\MNote{How Chriſt's merits are applied to vs, without any iniurie to his
death.}
Hereupon it is plaine now, that this accomplishment of the wants of
Chriſtes Paſſions, which the Apoſtle and other Saints make vp in their
flesh, is not meant but of the penal & ſatisfactorie workes of Chriſt in
his members, euery good man adding continually (and ſpecially Martyrs)
ſom-what to accomplish the ful meaſure therof: and theſe be the
plenitude of his paſſions and ſatisfactions, as the Church is the
plenitude of his perſon: and therfore theſe alſo through the communion
of Saints & the ſocietie that is not only between the Head & the body,
but alſo between one member & another are not only ſatisfactorie and many
waies profitable for the ſufferers themſelues, but alſo for other their
fellow-members in Chriſt.
\MNote{The workes of one may ſatisfie for another.}
For though one member can not merit for another properly, yet may one
beare the burden and diſcharge the debt of another, both by the Law of
God and nature. And it was a ridiculous Hereſie of Wicleffe to deny the
ſame. Yea (as we ſee here) the paſſions of Saints are alwaies ſuffered
for the common good of the whole body, and ſometimes withal by the
ſufferers ſpecial intention they are applicable to ſpecial perſons one
or many: as here the Apoſtle ioyeth in his paſſions for the Coloſſians,
\CNote{\XRef{2.~Cor.~1,~6.}}
in another place his afflictions be for the ſaluation of the
Corinthians,
\CNote{\XRef{Ro.~9,~3.}}
ſometimes he wisheth to be \Emph{Anathema}, that is
according to Origens expoſition
\Cite{(in li. Nu. ho.~10. &~24.)}
a Sacrifice for the Iewes, and
\CNote{\XRef{Phil.~2.}}
he often ſpeaketh of his death as of a libation, hoſt, or offering, as
the Fathers doe of al Martyrs paſſions. Al which dedicated & ſanctified
in Chriſtes bloud & Sacrifice, make the plenitude of his Paſſion, and
haue a forcible crie, interceſſion, & ſatisfaction for the Church & the
particular neceſſities therof. In which, as ſome doe abound in good
workes & ſatisfactions (as S.~Paul, who
\CNote{\XRef{2.~Tim.~4.}}
rekneth vp his afflictions and glorieth in them
\XRef{2.~Cor.~11.}
and
\SNote{Iob.~6.}
\XRef{Iob.}
who auoucheth that his penalties farre ſurmounted his ſinnes; and our
Ladie much more, who neuer ſinned, and yet ſuffered ſo great dolours) ſo
other-ſome doe want, and are to be holpen by the aboundance of their
fellow-members.

Which
\MNote{The ground of Indulgences or pardons.}
entercourſe of ſpiritual offices and the recompenſe of the wants of one
part by the ſtore of the other, is the ground of the old libels of
Indulgence, wherof is treated before out of S.~Cyprian (See the
\XRef{Annotations 2.~Cor.~2. v.~10.)}
and of
\Fix{a}{al}{obvious typo, fixed in other}
indulgences or pardons, which the Church daily diſpenſeth with
great iuſtice and mercie, by their hands in whõ Chriſt hath put the word
of our recõcilement, to whom he hath committed the keies to keep and
looſe, his commiſſion to remit and reteine, and the ſtewardship of his
familie to geue euery one their meat and ſuſtenance in due ſeaſon.}
doe accompliſh thoſe things that want of the paſſions of Chriſt, in my
fleſh for his body which is the \Sc{Chvrch}: \V wherof I am made a
Miniſter according to the diſpenſation of God, which is giuen me toward
you, that I may fulfil the word of God, \V the myſterie that hath been
hidden from worlds and Generations, but now is manifeſted to his
Saints, \V to whom God would make knowen the riches of the glorie of
this Sacrament in the Gentiles, which is Chriſt, in you the hope of
glorie, \V whom we preach, admoniſhing euery man, and teaching euery man
in al wiſedõ, that we may preſent euery man perfect in
Chriſt \Sc{Iesvs}. \V Wherin alſo I labour ſtriuing according to his
operation which he worketh in me in power.


\stopChapter


\stopcomponent


%%% Local Variables:
%%% mode: TeX
%%% eval: (long-s-mode)
%%% eval: (set-input-method "TeX")
%%% fill-column: 72
%%% eval: (auto-fill-mode)
%%% coding: utf-8-unix
%%% End:

