%%%%%%%%%%%%%%%%%%%%%%%%%%%%%%%%%%%%%%%%%%%%%%%%%%%%%%%%%%%%%%%%%
%%%%
%%%% The (original) Douay Rheims Bible 
%%%%
%%%% New Testament
%%%% Colossians
%%%% Chapter 02
%%%%
%%%%%%%%%%%%%%%%%%%%%%%%%%%%%%%%%%%%%%%%%%%%%%%%%%%%%%%%%%%%%%%%%




\startcomponent chapter-02


\project douay-rheims


%%% 2779
%%% o-2638
\startChapter[
  title={Chapter 2}
  ]

\Summary{He is careful for them though he were neuer with them: that
  they reſt in the wonderful wiſedom which is in Chriſtian religion, and
  be not caried away either with Philoſophie, to leaue Chriſt and to
  ſacrifice to Angels; or with Iudaiſme, to receiue any ceremonies of
  Moyſes law.}

%%% o-2639
For I wil haue you know, Brethren, what manner of care I haue for you and
for them that are at Laodicia, and whoſoeuer haue not ſeen my face in
the fleſh: \V that their harts may be comforted, inſtructed in charitie,
and vnto al the riches of the fulnes of vnderſtanding, vnto the
knowledge of the myſterie of God the Father of Chriſt \Sc{Iesvs}, \V in
whom be al the treaſures of wiſedom and knowledge hid. \V But this I ſay
\SNote{Heretikes doe moſt commonly deceiue the people with eloquẽce
namely ſuch as haue it by the guift of nature, as the Heretikes of al
Ages had, & lightly al ſeditious perſõs, which draw the vulgar ſort to
ſedition by allurement of their tongue. Nothing (ſaith S.~Hierom,
\Cite{ep.~2. ad Nepotian.)}
is ſo eaſie as with volubilitie of tongue to deceiue the vnlearned
multitude, which whatſoeuer it vnderſtandeth not, doth the more admire &
wonder at the ſame. The Apoſtle here calleth it, \G{πιθανολογίᾳ},
perſuaſible ſpeach.}
that no man deceiue you in loftines of words. \V For although I be
abſent in body, yet in Spirit I am with you; reioycing, and ſeeing your
order, and the conſtancie of that your faith which is in Chriſt. \V
Therfore as you haue
%%% 2780
receiued \Sc{Iesvs} Chriſt our Lord, walke in him, \V rooted and built
in him and confirmed in the faith, as alſo you haue learned, abounding
\Var{in him}{in it}
in thanks-giuing.

\V Beware leſt any man deceiue you
\LNote{By Phyloſophie.}{Philiſophie
\MNote{Philoſophie & other humane ſciences how profitable or hurtful to
the Church of God.}
and al humane ſcience, ſo long as they be ſubiect and obedient to Chriſt
(as they be in the Schooles of Chriſtian Catholike men) be nor
forbidden, but are greatly commended and be very profitable in the
Church of God. Otherwiſe where ſecular learning is made the rule of
religion and commandeth faith, there it is pernicious & the cauſe of al
hereſie & infidelitie. For the which, S.~Hierom & before him Tertul. cal
Philoſophers, \Emph{the Patriarkes of Heretikes}, & declare that al the
old hereſies roſe only by too much admiring of prophane Philoſophie,
\Cite{Hier. ad Cteſiph. cont. Pelaf. c.~1.}
\Cite{Tertul. de præſ. & cõt Marcio. li.~5.}
And ſo doe theſe new Sects no doubt in many things.
\MNote{The Proteſtãts abuſe Philoſophie againſt the B.~Sacrament.}
For, other argumẽts haue they none againſt the preſence of Chriſt in the
B.~Sacramẽt but ſuch as they borow of Ariſtotle & his like, cõcerning
quantitie, accidents, place, poſition, dimenſions, ſenſes, ſight, taſt,
and other ſtraits of reaſon, to which they bring Chriſtes myſteries. Al
Philoſophical arguments therfore againſt any article of our faith be
here condemned as deceitful, and are called alſo here, \Emph{the
tradition of men, and the elements of the world}.
\MNote{Schoole learning.}
The better to reſiſt
which fallacies and traditions of Heathen men, the Schoole learning is
neceſſarie, which keepeth Philoſophie in awe and order of faith, and
vſeth the ſame to withſtand the Philoſophical and ſophiſtical deceits of
the Heretikes and Heathen. So the great Philoſophers S.~Denys,
S.~Auguſtin, Clemens Alexandrinus, Iuſtine, Lactantius and the reſt,
vſed the ſame to the great honour of God and benefit of the Church. So
came S.~Cyprian, S.~Ambroſe, S.~Hierom, and the Greek Fathers, furnished
with al ſecular learning vnto the ſtudie of Diuinitie, wherof ſee
S.~Hierom, 
\Cite{ep.~84. ad Magnum Oratorem.}}
by Philoſophie, & vaine fallacie; according to the tradition of men,
according to the elements of the world, and not according to Chriſt. \V
For in him dwelleth al the fulneſſe of the Godhead corporally: \V and
you are in him repleniſhed, who is the Head in al principalitie and
power: \V in whom al you are circũciſed with circumciſion not made by
hand in ſpoiling of the body of the fleſh, in the circumciſiõ of Chriſt, \V
buried with him in Baptiſme: in whom alſo you are riſen againe by the
faith of the operation of God, who raiſed him vp from the dead. \V And
you
\CNote{\XRef{Eph.~2,~1.}}
when you were dead in the offenſes and the prepuce of your fleſh, did he
quicken together with him; pardoning you al offenſes, \V wyping out the
hand-writing
\Var{of decree}{by decree}
that was againſt vs, which was contrarie to vs. And the ſame he hath
taken out of the way, faſtning it to the croſſe: \V and ſpoiling the
Principalities & Poteſtates,
\TNote{\G{ἐδειγμάτισεν}.}
hath lead them confidently in open ſhew, triumphing them in himſelf. \V
Let no man therfore iudge you
\LNote{In meate.}{The
\MNote{Scriptures abuſed by the Proteſtants againſt Chriſtian faſting,
and holydaies.}
Proteſtants wilfully or ignorantly applie al theſe kinds of forbearing
meats, to the Chriſtian faſts: but it is by the circumſtãce of the text
plaine (as
\CNote{\Cite{Aug. ep.~59. ad Paulin. in ſolut.~7. quæſt.}}
S.~Auguſtin alſo teacheth) that the Iudaical obſeruation and diſtinction
of certaine cleane and vncleane meats is forbidden to the Coloſſians,
who were in danger to be ſeduced by certaine Iewes, vnder pretence
of holines to keep the Law touching meats & feſtiuities & other like,
which the Apoſtle sheweth were only shadowes of things to come: which
things are come, & therfore the ſaid shadowes to ceaſe. Where he nameth
the Sabboth & feaſts of the new moone, that no mã need to doubt but that
he ſpeaketh only of the Iewish daies & kinds of faſts and feaſts, and
not of Chriſtian holidaies or faſting daies at al.}
in meat or in drinke, or in part of a feſtiual day, or of the New-moon,
or of Sabboths: \V which are a ſhadow of things to come, but the body
Chriſts.

\V Let no man ſeduce you,
\SNote{That is, wilful or ſelfwilled in voluntarie religiõ. For that is,
\G{θέλων ἐν θρησκείᾳ} wherof commeth the word
following \G{ἐθελοθρησκίᾳ} \Emph{Supeſtition.} 
\XRef{v.~23.}
See
\XRef{Annot. v.~23.}}
willing in the humilitie and
\LNote{Religion of Angels.}{By
\MNote{S.~Paules place concerning religion of Angels, explicated and
that the Proteſtãts wickedly abuſe it againſt the due honour &
inuocation the Angels.}
the like falſe application of this text as of the other before, the
Heretikes abuſe it againſt the inuocation or honour of Angels vſed in
the Catholike Church, where the Apoſtle noteth the wicked doctrine of
Simon Magus & others (See S.~Chryſ.
\Cite{ho.~7. in hunc locum}
and
\Cite{Epiph. hær.~21.)}
who taught, Angels to be our Mediatours and not Chriſt, \L{non tenens
Caput}, \Emph{not holding the Head}, as the Apoſtle ſpeaketh, &
preſcribed Sacrifices to be offered vnto them, meaning indifferently as
wel the il Angels as the good. Which doctrine the ſaid Heretike had of
Plato, who taught, that ſpirits (which he calleth \L{dæmons}) were to be
honoured as Mediatours next to God. Againſt which S.~Auguſtin diſputeth
\Cite{li.~8,~9,~&~10. de ciuit.}
as he condemneth alſo the ſame vndue worship
\Cite{li.~10. confeſ. cap.~42.}
S.~Hierom
\Cite{(q.~10. ad algaſiam)}
expoundeth this alſo of al ſpirits or Diuels, whom he proueth (out of
S.~Steuen's ſermon
\XRef{Act.~7.)}
that the Iewes did worship, auouching that they ſerue them ſtil, ſo many
of them and ſo often as they obſerue the Law. Of which Idolatrie alſo to
Angels Theodoret ſpeaketh
\Cite{vpon this place,}
declaring, that the Iewes defended their ſuperſtition towards Angels by
that, that the Law was giuen by them, deceitfully at once inducing the
Coloſſians, both to keep the law, & to honouring of the Angels as the
giuers of the ſame. Wherby diuers of the faithful were ſo ſeduced, that
they forſooke Chriſt and his Church and ſeruice, and committed idolatrie
to the ſaid Angels. Againſt which abominations the
\Cite{Councel of Laodicea Cap.~35.}
tooke order, accurſing al that forſooke our Sauiour and cõmitted
idolatrie to Angels, & contemning Chriſt, kept conuenticles in the name
of ſpirits and Idols. Of which kind of worship of Angels and Diuels ſee
Clemens Alexand.
\Cite{Strom.~3.}
Tertullian
\Cite{(li.~5. cont. Marc.)}
expoundeth this place of the falſe Teachers that feined themſelues to
haue reuelation of Angels, that the Law should be kept touching
difference of cleane and vncleane meats. Which is very agreable to that
\CNote{\XRef{1.~Tim.~4,~1.}}
in the Epiſtle to Timothee, where S.~Paul calleth abſtaining from meats
after the Iewish or heretical manner, \Emph{the doctrine of Diuels}:
wherof ſee more in the
\XRef{annotation vpon that place.}
Haimo a godly ancient writer,
\Cite{vpon this place,}
ſaith further, that ſome Philoſophers of the Gentils and ſome of the
Iewes alſo taught, that there were foure Angels Preſidents of the foure
elements of man's body, and that in feined hypocriſie (which the Apoſtle
here calleth humilitie) they pretended to worship by Sacrifice the ſaid
Angels. Theophylact expoundeth this feined humilitie, of certaine
Heretikes, that pretẽding the mediatourship to be a derogation to
Chriſt's maieſtie, worshipped Angels as the only Mediatours. Al which we
ſet downe with more diligence, that the Heretikes may be ashamed to
abuſe this place againſt the due reuerence & reſpect or praiers made to
the holy Angels. Whom the Scriptures record ſo often to offer our
praiers vp to God, & to haue been lawfully reuerẽced of the Patriarkes,
neuer as Gods, but as God's Miniſters and meſſengers.
\XRef{Ioſ.~5,~14.}
\XRef{Tob.~12.}
\XRef{Gen.~48,~16.}
\L{Angelus qui eruit me},
\XRef{1.~Tim.~5,~21.}
And that they may be praied vnto, & can help & heare vs, See S.~Hierom
\Cite{in cap.~10. Danielis.}
S.~Ambroſe
\Cite{in Pſ.~118. ſerm.~1.}
S.~Auguſtin
\Cite{li.~10. de ciuit. Dei. c.~12.}
Bede
\Cite{li.~4. de Cantic. c.~24.}}
religion of Angels, walking in the things which he hath not ſeen, in
vaine puffed vp by the ſenſe of his flesh, \V and
%%% o-2640
\LNote{Not holding the Head.}{Becauſe he hath much adoe with ſuch falſe
Preachers as taught the people to preferre the Angels which gaue the
Law, or other whatſoeuer, before Chriſt, in this Epiſtle and to the
Epheſians, he often affirmeth Chriſt to be our Head, yea and to be
exalted aboue al creatures, Angels, Poteſtates, Principalities, or
whatſoeuer.}
not holding the Head, wherof the whole body by ioynts and bands being
\SNote{\G{ἐπιχορηγούμενον}, That is taking ſubminiſtration of ſpiritual
life & nourishmẽt by grace from Chriſt the head.}
ſerued and compacted, groweth to the increaſe of God. \V If then you be
dead with Chriſt, from the elemẽts of this world;
\LNote{Why doe you.}{A
\MNote{Heretical tranſlation.}
maruelous impudẽt tranſlation of theſe words in the English Bibles thus:
\Emph{Why are you burdned with traditions?} Wheras the
\TNote{\G{δογματίζεσθε}}
Greek hath not that ſignification: but to make the name of Tradition
odious here they put it of purpoſe, not being in the Greek & in other
places where Traditions are cõmended
\XRef{(1.~Cor.~11.}
&
\XRef{2.~Theſſ.~2.)}
& where the Greek is ſo moſt flatly (\G{παράδοσις}) there they tranſlate
it, \Emph{Inſtructions, Ordinances,} &c.}
why doe you yet
\TNote{\G{δογματίζεσθε}}
decree as liuing in the world? \V
%%% !!! Not marked in either
\LNote{Touch not.}{The
\MNote{Scriptures abuſed againſt the Churches faſts.}
Heretikes (as before and alwaies) very vainely alleage this againſt the
Catholike faſtings: when it is moſt cleer that the Apoſtle reprehendeth
the foreſaid falſe Teachers that thought to make the Chriſtians ſubiect
to the obſeruation of the ceremonies of the old Law, of not eating hogs,
conies, hares-flesh, and ſuch like, not to touch a dead corps nor any
place where a woman in her floures had ſittẽ, & other infinit doctrines
of touching, taſting, washing, eating, and the reſt, either commanded to
the old people by God, or (as many things were) voluntarily taken vp by
themſelues, ſometime cleane againſt God's ordinance, & often friuolous
and ſuperſtitious. Which ſort as Chriſt in the Ghoſpel, ſo here S.~Paul
calleth the precepts and doctrines of men, and ſuperſtitiõ, and (as the
Greek word ſignifieth)
\TNote{\G{ἐθελοθρησκίᾳ}}
voluntarie worship, that is inuented by Heretikes of their owne head
without the warrant of Chriſt in the Scriptures, or the Holy Ghoſt in
the Church, or any lawful authoritie of ſuch whom Chriſt commandeth vs
to obey. Againſt ſuch Sect-Maiſters therfore as would haue yoked the
faithful againe with the Iewish or Heretical faſts of Symon Magus and
the like, S.~Paul ſpeaketh, and not of the Churches faſts or doctrines.}
Touch not, taſt not, handle not: \V
which things are al vnto deſtruction by the very vſe, according to the
precepts and doctrines of men. \V Which are indeed
\LNote{Hauing a shew.}{Againe
\MNote{The hypocritical abſtinence of old Heretikes, maketh nothĩg
againſt true & ſincere faſting, but cõmendeth it.}
the Heretikes of our time obiect, that theſe foreſaid falſe Teachers
pretended holines, wiſedom, & chaſtiſement of their bodies (for ſo
S.~Paul ſaith) by forbidding certaine meats according to the Iewes
obſeruation, euen as the Catholikes doe: It is true they did ſo, and ſo
doe moſt vices imitate vertues. For if chaſtiſing of mens bodies &
repreſſing their cõcupiſcences & luſtes were not godly, and if
abſtinence frõ ſome meats were not laudably & profitably vſed in the
Church for the ſame purpoſe, no Heretikes (to induce the abolished
obſeruations & differences of meats of the Iewes, or the condemnation of
certaine meats & creatures as abominable, according to others) would
haue falſely pretended the chaſtiſement of their flesh, or made other
shew of wiſedom and pietie, to found their vnlawful Heretical or
Iudaical ſuperſtition concerning the ſame. The Catholike Church & her
children, by the example of Chriſt, S.~Iohn Baptiſt, the Apoſtles, and
other bleſſed men, doe that lawfully, godly, religiouſly, & ſincerely
indeed to the end a foreſaid, which theſe falſe Apoſtles only pretended
to doe. So
\CNote{\XRef{1.~Cor.~9.~27.}
\XRef{2.~Cor.~11,~27.}}
S.~Paul did chaſtiſe his body indeed, by watching, faſting, and many
other afflictions, and that was lawful, and was true wiſedom and pietie
indeed. The foreſaid Heretikes not ſo, but to induce the Coloſſians to
Iudaiſme & other abominable errours, did but pretend theſe things in
hypocriſie.}
hauing a ſhew of wiſedom in ſuperſtition and humilitie, and not to ſpare
the body, not in any honour to the filling of the fleſh.


\stopChapter


\stopcomponent


%%% Local Variables:
%%% mode: TeX
%%% eval: (long-s-mode)
%%% eval: (set-input-method "TeX")
%%% fill-column: 72
%%% eval: (auto-fill-mode)
%%% coding: utf-8-unix
%%% End:

