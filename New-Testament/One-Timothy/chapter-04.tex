%%%%%%%%%%%%%%%%%%%%%%%%%%%%%%%%%%%%%%%%%%%%%%%%%%%%%%%%%%%%%%%%%
%%%%
%%%% The (original) Douay Rheims Bible 
%%%%
%%%% New Testament
%%%% Epistles
%%%% One Timothy
%%%% Chapter 04
%%%%
%%%%%%%%%%%%%%%%%%%%%%%%%%%%%%%%%%%%%%%%%%%%%%%%%%%%%%%%%%%%%%%%%




\startcomponent chapter-04


\project douay-rheims


%%% 2812
%%% o-2673
\startChapter[
  title={Chapter 4}
  ]

\Summary{He prophecieth that certaine should depart from the Catholike
  faith, willing Timothee therfore to inculcate to the people thoſe
  articles of the ſaid faith. 7.~Item to exerciſe himſelf in ſpiritual
  exerciſe, 12.~to get authoritie by example of good life, 13.~to
  ſtudie, to teach, to increaſe in the grace giuen him by holy Orders.}

And
\CNote{\XRef{2.~Tim.~3.}
\XRef{3.~Pet.~3.}
\XRef{Iude v.~18.}}
the Spirit manifeſtly ſaith that in the laſt times certaine
\LNote{Shal depart.}{It
\MNote{Al Heretikes are \L{Apoſtataes} from the faith.}
is the proper deſcription of Heretikes, to forſake their former faith,
and to be Apoſtates, as the
\TNote{\G{ἀποστήσονταί}}
Greek word importeth; to giue eare to particular ſpirits of errour &
deception, rather then to the Spirit of Chriſt in his Church, to follow
in hypocriſie and ſhew of vertue the pernicious doctrine of Diuels, who
are the ſuggeſters and prompters of al Sects, and are lying ſpirits in
the mouths of al Heretikes and falſe Preachers: men that haue put their
conſcience to ſilence & made it ſenſles to the Holy Churches admonition:
the Apoſtle noting
\CNote{\XRef{cap.~1,~19.}}
once before alſo in this ſame Epiſtle, that Heretikes haue no
conſcience, which is the cauſe both of their fal and of their obduration
in hereſie.}
ſhal depart from the faith attending to ſpirits of errour, and doctrines
of diuels, \V ſpeaking lies in hypocriſie, and hauing their conſcience
ſeared, \V
\LNote{Forbidding to marrie.}{He
\MNote{The old Hereſies againſt matrimonie.}
ſpeaketh (ſaith S.~Chryſoſtom) of the Manichees, Encrarites, &
Marcioniſtes.
\Cite{ho.~12. in 1.~Tim.}
S.~Ambroſe
\Cite{vpon this place,}
addeth to theſe the Patritians alſo. S.~Ireaæus
\Cite{li.~1. c.~30.}
S.~Epiphanius
\Cite{hær.~45.~26.~61.~30.}
S.~Hierom
\Cite{1.~cont. Iouin. c.~1.}
&
\Cite{ep.~50. c.~1.}
&
\Cite{3.}
S.~Auguſtin
\Cite{hær.~24.~40.}
and generally al
\Fix{iniquitie}{antiquitie}{obvious typo, fixed in other}
affirme the ſame both of them, and alſo of the Heretikes called
Apoſtolici, Ebionitæ, and the like. Their hereſie about marriage was,
that to marrie or to vſe the act of matrimonie, is of Satan, as
S.~Irenæus witneſſeth
\Cite{li.~1. c.~22.}
and that the diſtinction of male and femal, & the creation of man and
woman for generation, came of an il God. They taught their hearers,
\CNote{Aug. hær.~46.}
ſaith S.~Auguſtin, that if they did vſe women, they ſhould in any wiſe
prouide, that they might not conceiue or beare children.  Clemens
Alexandrinus
\Cite{(li.~3. Strom. in principio)}
writeth that ſuch admit no marriage nor procreation of children, leſt
they ſhould bring into the world creatures to ſuffer miſerie and
mortalitie. And this is the damnable opinion concerning marriage, noted
here by the Apoſtle.

For
\MNote{The old Hereſies about abſtinence from meats.}
the ſecond point conſiſting in the prohibition of meats or vſe of
certaine creatures made to be eaten, the ſaid Heretikes or diuers of
them (for they were not al of one ſect touching theſe points) taught,
that men might not eate certaine ſorts of meats, ſpecially of beaſts and
liuing creatures, for that they were not made (ſay they) of the good
God, but of the euil. And wine they called the gal of the Prince of
darkenes, and not to be drunke at al, and the vine, whereof it came, to
be of the Diuels creation. And diuers other creatures they condemned as
things by nature and creation polluted and abominable. 
\Cite{Auguſt. hær. Manich.~46.}
&
\Cite{hær.~25. Tatian.}
&
\Cite{toto de libro nor. Manich. to.~1.}
Loe theſe were the Heretikes and their hereſies which S.~Paul here
prophecieth of, that forbid marriage and meats as you haue heard, for
which they and their followers were condemned in diuers Councels.

Is
\MNote{The Catholikes impudẽtly charged with the ſaid old hereſies.}
it not now an intolerable impudencie of the Proteſtants, who for a ſmal
ſimilitude of words in the eares of the ſimple, apply this text to the
faſts of the Church, & of the chaſtitie of Prieſts and Religious?
\MNote{Abſtinence from certaine meats is no cõdemnation of the meats.}
As though either by appointing or vſing ſome daies of abſtinence from
certaine meats, the Church or any Catholike man cõdemned the ſaid meats:
Vnles the Rechabites
\Cite{Hierom.~35.}
or the Nazarites
\XRef{Num.~6.}
or the Niniuites
\XRef{Ion.~3.}
or Moyſes
\XRef{Exod.~34.}
or Elias
\XRef{3.~Reg.~19.}
of holy Anna the widow
\XRef{Luc.~2.}
or Iohn Baptiſt
\XRef{Mat.~3.}
&
\XRef{9.}
or Chriſt himſelf
\XRef{Mt.~4.}
commending, vſing, & following a preſcript number of faſting-daies, or
God himſelf that in the very beginning, in Paradiſe, preſcribed
abſtinence from the fruit of one certaine tree, and after appointed ſo
many faſts in the Law, vnles he therfore, condemned his owne creatures,
& the reſt, thoſe creatures from which they abſtained.
\MNote{Diuers good cauſes of abſtinence.}
No, there be many good and lawful cauſes to forbid ſome, or to abſtaine
from ſome meats: as, for obedience, as in Paradiſe: for ſignification,
as the Iewes: for that they haue been offered to idols, as in the
Epiſtle to the Corinthians: for chaſtning the bodie and penance, for
health alſo: and only thoſe cauſes are vnlawful for which the Manichees
and other Heretikes abſtained.

Concerning
\MNote{Forbidding certaine perſons to marrie is no condemnation of
matrimonie.}
marriage likewiſe, they may as wel charge God or the Church for
forbidding the father to marrie the daughter, or the brother the ſiſter,
or other prohibitied perſons in the law: as wel might they charge Chriſt
and the Apoſtle for prohibiting the man to marrie, during his wiues
life: and appointing widowes that ſerue the Church to liue vnmarried,
and not admitting a married woman as wel as a widow, nor her that hath
had moe huſbands, as wel as her that hath been married but once; as they
charge the Church for not admitting married perſons to the altar, and
for forcing them and Religious perſons to keep their promiſe of
chaſtitie.
\MNote{Catholikes eſteem matrimonie more then the Proteſtants doe.}
No, the holy Church is ſo farre from condemning wedlock, that
she honoureth it much more then the Proteſtants, accounting it an holy
Sacrament, which they doe not, who onely vſe it to luſt as the Heathen
doe, and not to religion.

But it is an old deceitful practiſe of Heretikes to charge Catholike men
with old condemned hereſies. The Eutichians ſlandred the Councel of
Chalcedon and S.~Leo to be Neſtorians, & to make two perſons in Chriſt,
becauſe they ſaid there were two natures.
\Cite{Vigilius li.~5. cont. Eurychon.}
Arius charged Alexander his Biſhop of Sabellianiſme, for auouching the
vnitie of ſubſtance in Trinitie.
\Cite{Socrat. li.~1. c.~3.}
Iulianus accuſed S.~Auguſtine of the hereſie of Apollinaris.
\Cite{li.~5. cont. Iulian. c.~15.}
Other Pelagians chalenged him for condemning marriage.
\Cite{Retract. li.~2. c.~53.}
And that our Proteſtants bragge not too much of their goodly inuention,
Iouinian the old Heretike, their Maiſter in this point, accuſed
\CNote{\Cite{Aug. li.~2. c.~5. de nupt. & concupiſc.}}
the holy Doctours and Catholikes vpon this ſame place, to be Manichees,
and to condemne meats, and marriage, as both S.~Hierom and S.~Auguſtin
doe teſtifie.
\MNote{The Proteſtants anſwered long agoe by S.~Hierom and S.~Auguſtin.}
And they both anſwer to the Heretike, that the Church indeed &
Catholikes doe abſtaine from ſome for euer, & ſome for certaine daies, &
euery Chriſtian man lightly al the 40.~daies of Lent faſt: not for that
they thinke the meats vncleane, abominable, or of an il creation, as the
Manichees doe: but for puniſhment of their bodies and taming their
concupiſcences. 
\Cite{Hiero. li.~2. cont. Iouin. c.~11.}
\Cite{Aug. cont. Adimantum c.~14.}
\Cite{Li. de mor. Cath. Ec. Hiero. in c.~4. ad Galat.}
And as for marriage, the ſaid Doctours anſwer, that no Catholike man
condemneth it for vnlawful, as the old Heretikes did, but only
preferreth virginitie and continencie before it, as a ſtate in it-ſelf
more agreable to God & more meet for the Clergie. See S.~Auguſtin
againſt Fauſtus the
\Fix{Maniche.}{Manichee.}{obvious typo, fixed in other}
\Cite{li.~30. c.~5.~6.}
and
\Cite{hær.~25. in the name Apoſtolici}
S.~Hierom
\Cite{ep.~50. c.~2. &~3.} Al this the Catholikes continually tel the
Aduerſaries, and they can not but ſee it. Yet by accuſtomed audacitie
and impudencie they beare it out ſtil.}
forbidding to marrie, to abſtaine from meats which God created to
receaue with thankes-giuing for the faithful, and them that haue knowen
the truth. \V For
\SNote{We ſee plainely by theſe words ſuch abſtinence only to be
diſalowed as condemneth the creatures of God to be naught by nature and
creation.}
euery creature of God is good, and nothing to be reiected that is
receiued
\LNote{With thankes-giuing.}{By
\MNote{Bleſſing of the table or of meates, ſpecially by a Prieſt.}
the moſt ancient cuſtom of the faithful both before Chriſt and ſithence,
men vſe to bleſſe their table and meats, by the hand and word of a
Prieſt, if any be preſent, otherwiſe by ſuch as can conueniently doe
it. And in husbandmens houſes where they haue no other meanes, they
ſhould at leaſt bleſſe God's guifts and themſelues with a \L{Pater
noſter} or the ſigne of the Croſſe: not only to acknowledge from whom
they haue their continued ſuſtenance, but alſo to bleſſe their meate and
ſanctifie it. For the
\TNote{\G{μετὰ εὐχαριϛίας}}
Greek word vſed of S.~Paul, by Eccleſiaſtical vſe, when it concerneth
meats, ſignifieth not only thankes-giuing, but bleſſing or ſanctifying
the creatures to be receiued, as being al one with \G{ἐυλογία}, and in
Engliſh we cal it grace, not only that after meat, which is only thankes
to God but that before meat which is alwaies a benediction of the
creatures, as it is plaine in the preſcript and vſual formes of
grace.
\MNote{To bleſſe is a preeminence of the better perſon.}
For which cauſe a Prieſt ſhould euer doe it rather then a lay man or any
of inferiour order in the Clergie. In ſo much that S.~Hierom
\Cite{(ep.~85.)}
reprehendeth certaine Deacons whom he ſaw ſay grace or bleſſe the meat &
the companie, in the preſence of a Prieſt. Who alſo recordeth (in the
life of S.~Paul the holy Eremit) the great curteſie and humilitie of him and
S.~Antonie, yealding one to the other the preeminence of bleſſing their
poore dinner. For to bleſſe is a great thing, and a Prieſtly prerogatiue
as the
\CNote{\XRef{Heb.~7.}}
Apoſtle witneſſeth, declaring the preeminence of Melchiſedech in that
he bleſſed Abraham. Read the note following.}
with thankes-giuing. \V For it is
\LNote{Sanctified.}{Al
\MNote{No creature il by nature, yet one more ſanctified then another.}
creatures be of God's creation, none of the Diuel, or of any other cauſe
and beginning, as the Manichees blaſphemed: and therfore none are il,
abominable, or vncleane by creation, nature, and condition, but al good
and made for mans vſe, though al be not alike holy nor equally
ſanctified.
\MNote{Holy times and places, & euery thing deputed to the ſeruice of
God holy.}
God made ſeuen daies, but he ſanctified only one of them. He
made al places, but he ſanctified none but the Temple and ſuch like
deputed to his ſeruice, as the Arke, the altar, and the reſt which were
by ſacred vſe both holy themſelues, & gaue alſo holines & ſanctification
to things that touched them or were applied vnto them.
\CNote{\XRef{Mat.~22.}}
So our Sauiour ſaith, that the Temple ſanctified the gold, and the altar
the guift; and generally al creatures ſeuered from common and profane
vſe, to religion & worſhip of God, are made ſacred thereby. So the
places and daies of God's apparition or working ſome ſpecial wonders or
benefits toward the people, were holy, as Bethel, Sinai, and others. And
much more thoſe times and places of Chriſtes natiuitie, Paſſion, burial,
Reſurrection, Aſcenſion: which is ſo plaine a caſe, that the hil where
he was transfigured only, is called therfore by S.~Peter,
\CNote{2.~Pet.~1.}
\Emph{the holy mount.}

Theſe
\MNote{Creatures hallowed by the ſigne of the Croſſe.}
therfore be holy memories and monuments of al ſorts ſanctified, beſides
that creatures, (as we ſee here) be ſanctified alſo by the word of God
and prayer, that is to ſay, by benediction and inuocation of our Lordes
holy name vpon them, ſpecially by the ſigne of the Croſſe, as
S.~Chryſoſtom noteth on this place,
\Cite{ho.~12. in 1. ad Tim.}
by the which the aduerſarie power of Satan vſurping vniuſtly vpon God's
creatures through man's ſinne, and ſeeking deceitfully in or by the ſame
to annoy man's body or ſoule, is expelled, and the meats purged from him
and made holeſom.
\MNote{The bleſſing of our meat what a vertue it hath.}
S.~Gregorie
\Cite{(lib.~1. Dialog. c.~4.)}
recordeth that the Diuel entred into a certaine religious woman by
eating the herbe lettuce vnbleſſed. And S.~Auguſtin
\Cite{li.~18. de ciu. Dei c.~18.}
sheweth at large, what waies he hath by meats and drinkes and other
vſual creatures of God, to annoy men: though his power be much leſſe
then it was before Chriſt. But ſtil much deſire he hath on al ſides to
moleſt the faithful by abuſing the things moſt neer and neceſſarie vnto
them, to their hurt both bodily and Ghoſtly. For remedie whereof, this
ſanctification which the Apoſtle ſpeaketh of, is very ſoueraigne,
pertaining not only to this common and more vulgar benediction of our
meats & drinkes, but much more (as the proprietie of the
\TNote{\G{ἁγιάζεται}}
Greek word vſed by the Apoſtle for ſanctification, doth import) to other
more exact ſanctifying & higher applying of ſome creatures, & bleſſing thẽ to
Chriſtes honour in the Church of God, & to man's ſpiritual & corporal
benefits.

For as S.~Auguſtin writeth
\Cite{lib.~2. de pec. merit. c.~26.}
beſides this vſual bleſſing of our daily food, the Cathecumens (that is,
ſuch as were taught toward Baptiſme) are ſanctified by the ſigne of the
Croſſe, and the bread, (ſaith he) which they receiue, though it be not
the body of Chriſt, yet it is holy, and more holy then the vſual bread
of the table.
\MNote{Holy bread.}
He meaneth a kind of bread then hallowed, ſpecially for ſuch as were not
yet admitted to the B.~Sacrament: either the ſame, or the like to our
holy bread, vſed in the Church of England and France on Sundaies. And
it was a common vſe in the primitiue Church to bleſſe loaues, and ſend
them for ſacred tokens from one Chriſtian man to another. And that not
among the ſimple and ſuperſtitious (as the Aduerſaries may imagine) but
among the holieſt, learnedſt, and wiſeſt.
\CNote{\Cite{Aug. Ep.~31.}
\Cite{34.}
\Cite{35.}
\Cite{36.}}
Such hallowed breads did
S.~Paulinus ſend to S.~Auguſtin and Alipius, and they to him againe,
calling them bleſſings. Read S.~Hierom in
\Cite{the life of Hilarion (poſt medium)}
how Princes and learned Biſhops & other of al ſorts came to that holy man
for holy bread, \L{panem benedictum}. In the primitiue Church the people
commonly brought bread to the Prieſts to be hallowed.
\Cite{Author op. imp. ho.~14. in Mt.}
\Cite{The 3.~Councel of Carthage cap.~14.}
maketh mention of the bleſſing of milke, honie, grapes, and corne. See
the
\Cite{4.~Canon of the Apoſtles.}
And not only diuers other creatures vſed at certaine times in holy
Churches ſeruice, as waxe, fire, palmes, aſhes, but alſo the holy oile,
Chryſme, & the water of Baptiſme, that alſo which is the cheefe of al
Prieſtly bleſſing of creatures, the bread and wine in the high
Sacrifice, be ſanctified. For without ſanctification,
\MNote{The ſigne of the Croſſe vſed in bleſſing.}
yea (as S.~Auguſtin affirmeth
\Cite{tract.~118. in Ioan.)}
without the ſigne of the Croſſe none of theſe things can rightly be done.

Can any man now maruel that the Church of God by this warrant of
S.~Paules word expounded by ſo long practiſe & tradition of the firſt
Fathers of our religion, doth vſe diuers elements and bleſſe them for
man's vſe and the ſeruice of God,
\MNote{The Churches exorciſmes.}
expelling by the inuocation of
Chriſtes name, the aduerſarie power from them, according to the
authoritie giuen by Chriſt,
\CNote{\XRef{Luc.~9.}}
\L{Super onmia dæmonia}, \Emph{ouer al Diuels}: and \Emph{by praier},
which importeth as the Apoſtle here ſpeaketh deſire of help, as it were
by the vertue of Chriſt, to combat with the Diuel, & ſo to expel him out
of God's creatures, which is done by holy exorciſme, and euer beginneth,
\L{Adiutorium noſtrum in nomine Domini}, as we ſee in the bleſſing of
holy water and the like ſanctification of elements? Which exorciſmes,
namely of children before they come to Baptiſme, ſee in S.~Auguſtin
\Cite{li.~6. cont. Iulian c.~5.}
&
\Cite{de Ec. dogmat. c.~31.}
\Cite{De nupt. & concupiſc. li.~1. c.~10.}
\MNote{Holy water.}
& of holy water, that hath been vſed theſe 1400.~yeares in the Church by
the inſtitution of Alexander the firſt, in al Chriſtian countries, and
of the force thereof againſt Diuels, ſee a famous hiſtorie in Theodoret
\Cite{li.~5. c.~21.}
and in Epiphanus
\Cite{hær.~30. Ebionitarum}
See S.~Gregorie to S.~Auguſtin our Apoſtle, of the vſe thereof in
hallowing the Idolatrous temples to be made the Churches of Chriſt.
\Cite{apud. Bedam li.~1. c.~30. hiſt. Angl.}
\MNote{The force of ſanctified creatures.}
Remember how the Prophet Eliſeus applied ſalt to the healing & purifying
of waters,
\XRef{4.~Reg.~2:}
how the Angel Raphael vſed the liuer of the fiſh to driue away the
Diuel,
\XRef{Tob.~6,~8:}
how Dauids harp and Pſalmodie kept the euil ſpirit from Saul,
\XRef{1.~Reg.~16:}
\MNote{The holy land.}
how a peece of the holy earth ſaued ſuch a man's chamber from
infeſtation of Diuels,
\Cite{Aug. de Ciuit. dei. li.~22. c.~8:}
how Chriſt himſelf, both in Sacraments, & out of them, occupied diuers
ſanctified elements, ſome for the health of the body, ſome for grace and
remiſſion of ſinnes, and ſome to worke miracles by.
\MNote{Relikes.}
See 
\Cite{S.~Hierom againſt Vigilantius c.~1.}
how holy Relikes torment them.
\CNote{Theodoret li.~3. c.~3.}
\MNote{The Croſſe.}
In the hiſtorie of Iulianus the Apoſtata, how the ſigne of the Croſſe;
\MNote{The name of \Sc{Iesvs}}
in the Actes
\Cite{(cap.~19.)}
how the name of \Sc{Iesvs} yea and of Paul putteth them to flight.

Furnish your ſelues with ſuch examples and grounds of Scriptures and
antiquitie, and you shal contemne the Aduerſaries cauillations, and
blaſphemies againſt the Churches practiſe in ſuch things, and further
alſo find theſe ſacred actions and creatures, not only by increaſe of
faith, feruour, and deuotion, to purge the impuritie of our ſoules, and
procure remiſſion of our daily infirmities, but that the cheefe
Miniſters of Chriſtes Church, by their ſoueraigne authoritie granted of
our Lord,
\MNote{Remiſſion of venial ſinnes annexed to halowed creatures.}
may ioyne vnto the ſame, their bleſſing and remiſſion of our venial
ſinnes or ſpiritual debts: as we ſee in
\CNote{\XRef{Ia. c.~5.}}
S.~Iames, remiſſion of al ſinnes
to be annexed to the vnction with holy oile, which to the Catholikes is
a Sacrament, but to the Proteſtants was but a temporal ceremonie, and to
ſome of them not of Chriſtes inſtitution, but of the Apoſtles only. In
their owne ſenſe therfore they should not maruel that ſuch ſpiritual
effectes should proceed of the vſe of ſanctified creatures, whereas
venial treſpaſſes be remitted many waies, though mortal ordinarily by
the Sacraments only.
\MNote{S.~Gregorie.}
S.~Gregorie did commonly ſend his benediction and remiſſion of ſinnes,
in and with ſuch holy tokens as were ſanctified by his bleſſing &
touching of the Apoſtles bodies and Martyrs Relikes, as now his
Succeſſours doe in the like hallowed remembrances of religion. See
\Cite{his 7.~booke, epiſtle~126:}
and
\Cite{9.~booke, epiſtle~60.}
Thus therfore and to the effects aforeſaid the creatures of God be
ſanctified.

If
\MNote{The difference betweene the Churches exorciſmes & other
coniurations.}
any man obiect that this vſe of creatures is like coniuration in
Necromancie, he muſt 
know the difference is, that in the Churches ſanctifications and
exorciſmes, the Diuels be commanded, forced, and tormented by Chriſtes
word & by praiers: but in the other wicked practiſes, they be pleaſed,
honoured, and couenanted withal: and therfore the firſt is godly and
according to the Scriptures, but Necromancie abominable and againſt
the Scriptures.}
ſanctified by the word of God and praier.

\V Theſe things propoſing to the Brethren, thou ſhalt be a good Miniſter
of Chriſt \Sc{Iesvs}, nouriſhed in the words of the faith and the good
doctrine which thou haſt attained vnto. \V
\CNote{\XRef{1.~Tim.~1,~4.}
\XRef{Tit.~3,~9.}}
But fooliſh and old wiues fables auoid: and
%%% o-2674
exerciſe thy ſelf to pietie. \V For
\SNote{Some (ſaith S.~Chryſoſtome) expound this of faſting, but they are
deceiued: for faſting
\Fix{in}{is}{obvious typo, fixed in other}
a ſpiritual exerciſe. See a goodly cõmẽtarie of theſe words in
\Cite{S.~Aug. li de mor. Eccl. Cath. c.~33.}}
corporal exerciſe is profitable to little: but pietie is profitable to al
things: hauing promiſe of the life that now is, and of that to come. \V
A faithful ſaying and worthie of al acceptation: \V For to this purpoſe
we labour and are reuiled, becauſe we hope in the liuing God which is
the Sauiour of al men, ſpecially of the faithful. \V Command theſe
things and teach.

\V Let no man contemne thy youth: but be an example of the faithful, in
word, in cõuerſation, in charitie, in faith, in chaſtitie. \V Til I
come, attend vnto reading, exhortation,
%%% !!! a Var with an empty first part goes here? Marked here as a
%%% LNote? Not marked in other.
%%% \Var{}{and}
doctrine. \V Neglect not
\LNote{The grace.}{S.~Auguſtin
\MNote{Grace giuen in the Sacrament of Orders.}
declareth this grace to be the guift of the holy Ghoſt giuen vnto him by
receiuing this holy Order, whereby he was made fit to execute the office
to his owne ſaluation and other mens. And note withal, that grace is not
only giuen in or with the Sacraments, by the receiuers faith or
deuotion, but by the Sacrament, \L{per impoſitionem}, \Emph{by impoſitiõ
of hands}. For ſo he ſpeaketh
\XRef{2.~tim.~1.}
which is here ſaid, \L{cum impoſitione}, \Emph{with impoſitiõ}.}
the grace that is in thee: which is giuen thee by prophecie,
\LNote{With impoſition.}{S.~Ambroſe
\MNote{Conſecration of Prieſts by impoſition of handes.}
\Cite{vpon this place,}
implieth in the word \Emph{impoſition of hands}, al the
holy action and ſacred words done and ſpoken ouer him when he was made
Prieſt: \Emph{Whereby} (ſaith he) \Emph{he was deſigned to the worke,
and receiued authoritie, that he durſt offer Sacrifice in our Lordes
ſteed vnto God.} So doth the holy Doctour allude vnto the words that are
ſaid now alſo in the Catholike Church to him that is made Prieſt:
\L{Accipe poteſtatem offerendi pre piuis & mortuis in nomine Domini}:
That is, \Emph{Take or receiue thou authoritie to offer for the liuing
and the dead in the name of our Lord.} For the which
\CNote{In Eſa. 6,~58.}
S.~Hierom alſo (as is noted before) ſaith, that the ordering of Prieſts
is, \Emph{by impoſition of hands and imprecation of voice}.}
with impoſition of the hands
\LNote{Of Prieſthood.}{The
\MNote{Holy Orders a Sacrament.}
practiſe of the Church giueth vs the ſenſe of this place, which the
ancient
\CNote{\Cite{Conc. Carth.~4. c.~3.}}
Councel of Carthage doth thus ſet downe. \Emph{When a Prieſt taketh
orders, the Bishop bleſſing him and holding his hand vpon his head, let
al the Prieſts preſent lay alſo their hands on his head by the Bishops
hands, &c.} Who ſeeth not now, that holy Orders giuing grace by an
external ceremonie and worke, is a Sacrament? So al the old Church
counteth it. And S.~Auguſtin
\Cite{(cont. ep. Parmen li.~2. c.~13.)}
plainely ſaith that no man doubteth but it is a Sacrament. And leſt any
man thinke that he vſeth not the word Sacrament properly and preciſely,
he ioyneth it in nature and name with Baptiſme. Againe who ſeeth not by
this vſe 
of impoſition of hãds in giuing Orders & other Sacramẽts, that Chriſt,
the Apoſtles, and the Church may borow of the Iewish rites, certaine
cõuenient ceremonies & Sacramental actiõs, ſeeing this ſame (
\CNote{\Cite{Beza in cap.~6. Act.}}
as the Heretikes can not deny) was receiued of the manner of ordering
Aaron and the Prieſts of the old law or other Heads of the people? See
\XRef{Exod.~39.}
\XRef{Num.~17,~23.}}
of prieſthood. \V
%%% 2813
Theſe things doe thou meditate, be in theſe things: that thy profiting
may be manifeſt to al. \V Attend to thy ſelf, and to doctrine: be
earneſt in them. For, this doing, thou ſhalt
\LNote{Saue both thy ſelf.}{Though
\MNote{Men alſo are called Sauiours without derogation to Chriſt.}
Chriſt be our only Sauiour, yet the Scriptures forbeare not to ſpeake
freely and vulgarly & in a true ſenſe, that man alſo may ſaue himſelf &
others. But the Proteſtants notwithſtanding follow ſuch a captious kind
of Diuinitie that if a man ſpeake any ſuch thing of our Lady or any
Saint in heauen, or other meane of procuring ſaluation, they make it a
derogation to Chriſtes honour. With ſuch hypocrites haue we not adaies
to doe.}
ſaue both thy ſelf and them that heare thee.


\stopChapter


\stopcomponent


%%% Local Variables:
%%% mode: TeX
%%% eval: (long-s-mode)
%%% eval: (set-input-method "TeX")
%%% fill-column: 72
%%% eval: (auto-fill-mode)
%%% coding: utf-8-unix
%%% End:

