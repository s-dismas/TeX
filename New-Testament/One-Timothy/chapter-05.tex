%%%%%%%%%%%%%%%%%%%%%%%%%%%%%%%%%%%%%%%%%%%%%%%%%%%%%%%%%%%%%%%%%
%%%%
%%%% The (original) Douay Rheims Bible 
%%%%
%%%% New Testament
%%%% Epistles
%%%% One Timothy
%%%% Chapter 05
%%%%
%%%%%%%%%%%%%%%%%%%%%%%%%%%%%%%%%%%%%%%%%%%%%%%%%%%%%%%%%%%%%%%%%




\startcomponent chapter-05


\project douay-rheims


%%% 2817
%%% o-2677
\startChapter[
  title={Chapter 5}
  ]

\Summary{How to behaue himſelf towardes yong & old. 3.~To beſtow the
  Churches oblations vpon the needy widowes, 9.~and not to admit the
  ſaid Churches widowes vnder threeſcore yeares old. 17.~In diſtribution
  to reſpect wel the Prieſts that are painful: 19.~& how in his
  Conſiſtorie to heare accuſations againſt Prieſts. 22.~To be ſtrait in
  examining before he giue Orders. To be chaſt, and to remit ſomewhat of
  his drinking water.}

%%% o-2678
A Seniour rebuke not: but beſeech as a father: yong men, as brethren: \V
old women, as mothers: yong women, as ſiſters, in al chaſtitie.

\V Honour widowes, which are
\LNote{Widowes indeed.}{S.~Ambroſe
\CNote{\Cite{Ambr. in hunc loc.}}
\MNote{Widowhood.}
calleth them widowes and deſolate indeed, that might marrie, but to make
themſelues better and more worthy of God, refuſe marriage, which they
know to be but once bleſſed, imitating
\CNote{\XRef{Luc. c.~2,~37.}}
holy Anne, who in faſting and praiers ſerued God night and day, neuer
knowing but one huſband. Such profeſſed widowes then are to be honoured
and ſuccoured. Neither doth he ſpeake only of the Churches widowes (of
whom ſpecially afterward) but of al that by profeſſion kept their
widowhood, exhorting them to paſſe their time in praier and faſting,
\XRef{v.~5.}
Which was an honourable and holy ſtate, much written of and commended in
the primitiue Church, namely by S.~Ambroſe and by S.~Auguſtin, who wrote
bookes intitled thereof, and make in next to virginitie.
\Cite{Ambr. de viduit.}
\Cite{Auguſt. de bono viduitatis.}}
widowes indeed. \V But if any widow haue children or nephewes; let her
learne firſt to rule her owne houſe, & to render mutual dutie to her
parents. For this is acceptable before God. \V But ſhe that is a widow
indeed and deſolate, let her hope in God, and continue in obſecrations &
\SNote{Becauſe of this continual prayer which ſtandeth not with coniugal
& carnal actes of matrimonie (as the Apoſtle ſignifieth
\XRef{1.~Cor.~7,~5.)}
therfore were theſe widowes to liue in the ſtate of perpetual
continencie.}
praiers night and day. \V For she that is in deliciouſnes, liuing is
dead. \V And this command that they be blameleſſe. \V But if any man
haue not care of his owne and eſpecially of his domeſticals,
\LNote{He hath denied.}{Not that by this or by any other deadly ſinne
(except incredulitie or doubtfulnes in beleefe) they loſe their faith:
but that their facts be not anſwerable to their faith and to Chriſtian
religion, which preſcribeth al ſuch duties.}
he hath denied the faith, and is worſe then an infidel. \V
\LNote{Let a widow be choſen.}{Now
\MNote{Widowes called \L{Diaconiſſæ}, and their office.}
he ſpeaketh more particularly and ſpecially of ſuch widowes as were
nouriſhed and found by the oblations of the faithful and the almes of
the Church, and did withal ſome neceſſarie ſeruices about women that
were to be profeſſed or baptized, for their inſtruction and addreſſing
to that and other
\Fix{Sacramens,}{Sacraments,}{obvious typo fixed in other}
and alſo about the ſicke and impotent: and
withal ſometimes they had charge of the Church goods or the diſpoſition
of them vnder the Deacons: in reſpect whereof they alſo and the like are
called \L{Diaconiſſæ}. Euſebius
\Cite{li.~6. c.~35.}
reciteth out of Cornelius Epiſtle, that in the Church of Rome there is
one Biſhop, 40.~Prieſts, ſixe Deacons, ſeuen Subdeacons, Acoluthi~42,
Exorciſtes, Lectors, & Oftiarij~52, widowes together with the poore~150,
al which God nouriſheth in his Church. See
\XRef{Act. Apoſt. c.~6.}
\Cite{S.~Chryſoſtom li.~2. de Sacerdotio propius finem.}
\Cite{S.~Epiphanius in hæreſi~39. Collyridianorum.}
Now then, what manner of women ſhould be taken into the fellowſhip of
ſuch as were found of the Church, he further declareth.}
Let a widow be choſen of no leſſe then three-ſcore yeares, which hath
been the
\LNote{The wife of one husband.}{If
\MNote{Theſe widowes muſt haue had but one huſbãd: wherof many Catholike
concluſions are deduced.}
you would haue a plaine patterne of Heretical fraud, corruption, and
adulteration of the natiue ſenſe of God's word, and an inuincible
demonſtration that theſe new Gloſſers haue their conſciences ſeared and
harts obdurated, willingly peruerting the Scriptures againſt that which
they know is the meaning thereof, to the maintenance of their Sects:
marke wel their handling of this place about theſe widowes of the
Church. S.~Paul preſcribeth ſuch only to be admitted as haue been the
wiues of one huſbãd, that is to ſay, once only married, not admitting
any that hath been twiſe married.
\MNote{1.}
By which words the Catholikes proue firſt, that the like phraſe
\CNote{\XRef{c.~3,~2.}}
vſed before of Biſhops and Deacons, that they ſhould be the huſbands of
one wife, muſt needs ſignifie that they can not be twiſe married, nor
admitted to theſe and the like functions, if they were more then once
married before.
\MNote{2.}
Secondly, we proue by this place againſt the Aduerſaries, that the ſtate
of widowhood is more worthy, honourable, decent, and pure in reſpect of
the ſeruice of the Church, and more to be relieued of the reuenues
thereof, then the ſtate of married folkes. And that not only (as the
Aduerſaries perhaps may anſwer) for their greater neceſſitie, or more
leiſure, freedom, or expedition to ſerue, in that they be not cumbered
with huſband and houſhold, but in reſpect of their vidual continencie,
chaſtitie, and puritie. For els ſuch as were widowes with intention and
freedom to marrie afterward, might haue been admitted by the Apoſtle, as
wel as thoſe that were neuer to marrie againe.

Thirdly,
\MNote{3.}
we proue that ſecond marriage not only after admiſſion to the almes or
ſeruice of the Church, but before alſo, is diſagreable & a ſigne of
incontinencie or more luſt and fleſhlines then is agreable or comely for
any perſon belonging to the Church: and conſequently, that the Apoſtle
in the laſt chapter treating of the holy functions of Biſhops, Prieſts,
Deacons, and of the Churches refuſing generally \L{bigamos} or twiſe
married perſons, muſt needs much more meane that no man twiſe married
ſhould be receiued to holy Orders: and further, that as none were
admitted to be widowes of the Church, that euer intended to marrie
againe, ſo none ſhould euer be receiued to miniſter the Sacraments
(which is a thing infinitly more, and requireth more puritie, and
continencie, then the office or ſtate of the ſaid widowes) that intended
to marrie againe. To receiue the body of Chriſt (ſaith S.~Hierom
\Cite{in Apolog. pro lib. cont Iouin. ep.~50. c.~6.)}
is a greater and holier thing then praier, and therfore Prieſts that
muſt both continually pray and alſo be occupied about the receiuing or
miniſtring the holy Sacrament daily, muſt liue continently.

Fourthly,
%%% !!! only marked in other
\MNote{4.}
we proue that it is not vnlawful to annexe, by precept or the parties
promiſe, ſingle life or chaſtitie to a whole State or Order of the
faithful: becauſe the Apoſtle & the whole Church in his time ioyned to
this State of the Churches widowes perpetual continencie.
%%% !!! only marked in other
\MNote{5.}
Fifthly, we proue hereby that to refuſe and not to accept the twiſe
married or ſuch as wil not liue ſingle, into the State of widowes or
holy Orders, is not to contemne or forbid ſecond marriage, or once &
often marrying, with the Manichees according to the doctrine of Diuels,
as the Proteſtants (and before them the old condemned Iouinianiſtes) doe
blaſpheme the Church. For then did S.~Paul allow and teach doctrine of
Diuels, who refuſeth a twiſe married woman, and bindeth others by their
entring into this State, neuer to marrie againe: as no doubt he did the
Clergie men much more in the
\XRef{3.~chapter before.}
Thus loe we Catholikes conferre & conſter the Scriptures, and for this
meaning we haue al the Doctours without exception. What ſhift then haue
the Heretikes here? For marrie and remarrie they muſt, let the
Scriptures, & al the Doctours in the world ſay nay to it. In truth they
doe not expound the word of God, but fly from the euidence of it, ſome
one way & ſome another.

And
\MNote{The Caluiniſts moſt abſurd expoſition of the Apoſtles wordes.}
of al other, their extremeſt and moſt ſhameful tergiuerſation is,
that the Apoſtle here forbiddeth
\CNote{\Cite{Beza vpõ this place.}}
not the admiſſion of ſuch widowes as haue been twiſe married, but only
them that haue had two huſbands at once. Which was a very vnprobable and
extorted expoſition before, concerning Biſhops and Deacons,
\XRef{c.~3.}
and (as S.~Hierom ſaith
\Cite{ep.~83.)}
\L{malo nodo malus cuneus}: but here that an exception ſhould be made
only againſt widowes that had two husbands together (which was a thing
neuer lawful nor neuer heard of) that is a moſt intolerable impudencie,
and a conſtruction that neuer came to any wiſe mans cogitation before; &
yet theſe their fanſies muſt be God's word, and \L{bigamus} or
\L{bigamia} muſt againſt their old natures, and vſe of al Writers, be
al one with \L{Poligamus} and \L{Poligamia}. They giue an example of ſuch
widowes, in women diuorced iuſtly from their husbands in the old law. As
though S.~Paul here tooke order for the Iewes widowes only, or that had
been ſuch a common caſe among the Iewes alſo, that the Apoſtle needed to
take ſo careful order for it.
\MNote{Their blaſphemie againſt the plaine text.}
Finally, they let not to ſay that if the Apoſtle ſhould be vnderſtood to
refuſe a widow twiſe married at ſundrie times, it were vnreaſonable and
iniurious to ſecond marriages, which haue no more indecencie or ſigne of
incontinencie (ſay they) then the firſt. Thus bold they are with the
Apoſtle and al antiquitie.}
wife of one huſband, \V hauing teſtimonie in good workes, if ſhe haue
brought vp her children, if she haue receiued to harbour, if she haue
washed the Saints feet, if she haue miniſtred to them that ſuffer
tribulation, if she haue followed euery good worke. \V But the yonger
widowes auoid. For when they shal be
\LNote{Wanton in Chriſt.}{Widowes waxing warme, idle, and wel fedde by
the Church, luſt after husbands, as alſo Apoſtate-Prieſts and
Superintendents marrie, ſpecially after they haue gotten good
Eccleſiaſtical liuings. Which
is to waxe wanton in Chriſt, or againſt Chriſt \G{κατὰ Χριϛοῦ}.
\TNote{\G{καταϛρηvιᾶv}}
The Greek word ſignifieth to caſt off the raines or bridle, that is, the
bond or promiſe of continencie which they had put vpon them.}
wanton in Chriſt,
\LNote{They wil.}{\Emph{In
\MNote{There very wil to breake the vow of chaſtitie, is damnable.}
the chaſtitie of widowhood or Virginitie} (ſaith S.~Auguſtin) \Emph{the
excellencie of a greater guift is ſought for. Which being once deſired,
choſen, & offered to God by vow, it is not only damnable to enter
afterward into marriage, but though it come not actually to marriage,
only to haue the wil to marrie is damnable.}
\Cite{Aug. li. de bono. viduit. cap.~9.}}
they wil marrie: \V
\LNote{Hauing damnation.}{It ſignifieth not blame, check, or
reprehenſion of men, as ſome to make the fault ſeeme leſſe, would haue
it: but
\TNote{\G{κρίμα}}
iudgement or eternal damnation, which is a heauy ſentence. God grant al
married Prieſts and Religious may conſider their lamentable caſe. What a
grieuous ſinne it is, ſee S.~Ambroſe
\Cite{ad virginem lapſaam cap.~5. &~8.}}
hauing damnation, becauſe they haue made void
\LNote{Their firſt faith.}{Al
\MNote{\Emph{Breaking of their firſt faith}, is (by the conſent of al
antiquitie) when they breake their vow of chaſtitie.}
the Ancient Fathers that euer wrote commentaries vpon this Epiſtle,
Greek and Latin, as S.~Chryſoſtom, Theodoret, Oecumenius, Theophylatus,
Primaſius, S.~Ambroſe, Ven.~Bede, Anſelme, & the reſt: alſo al others
that by occaſion vſe this place, as the
\Cite{4.~Councel of Carthage cap.~104.}
& the
\Cite{4.~of Toleto. cap.~55.}
S.~Athanaſius
\Cite{li. de virginitate.}
S.~Epiphanius
\Cite{hær.~48.}
S.~Hierom
\Cite{cont. Iouinianum li.~1. c.~7.}
& in
\Cite{c.~44. Ezech. Prope finem.}
S.~Auguſtin in exceeding many places: al theſe expound the Apoſtles
words of the vow of Chaſtitie or the faith and promiſe made to Chriſt to
liue continently. \Emph{What is to breake their firſt faith?} ſaith
S.~Auguſtin. \Emph{They vowed, and performed not.}
\Cite{In Pſ.~75. prope finem.}
Againe in another place, \Emph{They breake their firſt faith, that ſtand
not in that which they vowed.}
\Cite{Li.~de Sancta virgin. c.~33.}
Againe he and al the Fathers with him in
\Cite{Carthage Councel before named:}
\MNote{Why this vow is called faith or fidelitie.}
\Emph{If any widowes, how yongue ſo euer they were left of their
husbands deceaſed, haue vowed themſelues to God, left their laical
habit, and vnder the teſtimonie of the Bishop and Church haue appeared
in religious weed, & afterward got any more to ſecular marriage,
according to the Apoſtles ſentence they shal be damned, becauſe they
were ſo bold to make void the faith or promiſe of Chaſtitie which they
vowed to our Lord.} So ſaith he and 215.~Fathers moe in that Councel.

And this promiſe of chaſtitie is called, \Emph{faith}, becauſe the
fidelitie betwixt married perſons is ordinarily called of holy Writers,
\Emph{faith}: and the vow of chaſtitie made to God, ioyneth him and the
perſons, ſo vowing, as it were in marriage, ſo farre, that if the ſaid
perſons breake promiſe, they are counted and called in the laſt alleaged
Councel, \Emph{God's adulterers}. In the
\XRef{3.~to the Romanes}
alſo and often els where, faith is taken for promiſe or fidelitie. And
that it is ſo taken here, the words \L{irritam fecere} (to fruſtrate and
make void) doe proue: for that terme is commonly vſed in matter of vow,
promiſe, or compact.
\XRef{Gen.~17.}
\XRef{Num.~30.}
\MNote{Why the firſt faith.}
This promiſe is called here \L{prima fides} (the firſt faith) in reſpect
of the later promiſe which vow-breakers make to them with whom they
pretend to marrie. So ſaith S.~Auguſtin
\Cite{lib. de bono viduit. c.~8. &~9.}
and Innocentius~1.
\Cite{ep.~2. cap.~23. to.~1. Conc.}
And this is the only natiue, euident, and agreable ſenſe to the
circumſtance of the letter. And the vaine euaſion of the Heretikes to
ſaue the Apoſtate-Monkes, Friers, Nunnes, and Prieſts from damnation for
their pretended marriages, is friuolous: to wit, that \Emph{firſt faith}
here ſignifieth the faith of Baptiſme or Chriſtian beleefe, and not the
promiſe or vow of Chaſtitie.
\MNote{The heretikes expoſition of this firſt faith, impoſſible and
againſt the text.}
But we aske them if this faith or Baptiſme be broken by marriage or
no. For the text is plaine that by intending to marrie, they breake
their faith, and by breaking their faith they be damned, if they die
without repentance. In truth which way ſo euer they writh themſelues to
defend their ſacriledge or pretended marriages, they loſe their labour
and ſtruggle againſt their owne conſcience and plaine Scripture.}
their firſt faith. \V And withal idle alſo they learne to goe about from
houſe to houſe: not only idle, but alſo ful of words & curious, ſpeaking
things which they ought not. \V
\LNote{I wil the yonger.}{He
\MNote{S.~Paul meaneth not that widowes profeſſed ſhould marrie.}
ſpeaketh of ſuch yong ones as were yet free. For ſuch as had already
made vow, neither could they without damnation marrie, were they yong
or old, nor he without ſinne command or counſel them to it. Neither (as
S.~Hierom proueth to
\CNote{otherwiſe Ageruchia ep.~11.}
Gerontia, and S.~Chryſoſtom
\Cite{vpon this place)}
doth he preciſely command or counſel the yong ones that were free, to
marrie, or abſolutely forbid them to vow chaſtitie: God forbid ſay
they.
\MNote{It is better for the fraile ſort, that are in dãger of falling,
to marrie rather then to vow.}
But his ſpeach conteineth only a wiſe admonition to the frailer
ſort, that it were farre better for them not to haue vowed at al, but to
haue married againe, then to haue fallen to aduoutrie and Apoſtaſie
after profeſſion. Which is no more but to preferre ſecond marriage
before fornication: and a good warning, that they which are to profeſſe,
looke wel what they doe. S.~Paules experience of the fal of ſome yong
ones to marriage, cauſed him to giue this admonition here: as alſo that
before, that none ſhould be receiued to the Churches almes vnder
threeſcore yeares of age. Not forbidding the Church for euer, to accept
any vowes of widowes or virgins til that age, as the Heretikes falſely
affirme: but ſhewing what was meet for that time and the beginning of
Chriſtianitie, when as yet there were no Monaſteries builded, no
preſcript rule, no exact order of obedience to Superiours: but the
profeſſed (as S.~Paul here noteth) courſed and wandered vp and downe
idly, as now our profeſſed virgins or Nunnes doe not, neither can
doe. Of whom therfore, where diſcipline is obſerued, there is no cauſe
of ſuch danger. Beſides that widowes hauing had the vſe of carnal
copulation before, are more dangerouſly tempted, then virgins that are
brought vp from their tender age in pietie and haue no experience of
ſuch pleaſures.
\MNote{Yong women may be profeſſed and taken into religion.}
See S.~Ambroſe
\Cite{lib. de viduit.}
prouing by the example of holy Anna who liued a widow euen from her
youth til 80.~yeares of age, in faſting and praying night and day, that
the Apoſtle doth not here without exception forbid al yong widowes to
vow, yea he eſteemeth that profeſſion in the yonger women much more
laudable, glorious, and meritorious. See his booke
\Cite{de Vidiut in initio.}}
I wil therfore the yonger to marrie, to bring forth children, to be
houſe-wiues: to giue no occaſion to the aduerſarie for to ſpeake
euil. \V For now certaine are turned backe
\LNote{After Satan.}{We
\MNote{To marrie after the vow of chaſtitie, it to goe after Satan.}
may here learne, that for thoſe to marrie which are profeſſed, is to
turne backe after Satan. For he ſpeaketh of ſuch as were married
contrarie to their vow. And hereupon we cal the Religious that marrie
(as Luther, Bucer, Peter Martyr and the reſt) Apoſtatae. More we learne,
that ſuch yong ones haue no excuſe of their age, or that they be
vehemently tempted and burne in their concupiſcences, or that they haue
not the guift of Chaſtitie. For notwithſtanding al theſe excuſes, theſe
yong profeſſed widowes if they marrie, goe backward after Satan, and be
Apoſtataes, and damned except they repent. For as for the Apoſtles words
to the Corinthians,
\CNote{\XRef{1.~Cor.~7.}}
\Emph{It is better to marrie then to burne}, we haue before declared out
of the Fathers, and here we adde, that it pertaineth only to perſons
that be free and haue not vowed to the contrarie. As S.~Ambroſe
\Cite{li. ad virg. lapſ. c.~5.}
S.~Auguſtin
\Cite{de bono vid. c.~8.}
and S.~Hierom
\Cite{li.~1. cont. Iouin. c.~7.}
expound it.

The
\MNote{The heretikes only remedie againſt concupiſcence is marriage.}
Heretikes of our time thinke there is no remedie for fornication or
burning, but marriage, and ſo did S.~Auguſtin when he was yet a
Manichee. \L{Putabam me miſerum &c.} \Emph{I thought} (ſaith he
\Cite{li.~6. Confeſ. c.~13.)}
\Emph{that I should be an vnhappie and miſerable man if I should lacke
the companie of a woman, and the medicine of thy mercie to heale the
ſame infirmitie I thought not vpon, becauſe I had not tried it: and I
imagined that Continencie was in a mans owne power and libertie, which
in my ſelf I did not feele: being ſo foolish not to vnderſtand that no
man can be continent vnles thou giue it. Verily thou wouldeſt giue it,
if with inward mourning I would knocke at thy eares, and with ſound
faith would caſt my care vpon thee.}

By
\MNote{The vow of chaſtitie lawful, poſſible to be kept, more grateful
to God.}
al which you may eaſily proue, that chaſtitie is a thing that may
lawfully be vowed, that it is not impoſſible to be fulfilled by praier,
faſting, and chaſtiſement of mens concupiſcence, that it is a thing more
grateful to God then the condition of married perſons: for els it ſhould
not be required either in the Clergie or in the Religious. Finally that
it is moſt abominable to perſuade the poore virgins or other profeſſed
to ſuch ſacrilegious wedlocke, which S.~Auguſtin auoucheth to be worſe
then aduoutrie.
\Cite{de bono vidu. c.~4.}
\Cite{11.}
\MNote{Iouinians hereſie in this point, cõdemned of old, is called of
the Proteſtants, Gods word.}
Iouinian was the firſt that euer made marriage equal with virginitie or
chaſt life, for which he was condemned of hereſie.
\Cite{Aug. in argumento li. de bono Coniugalis.}
\Cite{De pec. merit. li.~3. c.~7.}
\Cite{Li. de hæreſ. hær.~82.}
He was the firſt that perſuaded profeſſed virgins to marrie, which
S.~Auguſtin ſaith was ſo clerely and without queſtion wicked, that it
could neuer infect any Prieſt, but certaine miſerable Nunnes. Yea for
this ſtrange perſuaſion he calleth Iouinian a monſter, ſaying of him
thus
\Cite{Retract. cap.~22.}
\Emph{The holy Church that is there} (at Rome) \Emph{moſt faithfully and
ſtoutly reſiſted this monſter.} S.~Hierom called the ſaid Heretike and
his Complices, \Emph{Chriſtian epicures}.
\Cite{li.~2. cont. Iouin. c.~19.}
See S.~Ambroſe
\Cite{ep.~82. ad Vercellenſem epiſcorum in initio.}
But what would theſe holy Doctours haue ſaid, if they had liued in our
doleful time, when the Proteſtants goe quite away with this wickednes,
and cal it God's word?}
after Satan. \V If any faithful man haue widowes, let him miniſter to
them, and let not the Church be burdned: that there may be ſufficient
for them that are widowes indeed.

\V The Prieſts that rule wel, let them be eſteemed,
\SNote{Double honour & liuelihood due to good Prieſts.}
worthie of double honour: eſpecially they that labour
\LNote{In word and doctrine.}{Such
\MNote{Many good and worthie Biſhops, that haue not the guift of
preaching and teaching.}
Prieſts ſpecially and Prelates are worthy of double, that is, of the
more ample honour, that are able to preach and teach, and doe take
paines therin. Where we may note, that al good Biſhops or Prieſts in
thoſe daies were not ſo wel able to teach as ſome others, and yet for
the miniſterie of the Sacraments, and for wiſedom and gouernement were
not vnmeet to be Biſhops and Paſtours. For though it be one high
commendation in a Prelate, to be able to teach, as the Apoſtle before
noted: yet al can not haue the like grace therin, and it is often
recompenſed by other ſingular guifts no leſſe neceſſarie. S.~Auguſtin
laboured in word and doctrine, Alipius and Valerius were good Biſhops,
and yet had not that guift.
\Cite{Poſsid. in vit. Aug. c.~5.}
And ſome times and countries require Preachers more then other. Al which
we note, to diſcouer the pride of Heretikes, that contemne ſome of the
Catholike Prieſts or Biſhops, pretending that they can not preach as
they doe, with meretricious and painted eloquence.}
in the word and doctrine. \V For the Scripture ſaith:
\CNote{\XRef{Deut.~25.}
\XRef{1.~Cor.~9.}}
\Emph{Thou shalt not mooſel the mouth to the oxe that treadeth out the
corne}; and,
\CNote{\XRef{Mt.~10,~10.}}
\Emph{The worke-man is
\Fix{vorthie}{worthie}{obvious typo, fixed in other}
of his hire.} \V
\SNote{Here the Apoſtle wil not haue euery light fellow to be heard
againſt a Prieſt. So S.~Aug. for the like reuerence of prieſthood,
admoniſheth Pancarius that in no wiſe he admit any teſtimonies or
accuſations of Heretikes againſt a Catholike Prieſt.
\Cite{ep.~212.}}
Againſt a Prieſt receiue not accuſation, but vnder two or three
witneſſes. \V Them that ſinne, reproue before al: that the reſt alſo may
haue feare.

\V I teſtifie before God and Chriſt \Sc{Iesvs}, and the elect Angels,
that thou keep theſe things without preiudice,
%%% o-2679
doing nothing by declining to the one part. \V Impoſe hands on no man
\SNote{Biſhops muſt haue great care that they giue not Orders to any
that is not wel tried for his faith, learning, and good behauiour.}
lightly, neither doe thou communicate with other mens ſinnes. Keep thy
ſelf chaſt. \V Drinke not yet
\LNote{Water.}{You ſee how lawful and how holy a thing it is, to faſt
from ſome meates or drinkes, either certaine daies, or alwaies, as this
B.~Biſhop Timothee did: who was hardly induced by the Apoſtle to drinke
a litle wine with his water in reſpect of his infirmities. And marke
withal, what a calumnious and ſtale cauillation it is, that to abſtaine
from certaine meates and drinkes for puniſhment of the body or deuotion,
is to condemne God's creatures. See an homilie of S.~Chryſoſtom
\Cite{vpon theſe words, to.~5.}}
water; but vſe a litle wine for thy ſtomake, and thy often
infirmities. \V Certaine mens ſinnes be manifeſt, going before to
iudgement: and certaine men they follow. \V In like manner alſo good
deeds be manifeſt, and they that are otherwiſe, can not be hid.


\stopChapter


\stopcomponent


%%% Local Variables:
%%% mode: TeX
%%% eval: (long-s-mode)
%%% eval: (set-input-method "TeX")
%%% fill-column: 72
%%% eval: (auto-fill-mode)
%%% coding: utf-8-unix
%%% End:

