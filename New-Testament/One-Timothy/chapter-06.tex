%%%%%%%%%%%%%%%%%%%%%%%%%%%%%%%%%%%%%%%%%%%%%%%%%%%%%%%%%%%%%%%%%
%%%%
%%%% The (original) Douay Rheims Bible 
%%%%
%%%% New Testament
%%%% Epistles
%%%% One Timothy
%%%% Chapter 06
%%%%
%%%%%%%%%%%%%%%%%%%%%%%%%%%%%%%%%%%%%%%%%%%%%%%%%%%%%%%%%%%%%%%%%




\startcomponent chapter-06


\project douay-rheims


%%% 2822
%%% o-2682
\startChapter[
  title={Chapter 6}
  ]

\Summary{What to teach ſeruants. 3.~If any teach againſt the doctrine of
  the Church obſtinately, he doth it of pride and for lucre. 11.~But
  the Catholike Bishop muſt follow vertue, hauing his eye alwaies to
  life euerlaſting and to the comming of Chriſt. 17.~What to command the
  rich. 20.~Finally, to keep moſt carefully the Catholike Churches
  doctrine, without mutation.}

Whoſoeuer are ſeruants vnder yoke, let them count their Maſters worthie
of al honour; leſt the name of our Lord and his doctrine be
blaſphemed. \V But they that haue faithful Maſters, let them not
contemne them becauſe they are Brethren, but ſerue the rather, becauſe
they be faithful and beloued, which are partakers of the benefit. Theſe
things teach and exhort.

\V If any man
\SNote{See the
\XRef{annotation before cap.~1,~3.~4.}}
\TNote{\G{ἑτεροδιδασκαλεῖ}}
teach otherwiſe, and conſent not to the ſound words of our
Lord \Sc{Iesvs} Chriſt, and to that
%%% o-2683
doctrine which is according to pietie, \V he is proud, knowing nothing,
but
\LNote{Languishing.}{Euen theſe be the good diſputes of our new
Sect-maiſters; and the world hath too long proued theſe inconueniences
here named, to be the fruits of ſuch endles altercations in religion as
theſe vnhappie Sects haue brought forth.}
languiſhing about queſtions and ſtrife of words: of which riſe
enuies, contentions, blaſphemies, euil ſuſpicions, \V conflicts of men
corrupted in their mind, and that are depriued of the truth, that eſteem
gaine to be pietie. \V But pietie with ſufficiencie is great gaine. \V
For we
\CNote{\XRef{Iob.~1,~21.}}
brought nothing into this world: doubtleſſe, neither can we take away
any thing. \V But
\CNote{\XRef{Mat.~6,~25.}}
hauing food, and wherwith to be couered, with theſe we are content. \V
For they that wil be made rich, fal into tentation and the ſnare of the
Diuel, and many deſires vnprofitable and hurtful, which drowne men into
deſtruction and perdition. \V For the root of al euils is couetouſenes;
\SNote{As in the
\XRef{1.~chap.}
lacke of faith and good conſcience, ſo here couetouſnes or deſire of
theſe temporal things, and in
\XRef{the end of this chap.}
preſumption and boaſting of knowledge, are cauſes of falling from the
faith: hereſie often being the punishment of former ſinnes.}
which certaine deſiring haue erred from the faith, and haue intangled
themſelues in many ſorrowes.

\V But thou, ô man of God, fly theſe things; and purſue iuſtice, pietie,
faith, charitie, patience, mildnes. \V Fight the good fight of faith:
apprehend eternal life, wherin thou art called and haſt
\Fix{cofeſſed}{confeſſed}{obvious typo, fixed in other}
a good confeſſion before many witneſſes. \V I command thee before God
who quickneth al things, and Chriſt \Sc{Iesvs} who
\CNote{\XRef{Io.~18,~37.}}
gaue teſtimonie vnder Pontius Pilate a good confeſſion, \V that thou
keep the commandment without ſpot, blameleſſe vnto the comming of our
Lord \Sc{Iesvs} Chriſt. \V Which in due times the Bleſſed and only
Mightie wil ſhew, the
\CNote{\XRef{Apoc.~17,~14.}
\XRef{19,~16.}}
King of kings and Lord of lords, \V who only hath immortalitie, and
inhabiteth light not acceſsible,
\CNote{\XRef{Io.~1,~18.}}
whom no man hath ſeen, yea neither can ſee, to whom be honour and empire
euerlaſting. Amen.

\V Command the rich of this world not to be high minded, nor to truſt in
the vncertaintie of riches, but in the liuing God (who giueth vs al
things aboundantly to enioy) \V to doe wel, to become rich in good
workes, to giue eaſily, to communicate, \V to heap vnto themſelues a
good
\SNote{Almes deedes and good workes laid for a foundation and ground to
attaine euerlaſting life. So ſay the Doctours vpon this place.}
foundation for the time to come, that they may apprehend the true life.

\V Ô Timothee, keep the
\LNote{Depoſitum.}{The
\MNote{\L{Depoſitum}, is the Catholike truth deſcending from the
Apoſtles by ſucceſſion of Biſhops, euen vnto the end.}
whole doctrine of our Chriſtianitie being taught by the Apoſtles, &
deliuered to their Succeſſours, and comming downe from one Biſhop to
another, is called the \L{Depoſitum}, as it were a thing laid into their
hands, and committed vnto them to keep. Which becauſe it paſſeth from
hand to hand, from Age to Age, from Biſhop to Biſhop without corruption,
change, or alteration, is al one with Tradition, and is the truth giuen
vnto the holy Biſhops to keep, and not to lay men. See the notable
diſcourſe of Vicentius Lirinenſis vpon this text:
\Cite{li. con. profan. hær. Nouationes.}
And it is for this great, old, and knowen treaſure committed to the
Biſhops cuſtodie, that S.~Irenæus calleth the Catholike Church
\L{Depoſitorium diues}, \Emph{the rich treaſure of truth.}
\Cite{lib.~3. c.~4.}
And as Clemens Alexandrinus writeth
\Cite{li.~2. Strom.}
this place maketh ſo much againſt al Heretikes who doe al change this
\L{Depoſitum}, that for it only ſuch men in his daies denied this
Epiſtle.
\MNote{The Proteſtants can ſhew no ſuch \L{depoſitum}.}
The Heretikes of our daies change alſo the truth, and ſay it is the old
truth. But they leap 14.~or 15.~hundreth yeares for it ouer mens heads
to the Apoſtles. But we cal for the \L{Depoſitum}, and aske them in
whoſe hands that truth which they pretend, was laid vp, and how it came
downe to them. For it can not be Apoſtolical, vnles it
were \L{Depoſitum} in ſome Timothees hand, ſo to continue from one
Biſhop to another vntil our time and to the end.}
\L{depoſitum}, auoiding the
\LNote{Profane nouelties.}{\L{Non dixit antiquitates} (ſaith Vincentius
Lirinenſis) \L{non dixit vetuſtates, ſed prophanas nouitates. Nam ſi
vitanda eſt nouitas, tenenda eſt antiquitas: ſi prophana eſt nouitas,
ſacrata eſt vetuſtas}; that is, \Emph{He ſaid not}, \Sc{Antiqvities}:
\Emph{he ſaid not}, \Sc{Ancientnes}: \Emph{but} \Sc{Prophane
Novelties}. \Emph{For if noueltie is to be auoided, antiquitie is to be
kept: if noueltie be profane, ancientnes is holy and ſacred.} See his
%%% !!! Cite?
whole booke againſt the profane nouelties of hereſies.

We
\MNote{Prophane nouelties of words how to be tried and examined.}
may not meaſure the newnes or oldnes of words and termes of ſpeaking in
religion, by holy Scriptures only: as though al thoſe or only thoſe
were new and to be reiected, that are not expreſly found in holy writ:
but we muſt eſteeme them by the agreablenes or diſagreablenes they
haue to the true ſenſe of Scriptures, to the forme of Catholike faith
and doctrine, to the phraſe of the old Chriſtians, to the Apoſtolike vſe
of ſpeach come vnto vs by tradition of al Ages and Churches, & to the
preſcription of holy Councels and Schooles of the Chriſtian world: which
haue giuen out (according to the time and queſtions raiſed by heretikes
and contentious perſons) very fit, artificial, and ſignificant words, to
diſcerne and defend the truth by, againſt falſhood.

Theſe
\MNote{Catholike termes not expreſly in the Scriptures, but in ſenſe,
are no ſuch nouelties of words.}
termes, \Emph{Catholike, Trinitie, Perſon, Sacrament, Incarnation,
Maſſe}, and many more, are not (in that ſenſe wherein the Church vſeth
them) in the Scriptures at al, and diuers of them were ſpoken by the
Apoſtles before any part of the new Teſtament was written, ſome of them
taken vp ſtraight after the Apoſtles daies in the writings and
preachings of holy Doctours, and in the ſpeach of al faithful people,
and therfore can not be counted Nouelties of words. Others beſide
theſe, as, \Emph{Conſubſtantial, Deipara, Tranſſubſtantiation}, & the
like, which are neither in expreſſe termes found in Scriptures, nor yet
in ſenſe (if we ſhould follow the iudgement of the ſpecial Sects againſt
Nicene Councel, for the firſt; the Neſtorians againſt the Epheſine
Councel, for the ſecond; the Lutherans and Caluiniſts againſt the
Lateran and the later Councels, for the third) theſe words alſo
notwithſtanding, by the iudgement of holy Church, and Councels approued
to be conſonant to God's word, and made authentical among the faithful,
are ſound and true words, and not of thoſe kind which the Apoſtle
calleth \Emph{Nouelties}.

Theſe
\MNote{Heretical nouelties of words.}
words then here forbidden, are the new prophane termes and ſpeaches
inuented or ſpecially vſed by heretikes, ſuch as S.~Irenee recordeth the
Valentinians had a number moſt monſtrous: as the Manichees had alſo
diuers, as may be ſeen in
%%% !!! Cite?
S.~Auguſtin. The Arians had their
\TNote{\G{ὁμοιούσιον}}
\L{Similis ſubſtantiæ}, and Chriſt to be \L{ex non exiſtentibus}: the
other heretikes after thoſe daies had their
\TNote{\G{Χριϛοτοκον}}
\L{Chriſtiparam}, and ſuch like, agreable to their Sects.
\MNote{The Proteſtants prophane nouelties of words.}
But the Proteſtants paſſe in this kind, as they exceed moſt heretikes in
the number of new opinions: as their \L{Seruum arbitrium}, their
\Emph{ſole faith}, their \Emph{fiduce}, their \Emph{apprehenſion of
Chriſtes iuſtice}, their \Emph{imputatiue righteouſnes}: their horrible
termes of terrours, anguiſhes, diſtreſſes, diſtruſt, feares and feeling
of hel paines in the ſoule of our Sauiour, to expreſſe their blaſphemous
fiction of his temporal damnation, which they cal his deſcending to
hel: Their \Emph{markes, tokens}, and \Emph{badges Sacramental}, their
\Emph{Companation, Impanation, Circumpanation}, to auoid the true
conuerſion in the Euchariſt: their preſence \Emph{in figure, in faith,
ſigne, ſpirit, pleadge, effect}, to auoid the real preſence of Chriſtes
body. Theſe and ſuch like innumerable which they occupie in euery part of
their falſe doctrine, are in the ſenſe that they vſe them, al falſe,
captious, and deceitful words, and are \L{nouitates vocum} here
forbidden.

And though ſome of the ſaid termes haue been by ſome occaſion obiter
without il meaning ſpoken by Catholikes before theſe Heretikes aroſe,
yet now knowing them to be the proper ſpeaches of Heretikes, Chriſtian
men are bound to auoid them. Wherein the Church of God hath euer been as
diligent to reſiſt Nouelties of words, as her Aduerſaries are buſy to
inuent them.
\MNote{Catholikes muſt abhorre from heretical phraſes & words.}
For which cauſe ſhe wil not haue vs communicate with them, nor follow
their faſhion and phraſe newly inuented, though in the nature of the
words ſometime there be no harme. In S.~Auguſtines daies when Chriſtian
men had any good befallen them, or entred into any man's houſe, or met
any freind by the way, they vſed alwaies to ſay, \L{Deo gratias}. The
Donatiſtes and Circumcellians of that time being new-fangled, forſooke
the old phraſe, and would alwaies ſay, \L{Laus Deo}: from which the
Catholike men did ſo abhorre (as the ſaid Doctour
\CNote{\Cite{in Pſ.~132.}}
writeth) that they had as leefe met a theefe as one that ſaid to them,
\L{Laus Deo}, inſteed of \L{Deo gratias}. As now we Catholikes muſt not
ſay, \Emph{The Lord}, but, \Emph{Our Lord}: as we ſay, \Emph{Our Lady},
for his mother, not, \Emph{The Lady}. Let vs keep our forefathers words,
and we ſhal eaſily keep our old and true faith that we had of the firſt
Chriſtians. Let them ſay, \Emph{Amendment, abſtinence, the Lordes
Supper, the Communion table, Elders, Miniſters, Superintendent,
Congregation, ſo be it, praiſe ye the Lord, Morning-Praier,
Euening-Praier}, and the reſt, as they wil: Let vs auoid theſe Nouelties
of words, according to the Apoſtles preſcript, and keep the old termes,
\Emph{Penance, Faſting, Prieſt, Church, Bishop, Maſſe, Mattins,
Euenſong, the B.~Sacrament, Altar, Oblation, Hoſt, Sacrifice, Alleluia,
Amen, Lent, Palme-Sunday, Chriſtians}, and the very words wil bring vs
to the faith of our firſt Apoſtles, and condemne theſe new  Apoſtataes
new faith and phraſes.}
profane
\TNote{\G{καιvοφωvιας}
S.~Chryſoſtom.}
nouelties of voices, and oppoſitions of
\LNote{Falſely called knowledge.}{It
\MNote{Heretikes arrogate knowledge falſely ſo called.}
is the propertie of al Heretikes to arrogate to themſelues great
knowledge, and to condemne the ſimplicitie of their Fathers, the holy
Doctours, and the Church. But the Apoſtle calleth their pretended skil,
a knowledge falſely ſo called, being in truth high and deep blindnes.
\Emph{Such} (ſaith S.~Irenæus
\Cite{lib.~5. c.~17.)}
\Emph{as forſake the preaching of the Church, argue the holy Prieſts of
vnskilfulnes, not conſidering how farre more worth a religious idiote
is, then a blaſphemous and impudent ſophiſter, ſuch as al Heretikes be.}
And againe Vicentius Lirinenſis ſpeaking in the perſon of Heretikes
ſaith, \Emph{Come, ô ye foolish and miſerable men, that are commonly
called Catholikes, and learne the true faith which hath  been hid many
Ages heretofore, but is reuealed & shewed of late, &c.} See
%%% !!! Better cite?
\Cite{his whole booke concerning theſe matters.}}
falſely called knowledge. \V Which certaine promiſing, haue erred about
the faith. Grace
\Fix{b}{be}{obvious typo, fixed in other}
with thee. Amen.



\stopChapter


\stopcomponent


%%% Local Variables:
%%% mode: TeX
%%% eval: (long-s-mode)
%%% eval: (set-input-method "TeX")
%%% fill-column: 72
%%% eval: (auto-fill-mode)
%%% coding: utf-8-unix
%%% End:

