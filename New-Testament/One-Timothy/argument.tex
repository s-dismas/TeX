%%%%%%%%%%%%%%%%%%%%%%%%%%%%%%%%%%%%%%%%%%%%%%%%%%%%%%%%%%%%%%%%%
%%%%
%%%% The (original) Douay Rheims Bible 
%%%%
%%%% New Testament
%%%% Epistles
%%%% One Timothy
%%%% Argument
%%%%
%%%%%%%%%%%%%%%%%%%%%%%%%%%%%%%%%%%%%%%%%%%%%%%%%%%%%%%%%%%%%%%%%




\startcomponent argument


\project douay-rheims


%%% 2802
%%% o-2663
\startArgument[
  title={\Sc{The Argvment of the First Epistle of S.~Pavl to Timothee.}},
  marking={Argument of One Timothee}
  ]


After the Epiſtles to the Churches, now follow his Epiſtles to
particular perſons, as to Timothee, to Titus, who were Bishops; and to
Philemon.

Of Timothee we read
\XRef{Act.~16.}
how S.~Paul in his viſitation took him in his traine at Lyſtra,
circumciding him before, becauſe of the Iewes. He was then a Diſciple,
that is to ſay, a Chriſtian man. Afterward the Apoſtle gaue him holy
Orders, and conſecrated him Bishop, as he teſtifieth in both theſe
Epiſtles vnto him.
\XRef{1.~Tim.~4. v.~14.}
and
\XRef{2.~Tim.~1. v.~6.}

He writeth therfore vnto him as to a Bishop, and himſelf expreſſeth the
ſcope of his firſt Epiſtle, ſaying:
\CNote{\XRef{1.~Timoth.~3.}}
\Emph{Theſe things I write to thee, that thou maieſt know how thou
oughteſt to conuerſe in the Houſe of God, which is the Church.} And ſo
he inſtructeth him, (and in him, al Bishops) how to gouerne both
himſelf, and others. And touching himſelf, to be an example and a
ſpectacle to al ſorts, in al vertue. As touching others, to prohibit al
ſuch as goe about to preach otherwiſe then the Catholike Church hath
receiued, and to inculcate to the people the Catholike faith: to preach
vnto yong and old, men and women: to ſeruants, to the rich, to euery
ſort conueniently. With what circumſpection to giue orders, and to what
perſons: for whom to pray: whom to admit to the vow of widowhood, &c.

This Epiſtle was written, as it ſeemeth, after his firſt impriſonment in
Rome, when he was diſmiſſed and ſet at libertie. And therupon it is,
that he might ſay here:
\CNote{\XRef{1.~Timoth.~3.}}
\Emph{I hope to come to thee quickly}, to wit vnto Epheſus, where
\CNote{\XRef{1.~Tim.~1.}}
he had deſired him to remaine. Although in his voiage to Hieruſalem,
before his being at Rome, he ſaid at Miletum to the Clergie of Epheſus,
vpon probable feare:
\CNote{\XRef{Act.~20. v.~25.~38.}}
\Emph{And now behold I know, that you ſhal no more ſee my face.}

Where it was written, it is vncertaine: though it be commonly ſaid, at
Laodicia. Which ſeemeth not, becauſe it is like he was neuer there, as
may be gathered by the
\CNote{\XRef{Col.~2,~1.}}
Epiſtle to the Coloſsians, written at Rome in his laſt trouble, when he
was put to death.


\stopArgument


\stopcomponent


%%% Local Variables:
%%% mode: TeX
%%% eval: (long-s-mode)
%%% eval: (set-input-method "TeX")
%%% fill-column: 72
%%% eval: (auto-fill-mode)
%%% coding: utf-8-unix
%%% End:
