%%%%%%%%%%%%%%%%%%%%%%%%%%%%%%%%%%%%%%%%%%%%%%%%%%%%%%%%%%%%%%%%%
%%%%
%%%% The (original) Douay Rheims Bible 
%%%%
%%%% New Testament
%%%% One Timothy
%%%% Chapter 01
%%%%
%%%%%%%%%%%%%%%%%%%%%%%%%%%%%%%%%%%%%%%%%%%%%%%%%%%%%%%%%%%%%%%%%




\startcomponent chapter-01


\project douay-rheims


%%% 2803
%%% o-2664
\startChapter[
  title={Chapter 1}
  ]

\Summary{He recommendeth vnto him, to inhibit certaine Iewes who iangled
  of the law as though it were contrarie to his preaching. 11.~Againſt
  whom he auoucheth his miniſterie, though he acknowledge his
  vnworthines.}

Paul an Apoſtle of \Sc{Iesvs} Chriſt according to the commandment of God
our Sauiour, and of Chriſt \Sc{Iesvs} our hope: \V to Timothee his
beloued ſonne in the faith. Grace, mercie, and peace from God the
Father, and from Chriſt \Sc{Iesvs} our Lord.

\V As I deſired thee to remaine at Epheſus when I went into Macedonia,
that thou ſhouldeſt denounce to certaine
\LNote{Not to teach otherwiſe.}{The
\MNote{Teaching otherwiſe then the doctrine receiued, is a ſpecial marke
of Heretikes.}
proper marke of Heretikes and falſe Preachers is to teach otherwiſe or
contrarie to that which they found taught and beleeued generally in the
vnitie of the Catholike Church before their time: al doctrine that is
odde, ſingular, new, differing from that which was firſt planted by the
Apoſtles, and deſcended downe from them to al Nations and Ages following
without contradiction, being aſſuredly erroneous.
\TNote{\G{ἑτεροδιδασκαλεῖν}}
The Greek word which the Apoſtle here vſeth, expreſſeth this point
ſo effectually, that in one compound terme he giueth vs to wit, that an
Heretike is nothing els but an after-teacher, or teacher
otherwiſe. Which euen it-ſelf alone is the eaſieſt rule euen for the ſimple
to diſcerne a falſe Prophet or Preacher by, ſpecially when an hereſie
firſt beginneth.
\MNote{Luthers teaching otherwiſe.}
Luther found al Nations Chriſtian at reſt and peace in one vniforme
faith, and al Preachers of one voice and doctrine touching the
B.~Sacrament and other Articles: ſo that whatſoeuer he taught againſt
that which he found preached and beleeued, muſt needs be another
doctrine, a later doctrine, an after-teaching or teaching-otherwiſe, and
therfore conſequently muſt needs be falſe. And by this admonition of
S.~Paul, al Biſhops are warned to take heed of ſuch, and ſpecially to
prouide that no ſuch odde Teachers ariſe in their dioceſes.}
not to teach otherwiſe, \V nor to attend
\LNote{To fables.}{He ſpeaketh ſpecially of the Iewes after-doctrines
and humane conſtitutions repugnant to the lawes of God, whereof Chriſt
giueth warning
\XRef{Mt.~23.}
and in other places, which are conteined in their Cabala and Talmud:
\MNote{Al heretical doctrine is fables.}
generally of al heretical doctrines, which indeed, as we may ſee in the
Valentinians, Manichees, and other of old: by the Brethren of loue,
Puritans, Anabaptiſtes, and Caluiniſtes of our time. For which cauſe
Theodoret entitleth his book againſt Heretikes,
\Cite{Hæreticarum fabularum}
\Emph{Of Heretical fables.}}
to fables and genealogies hauing no end: which Miniſter
\LNote{Queſtions.}{Let
\MNote{Curious queſtioning in religion.}
our louing Brethren conſider whether theſe contentions and curious
queſtionings & diſputes in religion, which theſe vnhappie hereſies haue
ingendered, haue brought forth any increaſe of good life, any deuotion,
or edification of faith and religion in our daies, and then ſhal they
eaſily iudge of the truth of theſe new opinions, and the end that wil
follow of theſe innouations. In truth al the world now ſeeth they edifie
to Atheiſme and no otherwiſe.}
queſtions rather then the edifying of God which is in faith. \V But
\LNote{The end charitie.}{Here
\MNote{Charitie the very formal cauſe of our iuſtification.}
againe it appeareth, that Charitie is the cheefe of al vertues, and the
end, conſummation, and perfection of al the law and precepts. And yet
the Aduerſaries are ſo fond as to preferre faith before it, yea to
exclude it from our iuſtification. Such obſtinacie there is in them
that haue once in pride and ſtubbernes forſaken the euident
truth. Charitie doubtles which is here commended, is iuſtice it-ſelf,
and the very formal cauſe of our iuſtification, as the workes proceeding
therof, be the workes of iuſtice. \L{Charitas inchoata} (ſaith
S.~Auguſtin) \L{inchoata iuſtitia: charitas prouecta, prouecta iuſtitia:
Charitas magna, magna iuſtitia: Charitas perfecta, perfecta iuſtitia
eſt.} \Emph{Charitie now beginning, is iuſtice beginning: Charitie
growen or increaſed is iuſtice growen or increaſed: great Charitie, is
great iuſtice: perfect Charitie, is perfect iuſtice.}
\Cite{Li. de nat. & grat. c.~70.}}
the end of the precept is charitie from a pure hart, and
\SNote{S.~Auguſtin ſaith: He that liſt to haue the hope of Heauen: let
him look that he haue a good conſcience. To haue a good conſcience, let
him beleeue and worke wel. For that he beleueth, he hath of faith; that
he worketh, he hath of charitie.
\Cite{Præfat. in Pſ.~31.}}
a good conſcience, and a faith not feined. \V From the which things
certaine ſtraying, are turned into
\TNote{\G{ματαιολογίαν}}
vaine-talke, \V
\LNote{Deſirous to be Doctours.}{It
\MNote{Heretikes great boaſters, but vnlearned.}
is the proper vice both of Iudaical & of Heretical falſe Teachers, to
profeſſe knowledge and great skil in the Law and Scriptures, being
indeed in the ſight of the learned moſt ignorant of the word of God, not
knowing the very principles of diuinitie, euen to the admiration truely
of the learned that read their books, or heare them preach.}
deſirous to be Doctours of the Law, not vnderſtanding neither what
things they ſpeake, nor of what they affirme. \V But we know that
\CNote{\XRef{Ro.~7,~18.}}
the Law is good, if a man vſe it lawfully: \V knowing this, that
\LNote{The law not made to the iuſt.}{By
\MNote{Libertines alleadge Scripture.}
this place and the like, the Libertines of our daies would diſcharge
themſelues (whom they count iuſt) from the obedience of lawes. But the
Apoſtles meaning is that the iuſt man doth wel, not as compelled by law
or for feare of puniſhment due to the tranſgreſſours thereof, but of
grace and mere loue toward God and al goodnes, moſt willingly, though
there were no law to command him.}
the Law is not made to the iuſt man, but to the vniuſt, & diſobedient,
to the impious & ſinners, to the wicked & contaminate, to killers of
fathers & killers of mothers, to murderers, \V to fornicatours, to lyers
with mankind, to man-ſtealers, to liers, to periured perſons, and what
other thing ſoeuer is contrarie to ſound doctrine, \V which
%%% o-2665
is according to the Ghoſpel of the glorie of the bleſſed God, which is
committed to me.

\V I giue him thankes which hath ſtrengthned me, Chriſt \Sc{Iesvs} our
Lord, becauſe he hath eſteemed me faithful, putting me in the
miniſterie. \V Who before was blaſphemous and a perſecutour and
contumelious. But I obteined the mercie of God, becauſe I did it being
ignorant in incredulitie. \V And the grace of our Lord ouer-abounded
with faith and loue, which is in Chriſt \Sc{Iesvs}. \V A faithful
ſaying, and worthie of al acceptation, that Chriſt \Sc{Iesvs} came into
this world
\CNote{\XRef{Mat.~9,~13.}
\XRef{Mr.~2,~17.}}
to ſaue ſinners, of whom I am the cheefe. \V
%%% 2804
But therfore haue I obtained mercie: that in me firſt of al
Chriſt \Sc{Iesvs} might ſhew al patience,
\TNote{\G{πρὸς ὑποτύπωσιν}}
to the information of them that ſhal beleeue on him vnto life
euerlaſting. \V And to the King of the worlds, immortal, inuiſible, only
God, honour & glorie for euer and euer. Amen.

\V This precept I cõmend to thee, ô Timothee: according to the
prophecies going before
\TNote{\G{ἐπὶ σὲ}}
on thee, that thou warre in them a good warfare, \V hauing faith and a
good conſcience,
\SNote{Euil life and no good conſcience is often the cauſe that men fal
to Hereſie from the faith of the Catholike Church. Againe, this plainely
reproueth the Heretikes falſe doctrine, ſaying, that no man can fal from
the faith that he once truely had.}
which certaine repelling haue made ſhip-wrack about the faith. \V Of
whom is Hymenæus & Alexander: whom I haue
\LNote{Deliuered to Satan.}{Hymenæus
\MNote{Excommunication of Heretikes, and the effect therof.}
and Alexander are here excommunicated for falling from their faith and
teaching hereſie: an example vnto Biſhops to vſe their ſpiritual power
vpon ſuch. In the primitiue Church, corporal affliction through the
miniſterie of Satan was ioyned to excommunication. Where we ſee alſo the
diuels readines to inuade them that are caſt out by excommunication,
from the fellowſhip of the faithful, and the ſupereminent power of
Biſhops in that caſe. Wherof S.~Hierom
\Cite{(ep.~1. ad Heſiod. c.~7.)}
hath theſe memorable words: \Emph{God forbid} (ſaith he) \Emph{I should
ſpeake ſiniſtrouſly of them, who ſucceeding the Apoſtles in degree, make
Chriſtes body with their holy mouth, by whom we are made Chriſtians: who
hauing the keies of heauen, doe after a ſort iudge before the day of
iudgement: who in ſobrietie and Chaſtitie haue the keeping of the ſpouſe
of Chriſt.} And a litle after,
\MNote{The Prieſts high authoritie of Excommunication.}
\Emph{They may deliuer me vp to Satan, to the deſtruction of my flesh,
that the ſpirit may be ſaued in the day of our Lord Ieſus. And in the
old Law whoſoeuer was diſobedient to the Prieſts, was either caſt out of
the camp and ſo ſtoned of the people, or laying downe his neck to the
ſword, expiated his offenſe by his bloud:
\MNote{The terrible effects therof.}
but now the diſobedient is
cut-off with the ſpiritual ſword, or being caſt out of the Church, is
torne by the furious mouth of diuels.} So ſaith he. Which words would
God euery Chriſtian man would weigh.}
deliuered to Satan, that they may learne not to blaſpheme.


\stopChapter


\stopcomponent


%%% Local Variables:
%%% mode: TeX
%%% eval: (long-s-mode)
%%% eval: (set-input-method "TeX")
%%% fill-column: 72
%%% eval: (auto-fill-mode)
%%% coding: utf-8-unix
%%% End:

