%%%%%%%%%%%%%%%%%%%%%%%%%%%%%%%%%%%%%%%%%%%%%%%%%%%%%%%%%%%%%%%%%
%%%%
%%%% The (original) Douay Rheims Bible 
%%%%
%%%% New Testament
%%%% Epistles
%%%% One Timothy
%%%% Chapter 02
%%%%
%%%%%%%%%%%%%%%%%%%%%%%%%%%%%%%%%%%%%%%%%%%%%%%%%%%%%%%%%%%%%%%%%




\startcomponent chapter-02


\project douay-rheims


%%% 2805
%%% o-2666
\startChapter[
  title={Chapter 2}
  ]

\Summary{By his Apoſtolike authoritie he appointeth publike praiers to
  be made for al men without exception: 8.~alſo men to pray in al
  places: 9.~and women alſo in ſeemly attire, 11.~to learne of men, and
  not to be Teachers in any wiſe, but to ſeeke ſaluation by that which
  to them belongeth.}

I deſire therfore firſt of al things that
\LNote{Obſecrations.}{This
\MNote{The praiers and petitions in the Maſſe, deduced out of the
Apoſtles wordes by S.~Auguſtin & other fathers.}
order of the Apoſtle S.~Auguſtin
\Cite{(ep.~59.)}
findeth to be fulfilled ſpecially in the holy celebration of the Maſſe,
which hath al theſe kinds, expreſſed here in foure diuers words
pertaining to foure ſorts of praiers. The difference wherof he exactly
ſeeketh out of the proper ſignification and difference of the Greek
words. And he teacheth vs that the firſt kind of prayers which here be
called,
\TNote{\G{δεήσεις}}
\Emph{Obſecrations}, are thoſe that the Prieſt ſaith before the
conſecration: that the ſecond called,
\TNote{\G{προσευχάς}}
\Emph{Praiers}, be al thoſe which are ſaid in and after the
Conſecration, and about the Receiuing, including ſpecially the
\L{Pater noſter}, wherwith the whole Church (ſaith he) in a manner
endeth that part, as S.~Hierom alſo affirmeth, that
\MNote{\Sc{Pater noster} in the Maſſe.}
Chriſt taught his Apoſtles to vſe the \L{Pater noſter} in the Maſſe.
\L{Sic docuit}, &c. \Emph{So taught he his Apoſtles, that daily in the
Sacrifice of his body the faithful should be bold to ſay, Pater noſter
&c.}
\Cite{Li.~3. cont. Pelag. cap.~5.}
where he alludeth to the very words now vſed in the preface to the ſaid
\L{Pater noſter} in the ſaid Sacrifice, \L{audemus dicere, Pater
noſter}. The third ſort called here in the text,
\TNote{\G{ἐντεύξεις}}
\Emph{Poſtulations}, be thoſe which are vſed after the Communion, as it
were for dimiſſing of the people with benediction, that is, with the
Biſhops or Prieſts bleſſing. Finally the laſt kind, which is
\TNote{\G{εὐχαριστίας}}
\Emph{Thankes-giuings}, concludeth al,
\CNote{Theo. in hunc. loc.}
when the Prieſt and people giue thankes to God for ſo great a myſterie
then offered & receiued. Thus the ſaid holy father handleth this text.
\Cite{ep.~59. to Paulinus.}

S.~Epiphanius alſo inſinuateth theſe words of the Apoſtle to pertaine to
the Liturgie or Maſſe, when he thus writeth to Iohn Biſhop of
Hieruſalem. \Emph{When we accomplish our praiers after the rite of the
holy Myſteries, we pray both for al others, and for al thee alſo.}
\Cite{ep.~60. c.~2. ad Io. Hieroſolim. apud Hieronymum.}
\MNote{Praier in the Maſſe for Kings and others.}
And moſt of the other Fathers expound the foreſaid words, of publike
praiers made by the Prieſt, which are ſaid in al Liturgies or Maſſes
both Greek and Latin, for the good eſtate of al that be in high
dignitie, as Kings and others. See
\Cite{S.~Cryſ. ho.~6. in 1.~Tim.}
&
\Cite{S.~Ambr. in hunc loc.}
\Cite{Proſper de vocat. li.~1. c.~4.}
So exactly doth the practiſe of the Church agree with the precepts of
the Apoſtle and the Scriptures, and ſo profoundly doe the holy Fathers
ſeeke out the proper ſenſe of the Scriptures, which our Proteſtants doe
ſo prophanely, popularly, and lightly skim ouer, that they can neither
ſee nor endure the truth.}
obſecrations, praiers, poſtulations, thankes-giuings be made for al
men, \V
\SNote{Euen for heathen Kings and Emperours by whom the Church ſuffereth
perſecution: much more for al faithful Princes and Powers and people
both ſpiritual and temporal, for whom as members of Chriſtes body, &
therfore ioyning in praier and oblation with the Miniſters of the
Church, the Prieſtes more properly and particularly offer the holy
Sacrifice. See
\Cite{S.~Auguſtin de orig. anima li.~1. c.~9.}}
for Kings and al that are in preeminence: that we may lead a quiet and a
peaceable life in al pietie and chaſtitie. \V For this is good and
acceptable before our Sauiour God,
%%% o-2667
\V
\LNote{Who wil al men.}{The
\MNote{God wil no mans perdition but the ſaluation of al.}
perishing or damnation of men muſt not be imputed to God, who delighteth
not in any 
man's perdition, but hath prouided a general medicine and redemption to
ſaue al from periſhing that wil accept it, or that haue it applied vnto
them by his Sacraments and other meanes by him ordained, and ſo would
haue al ſaued by his conditional wil and ordinance: that is, if men wil
themſelues, by accepting, doing, or hauing done vnto them al things
requiſit by God's law. For God vſeth not his abſolute wil or power
towards al in this caſe. But he that liſt ſee the manifold ſenſes (al
good and true) that theſe words may beare, let him ſee S.~Auguſtin,
\Cite{ad articul. ſibi falſo impoſ. reſp.~2. to.~7.}
\Cite{Ench. c.~103.}
\Cite{Ep.~107.}
\Cite{Do. cor. & grat. c.~15.}
and
\Cite{S.~Damaſcene li.~2. de orthod. fide. c.~29.}}
who wil al men to be ſaued, and to come to the knowledge of the
truth. \V For there is one God,
\LNote{One Mediatour.}{The Proteſtants are too peeuiſh and pitifully
blind, that charge the Catholike Church and Catholikes, with making more
Mediatours then one, which is Chriſt our Sauiour, in that they deſire
the Saints to pray for them, or to be their patrones and interceſſours
before God.
\MNote{How there is but one Mediatour, Chriſt, & what it is to be ſuch a
Mediatour.}
We tel them therfore that they vnderſtand not what it is to be a
Mediatour, in this ſenſe that S.~Paul taketh the word, and in which it
is properly and only attributed to Chriſt. For, to be thus a Mediatour,
is,
\CNote{\Cite{Aug. li.~9. de Ciu. ca.~15.}
\Cite{De fid. ad Pet. c.~2.}}
by nature to be truely both God and man, to be that one eternal Prieſt
and Redeemer, which by his Sacrifice and death vpon the Croſſe hath
reconciled vs to God, and paied his bloud as a ful and ſufficient ranſom
for al our ſinnes, himſelf without need of any redemption,  neuer ſubiect
to poſſibilitie of ſinning: againe, to be the ſingular Aduocate and
Patrone of mankind, that by himſelf alone and by his owne merits
procureth al grace and mercie to mankind in the ſight of his Father,
none making any interceſſion for him, nor giuing any grace or force to
his praiers, but he to al: none asking or obtaining either grace in this
life, or glorie in the next, but by him. In this ſort then (as
S.~Auguſtin truely ſaith,
\Cite{Cont. ep. Parm. lib.~2. c.~8.)}
neither Peter nor Paul, no not our B.~Lady, nor any creature
whatſoeuer, can be our Mediatour. The Aduerſaries thinke to baſely of
Chriſtes mediation, if they imagin this to be his only prerogatiue, to
pray for vs, or that we make the Saints our Mediatours in that ſort as
Chriſt is, when we deſire them to pray for vs. Which is ſo farre
inferiour to the ſingular mediation of him, that no Catholike euer can
or dare thinke or ſpeake ſo baſely vnto him, as to deſire him to pray
for vs:
\MNote{The different manner of praying to Chriſt, and to Saints.}
but we ſay,
\TNote{\GG{Kyrie eleiſon, Chriſte eleiſon.}}
\Emph{Lord haue mercie vpon vs, Chriſt haue mercie vpon vs}: and not,
\Emph{Chriſt pray for vs}, as we ſay to our Ladie and the
reſt. Therfore to inuocate Saints in that ſort as the Catholike Church
doth, can not make them our Mediatours as Chriſt is, whom we muſt not
inuocate in that ſort. And as wel make we the faithful yet liuing, our
Mediatours (by the Aduerſaries arguments) when we deſire their praiers,
as the departed Saints.

But
\MNote{How there be many Mediatours, as there be many Sauiours, and
Redeemers, euen in the Scriptures.}
now touching the word, \Emph{Mediatour}, though in that ſingular ſenſe
proper to our Sauiour, it agreeth to no mere creature in Heauen or
earth, yet taken in more large and common ſort by the vſe of Scriptures,
Doctours, and vulgar ſpeach, not only the Saints, but good men liuing,
that pray for vs and help vs in the way of ſaluation, may and are
rightly called Mediatours. As
\Cite{S.~Cyril li.~12. Theſaur. c.~10.}
proueth, that Moyſes according to the Scriptures, and Ieremie, and the
Apoſtles, and others be Mediatours. Read his owne words, for they
plainely refute al the Aduerſaries cauillations in this caſe. And if the
name of
\CNote{\XRef{Iud.~3,~9.}
\XRef{2.~Eſd.~9,~17.}
\XRef{Act.~7,~35.}}
Sauiour and Redeemer be in the Scriptures giuen to men, without
derogation to him that is in a more excellent and incomparable manner
the only Sauiour of the world: what can they ſay, why there may not be
many Mediatours, in an inferiour degree to the only and ſingular
Mediatour? S.~Bernard ſaith, \L{Opus eſt mediatore ad Mediatorem
Chriſtum, nec alter nobis vtilior Maria}; that is, \Emph{We haue need of
a mediatour to Chriſt the Mediatour, and there is none more for our
profit then our Ladie.} Bernard Serm qui incipit, \L{Signum magnam
apparuit &c.}
\Cite{Poſt. Ser.~5. de Aſſumpt.}
S.~Baſil alſo in the ſame ſenſe, writing to Iulian the Apoſtata, deſireth
the mediation of our Ladie, of the Apoſtles, Prophets and Martyrs, for
procuring of God's mercie and remiſſion of his ſinnes. His words are
cited in
\Cite{Con. Nic.~2. Act.~4. pag.~110. &~111.}
Thus did and thus beleeued al the holy Fathers, moſt agreably to the
Scriptures, and thus muſt al the children of the Church doe, be the
Aduerſaries neuer ſo importunate and wilfully blind in theſe matters.}
one alſo Mediatour of God and men, man Chriſt \Sc{Iesvs}: \V who gaue
himſelf a redemption for al,
\Var{whoſe teſtimonie}{a teſtimonie}
in due times is confirmed. \V
\CNote{\XRef{2.~Tim.~1,~11.}}
Wherin I am appointed a Preacher and an Apoſtle (I ſay the truth, I lie
not) Doctour of the Gentils in faith and truth.

\V I wil therfore that men pray in euery place: lifting vp pure hands,
without anger and altercation. \V In like manner
\CNote{\XRef{1.~Pet.~3,~3.}}
women alſo in comely attire: with demurneſſe and ſobrietie adorning
themſelues, not in plaited haire, or gold, or pretious ſtones, or
gorgeous apparel, \V but that which becommeth women profeſſing pietie by
good workes. \V Let a woman learne in ſilence, with al ſubiection. \V
But
\CNote{\XRef{1.~Cor.~14,~34.}}
to teach
\LNote{I permit not.}{In
\MNote{Women great talkers of Scripture, and promoters of hereſie.}
times of licentiouſnes, libertie, and hereſie, women are much giuen to
reading, diſputing, chatting, and iangling of the holy Scriptures, yea
and to teach alſo if they might be permitted. But S.~Paul vtterly
forbiddeth it, & the
\CNote{\Cite{S.~Chryſ. Ho.~9. in 1.~Tim.}}
Greek Doctours vpon this place note that the woman taught but once, that
was when after her reaſoning with Satan, ſhe perſuaded her husband to
tranſgreſſion, and ſo ſhe vndid al mankind. And in the Eccleſiaſtical
Writers we find that women haue been great promoters of euery ſort of
hereſie (wherof ſee a notable diſcourſe in S.~Hierom
\Cite{Ep. ad Creſibp. cont. Pelag. c.~2.)}
which they would not haue done, if they had according to the Apoſtles
rule, followed pietie and good workes, and liued in ſilence and
ſubiection to their husbands.}
I permit not vnto a woman, nor to haue dominiõ ouer the man: but to be
in ſilence. \V For
\CNote{\XRef{Gen.~1,~27.}
\XRef{3,~6.}}
Adam was formed firſt; then Eue. \V And Adam was not ſeduced: but the
woman being ſeduced, was in preuarication. \V Yet she shal be ſaued by
generation of children: if
\Var{they}{she}
continue in faith and loue and ſanctification with ſobrietie.


\stopChapter


\stopcomponent


%%% Local Variables:
%%% mode: TeX
%%% eval: (long-s-mode)
%%% eval: (set-input-method "TeX")
%%% fill-column: 72
%%% eval: (auto-fill-mode)
%%% coding: utf-8-unix
%%% End:

