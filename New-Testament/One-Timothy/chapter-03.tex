%%%%%%%%%%%%%%%%%%%%%%%%%%%%%%%%%%%%%%%%%%%%%%%%%%%%%%%%%%%%%%%%%
%%%%
%%%% The (original) Douay Rheims Bible 
%%%%
%%%% New Testament
%%%% One Timothy
%%%% Chapter 03
%%%%
%%%%%%%%%%%%%%%%%%%%%%%%%%%%%%%%%%%%%%%%%%%%%%%%%%%%%%%%%%%%%%%%%




\startcomponent chapter-03


\project douay-rheims


%%% 2808
%%% o-2669
\startChapter[
  title={Chapter 3}
  ]

\Summary{Of what qualitie they muſt be, whom he ordaineth Bishops,
  8.~and Deacons, 14.~and the cauſe of his writing to be, the
  excellencie of the Catholike Church, and of Chriſt, who is the obiect
  of our religion.}

A Faithful ſaying. If a man deſire a Biſhops office, he deſireth
\LNote{A good worke.}{\Emph{Nothing}
\MNote{The great charge, and great merit, of Eccleſiaſtical functions.}
(ſaith S.~Auguſtin) \Emph{in this life, and ſpecially in this time, is
eaſier, pleaſanter, or more acceptable to men, then the office of a
Bishop, Prieſt, or Deacon, if the thing be done only for fashion ſake,
and flatteringly: but nothing before God more miſerable, more
lamentable, more damnable.} Againe, \Emph{There is nothing in this life,
and ſpecially at this time, harder, more laborious, or more dangerous,
then the office of a Bishop, Prieſt, or Deacon: but before God nothing
more bleſſed, if they warre in ſuch ſort as our Captaine commandeth.}
\Cite{Auguſt. ep.~148.}}
a good worke. \V
\CNote{\XRef{Tim.~1,~6.}}
It behoueth therfore
\LNote{A Bishop.}{That
\MNote{The Apoſtle vnder the name of Biſhop inſtructeth Prieſtes alſo.}
which is here ſpoken of a Biſhop (becauſe the words Biſhop & Prieſt in
the new Teſtament be often taken indifferently for both or either of the
twaine, as is noted in an other place) the ſame is meant of euery Prieſt
alſo: though the qualities here required, ought to be more ſingular in
the Biſhop then in the Prieſt, according to the difference of their
degrees, dignities, and callings.}
a Biſhop to be irreprehenſible, the huſband
\LNote{Of one wife.}{Certaine
\MNote{The Heretikes opinion concerning Prieſts marriage.}
Biſhops of Vigilantius Sect (whether vpon falſe conſtruction of this
text, or through the filthines of their fleshly luſt) would take none to
the Clergie, except they would be married firſt, \Emph{not beleeuing}
(ſaith S.~Hierom
\Cite{aduert. Vigilant. c.~1.)}
\Emph{that any ſingle man liueth chaſtly, shewing how holily they liue
themſelues, that ſuſpect il of euery man, and wil not giue the
Sacrament} (of Order) \Emph{to the Clergie, vnles they ſee their wiues
haue great bellies and children wailing at their mothers breaſts.} Our
Proteſtants though they be of Vigilantius Sect, yet they are ſcarſe come
ſo farre, to command euery Prieſt to be married. Neuertheles they
miſlike them that wil not marrie, ſo much the worſe, & they ſuſpect il
of euery ſingle perſon in the Church, thinking the guift of chaſtitie to
be very rare among them; & they doe not only make the ſtate of marriage
equal to chaſt ſingle life, with the Heretike Iouinian, but they are
bold to ſay ſometimes, that the Biſhop or Prieſt may doe his duety and
charge better married, then ſingle: expreſly againſt S.~Paul,
\CNote{\XRef{1.~Cor.~7.}}
who affirmeth that the vnmarried thinke of the things that belong to
God, and that the married be diuerſely diſtracted and intangled with the
world.

The
\MNote{S.~Paules place, \Emph{of one wife}, excludeth bigamos frõ holy
Orders.}
Apoſtle then, by this place we now treat of, neither commandeth, nor
counſeleth, nor wiſheth, nor would haue Biſhops or Prieſts to marrie, or
ſuch only to be receiued as haue been married: but, that ſuch an one as
hath been married (ſo it were but once, and that to a virgin) may be
made Biſhop or Prieſt. Which is no more then an inhibition that none
hauing been twiſe married or being \L{bigamus}, should be admitted to that
holy Order. And this expoſition only is agreable to the practiſe of the
whole Church, the definition of ancient Councels, the doctrine of al the
Fathers without exception, and the Apoſtles tradition. Which ſenſe
S.~Chryſoſtom wholy followeth vpon the Epiſtle to Titus (though here he
follow not wholy the ſame ſenſe)
\Cite{Hom.~2. in Epiſt. ad Tit.}
S.~Ambroſe alſo
\Cite{vpon this place}
& moſt plainely and largely in his
\Cite{82.~Epiſtle poſt med.}
giuing the cauſe why \L{bigamus} can not be made Bishop or Prieſt, in
fine affirmeth not only the Apoſtle but the holy Councel of Nice to haue
takẽ order that none should be receiued into the Clergie, that were
twiſe married. S.~Hierom
\Cite{Epiſt.~83. ad Oceanum c.~2.}
&
\Cite{epiſt.~2. c.~18.}
\Cite{ep.~11. c.~2.}
expreſly writeth that the Clergie is made of ſuch as haue had but one
wife, at leaſt after Baptiſme: for he thought that if one were often
married when he was yet no Chriſtian, he might notwithſtanding be
ordered Bishop or Prieſt. But S.~Ambroſe
\Cite{ep.~82.}
S.~Auguſtin
\Cite{de bono Coniug. c.~18.}
S.~Innocentius the firſt
\Cite{ep.~2. c.~5,~6. to.~1.}
S.~Leo
\Cite{ep.~87.}
%%% !!! Inline? Others like this anywhere?
\CNote{\Cite{li.~2. ep.~25.}}
S.~Gregorie, and after them the whole Church, exclude thoſe alſo which
haue been twiſe married when ſo-euer. Whereof S.~Auguſtin giueth a
goodly reaſon and example in the place alleaged. S.~Leo
\Cite{ep.~87.}
addeth further,
\MNote{Who are counted \L{bigami}.}
and proueth that the man is counted \L{bigamus}, and not the huſbãd of
one wife, in reſpect of holy Orders, not only if he hath had two wiues,
but if his one wife were not a virgin.
\CNote{\XRef{Leuit.~22.}}
Which being obſerued in the high
Prieſts of the old law, muſt needs be much rather now. See alſo the book
\Cite{de Eccleſtiaſticis dogmatibus c.~71.}
in S.~Auguſtines workes.

And
\MNote{The heretical Clergie nothing regardeth the Apoſtles preſcriptiõ
of one wife.}
by theſe few you may ſee how shamefully the ſtate of the new heretical
Clergie of our time is fallen from the Apoſtolike and al the Fathers
practiſe and doctrine herein. Who doe not only take men once or twiſe
married before, but (which was neuer heard of before in any perſon or
part of the Catholike Church) they marrie after they be Bishops or
Prieſts, once, twiſe, and as often as their luſts require.
\MNote{None euer married after holy Orders.}
Whereas it was neuer lawful in God's Church to marrie after Holy
Orders. Neither is there one authentical example therof in the
world. For theſe of whom Nice Councel ſpeaketh, were married before, &
were but tolerated only to vſe their wiues: the Fathers in the ſame
Councel prouiding expreſly at the ſame time, that none from thence-forth
should marrie after they came to holy Orders, \Emph{and that according
to the ancient tradition of the Church}, as
\CNote{Socrat. li.~1. c.~8.}
Socrates and
\CNote{Sozom. li.~1. c.~22.}
Sozomenus declare in moſt plaine words. See Suidas
\Cite{in the word Paphnuſius.}
\MNote{They that were made Prieſts of married men, abſteined from their
wiues.}
And in what countrie ſo-euer they haue been permitted to haue carnal
dealing euen with their wiues whom they had before, it was not according
to the exact rule of the Apoſtles and Churches tradition, by which al
that be in holy Orders, should wholy abſtaine, not only from marrying,
but euen from their wiues before married. Whereof thus writeth
\MNote{S.~Epiphanius.}
S.~Epiphanius
\Cite{hareſ.~59. cont. Catharos.}
\Emph{The holy preaching of God receiueth not, after Chriſt, them that
marrie againe after their wiues departure, by reaſon of the great
dignitie and honour of Prieſthood. And this the holy Church of God
obſerueth with al ſinceritie. Yea she doth not receiue the once married
perſon that yet vſeth his wife and begetteth children: but only ſuch an
one she taketh to be Deacon, Prieſt, Bishop, or Subdeacon, as abſtaineth
from his one wife, or is a widower, ſpecially where the holy canons be
ſincerely kept.
\MNote{Marriage of Prieſts is contrarie to the ancient canons.}
But thou wilt ſay vnto me, that in certaine places
Prieſts, Deacons, and Subdeacons doe yet beget children}, (belike this
holy Father neuer heard of any Bishop that did ſo, and therfore he
leaueth out that order, which he named with the other in the former part
of the ſentence) \Emph{but that is not done according to order and rule,
but according to man's mind, which by time ſlacketh, and for the great
multitude} (of Chriſtian people) \Emph{when there were not found
ſufficient for the miniſterie, &c.} the reſt of his words be goodly for
that purpoſe.

Euſebius
\MNote{Euſebius.}
alſo
\Cite{Euang. demonſt. li.~1. c.~9.}
ſaith, that ſuch as be conſecrated to the holy miniſterie, ſhould
abſtaine wholy from their wiues which they had before.
\MNote{S.~Hierom.}
S.~Hierom
\Cite{Apolog. ad Pammach. c.~8.}
proueth, that ſuch of the Apoſtles as were married, did ſo, and that the
Clergie ought to doe the ſame by their example. Yea in his time he
teſtifieth
\Cite{(Cont. Vigil. c.~3.)}
that they did liue ſingle in a manner through the world euen in the Eaſt
Church alſo. \Emph{What}, ſaith he, \Emph{shal the Churches of the Eaſt
doe, what they of Ægypt, of the See Apoſtolike: which take to the
Clergie, either virgins, or the continent and vnmarried, or ſuch as, if
they haue wiues, ceaſe to be husbands?} And againe he ſaith in
\Cite{Apol. ad Pammach c.~3.}
(See alſo
\Cite{c.~8.)}
\Emph{If married men like not wel of this, let them not be angrie with
me, but with the holy Scriptures, with al Bishops, Prieſts, Deacons, &
the whole companie of Prieſts & Leuites, that know they can not offer
Sacrifices, if they vſe the act of marriage.}
\MNote{S.~Auguſtin.}
S.~Auguſt.
\Cite{de adult. Coniug. li.~2. c.~20.}
maketh it ſo plaine a matter that al Prieſts ſhould liue chaſt, that he
writeth, that euen ſuch as were forced (as many were in the primitiue
Church) to be of  the Clergie, were bound to liue chaſt,
\MNote{See S.~Leo
\Cite{ep.~92. c.~3.}}
yea and did it
with great ioy and felicitie, neuer complaining of theſe neceſſities and
intolerable burdens, or impoſſibilities of liuing chaſt, as our fleſhly
companie of new Miniſters and Superintendents doe now, that thinke it no
life without women. Much like to S.~Auguſtin before his conuerſion, when
he was yet a Manichee, who (as himſelf reporteth
\Cite{Confeſ. li.~6. c.~3.)}
admiring in
\MNote{S.~Ambroſe.}
S.~Ambroſe al other his incomparable excellencies, yet counted al his
felicities leſſe, becauſe he lacked a woman, without which he thought
(in time of his infidelitie) no man could liue. But after his conuerſion
thus he ſaid to God of S.~Ambroſe: \Emph{What hope he had, and againſt
the tentations of his excellencie what a fight he felt, or rather what a
comfort and ſolace in tribulation, and his ſecret mouth which was within
his hart, what ſauourie and ſweet ioyes it taſted of thy bread, neither
could I coniecture, neither had I tried.}

See
\MNote{Tertullian.}
Tertullian
\Cite{li.~1. ad vxorum}
\MNote{S.~Cyprian}
S.~Cyprian
\Cite{de ſingul. Cleritor.}
\MNote{Councels.}
the 
\Cite{firſt Councel of Nice can.~3. conc. Toles.~2.}
\Cite{can.~3. conc. Aurelian.~3.}
\Cite{of Carthage the ſecond cap.~2.}
\Cite{of Neocæſarea cap.~2.}
\Cite{of Ancyra cap.~10.}
and you ſhal find that this was generally the Churches order euen from
the Apoſtles time, though in ſome places by the licentiouſnes of many,
it was ſometime not ſo religiouſly looked vnto. Wherby you may eaſily
refute the impudent clamours of Heretikes againſt Siricius, Gregorie 7,
and others, whom they falſely make the Authours
\Fix{or}{of}{obvious typo, fixed in other}
the Clergies ſingle life.}
of one wife, ſober, wiſe, comely, chaſt, a man of hoſpitalitie, a
Teacher, \V not giuen to wine, no fighter, but modeſt, no quareler, not
couetous, \V wel ruling his owne houſe,
\SNote{He ſaith, \Emph{hauing children}, not \Emph{getting children}.
\Cite{S.~Ambr. Ep.~82.}}
hauing his children ſubiect with al chaſtitie. \V But if a man know not
to rule his owne houſe, how ſhal he haue care of the Church of God? \V
\LNote{Not a Neophyt.}{That
\MNote{None rashly to be admitted to the Clergie.}
which is ſpoken here properly & principally of the newly baptized (for
ſo the word Neophyt doth ſignifie) the Fathers extend alſo to al ſuch as
be but newly retired from prophane occupations, ciuil gouernment,
warfare, or ſecular ſtudies, of whom good trail muſt be taken before
they ought to be preferred to the high dignitie of Biſhop or
Prieſt. Though for ſome ſpecial prerogatiue & excellencie, it hath in
certaine perſons been otherwiſe, as in S.~Ambroſe and ſome other notable
men. Tertullian
\Cite{(li. de præſcript.)}
noteth Heretikes for their lightnes in admitting euery one without
diſcretion to the Clergie.
\MNote{Heretikes admit al ſorts without exception.}
\Emph{Their Orders} (ſaith he) \Emph{are rash, light, inconſtant: now
they place Neophytes, then ſecular men, then our Apoſtates, that they
may tie them by glorie and preferments, whom with the truth they can
not. Nowhere may a man ſooner proſper and come forward, then in the
camp of rebelles, where to be only, is to deſerue much. Therfore one to
day a Bishop, to morrow ſome-what els: to day a Deacon, to morrow
Lector, that is, a Reader: to day a Prieſt, to morrow a lay man, for to
laie men alſo they enioyne the functions of Prieſts.} And S.~Hierom
\Cite{ep.~8. ad Oceanum c.~4.}
ſaith of ſuch, \Emph{Yeſterday a Cathecumen or newly conuerted, to day a
bishop: yeſterday in the theatre, to day in the Church: at night in the
place of games and maiſteries, in the morning at the altar: a while agoe
a great patrone of ſtage-plaiers, now a conſecratour of holy virgins.}
And in another place, \Emph{Out of the boſome of Plato and Ariſtophanes
they are choſen to a Bishoprike, whoſe care is, not how to ſuck out the
marow of the Scriptures, but how to ſooth the peoples eares with
flourishing declamations.}
\Cite{Dialog. cont. Lucifer. c.~5.}}
Not
\SNote{\L{Neophytus} is he that was lately chriſtned or newly planted in
the myſtical body of Chriſt.}
\TNote{\G{νεόφυτον}}
a neophit: leſt puffed into pride, he fal into the iudgement of the
Diuel. \V And he muſt haue alſo good teſtimonie of them that are
without: that he fal not into reproch and the ſnare of the Diuel.

\V
%%% !!! LNote not marked in either
\LNote{Deacons.}{Vnder
\MNote{The three holy Orders, only bound to chaſtitie.}
the name of Deacons are here conteined Subdeacons, as before vnder the
name of Biſhops, Prieſts alſo were comprehended.
\CNote{\Cite{Leo. ep.~92. c.~3.}
\Cite{Greg. 6. li.~1. ep.~42.}}
For to theſe foure
pertaineth the Apoſtles precept and order touching one wife, & touching
continencie and chaſtitie, as by the alleaged Councels and Fathers
(namely by the words of S.~Epiphanius) doth appeare. For they only be in
holy Orders, as ſeruing by their proper function about the Altar and the
B.~Sacrament: in reſpect whereof the law of chaſtitie pertaineth to
them,
\MNote{The 4.~inferiour orders not bound to chaſtitie.}
and not to the foure inferiour Orders of \L{Acolyti, Exorciſtæ,
Lectores} and \L{Oftiarij}, who neither by precept nor vow be bound to
perpetual chaſtitie, as the others of the holy and high Orders be bound,
both by precept and promiſe or ſolemne aſſent made when they tooke
Subdeaconship.

Al
\MNote{Al the ſeuen Orders ancient, euẽ from Chriſt and the Apoſtles
time.}
theſe degrees and orders to haue been euer ſince Chriſtes time in the
Church of God, it might be proued by al
\Fix{iniquitie.}{antiquitie.}{obvious typo, fixed in other}
But for as much as the Apoſtles purpoſe is not here to recken vp al the
Eccleſiaſtical Hierarchie, it need not be treated of in this place. But
we wiſh the learned to read the
\Cite{3.~4.~5.~6.~7.~8.~9. chapters of the 4.~Councel of Carthage},
whereat S.~Auguſtin was preſent: where they ſhal ſee the expreſſe
callings, offices, and manner of ordering or creating al the ſaid ſorts,
and ſhal wel perceiue theſe things to be moſt ancient and venerable. Let
them read alſo Euſebius hiſtorie, the
\Cite{35.~Chapter of the 6.~booke,}
where for al theſe orders he reciteth Cornelius epiſtle to Fabius,
concerning Nouatus. Likewiſe S.~Cyprian in many places, namely
\Cite{ep.~55. nu.~1.}
Where ſee the notes vpon the ſame.
\Cite{S.~Hier. ep.~2. c.~6.}
Of Subdeacon there is mention in S.~Auguſtin
\Cite{ep.~74.}
and
\Cite{ep.~20. de epiſtolis 22. in edit. Pariſ.}
\Cite{S.~Epiph. hær.~59.}
\Cite{S.~Cyprian ep.~74.}
\Cite{S.~Ignatius ep.~9. ad Antiochenos,}
and in the
\Cite{48.~canon of the Apoſtles.}
\Cite{Conc. Toles.~2. can.~1. &.~3.}
\Cite{Conc. Laodicen. cap.~23.}
\Cite{Epiſt. Epiph. apud Hiero.~60. c.~3.}}
Deacons in like manner
\TNote{\G{σεμνούς}}
chaſt, not double-tonged, not giuen to much wine, not followers of
filthie lucre: \V hauing the myſterie of faith in a pure conſcience. \V
And let theſe alſo be proued firſt:
\Fix{\V}{&}{obvious typo, fixed in other}
ſo let them miniſter, hauing no crime.

\V The women in like manner chaſt, not detracting, ſober, faithful in al
things. \V Let Deacons be the huſbands of one wife: which rule wel their
children, and their houſes. \V For they that haue miniſtred wel, ſhal
purchaſe to themſelues a good degree, and much confidence in the faith
which is in Chriſt \Sc{Iesvs}.

\V Theſe things I write to thee, hoping that I ſhal come to thee
quickly. \V But if I tary long, that thou maieſt know how thou oughteſt
to conuerſe
\LNote{In the houſe of God.}{\Emph{Al
\MNote{S.~Ambroſe calleth the B.~of Rome Rectour of the whole Church.}
the world being Gods, yet the Church only is his houſe, the Rectour or
Ruler whereof at this day}, (ſaith S.~Ambroſe
\Cite{vpon this place)}
\Emph{is Damaſus.} Where let our louing Brethren note wel, how cleare a
caſe it was then, that the Pope of Rome was not the Gouernour only of
one particular See, but of Chriſtes whole houſe, which is the Vniuerſal
Church, whoſe Rectour this day is Gregorie the thirteenth.}
in the houſe of God, which is the \Sc{Chvrch} of the liuing God,
\LNote{The piller of truth.}{This
\MNote{The heretikes ſay directly contrarie to the Apoſtle, that the
Church is not the piller of truth.}
place pincheth al Heretikes wonderfully, and ſo it euer did, and
therfore they oppoſe themſelues directly againſt the very letter and
confeſſed ſenſe of the ſame, that is, cleane contrarie to the Apoſtle:
Some ſaying, the Church to be loſt or hidden: ſome, to be fallen away
from Chriſt theſe many Ages: ſome, to be driuen to a corner only of the
world: ſome, that it is become a ſtewes and the Seat of Antichriſt:
laſtly the Proteſtants moſt plainely & directly that it may and doth erre
and hath ſhamefully erred for many hundred yeares together. And they ſay
herein like themſelues, and for the credit of their owne doctrine which
can not be true in very deed, except the Church erre, euen the Church of
Chriſt, which is here called the houſe of the liuing God.

%%% !!! CNote placement in this paragraph needs fixing.
But
\MNote{That the Church is the piller of truth & can not erre, is proued
by many reaſons.}
the Church which is the houſe of God, whoſe Rectour (ſaith S.~Ambroſe)
in his time was Damaſus, and now Gregorie the thirteenth, and in the
Apoſtles time S.~Peter, is the piller of truth, the eſtabliſhment of
al veritie: therfore it can not erre.
\CNote{\XRef{Io.~14,~16.}}
It hath the Spirit of God to lead it into al truth til the worlds end:
therfore it can not erre.
\CNote{\XRef{Mat.~16.}}
It is builded vpon a rocke, hel gates ſhal not preuaile againſt it:
therfore it can not erre.
\CNote{\XRef{Mat.~28.}}
Chriſt is in it til the end of the world,
\CNote{\XRef{Eph.~4.}}
he hath placed in it Apoſtles, Doctours, Paſtours, and Rulers to the
conſummation & ful perfection of the whole body, that in the meane time
we be not caried about with euery blaſt of doctrine: therfore it can not
erre.
\CNote{\XRef{Io.~17.}}
He hath praied for it, that it be ſanctified in veritie, that the faith
of the cheefe Gouernour thereof faile not: 
\CNote{\XRef{Luc.~22.}}
\CNote{\XRef{Pſal.~2.}}
\CNote{\XRef{Eph.~5.}}
it is his houſe, his ſpouſe, his body, his lot, Kingdom and inheritance
giuen him in this world: he loueth it as his owne fleſh, and it can not
be diuorced or ſeparated from him: therfore it can not erre. The new Teſtament,
Scriptures, Sacraments, and Sacrifice can not be changed, being the
euerlaſting dowrie of the Church, continued and neuer rightly occupied
in any other Church, but in this our Catholike Church: therfore it can
not erre. And therfore al thoſe points of doctrine, faith, and worſhip,
which the Arians, Manichees, Proteſtants, Anabaptiſtes, other old or new
Heretikes, vntruely thinke to be errours in the Church, be no errours
indeed but thẽſelues moſt ſhamefully are deceiued, and ſo ſhal be ſtil,
til they enter againe into this houſe of God, which is the piller and
ground of al truth: that is to ſay, not only it ſelf free from al errour
in faith and religion, but the piller and ſtay to leane vnto in al
doubts of doctrine and to ſtand vpon againſt al hereſies and errours
that il times yeald, without which there can be no certaintie nor
ſecuritie.
\MNote{The meaning of this article, \Emph{I beleeue the Cath. Church.}}
And therfore the holy Apoſtles, and Councels of Nice and Conſtantinople,
made it an article of our \Emph{Creed}, to beleeue the \Sc{Catholike}
and \Sc{Apostolike Chvrch}. Which is, not only to acknowledge that there
is ſuch a Church, as heretikes falſely ſay; but that that which is
called the Catholike Church, and knowen ſo to be, and communicateth with
the See Apoſtolike, is the Church: and that we muſt beleeue, heare, and
obey the ſame, as the touch-ſtone, piller, and firmament of truth. For,
al this is compriſed in that principle, \Emph{I beleeue the Catholike
Church}. And therfore the Councel of Nice ſaid,
\TNote{\G{πιστεύω εἰς τήv ἐκκλησίαν.}}
\Emph{I beleeue in the Church}, that is, I beleeue and truſt the ſame in
al things. 

Neither can the Heretikes eſcape by flying from the knowen viſible
Church, to the hid congregation or companie of the Predeſtinate. For
that is but a falſe phantaſtical apprehenſion of Wicleffe and his
followers. The companie of the Predeſtinate maketh not any one Societie
among themſelues, many of them being yet vnborne, and many yet Infidels
and heretikes, & therfore be not of the one houſe of God which is here
called,
\MNote{It is the viſible Church that is the piller of truth and can not
erre.}
\Emph{the piller of truth}. And thoſe of the Predeſtinate that be
already of the Church, make not a ſeueral cõpanie from the knowen
Catholike Church, but are baptiſed, houſeled, taught, they liue and die
in the common Catholike viſible Church, or els they can neither receiue
Sacrament, nor ſaluation. S.~Paul inſtructeth not Timothee how to teach,
preach, correct, and conuerſe in the inuiſible ſocietie of the
Predeſtinate, but in the viſible houſe of God. So that it muſt needs be
the viſible Church which can not erre.

If
\MNote{Whence the Church hath this priuiledge neuer to erre.}
any make further queſtion, how it can be that any companie or ſocietie
of men (as the Church is) can be void of errour in faith, ſeeing al men
may erre: he muſt know that it is not by nature, but by priuilege of
Chriſtes preſence, of the holy Ghoſts aſſiſtãce, of our Lordes promiſe
and praier. See S.~Auguſtin vpon theſe words of the
\Cite{118.~Pſalme Conc.~13.}
\MNote{S.~Auguſtin.}
\L{Ne auferos de ore meo verbum veritatis vſquequaque.} Where he hath
goodly ſpeaches of this matter. For the ſame purpoſe alſo theſe words of
Lactantius are very notable:
\MNote{Lactantius.}
\Emph{It is the Catholike Church only, that keepeth the true worship of
God; this is the fountaine of truth, this the houſe of faith, this the
Temple of God: whither if any man enter not, or frõ which if any man goe
out, he is an alien & ſtranger from the hope of euerlaſting life and
ſaluation. No man muſt by obſtinate contention flater himſelf, for it
ſtandeth vpon life and ſaluation, &c.} S.~Cyprian ſaith,
\MNote{S.~Ciprian.}
\Emph{The Church neuer departeth from that which she once hath knowen.}
\Cite{Ep.~55. ad Cornel. nu.~3.}
S.~Irenæus ſaith,
\MNote{S.~Irenæus.}
\Emph{That the Apoſtles haue laid vp in the Church as in a rich
treaſurie, al truth.} And, \Emph{that she keepeth with moſt ſincere
diligence, the Apoſtles faith and preaching.}
\Cite{li.~3. c.~4.}
&
\Cite{40.}
&
\Cite{li.~1. c.~3.}
It were an infinit thing to recite al that the Fathers ſay of this
matter, al counting it a moſt pernicious abſurditie to affirme, that the
Church of Chriſt may erre in religion.}
the piller and ground of truth. \V And manifeſtly it is a great
ſacrament of pietie, which was manifeſted in fleſh, was iuſtified in
ſpirit, appeared to Angels, hath been preached to Gentils, is beleeued
in the world, is aſſumpted in glorie.


\stopChapter


\stopcomponent


%%% Local Variables:
%%% mode: TeX
%%% eval: (long-s-mode)
%%% eval: (set-input-method "TeX")
%%% fill-column: 72
%%% eval: (auto-fill-mode)
%%% coding: utf-8-unix
%%% End:

