%%%%%%%%%%%%%%%%%%%%%%%%%%%%%%%%%%%%%%%%%%%%%%%%%%%%%%%%%%%%%%%%%
%%%%
%%%% The (original) Douay Rheims Bible 
%%%%
%%%% New Testament
%%%% Epistles
%%%% Two John
%%%% Chapter
%%%%
%%%%%%%%%%%%%%%%%%%%%%%%%%%%%%%%%%%%%%%%%%%%%%%%%%%%%%%%%%%%%%%%%




\startcomponent chapter


\project douay-rheims


%%% 2927
%%% o-2788
\startChapter[
  title={}
  ]

\Summary{He commendeth the Lady and her ſonnes for continuing in the
  old faith, bidding them ſo to doe hereafter alſo, leſt they loſe the
  reward of their workes in the day of iudgement: and to loue the true
  beleeuers, but with Heretikes to haue no ſocietie: expreſsing alſo the
  points then in controuerſie.}

The Seniour to the Lady Elect and her children, whom I loue in truth,
and not I only, but alſo al that haue knowen the truth, \V for the truth
which abideth in vs, and ſhal be with vs for euer. \V Grace be with you,
mercie, peace from God the Father, and from Chriſt \Sc{Iesvs} the Sonne
of the Father in truth, and charitie.

%%% o-2789
\V I was exceeding glad, becauſe I haue found of thy children walking in
truth, as we haue receiued commandement of the Father. \V And now I
beſeech thee Lady, not as writing a new commandement to thee, but that
which we haue had
\LNote{From the beginning.}{This
\MNote{To hold faſt the old receiued faith.}
is the Rule of a Chriſtian Catholike man, to walke in that faith and worship of
God which he hath receiued from the beginning. Which is that which we
now cal according to the Scriptures, \Emph{the tradition of the
Apoſtles}: that which is come to vs from man to man, from Bishop to
Bishop, and ſo from the Apoſtles. So shal a faithful man auoid ſeducers
that riſe vs in euery Age, teaching new doctrine.}
from the beginning,
\CNote{\XRef{Io.~15,~12.}
\XRef{1.~Io.~3,~11.}}
that we loue one another. \V And this is charitie, that we walke
according to his commandements. For this is the commandement, that as
you haue heard from the beginning, you walke in the ſame: \V becauſe many
ſeducers are gone out into the world, which doe not confeſſe \Sc{Iesvs}
Chriſt to haue come into fleſh: this is a ſeducer and an Antichriſt.

\V Looke to your ſelues, that you loſe not the things which you haue
wrought: but that you may receiue a ful
\SNote{Reward for keeping faſt the Catholike faith.}
reward. \V Euery one that
\SNote{To goe backe or reuolt from the receiued truth and doctrine
Apoſtolical, is damnable.}
reuolteth, and perſiſteth not in the doctrine of Chriſt, hath not
God. He that perſiſteth in the doctrine, the ſame hath both the Father,
and the Sonne. \V If
\CNote{\XRef{Ro.~16,~17.}}
any man come to you, and bring not
\LNote{This doctrine.}{The
\MNote{To bring wilfully another doctrine then the Catholike Church
ſetteth downe, is alwaies a marke of ſeducers & Heretikes.}
Apoſtles, and true Paſtours their lawful Succeſſours, and the Church of
God in holy Councel, vſe to ſet downe the true doctrine in thoſe points
which Heretikes cal into controuerſie. Which being once done and
declared to the faithful, they need no other marke or deſcription to
know an Heretike or falſe Teacher by, but that he commeth with an other
doctrine then that which is ſet downe to them. Neither can the Heretikes
shift themſelues, as now a-daies they would doe, ſaying, ô let vs firſt
be proued Heretikes by the Scriptures, let them define an Heretike. No,
this is not the Apoſtles Rule. Many a good honeſt shepheard knoweth a
woolfe, that can not define him. But the Apoſtle ſaith, If he bring not
this ſet doctrine, he is a ſeducer. So holy Church ſaith now, Chriſt is
really in the B.~Sacrament, vnder forme of bread and wine &c. If
therfore he bring not this doctrine, he is a ſeducer, and an Heretike
and we muſt auoid him, whether in his owne definitions and cenſures he
ſeeme to himſelf an Heretike or no.}
this doctrine,
\LNote{Receiue him not.}{Though
\MNote{When & wherein to cõuerſe with Heretikes, is tolerable, when &
wherein, it is damnable.}
in ſuch times and places where the communitie or moſt part be infected,
neceſſitie often forceth the faithful to conuerſe with ſuch in worldly
affaires, to ſalute them, to eate and ſpeake with them, & the Church by
decree of Councel, for the more quietnes of timorous conſciences prouideth,
that they incurre not excommunication or other cenſures for communicating
in worldly affaires with any in this kind, except they be by name
excommunicated or declared to be Heretikes: yet euen in worldly
conuerſation and ſecular actes of our life, we muſt auoid them as much
as we may, becauſe their familiaritie is many waies contagious and
noiſome to good men, namely to the ſimple: but in matter of religion, in
praying, reading their bookes, hearing their ſermons, preſence at their
ſeruice, partaking of their Sacraments, and al other communicating with
them in ſpiritual things, it is a great damnable ſinne to deale with
them.}
receiue him not into the houſe,
\LNote{Nor ſay, God ſaue you.}{S.~Irenæus
\MNote{S.~Iohn would not be in one bath with Cerinthus the Heretike.}
\Cite{(li.~3. c.~3.)}
reporteth a notable ſtorie of this holy Apoſtle touching this point, our
of Polycarpus, which is this. \Emph{There be ſome} (ſaith he) \Emph{that
haue heard Polycarpe ſay, that when Iohn the Diſciple of our Lord was
going to Epheſus, into a bath, to wash himſelf, and ſaw Cerinthus the
Heretike within the ſame, he ſodenly skips out, ſaying that he feared
leſt the bath should fal, becauſe Cerinthus the enemie of truth was
within.}
\MNote{The like zeale of S.~Polycarpe, and other Apoſtolike men in not
communicating with Heretikes.}
So ſaith he of S.~Iohn, and addeth alſo a like worthie example
of S.~Polycarpe himſelf: who on a time meeting Marcion the Heretike, and
the ſaid Marcion calling vpon him and asking whether he knew him not:
\Emph{Yes}, quoth Polycarpe, \Emph{I know thee for Satans ſonne and
heire. So great feare} (ſaith S.~Irenæus) \Emph{had the Apoſtles & their
diſciples to communicate in word only, with ſuch as were adulterers or
corrupters of the truth: as S.~Paul alſo warned, when he ſaid,
\CNote{\XRef{Tit.~3.}}
A man that is an Heretike, after the firſt and ſecond admonition auoid.}
So farre Irenæus. If then to ſpeake with them or ſalute them, is ſo
earneſtly to be auoided according to this Apoſtles example & doctrine;
what a ſinne is it to flatter them, to ſerue them, to marrie with them,
and ſo-forth?}
nor ſay, \Emph{God ſaue you}, vnto him. \V For he that ſaith vnto him,
\Emph{God ſaue you}, communicateth with his wicked workes.

\V Hauing moe things to write vnto you: I would not by paper and inke:
for I hope that I ſhal be with you, and ſpeake mouth to mouth: that your
ioy may be ful. \V The children of thy ſiſter elect ſalute thee.


\stopChapter


\stopcomponent


%%% Local Variables:
%%% mode: TeX
%%% eval: (long-s-mode)
%%% eval: (set-input-method "TeX")
%%% fill-column: 72
%%% eval: (auto-fill-mode)
%%% coding: utf-8-unix
%%% End:

