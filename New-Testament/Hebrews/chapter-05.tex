%%%%%%%%%%%%%%%%%%%%%%%%%%%%%%%%%%%%%%%%%%%%%%%%%%%%%%%%%%%%%%%%%
%%%%
%%%% The (original) Douay Rheims Bible 
%%%%
%%%% New Testament
%%%% Epistles
%%%% Hebrewes
%%%% Chapter 05
%%%%
%%%%%%%%%%%%%%%%%%%%%%%%%%%%%%%%%%%%%%%%%%%%%%%%%%%%%%%%%%%%%%%%%




\startcomponent chapter-05


\project douay-rheims


%%% 2847
%%% o-2708
\startChapter[
  title={Chapter 05}
  ]

\Summary{That Chriſt being a man and infirme, was therein but as al
  Prieſts; and that he alſo was called of God to this office: offering
  as the others: 8.~and ſuffered obediently for our example. 11.~Of
  whoſe Prieſthood he hath much to ſay, but that the Hebrewes haue need
  rather to heare their Catechiſme againe.}

For
\LNote{Euery high Prieſt.}{By
\MNote{The deſcriptiõ of a Prieſt, and his office.}
the deſcriptiõ of a Prieſt or high Prieſt (for to this purpoſe al is one
matter) he proueth Chriſt to be one in moſt excellẽt ſort. Firſt then, a
Prieſt muſt not be an Angel, or of any other nature but man's. Secõdly,
euery mã is not a Prieſt, but ſuch an one as is ſpecially choſen out of
the reſt, and preferred before other of the community, ſeuered,
aſſumpted, and exalted into a higher ſtate and dignitie then the
vulgar. Thirdly, the cauſe and purpoſe why he is ſo ſequeſtred and
picked out from the reſidue, is to take charge of Diuine things, to
deale as a Mediatour betwixt God and the people, to be the Deputie of
men in ſuch things as they haue to craue or to receiue of God, and to
preſent or giue to him againe. Fourthly, the moſt proper and principal
part of a Prieſts office is, to offer oblations, guifts, and Sacrifices
to God for the ſinnes of the people: without which kind of moſt
ſoueraigne dueties, no perſon, people, or common-wealth can appertaine
to God: and which can be done by none, of what other dignitie or calling
ſoeuer he be in the world, that is not a Prieſt: diuers Princes (as we
read in the
\CNote{\XRef{3.~Reg.~13.}
\XRef{2.~Par.~26.}
\XRef{1.~Reg.~13.}}
Scriptures) punished by God, and King Saul depoſed from his Kingdom,
ſpecially for attempting the ſame.

And
\MNote{The Princes temporal authoritie how farre it extendeth.}
generally we may learne here, that \L{in iis quæ ſunt ad Deum}, in al
matters touching God, his ſeruice, and religion, the Prieſt hath only
charge & authority: as the Prince tẽporal is the peoples Gouernour,
Guider, & Soueraigne, in the things touching their worldly affaires:
Which muſt for al that by him be directed and manneged no otherwiſe, but
as is agreable to the due worship and ſeruice of God. Againſt which if
the terrene Powers commit any thing, the Prieſts ought to admonish them
from God.

We
\MNote{There is a peculiar order & calling of Prieſts of the new
Teſtamẽt.}
learne alſo hereby, that euery one is not a Prieſt, and that the people
muſt alwaies haue certaine perſons choſen out from among them, to deale
in their ſutes and cauſes with God, to pray, to Miniſter Sacraments, and
to Sacrifice for them. And whereas the Proteſtants wil haue no Prieſt,
Prieſthood, nor Sacrifice, but Chriſt and his death, pretending theſe
words of the Apoſtle to be verified only in the Prieſthood and Seruice
of the old law, and Chriſtes Perſon alone, and after him of no moe;
therin they shew themſelues to be ignorant of the Scriptures, & of the
ſtate of the new Teſtament, and induce a plaine Atheiſme and Godleſneſſe
into the world.
\MNote{Prieſts and Sacrifice neceſſarie in the new Teſtament, and
nothing derogatorie to Chriſt's prieſthood or Sacrifice.}
For ſo long as man hath to doe with God, there muſt
needs be ſome deputed, & choſen out from among the reſt, to deale
according to this declaration of the Apoſtle, in things pertaining to
God, and thoſe muſt be Prieſts. For els, if men need to deale no more,
but immediately with Chriſt, what doe they with their Miniſters? Why let they
not euery man pray, and Miniſter for himſelf & to himſelf: What doe they
with Sacramẽts, ſeeing Chriſtes death is as wel ſufficiẽt without them,
as without Sacrifice? Why ſtandeth not his death as wel with Sacrifice,
as with Sacramẽts: as wel with Prieſthood, as with other Eccleſiaſtical
functiõ? There is no other cauſe in the world, but that (Sacrifice being
the moſt prĩcipal act of religiõ that mã oweth to God, both by his Law,
and by the Law of nature) the Diuel by theſe his Miniſters, vnder
pretence of deferring or attributing the more to Chriſtes death, would
abolish it.

This
\MNote{The difference & excellence of Chriſt's Prieſthood.}
definition of a Prieſt and his function, with al the properties thereto
belonging, holdeth not only in the law of Moyſes, and order of Aarons
Prieſthood, but it was true before, in the law of nature, in the
Patriarches, in Melchiſedech, and now in Chriſt, and al his Apoſtles,
and Prieſts of the new Teſtament. Sauing that it is a peculiar
excellencie in Chriſt, that he only offered for other mens ſinnes, and
not at al for his owne, as al other doe.}
euery high Prieſt taken from among men, is appointed for men in thoſe
things that pertaine to God: that he may offer guifts and Sacrifices for
ſinnes: \V that can haue compaſſion on them that be ignorant and doe
erre: becauſe himſelf alſo is compaſſed with infirmitie: \V & therfore
he ought, as for the people, ſo alſo for himſelf to offer for ſinnes. \V
\CNote{\XRef{1.~Par.~26,~18.}}
Neither doth any man
\LNote{Taketh to himſelf.}{A
\MNote{Al true Prieſts and Preachers muſt be lawfully called thereto.}
ſpecial prouiſo for al Prieſts, Preachers,
and ſuch as haue to deale for the people in thĩgs pertaining to God,
that they take not that honour or office at their owne hands, but by
lawful calling & conſecration, euen as Aaron did. By which clauſe if you
examine Luther, Caluin, Beza, and the like or if al ſuch as now a-daies
intrude themſelues into ſacred functions, looke into their conſciences,
great and foul matter of damnation wil appeare.}
take the honour to himſelf, but he that is called of God,
\CNote{\XRef{1.~Par.~23,~13.}}
as Aaron. \V So Chriſt alſo
\LNote{Did not glorifie himſelf.}{The
\MNote{The dignitie and function of Prieſthood is not to be vſurped.}
dignity of Prieſthood muſt needs be paſſing high and ſoueraigne, when it
was a promotion & preferment in the Sonne of God himſelf according to
his manhood, and when he would not vſurpe, nor take vpon him the ſame,
without his Fathers expreſſe commiſſion and calling thereunto. An
eternal example of humility, & an argument of condemnation to al mortal
men, that arrogate vniuſtly any function or power ſpiritual, that is not
giuen them from aboue, and by lawful calling and commiſſion of their
Superiours.}
did not glorifie himſelf that he might be made a high Prieſt; but he
that ſpake to him,
\CNote{\XRef{Pſ.~2,~7.}}
\Emph{My Sonne art thou, I this day haue begotten thee.} \V As alſo in
another place he ſaith,
\CNote{\XRef{Pſal.~109,~4.}}
\Emph{Thou art
\LNote{A Prieſt for euer.}{In
\MNote{Chriſt both Prieſt & King: but his Prieſthood more excellent of
the two.}
the
\XRef{109.~Pſalme,}
from whence this teſtimonie is taken, both Chriſtes Kingdom and
Prieſthood are ſet forth. But the Apoſtle vrgeth ſpecially his
Prieſthood, as the more excellent & preeminent ſtate in him, our
Redemption being wrought & atchieued by Sacrifice, which was an act of
his Prieſthood, and not of his Kingly power: though he was properly a
King alſo, as Melchiſedech was both Prieſt & King, being a reſemblance
of Chriſt in both, but much more in his Prieſthood. And our Lord had
this excellent double dignitie (as appeareth by the diſcourſe of
S.~Paul, & his allegations here out of the
\CNote{\XRef{Pſal.~2.}
\XRef{109.}}
Pſalmes) at the very firſt moment of his conception or incarnation.
\MNote{Chriſt a Prieſt as he is man not as he is God.}
For you muſt beware of the wicked hereſie of the Arians and Caluiniſts
(except in theſe later it be rather an errour proceeding of ignorance)
that ſticke not to ſay, that Chriſt was a Prieſt, or did Sacrifice,
according to his Godhead. Which is to make Chriſt God the Fathers
Prieſt, & not his Sonne, & to doe Sacrifice & homage to him as his Lord,
and not as his equal in dignity & nature. Therfore S.~Auguſtin ſaith
\Cite{in Pſal.~109.}
\Emph{That as he was man, he was Prieſt: as God, he was not Prieſt.} And
Theodorete
\Cite{in Pſal.~109.}
\Emph{As man he did offer Sacrifice: but as God, he did receiue
Sacrifice.} And againe, \Emph{Chriſt touching his humanity was called a
Prieſt, and he offered no other hoſt but his owne body, &c.}
\Cite{Dialog.~1. circa med.}
Some of our new
\CNote{\Cite{Retent. pag.~89.}}
Maiſters not knowing ſo much, did let fal out of their pennes the contrarie, and
being admonished of the errour, and that it was very Arianiſme, yet they
perſiſt in it of mere ignorance in the grounds of Diuinitie.}
a Prieſt for euer, according to the order of Melchiſedech.} \V Who in
the daies of his fleſh,
\LNote{With a ſtrong crie.}{Though
\MNote{The Sacrifice on the Croſſe was the principal acte of Chriſt's
prieſthood.}
our Sauiour make interceſſion for vs, according to his humane nature,
continually in heauen alſo, yet he doth not in any external creatures
make Sacrifice, nor vſe the praiers Sacrificial, by which our redemption
was atchieued, as he did in the time of his mortal life, and in the act
of his Paſſion, and moſt principally when with a loud voice, and with
this praier,
\CNote{\XRef{Luc.~23,~46.}}
\L{in manus tuas commendo spiritum meum}, he voluntarily depoſed his
ſoul, yealding it in moſt proper ſort for a Sacrifice. For in that laſt
point of his death, conſiſteth ſpecially his high Prieſtly office, and
the very worke and conſummation of our redemption.

Obſerue
\MNote{Prieſts praiers more effectual.}
more-ouer, that though commonly euery faithful perſon pray both for
himſelf and others, and offer his praiers to God, yet none offereth by
office and ſpecial deputation, and appointment, in the perſon of the
whole Church and people, ſauing the Prieſt. Whoſe praiers therfore be
more effectual in themſelues, for that they be the voice of al faithful
men together, made by him that is appointed & receiued of God for the
peoples Legate.
\MNote{Chriſt's Prieſtly actions.}
And of this kind were al Chriſtes praiers, in al his life and death, as
al his other actions were: his faſting, watching, preaching,
inſtituting, miniſtring, or receiuing Sacraments: euery one being done
as Prieſtly actions.}
with a ſtrong crie and teares, offering praiers and ſupplications to him
that could ſaue him from death, was heard
\LNote{For his reuerence.}{Theſe
\MNote{Notorious Heretical tranſlation to maintaine Caluin's horrible
blaſphemie.} 
words haue our English tranſlatours pernitiouſly and moſt preſumptuouſly
corrupted, turning them thus, \Emph{In that which he feared}, contrarie
to the verſion and ſenſe of al antiquity, and to Eraſmus alſo, and
contrarie to the ordinarie vſe of the Greek word, as Beza himſelf
defineth it
%%% !!! XRef or is this merely where Beza defined it? Cite Beza instead?
Luc.~2. v.~25.
and contrarie to the propriety of the
\TNote{\G{ἀπὸ εὐλαβείας}}
Greek phraſe, as not only the Catholikes, but
\CNote{\Cite{Flac. Illyr. vpon this place.}}
the beſt learned Lutherans doe shew & proue by many examples. They follow
herein the ſingular preſumption of Caluin, who was the firſt (as his
fellow Beza confeſſeth) that euer found out this interpretation. Which
neither S.~Chryſoſtom, nor any other, as perfect Grecians as they were,
could euer eſpie. Where, only to haue made choiſe of that impious and
arrogant Sectaries ſenſe, before the ſaid Fathers & al the Churches
beſides, had been shameful enough; but to ſet the ſame downe for very
Scripture of God's bleſſed word, that is intolerable, and paſſeth al
impiety. And we ſee plainely that they haue no conſcience,
indifferencie, nor other purpoſe, but to make the poore Readers beleeue,
that their opinions be Gods owne word, and to draw the Scriptures to
ſound after the fantaſie of their hereſies. But if the good Reader knew,
for what point of doctrine they haue thus framed their tranſlation, they
would abhorre them to the depth of Hel. Forſooth it is thus:
\MNote{Caluins blaſphemie that Chriſt ſuffered he paines vpõ the Croſſe,
and that his death otherwiſe were inſufficient.}
they would haue this Scripture meane, that Chriſt was in horrible feare
of damnation, & that he was not only in paines corporal vpon the Croſſe
(which they hold, not to haue been ſufficient for man's redemption) but
that he was in the very ſorrowes & diſtreſſes of the damned, without any
difference, but that it was not euerlaſting, as theirs is.

For this horrible blaſphemie (which is their interpretation of Chriſtes
deſcending into Hel) God's holy word muſt be corrupted, and the
Sacrifice of Chriſtes death (wherof they talke ſo preſumptuouſly) muſt
not be enough for our redemption, except he be damned for vs alſo to the
paines of Hel. Woe be to our poore Countrie, that muſt haue ſuch books,
and read ſuch tranſlations. See Caluin and Beza in their
\Cite{Commentaries and Annotations vpon this place,}
& you shal ſee, that for defenſe of the ſaid blaſphemies they haue thus
tranſlated this text. See
\XRef{Annotations before Act.~2,~27.}
and
\XRef{Mat.~27,~46}}
for his reuerence. \V And truely whereas he was the Sonne, he learned by
thoſe things which he ſuffered, obedience: \V and being
%%% !!! LNote not marked in either
\LNote{Conſummate.}{The
\MNote{Chriſt yealding vp the Ghoſt, accomplishing
\Fix{or}{our}{obvious typo, fixed in other}
redemption.}
ful worke of his Sacrifice, by which we were redeemed, was wholy
conſummate and accomplished, at the yealding vp of his ſpirit to God the
Father, when he ſaid:
\CNote{Io.~19,~30.}
\L{Consummatum est}: though for to make the ſame effectual to the
ſaluation of particular men, he himſelf did diuers things, and now doth
in heauen, and our ſelues alſo muſt vſe many meanes, for the application
thereof to our particular neceſſitie. See the
\XRef{next Annotation.}}
conſummate,
\LNote{Was made to al.}{The
\MNote{Chriſt's Paſſiõ ſufficient for al but profitable to them only
which obey, not by faith only, but by doing as he & his Church, command.}
Proteſtants vpon pretence of the ſufficiencie of Chriſtes Paſſion, and
his only redemption, oppoſe themſelues guilefully in the ſight of the
ſimple, againſt the inuocation of Saints, and their interceſſion, and
help of vs, againſt our penitential workes or ſuffering for our owne
ſinnes, either in this life or the next: againſt the merits of faſting,
praying, almes, and other things commended to vs in holy Writ, and
againſt moſt things done in the Church, in Sacrifice, Sacrament, and
ceremonie. But this place and many other shew, that Chriſtes Paſſion,
though it be of it-ſelf farre more ſufficient and forcible, then the
Proteſtants in their baſeneſſe of vnderſtanding can conſider, yet
profiteth none but ſuch, as both doe his commandements, and vſe ſuch
remedies and meanes to apply the benefit thereof to themſelues, as he
appointeth in his word, or by the holy Ghoſt in his Church. And the
Heretikes that ſay, faith only is the thing required to apply Chriſtes
benefits vnto vs, are hereby alſo eaſily refuted. For we doe not obey
him only by beleeuing, but by doing whatſoeuer he commandeth. Laſtly, we
note in the ſame words, that Chriſt appointeth not by his abſolute and
eternal election, men ſo to be partakers of the fruit of his redemption,
without any conditiõ or reſpect of their owne workes, obedience, or
free-wil: but with this condition alwaies, if men wil obey him, and doe
that which he appointeth. See S.~Auguſtin (or Proſper)
\Cite{to.~7. Reſponſ. Proſperi li.~2. articulo~1. ad obectiones
Vincentij,}
where he ſaith of the cup of Chriſtes paſſion, \Emph{It hath indeed in
it-ſelf, to profit al: but if it be not drunken, it healeth not.}}
was made to al that obey him, cauſe of eternal ſaluation, \V called of
God a high Prieſt according to the Order of Melchiſedech.

\V Of whom we haue great ſpeach and
\LNote{Inexplicable.}{Intending
\MNote{The Apoſtle omitteth to ſpeake of the B.~Sacrament as a Myſterie
then too deep for the Iewes capacitie.}
to treat more largely and particularly of Chriſtes or Melchiſedechs
Prieſthood, he fore-warneth them that the myſterie thereof is farre
paſſing their capacitie, and that through their feeblenes in faith and
weakenes of vnderſtanding, he is forced to omit diuers deep points
concerning the Prieſthood of the new law. Among which (no doubt) the
myſterie of the Sacrament and Sacrifice of the altar, called \Sc{Masse}
was a principal & pertinent matter: which the Apoſtles & Fathers of the
Primitiue Church vſed not to treat of ſo largely & particularly in their
writings, which might come to the hands of the vnfaithful, who of al
things tooke ſooneſt ſcandal of the B.~Sacrament, as we ſee
\XRef{Io.~6.}
\Emph{He ſpake to the Hebrewes} (ſaith S.~Hierom
\Cite{ep.~126.)}
\Emph{that is to the Iewes, and not to faithful men, to whom he might
haue been bold to vtter the Sacrament.} And indeed it was not reaſonable
to talke much to them of that Sacrifice which was the reſemblance of
Chriſtes death, when they thought not right of Chriſtes death
it-ſelf. Which the Apoſtles wiſedom and ſilence our Aduerſaries wickedly
abuſe againſt the holy Maſſe.}
inexplicable to vtter: becauſe you are become weake to heare. \V For
whereas you ought to be Maiſters for your time, you need to be taught
againe your ſelues what be the elements of the beginning of the words of
God: & you are become ſuch
%%% 2848
%%% o-2709
as haue need of milke, and not of ſtrong meat. \V For euery one that is
partaker of milke, is vnskilful of the word of iuſtice: for he is a
child. \V But ſtrong meate is for the perfect, them that by cuſtome haue
their ſenſes exerciſed to the diſcerning of good and euil.


\stopChapter


\stopcomponent


%%% Local Variables:
%%% mode: TeX
%%% eval: (long-s-mode)
%%% eval: (set-input-method "TeX")
%%% fill-column: 72
%%% eval: (auto-fill-mode)
%%% coding: utf-8-unix
%%% End:

