%%%%%%%%%%%%%%%%%%%%%%%%%%%%%%%%%%%%%%%%%%%%%%%%%%%%%%%%%%%%%%%%%
%%%%
%%%% The (original) Douay Rheims Bible 
%%%%
%%%% New Testament
%%%% Epistles
%%%% Hebrewes
%%%% Argument
%%%%
%%%%%%%%%%%%%%%%%%%%%%%%%%%%%%%%%%%%%%%%%%%%%%%%%%%%%%%%%%%%%%%%%

%%% Latin checked by KK.




\startcomponent argument


\project douay-rheims


%%% 2841
%%% o-2701
\startArgument[
  title={\Sc{The Argvment of the Epistle of S.~Pavl to The Hebrewes.}},
  marking={Argument of Hebrewes}
  ]

That the Hebrewes were not al the Iewes, but only a part of them, it is
manifeſt
\XRef{Act.~6.}
where the primitiue Church of Hieruſalem, although it conſiſted of Iewes
only, as we read
\XRef{Act.~2.}
yet it is ſaid to conſiſt of two ſorts, Greekes and Hebrewes. Which
againe is manifeſt
\XRef{Phil.~3.}
where S.~Paul comparing himſelf with the Iudaical falſe-Apoſtles, ſaith,
that he alſo is \Emph{an Hebrew of Hebrewes}. Finally, they ſeeme to
haue been thoſe Iewes which were borne in Iurie, which for the moſt part
dwelled alſo there. Therfore to the Chriſtian Iewes in Hieruſalme and in
the reſt of Iurie, S.~Paul writeth this Epiſtle, out of Italie: ſaying
thereupon, \Emph{The Brethren of Italie ſalute you.}
\XRef{Heb.~13.}
By which words, and by theſe other in the
\XRef{ſame place,}
\Emph{Know ye our brother Timothee to be diſmiſſed, with whom (if he
come the ſooner) I wil ſee you}, it is euident, that he wrot this, not
only after he was brought priſoner to Rome,
\Fix{wherein in}{wherein}{obvious typo, fixed in other}
S.~Luke endeth the Actes of the Apoſtles, but alſo after he was ſet at
libertie there againe.

Many cauſes are giuen of the Doctours, why writing to the Iewes, he doth
not put his name in the beginning, \Emph{Paul an Apoſtle, &c.} as he
doth lightly in
%%% o-2702
his Epiſtles to the Churches and Bishops of the Gentils. The moſt likely
cauſe is, for that he was
\CNote{\XRef{2.~Tim.~3.}}
\Emph{the Preacher and Apoſtle and Maiſter of the Gentils}. And againe
in another place he ſaith,
\CNote{\XRef{1.~Tim.~2.}}
that himſelf was appointed the Apoſtle of the Gentils, as Peter of the
Iewes.
\XRef{Gal.~2.}
Only S.~Peter therfore writing to the Iewes, doth vſe this ſtile:
\CNote{\XRef{1.~Pet.~1.}}
\Emph{Peter an Apoſtle of} \Sc{Iesvs} \Emph{Chriſt &c.} becauſe he was
more peculiarly their Apoſtle, as being the Vicar
%%% 2842
of Chriſt, who was alſo himſelf
\SNote{Yet was Chriſt head of the Gentils alſo. So likewiſe his vicar
S.~Peter, notwithſtanding his more peculiar Apoſtleship ouer the Iewes.}
more ſpecially
\CNote{\XRef{Rom.~5.}}
\Emph{the Miniſter of the Circumciſion}, that is (as himſelf ſpeaketh)
\Emph{not ſent but to the ſheep which were loſt of the houſe of Iſrael.}
\XRef{Mat.~15.}

The Argument of the Epiſtle S.~Paul himſelf doth tel vs in two words,
calling it
\CNote{\XRef{Heb.~13.}}
\L{verbum ſolatij}, \Emph{the word of ſolace and comfort}. Which alſo is
plaine in the whole courſe of the Epiſtle, namely in the
\XRef{tenth chapter v.~32. &c.}
Where he exhorteth them to take great comfort and confidence in their
manifold tribulations ſuſtained of their owne Countrie-men the Iewes,
whereof the Apoſtle alſo maketh mention to the Theſſallonians.
\XRef{1.~Theſs.~2. v.~14.}
Thoſe perſecutions then of the obſtinate incredulous Iewes their
countrie-men, was one great tentation vnto them. Another tentation was,
the perſuaſions that they brought vnto them out of Scriptures, to cleaue
vnto the Law, and not to beleeue in \Sc{Iesvs} the dead man.

And whereas the Iewes did magnifie their Law, by the Prophets, and by
the Angels by whom it was giuen, and by Moyſes, and by their land of
promiſe, into which Ioſue brought them, and by their father Abraham, and
by their Aaronical or Leuitical prieſthood and Sacrifices, by their
Tabernacle, and by their Teſtament: he sheweth, that our
Lord \Sc{Iesvs}, as being the natural Sonne of God, paſſeth incomparably
the Prophets, the Angels, and Moyſes: that the reſt or quietnes which
God promiſed, was not in their earthly land, but in heauen: that his
figure Melchiſedech farre paſſed Abraham: and that his prieſthood,
Sacrifice, Tabernacle, and Teſtament, farre paſſed theirs. In al which
he shooteth often at theſe three markes: to take away the ſcandal of
Chriſtes death, by giuing them ſundrie good reaſons and teſtimonies of
it: to erect their minds from viſible and earthly promiſes (to which
only, the Iewes were wholy bent) to inuiſible and heauenly: and to
inſinuate that the Ceremonies should now ceaſe, the time of their
correction by Chriſt being now come.

The Epiſtle may be deuided into theſe parts: the firſt, Of Chriſtes
excellencie aboue the Prophets, Angels, Moyſes, and Ioſue,
\XRef{c.~1.~2.~3.~4.}
The ſecond, of his prieſthood and excellencie thereof aboue the
Prieſthood of the old Teſtament:
\XRef{c.~5. vnto the middeſt of the~10.}
The laſt part is of exhortation
\XRef{c.~10. v.~9. to the end of the Epiſtle.}


\stopArgument


\stopcomponent


%%% Local Variables:
%%% mode: TeX
%%% eval: (long-s-mode)
%%% eval: (set-input-method "TeX")
%%% fill-column: 72
%%% eval: (auto-fill-mode)
%%% coding: utf-8-unix
%%% End:
