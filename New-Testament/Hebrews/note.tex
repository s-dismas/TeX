%%%%%%%%%%%%%%%%%%%%%%%%%%%%%%%%%%%%%%%%%%%%%%%%%%%%%%%%%%%%%%%%%
%%%%
%%%% The (original) Douay Rheims Bible 
%%%%
%%%% New Testament
%%%% Epistles
%%%% Hebrewes
%%%% Note
%%%%
%%%%%%%%%%%%%%%%%%%%%%%%%%%%%%%%%%%%%%%%%%%%%%%%%%%%%%%%%%%%%%%%%




\startcomponent note


\project douay-rheims


%%% 2843
%%% o-2703
\startChapter[
  title={}
  ]

Let
\MNote{Heretical corruption.}
the Chriſtian Reader note the corruption and impudent boldnes of our
Aduerſaries, that vpon a falſe priuate perſuaſion of their owne, that
S.~Paul was not the Authour of this Epiſtle,
\CNote{\Cite{In the English Bible of the yeare~1579.}}
leaue out his name in the title of the ſame, contrarie to the
authentical copies both Greeke and Latin. In old time there was ſome
doubt who should be the writer of it, but then when it was no leſſe
doubted whether it were Canonical Scripture at al.
\MNote{The Epiſtle to the Hebrewes is S.~Paules.}
Afterward the whole Church (by which only we know the true Scriptures
from other writings) held it and deliuered it, as now ſhe doth, to the
faithful for Canonical, and for S.~Paules Epiſtle. Notwithſtanding the
Aduerſaries would haue refuſed the Epiſtle, as wel as they doe the
Authour, but that they falſely imagin certaine places thereof to make
againſt the Sacrifice of the Maſſe.


\stopChapter


\stopcomponent


%%% Local Variables:
%%% mode: TeX
%%% eval: (long-s-mode)
%%% eval: (set-input-method "TeX")
%%% fill-column: 72
%%% eval: (auto-fill-mode)
%%% coding: utf-8-unix
%%% End:

