%%%%%%%%%%%%%%%%%%%%%%%%%%%%%%%%%%%%%%%%%%%%%%%%%%%%%%%%%%%%%%%%%
%%%%
%%%% The (original) Douay Rheims Bible 
%%%%
%%%% New Testament
%%%% Epistles
%%%% Hebrewes
%%%% Chapter 01
%%%%
%%%%%%%%%%%%%%%%%%%%%%%%%%%%%%%%%%%%%%%%%%%%%%%%%%%%%%%%%%%%%%%%%




\startcomponent chapter-01


\project douay-rheims


%%% 2843
%%% o-2703
\startChapter[
  title={Chapter 01}
  ]

\Summary{God ſpake to their fathers by the
\Fix{Prophet:}{Prophets:}{obvious typo, fixed in other}
but to themſelues by his owne Sonne, 14.~who incomparably paſſeth al the
Angels.}

Diuerſely and many waies in times paſt God ſpeaking to the Fathers in
the Prophets, \V laſt of al in theſe daies hath ſpoken to vs in his
Sonne, whom he hath appointed heire of al, by whõ he made alſo the
worlds. \V
\CNote{\XRef{Sap.~76,~26.}}
Who being the
\TNote{\G{ἀπαύγασμα}}
brightneſſe of his glorie, and
\LNote{The figure.}{To
\MNote{The B.~Sacramẽt a figure, and yet the true body.}
be the figure of his ſubſtance, ſignifieth nothing els but that which
S.~Paul ſpeaketh in other wordes to the Philipians
\XRef{c.~2. v.~6.}
that he is the forme and moſt expreſſe reſemblance of his Fathers
ſubſtance. So
%%% !!! Cite ?
S.~Ambroſe and others expound it, and the Greeke word
\TNote{\G{μορφὴ χαρακτὴρ}}
\Emph{Character} is very ſignificant to that purpoſe. Note alſo by this
place, that the Sonne, though he be a figure of his Fathers ſubſtance,
is notwithſtanding of the ſame ſubſtance. So Chriſtes body in the
Sacrament and his myſtical death and Sacrifice in the ſame, though
called a figure, image, or repreſentation of Chriſtes viſible body and
Sacrifice vpon the Croſſe, yet may be and is the ſelf-ſame in
ſubſtance.}
the
\TNote{\G{χαρακτὴρ ὑποστάσεως}}
figure of his ſubſtance, and carying al things by the word of his power,
making purgation of ſinnes, ſitteth on the right hand of the Maieſtie in
the high places: \V %%% !!! SNote ?
\MNote{The excellencie of Chriſt aboue Angels.}
being made ſo much better then Angels, as he hath
inherited a more excellent name aboue them.

\V For to which of the Angels hath he ſaid at any time,
\CNote{\XRef{Pſ.~2,~7.}}
\Emph{Thou art my Sonne, to day haue I begotten thee?} and againe,
\CNote{\XRef{2.~Reg.~7,~14.}}
\Emph{I wil be to him a Father, and he shal be to me a Sonne.} \V And
when againe he bringeth
%%% o-2704
in the firſt-begotten into the world, he ſaith,
\CNote{\XRef{Pſ.~96,~8.}}
\Emph{And
\LNote{Let al the Angels adore.}{The
\MNote{The adoration of Chriſt in the B.~Sacrament.}
heretikes maruel that we adore Chriſt in the B.~Sacrament, when they
might learne by this place, that whereſoeuer his perſon is, there it
ought to be adored both of men and Angels. And where they ſay it was not
made preſent in the Sacrament nor inſtituted to be adored: we anſwer
that no more was he incarnate purpoſely to be adored: but yet ſtraight
vpon his deſcending from heauen, it was the duety both of Angels and al
other creatures to adore him.}
let al the Angels of God adore him.} \V And to the Angels truly he
ſaith,
\CNote{\XRef{Pſ.~103,~4.}}
\Emph{he that maketh his Angels, ſpirits: and his Miniſters, a flame of
fire.} \V But to the Sonne:
\CNote{\XRef{Pſ.~44,~7.}}
\Emph{Thy throne ô God for euer and euer: a rod of equity, the rod of
thy Kingdom. \V Thou haſt loued iuſtice, and hated iniquitie: therfore
thee, God, thy God hath annointed with the oile of exultation aboue thy
fellowes.} \V And,
\CNote{\XRef{Pſ.~101,~16.}}
\Emph{Thou in the beginning ô Lord didſt found the earth: and the workes
of thy hands are the heauens. \V They shal perish, but thou shalt
continue: and they shal al waxe old as a garment. \V And as a veſture
shalt thou change them, and they shal be changed: but thou art the
ſelf-ſame, and thy yeares shal not faile.} \V But to which of the Angels
ſaid he at any time:
\CNote{\XRef{Pſ.~109,~1.}
\XRef{1.~Cor.~15,~25.}}
\Emph{Sit on my right hand, vntil I make thine enemies the foot-ſtoole
of thy feet?} \V Are they not al, a
\SNote{The holy Angels (ſaith S.~Auguſtin) to the ſocietie of whom we
aſpire in this our peregrination, as they haue eternitie to cõtinue, ſo
alſo facilitie to know and felicitie to reſt: for they doe help vs
without al difficultie, becauſe with their ſpiritual motiõs pure & free,
they labour or trauail not.
\Cite{De Ciuit. lib.~11. c.~31.}}
miniſtring ſpirits: ſent to miniſter for them which ſhal receiue the
inheritance of ſaluation?


\stopChapter


\stopcomponent


%%% Local Variables:
%%% mode: TeX
%%% eval: (long-s-mode)
%%% eval: (set-input-method "TeX")
%%% fill-column: 72
%%% eval: (auto-fill-mode)
%%% coding: utf-8-unix
%%% End:

