%%%%%%%%%%%%%%%%%%%%%%%%%%%%%%%%%%%%%%%%%%%%%%%%%%%%%%%%%%%%%%%%%
%%%%
%%%% The (original) Douay Rheims Bible 
%%%%
%%%% New Testament
%%%% Hebrews
%%%% Chapter 07
%%%%
%%%%%%%%%%%%%%%%%%%%%%%%%%%%%%%%%%%%%%%%%%%%%%%%%%%%%%%%%%%%%%%%%




\startcomponent chapter-07


\project douay-rheims


%%% 2853
%%% o-2713
\startChapter[
  title={Chapter 07}
  ]

\Summary{To proue the Prieſthood of Chriſt incomparably to excel the
  Prieſthood of Aaron (and therfore that Leuitical Prieſthood now to
  ceaſe, and that law alſo with it) he ſcanneth euery word of the verſe
  alleaged out of the Pſalme,
  %%% !!! Really Roman (upright) since Summary is already italic.
  \Emph{Our Lord hath ſworne: thou art a Prieſt for euer, according to
  the order of Melchiſedech.}}

For this
\LNote{Melchiſedech.}{The excellencie of this perſon was ſo great, that
ſome of the antiquitie tooke him to be an Angel, and ſome the holy
Ghoſt. Which opinion not only the Hebrewes, that auouch him to be Sem
the Sonne of Noe, but alſo the cheefe Fathers of the Chriſtians doe
condemne: not doubting but he was a mere man, and a Prieſt, and a King,
whoſoeuer he was. For els he could not in office and order and Sacrifice
haue been ſo perfect a type and reſemblance of our Sauiour, as in this
Chapter and other is shewed.}
Melchiſedech, the King of Salem, Prieſt of the God moſt high,
\CNote{\XRef{Gen.~14,~18.}}
who met Abraham returning from the ſlaughter of the Kings, and bleſſed
him: \V to whom alſo Abraham deuided tithes of al: firſt indeed by
%%% o-2714
interpretation,
\SNote{When the Fathers & Catholike Expoſitours pike out allegories and
myſteries out of the names of men the Proteſtãts not endowed with the
Spirit whereby the Scriptures were giuẽ, deride their holy labours in
the ſearch of the ſame: but the Apoſtle findeth high myſterie in the
very names of perſons & places, as you ſee.}
the King of iuſtice: & then alſo King of Salem, which is to ſay, King of
peace, \V
\LNote{Without father.}{Not
\MNote{The reſemblãce of Melchiſedech to Chriſt, in many points.}
that he was without father and mother, ſaith S.~Hierom
\Cite{ep.~136.}
for Chriſt himſelf was not without father, according to his diuinity,
nor without mother in his humanity: but for that his Pedegree is not ſet
out in the Geneſis, as the Genealogie of other Patriarches is, but is
ſodenly induced in the holy hiſtorie, no mention made of his ſtocke,
Tribe, beginning, or ending, and therfore in that caſe alſo reſembling
in a ſort the Sonne of God, whoſe generation was extraordinarie,
miraculous, and ineffable, according to both his natures, lacking a
father in the one, & a mother in the other: his Perſon hauing neither
beginning nor ending, & his Kingdom, & Prieſthood ſpecially, in himſelf
& in the Church, being eternal, both in reſpect of the time paſt, and
the time to come; as the ſaid Doctour in the
\Cite{ſame epiſtle}
writeth.}
without father, without mother, without Genealogie, hauing neither
beginning of daies nor end of life, but likened to the Sonne of God,
continueth a Prieſt for euer.

\V And
\LNote{Behold.}{To
\MNote{By the ſundrie excellencies of Melchiſedechs Prieſthood is proued
the excellẽcie of the Prieſthood and Sacrifice of the new Teſtamẽt.}
proue that Chriſtes Prieſthood farre paſſeth the Prieſthood of Aaron;
and the Prieſthood of the new Teſtament, the Prieſthood of the old law; &
conſequently that the Sacrifice of our Sauiour and the Sacrifice of the
Church doth much excel the Sacrifice of Moyſes law, he diſputeth
profoundly of the preeminences of Melchiſedech aboue the great Patriarch
Abraham who was father of the Leuites.}
behold how great this man is, to whom alſo Abraham the Patriarke gaue
\LNote{Tithes.}{The
firſt preeminence, that Abraham paied tithes, and that of the beſt and
moſt cheefe things that he had, vnto Melchiſedech, as a duty and homage,
not for himſelf only in perſon, but for Leui, who yet was not borne, and ſo
for the whole Prieſthood of Leuites ſtocke, acknowledging thereby,
Melchiſedech not only to be a Prieſt, but his Prieſt and Superiour, & ſo
of al the Leuitical order.
\MNote{He receiued tithes of Abrahã, and conſequẽtly of Leui and Aaron.}
And it is here to be obſerued, that wheras in the
\XRef{14.~of Geneſis}
whence this holy narration is taken, both in the Hebrew, and in the
70. it ſtandeth indifferent or doubtful whether Melchiſedech paied
tithes to Abraham or tooke tithes of him; the Apoſtle here putteth al
out of controuerſie, plainely declaring that Abraham paied tithes to the
other, as the inferiour to his Prieſt and Superiour.
%%% !!! MNote only in other
\MNote{Tithes.}
And touching
paiment of tithes, it is a natural duety, that men owe to God in al
lawes & to be giuẽ to his Prieſts in his behalfe, for their honour &
liuelihood. Iacob promiſed or vowed to pay them,
\XRef{Gen.~28.}
Moyſes appointed thẽ 
\XRef{Leuit.~27.}
\XRef{Nu.~18.}
\XRef{Deut.~12.}
\XRef{14.}
\XRef{26.}
Chriſt confirmeth that duety
\XRef{Mat.~23.}
and Abraham ſpecially here giueth thẽ to Melchiſedech  plainely thereby
approuing them or their equiualent to be due to Chriſt and the
Prieſthood of the new Teſtament, much more then either in the Law of
Moyſes, or in the Law of Nature. Of which tithes due to the Clergie of
Chriſtes Church ſee S.~Cypr.
\Cite{ep.~66.}
S.~Hierom
\Cite{ep.~1. c.~7.}
and
\Cite{ep.~2. c.~5.}
to Heliodorus and Neptianus, S.~Auguſtin
\Cite{ſer.~219 de tempore.}}
tithes of the principal things. \V And certes
\CNote{\XRef{Nu.~18,~21.}
\XRef{Deut.~18,~1.}
\XRef{Ioſ.~14,~4.}}
they of the ſonnes of Leui that take the prieſthood haue commandement to
take tithes of the people according to the Law, that is to ſay, of their
Brethren: albeit themſelues alſo iſſued out of the loines of Abraham. \V
But he whoſe Generation is not numbred among them, tooke tithes of
Abraham, and bleſſed him that had the promiſes. \V But without al
contradiction, that which is leſſe,
\LNote{Is bleſſed of the better.}{The
\MNote{He bleſſed Abraham.}
ſecond preeminence is, that Melchiſedech did bleſſe Abraham: which we
ſee here S.~Paul maketh a great and ſoueraigne holy thing, grounding our
Sauiours prerogatiue aboue the whole Order of Aaron therein: and we ſee
that in this ſort it is the proper act of Prieſthood:
\MNote{Bleſſing a great preeminence, ſpecially in Prieſts.}
and that without al controuerſie as the Apoſtle ſaith, he is greater in
dignitie, that hath authoritie to bleſſe, then the perſon that hath not,
and therfore the Prieſts vocation to be in this behalfe farre aboue any
earthly King, who hath no power to giue benediction in this ſacred
manner, neither to man, nor other creature. As here Melchiſedech, ſo
Chriſt bleſſed much more, and ſo haue the Bishops of his Church done,
and doe. Which no man can maruel that our Fore-fathers haue ſo highly
eſteemed and ſought for, if he marke the wonderful myſterie and grace
thereof here expreſſed. This Patriarch alſo which here taketh bleſſing
of Melchiſedech, himſelf (though in an inferiour ſort) bleſſed his
ſonnes, as the other Patriarches did, and fathers doe their children by
that example.}
is bleſſed of the better. \V And here indeed,
\SNote{The tithes giuen to Melchiſedech were not giuen as to a mere
mortal mã, as al of the Tribe of Leui & Aarons order were: but as to one
repreſenting the Sonne of God, who now liueth & reigneth & holdeth his
Prieſthood & the fũctions thereof for euer.}
men that die, receiue tithes: but there he hath witnes, that he
liueth. \V And (that it may ſo be ſaid) by Abraham Leui alſo, which
receiued tithes, was tithed. \V For as yet he was in his Fathers loines,
when Melchiſedech met him. \V
%%% !!! LNote not marked in either
\LNote{If conſummation.}{The
\MNote{The ful accõplishment of man's redemptiõ was not by Aaron, but by
Melchiſedechs Prieſthood.}
principal propoſition of the whole Epiſtle and al the Apoſtles
diſcourſe, is inferred and grounded vpon the former prerogatiues of
Melchiſedech aboue Abraham and Leui: that is, that the end, perfection,
accomplishment, and conſummation of ala man's dueties and debts to God,
by the general redemption, ſatisfaction, ful price and perfect ranſom of al
man-kind, was not atchieued by any or al the Prieſts of Aarõs Order, nor
by any Sacrifice or act of that Prieſthood; or of al the law of
Moyſes, which was grounded vpon the Leuitical Prieſthood, but by Chriſt
and his Prieſthood, which is of the Order and rite of Melchiſedech.}
If then conſummation was by the Leuitical
Prieſthood (for vnder it the people receiued the Law)
\LNote{What neceſsitie.}{This
\MNote{The Apoſtle to confute the Iewes falſe perſuaſion of Aarons
Prieſthood and Sacrifices, ſpeaketh altogether of the Sacrifice of the
Croſſe.}
diſputatiõ of the preeminẽce of Chriſtes Prieſthood aboue the Leuitical
Order, is againſt the erroneous perſuaſion of the Iewes, that thought
their law, Prieſthood, and Sacrifices to be euerlaſting, & to be
ſufficient in themſelues without any other Prieſt then Aaron and his
Succeſſours, and without al relation to Chriſtes Paſſion or any other
redemption or remiſſion, that that which their Leuitical offices did
procure: not knowing that they were al figures of Chriſtes death, and to
be ended and accomplished in the ſame. Which point wel vnderſtood and
kept in mind, wil cleere the whole controuerſie betwixt the Catholikes
and Proteſtants, concerning the Sacrifice of the Church. For, the ſcope
of the Apoſtles deputation being, to auouch the dignity, preeminence,
neceſſitie, and eternal fruit and effect of Chriſtes paſſion, he had not
to treat at al of the other, which is a Sacrifice depending of his
Paſſion, ſpecially writing to the Hebrewes, that were to be inſtructed &
reformed firſt touching the Sacrifice of the Croſſe before they could
fruitfully heare any thing of the other. Though in couert and by moſt
euident ſequele of diſputation, the learned and faithful may eaſily
perceiue whereupon the ſaid Sacrifice of the Church (which is the Maſſe)
is grounded. And therfore S.~Hierom ſaith,
\Cite{ep.~26.}
that al theſe commendations of Melchiſedech are in the type of Chriſt
\L{Cuius profectus Eccleſiæ ſacramenta ſunt.}}
what neceſſitie was there yet another Prieſt to riſe according to the
order of Melchiſedech, and not to be called according to the order of
Aaron? \V For the Prieſthood being
\LNote{Tranſlated.}{Note
\MNote{No lawful State of people without an external
\Fix{Prieſt.}{Prieſthood.}{obvious typo, fixed in other}}
wel this place, and you shal perceiue thereby, that euery lawful forme
and manner of law, ſtate, or gouernement of God's people dependeth on
Prieſthood; riſeth, ſtandeth, falleth, or altereth with the Prieſthood. In
the Law of Nature, the ſtate of the people hanged on one kind of
Prieſthood: in the law of Moyſes, of another: in the ſtate of
Chriſtianity, of another; & therfore in the former ſentence, the Apoſtle
ſaid, that the Iewish people or Common-wealth had their law vnder the
Leuitical Prieſthood, and the Greek more properly expreſſeth the matter,
that they were
\TNote{\G{νενομοθέτηται}}
\Emph{legitimated}, that is to ſay, made a lawful people, or communitie
vnder God, by the Prieſthood.
\MNote{External Prieſthood neceſſarie for the ſtate of the new
Teſtament.}
For there is no iuſt nor lawful Common-wealth in the world, that is not
made legal & God's peculiar, and diſtinguished from vnlawful
Common-weales that hold of falſe
\Fix{God's,}{goddes,}{obvious typo, fixed in other}
or of none at al, by Prieſthood. Whereupon it is cleere, that the new
law, & al Chriſtian peoples holding of the ſame, is made lawful by the
Prieſthood of the new Teſtament, and that the Proteſtants shamefully are
deceiued, and deceiue others, that would haue Chriſtian Common-weales to
lacke an external Prieſthood, or Chriſtes death to abolish the
ſame. For, this is a demonſtratiõ, that if Chriſt haue abolished
Prieſthood, he hath abolished the new Law, which is the new Teſtament &
ſtate of Grace, which al Chriſtian
\Fix{Common-weaths}{Common-wealths}{obvious typo, fixed in other}
liue vnder. Neither were it true, that the Prieſthood were tranſlated
with the Law, if al external Prieſthood ended by Chriſtes death, where
the new law began. For ſo the law should not depend on Prieſthood, but
dure whẽ al Prieſthood were ended: which is againſt S.~Paules doctrine.

Furthermore
\MNote{External Sacrifice alſo neceſſarie for the ſame.}
it is to be noted, that this legitimation or putting Communities vnder
law, & Prieſthood, of what Order ſoeuer, is no otherwiſe, but by ioyning
one with another in one homage of Sacrifice external, which is the
proper act of Prieſthood. For, as no lawful ſtate can be without
Prieſthood, ſo no Prieſthood can be without Sacrifice. And we meane
alwaies of Prieſthood & Sacrifice taken in their owne proper
ſignificatiõ, as here S.~Paul taketh them. For, the conſtitution,
difference, alteration, or tranſlation of ſtates & lawes riſe not vpon
any mutation of ſpiritual or metaphorically taken Prieſthood, or
Sacrifice: but vpon thoſe things in proper acception, as is moſt
plaine.

Laſtly,
\MNote{The tranſlatiõ of the old Prieſthood & Sacrifices, muſt needes be
into the ſaid Prieſthood & Sacrifice of the Church.}
it followeth of this, that though Chriſt truely ſacrificed himſelf vpon
the Croſſe, (there alſo a Prieſt according to the Order of
Melchiſedech) and there made the ful redemption of the world, confirmed,
and conſummated his compact, and Teſtament, and the law and Prieſthood
of this his new and eternal ſtate, by his bloud: yet that can not be the
forme of Sacrifice into which the old Prieſthood and Sacrifices were
tranſlated, whereupon the Apoſtle inferreth the tranſlation of the
Law. For they al were figures of Chriſtes death, and ended in effect at
his death, yet they were not altered into that kind of Sacrifice, which
was to be made but once at his death, and was executed in ſuch a ſort,
that peoples and Nations Chriſtned could not meet oftẽ to worship at it,
nor haue their law & Prieſts conſtituted in the ſame. Though for the
honour and duety, remembrance and repreſentation thereof, not only we
Chriſtians, but alſo al peoples faithful, both of Iewes & Gentils, haue
had their Prieſthood and Sacrifices according to the difference of their
ſtates. Which kind of Sacrifices were tranſlated one into another: and
ſo no doubt is the Prieſthood Leuitical properly turned into the
Prieſthood and Sacrifice of the Church, according to Melchiſedechs rite,
and Chriſtes inſtitution in the formes of bread and wine. See
\XRef{the next note.}}
tranſlated, it is neceſſarie that a tranſlation of the Law alſo be
made. \V For he on whom theſe things be ſaid, is of another Tribe, of
the which, none attended on the altar. \V For it is manifeſt that our
Lord ſprung of Iuda: in the which Tribe Moyſes ſpake nothing of 
\Var{Prieſtes.}{Prieſthood.}
\V And yet it is much more euident: if according to the ſimilitude of
Melchiſedech there ariſe another Prieſt, \V which was not made according
to the Law of carnal commandment, but according to the power of life
indiſſoluble. \V For he witneſſeth,
\CNote{\XRef{Pſ.~109,~4.}}
\Emph{That thou art
\LNote{A Prieſt for euer.}{Chriſt
\MNote{How Chriſt is a Prieſt for euer.}
is not called a Prieſt for euer, only for that his Perſon is eternal, or
for that he ſitteth on the right hãd of God, & perpetually praieth or
maketh interceſſion for vs, or for that the effect of his death is
euerlaſting: for al this proueth not that in proper ſignification his
Prieſthood is perpetual: but according to the iudgement of al the
Fathers grounded vpon this deep and diuine diſcourſe of S.~Paul, and
vpon the very nature, definition, and propriety of Prieſthood, and the
excellent act and Order of Melchiſedech, and the ſtate of the new law,
he is a Prieſt for euer according to Melchiſedechs Order,
\MNote{Chriſts eternal Prieſthood cõſiſteth in the perpetual Sacrifice
of his body and bloud in the Church.}
ſpecially in
reſpect of the Sacrifice of his holy body and bloud, inſtituted at his
laſt ſupper, and executed by his commiſſion, commandement, and perpetual
concurrence with his Prieſts, in the formes of bread and wine: In which
things only the ſaid high Prieſt Melchiſedech did Sacrifice. For though
S.~Paul make no expreſſe mention hereof, becauſe of the depth of the
myſterie, and their incredulity or feebleneſſe to whom he wrot: yet it
is euident in the iudgement of al the learned Fathers (without
exception) that euer wrot either vpon this Epiſtle, or vpon the 14.~of
Geneſis, or the Pſalme~109, or by occaſion haue treated of the Sacrifice
of the altar, that the eternity and proper act of Chriſtes Prieſthood,
and conſequently the immutabilitie of the new Law, conſiſteth in the
perpetual offering of Chriſtes body and bloud in the Church.

Which
\MNote{The Proteſtãts cauilling vpon particles, againſt Melchiſedechs
ſacrifice & Prieſthood directly agaĩſt the Apoſtle.}
thing is ſo wel knowen to the Aduerſaries of Chriſts Church and
Prieſthood, and ſo granted, that they be forced impudently to cauil vpon
certaine Hebrew particles, that Melchiſedech did not offer in bread and
wine: yea & when that wil not ſerue, plainly to deny him to haue been a
Prieſt: which is to giue check-mate to the Apoſtle, and ouerthrow al his
diſcourſe. Thus whiles theſe wicked men pretend to defend Chriſtes only
Prieſthood, they indeed abolish as much as in them lieth, the whole
Order, office, and ſtate of his eternal law and Prieſthood.

Arnobius
\MNote{Chriſts eternal Prieſthood and Sacrifice in the Church is proued
out of the Fathers.}
ſaith, \Emph{By the myſterie of bread and wine he was made a
Prieſt for euer.} And againe, \Emph{The eternal memorie, by which he
gaue the food of his body to them that feare him,}
\Cite{in Pſal.~109.}
\Cite{110.}
Lactanius, \Emph{In the Church he muſt needes haue his eternal
Prieſthood according to the Order of Melchiſedech.}
\Cite{Li.~14. inſtitut.}
S.~Hierom to Euagrius,
\CNote{Ep.~126.}
\Emph{Aarons Prieſthood had an end, but Melchiſedechs, that is Chriſtes
and the Churches is perpetual,
%%% !!! SNote in LNote
\SNote{That is from Adam to the end of the world, repreſented by
Sacrifice.}
both for the time paſt and to come.} S.~Chryſoſtom therfore calleth the
Churches Sacrifice, \L{hoſtiam inconſumptibilem}, \Emph{an hoſt or
Sacrifice that can not be conſumed}.
\Cite{ho.~27. in 9.~Hebr.}
S.~Cyprian, \L{hoſtiam qua ſublata, nulla eſſit futuræ
religio}, \Emph{an hoſt which being taken away, there could be no
religion.}
\Cite{de Cæna domini nu.~2.}
Emiſſenus, \L{perpetuam oblationem & perpetuo currentum redemptionem},
\Emph{A perpetual oblation and a redemption that runneth or continueth
euerlaſtingly.}
\Cite{ho.~5. de Paſch.}
And our Sauiour expreſſeth ſo much in the very inſtitution of the
B.~Sacrament of his body and bloud: ſpecially when he calleth the later
kind, \Emph{the new Teſtament in his bloud}, ſignifying that as the old
law was eſtablished in the bloud of beaſts, ſo the new (which is his
eternal Teſtamẽt) should be dedicated and perpetual in his bloud: not
only as it was shed on the Croſſe, but as giuen in the Chalice. And
therfore into this Sacrifice of the altar (ſaith S.~Auguſtin
\Cite{li.~17. ce Ciuit. c.~20.}
S.~Leo
\Cite{ſer.~8. de Paſsione,}
and the reſt) were the old ſacrifices to be tranſlated. See S.~Cyprian
\Cite{ep/~63. ad Cecil nu.~2.}
S.~Ambroſe
\Cite{de Sacram. li.~5. c.~4.}
S.~Auguſtin
\Cite{in Pſal.~33. conc.~2.}
and
\Cite{li.~17. de Ciuit. c.~17.}
S.~Hierom
\Cite{ep.~17. c.~2.}
&
\Cite{ep.~126.}
Epiph.
\Cite{hær.~55.}
Theodoret
\Cite{in Pſal.~109.}
Damaſcene
\Cite{li.~4. c.~14.}

Finally if any of the Fathers, or al the Fathers, had either wiſedom,
grace, or intelligence of Gods word and myſteries, this is the truth. If
nothing wil ſerue our Aduerſaries, Chriſt \Sc{Iesvs} confound them, and
defend his eternal Prieſthood, and ſtate of his new Teſtament
eſtablished in the ſame.}
a Prieſt for euer, according to the order of Melchiſedech.} \V
Reprobation certes is made
\LNote{Of the former commandement.}{The
\MNote{The old commandement & the new.}
whole law of Moyſes cõteining al their old Prieſthood, Sacrifice,
Sacraments, and ceremonies is called the \Emph{Old commandement}: and the
new Teſtament conteining the Sacrifice of Chriſtes body and bloud, and
al the Sacraments & graces giuen by the ſame, is named the \Emph{new
mandatum}:
\MNote{Maundy thurſday why ſo called.}
for which our forefathers called the Thurſday in the holy
week, \Emph{Maundy thurſday}, becauſe that in it, the new law and
Teſtament was dedicated in the Chalice of his bloud: the
old \Emph{mandatum}, law, Prieſthood, & Sacrifices, for that they were
inſufficient and vnperfect, being taken away: and this new Sacrifice,
after the order of Melchiſedech giuen in the place thereof.}
of the former commandement, becauſe of the weakeneſſe and
vnprofitableneſſe thereof. \V For the Law brought nothing to perfection,
but
%%% !!! LNote not marked in either
\LNote{The introduction.}{Euer
\MNote{The introduction of a new Prieſthood.}
obſerue, that the abrogation of the old law, is not an abolishing of al
Prieſthood, but an introduction of a new, conteining the hope of eternal
things, where the old had but temporal.}
an introduction of a better hope, by the which we approch to God. \V
And in as much as it is not without an othe, (the other truely without
an othe were made Prieſtes: \V but this
\LNote{With an othe.}{This
\MNote{The eternitie of the new Prieſthood cõfirmed by the Fathers othe
& Chriſts paſſion.}
othe ſignifieth the infallible and abſolute promiſe of the eternitie of
the new Prieſthood and ſtate of the Church. Chriſt by his death, and
bloud shed in the Sacrifice of the Croſſe, confirming it, ſealing it,
and making himſelf the ſurety & pledge thereof. For though the new
Teſtament was inſtituted, giuen and dedicated in the Supper, yet the
warrant, confirmation, and eternal operation thereof, was atchieued vpon
the Croſſe, in the one oblation and one general and euerlaſting redemption
there made.}
with an
%%% 2854
othe, by him that ſaid vnto him:
\CNote{\XRef{Pſ.~109,~4.}}
\Emph{Our Lord hath ſworne, & it shal not repent him: thou art a Prieſt
for euer})
%%% o-2715
\V by ſo much, is \Sc{Iesvs} made a ſuretie of a better Teſtament. \V And
the other indeed were made Prieſtes,
\LNote{Being many.}{The
\MNote{By the compariſon of many Prieſts, & one, is not meant, that there
is but one Prieſt of the new Teſtament.}
Proteſtants not vnderſtanding this place, feine very foolishly, that the
Apoſtle should make this difference betwixt the old ſtate and the new:
that in the old, there were many Prieſts, in the new, none at al but
Chriſt. Which is againſt the Prophet Eſay, ſpecially prophecying of the
Prieſts of the new Teſtamẽt (as S.~Hierom declareth
\Cite{vpon the ſame place)}
in theſe words,
\CNote{Eſa. c.~61.}
\Emph{You shal be called the
\TNote{\G{ἱερεῖς}}
Prieſts of God: the
\TNote{\G{λειτουργοὶ}}
Miniſters of our God, shal it be ſaid to you}: & it taketh away al
viſible Prieſthood, and conſequently the lawful ſtate that the Church
and Gods people haue in earth, with al Sacraments and external worship.

The
\MNote{The meaning is, that the abſolute Sacrifice of eternal redemptiõ
could not be done by thoſe many Aaronical Prieſts, but by one only,
Chriſt \Sc{Iesvs} who liueth a Prieſt for euer, hath no Succeſſour, and
as cheefe Prieſt, worketh & cõcurreth with al Prieſts in their prieſtly
functions.}
Apoſtle then meaneth firſt, that the abſolute Sacrifice of conſummation,
perfection, and vniuerſal redemption, was but one, once done, and by one
only Prieſt done, and therfore it could not be any of the Sacrifices, or
al the Sacrifices of the Iewes law, or wrought by any or by al of them,
becauſe they were a number at once, and ſucceeding one another, euery of
their offices and functions ending by their death, and could not worke
ſuch an eternal redemption as by Chriſt only was wrought vpon the
Croſſe. Secondly, S.~Paul inſinuateth thereupon that Chriſt neuer loſeth
the dignitie or practiſe of his eternal Prieſthood, by death nor
otherwiſe, neuer yealdeth it vp to any, neuer hath Succeſſours after
him, that may enter into his roome or right of Prieſthood, as Aaron
and al other had in the Leuitical Prieſthood, but that himſelf worketh
and concurreth with his Miniſters the Prieſts of the new Teſtament, in
al their actes of Prieſthood, as wel of Sacrifice as Sacrament,
bleſſing, preaching, praying, and the like what ſo-euer.

This therfore was the fault of the Hebrewes, that they did not
acknowledge their Leuitical Sacrifices and Prieſthood to be reformed and
perfited by Chriſtes Sacrifice of the Croſſe: and againſt them the
Apoſtle only diſputeth, and not againſt our Prieſts of holy Church, or
the number of them, who al confeſſe their Prieſthood and al exerciſes of
the ſame, to depend vpon Chriſtes only perpetual Prieſthood.}
being many, becauſe that by death they were prohibited to continue: \V
but this, for that he continueth for euer, hath an euerlaſting
prieſthood. \V Whereby he is able to ſaue alſo for euer
\Var{going}{them that goe}
by himſelf to God:
\SNote{Chriſt according to his humane nature praieth for vs, &
continually repreſenteth his former paſſion and merits to God the
Father.}
alwaies liuing to make interceſſion for vs.

\V For it was ſeemely that we ſhould haue ſuch a high Prieſt, holy, innocent,
impolluted, ſeparated from ſinners, and made higher then the Heauens. \V
Which hath not neceſſitie daily (as the Prieſtes) firſt
\CNote{\XRef{Leu.~9,~7.}
\XRef{16,~6.}}
for his owne ſinnes to offer Hoſtes, then for the peoples. For
\LNote{This did he once.}{This is the ſpecial preeminence of Chriſt,
that he offereth for other mens ſinnes only, hauing none of his owne to
offer for, as al other Prieſts both of the old and new law haue. And
this againe is the ſpecial dignitie of his owne Perſon, not communicable
to any other of what order of Prieſthood ſo-euer, that he by his death
(which is the only oblation that is by the Apoſtle declared to be
irreiterable in it-ſelf) paied the one ful ſufficient ranſom for the
redemption of al ſinnes.}
this he did once, in offering himſelf. \V For the Law appointeth
Prieſtes men that haue infirmitie: but the word of the othe which is
after the Law, the Sonne for euer perfected.


\stopChapter


\stopcomponent


%%% Local Variables:
%%% mode: TeX
%%% eval: (long-s-mode)
%%% eval: (set-input-method "TeX")
%%% fill-column: 72
%%% eval: (auto-fill-mode)
%%% coding: utf-8-unix
%%% End:

