%%%%%%%%%%%%%%%%%%%%%%%%%%%%%%%%%%%%%%%%%%%%%%%%%%%%%%%%%%%%%%%%%
%%%%
%%%% The (original) Douay Rheims Bible 
%%%%
%%%% New Testament
%%%% Hebrews
%%%% Chapter 08
%%%%
%%%%%%%%%%%%%%%%%%%%%%%%%%%%%%%%%%%%%%%%%%%%%%%%%%%%%%%%%%%%%%%%%

%%% Latin checked by KK.



\startcomponent chapter-08


\project douay-rheims


%%% 2858
%%% o-2718
\startChapter[
  title={Chapter 08}
  ]

\Summary{Out of the ſame
\XRef{Pſalme~109.}
he vrgeth this alſo,
%%% !!! Really Roman (upright) since Summary is already italic.
\Emph{Sit thou on my right hand}, shewing that the Leuitical tabernacle
on earth, was but a shadow of his true Tabernacle in heauen: without
which he should not be a Prieſt at al: 6.~whereas he is of a better
Prieſthood then they, as alſo he proueth by the excellencie of the new
Teſtament aboue the old.}

%%% o-2719
But the ſumme concerning thoſe things which be ſaid, is: We haue ſuch an
high Prieſt, who is ſet on the right hand of the ſeat of maieſtie in the
heauens. \V A
\SNote{Chriſt liuing &
\Fix{reiging}{reigning}{obvious typo, fixed in other}
in heauẽ continueth his prieſtly
function ſtil, & is Miniſter not of Moyſes Sancta & tabernacle, but of
his owne body & bloud, which be the true holies, and tabernacle not
formed by man, but by Gods owne hand.}
Miniſter of the Holies, and of the true tabernacle, which our Lord pight
and not man. \V For euery high Prieſt is appointed to offer guifts and
hoſts, wherfore it is
\LNote{Neceſſarie that he alſo.}{Euen
\MNote{Chriſts Prieſthood & Sacrifice is external, not ſpiritual, only.}
now being in heauen, becauſe he is a Bishop and Prieſt, he muſt needs
haue ſome-what to offer, and wherein to doe Sacrifice: and that not in
ſpiritual ſort only, for that could not make him a Prieſt of any
certaine Order. And it is moſt falſe and wicked to hold with the
Caluiniſts,
\CNote{Beza in ſchol. Teſt. Græcol. in c.~7. Heb.}
that Melchiſedechs Prieſthood was wholy ſpiritual. For then Chriſts
death was not a corporal, external, viſible, and truely named Sacrifice:
neither could Chriſt or Melchiſedech be any otherwiſe a Prieſt then
euery faithful man is: which to hold (as the Caluiniſts following their
owne doctrine muſt needs doe) is directly againſt the Scriptures, and no
leſſe againſt Chriſtes one oblation of his body vpon the Croſſe, then it
is againſt the daily Sacrifice of his body vpon the altar. Therfore he
hath a certaine hoſt in external and proper manner, to make perpetual
oblation thereby in the Church: for, viſible and external act of
ſacrifycing in heauen he doth
\Fix{not not}{not}{obvious typo, fixed in other}
exerciſe.}
neceſſarie that he alſo haue ſome thing that he may offer: \V
\LNote{If vpon the earth.}{It
\MNote{How Chriſtes body is made fit to be ſacrificed and eaten
perpetually.}
is by his death, and reſurrection to life againe, that his body is
become apt and fit in ſuch diuine ſort to be ſacrificed perpetually. For
if he had liued in mortal ſort ſtil, that way of myſtical repreſentation
of breaking his body and ſeparating the bloud from the ſame, could not
haue been agreable. And ſo the Church and Chriſtian people should haue
lacked a prieſthood and Sacrifice, and Chriſt himſelf should not haue
been a Prieſt of a peculiar Order, but either muſt haue offered in the
things that Aarons Prieſts did, or els haue been no Prieſt at al. For to
haue offered only ſpiritually, as al faithful men doe, that could not be
enough for his vocation, and our redemption, and ſtate of the new
Teſtament. How his flesh was made fit to be offered and eaten in the
B.~Sacrament, by his death, ſee Iſychius
\Cite{li.~1. in Leuit. cap.~2.}}
if thẽ he were vpon the earth, neither were he a Prieſt: whereas there
were that did offer guifts according to the Law, \V that
\TNote{\G{λατρεύουσι}.}
ſerue the exampler and ſhadow of
\LNote{Heauenly things.}{As
\MNote{\Emph{Kingdom of heauen & heauenly things}, ſpoken of the
Church.}
the Church or ſtate of the new Teſtament is commonly called \L{Regnum
cælorum & Dei}, in the Scriptures, ſo theſe heauenly things be probably
taken by learned men, for the myſteries of the new Teſtament. And it
ſeemeth that the paterne giuen to Moyſes to frame his tabernacle by,
was the Church, rather then the heauens themſelues: al S.~Paules
diſcourſe tending to shew the difference betwixt the new Teſtament and
the old, & not to make compariſon between the ſtate of heauen and the
old law. Though incidently, becauſe the condition of the new Teſtament
more neerly reſẽbleth the ſame thẽ the old ſtate doth, he ſometime may
ſpeake ſome-what therof alſo.}
heauenly things. As it was anſwered Moyſes, when he finiſhed the
tabernacle,
\CNote{\XRef{Exo.~25,~9.~40.}}
See (quoth he) that thou make al things according to the exampler which
was ſhewed thee in the mount.

\V But now he hath obtained a better miniſterie, by ſo much as he is
Mediatour of a better Teſtamẽt, which is eſtabliſhed in better
promiſes. \V For
\SNote{The promiſes and effects of the Law were temporal, but the
promiſes & effects of Chriſtes Sacramẽts in the Church be eternal.}
if that former had been void of fault, there ſhould not certes a place
of a ſecond been ſought. \V For blaming them, he ſaith:
\CNote{Hier.~31,~31.}
\Emph{Behold the daies shal come, ſaith our Lord: and I wil conſummate
vpon the houſe of Iſrael, and vpon the houſe of Iuda a new Teſtamẽt: \V
not according to the Teſtament which I made to their Fathers in the day
that I tooke their hand to bring them out of the land of Ægypt: becauſe
they did not continue in my Teſtamẽt: and I neglected them. ſaith our
Lord. \V For this is the Teſtament which I wil diſpoſe to the houſe of
Iſrael after thoſe daies, ſaith our Lord: Giuing my lawes
\LNote{Into their mind.}{This
\MNote{Grace, the effect of the new Teſtament.}
alſo and the reſt following is fulfilled in the Church, and is the
proper effect of the new Teſtament, which is the grace and ſpirit of
loue graffed in the harts of the faithful by the
\Fix{Goly}{Holy}{obvious typo, fixed in other}
Ghoſt, working in the Sacraments and Sacrifice of the new law to that
effect.}
into their mind, and in their hart wil I ſuperſcribe them, & I wil be
\LNote{Their God.}{Their
\MNote{The new Teſtament or couenãt between God & man.}
mutual couenant made betwixt God and the faithful, is that which was
dedicated and eſtablished, firſt in the chalice of his bloud, called
therfore
\CNote{\XRef{Luc.~22.}}
\Emph{the new Teſtament in his bloud}: and which was ſtraight after
ratified by the death of the Teſtatour, vpon the Croſſe.}
their God, and they shal be my people: \V and euery one
\LNote{Shal not teach.}{So
\MNote{Scriptures abuſed for phantaſtical inſpirations.}
it was in the primitiue Church, in ſuch ſpecially as were the firſt
founders of our new ſtate in Chriſt. And that which was verified in the
Apoſtles and other principal men, the Apoſtle ſpeaketh generally as
though it were ſo in the whole, as
\CNote{\XRef{Act.~2.}}
S.~Peter applieth the like out of Ioël, and
\CNote{\XRef{Io.~14. v.~12.}}
our Sauiour ſo ſpeaketh when he ſaith that ſuch as beleeue in him, shal
worke miracles of diuers ſorts. Chriſtian men then muſt not abuſe this
place to make chalenge of new inſpirations and ſo great knowledge that
they need no Scriptures or teaching in this life, as ſome Heretikes doe:
with much like reaſon and shew of Scriptures as the Proteſtants haue to
refuſe external Sacrifice. And it is no leſſe phantaſtical madneſſe to
deny external Sacrifice, Sacraments, or Prieſthood, then it is to
abolish teaching and preaching.}
shal not teach his neighbour, and euery one his brother, ſaying, Know
our Lord: becauſe al shal know me from the leſſer to the greater of
them: \V becauſe I wil be merciful to their iniquities, & their ſinnes I
wil not now remember.} \V And in ſaying a new, the former he hath made
old. And that which groweth ancient and waxeth old is nigh to vtter
decay.


\stopChapter


\stopcomponent


%%% Local Variables:
%%% mode: TeX
%%% eval: (long-s-mode)
%%% eval: (set-input-method "TeX")
%%% fill-column: 72
%%% eval: (auto-fill-mode)
%%% coding: utf-8-unix
%%% End:

