%%%%%%%%%%%%%%%%%%%%%%%%%%%%%%%%%%%%%%%%%%%%%%%%%%%%%%%%%%%%%%%%%
%%%%
%%%% The (original) Douay Rheims Bible 
%%%%
%%%% New Testament
%%%% Epistles
%%%% Hebrewes
%%%% Chapter 13
%%%%
%%%%%%%%%%%%%%%%%%%%%%%%%%%%%%%%%%%%%%%%%%%%%%%%%%%%%%%%%%%%%%%%%




\startcomponent chapter-13


\project douay-rheims


%%% 2875
%%% o-2736
\startChapter[
  title={Chapter 13}
  ]

\Summary{He commendeth vnto them mutual loue, 2.~hoſpitality,
  3.~compaſsion, 4.~chaſtitie, 5.~contentation, 7.~imitation of the
  faith of their Catholike Prelates and Martyrs (not harkning to the
  doctrines of Heretikes, nor fearing the caſting out of the Iewes
  Synagogue) 17.~and obedience to their preſent Paſtours. 18.~And ſo
  with requeſting their praiers, and praying for them, he endeth the
  Epiſtle.}

Let the
\TNote{\G{Ἡ φιλαδελφία}}
charitie of the fraternitie abide in you. \V And
\LNote{Hoſpitality.}{Hoſpitality,
\MNote{Hoſpitalitie.}
that is, receiuing & harbouring of poore pilgrimes, perſecuted and
deſolate perſons, is ſo acceptable to God and ſo honourable, that
\MNote{Angels harboured.}
often-times it hath been mens good hap to harbour Angels inſteed of
poore folke vnawares. Which muſt needs be euer a great benediction to
them and their families, as we ſee by Abraham and Lot
\XRef{Gen.~18.}
&
\XRef{19.}
(and the like fel alſo to S.~Gregorie, as Io. Diaconus writeth, to whoſe
ordinarie table of poore men, not only Angels but Chriſt alſo came in
pilgrimes weed.
\Cite{in vit. li.~1. c.~10.}
&
\Cite{li.~2. c.~22.~23.)}
whereof if we had not example and warrant by S.~Paules words in this
place, and many other expreſſe Scriptures of the old Teſtament, theſe
ſcorneful miſcreants of this time making ſo litle account both of good
workes and ſuch miraculous entrance of Chriſt and his Angels into holy
mens harbour, would make this alſo ſeeme fabulous, as they doe other
like things.}
hoſpitalitie doe not forget, for by this certaine being not aware,
\CNote{\XRef{Rom.~12,~10.}
\XRef{1.~Pet.~4.}
\XRef{Gen.~8,~3.}
\XRef{19,~2.~3.}}
haue receiued Angels to harbour. \V Remember them in bands, as if you
were bound with them; & them that labour, as your ſelues alſo remaining in
bodie. \V
\LNote{Marriage honourable.}{\Emph{The
\MNote{How marriage is honourable in al, if the Apoſtle did ſo ſay, as
he doth not.}
Apoſtle} (ſaith a holy Doctour) \Emph{ſaith, Marriage honourable in al,
and the bed vndefiled. And therfore the ſeruants of God in that they are
not married, thinke not the good of marriage to be a fault, but yet they
doubt not perpetual continencie to be better then good marriage,
ſpecially in this time when it is ſaid of continencie, He that can take,
let him take.}
\Cite{De fid. ad Pet. c.~3. apud Aug. in fine.}
Marke the doctrine of the Fathers and of the Catholike Church concerning
matrimonie, that it is honourable, and ſo honourable, that it is a holy
Sacrament, but yet
\CNote{1.~Cor.~7. v.~28.}
inferiour to virginity and perpetual continencie: honourable in al, that
is, al ſuch as may lawfully marrie and are lawfully married: not in
brother and ſiſter, not in perſons that haue vowed the contrarie, to
whom the ſame Apoſtle ſaith it is damnable.
\XRef{1.~Tim.~5. v.~11.}
And this were the meaning of this place, if it were to be read thus,
\Emph{Marriage is honourable.}

But
\MNote{One short place manifoldly corrupted by the Proteſtãts.}
to ſee how the Proteſtants in al their tranſlations, to abuſe the ſimple,
doe falſifie this ſentence of the Apoſtle, to make it ſerue for the
marriage of Votaries, it is notorious.
\MNote{They reſtraine the ſenſe to their Heretical fantaſie.}
Firſt, they vſe deceit in
ſupplying the verbe ſubſtantiue that wanteth, making it the Indicatiue
mood thus, \Emph{Marriage is honourable &c.} as though the Apoſtle
affirmed al marriage to be honourable or lawful, where the verbe to be
ſupplied ought rather to be the Imperatiue mode, \Emph{Let Marriage be
honourable}, that ſo the ſpeach may be an exhortation or commandement to
them that be or wil be married, to vſe themſelues in that ſtate in al
fidelity, cleanlineſſe, & coniugal cõtinencie one toward another: as whẽ
\CNote{\XRef{1.~Pet.~3.}}
S.~Peter alſo &
\CNote{\XRef{1.~Theſ.~4.}}
this Apoſtle exhort married men to giue honour to their wiues as to the
weaker veſſels, and to poſſeſſe their veſſel in honour not in the
paſſions of ignominie and vncleanlineſſe. This is honourable or chaſt
marriage, to which he here exhorteth. And that it is rather exhortation,
then an affirmation, it is euident by the other parts and circumſtances
of this place both before & after: al which are exhortation in their
owne tranſlations. This only being in the middes, & as indifferent to be
an exhortation as the reſt (by their owne confeſſion) they reſtraine of
purpoſe. Our text therfore & al Catholike tranſlations leaue the
ſentence indifferent
\TNote{\G{τίμιος ὁ γάμος ἐν πᾶσιν.}}
as it is in the Greek, and as true tranſlatours ought to doe, not
preſuming to addict it to one ſide, leſt they should reſtraine the ſenſe
of the Holy Ghoſt to their owne particular fantaſie.

Againe,
\CNote{\Cite{The Eng. Bib.~1577.}}
our new tranſlatours corrupt the text in that they tranſlate, \L{in
omnibus}, \Emph{among al men}, becauſe ſo they thinke it would ſound
better to the ignorant, that Prieſts, Religious, and al whoſoeuer, may
marrie: where they can not tel either by the Greek, or Latin, that
\L{in omnibus} should be the maſculine gendre, rather then the neutre
(as not only Eraſmus, but
\CNote{\Cite{Oecum. in collect.}}
the Greek Doctours alſo take it) to ſignifie that marriage should be
honourably kept between man & wife in al points and in al reſpects. See
\Cite{S.~Chryſoſtom}
&
\Cite{Theoph. in hunc locũ.}
For there may be many filthy abuſes in wedlocke, which the Apoſtle
warneth them to take heed of, and to keep their marriage-bed
vndefiled. But the third corruption for their purpoſe aforeſaid, and moſt
impudent, is,
\CNote{\Cite{Beza in no. Teſt. Græcolat. an.~1585.}}
that ſome of the Caluiniſtes for, \L{in omnibus}, tranſlate, \L{inter
quoſuis}, with a marginal interpretation to ſignifie al orders,
conditions, ſtates, and qualities of men. So boldly they take away al
indifferencie of ſenſes, and make Gods word to ſpeake iuſt that which
themſelues would, and their hereſie requireth, in which kind they paſſe
al impudencie and al heretikes that euer were.}
Marriage honourable in al, & the bed vndefiled. For fornicatours and
aduouterers God wil iudge. \V Let your manners be without auarice:
contented with things preſent. For he ſaid,
\CNote{\XRef{Deu.~31.}
\XRef{Ioſ.~1.}}
\Emph{I wil not leaue thee, neither wil I forſake thee.} \V So that we
doe confidently ſay:
\CNote{\XRef{Pſal.~55,~12.}
\XRef{117,~6.}}
\Emph{Our Lord is my helper: I wil not feare what man shal doe to me.}

\V
\LNote{Remember your Prelates.}{We
\MNote{We muſt haue regard to the faith and doctrine of the Fathers.}
be here warned to haue great regard in our life and beleefe, to the holy
Fathers, Doctours and glorious Bishops gone before vs in God's Church,
not doubting but they being our lawful Paſtours, had and taught the
truth: of whom S.~Auguſtin ſaid, \Emph{That which they found in the
Church, they held faſt: that which they learned, they taught: that which
they receiued of their Fathers, the ſame they deliuered to their
children.}
\Cite{Cont. Iulian. li.~2. c.~10.}
Which reſpect to our holy forefathers in faith, is now in this wicked
contempt of the Heretikes, ſo much the more to be had. See the ſaid holy
Doctours
\Cite{ſecond booke againſt Iulian the Pelagian}
throughout, what great account
\Fix{be}{he}{obvious typo, fixed in other}
maketh of them in the confutation of hereſies, and how farre he
preferreth thẽ aboue the proud Sectmaiſters of that time: as we muſt now
doe agaĩſt our new Doctours.
\MNote{Memories and feaſts of Saints.}
This place alſo is rightly vſed to proue that the Church of God should
keep the memories of Saints departed, by ſolemne holidaies & other
deuout waies of honour.}
Remember your Prelates, which haue ſpoken the word of God to you: the
end of whoſe conuerſation beholding, imitate their faith. \V
\Sc{Iesvs} Chriſt yeſterday, and to day: the ſame alſo for euer. \V With
\SNote{New, diuers, changeable, & ſtrange doctrines to be auoided, for
ſuch be heretical. Againſt which the beſt remedie or preſeruatiue is
alwaies to looke back to our firſt Apoſtles & the holy Fathers
doctrine.}
various & ſtrange doctrines be not led away. For it is beſt that
the hart be eſtablished with grace,
\LNote{Not with meats.}{He
\MNote{Iudaical abſtinence from meats.}
ſpeaketh not of Chriſtian faſts, but of the legal difference of meats,
which the Hebrewes were yet prone vnto, not conſidering that by Chriſtes
faith they were made free from al ſuch obſeruations of the Law.}
not with meats: which haue not profited thoſe that walke in them.

\V
\LNote{We haue an altar.}{He
\MNote{Material altars for the Sacrifice of Chriſtes body.}
putteth them in mind by theſe words, that in following too much their
old Iewish rites, they depriued themſelues of another manner and a more
excellent Sacrifice and meat: meaning, of the holy altar, and Chriſtes
owne bleſſed body offered and eaten there. Of which, they that continue
in the figures of the old Law, could not be partakers. \Emph{This
altar}, (ſaith Iſychius) \Emph{is the altar of Chriſtes body, which the
Iewes for their incredulity muſt not behold.}
\Cite{Li.~6. c.~21. in Leuit.}
And the
\TNote{\G{θυσιαστήριον}}
Greek word (as alſo the
\TNote{\H{מזבח}}
Hebrew, anſwering thereunto in the old Teſtament) ſignifieth properly an
altar to Sacrifice on and not a metaphorical and ſpiritual
altar. Whereby we proue againſt the Heretikes, that we haue not a common
table or profane cõmunion-bord, to eate meer bread vpon, but a very altar in
the proper ſenſe, to Sacrifice Chriſtes body vpon: and ſo called of the
Fathers in reſpect of the ſaid body ſacrificed.
\Cite{Greg. Nazianz. in orat. de ſorore. Gorgonia.}
\Cite{Chryſ. demmonſt. quid Chriſtus ſit Deus},
\Cite{Socrat. li.~1. c.~20.~25.}
\Cite{Aug. ep.~86.}
\Cite{De diu. Dei. li.~8. c.~27.}
&
\Cite{li.~21. c.~10.}
\Cite{Confeſſ. li.~9. c.~11.~13.}
\Cite{Cont. Fauſt. Manich. li.~20. c.~21.}
\Cite{Theophyl. in 13.~Mat.}
And when it is called a table, it is in reſpect of the heauenly food of
Chriſts body and bloud receiued.}
We haue an altar: whereof they haue not power to eate which ſerue the
tabernacle. \V For
\CNote{\XRef{Leu.~16,~27.}}
the bodies of thoſe beaſts, whoſe bloud for ſinne is caried into the
holies by the high Prieſt, are burned without the camp. \V For the which
thing \Sc{Iesvs} alſo, that he might ſanctifie the people by his owne
bloud, ſuffered without the gate. \V Let vs goe forth therfore to him
without the cãp; carying his reproche. \V For we haue not here a
permanẽt citie: but we ſeeke that which is to come. \V By him therfore
let vs offer
\LNote{The hoſt of praiſe.}{Though
\MNote{The Sacrifice of the altar is the principal hoſt of praiſe and
thankes-giuing, therfore called, \GG{Euchariſtia}.}
it may ſignifie the ſpiritual Sacrifice of praiſe and thãks-giuing of
what ſort ſoeuer: yet it ſpecially may be thought to ſignifie the great
Sacrifice of the B.~body and bloud of Chriſt: not as vpon the Croſſe,
which was but once done in bloudy ſort, but as in the Church and new
Teſtament, where it is daily done vnbloudily, being the proper hoſt of
laud and thankes-giuing and therfore called the \Emph{Euchariſt}, and
being the fruit and effect of Chriſt and his Prieſtes lips or words,
that is of conſecration. Becauſe this Sacrifice is made by the force of
the holy words. And when we read in the pſalme and other places of the
old Teſtament, of the hoſt of praiſe, it may be thought to be a
prophecie of the new Sacrifice, and not of euery vulgar
thankes-giuing. And ſo the old Fathers in the primitiue Church to hide
the myſteries from the vnworthy or heathen, often ſpeake. \Emph{What is}
(ſaith S.~Auguſtin) \Emph{a more holy Sacrifice of praiſe, then that
which conſiſteth in thankes-giuing, al which the faithful doe know in
the ſacrifice of the Church.}
\Cite{Li.~1. cont. aduerſ. leg. & proph. c.~18.}
Againe,
\Cite{c.~20.}
\Emph{The Church from the times of the Apoſtles by the moſt certaine
ſucceſsion of Bishops, offereth to God in the body of Chriſt the
Sacrifice of praiſe.} And a litle afterward: \Emph{Now Iſrael according
to the ſpirit, that is, the Church offereth a ſingular Sacrifice
according to the ſpirit: of whoſe houſe he wil not take calues nor
goats, but wil take the Sacrifice of praiſe, not according to the order
of Aarõ, but according to the order of Melchiſedech.} See
\Cite{ep.~120. c.~19.}
&
\Cite{ep.~57. ad q.~1. in fine.}
Thus you ſee, when the holy Fathers handle the Scriptures, they find
Maſſe and Sacrifice in many places, where the ignorant heretikes or the
ſimple might thinke they ſpeake only of a common thankes-giuing.}
the hoſt of praiſe alwaies to God, that is to ſay,
\CNote{\XRef{Oſe.~14,~3.}}
the fruits of lips confeſsing to his name.

%%% o-2737
\V And beneficence and communication doe not forget, for with ſuch
hoſtes
\LNote{God is promerited.}{This
\MNote{The Proteſtants auoid the word merit.}
latin word \L{promeretur}, can not be expreſſed effectually in any one
English word. It ſignifieth, Gods fauour to be procured by the foreſaid
workes of 
alme and charitie, as by the deſerts and merits of the doers. Which
doctrine & word of merits the Aduerſaries like ſo il, that they flye
both here and els-where from the word, trãſlating here for
\L{promeretur} \Emph{Deus, God is pleaſed}, more neere to the
\TNote{\G{εὐαρεστεῖται}}
Greek, as they pretend.
\MNote{Good workes meritouious.}
Which indeed maketh no more for them then the latin, which is agreable
to moſt ancient copies, as we ſee by
%%% !!! Cite?
Primaſius S.~Auguſtines ſcholer. For if God be pleaſed with good workes
and shew fauour for them, then they are meritorious, and then only faith
is not the cauſe of Gods fauour to men.}
God is promerited. \V
\LNote{Obey your Prelates.}{There
\MNote{The Apoſtle doth inculcate obedience to the Prieſts and Bishops
of Gods Church.}
is nothing more inculcated in the holy Scriptures, then obedience of
the lay people to the Prieſts and Prelates of Gods Church, in matters of
ſoule, conſcience, and religion. Whereof the Apoſtle giueth this reaſon,
becauſe they haue the charge of mens ſoules, and muſt anſwer for them:
which is an infinit preeminence and ſuperiority, ioyned with burden, and
requireth maruelous ſubmiſſion and moſt obedient ſubiection of al that
be vnder them and their gouernement. From this obedience there is no
exception nor exemption of Kings nor Princes, be they neuer ſo great.
\MNote{No perſon exempted from this obediẽce, in matters of religion.}
If they haue ſoules, and be Chriſtian men, they muſt be ſubiect to ſome
Bishop, Prieſt, or other Prelate. And whatſoeuer he be (though Emperour
of al the world) if he take vpon him to preſcribe and giue lawes of
religion to the Bishops and Prieſts, whom he ought to obey and be
ſubiect vnto in religiõ, he shal be damned vndoubtedly, except he
repent, becauſe he doth againſt the expreſſe word of God and law of
nature. And by this you may ſee the difference of an heretical and a
diſordered time, from other Catholike Chriſtian daies. For hereſie and
the like damnable reuolts from the Church of God, is no more but a
rebellion and diſobedience to the Prieſt of Gods Church, when men refuſe
to be vnder their diſcipline, to heare their doctrine, and interpretation of
Scriptures, to obey their lawes and counſels. This diſobedience and
rebellion from the 
Spiritual Gouernour, vnder pretence of obedience to the Temporal, is the
bane of our daies, and ſpecially of our Countrie, where theſe new Sects
are properly mainteined by this falſe principle, That the Prince in
matters of ſoule and religion may command the Prelate: which is directly
and euidently againſt this Scripture and al other, that command the
sheep of Chriſtes fold to obey their ſpiritual Officers.}
Obey your Prelates, and be ſubiect to them. For they watch as being to
rẽder acount for your ſoules: that they may doe this with ioy, and not
mourning. For this is not expedient for you. \V Pray for vs. For we haue
confidence that we haue a good conſcience, willing to conuerſe wel in
al. \V And I beſeech you the more to doe this, that I may the more
ſpeedily be reſtored to you. \V And the God of peace which brought out
from the dead the great Paſtour of the sheep, in the bloud of the
eternal teſtament, our Lord \Sc{Iesvs} Chriſt, \V
\SNote{\G{καταρτίσαι}, that is, \Emph{make you perfect and abſolute in
al goodnes}.}
\TNote{\L{aptet vos}}
fit you in al goodnes, that you may doe his wil doing in you that which
may pleaſe before him by \Sc{Iesvs} Chriſt: to whom is glorie for euer
and euer. Amen.

%%% 2876
\V And I deſire you, Brethren, that you ſuffer the word of
conſolation. For in very few words haue I written to you. \V Know you
our brother Timothee to be diſmiſſed: with whom (if he come the ſooner)
I wil ſee you. \V Salute al your Prelates, and al the Saints. The
Brethren of Italie ſalute you. \V Grace
\Fix{he}{be}{obvious typo, fixed in other}
with you al. Amen.


\stopChapter


\stopcomponent


%%% Local Variables:
%%% mode: TeX
%%% eval: (long-s-mode)
%%% eval: (set-input-method "TeX")
%%% fill-column: 72
%%% eval: (auto-fill-mode)
%%% coding: utf-8-unix
%%% End:

