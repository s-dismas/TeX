%%%%%%%%%%%%%%%%%%%%%%%%%%%%%%%%%%%%%%%%%%%%%%%%%%%%%%%%%%%%%%%%%
%%%%
%%%% The (original) Douay Rheims Bible 
%%%%
%%%% New Testament
%%%% Hebrews
%%%% Chapter 06
%%%%
%%%%%%%%%%%%%%%%%%%%%%%%%%%%%%%%%%%%%%%%%%%%%%%%%%%%%%%%%%%%%%%%%




\startcomponent chapter-06


\project douay-rheims


%%% 2851
%%% o-2711
\startChapter[
  title={Chapter 06}
  ]

%%% !!! There appear to be some marginal notes that I cannot read in
%%% this, missing in other.
\Summary{He exhorteth them to be perfect ſcholers, and not to need to be
  Catechumens againe, 4.~conſidering they can not be baptized againe:
  9.~and remembring their former good workes, for the which God wil not
  faile to performe them his promiſe, if they faile not to imitate
  Abraham by perſeuerance in the faith with patience. 20.~And ſo endeth
  his digreſsion, and returneth to the matter of Chriſtes Prieſthood.}

Wherfore intermitting the word of the beginning of Chriſt, let vs
proceed to perfection, not againe laying
\LNote{The foundation of penance.}{We
\MNote{The Apoſtles forme of Catechiſme, and the points thereof.}
ſee hereby, what the firſt grounds of Chriſtian inſtitution or
Catechiſme were in the Primitiue Church, and that there was euer a
neceſſarie inſtruction and beleefe of certaine points had by word of
mouth and tradition, before men came to the Scriptures: which could not
treat of things ſo particularly, as was requiſit for the teaching of al
neceſſarie grounds. Among theſe points were the 12.~Articles conteined
in the Apoſtles Creed: the doctrine of penance before Baptiſme: the
mãner and neceſſitie of Baptiſme: the Sacrament of Impoſition of hands
after Baptiſme, called Confirmation: the
\Fix{aricles}{articles}{obvious typo, fixed in other}
of the Reſurrection,
Iudgement, and ſuch like. Without which things firſt laid, if one should
be ſent to picke faith out of the Scripture, there would be madde rule
quickly.  See S.~Auguſtin
\Cite{in expoſit. inchoat. ep. ad Rom. verſus finem.}}
the foundation of penance from dead workes, & of faith toward God, \V of
the doctrine of Baptiſmes, & of impoſition of hands, & of the
reſurrection of the dead, & of eternal iudgment. \V And this ſhal we
doe, if God wil permit. \V For
\CNote{\XRef{Heb.~10,~26.}}
it is
\LNote{Impoſsible.}{How
\MNote{The Nouatians (as al Heretikes) made Scripture the groũd of their
hereſies.}
hard the holy Scriptures be, and how dangerouſly they be read of the
vnlearned, or of the proud be they neuer ſo wel learned, this one place
might teach vs.
\CNote{\Cite{Ambr. de pænit. li.~2. c.~2.}}
Wherat the Nouatians of old did ſo ſtumble, that they thought, &
heretically taught that none, falling into any mortal ſinne after
Baptiſme, could be receiued to mercie or penance in the Church: and ſo
to a contentious man, that would follow his owne ſenſe, or the bare
words, without regard of the Churches ſenſe and rule of faith (after
which euery Scripture muſt be expounded) the Apoſtles ſpeach doth here
ſound.
\MNote{Other places make no more for the Proteſtants then this doth for
Nouations.}
Euen as to the ſimple, and to the Heretike that ſubmitteth not his
ſenſe to the Churches iudgement, certaine place of this ſame Epiſtle
ſeeme at the firſt ſight, to ſtand againſt the daily oblation or
Sacrifice of the Maſſe: which yet in truth make no more for that
purpoſe, then this text we now ſtand on, ſerueth the Nouations: as when
we come to the places, it shal be declared.

And
\MNote{Caluins hereſie vpõ this place, worſe then the Nouatians.}
let the good Readers beware here alſo of the Proteſtants expoſition, for
they are herein worſe then Nouatians, ſpecially ſuch as preciſely follow
Caluin; holding impiouſly, that it is impoſſible for one that forſaketh
entirely his faith, that is, becommeth an Apoſtata or an Heretike, to be
receiued to penance or to God's mercie. To eſtablish which falſe and
damnable ſenſe, theſe fellowes make nothing of 
\CNote{\Cite{Ambr. loce cit.}
&
\Cite{in ep. ad Heb.}}
S.~Ambroſe's,
\CNote{\Cite{Chry. ho.~9. in c.~6. ad Heb.}}
S.~Chryſoſtom's, and other Fathers expoſitions, which is the holy
Churches ſenſe,
\MNote{The fathers expoſition of this place.}
That the Apoſtle meaneth of that penance which is done before and in
Baptiſme. Which is no more to ſay, but that it is impoſſible to be
baptized againe, and thereby to be renouated and illuminated, to die, be
buried, and riſe againe the ſecond time in Chriſt, in ſo eaſie and
perfect penance and cleanſing of ſinnes, as that firſt Sacrament of
generation did yeald: which applieth Chriſtes death in ſuch ample manner
to the receiuers, that it taketh away al paines due for ſinnes before
committed: and therfore requireth no further penance afterward, for the
ſinnes before committed, al being washed away by the force of that
Sacrament duely taken. S.~Auguſtin calleth the remiſſion in Baptiſme,
\L{Magnam indulgentiam}, a great pardon.
\Cite{Encir. c.~64.}

The
\MNote{The Sacramẽt of penance is ready for al ſinners whatſoeuer.}
Apoſtle therfore warneth them, that if they fal from their faith, and
from Chriſt's grace and Law which they once receiued in their Baptiſme,
they may not looke to haue any more that firſt great and large remedie
applied vnto them, nor no man els that ſinneth after Baptiſme: though
the other penance, which is called the
\CNote{\Cite{Hier. ep.~8. ad Demetriad. c.~6.}}
\Emph{Second table after
shipwracke}, which is a more paineful medicine for ſinne then Baptiſme,
requiring much faſting, praying, and other afflictions corporal, is open
not only to other ſinners, but to al once baptized, Heretikes, or
oppugners of the truth malitiouſly, and of purpoſe, or what way ſo-euer,
during this life. See S.~Cyprian
\Cite{ep.~52.}
S.~Ambroſe
\Cite{vpon this place.}
S.~Auguſtin
\Cite{cont. ep. Parm. li.~2. c.~13.}
and
\Cite{ep.~50.}
S.~Damaſcus
\Cite{li.~4. c.~16.}}
impoſſible for them that were once illuminated, haue taſted alſo the
heauenly guift, & were made pertakers of the holy Ghoſt, \V haue
more-ouer taſted the good word of God, & the powers of the world
%%% o-2712
to come, \V and are fallen, to be renewed againe to penance, crucifying
againe to thẽſelues the Sonne of God, and making him a mockerie. \V For
the earth drinking the raine often comming vpon it, & bringing forth
graſſe commodious for them by whom it is tilled, receiueth bleſſing of
God. \V But bringing forth thornes and bryers, it is reprobate, and very
neer a curſe, whoſe end is, to be burnt.

\V But
\SNote{It is euident by theſe wordes, againſt the Nouatians and the
Caluiniſts, that S.~Paul meant not preciſely, that they had done, or
could doe any ſuch ſinne, whereby they should be put out of al hope of
ſaluation, & be ſure of damnation, during their life.}
we confidently truſt of you, my beſt Beloued, better things and neerer
to ſaluation; although we ſpeake thus. \V For
\LNote{God is not vniuſt.}{It
\MNote{Gods iuſtice in rewarding meritorious workes.}
is a world to ſee, what wringing & writhing the Proteſtãts make to shift
themſelues from the euidence of theſe words, which make it moſt cleere
to al not blinded in pride and contention, that good workes be
meritorious, and the very cauſe of ſaluation, ſo farre that God should
be vniuſt, if he rendered not Heauen for the ſame. \L{Reuera grandis
iniuſtitia Dei} (ſaith Hierom) \L{ſi tantum peccata puniret, & bona opera
non ſuſciperet.} That is, \Emph{Indeed great were God's iniuſtice, if he
would only punish ſinnes, and would not receiue good workes.}
\Cite{Li.~2. cont. Iouin. c.~2.}}
God is not vniuſt, that he ſhould forget your worke & loue which you
haue ſhewed in his name, which haue miniſtred to the Saints and doe
miniſter. \V And our deſire is that euery one of you ſhew forth the ſame
carefulneſſe to the accompliſhing of hope vnto the end: \V that you
become not ſlouthful, but imitatours of them which by faith and patience
ſhal inherit the promiſes. \V For God promiſing to Abraham, becauſe he
had none greater by whom he might ſweare, he ſware by himſelf, \V ſaying:
\CNote{\XRef{Gen.~22,~16.}}
Vnles bleſſing I ſhal bleſſe thee, and multiplying ſhal multiplie
thee. \V And ſo patiently enduring he obtained the promiſe. \V For men
ſweare by a greater then themſelues: and the end of al their
controuerſie, for the cõfirmatiõ, is an oth. \V Wherein God meaning more
aboundantly to ſhew to the heires of the promiſe the ſtabilitie of his
counſel, he interpoſed
\Fix{on}{an}{obvious typo, fixed in other}
othe: \V that by two things vnmoueable, wherby it is impoſſible for God
to lie, we may haue a moſt ſtrong comfort. Who haue fled to hold faſt
the hope propoſed, \V which we haue as an anker of the ſoule, ſure and
firme, and going in into the inner parts of the veile, \V
where \Sc{Iesvs} the Precurſour for vs is intred, made a high Prieſt for
euer according to the order of Melchiſedech.


\stopChapter


\stopcomponent


%%% Local Variables:
%%% mode: TeX
%%% eval: (long-s-mode)
%%% eval: (set-input-method "TeX")
%%% fill-column: 72
%%% eval: (auto-fill-mode)
%%% coding: utf-8-unix
%%% End:

