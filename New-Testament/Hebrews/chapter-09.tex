%%%%%%%%%%%%%%%%%%%%%%%%%%%%%%%%%%%%%%%%%%%%%%%%%%%%%%%%%%%%%%%%%
%%%%
%%%% The (original) Douay Rheims Bible 
%%%%
%%%% New Testament
%%%% Hebrews
%%%% Chapter 09
%%%%
%%%%%%%%%%%%%%%%%%%%%%%%%%%%%%%%%%%%%%%%%%%%%%%%%%%%%%%%%%%%%%%%%




\startcomponent chapter-09


\project douay-rheims


%%% 2860
%%% o-2720
\startChapter[
  title={Chapter 09}
  ]

\Summary{In the old Teſtament, that ſecular ſanctuarie had two partes:
  the one ſignifying that time, with the ceremonies therof for the
  emundation of the flesh: the other ſignifying heauen, which then was
  shut, vntil our High Prieſt Chriſt entred into it, & that with his
  owne bloud, shed for the emundation of our conſciences. Whereupon he
  concludeth the excellencie of his tabernacle and hoſt aboue the
  old. 25.~Noting alſo the difference, that he entred but once (ſo
  effectual was that one bloudy offering of himſelf, for euer) whereas
  the Leuitical High Prieſt entred euery yeare once.}

%%% o-2721
The former alſo indeed had iuſtification
\CNote{\XRef{Exo.~25.}
\XRef{26.}
\XRef{1,~36.}}
\TNote{\G{λατρείας}}
of ſeruice, and a ſecular ſanctuarie. \V For the tabernacle was made,
the firſt, wherein were the candleſtickes, and the table, and the
propoſition of loaues, which is called Holy. \V But after the ſecond
veile, the tabernacle, which is called \L{Sancta Sanctorum}: \V hauing a
golden cenſar, and the arke of the Teſtament couered about on euery part
with gold, in the which was
\LNote{A golden pot.}{The
\MNote{Relikes.}
Proteſtants count it ſuperſtitious to keep with honour & reuerence the
holy memories or monuments of Gods benefits & miracles, or the tokens of
Chriſts Paſſion, as his Croſſe, garments, or other things appertaining
to him or his Saints, and thinke it impoſſible that ſuch things should
dure ſo long:
\MNote{They continue without putrefaction.}
when they may here ſee the reuerent & long reſeruation of Manna, which
of it-ſelf was moſt apt to putrifie, and of Aarons rod, only for that it
ſodenly flourished by miracle, the tables of the Teſtament &c. See a
notable place in S.~Cyril
\Cite{li.~6. cont Iulian,}
where he defendeth againſt Iulian the Apoſtataes blaſphemie,
\MNote{The holy \Sc{Crosse}.}
the keeping and honouring of that Croſſe or wood which Chriſt died
on. See alſo S.~Paulinus
\Cite{ep.~11.}
&
\MNote{The ſepulchres of Chriſt and his Saints.}
what reuerence S.~Hierom and the faithful of his time did to the
ſepulchres of Chriſt and his Martyrs, & to their relikes.
\CNote{Ep.~17. c.~5.}
\Emph{We reuerence and worship} (ſaith he) \Emph{euery-where Martyrs
ſepulchres, and putting the holy ashes to our eyes, if we may we touch
it with our mouth alſo; and doe ſome thinke, that the monument wherein
our Lord was buried, is to be neglected?} But our Proteſtants can not
skil of this. They had rather follow Vigilantius, Iulianus the Apoſtata,
and ſuch Maiſters, then the holy Doctours and euident practiſe of the
Church in al Ages.}
a golden pot hauing Manna, and the rod of Aaron that had bloſſomed, and
\CNote{\XRef{3.~Reg.~8.}
\XRef{2.~Par.~5.}}
the tables of the Teſtament, \V and ouer it were
\CNote{Exo.~25,~22.}
the
\LNote{Cherubins.}{You
\MNote{Images in Salomons temple commanded by God.}
ſee it is a fond thing to conclude vpon the firſt or ſecond
commandement, that there should be no ſacred images in the Church, when
euen among theſe people that were moſt prone to idolatrie, and groſſe in
imagination of ſpiritual things ſuch as Angels are, & to whõ the precept
was ſpecially giuẽ, the ſame God that forbad thẽ grauẽ Idols, did
command theſe images of Angels to be made & ſet in the ſoueraigne
holieſt place of al the tabernacle or Temple. By which it is plaine,
that much more the images of Chriſt and his B.~Mother & Saints, that may
be more truely pourtered then mere ſpiritual ſubſtances can be, are not
contrarie to Gods commandement, nor againſt his honour, or repugnant to
any other Scripture at al, which condemne only the Idols or
pourtraitures of the Heathen made for adoration of falſe Gods.}
Cherubins of glorie ouer-ſhadowing the propitiatorie, of which things it
is not needful to ſpeake now particularly. \V But theſe things being ſo
ordered, in the firſt tabernacle indeed the Prieſts alwaies entred,
accompliſhing offices of the Sacrifices. \V But in the ſecond,
\CNote{\XRef{Exo.~30,~10.}
\XRef{Leu.~16,~2.}
\XRef{30.}}
once a yeare the high Prieſt only: not without bloud which he offereth
for his owne and the peoples ignorance: \V the Holy Ghoſt ſignifying
this, that the way of the Holies was
\SNote{The way to heauen was not open before Chriſts paſſion; & therfore
the Patriarches & good men of the old Teſtament were in ſome other place
of reſt vntil then.}
not yet manifeſted, the former tabernacle as yet ſtanding. \V Which is a
\SNote{Al things done in the old Teſtament and prieſthood were figures
of Chriſtes actiõs.}
parable of the time preſent: according to which are offered guifts and
hoſts, which can not concerning the conſcience make perfect
\TNote{\G{τὸν λατρεύοντα}}
him that ſerueth, \V only in meats and in drinkes, and diuerſe
baptiſmes, and iuſtices of the fleſh laid on them
\LNote{Vntil the time of correction.}{Al thoſe groſſe and carnal
Sacrifices, ceremonies, and obſeruations inſtituted to cleanſe and
purifie the flesh from legal irregularities & impurities only, & not
reaching to the purging of the ſoules & conſciences of men, being
commanded not for euer, but til Chriſtes comming, ceaſed then: and
better, more forcible, and more ſpiritual Sacraments were inſtituted in
their place.
\MNote{Sacrifice not taken away by the new Teſtamẽt, but changed into a
better.}
For we may not imagin Chriſt to haue taken away the old, and put none in
their place: or to alter the Sacramẽts only into other Sacraments
external, and not alſo to tranſlate the Sacrifices to ſome other more
excellent. For it is called, \L{tempus correctionis, non abolitiones
Sacrificij aut legis}: \Emph{the time of correction not of abolishing
Sacrifice or Law.} Neither haue they more reaſon to affirme Chriſtes one
oblation vpon the Croſſe to haue rather taken away al kind of Sacrifice,
then al manner of Sacraments. The time and ſtate of the new Teſtament is
not made lawleſſe, hoſtleſſe, or without Sacrifice, but it is the time
of correction or reformation and abettering al the foreſaid things.}
vntil the time of correction.

\V But Chriſt aſſiſting an high Prieſt of the good things to come, by a
more ample and more perfect tabernacle not made with hand, that is, not
of this creation: \V neither by the bloud of goats or of calues, but by
his owne bloud entred in once into the Holies,
\LNote{Eternal redemption.}{No
\MNote{One only Sacrifice on the Croſſe the redẽption of the world: &
one only Prieſt (Chriſt) the Redemer thereof.}
one of the Sacrifices, nor al the Sacrifices of the old law, could make
that one general price, ranſom, and redemption of al mankind, and al
ſinnes, ſauing this one higheſt Prieſt Chriſt, and the one Sacrifice of
his bloud once offered vpon the Croſſe. Which Sacrifice of redemption
can not be often done, becauſe Chriſt could not die but once. Though the
figures alſo therof in the law of nature & of Moyſes, were truely called
Sacrifices, as ſpecially this high and maruelous commemoration of the
ſame in the holy Sacrament of the altar, according to the rite of the
new Teſtament, is moſt truely and ſingularly (as S.~Auguſtin
\CNote{Li. de Sp. & lit. c.~11.}
calleth it) a Sacrifice. But neither this ſort, nor the other of the old
law, being often repeated and done by many Prieſts (al which were and
are ſinners themſelues) could be the general redeeming and conſummating
Sacrifice: nor any one of thoſe Prieſts, nor al the Prieſts together,
either of the law of Nature, or of Aarons, or Melchiſedechs Order
(except Chriſt alone) could be the general Redeemers of the world.

And
\MNote{The Apoſtles diſputation being only
\Fix{agaiſt}{againſt}{obvious typo, fixed in other}
the errour of the Iewes concerning their Sacrifices and Prieſts: the
Proteſtants applying it againſt the Sacrifice of the Maſſe and Prieſts
of the new Teſtamẽt.}
this is the Apoſtles meaning in al this compariſon and oppoſition of
Chriſtes death to the old Sacrifices, and of Chriſt to their Prieſts:
and not that Chriſtes death or Sacrifice of the Croſſe should take away
al Sacrifices, or proue that theſe Aaronical offices were no true
Sacrifices at al, nor thoſe Prieſts, verily Prieſts. They were true
Prieſts & true Sacrifices, though none of thoſe Sacrifices were the
high, capital, and general Sacrifice of our price and redemption: nor
none of them, or thoſe Prieſts, could without reſpect to this one
Sacrifice of Chriſtes death, worke any thing to Gods honour, or
remiſſion of ſinnes, as the Iewes did falſely imagin, not referring them
at al to this general redemption and remiſſion by Chriſt, but thinking
them to be abſolute Sacrifices in themſelues. And that to haue been the
errour of the Hebrues, you may read in S.~Auguſtin
\Cite{li.~3. doct. Chriſt. c.~6.}
And this, we tel the Proteſtants, is the only purpoſe of the Apoſtle.

But they be ſo groſſe, or ignorant in the Scriptures, and ſo malitiouſly
ſet againſt Gods and the Churches truth, that they peruerſely and
foolishly turne the whole diſputation againſt the Sacrifice of the
B.~Maſſe, & the Prieſts of the new Teſtamẽt: as though we held, that
the Sacrifice of the altar were the general redemption or redeeming
Sacrifice, or that it had not relation to Chriſtes death, or that it were
not the repreſentation and moſt liuely reſemblance of the ſame, or were
not inſtituted and done, to apply in particular to the vſe of the
partakers, that other general benefit of Chriſtes one oblation vpon the
Croſſe. Againſt the Iewes then only S.~Paul diſputeth, and againſt the
falſe opinion they had of their Prieſts and Sacrifices, to which they
attributed al remiſſion and redemption, without reſpect of Chriſtes
death.}
eternal redemptiõ being found. \V For
\CNote{\XRef{Leu.~9,~8.}
\XRef{16,~6.}
\XRef{14.}
\XRef{Nu.~19.}}
if the bloud of goats and of oxen and the aſhes of an heifer being
ſprinkled, ſanctifieth the polluted to the cleanſing of the fleſh, \V
how much more
\Var{hath}{shal}
the bloud of Chriſt who by the Holy Ghoſt offered himſelf vnſpotted vnto
God,
\Var{cleanſed}{cleanſe}
our conſcience from dead workes, to ſerue the liuing God? \V And
therfore he is the Mediatour of the new Teſtament: that death being a
meane, vnto the redemption
\LNote{Of thoſe preuarications.}{The Proteſtãts doe vnlearnedly imagin,
that becauſe al ſinnes be remitted by the force of Chriſtes paſſiõ, that
therfore there should be no other Sacrifice after his death. Whereas
indeed they might as wel ſay, there ought neuer to haue been Sacrifice
appointed by God, either in the law of Nature, or of Moyſes: as al their
argumẽts made againſt the Sacrifice of the Church vpõ the Apoſtles
diſcourſe, proue as wel, or rather only, that there were no Sacrifices
of Aarõs Order or Leuitical law at al. For againſt the Iewes falſe
opinion concerning them, doth he diſpute, and not a word touching the
Sacrifice of the Church, vnto which in al this diſcourſe he neuer
oppoſeth Chriſtes Sacrifice vpon the Croſſe: al Chriſtian men wel
knowing that the hoſt and oblation of thoſe two, though they differ in
manner and external forme, yet it is indeed al one.

The Apoſtle then sheweth here plainely, that al the ſinnes that euer
were remitted ſince the beginning of the world, were no otherwiſe
forgiuen, but by the force and in reſpect of Chriſtes Paſſion. Yet it
followeth not thereupon, that the oblations of Abel, Abraham, Aaron,
&c. were no Sacrifices, as by the Heretikes foolish deduction it should
doe: S.~Paul not oppoſing Chriſtes Paſſion to them, for the intent to
proue them to haue been no Sacrifices, but to proue, that they were not
abſolute Sacrifices, nor the redeeming or conſummating Sacrifice, which
could not be many, nor done by many Prieſts, but by one, and at one
time, by a more excellent Prieſt then any of them, or any other mere
mortal man.

And
\MNote{Caluins argumẽt againſt the Sacrifice of the altar, maketh no
leſſe againſt the Sacrifices of the old Law.}
that you may ſee the blaſphemous pride and ignorance of Caluin, and in
him, of al his fellowes: read (ſo many as may read Heretical bookes) his
cõmentarie
\Cite{vpon this place,}
and there you shal ſee him gather vpõ this that Chriſtes death had force
from the beginning and was the remedie for al ſinnes ſince the creation
of the world, therfore there muſt be no moe but that one Sacrifice of
Chriſtes death. Which muſt needes by his deduction hold (as it doth
indeed) no leſſe againſt the old Sacrifices then the new Sacrifice of
the Church, and ſo take away al, which is againſt the Apoſtles meaning
and al religion.}
of thoſe preuarications which were vnder the former Teſtament, they that
are called may receiue the promiſe of eternal inheritance. \V For
\CNote{\XRef{Gal.~3,~5.}}
where there is a teſtament: the death of the teſtatour muſt of
%%% o-2722
neceſſitie come between. \V For a Teſtament is confirmed in
%%% 2861
the dead: otherwiſe it is yet of no value, whiles the teſtatour
liueth. \V Whereupon neither was the firſt certes dedicated without
bloud. \V For al the commandement of the Law being read of Moyſes to al
the people: he taking the bloud of calues and goats with
\SNote{Here we may learne that the Scriptures cõteine not al neceſſarie
rites or truths, when neither the place to which the Apoſtle alludeth,
nor any other mentianeth half theſe ceremonies, but he had thẽ by
tradition.}
water and ſcarlet wool and hyſop, ſprinkled the very book alſo it ſelf
and al the people, \V ſaying,
\CNote{\XRef{Exo.~24,~8.}}
\LNote{This is the bloud.}{Chriſtes
\MNote{The correſpõdence of wordes in dedicating both Teſtaments proueth
the real preſence of bloud in the Chalice.}
death was neceſſarie for the ful confirmation, ratification, and
accomplishment of the new Teſtament, though it was begun to be dedicated
in the Sacrifice of his laſt ſupper, being alſo within the compaſſe of
his Paſſion. Which is euident by the wordes pronounced by Chriſt ouer
the holy chalice, which be correſpondent to the wordes that were ſpoken
(as the Apoſtle here declareth) in the firſt Sacrifice of the dedication
of the old law, hauing alſo expreſſe mention of remiſſion of ſinnes
therby as by the bloud of the new Teſtamẽt. Whereby it is plaine, that
the B.~Chalice of the altar hath the very ſacrificial bloud in it that
was shed vpon the Croſſe, in & by which, the new Teſtament (which is the
law of ſpirit, grace, and remiſſion) was dedicated, and doth
conſiſt. And therfore it is alſo cleere, that many diuine things, which
to the Heretikes or ignorant may ſeeme to be ſpoken only of Chriſtes
Sacrifice vpon the Croſſe, be indeed verified & fulfilled alſo in the
Sacrifice of the altar. Wherof S.~Paul for the cauſes aforeſaid would
not treate in plaine termes. See Iſychias
\Cite{li.~4. in Leuit. c.~4. paulo poſt initium,}
applying al theſe things to the immolation of Chriſt alſo in the
Sacrament.}
This is the bloud of the Teſtament, which God hath commanded vnto
you. \V The tabernacle alſo & al the veſſel of the miniſterie he in like
manner ſprinkled with bloud. \V And al things almoſt according to the
law are cleanſed with bloud: and without ſheading of bloud there is not
remiſſion.

\V It is neceſſarie therfore that
\LNote{The examplers.}{Al
\MNote{In the old Teſtament were figures of the new: in the new, is
reſemblance of the heauenly ſtate.}
the offices, places, veſſels, and inſtruments of the old law, were but
figures and reſemblances of the ſtate and Sacraments of the new
Teſtament, which are here called \Emph{celeſtials}, for that they are
the liuely image of the heauenly ſtate next enſuing: which be therfore
ſpecially dedicated and ſanctified in Chriſtes bloud, ſacrificed on
\Fix{the the}{the}{obvious typo, fixed in other}
altar, and ſprinkled vpon the faithful, as the old figures and people
were cleanſed by the bloud of beaſts. And therfore by a tranſition vſual
in the holy Scriptures, the Apoſtle ſodenly paſſeth in the ſentence
immediatly following, and turneth his talke to Chriſtes entrance into
heauen, the ſtate whereof, both by the Sacraments of the old law, and
alſo more ſpecially by them of the new, is prefigured.}
the examplers of the celeſtials be cleanſed with theſe: but the
celeſtials themſelues with better hoſts then theſe. \V For \Sc{Iesvs} is
not entred into Holies made with hand, examplers of the true: but into
Heauen it-ſelf, that he may appeare now to the countenance of God for
vs. \V Nor that he ſhould
\LNote{Offer himſelf often.}{As
\MNote{Chriſt once offered in bloudy ſort, but vnbloudily often, namely
in the Sacrifice of the altar.}
Chriſt neuer died but once, nor neuer shal die againe, ſo in that
violent, painful, and bloudy ſort he cã neuer be offered againe, neither
needeth he ſo to be offered any more, hauing by that one actiõ of
Sacrifice vpon the Croſſe, made the ful ranſom, redemption, and remedie
for the ſinnes of the whole world. Neuertheleſſe, as Chriſt died & was
offered after a ſort in al the Sacrifices of the Law and Nature, ſince
the beginning of the world (al which were figures of this one oblation
vpon the Croſſe) ſo is he much rather offered in the Sacrifice of the
altar of the new Teſtament, incomparably more neerly, diuinely, and
truly expreſſing his death, his body broken, his bloud shed, then did
any figure of the old law, or other ſacrifice that euer was: as being
indeed (though in hidden, ſacramental, and myſtical, and vnbloudy
manner) the very ſelf-ſame B.~body and bloud, the ſelf-ſame hoſt,
oblation and Sacrifice, that was done vpon the Croſſe.

And
\MNote{The Sacrifice of the altar & that on the Croſſe, both one.}
this truth is moſt euident by the very forme of wordes vſed by our
Sauiour in the inſtitution and conſecration of the holy Sacrament, and
by the profeſſion of al the holy Doctours, \Emph{Our Sacrifice}, ſaith
S.~Cyprian, \Emph{is correſpondent to the Paſsion of Chriſt.} And,
\Emph{The ſacrifice that we offer, is the Paſsion of Chriſt.}
\Cite{ep.~63. nu.~4.}
&
\Cite{nu.~7.}
S.~Auguſtin
\Cite{de fid. ad Pet. c.~19.}
\Emph{In thoſe carnal Sacrifices was the prefiguring of the flesh of
Chriſt, which he was to offer for ſinnes, and of the bloud, which he was
to shead. But in this Sacrifice is the commemoration of the flesh of
Chriſt which he hath now giuen, and of the bloud which he hath shed}:
\L{in illis prænunciabatur occidendus, in hoc annunciatur occiſus.}
\Emph{In them he was forshewed as to be killed: in theſe he is shewed,
as killed.} And S.~Gregorie Nazianzene ſaith,
\Cite{orat. in morbum,}
that the Prieſt in this Sacrifice, \L{immiſcet ſe magnis Chriſti
Paſsionibus.}
S.~Ambroſe,
\Cite{1.~Off. c.~48.}
\L{Offertur Chriſtus in imagine quaſi recipiens Paſsionem.} Alexander
the firſt,
\Cite{ep. ad omnes Orthodox. nu.~4, to.~1. Conc.}
\L{Cuius corpus & ſanguis conſicitur, paſsio etiam celebratur.}
S.~Gregorie,
\Cite{ho.~37. in Euangel.}
\Emph{So often as we offer the hoſt of his Paſsion, ſo often we renew
his Paſsion.} And, \Emph{He ſuffereth for vs againe in myſterie.} And
Iſichius,
\Cite{in Leuit. poſt med.}
\Emph{By the Sacrifice of the only-begotten many things are giuen vnto
vs, to wit, the remiſsion or pardoning of al mankind, and the ſingular
introduction or bringing in of the myſteries of the new Teſtament.}

And
\MNote{The Fathers cal it the vnbloudy Sacrifice of the altar.}
the ſaid Fathers and others, by reaſon of the difference in the manner
of Chriſtes preſence and oblation in reſpect of that on the Croſſe,
called this \Emph{the vnbloudy Sacrifice}, as
\CNote{\Cite{Commẽt in 9.~Heb.}}
Caluin himſelf confeſſeth, but anſwereth them in the pride of Heretical
ſpirit, with theſe words:
\MNote{Caluins contempt of the Fathers.}
\L{Nihil moror quod, ſic loquantur vetuſti
Scriptores}; that is, \Emph{I paſſe not for it, that the ancient Writers
doe ſo ſpeake}: calling the diſtinction of bloudy and vnbloudy
Sacrifice, ſcholaſtical and friuolous, and \L{diabolicum commentum},
\Emph{a diuelish deuiſe}. With ſuch ignorant and blaſphemous men we haue
to doe, that thinke they vnderſtand the Scriptures better then al the
Fathers.}
offer himſelf often, as the high Prieſt entreth into the Holies, euery
yeare in the bloud of others: \V otherwiſe he ought to haue ſuffred
often from the beginning of the world: but now once in the cõſũmation of
the worlds, to the deſtruction of ſinne, he hath appeared by his owne
hoſt. \V And as it is appointed to men to die once, and after this, the
iudgement: \V ſo alſo Chriſt was offered once
\SNote{By this word which ſignifieth to emptie or draw out euen to the
bottom, is declared the plentiful and perfect redemption of ſinne by
Chriſt.}
\TNote{\L{ad exhaurienda peccata.}}
to exhauſt the ſinnes of many. The ſecond time he ſhal appeare without
ſinne to them that expect him, vnto ſaluation.


\stopChapter


\stopcomponent


%%% Local Variables:
%%% mode: TeX
%%% eval: (long-s-mode)
%%% eval: (set-input-method "TeX")
%%% fill-column: 72
%%% eval: (auto-fill-mode)
%%% coding: utf-8-unix
%%% End:

