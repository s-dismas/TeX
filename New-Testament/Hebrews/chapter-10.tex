%%%%%%%%%%%%%%%%%%%%%%%%%%%%%%%%%%%%%%%%%%%%%%%%%%%%%%%%%%%%%%%%%
%%%%
%%%% The (original) Douay Rheims Bible 
%%%%
%%%% New Testament
%%%% Epistles
%%%% Hebrewes
%%%% Chapter 10
%%%%
%%%%%%%%%%%%%%%%%%%%%%%%%%%%%%%%%%%%%%%%%%%%%%%%%%%%%%%%%%%%%%%%%




\startcomponent chapter-10


\project douay-rheims


%%% 2864
%%% o-2725
\startChapter[
  title={Chapter 10}
  ]

\Summary{Becauſe in the yearely feaſt of Expiation was only a
  commemoration of ſinnes, therfore in place of al thoſe old Sacrifices
  the Pſalme telleth vs of the oblation of Chriſtes body. 10.~Which he
  offered bloudily but once (the Leuitical Prieſts offering ſo euery day)
  becauſe 
  that once was ſufficient for euer, 15.~in that it purchaſed (as the
  Prophet alſo witneſſeth) remiſsion of ſinnes. 19.~After al this he
  proſecuteth and exhorteth them vnto perſeuerance, partly with the
  opening of Heauen by our high Prieſt, 26.~partly with the terrour of
  damnation if they fal againe: 32.~bidding them remember how much they
  had ſuffered already, and not loſe their reward.}

For the law
\LNote{A shadow.}{The
\MNote{The old Sacrifices obſcurely shadowed, but the Sacrifice of the
altar moſt plainely repreſenteth the Sacrifice on the Croſſe.}
Sacrifices and ceremonies of the old law, were ſo farre from the truth
of Chriſts Sacraments, and from giuing ſpirit, grace, remiſsion,
redemption, and iuſtification, and thereupon the entrance into heauen
and ioyes celeſtial, that they were but mere shadowes, vnperfectly and
obſcurely repreſenting the graces of the new Teſtament and of Chriſtes
death: whereas al the holy Churches rites and actions inſtituted by
Chriſt in the Prieſthood of the new law, conteine and giue grace,
iuſtification, and life euerlaſting to the faithful and worthy
receiuers: and therfore they be not shades or darke reſemblances of
Chriſtes paſſion, which is the fountaine of al grace and mercie, but
perfect images and moſt liuely repreſentations of the ſame, ſpecially
the Sacrifice of the altar, which becauſe it is the ſame oblation, the
ſame hoſt, and offered by the ſame Prieſt Chriſt \Sc{Iesvs} (though by
the miniſterie of man and in myſterie) is the moſt pure and neer image,
character, and correſpondence to the
\Fix{Sacrifice}{Sacrifice of}{obvious typo, fixed in other}
Chriſtes paſſion, both in ſubſtance, force, and effect, that can be.}
hauing a ſhadow of good things to come, not the very image of the
things:
\CNote{\XRef{Leu.~16,~14.}}
euery yeare with the ſelf-ſame hoſts which they offer inceſſantly, can
neuer make the commers thereto perfect: \V otherwiſe
\LNote{They should haue ceaſed.}{If
\MNote{The Iewes Sacrifices were not abſolute & indepẽdẽt, becauſe they
were often repeated.}
the hoſts and offerings of the old Law had been of them ſelues perfect
to al effects of redemption and remiſsion: as the Hebrewes (againſt whom
the Apoſtle diſputeth) did thinke, and had had no relation to Chriſtes
Sacrifice on the Croſſe or any other abſolute and vniuerſal oblation or
remedie for ſinne, but by and of their owne efficacie could haue
generally purged & cleanſed man of al ſinne & damnation: then they
should neuer haue needed to be ſo often repeated and reiterated. For being
both generally auailable for al, by their opinion, and particularly
applied (in as ample ſort as they could be) to the ſeueral infirmities
of euery offender, there had been no ſinnes left. But ſinnes did
remaine, euen thoſe ſinnes for which they had offered Sacrifices before
notwithſtanding their Sacrifices were particularly applied vnto thẽ. For,
offering yearely they did not only offer Sacrifices for the new cõmitted
crimes, but euen for the old, for which they had oftẽ ſacrificed before:
the Sacrifices being rather records and atteſtations of their ſinnes,
then a redemption or ful remiſsion, as Chriſtes death is. Which being
once applied to mã by Baptiſme, wipeth away al ſinnes paſt, God neuer
remẽbring them any more, nor euer any Sacrifice or Sacrament or
ceremonie being made or done for them any more, though for new ſinnes
other remedies be daily requiſit. Their Sacrifices then could not of
themſelues remit ſinnes, much leſſe make the general redẽption, without
relation to Chriſtes Paſsion.
\MNote{The Apoſtle proueth by the oftẽ repeating of the Iewes
Sacrifices, not that they were none, but that they were not abſolute &
ſufficient.}
And ſo you ſee it is plaine euery-where, that the Apoſtle proueth not by
the often repetition of the Iewish Sacrifices, that they were no
Sacrifices at al, but that they were not of that abſolute force or
efficacie, to make redemption or any remiſsion, without dependance of
the one vniuerſal redemption by Chriſt: his whole purpoſe being, to
inculcate vnto them the neceſsitie of Chriſtes death and the oblation of
the new Teſtament. As for the Churches holy Sacrifice, it is cleane of
another kind then thoſe of the Iewes, and therfore he maketh no
oppoſition betwixt it, and Chriſtes death or Sacrifice on the Croſſe, in
al this Epiſtle: but rather as a ſequele of that one general oblation,
couertly alwaies inferreth the ſame: as being in a different manner the
very ſelf-ſame hoſt and offering that was done vpon the Croſſe, &
continually is wrought by the ſelf-ſame Prieſt.}
they should haue ceaſed to be offered, becauſe the worſhippers once
cleanſed ſhould haue no conſcience of ſinne any longer. \V But in them
there is made a commemoration of ſinnes euery yeare. \V For it is
\LNote{Impoſſible.}{The
\MNote{The old Sacrifices remitted not ſinnes but were only ſignes
thereof.}
Hoſts and Sacrifices of the old Law, which the carnal Iewes made al the
count of, without relation to Chriſtes death, were not only not perfect
and abſolute ſufficient in themſelues, but they did not, nor could not
remit any ſinnes at al, being but only ſignes thereof, referring the
offenders for remiſsion indeed, to Chriſtes Paſſion. For the bloud of
bruit beaſts could haue no other effect, nor any other element or
creature, before Chriſtes death. The fruit whereof, before it was
extant, could be no otherwiſe properly applied vnto them, but by beleefe
in him.}
impoſſible that with the bloud of oxen and goats ſinnes ſhould be taken
away. \V Therfore comming into the world he ſaith:
\CNote{\XRef{Pſ.~39,~7.}}
\LNote{Hoſt and oblation.}{He
\MNote{God refuſeth the Iewes Sacrifices, not al Sacrifice.}
meaneth not that God would no hoſt nor Sacrifice any more as the
Proteſtants falſely imagin: for that were to take away not only the
Sacrifice of Chriſtes body vpon the altar, but the Sacrifice of the ſame
body vpon the Croſſe alſo. Therfore the Prophet ſpeaketh only of the
legal and carnal Sacrifices of the Iewes, ſignifying that they did neuer
of themſelues pleaſe God, but in reſpect of Chriſt, by whoſe oblation of
his owne body they should pleaſe.}
\Emph{Hoſt and oblation thou wouldeſt not:
\LNote{But a body.}{If
\MNote{That Chriſt should haue a body was neceſſarie for his Prieſthood,
& Sacrifice.}
Chriſt had not had a body, he could not haue had any worthy matter or
any matter at al to Sacrifice in viſible manner, other then the hoſts of
the old Law. Neither could he either haue made the general redemption by
his one oblation vpon the Croſſe, nor the daily Sacrifice of the Church:
for both which, his body was fitted by the diuine wiſedom. Which is an
high concluſion, not vnderſtood of Iewes, Pagans, nor the Heretikes of
our time, that Chriſtes humane nature was taken to make the Sonne of God
(who in his diuine nature could not be either Prieſt or Hoſt) fit to be
the Sacrifice & Prieſt of his Father, in a more worthy ſort, thẽ al the
Prieſts or oblatiõs of the old law.
\MNote{The body of Chriſts is the Sacrifice of the altar.}
And that this body was giuen him, not only to be the Sacrifice vpon the
Croſſe, but alſo vpon the altar, S.~Auguſtin affirmeth in theſe wordes:
\Emph{The table which the Prieſt of the new Teſtament doth exhibit, is
of his body and bloud: for that is the Sacrifice which ſucceeded al
thoſe Sacrifices that were offered in shadow of that to come. For the
which alſo we acknowledge that voice of the ſame Mediatour in the
Pſalme,}
\CNote{\XRef{Pſ.~39.}}
\Sc{Bvt a Body Thov Haſt Fitted to Me}, \Emph{becauſe inſteed of al
thoſe Sacrifices and oblations his body is offered, & is miniſtred to
the partakers or receiuers.}
\Cite{Li.~17. Ciuit Dei c.~~20.}
And againe,
\Cite{li.~4. de Trinit. c.~14.}
\Emph{Who ſo iuſt and holy a Prieſt, as the only Sonne of God? What
might ſo conueniently be offered for men, of men, as man's flesh? and
what ſo fit for this immolation or offering, as mortal flesh? what ſo
cleane for cleanſing the vices of mortal man, as the flesh borne of the
virgins womb? and what can be offered and receiued ſo greatfully, as the
flesh of our Sacrifice, made the body of our Prieſt?}}
but a body thou haſt fitted to me: \V Holocauſts and
\SNote{\Emph{For ſinne}, is the proper name of a certaine Sacrifice
called in Hebrew \H{֝חטאה}, as Holocauſt is another kind. See the
\XRef{Annot. 2.~Cor.~5. v.~21.}}
for ſinne did not pleaſe thee. \V Then ſaid I, Behold I come: in the
head of the booke it is written of me: That I may doe thy wil, ô
God.} \V Saying before, \Emph{Becauſe hoſts and oblations & holocauſts,
& for ſinne thou wouldeſt not,
\LNote{Neither did they pleaſe thee.}{By
\MNote{The Iewes Sacrifices refuſed, not al Sacrifice.}
that he ſaith, the things offered in the Law, did not pleaſe God, &
likewiſe by that he ſaith, the former to be taken away, that the ſecond
may haue place, it is euident, that al hoſtes and Sacrifices be not
taken away by Chriſt as the Heretikes foolishly conceiue: but that the
old Hoſts of brute beaſts be abrogated to giue place to that which is
the proper hoſt of the new law, that is, Chriſtes owne body.}
neither did they pleaſe thee}, which are offered according to the
law, \V \Emph{then ſaid I, Behold I come that I may doe thy wil, ô God}:
he taketh away the firſt, that he may eſtabliſh that that followeth. \V
In the which wil, we are ſanctified by the oblation of the body
of \Sc{Iesvs} Chriſt once. \V And euery Prieſt indeed is ready daily
%%% 2865
miniſtring, and
\LNote{Often offering the ſame Hoſts.}{As
\MNote{We muſt often note that the Apoſtles ſpeach of many Prieſts and often
Sacrificing, concerneth only the Iewes Prieſts and Sacrifices, not the
Prieſts and Sacrifices of the new Teſtamẽt.}
S.~Paul is forced often to inculcate that one principle of the efficacie
& ſufficiencie of Chriſtes death, becauſe of the Hebrues too much
attributing to their legal Sacrifices, and for that they did not referre
them to Chriſtes only oblation: ſo we, through the intolerable ignorance
and importunity of the Heretikes of this time (abuſing the words of the
Apoſtle ſpoken in the due defence and declaration of the valure and
efficacie of Chriſtes paſſion aboue the Sacrifices of the Law) are
forced to repeat often, that the Apoſtles reaſon of many Prieſts & often
repetition of the ſelf-ſame Sacrifices, concerneth the Sacrifices of the
Law only, vnto which he oppoſeth Chriſtes Sacrifice and Prieſthood; &
ſpeaketh no word of or againſt the Sacrifice of the new Teſtament:
which is the Sacrifice of Chriſtes owne Prieſthood, Law, and
inſtitution, yea, the ſame Sacrifice done daily vnbloudily, that once was
done bloudily: made by the ſame Prieſt Chriſt \Sc{Iesvs}, though by his
miniſters hands: and not many Hoſts, as thoſe of the old Law were, but
the very ſelf-ſame in number, euen Chriſtes owne body that was crucified.
\MNote{The Caluiniſts arguments againſt Chriſts body often offered, and
in many places anſwered by the Fathers long a-goe.}
And
that you may ſee that this is the iudgement of al antiquity, and their
expoſition of theſe and the like words of this Epiſtle, and that they
ſeeing the very ſame arguments that the Proteſtants now make ſo much a
doe withal among the ſimple and vnlearned, yet wel perceiued that they
made nothing againſt the daily oblation or Sacrifice of the altar, and
therfore anſwered them before the Proteſtants were extant, 1200.~yeares;
we wil ſet downe ſome of their words, whoſe authoritie and expoſition of
the Scriptures muſt preuaile in al that haue wiſedom or the feare of
God, aboue the falſe and vaine gloſſes of Caluin and his followers.

Thus then firſt ſaith S.~Ambroſe:
\CNote{\Cite{in 10.~cap. Hebr.}}
\L{Quid ergo nos &c.} \Emph{What we then? doe not we offer euery day? We
offer ſurely: but this Sacrifice is an exampler of that: for we offer
alwaies the ſelf-ſame, and not now one lamb, tomorrow another, but
alwaies the ſelf-ſame thing: therfore it is one Sacrifice. Otherwiſe, by
this reaſon, becauſe it is offered in many places, there should be many
Chriſtes: not ſo, but it is one Chriſt in euery place, here whole, and
there whole, one body. But this which we doe is done for a commemoration
of that which was done. For we offer not another Sacrifice, as the high
Prieſt of the old Law, but alwaies the ſelf-ſame. &c.}  Primaſius
S.~Auguſtines Scholer doth alſo preoccupate Proteſtants obiections thus:
%%% !!! CNote only in other
%%% !!! ???
\CNote{\Cite{ibidem.}}
\Emph{What shal we ſay then? doe not our Prieſts daily offer Sacrifice?
They offer ſurely, becauſe we ſinne daily, and daily haue need to be
cleanſed: and becauſe he can not die, he hath giuen vs the Sacrament of
his body and bloud: that as his Paſsion was the redemption and abſolution
of the world, ſo alſo this oblation may be redemption and cleanſing
to al that offer it in truth and veritie.} So ſaith this holy Father, to
wit,
\MNote{The general redemption vpon the Croſſe is particularly applied in
the Sacrifice of the altar.}
that as the Sacrifice of the Croſſe was a general redemption, ſo this of
the altar is, to al that vſe it, a particular redemption or application
of Chriſtes redemption to them. In which ſenſe alſo V.~Bede calleth the
holy Maſſe, \L{redemptionem corporis & anime ſempiternam}, \Emph{the
euerlaſting redemption of body and ſoule.}
\Cite{li.~4. c.~22. hiſtor.}
Againe the ſame Primaſius,
\CNote{Primaſ. loco citato.}
\Emph{The diuinity of the Word of God which is euery where, maketh that
there are not many Sacrifices, but one, although it be offered of many,
and that as it is one body which he tooke of the Virgins womb, not many
bodies, euen ſo alſo one Sacrifice, not diuers, as thoſe of the Iewes
were.}

S.~Chryſoſtom
\CNote{\Cite{ho.~17. in ep. ad Heb.}}
alſo, and after him Theophylact, and Oecumenius, and of the Latines,
Haimo, Paſchaſius, Remigius, and others, obiect to themſelues thus:
\Emph{Doe not we alſo offer euery day? We offer ſurely. But this Sacrifice is
an exampler of that, for we offer alwaies the ſelf-ſame: and not now one
lamb, tomorrow another, but the ſelf-ſame: therfore this is one
Sacrifice. Otherwiſe, becauſe it is offered in many places, there should
be many Chriſtes.} And a litle after, \Emph{Not another Sacrifice, as
the high Prieſts of the old Law, but the ſelf-ſame we doe alwaies offer,
rather working a remembrance or commemoration of the Sacrifice.} See the
\XRef{Annotation Luke~22,~19. vpon theſe words \Emph{A commemoration.}}
Thus did al the ancient Fathers Greek and Latin treate of theſe matters,
and ſo they ſaid Maſſe, and offered daily, and many of them made ſuch
formes of celebrating the diuine Sacrifice, as the Greek and Latines doe
vſe in their Liturgies and Maſſes, and yet they ſaw theſe places of the
Apoſtle, and made commentaries vpon them, and vnderſtood them (I trow)
as wel as the Proteſtants.

He
\MNote{Councels and Fathers.}
that for his further confirmation or comfort liſt ſee what the ancient
Councels and Doctours beleeued, taught, and practiſed in this thing, let
him read 
\Cite{the firſt holy Councel of Nice cap.~14.}
&
\Cite{in fine Conc. ex Græco.}
\Cite{the Councel of Epheſus Anethematiſ.~11.}
\Cite{the Chalcedon Councel act.~3. pag.~112. Conc. Ancyram. c.~1.~4.
and~5.}
\Cite{Neocæſat. can.~13.}
\Cite{Laodic. can.~19.}
\Cite{Carthag.~2. cap.~8.}
\Cite{Carthag.~3. c.~24.}
&
\Cite{Carthag.~4. c.~33. &~41.}
\Cite{S.~Denys c.~3. Eccl. hier.}
\Cite{S.~Andrew in hiſtoria Paſsionis,}
\Cite{S.~Ignatius ep. ad Smyrenſes.}
\Cite{S.~Martialis ep. ad Burdegalenſes}
\Cite{S.~Iuſtine Dialog. cum Triphone}
\Cite{S.~Irenæus l.~4. c.~32.~34.}
\Cite{Tertullian de cultu fæminorum,}
&
\Cite{de Corona milit.}
\Cite{Origen homil.~13. in Leuit.}
\Cite{S.~Cyprian ep. ad Cecilium nu.~2.}
&
\Cite{de Coenæ Domini nu.~14.}
&
\Cite{Euſebius demonſt Euang. li.~1. cap.~10.}
and the reſt which we haue cited by occaſion before, and might cite but
for tediouſnes: a truth moſt knowen and agreed vpon in the Chriſtian
religion.}
often offering the ſame hoſts, which can neuer take away ſinnes: \V but
this offering one hoſt for ſinnes, for euer
\CNote{\XRef{Pſ.~109.}
\XRef{1.~Cor.~15,~25.}}
ſitteth on
%%% o-2726
the right hand of God, \V hence-forth expecting, vntil his enemies be
put the foot-ſtool of his feet. \V For by one oblation hath he
conſummated for euer them that are ſanctified. \V And the Holy Ghoſt
alſo doth teſtifie to vs. For after that he ſaid: \V
\CNote{\XRef{Hier.~31.~33.~34.}
\XRef{Heb.~8,~8.}}
\Emph{And this is the Teſtament which I wil make to them after thoſe
daies, ſaith our Lord, giuing my lawes
\SNote{This is partly fulfilled in & by the grace of the new Teſtament,
but it shal be perfectly accomplished in heauen.}
in their harts, and in their minds wil I ſuperſcribe them: \V and their
ſinnes and iniquities I wil now remember no more.} \V But where there is
remiſſion of theſe,
\LNote{Now there is not.}{Chriſtes
\MNote{When the Apoſtle ſeemeth to ſay, there is no remiſſion or oblatiõ
for ſinne he alwaies meaneth that ful remiſsion by Baptiſme.}
death can not be applied vnto vs in that ful and ample ſort as it is in
Baptiſme, but once: Chriſt appointing that large remiſſion and
application to be made but once in euery man, as Chriſt died but
once. For it is not meant, that al ſinne shal ceaſe after Chriſtes
Sacrifice vpon the Croſſe, not that there should be no oblation for
ſinnes committed after Baptiſme, or that a man could not ſinne at al
after Baptiſme, or that if he ſinned afterward, he could haue no remedie
or remiſſion by God's ordinance in the Church, which diuers falſehoods
ſundrie Heretikes gather of this and ſuch like places: but only the
Apoſtle telleth the Hebrewes, as he did before
\XRef{chap.~6.}
and as he doth ſtraight afterward, that if they fal now (whereunto they
ſeemed very prone) to their old law, and voluntarily after this
knowledge and profeſſion of the Chriſtian faith by Baptiſme, commit this
ſinne of incredulitie and apoſtaſie, they can neuer haue that aboundant
remiſſion applied vnto them by Baptiſme, which can neuer be miniſtred to
them againe. And that general ful pardon he calleth here \Emph{oblation}
& afterward in the
\XRef{26.~verſe,}
\L{hoſtiam pro peccato}, \Emph{an hoſt for ſinne}.}
now there is not an oblation for ſinnes.

\V Hauing therfore, Brethren, confidence in the entring of the Holies in
the bloud of Chriſt: \V which
\SNote{To dedicate, is to be authour & beginner of a thing. The
Proteſtants tranſlate, \Emph{he hath prepared}, for their hereſie that
Chriſt was not the firſt man that entred into heauen.}
\TNote{\G{ἐνεκαίνισεν}}
he hath dedicated to vs a new & liuing way by the veile, that is, his
fleſh, \V and a high Prieſt ouer the houſe of God, \V let vs approche
with a true hart in fulneſſe of faith, hauing our harts ſprinkled
from euil conſcience, and our body waſhed with cleane water. \V Let vs
hold the confeſſion of our hope vndeclining (for he is faithful that
hath promiſed) \V and let vs conſider one another vnto the prouocation
of charitie and of good workes: \V not forſaking our aſſemblie as ſome
are accuſtomed, but comforting, and ſo much the more as you ſee the day
approching. \V
\CNote{\XRef{Heb.~6,~4.}}
For
\LNote{If we ſinne willingly.}{As
\MNote{The Caluiniſts hereſie againſt remiſſion of ſinnes.}
the Caluiniſts abuſe other like places againſt the holy Sacrifice of the
Maſſe, ſo they abuſe this as the Nouations did before them, to proue
that an Heretike, Apoſtata, or any that wilfully forſaketh the truth,
can neuer be forgiuen. Which (as is before declared in the
\XRef{6.~chapter)}
is moſt wicked blaſphemie; the meaning hereof being, as is there ſaid,
only to terrifie the Hebrewes, that falling from Chriſt they can not ſo
eaſily haue the Hoſt of Chriſtes death applied vnto thẽ becauſe they can
not be baptized any more, but muſt paſſe by ſacramental penance, &
ſatisfaction, & other hard remedies which Chriſt hath preſcribed after
Baptiſme in the Churches diſcipline. Therfore S.~Cyril ſaith
\Cite{li.~5. in Io. c.~17.}
\MNote{Al ſinnes may be remitted by penance, but not ſo fully as by
Baptiſme.}
\Emph{Penance is not excluded by theſe words of Paul, but the renewing
by the lauer of regeneration. He doth not here take away the ſecond or
third remiſsion of ſinnes (for he is not ſuch an enemy to our ſaluation)
but the hoſt which is Chriſt he denieth that it is to be offered againe
vpon the Croſſe.} So ſaith this holy Doctour.
\MNote{Perilous reading of the Scriptures.}
And by this place & the like you ſee, how perilous a thing it is for
Heretikes & ignorant perſons to read the Scriptures. Which by following
their owne fantaſie
\CNote{\XRef{2.~Pet.~3.}}
they peruert to their damnation.}
if we ſinne willingly after the knowledge of the truth receiued, now
there is not left an hoſt for ſinnes, \V but a certaine terrible
expectation of iudgement & rage of fire, which ſhal conſume the
aduerſaries. \V A man making the Law of Moyſes fruſtrate, without any
mercie
\CNote{\XRef{Deu.~19,~15.}
\XRef{Mat.~18,~16.}
\XRef{Io.~8,~17.}}
dieth vnder two or three witneſſes. \V
\SNote{Hereſie and Apoſtaſie from the Catholike faith, punishable by
death.}
How much more thinke you, doth he deſerue worſe puniſhements which hath
troden the Sonne of God vnder-foot, and eſteemed
\LNote{The bloud of the Teſtament.}{Whoſoeuer
\MNote{Contempt of Chriſts bloud in the Sacrament.}
maketh no more account of the bloud of Chriſtes Sacrifice, either as
shed vpon the Croſſe, or as in the holy Chalice of the altar (for our
Sauiour calleth that alſo
\CNote{\XRef{Luc.~22.}}
the bloud of the new Teſtament) then he doth of the bloud of calues and
goats, or of other common drinkes, is worthy death, and God wil in the
next life, if it be not punished here, reuenge it with greiuous
punishment.}
the bloud of the Teſtament polluted, wherein he is ſanctified, and hath
done contumelie to the Spirit of grace? \V For we know him that ſaid,
\CNote{\XRef{Deu.~32,~35.}
\XRef{Ro.~12,~19.}}
\Emph{Reuenge to me, I wil repay.}
And againe,
\CNote{\XRef{Pſ.~134,~14.}}
\Emph{That our Lord wil iudge his people.} \V
\LNote{It is horrible.}{Let
\MNote{Penance.}
al Chriſtian people doe ſatisfaction and
penance for their ſinnes in this life. For the iudgements of God in the
next life done by God himſelf, of what ſort ſoeuer, whether temporal as in
Purgatorie, or eternal as in Hel be exceeding grieuous.}
It is horrible to fal into the hands of the liuing God.

\V But cal to mind the old daies: wherein being illuminated, you
ſuſtained a great fight of paſſions. \V And on the one part certes by
reproches and tribulations made a ſpectacle; & on the other part made
companions of them that conuerſed in ſuch ſort. \V For,
\LNote{You had compaſsion.}{To
\MNote{Mercie to the empriſoned for religion.}
be merciful to the afflicted for religion, & to be partakers of their
miſeries, is a very meritorious worke, and giueth great confidence
before God in the day of repaiment or remuneration for the ſame.}
you both had compaſſion on them that were in bands: and the ſpoile of
your owne goods you tooke
\LNote{With ioy.}{If
\MNote{Loſſe of goods for religion.}
al Chriſtian men would conſider this, they would not thinke it ſo great
a matter to loſe their land or goods for defence of the Catholike
faith.}
with ioy, knowing that you haue a
%%% o-2727
better and a permanent ſubſtance. \V Doe not therfore leefe your
\SNote{Good workes make great cõfidence of ſaluation, & haue great
reward.}
confidence, which hath a great remuneration. \V For patience is
neceſſarie for you: that doing the wil of God, you may receiue the
promiſe. \V For
\CNote{\XRef{Abac.~2,~3.}
\XRef{Ro.~1,~17.}
\XRef{Gal.~3.~12.}}
yet a litle and a very litle while, he that is to come, wil come, and
wil not ſlacke. \V And my iuſt
\LNote{Liueth of faith.}{Faithful
\MNote{Faith is the comfort of the afflicted.}
men afflicted in this life, haue their comfort in their aſſured faith
and hope of Chriſtes comming to deliuer them once from al theſe
miſeries; & ſo by that faith & comfort they liue, whereas otherwiſe this
miſerable life were a death.}
liueth of faith: but if he withdraw himſelf, he ſhal not pleaſe my
ſoule. \V But we are not the children of withdrawing vnto perdition: but
of faith to the winning of the ſoule.


\stopChapter


\stopcomponent


%%% Local Variables:
%%% mode: TeX
%%% eval: (long-s-mode)
%%% eval: (set-input-method "TeX")
%%% fill-column: 72
%%% eval: (auto-fill-mode)
%%% coding: utf-8-unix
%%% End:

