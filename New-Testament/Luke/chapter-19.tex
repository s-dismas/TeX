%%%%%%%%%%%%%%%%%%%%%%%%%%%%%%%%%%%%%%%%%%%%%%%%%%%%%%%%%%%%%%%%%
%%%%
%%%% The (original) Douay Rheims Bible 
%%%%
%%%% New Testament
%%%% Luke
%%%% Chapter 19
%%%%
%%%%%%%%%%%%%%%%%%%%%%%%%%%%%%%%%%%%%%%%%%%%%%%%%%%%%%%%%%%%%%%%%




\startcomponent chapter-19


\project douay-rheims


%%% 2461
%%% o-2292
\startChapter[
  title={Chapter 19}
  ]

\Summary{In Iericho he lodgeth in the houſe of Zachæus a Publicane, and
  againſt the murmuring Iewes openeth the reaſons of his ſo
  doing. 11.~He sheweth, that the laſt day should not be yet, 15.~and
  what then in the iudgement he wil doe both to vs of his Church as wel
  good as bad, 27.~and alſo to the reprobate Iewes. 29.~Being now come
  to the place of his Paſsion, he entreth (weeping and foretelling the
  deſtruction of blind Hieruſalem): with triumph as their Chriſt. 45.~He
  sheweth his zeale for the houſe of God, and teacheth therein euery
  day. 47.~The rulers would deſtroy him, but for feare of the people.}

%%% o-2293
And entring in, he walked through Iericho. \V And behold a man named
Zachæus: and this was a Prince of the Publicans, and he rich. \V And
he ſought to ſee \Sc{Iesvs} what he was, and he could not for the
multitude, becauſe he was litle of ſtature. \V And running before, he
\LNote{Went vp}{Not
\MNote{External deuotion.}
only inward deuotion of faith and charitie towards Chriſt, but external
offices of ſeeing, following, touching, receiuing, harbouring him, are
recommended to vs in this example: euen ſo our manifold exteriour
deuotion towards his Sacraments, Saints, and ſeruants, be grateful:
ſpecially the endeauour of good people not only to be preſent at Maſſe
or in the Church, but to be neere the B.~Sacrament, and to ſee it with
al reuerence and deuotion according to the order of the Church, much
more to receiue it into the houſe of their body.}
went vp into a ſycomore tree that he might ſee him: becauſe he was to
paſſe by it. \V And when he was come to the place, \Sc{Iesvs} looking
vp, ſaw him, and ſaid to him: Zachæus, come downe in haſt: becauſe this
day I muſt abide in thy houſe. \V And he in haſt came downe, and
receiued him reioycing. \V And when al ſaw it, they murmured ſaying,
that he turned in, to a man that was a ſinner. \V But Zachæus ſtanding
ſaid to our Lord: Behold the halfe of my goods, Lord, I giue to the
poore: and if I haue defrauded any man of any thing,
\LNote{I reſtore fourefold}{That
\MNote{Reſtitution.}
which we giue of our owne, is almes and ſatisfaction for our ſinnes: but
that which we reſtore of il gotten goods by Extortion, Vſurie, Simonie,
Bribrie, Theft, or otherwiſe, that is called here Reſtoring. And it is
of duty and not of free almes, and muſt be rendred not to whõ we liſt,
but to the parties annoyed if it be poſſible; otherwiſe it muſt be
beſtowed vpon the Poore, or other good vſes, according to the aduiſe of
our ſuperiour & ſuch as haue charge of our ſoules.
\MNote{Satisfaction.}
But that he yealded foure-fold, that was more then he was bound, but
very ſatisfactorie for his former ſinnes alſo. And herewith we may note,
that it is not the giuing of a peny, grote, or crowne of a rich mans
ſuperfluitie, that is ſo much recommended to ſinners for redeeming their
faultes: but this large beſtowing vpon Chriſt, to ſel al and giue it in
almes, to giue the moytie of our goodes, to render foure times ſo much
for that which is wrongfully gotten, that extinguisheth ſinnes.
\CNote{\XRef{Lu.~21,~3.}}
The
poore widowes braſſe peny was very grateful, becauſe it was al or much
of that she had: but the rich mans pound of his ſuperfluitie, though it
be good, yet is nothing ſo grateful.}
I reſtore fourefold. \V \Sc{Iesvs} ſaid to him: That this day ſaluation
is made to this houſe: becauſe that he alſo is the ſonne of Abraham. \V
\CNote{\XRef{Mt.~18,~12.}}
For the Sonne of man is come to ſeeke and to ſaue that which was loſt.

\V They hearing theſe things, he added and ſpake a parable, for that he
was nigh to Hieruſalem, and becauſe they thought that forthwith the
Kingdom of God should be manifeſted. \V He ſaid therfore:
\CNote{\XRef{Mt.~25,~14.}}
A certaine
noble man went into a farre countrie to take to him ſelf a Kingdom, and
to returne. \V And calling his ten ſeruants, he gaue them ten poundes,
and ſaid to them: Occupie til I come. \V And his citizens hated
%%% 2462
him: and they ſent a legacie after him, ſaying: We wil not haue this man
reigne ouer vs. \V And it came to paſſe after he returned, hauing
receiued his Kingdom: and he commanded his ſeruants to be called, to
whom he gaue the money; that he might know how much euery man had gained
by occupying. \V And the firſt came, ſaying: Lord thy pound hath gotten
ten poundes. \V And he ſaid to him: Wel fare thee good ſeruant, becauſe
thou haſt been faithful in a litle, thou ſhalt haue power ouer
\SNote{Marke here againſt the aduerſaries, that the rewards of theſe two
good ſeruants be diuers & vnequal, according to the diuerſitie or
inequalitie of their gaines, that is, their merites: & yet one receiueth
the peny
\XRef{(Mt.~20,~9.)}
as wel as the other, that is, Heauen or life euerlaſting.}
ten cities. \V And the ſecond came ſaying: Lord, thy pound hath made
fiue poundes. \V And he ſaid to him: And be thou ouer fiue cities. \V
And an other came, ſaying: Lord, loe here thy pound, which I haue had laid
vp in a napkin. \V For I feared thee, becauſe thou art an auſtere man:
thou takeſt vp that thou didſt not ſet downe, and thou reapeſt
%%% o-2294
that which thou didſt not ſow. \V He ſaith to him: By thine owne mouth I
iudge thee, naughtie ſeruant. Thou didſt know that I am an auſtere man,
taking vp that I ſet not downe, and reaping that which I ſowed not: \V
and why didſt thou not giue my money to the banke, and I comming might
certes with vſurie haue exacted it? \V
\SNote{See
\XRef{annotations Mat.~25,~29. &c.}}
And he ſaid to them that ſtood by: Take the pound away from him, and
giue to him that hath the ten poundes. \V And they ſaid to him: Lord, he
hath ten poundes. \V But I ſay to you, that to euery one that hath ſhal
be giuen: and from him that hath not, that alſo which he hath ſhal be
taken from him.

\V But as for thoſe mine enemies that would not haue me reigne ouer
them, bring them hither; and kil them before me.

\V
\MNote{The fifth part of this Ghoſpel. Of the Holy weeke of his Paſſion
in Hieruſalem.}
\MNote{\Sc{Palme} ſunday.}
And hauing ſaid theſe things, he went before aſcending to Hieruſalem. \V
And it came to paſſe
\CNote{\XRef{Mt.~21,~1.}
\XRef{Mr.~11,~1.}
\XRef{Io.~12,~15.}}
when he was come nigh to Bethphage and Bethania
vnto the mount called Oliuet, he ſent two of his Diſciples, \V ſaying:
Goe into the towne which is ouer againſt, into the which as you enter,
you ſhal find the colt of an aſſe tied, on which no man euer hath
ſitten: looſe him, and bring him. \V And if any man aske you: Why looſe
you him? You ſhal ſay thus to him: Becauſe our Lord needeth his
ſeruice. \V And they that were ſent, went their waies, and found as he
ſaid to them, the colt ſtanding. \V And when they looſed the colt, the
owners thereof ſaid to them: Why looſe you the colt? \V But they ſaid:
Becauſe our Lord hath need of him. \V And they brought him
to \Sc{Iesvs}. And caſting their garments vpon the colt, they
ſet \Sc{Iesvs} thereupon. \V And as he went, they ſpred their garments
vnderneath in the way. \V And when he approched now to the deſcent of
mount-Oliuet, al the multitudes of
\Var{them that deſcended,}{his Diſciples,}
began with ioy to praiſe God with a loud voice, for al the miracles that
they had ſeen, \V ſaying: Bleſſed is he that commeth King in the name of
our Lord, peace in Heauen, and glorie on high. \V And certaine Phariſees
of the multitudes ſaid to him: Maiſter, rebuke thy Diſciples. \V To whom
he ſaid: I ſay to you, that if theſe hold their peace, the ſtones ſhal
crie. \V And as he drew neere, ſeeing the citie, he wept vpon it,
ſaying, \V Becauſe if thou alſo hadſt knowen, and that in this thy day,
the things that pertaine to thy peace: but now they are hid from thine
%%% o-2295
eyes. \V For
\SNote{This was fulfilled 40.\ yeares after the death of Chriſt by Titus
and Veſpaſianus, when beſides incredibile miſeries of famine and other
diſtreſſes, there perished eleuẽ hundred thouſand, and were taken
captiues 97000, the ſiege beginning in the very ſame feaſt & greateſt
ſolemnitie of Eaſter when they put Chriſt to death.
\Cite{Euſeb. li.~3. hiſt. c.~6,~7,~8.}
\Cite{Ioſeph. li.~7. c.~17.}}
the daies ſhal come vpon thee: and thy enemies ſhal
%%% 2463
compaſſe thee with a trench, and incloſe thee about, and ſtraiten thee
on euery ſide, \V and beate thee flat to the ground, and thy children
that are in thee: and they ſhal not leaue in thee a ſtone vpon a ſtone,
becauſe thou haſt not knowen the time of thy viſitation.

\V
\CNote{\XRef{Mt.~21,~12.}
\XRef{Mr.~11,~15.}}
\MNote{\Sc{Mvnday}.}
And entring into the Temple, he began to caſt out the ſellers therein
and the buyers, \V ſaying to them: It is written,
\CNote{\XRef{Es.~56,~7.}}
\Emph{That my houſe is the houſe of praier.} But you haue made it
\CNote{\XRef{Ier.~7,~11.}}
\Emph{a denne of theeues.} \V
And he was teaching daily in the Temple. And the cheefe Prieſts and the
Scribes and the Princes of the people ſought to deſtroy him: \V and they
found not what to doe to him. For al the people was ſuſpenſe, hearing
him.


\stopChapter


\stopcomponent


%%% Local Variables:
%%% mode: TeX
%%% eval: (long-s-mode)
%%% eval: (set-input-method "TeX")
%%% fill-column: 72
%%% eval: (auto-fill-mode)
%%% coding: utf-8-unix
%%% End:
