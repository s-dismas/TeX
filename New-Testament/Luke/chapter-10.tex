%%%%%%%%%%%%%%%%%%%%%%%%%%%%%%%%%%%%%%%%%%%%%%%%%%%%%%%%%%%%%%%%%
%%%%
%%%% The (original) Douay Rheims Bible 
%%%%
%%%% New Testament
%%%% Luke
%%%% Chapter 10
%%%%
%%%%%%%%%%%%%%%%%%%%%%%%%%%%%%%%%%%%%%%%%%%%%%%%%%%%%%%%%%%%%%%%%


%%% Latin checked by KK.



\startcomponent chapter-10


\project douay-rheims


%%% 2438
%%% o-2266
\startChapter[
  title={Chapter 10}
  ]

\Summary{He ſendeth yet 72.\ moe to preach to the Iewes, with power alſo
  of miracles. 13.~crying woe to the cities impenitent. 17.~At their
  returne he agniſeth the great power he gaue them, but yet teacheth
  them not to be proud thereof, 21.~and praiſeth God for his grace,
  23.~his Church alſo for her happy ſtate. 25.~To one of the Scribes he
  sheweth, that the loue of God and of his neighbour wil bring him to
  life euerlaſting, 29.~teaching him by the parable of the Samaritane,
  to take euery one for his neighbour that needeth his charitie. 38.~To
  Martha he sheweth that Maries Contemplatiue life is the better.}

And after this our Lord deſigned alſo other
\SNote{As the twelue Apoſtles did repreſent the higher degree of the
Clergie, called Biſhops: ſo theſe Seuentie two beare the figure of the
inferiour Clergie, called Prieſts.
\Cite{Beda.}}
ſeuentie two: and he ſent them two and two before his face into euery
citie and place whither himſelf would come. \V And he ſaid to them: The
harueſt truely is much; but the workmen few. Deſire therfore the Lord of
the harueſt, that he ſend workmen into his harueſt. \V Goe: Behold I
ſend you as lambes among wolues. \V Carie not purſe nor skrip, nor
shoes; and ſalute no body by the way. \V Into whatſoeuer houſe you enter,
firſt ſay: Peace to this houſe. \V And if the ſonne of peace be
%%% o-2267
there, your peace ſhal reſt vpon him: but if not, it ſhal returne to
you. \V And in the ſame houſe tarie you, eating and drinking ſuch things
as they haue.
\CNote{\XRef{1.~Tim.~5,~18.}}
For the workman is worthie of his  hire. Remoue not from houſe to
houſe. \V And 
%%% 2439
into what citie ſoeuer you enter, and they receiue you, eate ſuch things
as are ſet before you; \V and cure the ſicke that are in it, and ſay to
them: The
\Fix{kingdom}{Kingdom}{likely typo, same in other}
of God is come nigh vpon you.

\V And into whatſoeuer citie you enter, and they receiue you not, going
forth into the ſtreetes thereof, ſay: \V The duſt alſo of your citie
that cleaueth to vs, we doe wipe off againſt you. Yet this know ye that
the Kingdom of God is at hand. \V I ſay to you, it ſhal be
\SNote{Differences of paines and damnation in Hel according to the
differences of demerites.
\Cite{Aug. li.~5. c.~5. cont. Iulian.}}
more tolerable for Sodom in that day, then for that citie. \V Woe to
thee Corazaim, woe to thee Beth-ſaida: for if in Tyre and Sidon had been
wrought the miracles that haue been wrought in you, they had done
penance ſitting
\SNote{True penance not onely to lead a new life, but to punish the body
by ſuch things as here be recorded, for the il life paſt.}
in ſake cloth and aſhes long agoe. \V But it ſhal be more tolerable for
Tyre and Sidon in the iudgement, then for you. \V And thou Capharnaum
that art exalted vnto Heauen: thou ſhalt be thruſt downe euen vnto
Hel. \V
\SNote{It is al one to deſpiſe Chriſt, and to deſpiſe his Prieſts and
Miniſters in the Catholike Church: to refuſe his doctrine, & theirs.}
He that heareth you, heareth me; and he that deſpiſeth you, deſpiſeth
me. And he that deſpiſeth me, deſpiſeth him that ſent me.

\V And the Seuentie-two returned with ioy, ſaying: Lord, the Diuels alſo
are ſubiect to vs in thy name. \V And he ſaid to them: I ſaw Satan as a
lighting fal from Heauen. \V Behold, I haue giuẽ you power to tread vpon
ſerpents, and ſcorpions, and vpon al the power of the enemie, and
nothing ſhal hurt you. \V But yet reioyce not in this, that the ſpirits
are ſubiect vnto you; but reioyce in this, that your names are written
in Heauen.

\V In that very houre he reioyced in ſpirit, and ſaid: I confeſſe to
thee O Father, Lord of Heauen and earth, becauſe thou haſt hid theſe
things from the wiſe and prudent, and haſt reuealed them
\LNote{The litle ones}{By this place euery vulgar artificer may not
preſume that God
hath reuealed al truth to him, and therfore refuſe to be taught of the
learned: for Chriſt did not afterward indow fishers and vulgar men nor
any other with the guifts of wiſedom and tongues, without their
induſtrie, ſtudy, and teaching: though at the beginning, of great
prouidence he did it, that it might be cleere to the world, that al
Nations were conuerted to him, not by perſuaſion of cunning Oratours or
ſubtil Diſputers, but by the plaine force of his grace and truth, which
S.~Auguſtine counteth greater then al other miracles.
\MNote{The humble vnlearned Catholike knoweth Chriſt better then the
proud learned Heretike.}
Further we are taught by this place, that the poore humble obedient
children of the Church know by their faith the high myſteries of
Chriſtes Diuinity, and his preſence in the B.~Sacrament, and ſuch like,
rather then Arius, Caluin, and other like proud Scribes and phariſees.}
to litle ones. Yea Father, for ſo hath it wel pleaſed thee. \V Al things
are deliuered to me of my Father. And no man knoweth who the Sonne is,
but the Father; and who the Father is, but the Sonne, and to whom the
Sonne wil reueale. \V And turning to his Diſciples, he ſaid: Bleſſed are
the eyes that ſee the things that you ſee. \V For I ſay to you, that
many Prophets and Kings deſired to ſee the things that you ſee, and ſaw
them not; and to heare the
%%% o-2268
things that you heare, and heard them not.

\V And behold a certaine lawyer ſtood vp, tempting him and ſaying:
Maiſter, by doing of what thing ſhal I poſſeſſe life euerlaſting? \V But
he ſaid to him: In the law what is written? how readeſt thou? \V He
anſwering ſaid:
\CNote{\XRef{Deu.~6,~3.}}
\Emph{Thou shalt loue the Lord thy God with thy whole
hart, and with thy whole ſoule, and with al thy ſtrength, and with al thy
mind:
\CNote{\XRef{Leu.~19,~18.}}
and thy neighbour as thy ſelf.} \V And he ſaid to him: Thou haſt
anſwered right,
\LNote{This doe}{Not by faith only, but by keeping Gods Commandements we
obtaine
life euerlaſting: not only by beleeuing, but by doing.
\MNote{The commandements poſsible to be kept.}
The heretikes ſay that is impoſſible to keepe this commandement of
louing God with al our hart. But the Scriptures giue vs examples of
diuers that haue kept and fulfilled it, as far as is requiſite in this
life. 
\XRef{3.~Reg.~14,~8.}
\XRef{2.~Par.~15,~15.}
\XRef{Ps.~118,~10.}
\XRef{Eccleſiaſtici. 47,~9,~10.}
\XRef{4.~Reg. 10,~3,5.}
\XRef{Luc.~1,~5.}
And if it were impoſsible to keepe it, and yet by Chriſt propoſed for
the meane to obtaine life euerlaſting, he had mocked this Lawyer and
others, and not taught them.}
this doe and thou ſhalt liue. \V But he deſirous to iuſtifie himſelf,
ſaid to \Sc{Iesvs}: And who is my neighbour? \V And \Sc{Iesvs} taking
it, ſaid: A certaine man went downe from Hieruſalem into Iericho, and
fel among theeues, who alſo ſpoiled him, and giuing him woundes went
away leauing him
\LNote{Halfe dead}{Here
\CNote{\Cite{Con. Araus.~2. c.~25. to.~1.}}
is ſignified man wounded very ſore in his vnderſtanding
and free wil, and al other powers of ſoule and body, by the ſinne of
Adam: but yet that neither vnderſtanding, nor free-wil, nor the reſt,
were extinguished in man or taken away. The Prieſt and Leuite ſignifie
the Law of Moyſes: this Samaritane is Chriſt the Prieſt of the new
Teſtament: the oile and wine, his Sacraments: the hoſt, the prieſts his
miniſters.
\CNote{\Cite{Trid. Seſs.~6. c.~1.}}
\MNote{The parable of the wounded man, explicated.}
Whereby is ſignified, that the Law could not recouer the ſpiritual life
of mankind from the death of ſinne, that is, iuſtifie man; but Chriſt
only, who by his Paſsion and the grace and vertue thereof miniſtred in
and by his Sacraments, iuſtifieth, and increaſeth the iuſtice of man,
healing and abling free-wil to doe al good workes.}
halfe-dead. \V And it chanced that a certaine Prieſt went downe the ſame
way; and ſeeing him, paſſed by. \V In like manner alſo a Leuite, when he
was neere the place, and ſaw him, paſſed by. \V But a certaine
Samaritane going his iourney, came neere him; and ſeeing him, was moued
with mercie. \V And going vnto him, bound his woundes,
%%% 2440
powring in oile and wine: and ſetting him vpon his owne beaſt, brought
him into an inne, and tooke care of him. \V And the next day he tooke
forth two pence, and gaue to the hoſt, and ſaid: Haue care of him; and
whatſoeuer thou shalt
\SNote{S.~Auguſtin ſaith that the Apoſtle
\XRef{(1.~Cor.~9.)}
according to this place did ſupererogate, that is, did more then he
needed or was bound to doe, when he might haue required al duties for
preaching the Ghoſpel, but would not.
\Cite{li. de op. Monach. c.~5.}
Whereof it cõmeth, that the workes which we doe more then precept, be
called workes of Supererogation: & whereby it is alſo euident againſt
the
\Fix{proteſtants}{Proteſtants}{obvious typo, fixed in other}
that there be ſuch workes. See
\Cite{Optatus li.~6. cont. Parmen.}
how aptly he applyeth this parable to S.~Paules counſel of virginitie
\XRef{(1.~Cor.~7.)}
as to a worke of ſupererogation.}
\TNote{\L{supererogaueris}, \G{προσδαπανήσῃς}.}
ſupererogate, I at my returne wil repay thee. \V Which of theſe three in
thy opinion was neighbour to him that fel among theeues? \V But he ſaid:
He that did mercie vpon him. And \Sc{Iesvs} ſaid to him: Goe, and doe
thou in like manner.

\V And it came to paſſe as they went, and he entred into a certaine
towne; and a certaine woman named Martha, receiued him into her
houſe, \V and ſhe had a ſiſter called Marie. Who ſitting alſo at our
Lords feete, heard his word. \V But Martha was buſie about much
ſeruice. Who ſtood and ſaid: Lord, haſt thou no care that my ſiſter hath
left me alone to ſerue? ſpeake to her therfore, that she help me. \V And
our Lord anſwering ſaid to her: Martha, Martha, thou art careful, and
art troubled about many things. \V But one thing is neceſſarie,
\LNote{Marie the beſt part}{Two notable examples, one of the life
Active, in Martha, the
other of the life
\Fix{contemplatiue,}{Contemplatiue,}{likely typo, fixed in other}
in Marie: repreſenting vnto vs, that in holy Church there should be
alwaies ſome to ſerue God in both theſe ſeueral ſorts.
\MNote{The Contemplatiue or Religious life, better then the Actiue and
ſecular.}
The life contemplatiue is here preferred before the actiue. The
Religious of both ſexes are of that more excellent ſtate. And therfore
our Proteſtants haue wholy abandoned them out of their cõmon-wealth,
which the true Church neuer wanted. But to ſay truth, they haue neither
Martha nor Marie. Our Lord giue them grace to ſee their miſerie. If ours
were not anſwerable to their profeſsiõ, or were degenerated, why haue
they no new ones? if our Churches Votaries vowed vnlawful things,
Chaſtitie, Pouertie, Obedience, Pilgrimage: what other Votaries or
lawful vowes haue they?
\MNote{Vowes and votaries.}
For, to offer voluntarily by vow (beſides the
keeping of Gods
\Fix{commandemens,}{commandements,}{obvious typo, fixed in other}
wherevnto we are bound by precept and
promiſe in our Baptiſme) our ſoules, bodies, goods, or any other
acceptable thing to God, is an acte of ſoueraigne worship belonging to
God only: & there was neuer true religion without ſuch vowes and
Votaries. If there be none in their whole Church that profeſſe
contemplation, or that vow any thing at al to God voluntarily, neither
in their bodies nor in their goods; God and the world know they haue no
Church nor religion at al.}
Marie hath choſen the beſt part which shal not be taken away from her.


\stopChapter


\stopcomponent


%%% Local Variables:
%%% mode: TeX
%%% eval: (long-s-mode)
%%% eval: (set-input-method "TeX")
%%% fill-column: 72
%%% eval: (auto-fill-mode)
%%% coding: utf-8-unix
%%% End:
