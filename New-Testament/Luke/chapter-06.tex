%%%%%%%%%%%%%%%%%%%%%%%%%%%%%%%%%%%%%%%%%%%%%%%%%%%%%%%%%%%%%%%%%
%%%%
%%%% The (original) Douay Rheims Bible 
%%%%
%%%% New Testament
%%%% Luke
%%%% Chapter 06
%%%%
%%%%%%%%%%%%%%%%%%%%%%%%%%%%%%%%%%%%%%%%%%%%%%%%%%%%%%%%%%%%%%%%%




\startcomponent chapter-06


\project douay-rheims


%%% 2425
%%% o-2251
\startChapter[
  title={Chapter 6}
  ]

\Summary{For reprouing by Scripture and miracle (as alſo by reaſon) the
  Phariſees blindnes about the obſeruation of the Sabboth, 11.~they ſeeke
  his death. 12.~Hauing in the mountaine prayed al night, he chooſeth
  twelue Apoſtles, 17.~and after many miracles vpon the diſeaſed, 20.~he
  maketh a ſermon to his Diſciples before the people: propoſing Heauen
  to ſuch as wil ſuffer for him, 24.~and woe to ſuch as wil not. 27.~Yet
  with al exhorting to doe good euen to our enemies alſo. 
  \Fix{19.~and}{39.~and}{obvious typo, same in other}
  that the Maiſters muſt firſt mend themſelues. 46.~finally to doe good
  works, becauſe only faith wil not ſuffice.}

And
\CNote{\XRef{Mt.~12,~1.}
\XRef{Mar.~2,~23.}}
it came to paſſe on the
\SNote{S.~Hierom
\Cite{(ep.~2. ad Nepotian.)}
writeth of himſelf, that being at Conſtãtinople, he aſked his maiſter
Gregorie Naziãzene the famous Doctour, then Bishop there, what Sabboth
this was. Who by his anſwer declared that it was very hard to tel:
neither is it yet knowẽ to the beſt learned. Yet the Proteſtants are wont to
ſay, Al is very eaſie.}
Sabboth ſecõd-firſt, when he paſſed through the corne, his Diſciples did
pluck the eares, & did eate rubbing them with their hands. \V And
certaine of the Phariſees ſaid to them: Why doe you that which is not
lawful on the Sabboths? \V And \Sc{Iesvs} anſwering thẽ, ſaid:
\LNote{Neither this haue you read}{The
\MNote{Heretikes vnderſtãd not the Scriptures.}
Scribes and Phariſees boaſted moſt of their knowledge of the Scripture;
but our Sauiour often ſheweth their great ignorance. Euen ſo the
Heretikes that now adaies vaunt moſt of the Scriptures and of their
vnderſtanding of them, may ſoone be proued to vnderſtand litle or
nothing.}
Neither this haue you read which Dauid did, when himſelf
%%% 2426
was an hungred & they that were with him: \V
\CNote{\XRef{1.~Re.~21,~4.}}
how he entred into the
houſe of God, and tooke the loaues of Propoſition, and did eate, and
gaue to them that were with him, which it is not lawful to eate
\CNote{\XRef{Leu.~24,~9.}}
but only
for Prieſts? \V And he ſaid to them: That the Sonne of man is Lord of
the Sabboth alſo.

\V And it came to paſſe on another Sabboth alſo, that he entred into the
Synagogue, and taught.
\CNote{\XRef{Mt.~12,~10.}
\XRef{Mr.~3,~1.}}
And there was a man, and his right hand was
withered. \V And the Scribes and Phariſees watched if he would cure on
the Sabboth; that they might find how to accuſe him. \V But he knew
their cogitations; and he ſaid to the man that had the withered hand:
Ariſe, and ſtand forth into the middes. And riſing he ſtood. \V
And \Sc{Iesvs} ſaid to them: I aske you, if it be lawful on the Sabboths
to doe wel or il; to
\LNote{Saue a ſoule}{Hereby it ſeemeth that Chriſt (as at other times lightly
alwaies) did not only heale this man in body, but of ſome correſpondent
diſeaſe in his ſoule.}
ſaue a ſoule or to deſtroy? \V And looking about vpon thẽ al, he ſaid to
the
%%% o-2252
mã: Stretch forth thy hãd. And he ſtretched it forth; & his hand was
reſtored. \V And they were repleniſhed with madnes; & they communed one
with another what they might doe to \Sc{Iesvs}.

\V And it came to paſſe in thoſe daies, he went forth into the mountaine
to pray, and he paſſed
\LNote{The whole night}{Our
\MNote{The Churches praiers at the times of giuing holy orders.}
Sauiour inſtantly prayed, alone in the mount without doore, al night
long, as a preparation to the deſignement of his Apoſtles the day after:
to giue example to the Church of praying inſtantly when Prieſts are to
be ordered, and a leſſon to vs al what we ſhould doe for our owne
neceſſities, when Chriſt did ſo for other mens.}
the whole night in the prayer of God. \V
\CNote{\XRef{Mt.~10,~1.}
\XRef{Mr.~3,~1.}
\XRef{6,~7.}
\XRef{Lu.~9,~1.}}
And when day was come, he
called his Diſciples; and he choſe twelue of them (
\LNote{Whom he named Apoſtles}{Here
\MNote{The name and dignitie of Apoſtles.}
it is to be noted againſt our Aduerſaries that deceitfully
meaſure to the ſimple the whole nature and qualitie of certaine
\Fix{ſacret}{ſacred}{obvious typo, fixed in other}
functions, by the primitiue ſignification & compaſſe of the names or
words whereby they be called; with whom as a Prieſt is but an elder, and
a Biſhop, a watchman or Superintendent, ſo an Apoſtle is nothing but a
Legate or Meſſenger, and therfore (as they argue)
\CNote{\Cite{Cal. Inſt. li.~4. c.~8.}}
can make no Lawes nor
preſcribe or teach any thing not expreſſed in his \Emph{mandatum}. Know
therfore againſt ſuch deceiuers, that ſuch things are not to be ruled by
the vulgar ſignification of the word or calling, but by vſe and
application of the holy writers, and in this point by Chriſts owne
expreſſe impoſition. And ſo this word \Emph{Apoſtle}, is a calling of
Office, gouernement, authoritie, and moſt high dignitie giuen by our
Maiſter, ſpecially to the College of the Twelue: whom he indowed aboue
that which the vulgar etymologie of their name requireth, with power to
bind and looſe, to puniſh and pardon, to teach and rule his Church. Out
of which roome and dignitie (which is called in the
\CNote{Pſ.~108,~8.}
Pſalme and in the
\CNote{Act.~1,~20.}
Actes a Biſhoprike) when Iudas fel, Mathias was choſen to ſupply it, &
was numbred among the reſt, who were as founders or
\CNote{Eph.~2,~20.}
foundations of our
religion, as the Apoſtle termeth them. Therfore to that college this name
agreeth by ſpecial impoſition & prerogatiue, though afterward it was by
vſe of the Scriptures
\CNote{Act.~14.}
extended to S.~Paul and S.~Barnabas, and ſometimes to the
%%% !!! Where do these go.
\CNote{1.~Cor.~12. Eph.~4,~11. 1.~Cor.~9,~2. Phil.~2,~25.}
Apoſtles Succeſſours: as alſo (by the like vſe of Scriptures) to
the firſt conuerters of countries to the faith, or their coadiutours in
that function. In which ſenſe S.~Paul chalengeth to be the Corinthians
Apoſtle, and nameth Epaphroditus the Philippians Apoſtle: as we cal
S.~Gregorie & his Diſciple S.~Auguſtin, our Apoſtles of England. In al
which taking, it euer ſignifieth Dignitie, Regiment, Paternitie,
Principalitie, and Primacie in the Church of God: according to S.~Paul
\XRef{1.~Cor.~12.}
\Emph{He hath placed in his Church, firſt indeed Apoſtles, &c.} Whereby
we may ſee that S.~Peters dignitie was a wonderful eminent prerogatiue
and ſoueraigntie, when he was the Head not only of other Chriſtian men,
but the Head of al Apoſtles, yea euen of the College of the Twelue. And
if our Aduerſaries liſt to haue learned any profitable leſſon by the
word Apoſtle, more profitably and truely they might haue gathered, that
Chriſt called theſe his principal officers,
\CNote{Lu.~4,~18. Heb.~3,~1.}
\Emph{Apoſtles},
or \Emph{Sent}, him ſelf alſo ſpecially and aboue al other
being \Emph{Miſſus}, that is, \Emph{Sent}, and called alſo Apoſtle in the
Scriptures; to warne vs by the nature of the word, that none are true
Apoſtles, Paſtours, or Preachers, that are not ſpecially ſent and called,
or that can not ſhew by whom they be ſent, & that al Heretikes therfore
be rather Apoſtates then Apoſtles, for that they be not ſent, not duely
called, nor choſen to preach.}
whom alſo he named \Emph{Apoſtles}) \V
\LNote{Simon}{Peter
\MNote{Peters preeminence.}
in the numbering of the Apoſtles, alwaies firſt named and preferred
before Andrew his elder brother and ſenior by calling. See
\XRef{Annotat. Mt.~10,~2.}}
Simon whom he ſurnamed Peter, and Andrew his brother, Iames and Iohn,
Philippe and Bartholomew, \V Matthew and Thomas, Iames of Alphæus and
Symon that is called Zelotes, \V and Iude of Iames, and Iudas Iſcariote
which was the traitour. \V And deſcending with them he ſtood in a plaine
place, and the multitude of his Diſciples, and a very great companie of
People from al Iewrie and Hieruſalem, and the ſea coaſt both of Tyre &
Sidon, \V which were come to heare him, and to be heaed of their
maladies. And they that were vexed of vncleane Spirits, were cured. \V
And al the multitude
\SNote{See
\XRef{S.~Mar. Annot. c.~5,~28.}}
ſought to touch him, becauſe vertue went forth from him, and healed
al. \V And he lifting vp his eyes vpon his Diſciples, ſaid:

\CNote{Mt.~5,~2. 6,~7.}
Bleſſed are ye poore: for yours is the Kingdom of God. \V Bleſſed are
you that now are an hungred: becauſe you ſhal be filled. Bleſſed are
you that now doe weepe: becauſe you ſhal laugh. \V Bleſſed ſhal you be
when men ſhal hate you, and when they ſhal ſeparate you, and vpbraid
you, and abandon your name as euil, for the Sonne of mans ſake. \V
\LNote{Be glad}{The
\MNote{Al perſecution for Chriſt is a bleſsing.}
common miſeries that fal to the true preachers and other Catholike men
for Chriſts ſake, as pouertie, famin, mourning, & perſecutions, be
indeed the greateſt bleſsing that can be, and are meritorious of the
reward of Heauen. Contrariewiſe, al the felicities of this world
without Chriſt, are indeed nothing but woe, and the entrance to
euerlaſting miſerie.}
Be glad in that day and reioyce; for behold, your reward is much in
Heauen. For according to theſe things did their Fathers to the
Prophets. \V But woe to you that are rich: becauſe you haue your
conſolation. \V Woe to you that are filled: becauſe you ſhal be
hungrie. Woe to you that now doe laugh: becauſe you ſhal mourne and
weep. \V Woe, when al men
\LNote{Shal bleſſe you}{This
\MNote{The vanitie of Heretical preachers.}
woe pertaineth to the Heretikes of our daies, that delight to haue the
Peoples praiſes and bleſsings & ſhouts, preaching pleaſant things of
purpoſe to their itching eares: as did the Falſe-Prophets, when they
were magnified and commended therfore of the carnal Iewes.}
ſhal bleſſe you: For according to theſe things did their Fathers to the
falſe-Prophets.

\V But to you I ſay that doe heare: Loue your enemies, doe good to them
that hate you. \V Bleſſe them that curſe you, and pray for them that
calumniate you. \V And he that ſtriketh thee on the cheeke, offer alſo
the other. And from him that taketh away from thee thy robe, prohibit
not thy coate alſo. \V And
\SNote{That is, to euery one iuſtly aſking. For that which is vniuſtly
aſked, may be iuſtly denyed.
\Cite{Aug. li.~1. c.~40. de Serm. Do. in monte.}}
to euery one that asketh thee, giue,
%%% 2427
and of him
%%% o-2253
that taketh away the things that are thine, aske not againe. \V And
according as you wil that men doe to you, doe you alſo to thẽ in like
manner. \V And if you loue them that loue you, what thanke is to you?
for ſinners alſo loue thoſe that loue them. \V And if ye doe good to
them that doe you good: what thanke is to you? for ſinners alſo doe
this. \V And if ye lend to them of whom ye hope to receaue; what thanke
is to you? for ſinners alſo lend vnto ſinners, for to receaue as
much. \V But loue ye your enemies; doe good and
\LNote{Lend, hoping nothing}{In that we may here ſeeme to be moued to
lend to thoſe whom we
thinke not able nor like euer to repay againe, it muſt be holden for a
counſel rather then a cõmandemẽt, except the caſe of neceſsitie.
\MNote{Againſt vſurie.}
But it
may be takẽ rather for a precept, wherein vſurie, that is to ſay, the
expectatiõ not of the money lẽt, but of vantage for lone, is forbiddẽ:
as by other places of Scripture it is condẽned, & is a thing againſt the
Law of nature & Nations. And great ſhame & pitie it is, that it ſhould
be ſo much vſed or ſuffered amõg Chriſtiãs, or ſo couered & cloked vnder
the habite of other cõtracts, as it is.}
lend, hoping for nothing thereby: and your reward ſhal be much, and you
ſhal be the Sonnes of the Higheſt, becauſe him ſelf is beneficial vpon
the vnkind and the euil. \V Be ye therfore merciful as alſo your Father
is merciful. \V Iudge not, & you ſhal not be iudged. Condemne not, & you
ſhal not be condemned. Forgiue, and you ſhal be forgiuen. \V Giue, and
there ſhal be giuen to you. Good meaſure & preſſed downe and ſhaken
togeather and running ouer ſhal they giue into your boſome. For with the
ſame meaſure that you doe mete, it ſhal be meaſured to you againe.

\V And he ſaid to them a ſimilitude alſo: Can the blind lead the blind?
doe not both fal into the ditch? \V The Diſciple is not aboue his
Maiſter: but euery one ſhal be perfect, if he be as his Maiſter. \V And
why ſeeſt thou the mote in thy brothers eye: but the beame that is in
thine owne eye thou conſidereſt not? \V Or how canſt thou ſay to thy
brother: Brother, let me caſt out the mote out of thine eye: thy ſelf
not ſeeing the beame in thine owne eye? Hypocrite, caſt firſt the beame
out of thine owne eye; and then ſhalt thou ſee clerely to take forth the
mote out of thy brothers eye.

\V For there is no good tree that yealdeth euil fruits; nor euil tree,
that yealdeth good fruit. \V For euery tree is knowen by his fruit. For
neither doe they gather figges of thornes; neither of a buſh doe they
gather the grape. \V The good man of the good treaſure of his hart
bringeth forth good; and the euil man of the il treaſure bringeth forth
euil. For of the aboundance of the hart the mouth ſpeaketh.

\V And why cal you me, Lord, Lord: and doe not the things which I
ſay? \V Euery one that commeth to me, and heareth my words, and doeth
them, I wil ſhew you to whom he is like. \V He is like to a man
\SNote{He buildeth right & ſurely, that hath both faith and good works:
he buildeth on ſand, that truſteth to his faith or reading or knowledge
of the ſcripture, & doth not worke or liue accordingly.}
building a houſe, that digged deep, and laid the foundation vpon a
rock. And when an inundation roſe, the riuer beatt againſt that
%%% o-2254
houſe, and it could not moue it; for it was founded vpon a rock. \V But
he that heareth, and doeth not; is like to a man building his houſe vpon
the earth without a foundation: againſt the which the riuer did beat;
and incontinent it fel, and the ruine of that houſe was great.


\stopChapter


\stopcomponent


%%% Local Variables:
%%% mode: TeX
%%% eval: (long-s-mode)
%%% eval: (set-input-method "TeX")
%%% fill-column: 72
%%% eval: (auto-fill-mode)
%%% coding: utf-8-unix
%%% End:
