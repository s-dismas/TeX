%%%%%%%%%%%%%%%%%%%%%%%%%%%%%%%%%%%%%%%%%%%%%%%%%%%%%%%%%%%%%%%%%
%%%%
%%%% The (original) Douay Rheims Bible 
%%%%
%%%% New Testament
%%%% Luke
%%%% Argument
%%%%
%%%%%%%%%%%%%%%%%%%%%%%%%%%%%%%%%%%%%%%%%%%%%%%%%%%%%%%%%%%%%%%%%




\startcomponent argument


\project douay-rheims


%%% 2409
%%% o-2233
\startArgument[
  title={\Sc{the argvment of s.~lvkes ghospel.}},
  marking={Argument of S.~Luke's Gospel}
  ]

S.~Lukes Ghoſpel may be deuided into fiue partes.

The firſt part is, of the Infancie both of the Precurſour, and of Chriſt
himſelfe: chap.~1.\ and 2.

The ſecond, of the Preparation that was made to the manifeſtation of
Chriſt: chap.~3.\ and a piece of the 4.

The third, of Chriſts manifeſting himſelfe, by preaching and miracles
ſpecially in Galilee: the other piece of the 4.~chap.\ vnto the middes
of the 17.

The fourth of his comming into Iurie towards his Paſsion: the other
piece of the 17.~chap.\ vnto the middes of the 19.

The fifth, of the Holy weeke of his Paſsion in Hieruſalem: the other
part of the 19.~chap.\ vnto the end of the booke.

S.~Luke was Sectatour
\CNote{\Cite{Hier. in Catalogo.}}
(ſaith S.~Hierome) \Emph{that is, a Diſciple of
the Apoſtle Paul, and a companion of al his peregrination.} And the ſame
we ſee in the Actes of the Apoſtles: Where, from the
\XRef{16.~chap.}
S.~Luke putteth himſelf in the traine of S.~Paul, writing thus in the
ſtorie. \Emph{Forthwith we ſought to goe into Macedonia.} And in like
manner, in the firſt perſon, commonly through the reſt of that booke. Of
him and his Ghoſpel, S.~Hierom vnderſtandeth this ſaying of
S.~Paul:
\CNote{\XRef{2.~Cor.~8,~18.}}
\Emph{We haue ſent with him the brother, whoſe praiſe is in the
Ghoſpel through al Churches.} Where alſo he addeth: \Emph{Some ſuppoſe,
ſo often as Paul in his Epiſtles ſaith,} According to my
Ghoſpel, \Emph{that he meaneth of Lukes booke.} And againe: \Emph{Luke
learned the Ghoſpel not only of the Apoſtle Paul, who had not been with
our Lord in fleſh, but of the other Apoſtles; which himſelf alſo in the
beginning of his booke declareth, ſaying:
\CNote{\XRef{Luc.~1,~2.}}
As they deliuered to vs who
them ſelues from the beginning ſaw, & were Miniſters of the Word.} It
foloweth in S.~Hierome: \Emph{Therfore he wrote the Ghoſpel, as he had
heard; but the Actes of the Apoſtles he compiled as he had ſeen.}
S.~Paul writeth of him by name to the Coloſsians:
\CNote{\XRef{Col.~4,~14.}}
\Emph{Luke the
Phiſiciõ ſaluteth you.} And to Timothie:
\CNote{\XRef{2.~Tim.~4,~11.}}
\Emph{Luke alone is with me.}
Finally of his end thus doth S.~Hierom write:
\CNote{\Cite{Hiero. in Catalogo.}}
\Emph{He liued foureſcore
and foure yeares, hauing no wife. He is buried at Conſtantinople; to
which citie his bones with the Relikes of Andrew the Apoſtle were
tranſlated out of Achaia the twentith yeare of
\Var{Conſtantinus:}{Conſtantius.}
} And of the ſame Tranſlation alſo in an
other place againſt Vigilantius the Heretike:
\CNote{\Cite{Hier. con. Vigil. c.~2.}}
\Emph{It grieueth him that the
Relikes of the Martyrs are couered with pretious couerings, and that
they are not either tied in cloutes or throwen to the dunghil.
\MNote{The Heretike ſo counted the Catholikes for their honouring of
Saints and Relikes.}
Why, are
we then ſacrilegious, when we enter the Churches of the Apoſtles? Was
\Var{Conſtantinus}{Conſtantius.}
the Emperour ſacrilegious, who tranſlated to Conſtantinople the holy
Relikes of Andrew, Luke, and Timothie, at which the Diuels rore, and the
inhabiters of Vigilantius confeſſe that they feele their preſence?}

His ſacred body is now at Padua in Italie; Whither it was againe
tranſlated from Conſtantinople.

\stopArgument


\stopcomponent


%%% Local Variables:
%%% mode: TeX
%%% eval: (long-s-mode)
%%% eval: (set-input-method "TeX")
%%% fill-column: 72
%%% eval: (auto-fill-mode)
%%% coding: utf-8-unix
%%% End:
