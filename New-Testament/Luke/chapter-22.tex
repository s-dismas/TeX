%%%%%%%%%%%%%%%%%%%%%%%%%%%%%%%%%%%%%%%%%%%%%%%%%%%%%%%%%%%%%%%%%
%%%%
%%%% The (original) Douay Rheims Bible 
%%%%
%%%% New Testament
%%%% Luke
%%%% Chapter 22
%%%%
%%%%%%%%%%%%%%%%%%%%%%%%%%%%%%%%%%%%%%%%%%%%%%%%%%%%%%%%%%%%%%%%%




\startcomponent chapter-22


\project douay-rheims


%%% 2468
%%% o-2300
\startChapter[
  title={Chapter 22}
  ]

\Summary{Iudas doth ſel him to the Iewes. 7.~After the old Paſcal,
  19.~he giueth to his Diſciples the bread of life in a myſtical
  Sacrifice of his body and bloud, for an euerlaſting commemoration of
  his Paſsion. 21.~He couertly admonisheth the traitour. 24.~Againſt
  their ambitious contention he sheweth them that the maioritie of any
  among them in this world is for their ſeruice, as his owne alſo was: 28.~&
  how he wil exalt them al in the world to come: 31.~foretelling Peter
  the ſingular priuiledge of his faith neuer failing, 33.~and his three
  negations: 35.~and how they shal al now be put to their
  shiftes. 39.~And that night, after his praier with ſweating of bloud,
  42.~he is taken of the Iewes men, Iudas being their Capitaine: yet
  shewing them both by miracle and word, that they could doe nothing
  vnto him but by his owne permiſsion. 54.~Then in the cheefe Prieſtes
  houſe he is thriſe denied of Peter, 63.~shamefully abuſed of his
  keepers, 66.~and in the morning impiouſly condemned of their Councel,
  for confeſsing himſelf to be the Sonne of God.}

%%% o-2301
And
\CNote{\XRef{Mt.~26,~1.}
\XRef{Mr.~14,~11.}}
\MNote{\Sc{Tenebre} weneſday.}
the feſtiual day of the Azymes approched, which is called Paſche: \V &
the cheefe Prieſts & the Scribes ſought how they might kil him: but they
feared the people. \V And Satan entred into Iudas that was ſurnamed
Iſcariote, one of the Twelue. \V And he went, and talked with the cheefe
Prieſts and the Magiſtrates, how he might betray him to them. \V And
they were glad, and bargained to giue him money. \V And he promiſed. And
he ſought opportunitie to betray him apart from the multitudes.

\V
\CNote{\XRef{Mt.~26,~17.}
\XRef{Mr.~14,~12.}}
\MNote{\Sc{Mavndi} Thurſday.}
And the day of the Azymes came, wherein it was neceſſarie that the
Paſche ſhould be killed. \V And he ſent Peter and Iohn, ſaying: Goe and
prepare vs the Paſche, that we may eate. \V But they ſaid: Where wilt
thou that we prepare it? \V And he ſaid to them: Behold, as you enter
into the citie, there ſhal meete you a man carying a pitcher of water:
follow him into the houſe into which he entreth, \V and you ſhal ſay to
the Good-man of the houſe: The Maiſter ſaith to thee, where is the inne
where I may eate the Paſche with my Diſciples? \V And he wil ſhew you a
great refectorie adorned: and there prepare. \V And they going, found as
he ſaid to them, and prepared the Paſche.

\V And when the houre was come, he ſate downe, and the twelue Apoſtles
with him. \V And he ſaid to them:
\LNote{With deſire I haue deſired}{This
\MNote{The old Paſchal ceaſeth and a new is inſtituted.}
great deſire he had to eate this Paſchal lambe, was not for it ſelf,
which he had celebrated many yeares before: but becauſe he meant
immediatly after the Paſchal of the Law was ſacrificed & eaten, to
inſtitute the other new Paſchal in the oblation and eating of his owne
body, by which the old Paſchal should end and be fulfilled, and in which
the old Teſtament and Law ceaſing, the Kingdom of God (which is the
ſtate of the new Teſtament and of his Church) should begin. For, the
very paſſage from the old Law to the new was in this one ſupper.}
With deſire I haue deſired to eate this Paſche with you before I
ſuffer. \V For I ſay to you, that from this time I wil not eate it, til
it be fulfilled in the Kingdom of God. \V And
\LNote{Taking the chalice}{This chalice according to the very euidence
of the text it ſelf alſo, is not the ſecond part of the Holy Sacrament,
but that ſolemne cup of wine which belonged as a libament to the
offering and eating of the Paſchal lambe.
\MNote{Two cups or chalices at Chriſts laſt ſupper.}
Which being a figure ſpecially of the holy Chalice, was
there drunken by our Sauiour, and giuen to the Apoſtles alſo, with
declaration that it should be the laſt cuppe of the Law, not to be
drunken any more, til it should be drunken new in the Kingdom of God,
that is to ſay, in the celebration of the B.~Sacrament of his bloud of
the new Teſtament. And by this place it ſeemeth very like that the
wordes in S.~Matthew,
\CNote{\XRef{Mt.~26,~29.}}
\Emph{I wil not drinke of the fruit of the vine
&c}, were pertaining to this cuppe of the old Law, and not to the Holy
Sacrament, though they be there by repetition or recapitulation ſpoken
after the holy Chalice.}
taking the chalice he gaue thankes, and ſaid: Take and deuide among
you. \V For I ſay to you, that I wil not drinke of the generation of the
vine, til the Kingdom of God doe come.

\V
\CNote{\XRef{Mt.~26,~26.}
\XRef{Mr.~14,~22.}
\XRef{1.~Cor.~11,~24.}}
And taking bread, he gaue thankes, and brake; and gaue to them, ſaying:
\LNote{This is my body}{\Emph{Although
\MNote{The real preſence.}
ſenſe tel thee it is bread, yet it is the body, according to his wordes,
let faith confirme thee, iudge not by ſenſe. After the wordes of our
Lord let  no doubt riſe in thy mind.}
\Cite{Cyril. myſtag.~4.}
\Emph{Of the veritie of flesh and bloud there is left no place to doubt:
by the profeſsion of our Lord him ſelf, and by our faith it is flesh and
bloud indeed. Is not this truth? To them be it vntrue, which
deny} \Sc{Iesvs Christ} \Emph{to be true God.}
\Cite{Hilar. li.~2. de Trinit.}}
\Sc{This is my body
\LNote{Which is giuen}{As
\MNote{Chriſt ſacrificed his body and bloud in Sacrament at his ſupper.}
the former wordes make and proue his body preſent, ſo theſe wordes
plainely ſignifie, that it is preſent, as giuen, offered or ſacrificed
for vs: and being vttered in the
\TNote{\L{quod datur} \G{τὸ διδόμενον}}
preſent tence, it ſignifieth not only that it should afterward be giuen
or offered on the Croſſe, but that it was then alſo in the Sacrament
giuen and offered for vs. Whereby it is inuincibly proued that his Body
is preſent as an Hoſt or Sacrifice: and that the making or conſecrating
thereof muſt needes be Sacrificing. And therfore the holy Fathers in
this ſenſe cal it a Sacrifice.
\MNote{The Sacrifice of the
\Fix{Alter.}{Altar.}{obvious typo, fixed in other}}
%%% !!! Where does this go?
\CNote{\Cite{Cyril. Alex. anathẽ.~11.}}
\Cite{Niſſen. orat.~1. de reſur.}
\Cite{Leo ſer.~7. et~8. de Paſs.}
\Cite{Heſychius li.~2. in Leuit. c.~8.}
\Cite{Grego. ho.~37. in Euan. et Dial. li.~4. c.~59.}
\Cite{Cyrillus Hieroſ. myſtag.~5.}
\Cite{Dionys. Eccl. Hier. c.~3.}
\Cite{Ignat. ep.~6. ad Smyrn. Iuſtinus dial. cum Tryph. circ. med.}
\Cite{Iren. li.~4. c.~32. et~34.}
\Cite{Tertul. de cult fam. et vxor. li.~2.}
\Cite{Cypr. ep. ad Cæcil. et de Cæn. Do.}
\Cite{Euſeb. Semonſt. euang. li.~1. c.~10.}
\Cite{Nazian orat~1. cont. Iulianum}
\Cite{Chryſo. ho.~83. in 26.~Mat. et li.~6. de Sacerd.}
\Cite{Ambros. li.~4. de Sacram. c.~6. et li.~1. Offic. c.~48.}
\Cite{Hiero. in ep. ad Hebid. q.~2. et ad Euagr. ep.~126. to.~3.}
%%% !!! I no longer have any idea where separate citations begin or
%%% end. The rest is just all run together.
\Cite{Auguſt. in pſal.~33. conc.~1. et
alibi ſape. Græci omnes in 9.~Hebr. et
Primaſius. Conc. Nic.~1.~14. Ephes. ad
Neſtor. Conſtantinop.~6. can.~32. Nicen. 2.~act.6. to.~3.
Lateran. Conſtant. Flor. Trid.}}
which is given for yov.}
\LNote{Doe this}{In
\MNote{The Apoſtles are made Prieſtes, & the Sacrament of holy Orders
inſtituted.}
theſe wordes the holy Sacrament of Order is inſtituted, becauſe power
and commiſsion to doe the principal act & worke of Prieſthood, is giuen
to the Apoſtles: that is, to doe that which Chriſt then did concerning
his body: which was, to make & offer his body as a Sacrifice for vs and
for al that haue need of Sacrifice, & to giue it to be eaten as Chriſtes
body ſacrificed, to al faithful. For as the Paſchal lambe was firſt
ſacrificed, and then eaten; ſo was his body: and thus to doe he here
giueth commiſsion and authoritie to the Apoſtles, & to al Prieſts which
be their ſucceſſours in this matter.
\Cite{Dionys. cal. Hierar. c.~3.}
\Cite{Iren. li.~4. c.~32.}
\Cite{Cyp. ep. ad Cecil.}
\Cite{Chrys. ho.~17. in ep. ad Heb.}
\Cite{Ambros. in Pſ.~38. & in c.~10. ad Hebr.}}
Doe this
\LNote{For a commemoration}{This
\MNote{A commemoratiue Sacrifice is a true Sacrifice, no leſſe then the
prefiguratiue Sacrifices were true Sacrifices.}
Sacrifice and Sacrament is to be done perpetually in the Church for the
commemoration of Chriſt, ſpecially of his Paſsion: that is to ſay, that
it may be a liuely repreſentation, exemplar, and forme of his Sacrifice
vpon the croſſe. Of which one oblation on the croſſe, not only al other
Sacrifices of the Law were figures, but this alſo: though this in a more
nigh, high, myſtical, and maruelous ſort then any other. For in them
Chriſts death was ſignified as by reſemblance and ſimilitudes of
external creatures and bodies of brute beaſts: but in this of the new
Teſtament, his body viſibly ſacrificed on the croſſe, in and by the ſelf
ſame body ſacrificed and immolated in Sacrament and vnder the shapes of
bread and wine, is moſt neerely and perfectly reſembled. And therfore
this is moſt properly commemoratiue, as moſt neerely expreſsing the very
condition, nature, efficacie, ſort, and ſubſtance of that on the
croſſe. For which the
\CNote{\Cite{Ambr. in 10.~Hebr.}
\Cite{Chryſ. ho.~17. in ep. ad Hebr.}}
holy Fathers cal it the very ſelf ſame ſacrifice
(though in other manner) which was done on the croſſe, as it is the ſelf
ſame thing, that is offered in the Sacrament, & on the croſſe. Whereby
you may ſee the peruerſitie of the Proteſtants or their ignorance, that
thinke it therfore not to be Chriſts body becauſe it is a memorie of his
body or a figure of his body vpon the croſſe: nor to be a true Sacrifice
becauſe it is a commemoratiue Sacrifice. For as the thing that more
liuely, neerely, & truely reſembleth or repreſenteth, is a better figure
then that which shadoweth it a far off: ſo this his body in the
Sacrament, is more perfectly a figure of Chriſts body & Sacrifice, then
any other.
\MNote{To be a figure of a thing, and yet the thing it ſelf, repugneth
not.}
Chriſt himſelf the Sõne of God is a figure & character of his
Fathers Perſon, being yet of the ſelf ſame ſubſtãce. And Chriſts body
transfigured on the holy Mount, was a figure & reſemblance of his Perſon
glorified in Heauen. Euen ſo is his body in the Sacramẽt to a faithful
mã that knoweth by his beleefe grounded on Chriſts owne word, that in
the one forme is his body, in the other his bloud, the moſt perfect
repreſentatiõ of his death that cã be. As for the Sacrifice, it is no
leſſe a true Sacrifice, becauſe it is commemoratiue of Chriſts Paſsion,
then thoſe of the old Teſtament were the leſſe true, becauſe they were
prefiguratiue. For that is the condition annexed to al Sacrifice of
euery Law, to repreſent Chriſts Paſsion.}
for a commemoration of me. \V In like manner the chalice alſo, after he
had ſupped, ſaying:
\SNote{The Greeke is here ſo plaine, that there was very bloud in the
chalice shed for vs, that Beza ſaith it is a corruption in the
greeke. See the
\XRef{Annot. vpon this place.}}
\Sc{This is the chalice
\LNote{The new Teſtament in my bloud}{Moyſes tooke the bloud of the
firſt Sacrifice that was made after the giuing of the Law
\XRef{Exod.~14.}
and with bloud confirmed the couenant & compact betwixt God and his
people, and ſo dedicated the \Emph{old Teſtament}, which without bloud
\CNote{\XRef{Hebr.~9.}}
(ſaith S.~Paul) was not dedicated. Moyſes put that bloud alſo into a
ſtanding peece, & ſprinkled al the people &c. with the ſame, & ſaid
theſe formal wordes:
\MNote{Both Teſtaments dedicated in bloud.}
\Emph{This is the bloud of the couenant &c.} or (as it is read in
S.~Paul)
%%% !!! Only in other
\CNote{\XRef{verſ.~20.}}
\Emph{of the Teſtament which God hath deliuered vnto you.} Vnto
al which, Chriſt in this action about the ſecond part of this his
Sacrifice, in euery of the Euangeliſts moſt cleerely alludeth:
expreſsing that the \Emph{new Teſtament} is begun and dedicated in his
bloud in the Chalice, no leſſe then the old was dedicated, begun, and
ratified in that bloud of calues conteined in the goblet of
Moyſes. With which his owne bloud he ſprinkled inwardly his Apoſtles as
the firſt fruits of the new Teſtament, imitating the wordes of Moyſes,
and ſaying: \Emph{This is the Chalice of the new Teſtament &c}: Which
the other Euangeliſts ſpake more plainly: \Emph{This is my bloud of the
new Teſtament.}
\MNote{The external religion of the new Teſtament principally in the
Sacrifice of the Altar.}
By al which it is moſt certaine, that Chriſts bloud in the Chalice, is
the bloud of Sacrifice, and that in this Sacrifice of the Altar
conſiſteth the external religion and proper ſeruice of the new
Teſtament, no leſſe then the ſoueraigne worship of God in the old Law
did conſiſt in the Sacrifices of the ſame. For though Chriſts Sacrifice
on the Croſſe and his bloud shed for vs there, be the general price,
redemption, and ſatisfaction for vs al, and is the laſt & perfecteſt
ſealing or confirmation of the new law & Teſtament: yet the Seruice &
Sacrifice which the people of the new Teſtament might reſort vnto could
not be that violent action of the Croſſe, but this on the Altar, which
by Chriſts owne appointment is & shal be the eternal office of the new
Teſtament, & the continual application of al the benefites of his
Paſsion vnto vs.}
the new
%%% 2469
testament in my blovd,
\LNote{Which shal be shed}{It
\MNote{\Emph{The chalice shed for vs}, muſt needes ſignifie, the bloud
therein, not wine, and the ſame Sacrificed.}
\TNote{\L{calix qui} the chalice which \G{τὸ ποτήριον τὸ ἐκχυννόμενον}}
is much to be obſerued that the relatiue, \Emph{which}, in theſe wordes
is not gouerned or ruled (as ſome would perhaps thinke) of the
nowne \Emph{bloud}, but of the word \Emph{chalice}. Which is moſt plaine
by the Greeke: Which taketh away al cauillations and shifts from the
Proteſtants, both againſt the real preſence & the true Sacrificing. For
it sheweth euidently, that the bloud as the contents of the chalice, or
as in the chalice, is shed for vs (for ſo the Greeke readeth in the
preſent tenſe) & not only as vpon the croſſe. And therfore as it
followeth thereof inuincibly, that it is no bare figure, but his bloud indeed,
ſo it enſueth neceſſarily, that it is a Sacrifice and propitiatorie,
becauſe the chalice (that is the Bloud contained in the ſame) is shed
for our ſinnes. For al that know the manner of the Scriptures ſpeaches,
know alſo that, \Emph{Bloud to be shed for ſinnes}, is to be ſacrificed
for propitiation or for pardon of ſinnes.
\MNote{Beza condemneth the Ghoſpel it ſelf of falshood and
impoſſibilitie.}
And this text proueth al this ſo plainly, that
\CNote{\Cite{Annot. no. Teſt.~1556.}}
Beza turneth himſelf roundly vpon
the Holy Euangeliſt, charging him with Solœciſme or falſe Greeke, or els
that the wordes (which yet he confeſſeth to be in al copies Greeke &
Latin) are thruſt into the text out of ſome other place: which he rather
ſtandeth vpon then that S.~Luke ſhould ſpeake incongruouſly in ſo plaine
a matter. And therfore he ſaith plainely that it can not be truely ſaid
neither of the chalice it ſelf nor of the contents thereof: which is
indeed to giue the lie to the Bleſſed Euangeliſt, or to deny this to be
Scripture. So cleere is the Scripture for vs, ſo miſerable flights and
shifts is falshood put vnto, God be thanked.}
which shal be shed for yov.}

\V
\CNote{\XRef{Mt.~26,~21.}
\XRef{Mr.~14,~20.}
\XRef{Io.~13,~18.}}
But yet behold, the hand of him that betraieth me, is with me on the
table. \V And the Sonne of man indeed
%%% o-2302
goeth according to that which is
determined: but yet woe to that man by whom he ſhal be betrayed. \V And
they began to queſtion among them ſelues, which of them it ſhould be
that ſhould doe this.

\V
\CNote{\XRef{Mt.~20,~25.}
\XRef{Mr.~10,~42.}}
And there fel alſo a 
%%% !!! unmarked LNote, but marked in other
\LNote{Contention}{The
\MNote{Ambition.}
Apoſtles perceiuing Chriſts departure from them and his Kingdom to be
neere, as infirme men and not yet indowed with the Spirit of God, began
to haue emulation & cogitations of Superiority one ouer another which
our Maiſter repreſſeth in them by exhortation to humilitie and by his
owne example, that being their Lord, yet ſo lately ſerued them: not
forbidding Maioritie or Superioritie in them, but pride, tyranny, &
contempt of their inferiours.}
contention between them, which of them ſeemed to
be greater. \V And he ſaid to them: The Kinges of the Gentiles ouerrule
them; and they that haue power vpon them, are called beneficial. \V But
you not ſo: but he that is the greater among you, let him become as the
yonger: & he that is the leader, as the waiter. \V For which is greater,
he that ſitteth at the table, or he that miniſtreth? is not he that
ſitteth? but I am in the middes of you, as he that miniſtreth: \V & you
are they that haue remained with me in my tentations. \V And I diſpoſe
to you, as my Father diſpoſed to me, a Kingdom: \V that you may eate &
drinke vpon my table in my Kingdom, & may ſit
\SNote{Straight after the former louing checke & admonition, he
promiſeth to them al that haue beene partakers with him of his miſeries
in this life, greater preeminence in
\Fix{heauen,}{Heauen,}{likely typo, same in other}
then any Potentate can haue in this world, & therfore that they need not
be careful of dignitie or Supremacie.}
vpon thrones, iudging the twelue tribes of Iſrael.

\V And our Lord ſaid:
\LNote{Simon Simon}{Laſtly
\MNote{Peters faith ſhal neuer faile.}
to put them out of doubt, he calleth Peter twiſe by name, and telling him the
Diuels deſire to ſifte & trie them al to the vttermoſt (as he did that
night) ſaith that he hath ſpecially prayed for him, to this end that his
faith should neuer faile, & that he being once conuerted, should after
that for euer confirme, eſtablish or vphold the reſt in their
faith. Which is to ſay, that Peter is that man whom he would make
Superiour ouer them and the whole Church. Whereby we may learne that it
was thought fit in the prouidence of God, that he who should be the Head
of the Church, should haue a ſpecial priuiledge by Chriſtes praier &
promiſe neuer to faile in faith & that none other either Apoſtle,
Biſhop, or Prieſt may chalẽge any ſuch ſingular or ſpecial prerogatiue
either of his Office or perſon, otherwiſe thẽ ioyning in faith with
Peter & by holding of him. The danger
\CNote{\Cite{Serm.~3. Aſsump. ad Pont. Li.~q. Noui. Teſt. q.~75. to.~4.}}
(ſaith S.~Leo) was cõmon to al the
Apoſtles, but our Lord tooke ſpecial care of Peter, that the ſtate of al
the reſt might be more ſure, if the Head were inuincible: God ſo
diſpenſing the aide of his grace, that the aſſurance & ſtrength which
Chriſt gaue to Peter, might redoũd by Peter to the reſt of the Apoſtles:
S.~Auguſt. alſo, Chriſt praying for Peter, prayed for the reſt, becauſe
in the Paſtour & Prelate the people is corrected or cõmended. And
S.~Ambroſe writeth, that Peter after his tentation was made Paſtour of
the Church, becauſe it was ſaid to him: \Emph{Thou being conuerted,
confirme thy brethren.}
\MNote{The Romane faith of Peters ſucceſſours  cannot faile.}
Neither was this the priuiledge of S.~Peters perſon, but of his Office,
that he ſhould not faile in faith but euer confirme al other in their
faith. For the Church, for whoſe ſake that priuiledge was thought
neceſſarie in Peter the Head thereof, was to be preſerued no leſſe
afterward, thẽ in the Apoſtles time. Whereupon al the Fathers apply this
priuiledge of not failing & of confirming other in faith, to the Romane
Church & Peters ſucceſſours in the ſame. To which
\CNote{\Cite{Cypr. ep.~55. nu.~6.}}
(ſaith S.~Cyprian)
infidelitie or falſe faith can not come. And
\CNote{\Cite{Bern. ep.~190.}}
S.~Bernard ſaith writing to
Innocentius Pope, againſt Abailardus the Heretike: We muſt referre to
your Apoſtleſhip al the ſcandals and perils which may fal, in matter of
faith ſpecially. For there the defects of faith muſt be holpen, where
faith can not faile. For to what other See was it euer ſaid: \Emph{I
haue prayed for thee Peter, that thy faith doe not faile}?
\MNote{Popes may erre perſonally, not iudicially, or definitiuely.}
So ſay the Fathers: not meaning that none of Peters ſeat can erre in
perſon, vnderſtanding, priuate doctrine or writings, but that they can
not nor ſhal not euer iudicially conclude or giue definitiue ſentence
for falshood or hereſie againſt the Catholike faith, in their
Conſiſtories, Courts, Councels, Decrees, Deliberations or Conſultations
kept for deciſion and determination of ſuch controuerſies, doubts, or
queſtions of faith as ſhal be propoſed vnto them: becauſe Chriſtes
prayer and promiſe protecteth them therein for confirmation of their
brethren. And no maruel that our Maiſter would haue his Vicars
Conſiſtorie & Seat infallible, ſeeing euen in the
\CNote{\XRef{Deu.~17.}}
old Law the high
Prieſthood & Chaire of Moyſes wanted not great priuiledge in this caſe,
though nothing like the Churches and Peters prerogatiue. But in both,
any man of ſenſe may ſee the difference between the perſon, and the
Office, as wel in doctrine as life. Liberius in perſecution might yeald,
Marcellinus for feare might commit Idolatrie, Honorius might fal to
Hereſie, and more then al this, ſome Iudas might creepe into the
Office: and yet al this without preiudice of the Office and
Seat, \Emph{in which}
\CNote{\Cite{Aug. ep.~166. in fine.}}
(ſaith S.~Auguſtin) \Emph{our Lord hath ſet the
doctrine of truth.} Caiphas by priuiledge of his Office prophecied right
of Chriſt, but according to his owne knowledge and faith, knew not
Chriſt. The Euangeliſts and other penners of holy writ, for the
execution of that function had the aſſiſtãce of God, & ſo farre could
not poſsibly erre: but that Luke, Marke, Salomõ or the reſt might not
erre in other their priuate writings; that we ſay not.
\MNote{The learned
\Fix{fathers}{Fathers}{likely typo, same in other}
ſought to the B.~of Rome for reſolution of doubts.}
It was not the perſonal wiſedõ, vertue, learning, or faith of Chriſts
Vicars, that made
\CNote{\Cite{Bern. ep.~190.}}
S.~Bernard ſeeke to Innocentius the third:
S.~Auguſtine and the Bishops of Afrike to Innocentius the firſt, and to
Celeſtinus,
\Cite{ep.~90. 91.~95.}:
\CNote{\Cite{Chry. ep.~1. &~2.}}
S.~Chriſoſtome to the ſaid Innocentius: S.~Baſil to the Pope in his time
\Cite{ep.~52.}:
S.~Hierom to Damaſus
\Cite{ep.~57. 58. to.~2.}
but it was the prerogatiue of their Office and higher degree of Vnction,
& Chriſts ordinance, that would haue al Apoſtles and Paſtours in the
world, for their confirmation in faith and Eccleſiaſtical regiment,
depend on Peter. The lacke of knowledge and humble acceptation of which
Gods prouidence, that is, that one is not honoured and obeyed of al the
brotherhood, is the cauſe of al Schiſmes and Hereſies, ſaith
\CNote{\Cite{Cyp. ep.~55. nu.~2.}}
S.~Cyprian. A point of ſuch importance, that al the Twelue being in
Apoſtleſhip like, Chriſt would yet for the better keeping of vnity &
truth, haue one to be Head of them al, that a Head being once appointed,
occaſion of Schiſme might be taken away, ſaith 
\Cite{S.~Hierom. li.~1. adu. Iouinian. c.~14.}}
Simon, Simon, behold Satan hath required to haue you for to ſift as
wheate: \V \Sc{Bvt I have praied for thee}, that thy faith faile not:
and thou once conuerted, confirme thy brethren. \V Who ſaid to him:
Lord, with thee I am readie to goe both into priſon and vnto death. \V
And he ſaid:
\CNote{\XRef{Mt.~26,~34.}
\XRef{Mr.~14,~30.}}
I ſay to thee Peter, the cocke ſhal not crow to day, til
thou denie thriſe that thou knoweſt me. \V And he ſaid to them: when I ſent
you
\CNote{\XRef{Mt.~10,~9.}
\XRef{Luc.~10,~4.}}
without purſe and ſkrip and ſhoes, did you lacke any thing? But they
ſaid: Nothing. \V He ſaid therfore vnto them: But now he that hath a
purſe, let him take it, likewiſe alſo a skrip: and he that hath not, let
him ſel his coate, and buy a ſword. \V For I ſay to you, that yet this
that is written muſt be fulfilled in me:
\CNote{\XRef{Eſ.~53,~12.}}
\Emph{And with the wicked was
he reputed}. For thoſe things that are concerning me, haue
\Fix{and}{an}{obvious typo, fixed in other}
end. \V But they ſaid: Lord, loe two ſwordes here. But he ſaid to them:
It is enough. \V
\CNote{\XRef{Mt.~26,~36.}
\XRef{Mr.~14,~32.}
\XRef{Io.~18,~1.}}
\MNote{Thurſday night.}
And going forth he went according to his cuſtome into
mount-Oliuet. And his Diſciples alſo followed him. \V And when he was
come to the place, he ſaid to them: Pray, leſt ye enter into
tentation. \V And he was pulled away from them a ſtones caſt: and
kneeling he praied, \V ſaying: Father, if thou wilt, transferre this
chalice from me. But yet not my wil, but thine be done. \V And there
appeared to him an Angel from Heauen, ſtrengthening him. And being in an
agonie, he praied the longer. \V And his ſweat became
%%% o-2303
as drops of bloud trikling downe vpon the earth. \V And when he was
riſen vp from praier, and was come to his Diſciples, he found them
ſleeping for penſiuenes. \V And he ſaid to them: Why ſleep you? ariſe,
pray, leſt you enter into tentation.

\V As he was yet ſpeaking, behold a multitude: and he that was
called Iudas, one of the Twelue, went before them, and approched
to \Sc{Iesvs}, for to kiſſe him. \V And \Sc{Iesvs} ſaid to him: Iudas
with a kiſſe doeſt thou betray the Sonne of man? \V And they that were
about him, ſeeing what would be, ſaid to him: Lord, ſhal we ſtrike with
the ſword? \V And one of them ſmote the ſeruant of the high Prieſt: and
cut off
%%% 2470
his right eare. \V But \Sc{Iesvs} anſwering, ſaid: Suffer ye thus
farre. And when he had touched his eare, he healed him. \V
And \Sc{Iesvs} ſaid to them that were come vnto him, the cheefe Prieſts,
and Magiſtrates of the Tẽple, & Ancients: As it were to a theefe are you
come forth with ſwordes and clubs? \V When I was daily with you in the
Temple, you did not lay handes vpon me, but this is your houre, and the
power of darkeneſſe.

\V And apprehending him, they led him to the high Prieſts houſe: but
Peter followed a farre off. \V And a fire being kindled in the middes of
the court, & they ſitting about it, Peter was in the middes of them. \V
Whom when a certaine wench ſaw ſitting at the light, and had beheld him,
ſhe ſaid: This fellow alſo was with him. \V But he denied him, ſaying:
Woman, I know him not. \V And after a while another man ſeeing him,
ſaid: And thou art of them. But Peter ſaid: O man I am not. \V And after
the ſpace as it were of one houre, a certaine other man affirmed,
ſaying: Verily this fellow alſo was with him: for he is alſo a
Galilæan. \V And Peter ſaid: Man I know not what thou ſayeſt. And
incontinent as he was yet ſpeaking, the cocke crew. \V And our Lord
turning looked on Peter. And Peter remembred the word of our Lord, as he
had ſaid: That before the cocke crow thou ſhalt thriſe denie me. \V And
Peter going forth a doores, wept bitterly.

\V And the men that held him, mocked him, beating him. \V And they did
blind-fold him, and ſmote his face. And they asked him ſaying:
Prophecie, who it is that ſmote thee? \V And blaſpheming many other
things they ſaid againſt him.

%%% o-2304
\V And when it was day, there aſſembled the Ancients of the people and
cheefe Prieſts and Scribes, and they brought him into their Councel,
ſaying: \V If thou be Chriſt tel vs. And he ſaid to them: If I tel you,
you wil not beleeue me: \V if alſo I aske, you wil not anſwer me, nor
dimiſſe me. \V But from henceforth the Sonne of man ſhal be ſitting on
the right hand of the power of God. \V And they al ſaid: Art thou then the
Sonne of God? Who ſaid: You ſay that I am. \V But they ſaid: What need
we teſtimonie any further? For our ſelues haue heard of his owne
mouth.


\stopChapter


\stopcomponent


%%% Local Variables:
%%% mode: TeX
%%% eval: (long-s-mode)
%%% eval: (set-input-method "TeX")
%%% fill-column: 72
%%% eval: (auto-fill-mode)
%%% coding: utf-8-unix
%%% End:
