%%%%%%%%%%%%%%%%%%%%%%%%%%%%%%%%%%%%%%%%%%%%%%%%%%%%%%%%%%%%%%%%%
%%%%
%%%% The (original) Douay Rheims Bible 
%%%%
%%%% New Testament
%%%% Luke
%%%% Chapter 01
%%%%
%%%%%%%%%%%%%%%%%%%%%%%%%%%%%%%%%%%%%%%%%%%%%%%%%%%%%%%%%%%%%%%%%




\startcomponent chapter-01


\project douay-rheims


%%% 2411
%%% o-2234
\startChapter[
  title={Chapter 1}
  ]

\Summary{The Annunciation and Conception, firſt of the Precurſour:
  26.~and ſix months after, of Chriſt alſo him ſelf. 39.~The Viſitation
  of our Ladie, where both the mothers doe Prophecie. 57.~The Natiuitie
  and Circumciſion of the Precurſour, where his father doth
  prophecie. 80.~The Precurſour is from a child an Eremite.}

Becauſe
\MNote{The firſt part of the Infancie both of the Precurſour and of
Chriſt himſelf.}
many haue gone about to compile a Narration of the things that haue been
accompliſhed among vs; \V according as they haue deliuered vnto vs, who
from the beginning them ſelues ſaw and were Miniſters of the Word; \V it
ſeemed good alſo vnto me
\LNote{Hauing diligently atteined}{Hereby
\MNote{Sacred Writers & holy Councels.}
we ſee, that though the Holy Ghoſt ruled the penne of holy Writers that
they might not erre, yet did they vſe humane meanes to ſearch out and
find the truth of the things they wrote of. Euen ſo doe Councels, and
the Preſident of them, Gods Vicar, diſcuſſe and examine al cauſes by
humane meanes, the aſſiſtance of the Holy Ghoſt concurring and directing
them into al truth, according to Chriſtes promiſe
\XRef{Io.~16,~13:}
as in the very firſt Councel of the Apoſtles them ſelues at Hieruſalem
is manifeſt.
\XRef{Act.~15.~7. and 28.}
Againe here we haue a familiar preface of the Authour as to his friend
or to euery godly Reader (ſignified by Theophilus) concerning the cauſe and
purpoſe & manner of his writing, and yet the very ſame is confeſſed
Scripture, with the whole booke folowing.
\MNote{The ſecond booke of the Machabees.}
Maruel not then if the Authour
of the ſecõd booke of the Machabees
\CNote{\XRef{2.~Mac.~2.}
&
\XRef{15.}}
vſe the like humane ſpeaches both at
the beginning and in the later end; neither doe thou therfore reiect the
booke for no Scripture, as our Heretikes doe; or not thinke him a ſacred
Writer.}
hauing, diligently attained to al things from the beginning, to write to
thee in order, Good
\CNote{\XRef{Act.~1,~1.}}
Theophilus, \V that thou maiſt know the veritie of
thoſe wordes wherof thou haſt been inſtructed.

\V There was in the daies of Herod the King of Iewrie, a certaine Prieſt
named Zacharie, of the
\CNote{\XRef{1.~Par.~24,~10.}}
courſe of Abia; & his wife of the daughter of
Aaron, and her name Elizabeth. \V And they were both
\LNote{Iuſt before God}{Againſt the Heretikes of this time, here it is
euident that
holy men be iuſt, not only by the eſtimation of men, but in deed and
before God.}
iuſt before God, walking
\LNote{In al the commandements}{Three
\MNote{True iuſtification by obſeruing the commandements.}
things to be noted directly againſt the Heretikes of our time. firſt,
that good men doe keepe al Gods commandements: which (they ſay) are
impoſſible to be kept. Againe, that men be iuſtified not by only
imputation of Chriſtes iuſtice, nor by faith alone, but by walking in
the commandements. Againe, that the keeping and doing of the
commandements is properly our iuſtification.}
in al the commandements
\LNote{Iuſtifications}{This
\MNote{Corrupt tranſlation of Heretikes.}
\TNote{\G{δικαιώματα}}
word is ſo vſual in the Scriptures (namely in the
\XRef{Pſal.~118})
to ſignifie the commandemẽts of God, becauſe the keeping of them is
iuſtificatiõ, and the Greeke is alwaies ſo fully correſpondent to the
ſame, that the Heretikes in this place (otherwiſe pretending to eſteeme
much of the Greeke) blush not to ſay, that they auoid this word of
purpoſe againſt the iuſtification of the Papiſts.
\CNote{\Cite{Beza in Annot. no. Teſt. 1556.}}
And therfore one vſeth Tullies word forſooth, in Latin
\L{conſtituta}:
and his ſcholers in their English Bibles ſay, \Emph{Ordinances}.}
and iuſtifications of our Lord without blame, \V and they had no ſonne:
for that Elizabeth was barren, and both were wel ſtriken in their
daies. \V And it came to paſſe, when he executed the prieſtly function
in the order of his courſe before God, \V according to the cuſtome of
the Prieſtlie functiõ, he went forth by lot
\CNote{\XRef{Exo.~3,~17.}}
to offer incenſe, entring into the Temple of our Lord; \V and
\CNote{\XRef{Leu.~16,~16.}}
al the multitude of the People was
\SNote{We ſee here that the Prieſt did his dutie within, the People in
the meane time praying without; and that the Prieſts fũctions did profit
them, though they neither heard nor ſaw his doings.}
praying without at the houre of the incenſe. \V And there appeared to
him an Angel of our
%%% o-2235
Lord, ſtanding on the right hand of the Altar of incenſe. \V And
Zacharie was troubled, ſeeing him; and feare fel vpon him. \V But the
Angel ſaid to him: Feare not Zacharie, for thy praier is heard; and thy
wife
%%% 2412
Elizabeth ſhal beare thee a ſonne and thou ſhalt cal his name Iohn: \V
and thou ſhalt haue
\LNote{Ioy and exultation}{This was fulfilled, not only when he was
borne, but now alſo
through the whole Church for euer, in ioyful celebrating of his
Natiuitie.}
ioy and exultation, and many ſhal reioyce in his natiuitie. \V For he
ſhal be great before our Lord;
\SNote{This abſtinence foretold and preſcribed by the Angel, ſheweth
that it is a worthie thing, and an act of religion in S.~Iohn, as it was
in the Nazarites.}
and wine and ſicer he ſhal not drinke; and he ſhal be repleniſhed with
the Holy Ghoſt euen from his mothers womb. \V And he ſhal
\CNote{\XRef{Mal.~4,~6.}}
conuert many
of the children of Iſrael to the Lord their God. \V And he ſhal goe
before him
\CNote{\XRef{Mt.~11,~14.}}
in the ſpirit and vertue of Elias; that he may conuert the
harts of the Fathers vnto the children, and the incredulous to the
wiſedom of the iuſt, to prepare vnto the Lord a perfect People. \V And
Zacharie ſaid to the Angel: Whereby ſhal I know this? for I am old; and
my wife is wel ſtriken in her daies. \V And the Angel anſwering ſaid to
him: I am Gabriel that aſſiſt before God; and am ſent to ſpeake to thee,
and to Euangelize theſe things to thee. \V And behold,
\SNote{Zacharie punished for doubting of the Angels word.}
thou ſhalt be dumme, and ſhalt not be able to ſpeake vntil the day
wherein theſe things ſhal be done; for becauſe thou haſt
\Fix{uot}{not}{obvious typo, fixed in other}
beleeued my wordes, which ſhal be fulfilled in their time. \V And the
People was expecting Zacharie; and they marueled that he made tariance
in the Temple. \V And comming forth he could not ſpeake to them, and
they knew that he had ſeen a viſion in the Temple. And he made ſignes to
them, and remained dumme. \V And it came to paſſe, after the daies of
his office were expired,
\LNote{He departed}{In
\MNote{The cõtinencie of Prieſts.}
the old Law (ſaith S.~Hierom) they that offered Hoſtes for the People,
were not only not in their houſes, but were purified for the time, being
ſeparated from their wiues, and they dranke neither wine nor any ſtrong
drinke, which are wont to prouoke concupiſcence. Much more the Prieſts
of the new Law that muſt alwayes offer Sacrifices, muſt alwaies be free
from matrimonie.
\Cite{Li.~1. c.~6.~19. adu. Iouin.} and
\Cite{ep.~50. c.~3.}
\Fix{Se}{See}{obvious typo, fixed in other}
\Cite{S.~Ambroſe in 1.~Tim.~3.}
\MNote{Miniſters not ſo perfect as the Prieſts of the old law.}
And therfore if there were any religion in Caluins Communion, they would
at the leaſt giue as much reuerence in this point, as they in the old
Law did to their Sacrifices, and to the loaues of propoſition,
\XRef{1.~Reg.~21.}}
he departed into his houſe. \V And after theſe daies Elizabeth his wife
conceaued; and hid herſelf fiue months, ſaying: \V For thus hath our
Lord done to me in the daies wherein he had reſpect to take away my
reproch among men.

\V And in the ſixt month, the Angel Gabriel was ſent of God into a citie
of Galilee, called Nazareth, \V
\CNote{\XRef{Mt.~1,~18.}}
to a Virgin eſpouſed to a man whoſe name
was Ioſeph, of the houſe of Dauid; and the Virgins name
was \Sc{Marie}. \V And the Angel being entred in, ſaid vnto her:
\MNote{The beginning of the \Sc{Ave Marie}, See the reſt
\XRef{v.~42.}}
\LNote{Haile ful of grace}{Holy
\MNote{Often ſaying of the \Emph{Ave Marie}.}
Church and al true Chriſtian men doe much and often vſe theſe wordes
brought frõ Heauen by the Archangel, as wel to the honour of Chriſt and
our B.~Ladie, as alſo for that they were the wordes of the firſt glad
tidings of Chriſts Incarnation & our Saluation by the ſame; and be the
very abridgement and ſumme of the whole Ghoſpel.
\CNote{\Cite{Liturg. S.~Iacobi & Chryſ.}}
In ſo much that the Greeke Church vſed it daily in the Maſſe.}
\Sc{Haile}
\LNote{Ful of grace}{Note
\MNote{Corrupt tranſlation of Heretikes.}
the excellent prerogatiues of our B.~Lady, and abhore thoſe Heretikes
which make her no better then other vulgar women, and therfore to take
from her fulnes of grace,
\TNote{\G{κεχαριτωμένη}}
they ſay here \Emph{Haile freely beloued},
contrarie to al ſignificatiõ of the Greeke word, which is at the
leaſt, \Emph{endued with grace},
\TNote{\G{ἐχαρίτωσε}}
as S.~Paul vſeth it
\XRef{Epheſ.~1.}
by S.~Chryſoſtoms interpretation: or rather \Emph{ful of grace}, as both
\CNote{\Cite{S.~Atha. de S.~Deip.}
\Cite{S.~Ephrem. in orat. de laud. B.~Virg.}}
Greeke and Latin Fathers haue alwaies here vnderſtood it, and the
Latines alſo read it, namely S.~Ambroſe thus,
\CNote{\Cite{Amb. in 1.~Lu. l.~2.}
\Cite{Hier. ep.~140. in exp. Pſ.~44.}}
\Emph{wel is she only
called ful of grace, who only obtained the grace, which no other woman
deſerued, to be 
\Fix{repelniſhed}{replenished}{obvious typo, fixed in other}
with the authour of grace.} And if they did
as wel know the nature of theſe kind of Greeke words, as they would
ſeeme very ſkilful, they might eaſily obſerue that they ſignifie fulnes,
as when them ſelues tranſlate the like word
\TNote{\G{ἡλκωμένος}}
\XRef{(Luc.~16.~20.)}
ful of ſores Beza,
\L{vlceroſus}.}
\Emph{ful of grace, our Lord is with thee: Bleſſed art thou among
women.} \V Who hauing heard, was troubled at his ſaying, and thought
what manner of ſalutation this ſhould be. \V And the Angel ſaid to her:
Feare not \Sc{Marie}, for thou haſt found grace with God. \V
\CNote{\XRef{Eſa.~7,~14.}}
Behold
%%% o-2236
thou
ſhalt conceaue in thy womb, and ſhalt beare a Sonne; and thou ſhalt cal
his name \Sc{Iesvs}. \V He ſhal be great, and ſhal be called the Sonne
of the moſt High, and our Lord God ſhal giue him the ſeat of Dauid his
Father: \V
\CNote{\XRef{Dan.~7,~14,~27.}}
and he ſhal reigne in the houſe of Iacob for euer, and of his
Kingdom there ſhal be no end. \V And \Sc{Marie} ſaid to the Angel:
\SNote{She doubted not of the thing as Zacharie, but enquired, of the
meanes.}
How ſhal this be done,
\LNote{I know not man}{Theſe
\MNote{Our B.~Lady vowed virginitie.}
words declare (ſaith S.~Auguſtine) that ſhe had now vowed virginitie to
God. For otherwiſe neither would ſhe ſay, \Emph{How shal this be done?}
nor haue added, \Emph{becauſe I know not man.} Yea if ſhe had ſaid only
the firſt words, \Emph{how shal this be done?} it is euident that ſhe
would not haue aſked ſuch a queſtiõ, how a woman ſhould beare a ſonne
promiſed her, if ſhe had maried meaning to haue carnal copulatiõ
\Cite{c.~4. de Virg.}
As if he ſhould ſay, If she might haue knowen a man and ſo haue had a
child, ſhe would neuer haue aſked, How ſhal this be done; but becauſe
that ordinarie way was excluded by her vow of virginitie, therfore ſhe
aſketh, How? And in aſking, How? She plainly declareth that ſhe might
not haue a child by knowing man, becauſe of her vow. See
\Cite{S.~Grego. Nyſſene de ſancta Chriſti Natiuitate.}}
becauſe I know not man? \V And the Angel anſwering, ſaid to her: The
Holy Ghoſt ſhal come vpon thee, and the power of the moſt High ſhal
ouerſhadow thee. And therfore alſo that which of thee shal be borne
Holy, shal be called the Sonne of God. \V And behold
\LNote{Elizabeth thy coſin}{By
\MNote{Chriſt came of both Tribes, Iude and Leui.}
this that Elizabeth and our Lady were coſins, the one of the Tribe of
Leui the other of Iuda, is gathered that Chriſt came of both Tribes,
Iuda and Leui, of the Kings and the Prieſts himſelf both a King & a
Prieſt, and the Anointed (to wit) by grace ſpiritually, as they were
with oile materially and corporally.
\Cite{Auguſt. li.~2. de Conſenſ. Euang. c.~1.}}
Elizabeth thy coſin, she alſo hath conceaued a Sonne in her old age; and
this month, is the ſixt to her that is called barren; \V becauſe there
shal not be impoſſible with God any word. \V And \Sc{Marie} ſaid,
\SNote{At this very moment when the B.~Virgin gaue conſent, ſhe
conceaued him perfect God and perfect man.}
\Sc{Behold} \Emph{the handmaid of our Lord, be it done to me according
to thy word.} And the Angel departed from her.

\V And \Sc{Marie} riſing vp in thoſe daies, went vnto the hil countrie
with ſpeed into a citie of Iuda. \V And she entred into the houſe of
%%% 2413
Zacharie, and ſaluted Elizabeth. \V And it came to paſſe; as Elizabeth
heard the ſalutation of \Sc{Marie}, the
\SNote{Iohn the Baptiſt being yet in his mothers wõb, reioyced &
acknowledged the preſẽce of Chriſt and his mother.}
infant did leap in her womb. And Elizabeth was repleniſhed with the Holy
Ghoſt: \V and ſhe cried out with a loud voice, and ſaid,
\LNote{Bleſſed art thou}{At
\MNote{The Bleſſed Virgin Marie.}
the very hearing of our Ladies voice, the infant and
\Fix{She}{ſhe}{likely typo, fixed in other}
were repleniſhed with the Holy Ghoſt, and ſhe ſang praiſes not only to
Chriſt, but for his ſake to our B.~Lady, calling her Bleſſed and her
fruit Bleſſed, as the Church doeth alſo by her words and example in
the \Sc{ave marie}.}
\Sc{Blessed} \Emph{art thou among women, and Bleſſed is the fruit of thy
womb.} \V And whence is this to me, that the
\LNote{Mother of my Lord}{Elizabeth
\MNote{Her excellẽcie.}
being an exceeding iuſt and Bleſſed woman, yet the worthines of Gods
mother doth ſo far excel her and al other women, as the great light the
litle ſtarres.
\Cite{Hiero. Praſ. in Sophon.}}
mother of my Lord doth come to me? \V For behold as the voice of thy
ſalutation ſounded in mine eares, the infant in my womb did leap for
ioy. \V And Bleſſed is ſhe that beleeued becauſe thoſe things ſhal be
accompliſhed that were ſpoken to her by our Lord. \V And \Sc{Marie}
ſaid:

\Sc{My ſovle}
\MNote{Magnificat at Euenſong.}
\Emph{doth magnifie our Lord:}

\V \Emph{And my ſpirit hath reioyced in God my Sauiour.}

\V \Emph{Becauſe he hath regarded the humilitie of his handmaid; for
behold from hence forth
\SNote{Haue the Proteſtãts had alwaies Generations to fulfil this prophecie?
or doe they cal her Bleſſed, that derogate what they can from her
graces, bleſſings and al her honour?}
al Generations
\LNote{Shal cal me Bleſſed}{This
\MNote{Her honour in al the world.}
Prophecie is fulfilled, when the Church keepeth her Feſtiual daies, &
when the Faithful in al Generations ſay the Aue Marie, and other holy
Anthems of our Lady. And therfore the Caluiniſtes are not among thoſe
Generations which cal our Lady Bleſſed.}
shal cal me Bleſſed.}

\V \Emph{Becauſe he that is mightie hath done great things to me: and
holy is his name.}

\V \Emph{And his mercie from Generation vnto Generations, to them that
feare him.}

\V \Emph{He hath shewed might in his arme: he hath diſperſed the proud
in the conceit of their hart.}

\V \Emph{He hath depoſed the mightie from their feat, & hath exalted the
humble.}

\V \Emph{The hungrie he hath filled with good things: and the rich he
hath ſent away emptie.}

\V \Emph{He hath receaued Iſrael his child, being mindful of his
mercie,}

\V \Emph{As he ſpake to our Fathers, to Abraham and his ſeed for euer.}

%%% o-2237
\V And \Sc{Marie} taried with her about three months: and ſhe returned
into her houſe.

\V And Elizabeths ful time was come to be deliuered; and ſhe bare a
Sonne. \V And her neighbours and kinsfolke heard that our Lord did
magnifie his mercie with her, and they did congratulate her. \V And it came
to paſſe; on the eight day they came to circumciſe the child, and they
called him by his fathers name Zacharie. \V And his mother anſwering,
ſaid: Not ſo, but he shal be called Iohn. \V And they ſaid to her, That
there is none in thy kinred that is called by this name. \V And they
made ſignes to his father, what he would haue him called. \V And
demanding a writing table, he wrote, ſaying:
\CNote{\XRef{Luc.~1,~13.}}
\LNote{Iohn is his name}{We
\MNote{Myſterie and ſignification in names.}
ſee that names are of ſignification and importance, God him ſelf
changing or giuing names in both Teſtaments; as, Abraham, Iſrael, Peter,
and the principal of al others,
\Fix{Ieſus;}{\Sc{Iesvs};}{obvious typo, fixed in other.}
and here Iohn, which ſignifieth, Gods grace or mercie, or, God wil haue
mercie. For he was the Precurſour and Prophet of the mercie and grace
that enſued by
\Fix{Chriſt Ieſus.}{\Sc{Christ Iesvs.}}{obvious typo, fixed in other.}
Note alſo that as then in Circumciſion, ſo now in Baptiſme (which
anſwereth therevnto) names are giuẽ. And as
\Fix{wſee}{we ſee}{obvious typo, fixed in other}
here & in al the old Teſtamẽt, great reſpect was had of names:
\MNote{What names to be giuen in Baptiſme.}
ſo we
muſt beware of ſtrange, profane, & ſecular names (now a daies too
common) & rather according to the
\CNote{\Cite{c.~de Bap. in fine.}}
Catechiſme of the holy Councel of Trent, take names of Saints and holy
men, that
\Fix{mey Put}{may put}{obvious typo, fixed in other}
vs in mind of their vertues.}
Iohn is his name. And they al marueled. \V And forthwith his mouth was
opened, and his tongue, and he ſpake bleſſing God. \V And feare came
vpon al their neighbours; and al theſe things were bruited ouer al the
hil-countrie of Iewrie: \V and al that had heard, laid them vp in their
hart, ſaying: What an one, trow ye, ſhal this child be? For the hand of
our Lord was with him. \V And Zacharie his father was repleniſhed with
the Holy Ghoſt; and he prophecied, ſaying:

\V \Sc{Blessed be ovr lord}
\MNote{Benedictus at Laudes.}
\Emph{God of Iſrael: becauſe he hath viſited and wrought the redemption
of his People:}

\V \Emph{And hath erected the horne of ſaluation to vs, in the houſe of
Dauid his ſeruant.}

\V \Emph{As he ſpake by the mouth of his holy Prophets, that are from
the beginning:}

\V \Emph{Saluation from our enemies, and from the hand of al that hate
vs:}

\V \Emph{To worke mercie with our Father; and to remember his holy
Teſtament,} 

\V
\CNote{\XRef{Gen.~22,~6.}}
\Emph{The oth which he ſware to Abraham our father, \V that he would
giue to vs.}

%%% 2414
\Emph{That without feare being deliuered from the hand of our enemies,
we may ſerue him,}

\V \Emph{In holines and
\LNote{Iuſtice before him}{Here
\MNote{True iuſtice, not imputatiue.}
alſo we ſee that we may haue true iuſtice, not only in the ſight of men,
or by the imputation of God, but indeed before him and in his ſight and
that the comming of Chriſt was to giue men ſuch iuſtice.}
iuſtice before him, al our daies.}

\V \Emph{And thou child, shalt be called the Prophet of the Higheſt: for
\CNote{\XRef{Mal.~3,~1.}}
thou shalt goe before the face of our Lord to prepare his waies.}

\V \Emph{To giue knowledge of ſaluation to his People, vnto remiſsion of
their ſinnes,}

\V \Emph{Through the bowels of the mercie of our God, in which
\LNote{The Orient}{Maruel
\MNote{The Heretikes controle both Greeke and Latin text.}
not if Heretikes countrole the old authentical tranſlation, as though it
differed from the Greeke: wheras here they make ſuch a doe to controle
not only al the Greek Interpreters of the old Teſtament, but alſo
S.~Luke him ſelf,
\CNote{\Cite{Beza.}}
for the word
\G{ἀνατολὴ},
as differing from the Hebrew.}
the
\CNote{\XRef{Zac.~3,~9.}
\XRef{6,~12.}}
Orient, from on high, hath viſited vs.}

\V
\CNote{\XRef{Mal.~4,~2.}}
\Emph{To illuminate them that ſit in darkenes, and in the shadow of
death: to direct our feet into the way of peace.}

\V And the child grew, and was ſtrengthned in ſpirit, and was
\SNote{Marke that he was a voluntarie Eremite, and choſe to be ſolitarie
from a child, til he was to preach to the People, in ſo much that
antiquitie counted him the firſt Eremite.
\Cite{Hiero. in vit. Pauli.}}
in the deſerts vntil the day of his manifeſtation to Iſrael.


\stopChapter


\stopcomponent


%%% Local Variables:
%%% mode: TeX
%%% eval: (long-s-mode)
%%% eval: (set-input-method "TeX")
%%% fill-column: 72
%%% eval: (auto-fill-mode)
%%% coding: utf-8-unix
%%% End:
