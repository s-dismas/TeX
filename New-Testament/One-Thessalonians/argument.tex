%%%%%%%%%%%%%%%%%%%%%%%%%%%%%%%%%%%%%%%%%%%%%%%%%%%%%%%%%%%%%%%%%
%%%%
%%%% The (original) Douay Rheims Bible 
%%%%
%%%% New Testament
%%%% Epistles
%%%% One Thessalonians
%%%% Argument
%%%%
%%%%%%%%%%%%%%%%%%%%%%%%%%%%%%%%%%%%%%%%%%%%%%%%%%%%%%%%%%%%%%%%%




\startcomponent argument


\project douay-rheims


%%% 2785
%%% o-2645
\startArgument[
  title={\Sc{The Argvment of the Firſt Epistle of S.~Pavl to the
  Thessalonians.}},
  marking={Argument of One Thessalonians}
  ]

How S.~Paul with Silas (or Syluanus) and Timothee according to a viſion
calling him out of Aſia in Macedonia, came to Philippi being the firſt
citie therof, we read
\XRef{Act.~16.}
And how againe from Philippi, after ſcourging and impriſoning there, he
came to Theſſalonica being the head citie of that countrie, we read
\XRef{Act.~17.}
where after 3.~weekes preaching, the Iewes ſtirred the citie againſt
them, and purſued them alſo to Beræa: ſo that Paul was conueied from
thence to Athens, where he expected the comming of Silas & Timothee from
the foreſaid Beræa in Macedonia, but receiued them (as we haue
\XRef{Act.~18.)}
at Corinth in Achaia.

Hauing therfore left the Theſſalonians in ſuch perſecution, and being
careful to know how they did in it, he was deſirous to returne vnto
them, as he ſignifieth in the
\XRef{2.~chapter of this Epiſtle v.~17.}
But (as he there addeth) \Emph{Satan hindred vs}. Therefore tarying
himſelf at Athens, he ſendeth Timothee vnto them. At whoſe returne
vnderſtanding their conſtencie, he is much comforted, as he declareth,
\XRef{c.~3.}
So then they are al three together at the writing of this Epiſtle, as
alſo we haue in the title of it: \Emph{Paul and Syluanus and Timothee to
the Church of the Theſſalonians}. And therfore it ſeemeth to haue been
written at Corinth, not at Athens: becauſe after the ſending of Timothee
to Theſſalonica, they met not at Athens againe, but at Corinth.

The firſt three chapters of it are, to confirme and comfort them againſt
the tentations of thoſe perſecutions. The other two are of exhortation,
to liue according to his precepts, namely in ſanctification of their
bodies, & not in fornication: to loue one another: about their freinds
departed, with the doctrine of the Reſurrection, and with continual
preparation to die: the laietie to obey, and the Clergie to be diligent
in euery point of their office.


\stopArgument


\stopcomponent


%%% Local Variables:
%%% mode: TeX
%%% eval: (long-s-mode)
%%% eval: (set-input-method "TeX")
%%% fill-column: 72
%%% eval: (auto-fill-mode)
%%% coding: utf-8-unix
%%% End:
