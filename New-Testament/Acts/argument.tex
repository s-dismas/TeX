%%%%%%%%%%%%%%%%%%%%%%%%%%%%%%%%%%%%%%%%%%%%%%%%%%%%%%%%%%%%%%%%%
%%%%
%%%% The (original) Douay Rheims Bible 
%%%%
%%%% New Testament
%%%% Acts
%%%% Argument
%%%%
%%%%%%%%%%%%%%%%%%%%%%%%%%%%%%%%%%%%%%%%%%%%%%%%%%%%%%%%%%%%%%%%%

%%% Latin checked KK




\startcomponent argument


\project douay-rheims


%%% 2544
%%% o-2386
\startArgument[
  title={\Sc{the argvment of the actes of the apostles.}},
  marking={Argument of Actes of the Apostles.}
  ]

The Ghoſpel hauing shewed, how the Iewes moſt impiouſly reiected Chriſt
(as alſo Moyſes and the Prophetes had foretold of them:) and therfore
deſerued to be reiected themſelues alſo of him: now followeth this booke
of \Emph{the Actes of the Apoſtles}
\CNote{\Cite{(Hier. in Catal.)}}
written by S.~Luke in Rome the fourth yeare of Nero, An.~Dom.~61.) and
sheweth, how notwithſtanding their deſerts, Chriſt of his mercy (as the
Prophets alſo had foretold of him) offered himſelfe vnto that vnworthy
people, yea after that they had Crucified him, ſending vnto them his
twelue Apoſtles to moue them to penance, and ſo by Baptiſme to make them
of his Church: and whiles al the Twelue
\Fix{vere}{were}{obvious typo, fixed in other}
ſo occupied about the Iewes; how of a perſecuting Iewe he made
\Fix{and}{an}{obvious typo, fixed in other}
extraordinarie Apoſtle (who was Saint Paul) and to auoid the ſcandal of
the Iewes (to whom only himſelfe likewiſe for the ſame cauſe had
preached) ſent him, and not any of his Twelue by and by, who were his
knowen Apoſtles, vnto the Gentils, who neuer afore had heard of Chriſt,
and were worshippers of many Gods, to moue them alſo (for, that likewiſe
the Prophets had foretold) to faith and penance, and ſo by Baptiſme to
make them of his Church: and how the incredulous Iewes euery where
reſiſted the ſame Apoſtle and his preaching to the Gentils, perſecuting
him and ſeeking his death, and neuer ceaſing vntil he fel into the
handes of the Gentils: that ſo (as not only
\CNote{\XRef{Act.~13,~46.}
\XRef{18,~6.}
\XRef{19,~9.}
\XRef{28,~18.}}
he euery where, but alſo
\CNote{\XRef{Eſa.~1.}}
the Prophets afore him, and
\CNote{\XRef{Mat.~21,~23.}}
Chriſt had foretold) the Ghoſpel might be taken
away from them, and giuen to the Gentils: euen from Hieruſalem
\CNote{\XRef{Luc.~13,~33.}}
(whoſe
reprobation alſo by name had been often foretold) the head-citie of the
Iewes, where it began, tranſlated to Rome the head-citie of the
Gentils. Al this wil be euident by the partes of the booke: which may be
theſe ſixe.

Firſt, how Chriſt Aſcending in the ſight of his Diſciples, promiſed vnto
them the Holy Ghoſt, foretelling that of  him they should receiue
ſtrength, and ſo begin his Church in Hieruſalem: and from thence dilate
it into al that Countrie, that is into al Iurie: yea and into Samaria
alſo, yea into al Nations of the Gentils, be they neuer ſo farre off.
\Emph{You ſhal receiue} (ſaith he) \Emph{the vertue of the Holy Ghoſt
comming vpon you: and you ſhal be witneſſes vnto me in Hieruſalem, and
in al Iurie, and Samaria, and euen to the vtmoſt of the earth.}
\XRef{Chap.~1.}

Secondly, the beginning of the Church in Hieruſalem, accordingly.
\XRef{Chap.~2.}

Thirdly, the propagation of it conſequently into al Iurie, and alſo to
Samaria.
\XRef{Cha.~8.}

%%% 2545
Fourthly, the propagation of it to the Gentils alſo.
\XRef{Chap.~10.}

Fifthly, the taking of it away from the obſtinate Iewes, and giuing of
it to the
%%% o-2387
Gentils, by the miniſterie of S.~Paul and S.~Barnabee.
\XRef{Chap.~13.}

Sixthly, of taking it away from Hieruſalem it ſelfe, the head-citie of
the Iewes, and ſending it (as it were) to Rome the head-citie of the
Gentils, and that, in their perſecuting of Paul ſo farre,
\CNote{\XRef{Act.~25,~11.}}
that he appealed to Cæſar, and ſo deliuering him after a ſort vnto the
Romanes: as they had
\CNote{\XRef{Luc.~23,~5.}}
before deliuered to them alſo Chriſt himſelfe. Wheras S.~Peters firſt
comming thither, was vpon another occaſion, as shal be ſaid anone. Of
which Romanes and Gentils therfore, the ſame S.~Paul being now come to
Rome (the
\XRef{laſt Chap. of the Actes})
foretelleth the obſtinate Iewes there, ſaying:
\CNote{Act.~28,~28.}
\L{Et ipſi audient}: You wil not heare, but, \Emph{they wil heare}. That
ſo the prediction of Chriſt aboue might be fulfilled: \Emph{And euen to
the vtmoſt of the earth}. And there doth S.~Luke end the booke, not
caring to tel ſo much as the fulfilling of that which our Lord had
foretold
\XRef{(Act.~27,~24.)}
to S.~Paul: \Emph{Thou muſt appeare before Cæſar.} Becauſe his purpoſe
was no more but to shew the new Hieruſalem of the Chriſtians, where
Chriſt would place the cheefe ſeat of his Church: as alſo indeed the
Fathers, and al other Catholikes haue in al Ages looked thither, when
they were in any great doubt: no leſſe then the Iewes to Hieruſalem,
\Fix{las}{as}{obvious typo, fixed in other}
they were appointed in the old Teſtament.
\XRef{Deut.~17,~8.}

And ſo this Booke doth shew the true Church, as plainely, as the Ghoſpel
doth shew the true Chriſt, vnto al that doe not wilfully shut their owne
eyes. To wit, this to be the true Church, which beginning viſibly at
Hieruſalem, was taken from the Iewes, and tranſlated to the Gentils (and
namely to Rome) continuing viſibly, and viſibly to continue hereafter
alſo,
\CNote{\XRef{Rom.~11,~25.}}
\Emph{Vntil the fulnes of the Gentils ſhal be come in}: that then alſo
\Emph{Al Iſrael may be ſaued}. And then is come the end of the
world. For ſo did Chriſt moſt plainely foretel vs:
\CNote{\XRef{Mat.~24,~14.}}
\Emph{This Ghoſpel of the Kingdõ ſhal be preached in the whole world,
for a teſtimonie to al Nations: and then ſhal come the conſummation.}
For the conuerſion of which Nations and accomplishing the fulnes of al
Gentils, the foreſaid Church Catholike, being mindful of her office,
\Emph{to be Chriſtes witnes euen to the vtmoſt of the earth}, doth at
this preſent (as alwaies) ſend preachers to conuert and make them alſo
Chriſtians: whereas the Proteſtants and
\CNote{\Cite{Tertul. de præf.}}
al other Heretikes doe nothing els but ſubuert ſuch as before were
Chriſtians.

And this being the Summe and ſcope of this Booke, thus to giue vs
hiſtorically a iuſt ſight of the fulfilling of the Prophets & Chriſtes
prediction about the Church: it is not to be maruelled at, why it
telleth not of S.~Peters comming to Rome: conſidering that his firſt
comming thither was not, as S.~Paules was, by the Iewes deliuerie of
him, working ſo to their owne reprobation, but vpon another occaſion, to
wit, to confound Simon Magus.
\Cite{Euſ. Hiſt. li.~2. c.~12.~13.}
For who alſo ſeeth not, that it maketh no mention of his preaching to
any Gentils at al, thoſe few only
\XRef{Act.~10.}
excepted, who were the firſt, and
therfore (leſt the Gentils should ſeeme leſſe cared for of God, then the
Iewes) Peter being the Head of al, was elected of God, to incorporate
them into the Church, as before he had done the Iewes.
\CNote{\XRef{Act.~15,~7.}}
\Emph{God} (ſaith he) \Emph{among vs choſe, that by my mouth the Gentils
ſhould heare the word of the Ghoſpel, and beleeue.} And S.~Iames
thereupon:
\CNote{\XRef{Act.~15,~14.}}
\Emph{Simon hath told how God firſt viſited to take of the Gentils a
people to his name.} But otherwiſe (I ſay) here is no mention of Peters
preaching to any Gentils: no nor of the other eleuen Apoſtles. Wil any
man therfore inferre, that neither Peter, not the other Eleuen preached
to any Nation or
%%% 2546
citie of the Gentils? No, the meaning of the Holy Ghoſt was not to write
al the Actes of al the Apoſtles, no nor the preaching of Peter and his,
to the Gentils, but only to the Iewes: therby to ſet out vnto the world,
the great mercy of Chriſt toward thoſe vnworthy
%%% o-2388
Iewes, and conſequently their moſt worthy reprobation for contemning
ſuch grace and mercy. As alſo on the other ſide to shew, how readily the
Gentils in ſo many Nations, were conuerted by one Apoſtle only, who
\Emph{From Hieruſalem euen to Illyricum repleniſhed the Ghoſpel of
Chriſt.} And this parting of the worke ſo made by S.~Peter with the reſt
doth S.~Paul himſelfe touch:
\CNote{\XRef{Gal.~2,~9.}}
\Emph{That we vnto the Gentils, and they vnto the Circumciſion}. Neuer
theleſſe before his comming to Rome, not only was the Church come to
Rome (as is euident
\XRef{Act. the laſt chap.})
there planted by Saint Peter and others (as likewiſe by Saint Peter it
was planted in the firſt Gentils, before that S.~Paul began the taking
of it away from the multitude of the Iewes, and the tranſlating of it to
the multitude of the Gentils) but alſo ſo notable was the ſame Church of
Rome, that S.~Paul writing his Epiſtle to the Romanes, before he came
thither, ſaith:
\CNote{\XRef{Rom.~1,~8.}}
\Emph{Your faith is renowned in the whole world.} And therfore they with
the reſt of the Gentils, be that Nation whereof Chriſt told the Iewes,
ſaying:
\CNote{\XRef{Mat.~21,~43.}}
\Emph{The Kingdom of God ſhal be taken away from you, and ſhal be giuen
to a Nation yealding the fruits thereof.}


\stopArgument


\stopcomponent


%%% Local Variables:
%%% mode: TeX
%%% eval: (long-s-mode)
%%% eval: (set-input-method "TeX")
%%% fill-column: 72
%%% eval: (auto-fill-mode)
%%% coding: utf-8-unix
%%% End:
