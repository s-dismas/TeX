%%%%%%%%%%%%%%%%%%%%%%%%%%%%%%%%%%%%%%%%%%%%%%%%%%%%%%%%%%%%%%%%%
%%%%
%%%% The (original) Douay Rheims Bible 
%%%%
%%%% New Testament
%%%% Acts
%%%% Chapter 20
%%%%
%%%%%%%%%%%%%%%%%%%%%%%%%%%%%%%%%%%%%%%%%%%%%%%%%%%%%%%%%%%%%%%%%




\startcomponent chapter-20


\project douay-rheims


%%% 2603
%%% o-2450
\startChapter[
  title={Chapter 20}
  ]

\Summary{Hauing viſited the Churches of Macedonia and Achaia (as he
purpoſed
\XRef{Act.~19.})
and now about to ſaile from Corinth toward Hieruſalem, becauſe of the
Iewes lying in wait for him, he is conſtrained to returne into
Macedonia. 6.~And ſo at Philippis taking boat, commeth to Troas, where
vpon the Sunday, with a ſermon, and a miracle, he greatly confirmeth
that Church. 13.~Thence comming to Miletum, 17.~he ſendeth to Epheſus
for the Clergie of thoſe partes: to whom he maketh a Paſtoral ſermon,
committing vnto their charge the flocke begun by him there, and now
like to be ſeen of him no more, conſidering the troubles that by
reuelation he looketh for at Hieruſalem.}

%%% o-2451
And after that the tumult was ceaſed, Paul calling the Diſciples, and
exhorting them, tooke his leaue, and ſet forward to goe into
Macedonia. \V And when he had walked through thoſe parts, & had exhorted
thẽ with much ſpeach, he came to Greece: \V where when he had ſpent
three moneths, the Iewes laid wait for him as he was about to ſaile into
Syria: and he had counſel to returne
%%% 2604
through Macedonia. \V And there accompanied him Soſipater of Pyrrhus, of
Berœa: and of Theſſalonians, Ariſtarchus, and Secundus: and Caius of
Derbe, and Timothee: and of Aſia, Tychicus and Trophimus. \V Theſe going
before, ſtaied for vs at Troas. \V But we ſailed after the daies of
Azymes from Philippi, and came to them vnto Troas in fiue daies where we
abode ſeuen daies.

\V And in the firſt of the Sabboth when we were aſſembled to
\SNote{S.~Paul did here breake bread on the Sunday as it is broken in
the Sacramẽt of the body of Chriſt and had both before & after the
celebrating of the Sacrament a ſermon to the people.
\Cite{Aug. ep.~86. ad Caſulanũs.}
\Cite{Vener. Beda, in 20.~Act.}}
breake bread, Paul diſputed with them, being to depart on the morow: and
he continued the ſermon vntil mid-night. \V And there were a great
number of lampes in the vpper chamber where we were aſſembled. \V And a
certaine yong man named Eutychus, ſitting vpon the window, whereas he was
oppreſſed with heauy ſleep: (Paul diſputing long) driuen by ſleep, fel
from the third loft downe, and was taken vp dead. \V To whom when Paul
was gone downe, he lay vpon him: and embracing him he ſaid: Be not
troubled, for his ſoule is in him. \V And going vp and breaking bread
and taſting, and hauing talked ſufficiently to them vntil day light, ſo
he departed. \V And they brought the youth aliue, & were not a litle
comforted.

\V But we going vp into the ſhip, ſailed to Aſſon, from thence meaning
to receiue Paul; for ſo he had ordained, himſelf purpoſing to iourney by
land. \V And when he had found vs in Aſſon, taking him with vs we came
to Mitylene. \V And ſailing thence, the day following we came ouer
againſt Chios: and the other day we arriued at Samos: and the day
following we came to Miletum. \V For Paul had purpoſed to ſaile leauing
Epheſus, leſt any ſtay ſhould be made him in Aſia. For he haſtned, if it
were poſſible for him, to keep the day of
\LNote{Pentecoſt.}{Though
\MNote{The Chriſtian Pentecoſt.}
the Apoſtles might deſire to come to the Iewes Feſtiuities, by reaſõ of
the general cõcourſe of people to the ſame, the better to deale for
their ſaluatiõ & to ſpread the Ghoſpel of Chriſt, yet it is like that
they now kept ſolemnly the Chriſtiã Pentecoſt or whitſuntide, for
memorie of the Holy Ghoſt, and that S.~Paul went to that Feaſt of the
Chriſtians rather then the other of the Iewes. And Ven.~Bede ſaith here:
\Emph{The Apoſtle maketh haſt to keepe the fiftieth day, that is, of
remiſsion and of the holy Ghoſt.}
\MNote{Sunday.}
For, that the Chriſtians already kept the eight day, that is, the Sunday
or our Lordes day, & had altered already the ordinarie Sabboth into the
ſame, it is plaine by the Scriptures
\XRef{(1.~Cor.~16,~2.}
\XRef{Apoc.~1,~10.}
& by antiquitie,
\Cite{Iuſtin. Mart. Apolog. ad Anton. Pium in fine.})
And it is as like that they changed the Iewes Paſche and Pentecoſt as
that; ſpecially when it is euident that
\CNote{\Cite{Aug. ep.~118. c.~1.}}
theſe Feſtiuities be kept by Apoſtolike tradition, and approued by the
vſe of al ancient Churches and Councels.}
Pentecoſt at Hieruſalem.

\V And ſending from Miletum to Epheſus, he called the
\SNote{That is, \Emph{Prieſts} as
\XRef{Act.~15,~4.}
See the
%%% marginal
\XRef{Annot. there.}}
\TNote{\G{πρεσβυτέρους}}
Ancients of the Church. \V Who being come to him, and
%%% o-2452
aſſembled together, he ſaid to them: You know
\CNote{\XRef{Act.~19,~2.}}
from the firſt day that I entred into Aſia, in what manner I haue been
with you al the time, \V ſeruing our Lord with al humilitie and teares,
and tentations that did chance to me by the conſpiracies of the
Iewes: \V How I haue withdrawen nothing that was profitable, but that I
preached it to you, and taught you openly and from houſe to houſe, \V
teſtifying to Iewes and Gentils
\SNote{Apoſtolike preaching cõmendeth not faith only, but penance alſo
to the people.}
penance toward God, and faith in our Lord \Sc{Iesvs Chriſt}. \V And now
behold, being bound by the Spirit, I goe to Hieruſalem; not knowing what
things ſhal befal me in it, \V but that the Holy Ghoſt through out al
cities doth proteſt to me ſaying: that bands and tribulations abide me
at Hieruſalem. \V But I feare none of theſe things, neither doe I make my
life more pretious then my ſelf, ſo that I may conſummat my courſe &
miniſterie which I receiued of our Lord \Sc{Iesvs}, to teſtifie the
Ghoſpel of the grace of God. \V And now behold I doe know, that you ſhal
no more ſee my face al you, through whom I haue paſſed preaching the
Kingdom of God. \V Wherefore I take you to witneſſe this preſent day that
I am cleere from the bloud of al. \V For I haue not ſpared to declare
vnto you al the counſel of God. \V Take heed to your ſelues and to the
whole flocke wherin the
\SNote{Biſhops or Prieſts (for thẽ theſe names were ſometimes vſed
indifferẽtly) gouernours of the Church of God, & placed in that roome &
high functiõ by the Holy Ghoſt.}
Holy Ghoſt
%%% 2605
hath placed you Biſhops, to rule the Church of God which he hath
purchaſed with his owne bloud. \V I know that after my departure there
wil
\LNote{Rauening wolues.}{The
\MNote{Rauening wolues are the Heretikes of al Ages.}
Gouernours of the Church are foretold of the great danger that ſhould
fal to the people by wolues, that is to ſay, by Heretikes, whoſe cruelty
toward the Catholikes is noted by this terme. They be knowen by the
forſaking the vnitie of the Church wherof they were before, by going out
and drawing many Diſciples after them, and by their peruerſe
doctrine. Such wolues came afterward indeed in diuers Ages: Arius,
Macedonius, Neſtorius, Eutyches, Luther, Caluin, great bloud-ſucking
wolues, & waſters of the flocke of Chriſt.}
rauening wolues enter in among you, not ſparing the flocke. \V And out
of your owne ſelues ſhal ariſe men ſpeaking peruerſe things, to draw
away Diſciples after themſelues. \V For the which cauſe be vigilant,
keeping in memorie that for three yeares night & day I ceaſed not with
teares to admoniſh euery one of you. \V And now I commend you to God and
to the word of his grace, who is able to edifie, and to giue
inheritance in al the ſanctified. \V No mans ſiluer and gold or garment
haue I coueted. \V Your ſelues know that for ſuch things as were needful
for me and them that are with me, theſe hands haue miniſtred. \V I haue
ſhewed you al things, that ſo labouring, you muſt receiue the weake, and
remember the word of our Lord \Sc{Iesvs}, becauſe he ſaid:
\LNote{More bleſſed to giue.}{Among
\MNote{Chriſts ſpeaches not writtẽ in the Ghoſpel.}
many other infinit goodly things and ſpeaches which Chriſt ſpake and be
not written in the Ghoſpels, this ſentence is one: which S.~Paul heard
of ſome of the Apoſtles daily conuerſant with him, or els learned of
Chriſt himſelf, or of the Holy Ghoſt. And it ſignifieth, that whereas
the world commonly counteth him happie that receiueth any benefit, as
almes either temporal or ſpiritual,
\MNote{Great almes-men bleſſed.}
yet indeed he that giueth or
beſtoweth, is more happie. Which if the world did wel conſider, men
would giue almes faſter then they doe, if it were but for their owne
benefit.}
It is a more bleſſed thing to giue rather then to take.

\V And when he had ſaid theſe things, falling on his knees he praied
with al them. \V And there was great weeping
%%% o-2453
made of al; and falling vpon the necke of Paul, they kiſſed him, \V
being ſorie moſt of al for the word which he had ſaid, that they ſhould
ſee his face no more. And they brought him going vnto the ſhip.


\stopChapter


\stopcomponent


%%% Local Variables:
%%% mode: TeX
%%% eval: (long-s-mode)
%%% eval: (set-input-method "TeX")
%%% fill-column: 72
%%% eval: (auto-fill-mode)
%%% coding: utf-8-unix
%%% End:

