%%%%%%%%%%%%%%%%%%%%%%%%%%%%%%%%%%%%%%%%%%%%%%%%%%%%%%%%%%%%%%%%%
%%%%
%%%% The (original) Douay Rheims Bible 
%%%%
%%%% New Testament
%%%% Acts
%%%% Chapter 23
%%%%
%%%%%%%%%%%%%%%%%%%%%%%%%%%%%%%%%%%%%%%%%%%%%%%%%%%%%%%%%%%%%%%%%




\startcomponent chapter-23


\project douay-rheims


%%% 2610
%%% o-2458
\startChapter[
  title={Chapter 23}
  ]

\Summary{As the people in the tumult, ſo alſo the very cheefe of the
  Iewes in their Councel shew themſelues obſtinate, and wilful
  perſecutours of the truth in S.~Pauls perſon. Whoſe behauiour towardes
  them is ful of conſtancie, modeſtie, and wiſedom. 11.~(Chriſt alſo by
  a viſion encouraging him & foretelling that he shal to Rome.) 12.~Yea
  they conſpire with 40.~men to kil him traiterouſly. 16.~But the matter
  being detected, the Romane Tribune conueigheth him ſtrongly to
  Cæſarea.}

And Paul looking vpon the Councel, ſaid: Men Brethren, I with al good
conſcience haue conuerſed before God, vntil this preſent day. \V And the
high Prieſt Ananias commanded them that ſtood by him, to ſmite him on the
mouth. \V Then Paul ſaid to him:
\SNote{He ſaid not this through pertubation of mind, or of a paſſion,
but by way of prophecie, that this figuratiue high prieſthood thẽ trimmed
like a
\Fix{wihted}{whited}{obvious typo, fixed in other}
wal, was to be deſtroied; whereas now the true prieſthood of Chriſt was
cõe.
\Cite{Beda in hunc lo.}}
God ſhal ſtrike thee, thou whited wal. And thou ſitting iudgeſt me
according to the law, and contrarie to law doeſt thou command me to be
ſmitten? \V And they that ſtood by, ſaid: Doeſt thou reuile the high
Prieſt of God? \V And Paul ſaid:
\LNote{I knew not.}{\Emph{Our
\MNote{The honour of Prieſthood.}
Lord} (ſaith 
\CNote{\Cite{Cypr. ep.~65.~69. nu.~2.}}
S.~Cyprian) \Emph{in the Ghoſpel, when it was ſaid to him: Anſwereſt thou
the high Prieſt ſo? teaching that the honour of Prieſthood muſt be kept
ſaid nothing to the high Prieſt, but only purging his innocencie, ſaid:
If I haue ſpoken euil, beare witnes of euil; but if wel, why ſmiteſt
thou me? Alſo the Bleſſed Apoſtle when it was ſaid to him: Doeſt thou
aſſaile the high Prieſt ſo with il wordes? ſpake not any thing
contumaliouſly againſt the Prieſt, whereas he might haue put forth
himſelf ſtoutly againſt them which had both crucified our Lord, and
which had now alſo loſt their God and Chriſt, Temple and Prieſthood. But
though in falſe & ſpoiled Prieſts, yet conſidering the very bare shadow
of the name of Prieſts, he ſaid: I knew not, Brethren, that he was high
Prieſt.} By which wordes of the Apoſtle, either it may be thought he
knew not indeed that he was in that function, becauſe he had not been of
long time in thoſe partes; or els that he ſo ſaid in reſpect of the
abrogation of the high Prieſthood of the Iewes, wherby he knew this man
not to be truely any Prieſt: as alſo becauſe at this time they came not
orderly to it by ſucceſsion of Aaron and Law of Moyſes, but by the Roman
Emperours fauour
\CNote{See
\XRef{Annot. Io. c.~11,~51.}}
as is ſaid before: though (as it is lawful in ſuch a caſe) the leſſe to
irritate them, he frameth his ſpeach ſo as they might not take occaſion
of further accuſation againſt him.}
I knew not, Brethren, that he is the high Prieſt. For it is written:
\CNote{\XRef{Exo.~22,~28.}}
\Emph{The Prince of thy people thou shalt not mis-ſpeake.} \V And Paul
%%% o-2459
knowing that the one part was of Sadducees, and the other of Phariſees,
\SNote{Such prudent euaſions from danger are lawful. Which
S.~Chryſoſtome calleth (ſpecially in this Apoſtle) the wiſdom of the
ſerpent; as otherwiſe in his teaching and preaching & patience he vſed
the ſimplicitie of a doue.}
he cried out in the Councel: Men Brethren,
\CNote{\XRef{Phil.~3,~5.}}
I am a Phariſee, the ſonne of Phariſees: of the hope and reſurrection of
the dead I am iudged. \V And when he had ſaid theſe things, there roſe
diſſenſion between the Phariſees and Sadducees; and the multitude was
deuided. \V For the
\LNote{The Sadducees.}{This
\MNote{The Sadducees (as it ſeemeth) denied praier for the dead.}
was the worſt Hereſie among the Iewes, denying that there be any Angels,
or ſpirits, the Reſurrection alſo of the bodies: & conſequently (as it
may very wel be gathered by the booke of the 
\CNote{Mac. li.~2. c.~12,~43.}
Machabees) they denied
praier for the dead. For to offer or pray for the dead, & to thinke
rightly & religiouſly of the Reſurrection, are made there ſequels one of
another. Of this ſect of Sadducees was (as Euſebius writeth
\Cite{li.~2. c.~12. Ec. Hiſt.})
this Ananias the high Prieſt, that cauſed Paul to be ſmitten. For their
Prieſthood had now no more the protection of God to preſerue it in truth
and right iudgement, the Chriſtian Prieſthood being then eſtabliſhed.}
Sadducees ſay
\CNote{\XRef{Mt.~22,~23.}}
there is no reſurrection, nor Angel, nor ſpirit: but the Phariſees
confeſſe both. \V And there was made a great crie. And certaine of the
Phariſees riſing vp, ſtroue ſaying: We find no euil in this man. What
if a ſpirit hath ſpoken to him, or an Angel? \V And when there was riſen
great diſſenſion, the Tribune fearing leſt Paul ſhould be torne in
peeces by them, commanded the ſouldiars to goe downe, and to take him
out of the middes of them, and to bring him into the caſtel. \V And the
night following our Lord ſtanding by him, ſaid: Be conſtant; for as thou
haſt teſtified of me in Hieruſalem, ſo
\SNote{Though God who could not lie, had promiſed Paul that he should
goe to Rome; yet the Apoſtle omitted not humane meanes to defend himſelf
from his enemies & otherwiſe. Neither ſaid he as the Heretikes called
Predeſtinates, Let thẽ doe what they wil, they cã not hurt me, for I am
predeſtinate to goe to Rome. See his doings and ſayings to ſaue himſelf,
in
\XRef{the chap. following.}}
muſt thou teſtifie at Rome alſo.

\V And when the day was come, certaine of the Iewes gathered themſelues
together, &
\LNote{Vowed themſelues.}{Such
\MNote{Vnlawful othes & vowes muſt not be kept.}
vowes, othes, or execrations as this, bind no man before God, yea they
muſt in no wiſe be obſerued. It is a great offence either to vow
voluntarily, or to take any ſuch thing vpon a man, for feare or by
commandement. For example, if thou haue rashly by promiſe, or othe,
appointed to be reuenged vpon any man, thou bindeſt not thy ſelf thereby,
neither muſt thou keepe thy promiſe. If thou be put to an othe to
accuſe Catholikes for ſeruing God as they ought to doe, or to vtter any
innocent man to God's enemies and his, thou oughteſt firſt to refuſe
ſuch vnlawful othes: but if thou haue not conſtancie and courage ſo to
doe, yet know thou that ſuch othes bind not at al in conſcience & Law of
God, but may and muſt be broken vnder paine of damnation. For to make or
take ſuch vowes or othes is one ſinne, and to keep them, is another
farre greater: as
\CNote{\XRef{Mt.~14,~9.}}
when Herode, to keep his othe, killed Iohn Baptiſt. And ſuch vowes and
othes to God as theſe, are vnlawful & muſt be broken: and not the vowes
of Chaſtitie and Religion, as our new Miniſters teach by their wordes
and workes.}
vowed themſelues, ſaying: that they would neither eate nor drinke til
they killed Paul. \V And they were more then fourtie men that had made
this conſpiracie: \V who came to the cheefe Prieſts and the Ancients,
and ſaid: By execration we haue vowed our ſelues, that we wil eate
nothing, til we kil Paul. \V Now therfore giue you knowledge to the
Tribune with the Councel, that he bring him forth to you, as if you
meant to know ſome more certaintie touching him. But we, before he come
neere, are ready for to kil him. \V Which when Paules ſiſters ſonne had
heard, of their lying in wait, he came and entred into the caſtel and
told Paul. \V And Paul calling to him one of the Centurions, ſaid: Bring
this yong man to the Tribune,
%%% 2611
for he hath ſome thing to tel him. \V
\SNote{See the courteſie & equitie of Heathen officers toward their
priſoners, to ſaue them from al iniurie & villanie.}
And he taking him, brought him to the Tribune, and ſaid: The priſoner
Paul deſired me to bring this yong man vnto thee, hauing ſome thing to ſay to
thee. \V And the Tribune taking him by the hand, went aſide with him
apart, and asked him: What is it that thou haſt to tel me? \V And he
ſaid: The Iewes haue agreed to deſire thee, that to morow thou wilt
bring forth Paul into the Councel, as though they meant to inquire ſome
more certaintie touching him. \V But doe not thou credit them; for there
lie in wait for him more then fourtie men
%%% o-2460
of them, which haue vowed neither to eate nor to drinke, til they kil
him: and they are now ready, expecting thy promiſe. \V The Tribune
therfore dimiſſed the yong man, commanding that he ſhould ſpeake to no
man that he had notified theſe things vnto him. \V And calling two
Centurions, he ſaid to them: Make ready two hundred ſouldiars, to goe as
farre as Cæſarea, and ſeuentie horſe-men, and lances two hundred, from
the third houre of the night: \V and prepare beaſts: that ſetting Paul
on, they might bring him ſafe to Felix the Preſident. (\V For he feared
leſt perhaps the Iewes might take him away, and kil him, and himſelf
afterward ſhould ſuſtaine reproch, as though he would haue taken
money) \V writing a letter conteining thus much:

\Emph{Claudius Lyſias to the moſt excellent Preſident Felix,
greeting.} \V This man being apprehended of the Iewes, and ready to be
killed of them, I comming in with the band deliuered him, vnderſtanding
that he is a Roman: \V and meaning to know the cauſe that they obiected
vnto him, I brought him downe into their Councel. \V Whom I found to be
accuſed concerning queſtions of their law: but hauing no crime worthie
of death or of bands. \V And when it was told me of ambuſhments that
they had prepared againſt him, I ſent him to thee, ſignifying alſo to
the accuſers, to ſpeake before thee. Fare-wel. \V And the ſouldiars
according as it was commanded them, taking Paul, brought him by night to
Antipatris. \V And the next day ſending away the horſe-men to goe with
him, they returned to the caſtel. \V Who when they were come to Cæſarea,
and had deliuered the letter to the Preſident, they did ſet Paul alſo
before him. \V And when he had read, and had asked of what prouince he
was: and vnderſtanding that of Cilicia: \V I wil heare
\Fix{the,}{thee,}{possible typo, same in other}
ſaid he, when thy accuſers are come. And he commanded him to be kept in
Herods palace.


\stopChapter


\stopcomponent


%%% Local Variables:
%%% mode: TeX
%%% eval: (long-s-mode)
%%% eval: (set-input-method "TeX")
%%% fill-column: 72
%%% eval: (auto-fill-mode)
%%% coding: utf-8-unix
%%% End:

