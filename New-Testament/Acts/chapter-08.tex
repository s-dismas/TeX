%%%%%%%%%%%%%%%%%%%%%%%%%%%%%%%%%%%%%%%%%%%%%%%%%%%%%%%%%%%%%%%%%
%%%%
%%%% The (original) Douay Rheims Bible 
%%%%
%%%% New Testament
%%%% Acts
%%%% Chapter 08
%%%%
%%%%%%%%%%%%%%%%%%%%%%%%%%%%%%%%%%%%%%%%%%%%%%%%%%%%%%%%%%%%%%%%%




\startcomponent chapter-08


\project douay-rheims


%%% 2566
%%% o-2410
\startChapter[
  title={Chapter 8}
  ]

\Summary{So
\MNote{The 3.~part.

The propagation of the Church from Hieruſalẽ into al
  Iewrie, and Samaria.}
farre is perſecution from preuailing againſt the Church, that by it the
  Church groweth from Hieruſalem into al Iewrie and Samaria. 5.~The
  ſecond of the Deacons, Philip, cõuerteth with his miracles the citie
  it ſelf of Samaria, and baptizeth them, euen Simon Magus alſo himſelf
  among the reſt. 14.~But the Apoſtles Peter and Iohn are the Miniſters
  to giue them the Holy Ghoſt. 18.~Which miniſterie Simon Magus would
  buy of them. 26.~The ſame Philip being ſent of an Angel to a great man
  of Æthiopia, who came a Pilgrimage to Hieruſalem, firſt catechizeth
  him, 36.~and then (he profeſsing his faith and deſiring Baptiſme) doth
  alſo baptize him.}

%%% 2567
And the ſame day there was made a great perſecution in the Church, which
was at Hieruſalem, and al were diſperſed through the countries of Iewrie
and Samaria, ſauing the Apoſtles. \V And
\LNote{Deuout men.}{As
\MNote{S.~Steuens relikes.}
here great deuotion was vſed in burying his body, ſo afterward at the
Inuention & Tranſlation thereof. And the miracles wrought by the ſame,
and at euery litle memorie of the ſame, were infinite: as S.~Auguſtine
witneſſeth,
\Cite{li.~21. de Ciuit Dei. c.~8.}
&
\Cite{Sermon. de S.~Steph. to.~10.}}
deuout men
\TNote{\L{Curaverunt} \G{συνεκόμισαν}.}
tooke order for Steuens funeral, and made great mourning vpon him. \V But
Saul
\CNote{\XRef{Act.~22,~4.}}
waſted the Church: entring in from houſe to houſe, and drawing men and
women, deliuered them into priſon.

\V They therfore that were diſperſed, paſſed through,
\SNote{This perſecutiõ wrought much good, being an occaſion that the
diſperſed preached Chriſt in diuers Countries where they came.}
euangelizing the word.

\V And Philippe deſcending into the citie of Samaria,
preached \Sc{Christ} vnto them, \V and the multitudes were attent to
thoſe things which were ſaid of Philippe, with one accord hearing, and
ſeing the ſignes that he did. \V For many of them that had vncleane
Spirits, crying with a loud voice, went out. And many ſicke of the
palſey and lame were cured. \V There was made therfore great ioy in
that citie. \V And a certaine man named Simon, who before had been in
that citie a Magician, ſeducing the Nation of Samaria, ſaying himſelf to
be ſome great one: \V vnto whom al harkened from the leaſt to the
greateſt, ſaying: This man is the power of God, that is called great. \V
And they were attent vpon him, becauſe a long time he had bewitched them
with his magical practiſes. \V But when they had beleeued Philippe
euangelizing of the kingdom of God, and of the name of \Sc{Iesvs
Christ}, they were baptized, men and women. \V Then Simon alſo himſelf
beleeued: & being baptized, he cleaued to Philippe. Seing alſo ſignes
and very great miracles to be done, he was aſtoniſhed with admiration.

%%% o-2411
\V And when the Apoſtles who were in Hieruſalem, had heard that Samaria
had receiued the word of God, they
\LNote{Sent Peter.}{Some
\MNote{That Peter was ſent, is no reaſon againſt his Primacie.}
Proteſtants vſe this place to proue S.~Peter not to be Head of the
Apoſtles, becauſe he and S.~Iohn were ſent by the Twelue. By which
reaſon they might as wel conclud that he was not equal to the reſt. For
cõmonly the Maiſter ſendeth the man, & the Superiour the inferiour, when
the word of Sending is exactly vſed. But it is not alwayes ſo taken in
the Scriptures: for then could not the Sonne be ſent by the Father, nor
the Holy Ghoſt from the Father & the Sonne; nor otherwiſe in cõmon vſe
of the world ſeeing the inferiour or equal may intreate his freind or
Superiour to doe his buſines for him. And ſpecially a body Politike or a
Corporatiõ may be electiõ or otherwiſe chooſe their Head & ſend him. So
may the Citizẽs ſend their Maior to the Prince or Parlamẽt, though he be
the Head of the citie, becauſe he may be more fit to doe their
buſines. Alſo the Superiour or equal may be ſent by his owne conſent or
deſire. Laſtly, the College of the Apoſtles compriſing Peter with the
reſt (as euery ſuch Body implieth both the Head and the members) was greater
then Peter their Head alone, as the Prince and Parlament is greater then
the Prince alone. And ſo Peter might be ſent as by authoritie of the
whole Colledge, not withſtanding he were the Head of the ſame.}
ſent vnto them
\SNote{\L{Saepe ſibi ſociũ Petrus facit eſſe Ioannem: Eccleſia quia virgo
placet.} Peter often maketh Iohn his cõpanion, becauſe the Church loueth
a virgin.
\Cite{Arator apud Bedlam in Act.}}
Peter and Iohn. \V Who when they were come, prayed for them, that they
might receiue the holy Ghoſt. \V For he was not yet come vpon any of
them, but they were only baptized in the name of our Lord \Sc{Iesvs}. \V
Then did
\LNote{Did they impoſe.}{\Emph{If
\MNote{The Sacramẽt of Confirmation miniſtred by Biſhops only. And that
with ſolemne praier and impoſition of handes.}
this Philip had been an Apoſtle} (ſaith S.~Bede) \Emph{he might haue
impoſed his handes, that they might haue receiued the Holy Ghoſt; but
this none can doe ſauing Biſhops. For though Prieſts may baptize and
anoint the Baptized alſo vvith Chriſme conſecrated by a Biſhop; yet he
can not ſigne his forehead vvith the ſame holy oile, becauſe that
belongeth only to Biſhops, vvhen they giue the holy Ghoſt to the
Baptized.} So ſaith he touching the Sacrament of Confirmation in
\Cite{8.~Act.}
This impoſition therfore of hands together with the praiers here
ſpecified (which no doubt were the very ſame that the Church yet vſeth
to that purpoſe) was the miniſtration of the Sacrament of
Confirmation. Whereof 
\CNote{\Cite{ep.~73. nu.~3. ad Iubainum.}}
S.~Cyprian ſaith thus:
They that in Samaria were baptized of Philip, becauſe they had lawful
and Eccleſiaſtical Baptiſme, ought not to be baptized any more: but only
that which wanted, was done by Peter and Iohn, to wit, that by praier
made for them and impoſition of handes, the Holy Ghoſt might be powred
vpon them. Which now alſo is done with vs, that they which in the Church
are baptized be by the Rulers of the Church offered, and by our praier &
impoſition of hand receiue the Holy Ghoſt, and be ſigned with our Lords
ſeale. So S.~Cyprian. But the Heretikes obiect that yet here is no
mention of oile. To whom we ſay, that many things were done and ſaid in
the adminiſtration of this and other Sacraments, & al inſtituted by
Chriſt himſelf & deliuered to the Church by the Apoſtles, which are not
particularly written by the Euangeliſts or any other in the Scripture;
among which this is euident by al antiquitie and moſt general practiſe
of the Church, to be one.

S.~Denys
\MNote{Chriſme in Confirmation.}
\CNote{\Cite{Ec. Hier. c.~2. &~4.}}
ſaith, the Prieſts did preſent the baptized to the Biſhop, that he might
ſigne them, \L{diuina et deiſico vnguento}, with the diuine and deifical
ointment. And again: \L{Aduentum S.~Spiritus conſummans inunctio
largitur}, the inunction conſummating, giueth the comming of the Holy
Ghoſt. Tertullian
\Cite{de reſur. nu.~7}
&
\Cite{li.~1. adu. Marcio.}
ſpeaketh of this Confirmation by Chriſme thus: \Emph{The flesh is
anointed, that the ſoul may be conſecrated: the fleſh is ſigned, that
the ſoul may be ſenſed: the fleſh by impoſition of hand is ſhadovved,
that the ſoul by the Spirit may be illuminated.} S.~Cyprian likewiſe,
\Cite{ep.~70. nu.~1.}
\Emph{He muſt alſo be anointed, that is baptized, vvith the oile
ſanctified on the Altar.} And
\Cite{ep.~71.}
(ſee alſo
\Cite{ep.~73. nu.~32.)}
he expreſly calleth it a Sacrament, ioyning it with Baptiſme, as
Melchiadas doth
\Cite{(ep. ad omnes Hiſpania Epiſcopus nu.~2. to.~1. Conc.)}
ſhewing the difference betwixt it and Baptiſme. S.~Auguſtine alſo,
\Cite{cont. lit. Petil. li.~2. c.~104.}
\Emph{The Sacrament of Chriſme in the kind of viſible ſeales is ſacred
and holy, euen as Baptiſme it ſelf.} We omit S.~Cyril
\Cite{myſtog.~3.}
S.~Ambroſe
\Cite{li.~3. de Sacrem. c.~2.}
&
\Cite{de ijs qui myſteriis imitiantur c.~7.}
S.~Leo
\Cite{ep.~88.}
the ancient Councels alſo of Loadicea,
\Cite{can.~43.}
Carthage
\Cite{3.~can.~39.}
and Arauſicanum
\Cite{4.~can.~1.}
and others. And S.~Clement
\Cite{(Apoſt. conſt. li.~7. c.~44.)}
reporteth certaine conſtitutions of the Apoſtles touching the ſame. 
\CNote{\Cite{Ec. Hier. c.~4.}}
S.~Denys referreth the manner of conſecration of the ſame Chriſme to the
Apoſtles inſtruction. S.~Baſil
\Cite{de Sp. ſancto c.~27.}
calleth it a tradition of the Apoſtles. And the moſt ancient Martyr
S.~Fabian
\Cite{ep.~2. as omnes Orientales Epiſcopes in initio. to.~1. conc.}
ſaith plainely that Chriſt himſelf did ſo inſtruct the Apoſtles at the
time of the inſtitution of the B.~Sacrament of the Altar. And ſo doth
the Authour of the booke
\Cite{de vnctione Chriſmatis apud D.~Cyprianum nu.~1.}
telling the excellent effects and graces of this Sacrament, & why this
kind of oile and balſme was taken of the old Law, & vſed in the
Sacraments of the new Teſtament. Which thing the Heretikes can with
leſſe cauſe obiect againſt the Church, ſeeing they confeſſe
\CNote{\Cite{Beza in Act. c.~6. v.~6.}}
that Chriſt and his Apoſtles tooke the ceremonie of impoſition of hands
in this and other Sacraments, from the Iewes manner of conſecrating
their hoſtes deputed to ſacrifice.

To
\MNote{Old hereſies againſt confirmation and Chriſme.}
conclude, neuer none denied or contemned this Sacrament of Confirmation
and holy Chriſme, but knowen Heretikes. S.~Cornelius that B.~Martyr ſo
much praiſed of S.~Cyprian
\Cite{ep. ad Fabrium apud Euſeb. li.~6. c.~35.}
affirmeth, that Nouatus fel to Hereſie, for that he had not receiued the
Holy Ghoſt by the conſignation of a Biſhop. Whom al the Nouatians did
follow, neuer vſing that holy Chryſme, as Theodorete writeth,
\Cite{li.~3. Fabul. Hær.}
And Operatus
\Cite{li.~2. cont Parm.}
writeth that it was the ſpecial barbarous ſacrilege of the Donatiſts, to
conculcate the holy oile. But al this is nothing to the ſauage diſorder
of Caluiniſts in this point.}
they impoſe their handes vpon them, & they
\LNote{And they receiued the Holy Ghoſt.}{The Proteſtants charge the
Catholikes,
\CNote{\Cite{Kemnit. in exam. Conc. Trid. de Confir.}}
that by approuing & commending ſo much the Sacrament of Confirmation, &
by attributing to it ſpecially the guift of the Holy Ghoſt, they
diminiſh the force of Baptiſme, chalenging alſo boldly the ancient
Fathers for the ſame. As though any Catholike or Doctour euer ſaid more
then the expreſſe wordes of Scripture here and elſwhere plainely giue
them warrant for. If they diminiſh the vertue of Baptiſme, then did
Chriſt ſo, appointing his Apoſtles and al the Faithful euen after their
Baptiſme to expect the Holy Ghoſt and vertue from aboue; then did the
Apoſtles iniurie to Baptiſme, in that they impoſed hands on the baptized
and gaue them the holy Ghoſt.
\MNote{The effectes of Baptiſme and Confirmation differ.}
And this is the Heretikes blindnes in this
caſe, that they can not, or wil not ſee that the Holy Ghoſt is giuen in
Baptiſme to remiſſion of ſinnes, life, and ſanctification: and in
Confirmation, for force, ſtrength, and corroboration to fight againſt al
our ſpiritual enemies, and to ſtand conſtantly in confeſſion of our
faith, euen to death, in times of perſecution either of the Heathen or
of Heretikes, with great increaſe of grace.
\MNote{Heretical ſhiftes & euaſions againſt manifeſt Scriptures, &
againſt this Sacrament of Confirmation.}
And let the good Reader note here our Aduerſaries great peruerſity and
corruption of the plaine ſenſe of the Scriptures in this point: ſome of
them affirming the holy Ghoſt here to be no other but the guift of
wiſedom in the Apoſtles and a few more to the gouernment of the Church;
when it is plaine that not only the Gouerners but al that were baptized,
receiued this grace, both men and women: Some, that it was no internal
grace, but only the guift of diuers languages: Which is very falſe;
the guift of Tongues being but a ſequele and an accident to the grace,
and an external token of the inward guift of the Holy Ghoſt, and our
Sauiour calleth it vertue from aboue. Some ſay, that whatſoeuer it was,
it was but a miraculous thing, and dured no longer then the guift of
the Tongues ioyned thereunto: by which euaſion they deny alſo the Sacrament
of Extreme Vnction, and the force of Excommunication becauſe the
corporal puniſhments which were annexed often times in the primitiue
Church vnto it, ceaſeth: and ſo may they take away (as they meane to
doe) al Chriſts faith or religion, becauſe it hath not the like
operation of miracles as in the beginning. But S.~Auguſtin toucheth this
point fully.
\CNote{\Cite{Tract.~6. in ep.~Io.}}
\Emph{Is there any man} (ſaith he) \Emph{of ſo peruerſe an hart, to deny
theſe children on vvhom vve novv impoſed hands, to haue receiued the
Holy Ghoſt, becauſe they ſpeake not
\Fix{vvhich}{vvith}{obvious typo, fixed in other}
Tongues? &c.} Laſtly ſome of them make no more of Confirmation or the
Apoſtles fact, but as of a doctrine, inſtruction, or exhortation to
continue in the faith receiued. Whereupon they haue turned this holy
Sacrament
\CNote{See
\Cite{Conc. Trid. ſeſſ.~7. can.~1. de Confirmat.}}
into a Cathechiſme.
\CNote{\Cite{Conc. Trid. ſeſſ.~7. can.~14. de Bapt.}}
There are alſo that put the baptized comming to yeares of diſcretion, to
their owne choiſe, whether they wil continue Chriſtians or no. To ſuch
diueliſh and diuers inuentions they fal, that wil not obey Gods Church
nor the expreſſe Scriptures, which tel vs of praiers, of impoſition of
hands, of the Holy Ghoſt, of grace and vertue from aboue, and not of
inſtruction, which might and may be done as wel before Baptiſme, & by
others, as by Apoſtles and Biſhops, to whom only this Holy function
pertaineth, in ſo much that in our Countrie it is called 
\MNote{Biſhoping.}
\Emph{Biſhoping}.}
receiued the holy Ghoſt. \V And when Simon had ſeen that by the
impoſition of the hand of the Apoſtles, the holy Ghoſt was giuen, he
\LNote{Offered money.}{This wicked ſorcerer Simon is noted by S.~Irenæus
\Cite{li.~1. c.~20.}
& others to haue been the firſt Heretike, & father of al Heretikes to
come, in the Church of God. He taught, only faith in him, without good
life and workes, to be enough to ſaluation. He gaue the onſet to
purchaſe with his money a ſpiritual function, that is to be made a
Biſhop; for to haue power to giue the Holy Ghoſt by impoſition of handes,
is to be a Biſhop:
\MNote{Simonie.}
as to buye the power to remit ſinnes or to
conſecrate Chriſts body, is to buy to be a Prieſt, or to buy Prieſthood:
and to buye the authoritie to miniſter Sacraments, to preach or to haue
cure of ſoules, is to buy a benefice: and likewiſe in al other ſpiritual
things, whereof either to make ſale or purchaſe for money or money
worth, is a great horrible ſinne called Simonie: & in ſuch as thinke it
lawful (as here Simon iudged it) it is named \Emph{Simoniacal Hereſie},
of this deteſtable man who firſt attempted to buye ſpiritual function or
office.
\Cite{D.~Greg. apud Ioan. Diac. in vit. li.~3. c.~2. 3.~4.~5.}}
offered them money, \V ſaying: Giue me alſo this power, that on
whomſoeuer I impoſe my handes, he may receiue the holy Ghoſt. \V But
Peter ſaid to him: Thy money be with thee vnto perdition: becauſe thou
haſt thought that the guift of God is purchaſed with money. \V Thou haſt
no part, nor lot in this word. For thy hart is not right before God. \V
\LNote{Doe penance.}{S.~Auguſtine
\MNote{Penance.}
\Cite{(ep.~108.)}
vnderſtanding this of the penance done in the Primitiue Church for
heinous offenſes, doth teach vs to tranſlate this & the like places
\XRef{(2.~Cor.~12,~21.}
\XRef{Apoc. v.~21.)}
as we doe, and as it is in the vulgar Latin, and conſequently that the
Greeke \G{μετανοεῖν} doth ſignifie ſo much. Yea when he addeth, that
very good men doe daily penance for venial ſinnes by faſting, praier, &
almes, he warranteth this phraſe & tranſlation throughout the new
Teſtament, ſpecially himſelf alſo reading ſo as it is in the vulgar
Latin, & as we tranſlate.}
Doe
\TNote{\G{ μετανόησον ἀπὸ} See
\XRef{Apoc.~9,~21.}}
penance therfore from this thy wickedneſſe: and pray to God,
\LNote{If perhaps.}{You may ſee, great penance is here required for
remiſſion of ſinne, & that men muſt ſtand in feare & dread leſt they be
not worthy to be heard or to obteine mercie. Wherby al men that buy or
ſel any ſpiritual function, dignities, offices, or liuings, may
ſpecially be warned that the ſinne is exceeding great.}
if perhaps this cogitation of thy hart may be remitted thee. \V For I
ſee thou art in the gal of bitternes and the obligation of
iniquitie. \V And Simon anſwering ſaid:
\LNote{Pray you for me.}{As
\MNote{Simon Magus more religious then the Proteſtants.}
this Sorcerer had more knowledge of the true religion then the
Proteſtants haue, who ſee not that the Apoſtles and Biſhops can giue
the Holy Ghoſt in this Sacrament or other, which he plainely perceiued
and confeſſed: ſo ſurely he was more religious then they, that being ſo
sharply checked by the Apoſtles, yet blaſphemed not as they doe when
they be blamed by the Gouerners of the Church, but deſired the Apoſtles
to pray for him.}
Pray you for me to our Lord, that nothing come vpon me of theſe things
which you haue ſaid. \V And they indeed hauing teſtified and ſpoken the
word of our Lord, returned to Hieruſalem, and euangelized to many
countries of the Samaritans.

\V And an Angel of our Lord ſpake to Philippe, ſaying: Ariſe, and goe
toward the South, to the way that goeth downe from Hieruſalem
%%% 2568
into Gaza:
\LNote{This is deſert.}{Intolerable boldnes of ſome Proteſtants, here
alſo (as in other places) againſt al copies both Greeke and Latin, to
ſurmiſe corruption or falſhood of the text, ſaying it can not be
ſo. Which is to accuſe the holy Euangeliſt, and to blaſpheme the Holy
Ghoſt himſelf.
\MNote{Beza.}
See
\CNote{\Cite{Annot. no. Teſt. 1556.}}
Beza, who is often very ſaucie with S.~Luke.}
this is deſert. \V And riſing he went. And behold, a man of Æthiopia, an
Eunuch, of great authoritie vnder Candace the Queene of the Æthiopians,
who was ouer al her treaſures, was come to Hieruſalem
\SNote{Note that this Æthiopiã came to Hieruſalẽ to adore, that is, on
Pilgrimage: whereby we may learne that it is an acceptable act of
religion to goe from home to places of greater deuotion & ſanctification.}
to adore: \V and he was returning and ſitting vpon his chariot, and
reading Eſay the Prophet. \V And the Spirit ſaid to Philippe: Goe neere,
and ioyne thy ſelf to this ſame chariot. \V And Philippe running
thereunto, heard him reading Eſay the 
\Fix{Propet,}{Prophet,}{obvious typo, fixed in other}
and he ſaid: Troweſt thou that thou vnderſtandeſt the things which thou
readeſt? \V Who ſaid: And
\SNote{The Scriptures are ſo written that they cannot be vnderſtood
without an interpreter, as eaſy as our Proteſtãts make them. See
S.~Hierom
\Cite{Ep. ad Palinum de omnibus diuinæ hiſtoria librit}
ſet in the beginning of latin bibles.}
how can I, vnleſſe ſome man ſhew me? & he deſired Philippe that he would
come vp and ſit with him. \V And the place of the ſcripture which he did
read, was this:
\CNote{\XRef{Eſ.~53,~7.}}
\Emph{As a sheep to ſlaughter was he led: and as a lamb before his
shearer, without voice, ſo did he not open his mouth. \V In humilitie
his iudgement was taken away.
%%% o-2412
His generation who shal declare, for from the earth shal his life be
taken?} \V And the Eunuch anſwering Philip, ſaid: I beſeech thee, of
whom doth the Prophet ſpeake this? of himſelf, or of ſome other? \V And,
Philip opening his mouth, and beginning from this ſcripture,
euangelized vnto him \Sc{Iesvs}. \V And as they went by the way, they
came to a certaine water: and the Eunuch ſaid: Loe water,
\Var{who}{vvhat}
doth lett me to be baptized? \V And Philip ſaid: If thou beleeue with al
thy hart, thou maieſt. And he anſwering ſaid: I beleeue that \Sc{Iesvs
Christ} is the Sonne of God. \V And he commanded the chariot to ſtay:
and both went downe into the water, Philip and the Eunuch, and
\LNote{He baptized him.}{When
\MNote{The ceremonies of Sacraments done, though not mentioned.}
the Heretikes of this time find mention made in Scripture of any
Sacrament miniſtred by the Apoſtles or other in the Primitiue Church,
they imagine no more was done then there is expreſly told, nor ſcarſly
beleeue ſo much.  As if impoſition of hands in the Sacrament of
Confirmation be only expreſſed, they thinke there was no chriſme, nor
other worke or word vſed. So they thinke no more ceremonie was vſed in
the baptizing of this noble man, then here is mentioned. Whereupon
S.~Auguſtin hath theſe memorable wordes:
\CNote{\Cite{De fid. & op.~8.~9.}}
\Emph{In that that he ſaith, Philip baptized him, he vvould haue it
vnderſtood that al things vvere done, vvhich though in the Scriptures
for breuitie ſake, they are not mentioned, yet by order of tradition vve
knovv vvere to be done.}}
he baptized him. \V And when they were come vp out of the water, the
Spirit of our Lord tooke away Philip, & the Eunuch ſaw him no more. And
he went on his way reioycing. \V But Philip was found in Azorus, and
paſſing through, he euangelized to al the cities, til he came to
Cæſarea.


\stopChapter


\stopcomponent


%%% Local Variables:
%%% mode: TeX
%%% eval: (long-s-mode)
%%% eval: (set-input-method "TeX")
%%% fill-column: 72
%%% eval: (auto-fill-mode)
%%% coding: utf-8-unix
%%% End:

