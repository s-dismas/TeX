%%%%%%%%%%%%%%%%%%%%%%%%%%%%%%%%%%%%%%%%%%%%%%%%%%%%%%%%%%%%%%%%%
%%%%
%%%% The (original) Douay Rheims Bible 
%%%%
%%%% New Testament
%%%% Acts
%%%% Chapter 11
%%%%
%%%%%%%%%%%%%%%%%%%%%%%%%%%%%%%%%%%%%%%%%%%%%%%%%%%%%%%%%%%%%%%%%




\startcomponent chapter-11


\project douay-rheims


%%% 2577
%%% o-2421
\startChapter[
  title={Chapter 11}
  ]

\Summary{The Chriſtian Iewes reprehend the foreſaid fact of Peter in
  baptizing the Gentils. 4.~But he alleaging his foreſaid warrants, and
  shewing plainly that it was of God, 18.~they like good Catholikes doe
  yeald. 19.~By the foreſaid perſecution, the Church is yet further
  dilated, not only into al Iewrie, Galilee, and Samaria, but alſo into
  other Countries: ſpecially in Antiochia Syriæ the increaſe among the
  Greekes, is notable, firſt by the foreſaid diſperſed, 22.~then by
  Barnabas, thirdly by him and Saul together: ſo that there beginneth
  the name of Chriſtians, 27.~with perfite vnity between them and the
  Church that was before them at Hieruſalem.}

And the Apoſtles and Brethren that were in Iewrie, heard that the
Gentils alſo receiued the word of God. \V And when Peter was come vp to
Hieruſalem, they that were of the Circumciſion reaſoned againſt him,
ſaying: \V Why didſt thou enter in to men
\TNote{\L{præputium habentes.}}
vncircumciſed, and didſt eate with them? \V But Peter began and
declared to them the order, ſaying: \V
\CNote{\XRef{Act.~10,~9.}}
I was in the citie of Ioppe praying, & I ſaw in an exceſſe of mind a
viſion, a certaine veſſel deſcending as it were a great ſheet with foure
corners let downe from heauen, & it came euen vnto me. \V Into which I
looking conſidered, & ſaw foure footed beaſtes of the earth, & catel, &
ſuch as creep, &
%%% o-2422
foules of the aire. \V And I heard alſo a voice ſaying to me: Ariſe
Peter, kil and eate. \V And I ſaid: Not ſo Lord; for common or vncleane
thing neuer entred into my mouth. \V And a voice anſwered the ſecond
time from heauen: That which God hath made cleane, doe not thou cal
common. \V And this was done thriſe: and al were taken vp againe into
heauen. \V And behold, three men immediatly were come to the houſe
wherein I was, ſent to me from Cæſarea. \V And the ſpirit ſaid to me,
that I ſhould goe with them, doubting nothing. And there came with me
theſe ſix Brethren alſo: and we went into the mans houſe. \V And he told
vs, how he had ſeen an Angel in his houſe, ſtanding and ſaying to him:
Send to Ioppe, and cal hither Simon, that is ſurnamed Peter, \V who ſhal
ſpeake to thee wordes wherein thou ſhalt be ſaued and al thy houſe. \V
And when he had begun to ſpeake, the Holy Ghoſt fel vpon them, as vpon
vs alſo in the beginning. \V And I remembred the word of our Lord,
according as he ſaid:
\CNote{\XRef{Act.~1,~5.}}
\Emph{Iohn indeed baptized with water, but you shal be baptized with the
Holy Ghoſt.} \V If therfore God hath giuen them the ſame grace, as to vs
alſo that beleeued in our Lord \Sc{Iesvs Christ}: who was I that might
prohibit God? \V Hauing heard theſe things, they
\SNote{Good Chriſtians heare & obey gladly ſuch truths as be opened vnto
thẽ from God by their cheefe Paſtours, by viſion, reuelation, or
otherwiſe.}
held their peace: & glorified God, ſaying: God then to the Gẽtils alſo
hath giuẽ repentãce vnto life.

\V
\CNote{\XRef{Act.~8,~1.}}
And they truly that had been diſperſed by the tribulation that was made
vnder Steuen, walked throughout vnto Phœnice & Cypres
& Antioche, ſpeaking the word to none, but to the Iewes only. \V But
certaine of them were men of Cypres and Cyrene, who when they were
entred into Antioche, ſpake to the Greekes, preaching our Lord
%%% 2578
\Sc{Iesvs}. \V And the hand of our Lord was with them: and a great
number of beleeuers was conuerted to our Lord. \V And the report came to
the eares of the Church that was at Hieruſalem, touching theſe things:
and they ſent
\CNote{\XRef{Act.~4,~36.}}
Barnabas as farre as Antioche. \V Who when he was come, and ſaw the
grace of God, reioyced: and he exhorted al with purpoſe of hart to
continue in our Lord: \V becauſe he was a good man, and ful of the Holy
Ghoſt and faith. And a great
\LNote{Multitude added.}{As before
\XRef{(c.~10.)}
a few, ſo now great numbers of Gentils are adioyned alſo to the viſible
Church, conſiſting before only of the Iewes. 
\MNote{The Church viſible.}
Which Church hath been euer ſince Chriſts Aſcenſion, notoriouſly ſeen
and knowen: their preaching open, their Sacraments viſible, their
diſcipline viſible, their Heades and Gouernours viſible, the prouiſion
for their maintenance viſible, the perſecution viſible, their diſperſion
viſible: the Heretikes that went out from them, viſible: the ioyning
either of men or Nations vnto them, viſible: their peace and reſt after
perſecutions, viſible: their Gouernours in priſon, viſible: the Church
praieth for them viſibly, their Councels viſible, their guifts and
graces viſible, their name (Chriſtians) knowen to al the world. Of the
Proteſtants inuiſible Church we heare not one word.}
multitude was added to our Lord. \V And he went forth to
\CNote{\XRef{Act.~9,~30.}}
Tarſus, to ſeeke Saul: \V whom when he had found, he brought him to
Antioche. And they conuerſed there
%%% o-2423
in the church a whole yeare: and they
taught a great multitude, ſo that the Diſciples were at Antioche firſt
named
\MNote{The name of
\Fix{\Sc{Chistians}.}{\Sc{Christians}.}{obvious typo, fixed in other}}
\LNote{Chriſtians.}{This name, \Emph{Chriſtian}, ought to be common to
al the Faithful, and other new names of Schiſmatikes and Sectaries muſt
be abhorred. \Emph{If thou heare} (ſaith
\CNote{\Cite{Hierom. cont. Lucif. c.~7. in fine.}}
S.~Hierom) \Emph{anywhere, ſuch as be ſaid to be of Chriſt, not to haue
the names of our Lord} \Sc{Iesvs Christ}, \Emph{but to be called after
ſome other certaine name, as
\MNote{Names of Sectaries and Heretikes.}
Marcionites, Valentinians}, (as now alſo
the Lutherans, Caluiniſts, Proteſtants) \Emph{know thou that they belong
not to the Church of Chriſt, but to the Synagogue of Antichriſt.}
Lactantius alſo
\Cite{(li.~7. Diuin. inſtit. c.~30.)}
ſaith thus: \Emph{When Phrygians, or Nouatians, or Valentinians, or
Marcionites, or Anthropomorphites, or Arians, or any other be named,
they ceaſe to be Chriſtians, who hauing left the name of Chriſt, haue
done on the names of men.} Neither can our new Sectaries diſcharge
themſelues, for that they take not to themſelues theſe names, but are
forced to beare them as giuen by their Aduerſaries. For, ſo were the
names of Arians & the reſt of old, impoſed by others, and not choſen
commonly of themſelues: which notwithſtanding were callings that proued
them to be Heretikes.
\MNote{Proteſtants.}
And as for the name of Proteſtants, our men hold them wel content
therewith. But concerning the Heretikes turning of the argument againſt
the peculiar callings of our Religious,
\Fix{Dominicians,}{Dominicans,}{likely typo, fixed in other}
Franciſcans, Ieſuites, Thomiſts, or ſuch like,
\MNote{Diuers religious orders are not diuers Sectes.}
it is nothing, except
they could proue that the orders & perſons ſo named, were of diuers faithes &
Sectes, or differed in any neceſſarie point of religion, or were not al
of one Chriſtian name & Communion: and it is as ridiculous as if it were
obiected, that ſome be Ciceronians ſome Plinians, ſome good Auguſtine
men, ſome Hieronymians, ſome Oxford men, ſome Cambrige men, & (which is
moſt like) ſome
\CNote{\XRef{Ierem.~35.}}
Rechabites, ſome
\CNote{\XRef{Num.~6.}}
Nazareites.

Neither
\MNote{Papiſtes, Catholikes, and true Chriſtiãs, al one.}
doth their obiection, that we be called Papiſtes, helpe or excuſe them
in their new names. For beſides that it is by them ſcornfully inuented
(as the name Homouſians was of the Arians) this name is not of any one
man, Bishop of Rome or els where, knowen to be the authour of any
Schiſme or Sect, as their callings be: but it is of a whole ſtate and
order of Gouernours, and that of the cheefe Gouernours, to whom we are
bound to cleaue in religion and to obey in al things. So to be a Papiſt,
is to be a Chriſtian man, a child of the Church, and ſubiect to Chriſts
Vicar. And therfore againſt ſuch impudent Sectaries as compare the
faithful for following the Pope, to the diuerſitie of Heretikes bearing
the names of new Maiſters, let vs euer haue in readines this ſaying of
S.~Hierom to Pope Damaſus:
\CNote{\Cite{to.~2. ep.~57. &~58. ad Damaſ.}}
\MNote{Not to be with the Pope, is to be with Antichriſt.}
\Emph{Vitalis I know not, Meletius I refuſe, I know not Paulinus;
whoſoeuer gathereth not with thee, ſcattereth: that is to ſay, whoſoeuer
is not Chriſts, is Antichriſts.} And againe, \Emph{If any man ioyne with
Peters Chaire, he is mine.}

We muſt here further obſerue that this name,
\MNote{The name of \Sc{Christians}.}
Chriſtian, giuen to al beleeuers & to the whole Church, was ſpecially
taken to diſtinguish them from the Iewes & Heathens which beleeued not
at al in Chriſt: and the ſame now ſeuereth and maketh knowen al Chriſtian
men from Turkes and others that hold not of Chriſt at al. But when
Heretikes began to riſe from among the Chriſtians, who profeſſed Chriſts
name, and ſundry articles of faith, as true beleeuers doe, the name
\Emph{Chriſtian} was too common to ſeuer the Heretikes from true
faithful men; and thereupon the Apoſtles by the Holy Ghoſt impoſed this
name,
\MNote{The name of \Sc{Catholikes}.}
\Emph{Catholike}, vpon the Beleeuers which in al points were
obedient to the Churches doctrine. \Emph{When hereſies were riſen}
(ſaith S.~Pacianus
\Cite{ep. ad Symphorianum})
\Emph{& endeuoured by diuers names to teare the doue of God and Queene,
and to rent her in peeces, the Apoſtolical people required their
ſurname, whereby the incorrupt people might be diſtinguished, &c.} and
ſo thoſe that before were called Chriſtians, are now ſurnamed alſo
Catholikes. \Emph{Chriſtian is my name}, ſaith he, \Emph{Catholike my
ſurname.} And this word, Catholike, is the proper note whereby the holy
Apoſtles in their
\MNote{\L{\Sc{Credo Ecclesiam Catholicam}}.}
Creed taught vs to diſcerne the true Church from the
falſe heretical congregation of what ſort ſoeuer. And not only the
meaning of the word, which ſignifieth vniuerſalitie of times, places,
and perſons, but the very name and word itſelf, by Gods prouidence,
alwaies and only appropriated to the true beleeuers, and (though
ſometimes at the beginning of Sectes chalenged) yet neuer obtained by
Heretikes, giueth ſo plaine a marke and euidence, that S.~Auguſtine
ſaid: \Emph{In the lappe of the Church the very name of Catholike
keepeth me.}
\Cite{cont. ep. fund. c.~4.}
And againe
\Cite{tract.~32. in Io.}
\Emph{We receiue the Holy Ghoſt if we loue the Church, if we be ioyned
together by charitie, if we reioyce in the Catholike name and faith.}
And againe
\Cite{de ver. rel. c.~7. to.~1.}
\Emph{We muſt hold the communion of that Church which is named
Catholike, not only of her owne, but alſo of al her enemies. For, wil
they nil they, the Heretikes alſo and Schiſmatikes themſelues, when they
ſpeake not with their owne fellowes but with ſtrangers, cal the
Catholike Church nothing els but the Catholike Church: for they could
not be vnderſtood vnles they diſcerne it by this name, wherewith she is
called of al the world.}
\MNote{The Proteſtãts deride the name \Sc{Catholikes}.}
The Heretikes when they ſee themſelues preuented of this
name \Emph{Catholike}, then they plainely reiect it and deride the name,
as the Donatiſtes did, calling it an \Emph{humane forgerie or fiction}:
which S.~Auguſtine calleth wordes of blaſphemie,
\Cite{li.~1. c.~33. cont. Gaudens.}
and ſome Heretikes of this time cal them ſcornefully cartholikes, and
cacolikes. Another calleth it, \Emph{this moſt vaine terme Catholike}.
\Cite{Beza in præf. no Teſt. an.~1565.}
Another calleth the Catholike religion, \Emph{a Catholike
Apoſtaſie or defection}.
\Cite{Humfrey in vit. Iuel. pag.~213.}
Yea & ſome haue taken the word out of the Creed,
\CNote{In the Catechiſmes of the Lutherans.}
putting \Emph{Chriſtian} for it. But againſt theſe good fellowes let vs
follow that which S.~Auguſtine
\Cite{(de vtil. cred. c.~8. to.~6.)}
giueth as a rule to direct a man the right and ſure away from the
diuerſitie & doubtfulnes of al errour ſaying: \Emph{If after theſe
troubles of mind then thou ſeeme to thy ſelf ſufficiently toſſed and
vexed, & wilt haue an end of theſe moleſtation, folow the way of
Catholike diſcipline, which from Chriſt himſelf by the Apoſtles hath
proceeded euen vnto vs, and shal proceed from hence to the poſteritie.}
See the
\XRef{Annotation. 1.~Tim.~3,~15.}}
\Fix{\Sc{Chritians}.}{\Sc{Christians}.}{obvious typo, fixed in other}

\V And in theſe daies there came Prophets from Hieruſalem to
Antioche, \V and one of them riſing, named Agabus, did by the Spirit
ſignifie a great famine that ſhould be in the whole world, which fel
vnder Claudius. \V And the Diſciples according as each man had, purpoſed
euery one to ſend, for to ſerue the Brethren that dwelt in Iewrie: \V
which alſo they did, ſending to the Ancients by the hands of Barnabas
and Saul.


\stopChapter


\stopcomponent


%%% Local Variables:
%%% mode: TeX
%%% eval: (long-s-mode)
%%% eval: (set-input-method "TeX")
%%% fill-column: 72
%%% eval: (auto-fill-mode)
%%% coding: utf-8-unix
%%% End:

