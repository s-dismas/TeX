%%%%%%%%%%%%%%%%%%%%%%%%%%%%%%%%%%%%%%%%%%%%%%%%%%%%%%%%%%%%%%%%%
%%%%
%%%% The (original) Douay Rheims Bible 
%%%%
%%%% New Testament
%%%% Epistles Argument
%%%% Argument
%%%%
%%%%%%%%%%%%%%%%%%%%%%%%%%%%%%%%%%%%%%%%%%%%%%%%%%%%%%%%%%%%%%%%%




\startcomponent argument


\project douay-rheims


%%% 2626
%%% o-2477
\startArgument[
  title={\Sc{the argvment of the Epistles in General.}},
  marking={Argument of the Epistles.}
  ]

After the Ghoſpels, which is a ſtorie of Chriſt himſelfe, and after
the \Emph{Actes of the Apoſtles}, which is a ſtorie of Chriſtes Church:
now follow the \Emph{Epiſtles of the Apoſtles}, which they wrote of ſuch
matters, as they had then occaſion to write of. For, being the Founders and
the Doctours of the Church, they did in their time, as the Doctours that
ſucceeded them, did after them: who from the beginning vnto this day,
haue written Epiſtles & Bookes againſt hereſies, euer as they aroſe, and
of al other Eccleſiaſtical matters, as they had occaſion miniſtred vnto
them. Of which their doing the Apoſtles firſt gaue here the enſample: as
alſo S.~Luke in the Actes of the Apoſtles, led the way to al the Writers
of the Eccleſiaſtical Hiſtorie after him. For although there be no
compariſon between them for authoritie, for aſmuch as theſe are
Canonical Scripture, and ſo are not any writings of their Succeſſours;
yet the occaſions and matters (as I haue ſaid) are like.

Moſt of the Epiſtles are S.~Paules Epiſtles: the reſt are called
\CNote{\Cite{Euſeb. li.~2. Eccl. hiſt. c.~23.}}
\Emph{Catholicæ Epiſtolæ}, the \Emph{Epiſtles Catholike}. For S.~Paul
writeth not any Epiſtle to al (howbeit euery one of them is for al the
Church:) but ſome to particular Churches of the Gentils; as to the
Romanes, to the Corinthians, to the Galatians, to the Epheſians, to the
Philippians, to the Coloſsians, to the Theſſalonians: ſome to particular
Perſons, as to Timothee, to Tite (who were Bishops among the Gentils, to
wit, of Epheſus, and of Crete) and to Philemon, and then one to the
Hebrewes, who were the Iewes of Hieruſalem & Iurie. But the Epiſtles of
the other Apoſtles, that is, of S.~Iames, S.~Peter, S.~Iohn, and
S.~Iude, are not ſo intituled to any one Church or perſon (except
S.~Iohns two later short Epiſtles, which yet might not be ſeparated from
his firſt, becauſe they were al of one Authour) and therefore they are
termed \Emph{Catholike}, that is \Emph{vniuerſal}. For ſo writeth
S.~Iames: \Emph{To the twelue Tribes that are in diſpeſion, greeting.}
And S.~Peter in his firſt Epiſtle, thus: \Emph{To the elect ſtrangers of
the diſperſion of Pontus, Galatia, Capadocia, Aſia, & Bithynia}; in his
ſecond, thus: \Emph{To them that haue obteined equal faith with vs.}
Likewiſe S.~Iude: \Emph{To them that are in God the Father beloued, & in
Ieſus Chriſt preſerued, & called.} S.~Iohns firſt is without title.

Now, for the occaſions of their writing, whereby we shal perceiue the
matters of arguments that they handle; it muſt be remembred (as the
Storie of that time in the Actes of the Apoſtles doth at large declare)
that the Church then beginning, was planted by the Apoſtles not only in
the Iewes, but alſo in the Gentils: yea and ſpecially in the
Gentils. Which thing offended the Iewes many waies. For, they could not
abide to
%%% 2627
ſee ſo much as their owne Countrie to receiue him for \Sc{Christ}, whom
they had reiected and crucified; much leſſe, to ſee them preach him to
the Gentils alſo: that offended euen thoſe Iewes alſo, that
%%% o-2478
beleeued him to be Chriſt. Howbeit ſuch of them as were Catholikes, and
therfore not obſtinate, were ſatisfied when they vnderſtood by the
Apoſtles that it was Gods pleaſure, as
\XRef{Act.~11.}
we read. But others of them became heretikes, & preached to the
Chriſtian Gentils, that it was neceſſary for them to receiue alſo the
Iewes religion. Of ſuch we read
\XRef{Act.~15.}
\Emph{Vnles you be circumciſed, you can not be ſaued.} And as theſe did
ſo preach againſt the truth, ſo did the vnchriſtned Iewes not only
themſelues perſecute, but alſo ſtirre vp the Idolatrous Gentils euery
where to perſecute the Chriſtians; by ſuch obſtinacie prouoking God to
reprobate their Nation: which yet they thought vnpoſsible to be done,
becauſe they were the ſeed of Abraham, and were circumciſed, and
had receiued the Law by Moyſes. For ſuch carnal reſpects they truſted in
themſelues, as though God and Chriſt were vnſeparably bound vnto them:
attributing alſo ſo much to their owne workes, (which they thought they
did of themſelues, being holpen with the knowledge of their law,) that
they would not acknowledge the death of Chriſt to be neceſſarie for
their ſaluation: but looked for ſuch a Chriſt, as should be like other
Princes of this world, and make them great men temporally.

Hereupon did S.~Paul write his Epiſtles, to shew both the vocation of
the Gentils, and the reprobation of the Iewes. Moreouer, to admonish
both the Chriſtian Gentils, not to receiue Circumciſion and other
ceremonies of Moyſes law, in no wiſe: and the Iewes alſo, not to put
their truſt in the ſame, but rather to vnderſtand, that now Chriſt being
come, they muſt ceaſe. Againe, to shew the neceſsitie of Chriſts comming
and of his death, that without it neither the Gentils could be ſaued; no
nor the Iewes, by no workes that they could doe of themſelues, although
they were alſo holpen by the Law: telling them what was good & what bad:
for ſo much as al were ſinners, and therfore alſo impotent or infirme,
and the law could not take away ſinne and infirmitie, and giue ſtrength
to fulfil that which it gaue knowledge of. But this was God only able to
doe, and for Chriſts ſake only would he doe it. Therfore it is
neceſſarie for al to beleeue in Chriſt, and to be made his members, being
incorporate into his Body which is his Catholike Church. For ſo (although
they neuer yet did good worke, but al il) they shal haue remiſsion of
their ſinnes, and new ſtrength withal, to make them able to fulfil the
commandements of Gods law, yea & their workes after this shal be ſo
gracious in Gods ſight, that for them he wil giue them life
euerlaſting. This is the neceſsitie, this is alſo the fruit of Chriſtian
Religion. And therfore he exhorteth al, both Gentils and Iewes, as to
receiue it humbly, ſo alſo to perſeuer in it conſtantly vnto the end,
againſt al ſeduction of hereſie, and againſt al terrour of perſecution:
and to walke al their time in good workes, as now God had made them able
to doe.

The ſame doctrine doth the Catholike Church teach vnto this day moſt
exactly:
\MNote{The doctrine of the Cath. Church concerning good workes.}
to wit, that no workes of the vnbeleeuing or vnbaptized, whether they be
Iewes or Gentils, can ſaue them: no nor of any Heretike, or Schiſmatike,
although he be baptized, becauſe he is not a member of Chriſt. Yea more
then that, no worke of any that is not a liuely member of Chriſt,
although otherwiſe he be baptized, and continue within his Church, yet
becauſe he is not in grace but in mortal ſinne, no worke that he doth,
is meritorious or able to ſaue him.

This very ſame is S.~Paules doctrine:
\MNote{S.~Paules doctrine concerning faith and good workes.}
he denieth to the workes of ſuch as haue not the Spirit of Chriſt, al
vertue to iuſtifie or to ſaue; neither requireth he a mã to haue had
knowledge of the Law, or to haue kept it aforetime, as though otherwiſe
he might not be ſaued
%%% 2628
by Chriſt: but yet when he is Chriſtned, he requireth of neceſsitie,
that he keep Gods commandements, by auoiding of al ſinne, and doing good
workes: and to ſuch a mans good workes he attributeth as much vertue as
%%% o-2479
any Catholike of this time.

Neuertheleſſe there were certaine at that time (as alſo al the Heretikes
of this our time) whom S.~Peter termeth
\CNote{\XRef{2.~Pet.~3.}}
\Emph{vnlearned and vnſtable}, who reading S.~Paules Epiſtles, did
miſconſter his meaning, as though he required not good workes no more
after Baptiſme, then before Baptiſme; but held that only Faith did
iuſtifie and ſaue a man. Thereupon the other Apoſtles wrote their
Epiſtles, as S.~Auguſtin noteth in theſe wordes:
\CNote{\Cite{Aug. de fide & oper. ca.~14.}
Et
\Cite{præf. pſal.~31.}}
\Emph{Therfore becauſe this opinion (\L{Ad ſalutem obtinendam ſufficere
ſolam fidem}, that only faith is ſufficient to obteine ſaluation) was
then riſen, the other Apoſtolical Epiſtles, of Peter, Iohn, Iames, Iude,
doe againſt it ſpecially direct their intention: to auouch vehemently,
\L{fidem ſine operibus nihil prodeſſe}, that faith without workes
profiteth nothing. As alſo Paul himſelf did not define it to
be \L{quamlibet fidem, qua in Deum creditur}, whatſoeuer manner of faith
wherewith we beleeue in God, but that holeſome & expreſſe Euangelical
faith, whoſe workes proceed from loue, and}
\CNote{\XRef{Gal.~5.}}
the faith (\Emph{quoth he}) that worketh by loue. \Emph{Wherevpon that
faith, which ſome thinke to be ſufficient to ſaluation, he ſo affirmeth
to profit nothing, that he ſaieth:}
\CNote{\XRef{1.~Cor.~13.}}
If I should haue al faith, ſo that I could remoue mountaines, and haue
not charitie, I am nothing.

He therfore that wil not erre in this point, nor in any other, reading
either S.~Paules Epiſtles, or the reſt of the holy Scriptures, muſt
ſticke faſt to the doctrine of the Catholike Church, which Church
S.~Paul termeth 
\CNote{\XRef{1.~Tim.~3.}}
\Emph{the pillar & ground of the truth}: Aſſuring
himſelf that if any thing there ſound to him as contrarie hereunto, he
faileth of the right ſenſe; and bearing alwaies in his mind the
admonition of S.~Peter, ſaying:
\CNote{\XRef{2.~Pet.~3.}}
\Emph{As alſo our moſt deare brother Paul according to the wiſedom giuen
to him, hath written to you: as alſo in his Epiſtles, ſpeaking in them
of theſe things, in the which are certaine things hard to vnderſtand,
which the vnlearned and vnſtable depraue, as alſo the reſt of the
Scriptures, to their owne perdition. You therfore, Brethren,
foreknowing, take heed leſt ye be led amiſſe by the errour of the
vnwiſe, and fal away from your owne ſtedfaſtnes.}


\stopArgument


\stopcomponent


%%% Local Variables:
%%% mode: TeX
%%% eval: (long-s-mode)
%%% eval: (set-input-method "TeX")
%%% fill-column: 72
%%% eval: (auto-fill-mode)
%%% coding: utf-8-unix
%%% End:
