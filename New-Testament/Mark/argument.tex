%%%%%%%%%%%%%%%%%%%%%%%%%%%%%%%%%%%%%%%%%%%%%%%%%%%%%%%%%%%%%%%%%
%%%%
%%%% The (original) Douay Rheims Bible 
%%%%
%%%% New Testament
%%%% Mark
%%%% Argument
%%%%
%%%%%%%%%%%%%%%%%%%%%%%%%%%%%%%%%%%%%%%%%%%%%%%%%%%%%%%%%%%%%%%%%




\startcomponent argument


\project douay-rheims


%%% 2368
%%% o-2186
\startArgument[
  title={\Sc{the argvment of s.~markes ghospel.}},
  marking={Argument of S.~Mark's Gospel}
  ]

S.~Markes Ghoſpel may be wel diuided into foure parts.

The firſt part, of the preparation that was made to the manifeſtation of
Chriſt: Chap.~1.\ in the beginning.

The Second, of his manifeſting himſelf by Preaching & Miracles, & that
in Galilee: the reſidue of the 1.~chap.\ vnto the 10.~chap.

The third, of his comming into Iewrie, towards his Paſsion: chap.~10.

The fourth, of the Holy weeke of his Paſsion in Hieruſalem: chap.~11.\
to the end of the booke.

Of S.~Marke and his conuerſion with the two Apoſtles S.~Paul and
S.~Barnabee, we haue at large Act.~12.\ and 15.\ ſome what alſo Col.~4.\
and 2.~Tim.~4.\ and to Philemon. Moreouer of his familiaritie with the
Prince of the Apoſtles S.~Peter, we haue 1.~Pet.~5. For ſo it pleaſed
our Lord, that only two of the Euangeliſts should be of his twelue
Apoſtles, to wit, S.~Matthew and S.~Iohn. The other two, S.~Marke and
S.~Luke, he gaue vnto vs of the Diſciples of his two moſt principal and
moſr glorious Apoſtles S.~Peter and S.~Paul. Whoſe Ghoſpels therfore
were of Antiquitie counted as the Ghoſpels of S.~Peter and S.~Paul them
ſelues. Marke the Diſciple, and interpreter of Peter (ſaith S.~Hierom)
according to that which he heard of Peters mouth, wrote at Rome a briefe
Ghoſpel at the requeſt of the Brethren (about 10.\ or 12.\ yeares after
our Lordes Aſcenſion.) which when Peter had heard, he approued it, and
with his authoritie did publiſh it to the Church to be read, as Clemens
Alexandrinus writeth li.~6.\ Hypotypoſ.

In the ſame place S.~Hierom addeth, how he went into Ægypt to preach,
and was the firſt Bishop of the cheefe Citie there, named Alexandria:
and how Philo Iudæus at the ſame time ſeeing & admiring the life &
conuerſation of the Chriſtians there vnder S.~Marke, who were Monkes,
wrote a booke thereof, which is extant to this day. And not only
S.~Hierom (in Marco, & in Philone) but alſo Euſebius Hiſt.\ li.~2.\ ca.\
15.~16.~17. Epiphanius Secta~29. Nazaræorum li.~1.\ to.~2. Caſsianus de
Inſtit.\ Cænobiorum li.~2.\ c.~5. Sozomenus li.~1.\ c.~12. Nicephorus
lib.~2.\ c.~15. and diuerſe others doe make mention of the ſaid Monkes
out of the ſame Authour. Finally, He died Iſaith S.~Hierom) the 8.~yeare
of Nero, and was buried at Alexandria, Anianus ſucceeding in his
place. But from Alexandria he was tranſlated to Venice, Anno Dom.\ 830.

It is alſo to be noted, that in reſpect of S.~Peter, who ſent S.~Marke
his ſcholer to Alexandria, and made him the firſt Bishop there, this See
was eſteemed next in dignitie to the See of Rome, and the Bishop thereof
was accounted the cheefe Metropolitan of Patriarch of the Eaſt, and that
by the firſt Councel of Nyce. Whereof ſee S.~Leo ep.~53.\ S.~Gregorie
li.~5.\ ep.~60.\ & li.~6.\ ep.~37.

\stopArgument


\stopcomponent


%%% Local Variables:
%%% mode: TeX
%%% eval: (long-s-mode)
%%% eval: (set-input-method "TeX")
%%% fill-column: 72
%%% eval: (auto-fill-mode)
%%% coding: utf-8-unix
%%% End:
