%%%%%%%%%%%%%%%%%%%%%%%%%%%%%%%%%%%%%%%%%%%%%%%%%%%%%%%%%%%%%%%%%
%%%%
%%%% The (original) Douay Rheims Bible 
%%%%
%%%% New Testament
%%%% Mark
%%%% Chapter 05
%%%%
%%%%%%%%%%%%%%%%%%%%%%%%%%%%%%%%%%%%%%%%%%%%%%%%%%%%%%%%%%%%%%%%%




\startcomponent chapter-05


\project douay-rheims


%%% 2378
%%% o-2197
\startChapter[
  title={Chapter 5}
  ]

\Summary{To the Geraſens (and in them to al men) Chriſt manifeſteth how
  the Diuel of his malice would vſe them, if he would permit: 17.~and
  yet they like not their Sauiours preſence. 21.~A woman Gentil, that
  began her ſicknes when the Iewes daughter began her life (ſignifying
  Abrahams time) he cureth by the way as he was comming to heale the
  Iewes: And euen then the Iewes doe die; but yet them alſo he wil
  reuiue, as here the Iewes daughter.}

%%% o-2198
And
\CNote{\XRef{Mt.~8,~28.}
\XRef{Lu.~8,~26.}}
they came beyond the ſtrait of the ſea into the countrie of the
Geraſens. \V And as he went out of the boat, immediatly there met him
out of the ſepulchres a man in an vncleane Spirit, \V that had his
dwelling in the ſepulchres, and neither with chaines
\LNote{Could bind him}{We
\MNote{Profane and natural men.}
ſee here that mad men which haue extraordinary ſtrength are many times
poſſeſſed of the Diuel: as there is alſo a deafe and a dumme Diuel, and
vncleane ſpirits, which worke theſe effects in men poſſeſſing their
bodies. Al which things infidels & carnal men, folowing only nature and
reaſon, attribute to natural cauſes: and the leſſe faith a man hath, the
leſſe he beleeueth that the Diuel worketh ſuch things.}
could any man now bind him: \V For being often bound with fetters and
chaines, he had burſt the chaines, and broken the fetters, and no body
could tame him. \V And he was alwaies day and night in the ſepulchres
and in the mountaines, crying and cutting him ſelf with ſtones. \V And
ſeeing \Sc{Iesvs} a farre off, he ranne and adored him: \V And crying
with a great voice, ſaid: What to me and thee \Sc{Iesvs}, the Sonne of
God moſt high: I adiure thee by God that thou torment me not. \V For he
ſaid vnto him: Goe out of the man thou vncleane Spirit. \V And he asked
him, what is thy name? And he ſaith to him: My name is Legion;
becauſe we are many. \V And he beſought him much, that he would not
expel him out of the countrie. \V And there was there about the
mountaine a great heard of ſwine, feeding. \V And the Spirits beſought
him, ſaying: Send vs
\SNote{It is not with out myſterie that the Diuels deſired, and Chriſt
ſuffred them to enter into the ſwine; ſignifying that filthy liuers be
meet dwelling places for Diuels.
\Cite{Aug. tract.~9. in ep.~Io.}}
into the ſwine, that we may enter into them. \V And \Sc{Iesvs}
immediatly granted vnto them. And the vncleane Spirits going out, entred
into the ſwine: and the heard with great violence was caried headlong
into the ſea, about two thouſand, and were ſtifled in the ſea. \V And
they that
%%% 2379
fed them, fled, and caried newes into the citie and into the fields. And
they went forth to ſee what was done: \V And they come to \Sc{Iesvs},
and they ſee him that was vexed of the Diuel, ſitting, clothed, and wel
in his wits, & they were afraid. \V And they that had ſeen it, told thẽ,
in what mãner he had been dealt withal that had the Diuel; & of the
ſwine. \V And they began to deſire him, that he would depart from their
coaſts. \V And when he went vp into the boat, he that had been vexed of
the Diuel, began to beſeech him that he might be with him: \V And he
admitted him not, but ſaith to him: Goe into thy houſe to thine, and tel
them how great things the Lord hath done for thee, and hath had mercie
vpon thee. \V And he went his way, and began to publiſh in Decapolis how
great things \Sc{Iesvs} had done to him: and al marueled.

%%% o-2199
\V
\CNote{\XRef{Mt.~9,~18.}
\XRef{Lu.~8,~42.}}
And when \Sc{Iesvs} had paſſed in boat againe ouer the ſtrait, a
great multitude aſſembled togeather vnto him, and he was about the
ſea. \V And there commeth one of the Arch-ſynagogues,
\MNote{\Emph{Archſynagogue}, cheefe gouerner of a Synagogue.}
named Iairus: and ſeeing him, he falleth downe at his feet. \V And
beſought him much, ſaying: That my daughter is at the point of death,
come, impoſe thy hands vpon her, that ſhe may be ſafe and liue. \V And
he went with him, and a great multitude folowed him, and they thronged
him.

\V And a woman which was in an iſſue of bloud twelue yeares, \V and had
ſuffred many things of many Phyſicions, and had beſtowed al that ſhe
had, neither was any thing the better, but was rather worſe: \V when ſhe
had heard of \Sc{Iesvs}, ſhe came in the preaſſe behind him, and touched
his garment. \V For ſhe ſaid: That
\LNote{If I shal touch}{So
\MNote{The touch of Relikes.}
the good Catholike ſaith: If I might but touch one of his Apoſtles, yea
one of his Apoſtles napkins, yea but the shade of one of his Saints, I
ſhould be better for it.
\XRef{Act.~5. and 19.}
\Cite{See S.~Chrys. to.~5. cont. Gent. in principio, in vit. Babylæ}
Yea S.~Baſil ſaith:
\CNote{\Cite{Baſil in Pſ.~115.}}
He that toucheth the bone of a Martyr, receaueth in
ſome degree holineſſe of the grace or vertue that is therin.}
if I ſhal touch but his garment, I ſhal be ſafe. \V And forthwith the
fountaine of her bloud was dried; and ſhe felt in her body that ſhe was
healed of the maladie. \V And immediatly \Sc{Iesvs} knowing in him ſelf
\LNote{Vertue}{Vertue to heale this womans maladie, proceeded from Chriſt,
though she touched but his coate: ſo when the Saints by their Relikes or
garments doe miracles, the grace and force therof commeth from our
Sauiour, they being but the meanes or inſtrumẽts of the ſame.}
the vertue that had proceeded from him, turning to the multitude, ſaid:
Who hath touched my garments? \V And his Diſciples ſaid to him: Thou
ſeeſt the multitude thronging thee, & ſayeſt thou, who hath touched
me? \V And he looked about to ſee her that had done this. \V But the
woman fearing and trembling, knowing what was done in her, came and fel
downe before him, and told him al the truth. \V And he ſaid to her:
Daughter, thy faith hath made thee ſafe, goe in peace, and be whole of
thy maladie.

\V As he was yet ſpeaking, they come
\Var{to}{from}
the Archſynagogue, ſaying: That thy daughter is dead: why doeſt thou
trouble the Maiſter any further? \V But \Sc{Iesvs} hauing heard the word
that was ſpoken, ſaith to the Archſynagogue: Feare not;
\LNote{Only beleeue}{It is our common ſpeach, when we require one thing
ſpecially,
though other things alſo be as neceſſarie, and more neceſſarie. As the
Phyſicion to his patient, \Emph{Only haue a good hart}: when he muſt
alſo keep a diet and take potions, things more requiſit.  So Chriſt in
this great infidelity of the Iewes, required only that they would
beleeue he was able to doe ſuch a cure, ſuch a miracle, & thẽ he did it:
otherwiſe it foloweth in the next Chapter: \Emph{He could not doe
miracles there becauſe of their incredulity.} Againe, for this faith he
gaue thẽ here & in al like places health of the body, which they
deſired. And therfore he ſaith not: Thy faith hath iuſtified thee: but,
hath made thee ſafe or whole. Againe this was the fathers faith, which
could not iuſtifie the daughter.
\MNote{Scripture fõdly applied to proue only faith.}
Wherby it is moſt euident, that this
Scripture, and the like, are foolishly abuſed of the Heretikes to proue
that only faith iuſtifieth.}
only beleeue. \V And he admitted not any man to follow him, but Peter &
Iames and Iohn the brother of Iames. \V And they come to the
Archſynagogues houſe, and he ſeeth a tumult, and folke weeping and
wailing much. \V And going in, he ſaith to them: Why make you this adoe
and weep? the wench is not dead, but
\SNote{To Chriſt that can more eaſily raiſe a dead man then we can do
one that is but aſleep, death is but ſleep.
\Cite{Aug. de verb. Do. Ser.~44.}}
ſleepeth. \V And they derided him. But he hauing put forth al, taketh
the father and mother of the wench, and them that were with him, and
they goe in where the wench was lying. \V And holding the wenches hand,
he ſaith to her: \Emph{Talitha cumi}, which is being
interpreted,
\LNote{Wench ariſe}{Chriſts miracles, beſides they be wonders & wayes to
ſhew his
power, be alſo ſignificatiue:
\CNote{\Cite{Aug. de verb. Do. ſer.~44.}}
as theſe which he corporally raiſed frõ
death, put vs in mind of his raiſing our ſoules from ſinne. The
Scripture maketh ſpecial mention only of three raiſed by our Sauiour, of
which three, this wench is one, within the houſe: an other, the widowes
ſonne in Naim, now caried out toward the graue; the third, Lazarus
hauing been in the graue foure daies, and therfore ſtinking.
\MNote{By three dead, are ſignified three kinds of ſinners.}
Which diuerſity of dead bodies, ſignifie diuerſity of dead ſoules, ſome
more deſperate than other, ſome paſt al mans hope, and yet by the grace
of Chriſt to be reuiued and reclaimed.}
\Emph{wench} (I ſay to thee) \Emph{ariſe}. \V And
%%% o-2200
forthwith the wench roſe vp, and walked, and ſhe was twelue yeares old:
and they 
%%% 2380
were aſtoniſhed with great aſtoniſhment. \V And he commanded them
earneſtly that no body ſhould know it: and he bad that ſome thing ſhould
be giuen her to eate.

\stopChapter


\stopcomponent


%%% Local Variables:
%%% mode: TeX
%%% eval: (long-s-mode)
%%% eval: (set-input-method "TeX")
%%% fill-column: 72
%%% eval: (auto-fill-mode)
%%% coding: utf-8-unix
%%% End:
