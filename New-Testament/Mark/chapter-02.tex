%%%%%%%%%%%%%%%%%%%%%%%%%%%%%%%%%%%%%%%%%%%%%%%%%%%%%%%%%%%%%%%%%
%%%%
%%%% The (original) Douay Rheims Bible 
%%%%
%%%% New Testament
%%%% Mark
%%%% Chapter 02
%%%%
%%%%%%%%%%%%%%%%%%%%%%%%%%%%%%%%%%%%%%%%%%%%%%%%%%%%%%%%%%%%%%%%%




\startcomponent chapter-02


\project douay-rheims


%%% 2371
%%% o-2190
\startChapter[
  title={Chapter 2}
  ]

\Summary{Againſt the Scribes and Phariſees he defendeth firſt his power to
  remit ſinnes in earth, 13.~and his eating with ſinners (as being the
  Phyſicion of ſoules, ſignified in thoſe his miraculous cures vpon
  bodies): 18.~then alſo he defendeth his Diſciples, not hauing as yet
  any faſtes by him preſcribed vnto them, and plucking eares of corne
  vpon the Sabboth: ſignifying withal that he wil change their
  ceremonies.}

%%% 2372
And againe
\CNote{\XRef{Mt.~9,~1.}
\XRef{Lu.~5,~18.}}
he entred into Capharnaum after ſome daies, and it was heard
that he was in the houſe, \V and many came togeather, ſo that there was
no place; no not at the doore; & he ſpake to them the Word. \V And they
came to him bringing one ſick of the palſey, who was caried of foure. \V
And when they could not offer him vnto him for the multitude, they
\LNote{Vncouered}{Such diligence ought to be vſed to bring ſinners to
Chriſt in his Sacraments, as was vſed to procure this man and others, by
 Chriſt, the health of their bodies.}
vncouered the roofe where he was: and opening it they did let downe the
couch wherin the ſick of the palſey lay. \V And when \Sc{Iesvs} had
ſeene
\SNote{Our Lord is moued to be merciful to ſinners by other mens faith
and deſires, and not only by the parties owne meanes alway.}
their faith, he ſaith to the
\LNote{Sick of the palſey}{Such as this man was in body by diſſolution
of his limmes, ſuch alſo was he in ſoule, by the noiſome deſires of the
world occupying his hart, and withdrawing him from al good works.
\Cite{Aug. de Paſter. c.~6. to.~9.}}
ſick of the palſey: Sonne,
\LNote{The ſinnes}{Hereby it appeareth that Chriſt healed this ſick man
 firſt in his ſoule, before he tooke away his bodily infirmity:
\MNote{The Sacramẽts to be called for in ſicknes.}
which
 may be an inſtruction for al men in bodily diſeaſe, firſt to cal for
 the Sacraments, which be medicines of the ſoule. As hereby alſo may be
 gathered that many diſeaſes come for ſinne, and therfore can not be
 healed til the ſinnes be remitted.}
thy ſinnes are forgiuen thee. \V And there were certaine of the Scribes
ſitting there and thinking in their harts: \V Why doth he ſpeake ſo? he
blaſphemeth.
\CNote{\XRef{Iob.~14,~4.}
\XRef{Eſa.~43,~25.}}
Who can forgiue ſinnes but only God? \V Which by and
by \Sc{Iesvs} knowing in his Spirit, that they ſo thought within them
ſelues, ſaith to them: Why thinke you theſe things in your harts? \V
Whether is eaſier, to ſay to the ſick of the palſey: Thy ſinnes are
forgiuen thee; or to ſay: Ariſe, take vp they couch, and walke? \V But
that you may know that
\LNote{The Sonne of man}{As
\MNote{Mã hath power to remit ſinnes.}
Chriſt proueth vnto them, that him ſelf as man, and not as God only,
 hath power to remit ſinnes, by that in al their ſights he was able to
 doe miracles, and make the ſick man ſodenly ariſe; ſo the Apoſtles
 hauing power granted them to doe miracles, though they be not God, may
 in like manner haue authority from God to remit ſinnes, not as God, but
 as Gods miniſters.}
the Sonne of man hath power
\LNote{In earth}{This
\MNote{Chriſt remitteth ſinnes by the Prieſts miniſterie.}
power that the Sonne of man hath to remit ſinnes in earth, was neuer
 taken from him, but dureth ſtil in his Sacraments, and miniſters, by
 whom he remitteth ſinnes in the Church, and not in Heauen only. For
 concerning ſinne, there is one court of conſcience in earth, and an
 other in Heauen, and the iudgement in Heauen foloweth and approueth
 this on earth, as is plaine by the wordes of our Sauiour to Peter
 firſt, & then to al the Apoſtles:
\CNote{\XRef{Mt.~16,~16.}
\XRef{Mt.~18,~18.}}
\Emph{Whatſoeuer you shal bind vpon
 earth, shal be bound in Heauen: whatſoeuer you shal looſe vpon earth,
 shal be looſed in Heauen}: Wherevpõ S.~Hierom ſaith:
\CNote{\Cite{ad Heliod. ep.~1.}}
\Emph{That Prieſts
hauing the keies of the Kingdom of Heauen iudge after a ſort before the day
of iudgement.} And
\Cite{S.~Chryſoſt. li.~3. de Sacerd. paulo poſt princip.}
more at large.}
in earth to forgiue ſinnes (he ſaith to the ſick of the palſey) \V I ſay to
thee, ariſe, take vp thy couch, and goe into thy houſe. \V And forthwith
he aroſe; and taking vp his couch, went his way in the ſight of al, ſo
that al marueled, and glorified God, ſaying: That we neuer ſaw the
like.

\V And he went forth againe to the ſea; and al the multitude came to
him, and he taught them. \V And when he
%%% o-2191
paſſed by,
\CNote{\XRef{Mt.~9,~9.}
\XRef{Lu.~5,~27.}}
he ſaw Leui of Alphæus ſitting at the cuſtome place; and he
ſaith to him: Folow me. And riſing vp he folowed him. \V And it came to
paſſe, as he ſate at meate in his houſe, many Publicans and ſinners did
ſit downe togeather with \Sc{Iesvs} and his Diſciples. For they were
many, who alſo folowed him. \V And the Scribes & the Phariſees ſeeing
that he did eate with Publicans and Sinners, ſaid to his Diſciples: Why
doth your Maiſter eate & drinke with Publicans and
ſinners? \V \Sc{Iesvs} hearing this, ſaith to them: The whole haue not
need of a Phyſicion, but they that are il at eaſe. For I came not to cal
the iuſt, but ſinners.

\V And
\CNote{\XRef{Mt.~9,~11.}
\XRef{Lu.~5,~33.}}
the Diſciples of Iohn and the Phariſees did vſe to faſt: and they
come, and ſay to him: Why doe the Diſciples of Iohn and of the Phariſees
faſt; but thy Diſciples doe not faſt? \V And \Sc{Iesvs} ſaid to them:
Why, can the children of the mariage faſt, as long as the bridegrome is
with them? So long time as they haue the bridegrome with them, they can
not faſt. \V But the daies wil come when the bridegrome ſhal be taken
away from them; and then they ſhal
\SNote{He foretelleth that faſting ſhal be vſed in his Church, no leſſe
then in the old law, or in the time of Iohn the Baptiſt. See
\XRef{Mat. c.~9,~15.}}
faſt in thoſe daies. \V No body ſoweth a peece of raw cloth to an old
garment: otherwiſe he taketh away the new peecing from the old, and
there is made a greater rent. \V And no body putteth new wine into old
bottels: otherwiſe the wine burſteth the bottels, and the wine wil be
ſhed, and the bottels wil be loſt. But new wine muſt be put into new
bottels.

\V And
\CNote{\XRef{Mt.~12,~1.}
\XRef{Lu.~6,~1.}}
it came to paſſe againe when he walked through the corne on the
Sabboths, and his Diſciples began to goe forward and to
\Fix{pulck}{plucke}{obvious typo, fixed in other}
the eares. \V And the Phariſees ſaid to him: Behold, why do they on the
%%% 2373
Sabboths that which is not lawful? \V And he ſaid to them: Did you neuer
read what Dauid did, when he was
\LNote{In neciſsity}{In neceſsity many things be done without ſinne,
which els might not be done, and ſo
\CNote{\Cite{Amb. li.~2. off. c.~28.}}
the very chalices and conſecrated
 iewels and veſſels of the Church, in caſes of neceſsity, are by lawful
 authority turned to profane vſes, which otherwiſe to alienate to a mans
 priuate commoditie is ſacrilege.}
in neceſſitie, and himſelf was an hungred and they that were with
him? \V how
\CNote{\XRef{1.~Re.~21,~6.}}
he entred into the houſe of God vnder Abiathar the high
Prieſt, and did eate the loaues of Propoſition, which it was not lawful
to eate
\CNote{\XRef{Leu.~24,~9.}}
but for the Prieſts, and did giue vnto them which were with
him? \V And he ſaid to them: The Sabboth was made for man, and not man
for the Sabboth. \V Therfore the Sonne of man is
\SNote{The maker of the law may abrogate or diſpenſe when and where for
iuſt cauſe it ſeemeth good to him.}
Lord of the Sabboth alſo.

\stopChapter


\stopcomponent


%%% Local Variables:
%%% mode: TeX
%%% eval: (long-s-mode)
%%% eval: (set-input-method "TeX")
%%% fill-column: 72
%%% eval: (auto-fill-mode)
%%% coding: utf-8-unix
%%% End:
