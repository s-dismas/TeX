%%%%%%%%%%%%%%%%%%%%%%%%%%%%%%%%%%%%%%%%%%%%%%%%%%%%%%%%%%%%%%%%%
%%%%
%%%% The (original) Douay Rheims Bible 
%%%%
%%%% New Testament
%%%% Mark
%%%% Chapter 07
%%%%
%%%%%%%%%%%%%%%%%%%%%%%%%%%%%%%%%%%%%%%%%%%%%%%%%%%%%%%%%%%%%%%%%




\startcomponent chapter-07


\project douay-rheims


%%% 2383
%%% o-2204
\startChapter[
  title={Chapter 7}
  ]

\Summary{The Maſters of Hieruſalem comming ſo farre to carpe him 6.~he
  chargeth with traditions, partly friuolous, 9.~partly alſo contrarie
  to Gods commandements. 14.~And to the People he yealdeth the reaſon of
  that which they carped, 17.~and agayne to his Diſciples, shewing the
  ground of the Iewish washing (to wit, that meats otherwiſe
  %%% 2384
  defile the ſoule) to be falſe. 24.~But by and by among the Gentils, in
  a woman he findeth wonderful faith, vpon her therfore he beſtoweth the
  crumme that she asked, 31.~returning (becauſe the time of the Gentils
  was not yet come) to the Iewes with the lofe: 32.~where he sheweth his
  compaſſion towards mankind ſo deafe & dumme, 36.~and of the People
  is highly magnified.}

And
\CNote{\XRef{Mt.~15,~1.}}
there aſſemble togeather vnto him the Phariſees and certaine of the
Scribes, comming from Hieruſalem. \V And when they had ſeen certaine of
his Diſciples eate bread with
\LNote{Common}{Common and vncleane is al one. For the Iewes were
commanded by
the Law to eate certaine kinds of meats only, and not al indifferently:
and becauſe theſe were ſeparated from other meats, and as it were
ſanctified to their vſe, they called the other common and profane: and
becauſe the Law calleth thoſe cleane and theſe vncleane, thereof it is,
that vncleane and common is al one, as in this Chapter often, and
\XRef{Act.~10.}}
cõmon hãds, that is, not waſhed, they blamed them. \V For the Phariſees,
and al the Iewes, vnles they often waſh their hands, eate not, holding
the tradition of the Ancients: \V And from the market, vnles they be
waſhed, they eate not: and many other things there be that were
deliuered vnto them to obſerue, the waſhing of cups and cruſes, and of
braſen veſſels & beds. \V And the Phariſees and Scribes asked him: Why
doe not thy Diſciples walke according to the tradition of the Ancients,
but they eate bread with common hands? \V But he anſwering, ſaid to
them: Wel did Eſay Prophecie of your Hypocrites, as it is
written:
\CNote{\XRef{Eſa.~29.~13.}}
\Emph{This People honoureth me
\SNote{They that ſay wel, or teach & preach wel, or haue Chriſt & his
word in their mouth, & liue naughtily, be touched in this place.}
with their lips, but their hart is farre from me. \V And in vaine doe
they worship me, teaching doctrines
\LNote{Precepts of men}{Mens
\MNote{Commandements of men.}
ordinances which be repugnant to Gods commandements, be here condemned
as al obſeruations not edifying nor profitable to the fulfilling of
Gods commandements, be vaine and ſuperfluous: as many obſeruations of
the Phariſees were then, and the like traditions of Heretikes be
now, for howſoeuer they bragge of Scriptures, al their manner of
adminiſtration and miniſterie is their owne tradition and inuention
without al Scripture and warrant of Gods word.
\MNote{Traditions.}
But the traditions of the
Apoſtles & Ancients, and al the precepts of holy Church we are cõmanded
to keep, as things not preſcribed by man but by the Holy Ghoſt
\XRef{Act.~15.~28.~41.}
\XRef{2.~Theſal. 2.~13.}}
precepts of men.}
%%% o-2205
\V For leauing the cõmandement of God, you hold the traditions of men,
the waſhings of cruſes and cups: & many other things you doe like to
theſe. \V And he ſaid to them, wel doe you fruſtrate the precept of God,
that you may obſerue your owne tradition. \V For Moyſes
ſaid:
\CNote{\XRef{Exo.~20,~12.}
\XRef{Leu.~20,~9.}}
\Emph{Honour thy father and thy mother}; and, \Emph{He that shal
curſe father or mother, dying let him dye}. \V But you ſay: If a man ſay
to father or mother, \Emph{Corban} (which is a
\LNote{Guift}{To
\MNote{Dutie to parents.}
giue to the Church or Altar is not forbidden, but the forſaking of a
mans parents in their neceſsitie, pretending or excuſing the matter vpon
his giuing that which should relieue them, to God or the Altar, that is
impious and vnnatural. And theſe Phariſees teaching children ſo to
neglect their duties to their parents, did wickedly.}
guift) whatſoeuer proceedeth from me, ſhal profit thee: \V And further
you ſuffer him not to doe ought for his father or mother, \V defeating
the Word of God for your owne tradition which you haue giuen forth. And
many other things of this ſort you doe.

\V And calling againe the multitude vnto him, he ſaid to thẽ: Heare me
al you, and vnderſtand. \V
\LNote{Nothing entring into a man}{As
\MNote{Abſtinence from certaine meats.}
theſe wordes of our Sauiour doe not import, that the Iewes
then might haue eaten of thoſe meates which God forbade them: no more
doe they now, that we Chriſtians may eate of meates which the Church
forbiddeth vs. And yet both then and now al meates are cleane, and
nothing entring into a man, defileth a man. For neither they then, nor
we now abſtaine, for that any meates are of their nature abominable, or
defile the eaters, but they for ſignification, we for obedience and
chaſtiſement of our bodies.}
Nothing is without a man entring into him, that can defile him. But the
things that proceed from a man thoſe are they that make a man
\SNote{See the firſt annotatiõ vpõ this chapter.}
common. \V If any man haue eares to heare, let him heare. \V And when he
was entred into the houſe from the multitude, his Diſciples asked him
the parable. \V And he ſaith to them: So are you alſo vnskilful?
Vnderſtand you not that euery thing from without, entring into a man,
can not make him common: \V becauſe it entreth not into his hart, but
goeth into the belly, and is caſt out into the priuy, purging al the
meates? \V But he ſaid that the things which come forth from a man,
they make a man common. \V For from within out of the hart of men
proceed euil cogitations, aduouteries, fornications, murders, \V thefts,
auarices, wickedneſſe, guile, impudicities, an euil eye, blaſphemie,
pride, fooliſhnes. \V Al theſe euils proceed from within, and make a man
common.

\V And
\CNote{\XRef{Mt.~15,~21.}}
riſing from thence he went into the coaſts of Tyre and Sidon: and
entring into a houſe, he would that no man ſhould know, and he could not
be hid. \V For a woman immediatly as ſhe heard of him,
%%% 2385
whoſe daughter had an vncleane Spirit, entred in, and fel downe at his
feet. \V For the woman was a Gentile, a Syrophænician borne. And ſhe
beſought him that he would caſt forth the Diuel out of her daughter. \V
Who ſaid to her: Suffer firſt the children to be filled. For it is not
good to take the childrens bread, and caſt it to the dogs. \V But ſhe
anſwered, and ſaid to him: Yea Lord; for the whelpes alſo eate vnder the
table of the crummes
%%% 0-2206
of the children. \V And he ſaid to her: For this ſaying goe thy way, the
Diuel is gone out of thy daughter. \V And when ſhe was departed into her
houſe, ſhe found the maid lying vpon the bed, and the Diuel gone out.

\V And againe going out of the coaſts
\Var{of Tyre, he came by Sidon}{of Tyre and Sidon, he came}
to the ſea of Galilee through the middes of the coaſts of Decapolis. \V
And they bring to him one deafe and dumme; and they beſought him that he
would impoſe his hand vpon him. \V And taking him from the multitude
apart, he put his fingers into his eares, and
\LNote{Spitting}{Not
\MNote{Chriſt's ſpittle worketh miracles.}
only by Chriſts word and wil, but alſo by ceremonie and by application
of external creatures which be holy, miracles are wrought; as by Chriſts
ſpittle, which was not part of his Perſon, being a ſuperfluity of his
body, but yet moſt holy
\Cite{Theophyl. in 7.~Marci.}}
ſpitting, touched his tongue; \V And looking vp vnto Heauen, he groned,
and ſaid to him:
\LNote{Ephphetha}{The
\MNote{Exorciſmes & other ceremonies in Baptiſme.}
Church doth moſt godly imitate and vſe theſe very wordes and ceremonies
of our Sauiour in the Exorciſmes before Baptiſme, to the healing of
their ſoules that are to be baptized, as Chriſt here healed the bodily
infirmitie, and the diſeaſe of the ſoule togeather.
\Cite{Ambros. li. de Sacramen. c.~1.}}
\Emph{Ephpheta}, which is, \Emph{Be thou opened}. \V And immediatly his
eares were opened, and the ſtring of his tongue was looſed, and he ſpake
right. \V And he commanded them not tel any body. But how much he
commanded them, ſo much the more a great deale did they publiſh it. \V
And ſo much the more did they wonder, ſaying: He hath done al things
wel; he hath made both the deafe to heare, and the dumme to ſpeake.

\stopChapter


\stopcomponent


%%% Local Variables:
%%% mode: TeX
%%% eval: (long-s-mode)
%%% eval: (set-input-method "TeX")
%%% fill-column: 72
%%% eval: (auto-fill-mode)
%%% coding: utf-8-unix
%%% End:
