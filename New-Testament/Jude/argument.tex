%%%%%%%%%%%%%%%%%%%%%%%%%%%%%%%%%%%%%%%%%%%%%%%%%%%%%%%%%%%%%%%%%
%%%%
%%%% The (original) Douay Rheims Bible 
%%%%
%%%% New Testament
%%%% Jude
%%%% Argument
%%%%
%%%%%%%%%%%%%%%%%%%%%%%%%%%%%%%%%%%%%%%%%%%%%%%%%%%%%%%%%%%%%%%%%




\startcomponent argument


\project douay-rheims


%%% 2930
%%% o-2792
\startArgument[
  title={\Sc{The Argvment of the Epistle of S.~Ivde.}},
  marking={argument of the Epiſtle of S.~Iude.}
  ]

In the Ghoſpel
\CNote{\XRef{Mt.~13.}}
theſe are called \L{Fratres Ieſu}, \Emph{the Brethren of Ieſus}: Iames,
and Ioſeph, and Simon, and Iude.
\CNote{\XRef{Mt.~10.}}
Their father is called Alphæus, where Iames is termed, \Emph{Iames of
Alphæus}:
\CNote{\XRef{Mat.~13.}}
and their mother, \L{Maria Iacobi minoris}, \Emph{Marie the mother of
Iames the yonger & of Ioſeph}. Which Marie in
\CNote{\XRef{Ioh.~19.}}
another place being called \L{Maris Cleophæ}, we perceiue their father
was named both Alphæus and alſo Cleophas. And that this Cleophas was
brother to Ioſeph our Ladies husband,
\CNote{\Cite{Euſeb. hiſt. li.~3. c.~10.}}
Hegeſippus telleth vs. Therfore becauſe Ioſeph was called the father of
Chriſt, his brothers children were called the Brethren, that is
(according to the cuſtom of the ſcripture alſo) the kinſmen of our Lord;
& not becauſe they were the children of Ioſeph himſelf by another wife,
much leſſe (as Heluidius the Heretike did blaſpheme) by our B.~Ladie the
perpetual Virgin \Sc{Marie}. Howbeit ſome good Authours ſay, that their
mother Marie was the natural ſiſter of our Ladie, and that therfore they
are called, \L{Fratres Domini}, \Emph{the Brethren of our Lord}.

Howſoeuer that be,
\CNote{\XRef{Luc.~6.}
\XRef{Mt.~10.}}
three of them are reckened among the 12.~Apoſtles, Iames, and Simon
Cananæus, and Iude. Yea and that they were ſome-what more then Apoſtles,
though leſſe then Peter, S.~Paul ſignifieth, where he ſaith ſpeaking of
himſelf and Barnabas: \Emph{As alſo the other Apoſtles, and the Brethren
of our Lord, and Cephas.}
\XRef{1.~Cor.~9.}

And as S.~Luke calleth this Iude,
%%% !!! XRef?
\Emph{Iude of Iames}, ſo he calleth himſelf in this Epiſtle of his,
\Emph{Iude the ſeruant of Ieſus Chriſt, and the brother of
Iames}. S.~Matthew and S.~Marke doe cal him
\CNote{\XRef{Mt.~10.}
\XRef{Mar.~3.}}
\Emph{Thaddæus}, as \Emph{Lebbæus} alſo in the Greeke. His feaſt and his
brother Simons together, the Church keepeth
Octob.~28. called \Emph{Simon and Iudes day}.

His Epiſtle is an Inuectiue againſt al heretikes (as it were a
Commentarie of 2.~Pet.~2.) and namely (as
%%% !!! Fix XRef
\CNote{\XRef{pag.~342, 396.}}
S.~Aug hath told vs) againſt thoſe, which miſconſtred S.~Paules Epiſtles
and held \Emph{Only faith}, whom he calleth therfore, \Emph{Men that
transferre or peruert the grace of God into riotouſnes},
\XRef{v.~4.}
exhorting Catholikes to be conſtant and vnmoueable from their old faith,
and to contend for the keeping thereof,
\XRef{v.~3.}
and
\XRef{v.~20.}
For heretikes (ſaith he) \Emph{ſegregate themſelues} from the Church and
from her faith.
\XRef{v.~19.}


\stopArgument


\stopcomponent


%%% Local Variables:
%%% mode: TeX
%%% eval: (long-s-mode)
%%% eval: (set-input-method "TeX")
%%% fill-column: 72
%%% eval: (auto-fill-mode)
%%% coding: utf-8-unix
%%% End:
