%%%%%%%%%%%%%%%%%%%%%%%%%%%%%%%%%%%%%%%%%%%%%%%%%%%%%%%%%%%%%%%%%
%%%%
%%%% The (original) Douay Rheims Bible 
%%%%
%%%% New Testament
%%%% Two Thessalonians
%%%% Chapter 02
%%%%
%%%%%%%%%%%%%%%%%%%%%%%%%%%%%%%%%%%%%%%%%%%%%%%%%%%%%%%%%%%%%%%%%




\startcomponent chapter-02


\project douay-rheims


%%% 2793
%%% o-2654
\startChapter[
  title={Chapter 2}
  ]

\Summary{He requireth them, in no caſe to thinke that Domes-day is at
  hand, 3.~repeating vnto them that there muſt before come firſt a
  reuolt, ſecondly the reuelation alſo of Antichriſt himſelf in perſon,
  and that Antichriſt shal not permit any God to be worshipped but only
  himſelf: that alſo with his lying wonders he shal winne to him the
  incredulous Iewes. But Chriſt shal come then immediately in maieſtie,
  and deſtroy him and his. 13.~Therfore he thanketh God for the faith of
  the Theſſalonians, 15.~and biddeth them ſtick to his Traditions both
  written and vnwritten, and praieth God to confirme them.}

And we deſire you, Brethren, by the comming of our Lord \Sc{Iesvs}
Chriſt, & of our congregation into him; \V that you be not eaſily moued
from your ſenſe, nor be terrified, neither by ſpirit, nor by word, nor
by epiſtle as ſent by vs,
\LNote{As though the day.}{The
\MNote{The day of iudgement vncertaine, & to be left to God's ſecrets.}
curioſitie of man fed by Satans deceits, hath ſought to know and to giue
out to the world, ſuch things as God wil not impart to him, nor be
neceſſarie or profitable for him to know: ſo farre, that both in the
Apoſtles daies and often afterward, ſome haue feined reuelations, ſome
falſely gathered out of the
Scriptures,
\Fix{Sriptures,}{Scriptures,}{obvious typo, fixed in other}
ſome preſumed to calculate and
coniect by the ſtarres, and giuen forth to the world a certaine time of
Chriſtes comming to iudgement. Al which ſeducers be here noted in the
perſon of ſome that were about to deceiue the Theſſalonians therin. And
S.~Auguſtin (in his
\Cite{80.~Epiſtle ad Heſychium)}
proueth that no man can be aſſured by the Scriptures of the day, yeare,
or Age that the end of the world or the ſecond Aduent ſhal be.}
as though the day of our Lord were at hand. \V Let no man ſeduce you by
any meanes, for
\LNote{Vnles there come a reuolt firſt.}{Though
\MNote{Two ſpecial ſignes before the later day: a general apoſtaſie, and
the comming of Antichriſt.}
we can not be aſſured of the moment, houre, or any certaine time of our
Lordes comming, yet he warranteth vs that it wil not be before certaine
things be fulfilled, which muſt come to paſſe by the courſe of God's
prouidence and permiſſion before, which are diuers, wherof in other
places of Scriptures we be fore-warned. Here he warneth vs, of two
ſpecially, of a reuolt, defection or an apoſtaſie, and of the comming or
reuelation of Antichriſt. Which two pertaine in effect both to one,
either depending of the other, & ſhal fal (as it may be thought) neer
together and therfore S.~Auguſtin maketh them but one thing.

This apoſtaſie or reuolt, by the iudgement in a manner of al ancient
Writers, is the general forſaking & fal of the Romane Empire. So Tertullian
\Cite{li. de reſur. carnis.}
S.~Hierom
\Cite{q.~11. ad Algaſiam.}
S.~Chryſoſtom
\Cite{ho.~4.}
and S.~Ambroſe
\Cite{vpon this place.}
S.~Auguſtin
\Cite{De Ciuit. Dei li.~10. c.~19.}
\MNote{The heretikes interpretation of this apoſtaſie, & their
condemning of the Fathers.}
Al which Fathers and the reſt
\CNote{\Cite{Caluin in hunc locum.}}
Caluin preſumptuouſly condemneth of errour and follie herein, for that
their expoſition agreeth not with his & his fellowes blaſphemous fiction
that the Pope ſhould be Antichriſt. To eſtabliſh which falſe impietie,
they interpret this reuolt or apoſtaſie to be a general reuolt of the
viſible Church from God, whoſe houſe or building (they ſay) was ſodenly
deſtroied and lay many yeares ruined, and ruled only by Satan and
Antichriſt. So ſaith the foreſaid Arch-heretikes here: though for the
aduãtage of his defence & as the matter els-where requireth, he ſeemeth
(as al their faſhion is) to ſpeake in other places quite contrarie: but
with ſuch colour and colluſion of words, that neither other men nor
himſelf can tel what he would haue or ſay. And his Fathers Wicleffe and
Luther, his fellowes and followers Illyricus, Beza, and the reſt, are
(for the time of the Churches falling from Chriſt) ſo various among
themſelues, and ſo contrarie to him, that it is horrible to ſee their
confuſion, and a pitieful caſe that any reaſonable man wil follow ſuch
companions to euident perdition.

But
\MNote{There can be no apoſtaſie of the viſible Church from God.}
concerning this errour & falſhood of the Churches defection or reuolt,
it is refuted ſufficiently by S.~Auguſtin againſt the Donatiſtes in many
places. Where he proueth that the Church ſhal not faile to the worlds
end, no not in the time of Antichriſt: affirming them to deny Chriſt &
to robbe him of his glorie & inheritance bought with his bloud, which
teach that the Church may faile or perish.
\Cite{Li. de vnit. Ec. c.~12,~13.}
\Cite{De Ciuit. li.~20. c.~8.}
\Cite{In Pſal.~85. ad illud.}
\Cite{To ſolus Deus magnus.}
\Cite{Pſ.~70. Conc.~2.}
\Cite{Pſal.~60.}
\Cite{De vtil. cred. c.~8.}
S.~Hierom refuteth the ſame wicked Hereſie in the
\CNote{\Cite{Dial. adu. Lucifer. c.~6.}}
Luciferians, prouing againſt them, that they make God ſubiect to the
Diuel, and a poore miſerable Chriſt, that imagine the Church his body
may either periſh or be driuen to any corner of the world. Both of them
anſwer to the Heretikes arguments grounded on Scriptures falſely
vnderſtood, which were too long here to rehearſe. It is enough for the
Chriſtian Reader to know, that it is an old deceit and excuſe of al
Heretikes and Schiſmatikes, for defence of their forſaking God's Church,
that the Church is periſhed, or remaineth hidden, or in themſelues only
& in thoſe places where they & their followers dwel: to know alſo, that
this is reproued by the holy Doctours of the primitiue Church, and that
it is againſt Chriſtes honour, power, prouidence, and promiſe.

If the Aduerſaries had ſaid that this reuolt which the Apoſtle
fore-telleth ſhal come before the worlds end, is meant of great numbers
of Heretikes and Apoſtates reuolting from the Church, they had ſaid
truth of themſelues and ſuch others, whom
\CNote{\XRef{1.~Io.~2. v.~18.}}
S.~Iohn calleth Antichriſtes.
\MNote{It is very like, the Apoſtle ſpeaketh of a great apoſtaſie from
the See of Rome, & from moſt articles of the Catholike faith.}
And it is very like (be it ſpoken vnder the correction of God's Church
and al learned Catholikes) that this great defection or reuolt ſhal not
be only from the Romane Empire, but ſpecially from the Romane Church,
and withal from moſt points of Chriſtian religion: not that the
Catholike Chriſtians, either in the time of Antichriſt or before, ſhal
refuſe to obey the ſame; but for that neer to the time of Antichriſt and
the conſummation of the world, there is like to be a great reuolt of
Kingdoms, peoples,  and Prouinces from the open external obedience and
communion therof. Which reuolt hauing been begun and continued by
Heretikes of diuers Ages, reſiſting & hating the Seat of Peter (which
they called \L{cathedram peſtilentie}, \Emph{the chaire of peſtilence},
\CNote{\Cite{li.~2. cont. lit. Petil. 6,~52.}}
in S.~Auguſtines daies) becauſe it is Chriſtes fort erected againſt
Hel-gates and al Heretikes, and being now wonderfully increaſed by theſe
of our daies the next precurſours of Antichriſt, as it may ſeeme, ſhal
be fully atchieued a little before the end of the world by Antichriſt
himſelf. Though euen then alſo, when for the few daies of Antichriſtes
reigne the external ſtate of the Romane Church and publike entercourſe
of the faithful with the ſame may ceaſe, yet the due honour and
obedience of the Chriſtians toward it, and communion in hart with it,
and practiſe therof in ſecret, & open confeſſing therof if occaſion
require, ſhal not ceaſe, no more then it doth now in the Chriſtiãs of
Cypres & other places where open entercourſe is forbidden.

This
\MNote{The wonderful prouidence of God in preſeruing the See of Rome
more then al other States, notwithſtanding manifold dangers and
ſcandals.}
is certaine and wonderful in al wiſe mens eyes, & muſt needs be of God's
prouidence and a ſingular prerogatiue, that this Seat of Peter ſtandeth,
when al other Apoſtolike Sees be gone: that it ſtood there for certaine
Ages together with the ſecular Seat of the Empire: that the Popes ſtood
without wealth, power, or humane defenſe, the Emperours knowing,
willing, & ſeeking to deſtroy them, and putting to the ſword aboue
thirtie of them one after another, yea and being as much afraid of them
as if they had been \L{amuli Imperij}, \Emph{Comptetours of their
Empire}, as S.~Cyprian noteth
\Cite{(epiſt.~52. ad Antonianum num.~3.)}
of S.~Cornelius Pope in his daies, & Decius then Emperour: againe, that
the Emperours afterward yealded vp the citie vnto them, continuing for
al that in the Emperial dignitie ſtil: that the Succeſſours of thoſe
that perſecuted them, laid downe their crownes before their Seat and
ſepulchers honouring the very memories & Relikes of the poore men whom
their Predeceſſours killed: that now wel-neer theſe 1600~yeares this
Seat ſtandeth, as at the beginning in continual miſerie, ſo now of long
time for the moſt part in proſperitie, without al mutation in effect, as
no other Kingdom or State in the world hath done, euery one of them in
the ſaid ſpace being manifoldly altered. It ſtandeth (we ſay) al this
while (to vſe S.~Auguſtines words
\Cite{de vtil. cred. c.~17.)}
\L{Fruſtra circumlatrantibus Hæreticis}, \Emph{the Heretikes in vaine
barking about it}, not the firſt Heathen Emperours, not the Gothes and
Vandals, not the Turke, not any ſacks or maſſakers by Alaricus,
Genſericus, Attila, Borbon, and others; not the emulation of ſecular
Princes, were they Kings or Emperours, not the Popes owne diuiſions
among themſelues & manifold difficulties and dangers in their elections,
not the great vices which haue been noted in ſome of their perſons, not
al theſe nor any other endeauour or ſcandal could yet preuaile againſt the
See of Rome, nor is euer like to preuaile til the end of the world draw
neer, at which time this reuolt (here ſpoken of by the Apoſtle) may be
in ſuch ſort as is ſaid before, and more ſhal be ſaid in the Annotatiõs
next following.}
vnleſſe there come
\TNote{\G{ἀποστασία}}
a reuolt firſt, &
\LNote{The man of ſinne.}{There
\MNote{Many Antichriſts, as fore-runners of the great Antichriſt.}
were many euen in the Apoſtles time (as we ſee by the
\XRef{4.~Chapter of S.~Iohn's firſt epiſtle,}
and in the writings of the ancient Fathers) that were fore-runners of
Antichriſt, & for impugning Chriſtes truth & Church were called
Antichriſtes, whether they did it by force and open perſecution, as Nero
& others either Heathen or Heretical Emperours did, or by falſe teaching
& other deceits, as the Heretikes of al Ages. In which common and vulgar
acception S.~Hierom ſaith, al belonged to Antichriſt that were not of
the communion of Damaſus then Pope of Rome.
\Cite{Hieor. ep.~57. ad Damaſ.}
and in another place, al that haue new names after the peculiar calling
of Heretikes; as Arians, Donatiſtes, (and as we ſay now, Caluiniſtes,
Zuinglians, &c.) al ſuch (ſaith he) be Antichriſtes.
\Cite{Dial. cont. Lucifer. c.~9.}
Yea theſe later of our time much more then any of the former, for diuers
cauſes which ſhal afterward be ſet downe.
\MNote{The great Antichriſt ſhal be one ſpecial and notorious man.}
Neuertheleſſe they nor none of them are that great Aduerſarie, enemie,
and impugner of Chriſt, which is by a peculiar diſtinction and ſpecial
ſignification named,
%%% !!! The second and third are out of order? Fix this and put each
%%% with the correct English word?
\TNote{\G{ὁ ἀντίχριστος}

\G{ὁ ὑιὸς ἀπωλείας}

\G{ὁ ἀvθρωπος ἁμαρτίας}

\G{ὁ ἀvτικείμενος}}
\Emph{the Antichriſt},
\XRef{1.~Io.~2.}
and \Emph{the man of ſinne, the ſonne of perdition, the Aduerſarie},
deſcribed here and els where, to oppoſe himſelf directly againſt God and
our Lord \Sc{Iesvs Christ}. The Heathen Emperours were many,
\Fix{Turks many, Heretikes haue been and more are many: be many,
Heretikes haue been and now are many:}
{Turks be many, Heretikes haue been and now are many:}
{obvious typo, fixed in other}
Therfore they can not be that one great Antichriſt which here is ſpoken
of, and which by
%%% !!! Or should all that Greek go here?
the article alwaies added in the Greek, is ſignified to
be one ſpecial and ſingular man: as his peculiar & direct oppoſition to
Chriſt's perſon in the
\XRef{5.~chapter of S.~Iohn's Ghoſpel v.~43.}
the inſinuation of the particular ſtock and tribe wherof he should be
borne, to wit, of the Iewes (for of them he shal be receiued as their
Meſſias
\XRef{Io.~5. v.~43.)}
\CNote{\XRef{Gen.~49,~17.}}
and of the tribe of Dan.
\Cite{Iren. li.~5.}
\Cite{Hierom. com. in c.~11. Dan.}
\Cite{Auguſt. q. in Ioſ. q.~21.}
the note of his proper name
\XRef{Apoc.~13;}
the time of his appearing ſo neer the worlds end; his short reigne, his
ſingular waſt and deſtruction of God's honour and al religion, his
feined miracles, the figures of him in the Prophets and Scriptures of
the new & old Teſtament: al theſe & many other arguments proue him to be
but one ſpecial notorious Aduerſarie in the higheſt degree, vnto whom al
other perſecutours, Heretikes, Atheiſtes, and wicked enemies of Chriſt
and his Church, are but members and ſeruants.

And this is the moſt common ſentence alſo of al ancient Fathers. Only
Heretikes make no doubt but Antichriſt is a whole order or ſucceſſion of
men. Which they hold againſt the former euident Scriptures and reaſons,
only to eſtablish their foolish and wicked paradoxe, that Chriſtes
cheefe Miniſter is Antichriſt, yea the whole order.
\CNote{\Cite{Beza in hoc cap.}}
\MNote{The Caluiniſts place Antichriſt in the See of Rome in S.~Paules
daies.}
Wherin Beza ſpecially pricketh ſo high, that he maketh Antichriſt (euen
this great Antichriſt) to haue been in S.~Paules daies, though he was
not open to the world. Who it should be (except he meane S.~Peter,
becauſe he was the firſt of the order of Popes,) God knoweth. And ſure it
is, except he were Antichriſt, neither the whole order, nor any of the
order can be Antichriſt, being al his lawful Succeſſours both in
dignitie & alſo in truth of Chriſtes religion. Neither can al the
Heretikes aliue proue that they or any of them vſed any other regiment,
or iuriſdiction Eccleſiaſtical in the Church, or forced the people to
any other faith or worship of God, then Peter himſelf did preach &
plant. Therfore if the reſt be Antichriſt, let Beza boldly ſay that
S.~Peter was ſo alſo, and that diuers of the ancient Catholike Fathers
did ſerue and worke (though vnawares) towards the ſetting vp of the
great Antichriſt: for ſo doth that blaſphemous pen boldly write in his
Annotations vpon theſe words:
\CNote{\Cite{Againſt D.~Sanders rocke pag.~248 & pag.~278. }}
\MNote{They make S.~Leo & S.~Gregorie, great furtherers of Antichriſtes
pride.}
\Emph{As for Leo and Gregorie Bishops of Rome, although they were not
come to the ful pride of Antichriſt, yet the myſterie of iniquitie
hauing wrought in that Seat neer fiue or ſixe hundred yeares before
them, and then greatly increaſed, they were deceiued with the long
continuance of errour.} Thus writeth a malapert ſcholer of that impudent
ſchoole, placing the myſterie of Antichriſt as working in the See of
Rome euen in S.~Peters time, and making theſe two holy Fathers great
workers and furtherers of the ſame. Whereas another English Rabbin
doubted not at Paules croſſe to ſpeake of the ſelf-ſame Fathers as great
Doctours and Patrones of their new Ghoſpel, thus: 
\CNote{\Cite{Iuel.}}
\Emph{O Gregorie, ô Leo, if we be deceiued, you haue deceiued vs.}
Wherof we giue the good Chriſtian Reader warning, more diligently to
beware of ſuch damnable bookes and Maiſters, carying many vnaduiſed
people to perdition.}
the man of ſinne be reuealed, the ſonne of perdition, \V which is an
aduerſarie & is
\LNote{Extolled.}{The
\MNote{Antichriſt shal ſuffer no worship or adoration, but of himſelf
only: therfore the Pope can not be Antichriſt.}
great Antichriſt which muſt come neer the worldes end, shal abolish the
publike exerciſe of al other religions true and falſe, & pul downe both
the B.~Sacrament of the altar, wherin conſiſteth ſpecially the worship of
the true God, & alſo al Idols of the Gentils, & Sacrifices of the Iewes:
generally, al kind of religious
\Fix{whorship,}{worship,}{obvious typo, fixed in other}
ſauing that which muſt be done to himſelf alone. Which was partly
prefigure in ſuch Kings as published that no God nor man but themſelues
should be praied vnto for certaine daies, as
\CNote{\XRef{Dan. c.~6.}}
Darius and ſuch like. How can the Proteſtants then for shame & without
euident contradiction, auouch the Pope to be Antichriſt, who (as we
ſay) honoureth Chriſt the true God with al his power, or (as they ſay)
honoureth Idols, and chalengeth no diuine honour to himſelf, much leſſe
to
\Fix{himſef}{himſelf}{obvious typo, fixed in other}
only, as Antichriſt shal doe? He humbly praieth to God, & lowly
kneeleth downe in euery Church at diuers altars erected to God in the
memories of his Saints, & praieth to them. He ſayeth or heareth Maſſe
daily with al deuotion: he confeſſeth his ſinnes to a Prieſt as other
poore men doe; he adoreth the holy Euchariſt which Chriſt affirmed to be
his owne body, the Heretikes cal it an Idol (no maruel if they make the
Pope his Vicar Antichriſt, when they make Chriſt himſelf an Idol:)
theſe religious duties doth the Pope, wheras Antichriſt shal worship
none, nor pray to any, at the leaſt openly.}
extolled
\SNote{How then can the Pope be Antichriſt, as the Heretikes fondly
blaſpheme, who is ſo farre from being exalted aboue God, that he praieth
moſt humbly not only to Chriſt but alſo to his B.~mother and al his
Saints.}
aboue al that is called God, or that is worshipped, ſo that he ſitteth
\LNote{In the temple.}{Moſt
\MNote{In what temple Antichriſt shal ſit.}
ancient Writers expound this of the Temple in Hieruſalem, which they
thinke Antichriſt shal build vp againe, as being of the Iewes ſtock, & to
be acknowledged of that obſtinate people (according to our Sauiours
prophecie
\XRef{Io.~5.)}
for their expected & promiſed Meſſias,
\Cite{Iren. li.~5. in fine.}
\Cite{Hyppolit. de conſum. mundi.}
\Cite{Cyril. Hieroſ. Catech.~15.}
\Cite{Author ep. imp. ho.~49. in Mat.}
See 
\Cite{S.~Hierom in 11.~Dan.}
\Cite{Grego. li.~13. Moral. c.~11.}
Not that he shal ſuffer them to worship God by their old manner of
Sacrifices, (al which he wil either abolish, or conuert to the only
adoration of himſelf; though at the firſt to apply himſelf to the Iewes,
he may perhaps be circumciſed & keep ſome part of the law)
for it is here ſaid that he shal ſit in the Temple of God, that is, he
shal be adored there by Sacrifice and diuine honour, the name & worship
of the true God wholy defaced.
\MNote{The abomination of deſolation conſiſteth cheefely in aboliſhing
the Sacrifice of the Altar.}
And this they thinke to be \Emph{the abomination of deſolation}
\CNote{\XRef{Dan.~9.}}
fore told by Daniel,
\CNote{\XRef{Mat.~14.}}
mentioned by our Sauiour,
\CNote{\XRef{1.~Mach.~1.}}
prefigured and reſembled by Antiochus and others, that defaced the
worſhip of the true God by prophanation of that Temple, ſpecially by
abrogating the daily Sacrifice, which was a figure of the only Sacrifice
and continual oblation of Chriſtes holy body & bloud in the Church, as
the aboliſhing of that, was a figure of the aboliſhing of this, which
ſhal be done principally & moſt vniuerſally by Antichriſt himſelf (as
now in part by his fore-runners) through-out al Nations & Churches of
the world (though then alſo Maſſe may be had in ſecret, as it is now in
Nations where the ſecular force of ſome Princes prohibiteth it to be
ſayd openly.) For although he may haue his principal ſeat & honour in
the Temple and citie of Hieruſalem, yet he ſhal rule ouer the whole
world, and ſpecially prohibit that principal worſhip inſtituted by
Chriſt in his Sacraments, as being the proper Aduerſarie of Chriſtes
perſon, name, law, and Church. The prophanation and deſolation of which
Church by taking away the Sacrifice of the altar, is the proper
abomination of deſolation, and the worke of Antichriſt only.

S.~Auguſtin therfore
\Cite{li.~20. de ciuit. c.~19.}
and S.~Hierom
\Cite{c.~11. ad Algaſiam,}
doe thinke, that this ſitting of Antichriſt in the temple, doth ſignifie
his ſitting in the Church of Chriſt, rather then in Salomons temple.
\MNote{How Antichriſt ſhal ſit in the Church.}
Not as though he ſhould be a cheefe member of the Church of Chriſt, or a
ſpecial part of his body myſtical, and be Antichriſt and yet withal
continuing within the Church of Chriſt, as the Heretikes feine, to make
the Pope Antichriſt (whereby they plainely confeſſe and agniſe that the
Pope is a member of the Church, & 
\CNote{\Cite{Beza.}}
\L{in ipſo ſinu Eccleſiæ}, \Emph{in the very boſome of the Church}, ſay
they:)
\MNote{Neither Antichriſt nor his precurſours, are members of the
Church.}
for that is ridiculous, that al Heretikes whom S.~Iohn calleth
Antichriſtes as his precurſours, ſhould goe out of the Church, and the
great Antichriſt himſelf ſhould be of the Church, & in the Church, &
continue in the ſame. And yet to them that make the whole Church to
reuolt from God, this is no abſurditie. But the truth is, that this
Antichriſtian reuolt here ſpoken of, is from the Catholike Church: and
Antichriſt, if he euer were of or in the Church, ſhal be an Apoſtata and
a renegate out of the Church; & ſhal vſurp vpon it by tyrannie, and by
chalenging worſhip, religion, and gouernement thereof, ſo that himſelf
ſhal be adored in al the Churches of the world which he liſt to leaue
ſtanding for his honour. And this is to ſit in the temple, or
\TNote{\G{εἰς τὸν ναὸν}}
againſt the Temple of God, as ſome interpret. If any Pope did euer this,
or ſhal doe, then let the Aduerſaries cal him Antichriſt.

And
\MNote{Antichriſt (by interpretation, \Emph{One againſt Chriſt}) why ſo
called.}
let the good Reader obſerue, that there be two ſpecial cauſes why this
great man of ſinne is called Antichriſt. The one is, for impugning Chriſtes
kingdom in earth, that is to ſay, his ſpiritual regiment which he
conſtituted and appointed in his Church, and the forme of gouernement
ordained therein, applying al to himſelf by ſingular tyrannie and
vſurpation, in which kind S.~Athanaſius
\Cite{(ep. ad Solit. vit. degentes)}
is bold to cal the Emperour Conſtantius being an Arian Heretike,
Antichriſt, for making himſelf \L{Principem Epiſcoporum}, \Emph{Prince
ouer the Bishops & Preſident of Eccleſiaſtical iudgements, &c.} The
other cauſe is for impugning Chriſtes Prieſthood, which is only or moſt
properly exerciſed in earth by the Sacrifice of the holy Maſſe,
inſtituted for the commemoration of his death, & for the external
exhibition of godly honour to the B.~Trinitie, which kind of external
worſhip by Sacrifice no lawful people of God euer lacked.
\MNote{Proteſtants and Caluiniſts the neer fore runners of Antichriſt.}
And by theſe two things you may eaſily perceiue, that the Heretikes of
theſe daies doe more properly and neely prepare the way to Antichriſt
and to extreme deſolation, then euer any before: their ſpecial hereſie
being againſt the ſpiritual Primacie of Popes and Biſhops, & againſt the
Sacrifice of the altar, in which two the ſoueraigntie of Chriſt in earth
conſiſteth.}
in the Temple of God, ſhewing himſelf as though he were God. \V Remember
you not, that when I was yet with you, I told you theſe things? \V And
now
\LNote{What letteth.}{S.~Auguſtin
\MNote{S.~Auguſtin's humilitie in interpreting the Scriptures.}
\Cite{(li.~20. c.~19. de ciuit. Dei.)}
profeſſeth plainely that he vnderſtandeth not theſe words, nor that that
followeth of the myſterie of iniquitie, and leaſt of al that which the
Apoſtle addeth: \Emph{Only that he which holdeth now, doe hold &c.} Which
may humble vs al and ſtay the confident raſhnes of this time, namely of
Heretikes, that boldly feine hereof whatſoeuer is agreable to their
hereſie and phantaſie. The Apoſtle had told the Theſſaloniãs before by
word of mouth a ſecret point which he would not vtter in writing, and
therfore referreth them to his former talke.
\MNote{The myſterie of iniquitie is the couert working of heretikes
toward the manifeſt reuelation of Antichriſt himſelf.}
The myſterie of iniquitie
is commonly referred to Heretikes, who worke to the ſame, and doe that
that Antichriſt ſhal doe, but yet not openly, but in couert and vnder
the cloke of Chriſtes name, the
\Fix{Sriptures,}{Scriptures,}{obvious typo, fixed in other}
the word of the Lord, ſhew
of holines, &c. Whereas Antichriſt himſelf ſhal openly attempt and
atchieue the foreſaid deſolation, and Satan now ſeruing his turne by
Heretikes vnder-hand, ſhal toward the laſt end vtter, reueale, and bring
him forth openly. And that is here, \Emph{to be reuealed}, that is, to
appeere in his owne perſon.

Theſe other words, \Emph{Only that he which now holdeth, hold}; ſome
expound of the Emperour, during whoſe cõtinuance in his ſtate, God ſhal
not permit Antichriſt to come, meaning that the very Empire ſhal be
wholy deſolate, deſtroied, & taken away before or by his cõming: which
is more then a defection from the ſame, whereof was ſpoken before: for
there ſhal be a reuolt from the Church alſo, but it ſhal not be vtterly
deſtroied. Others ſay, that it is an admonition to al faithful, to hold
faſt their faith and not to be beguiled by ſuch as vnder the name of
Chriſt or Scriptures ſeeke to deceiue them, til they that now pretend
religion and the Ghoſpel, end in a plaine breach, reuolt, and open
apoſtaſie by the appearance of Antichriſt. Whom al Heretikes ſerue in
myſterie, that is, couertly and in the Diuel's meaning, though the world
ſeeth it not, nor thẽſelues at the beginning thought it, as now euery
day more & more al men perceiue they tend to plaine Atheiſme and
Antichriſtianiſme.}
what letteth, you know: that he may be reuealed in his time. (\V For now
the myſterie of iniquitie worketh: only that he which now holdeth, doe
hold, vntil he be taken out of the way.) \V And then that wicked one
ſhal be reuealed
\CNote{\XRef{Eſ.~11,~4.}}
whom our Lord \Sc{Iesvs} ſhal kil with the ſpirit of his mouth; and ſhal
deſtroy with the manifeſtation of his aduent, him, \V whoſe comming is
according to the operation of Satan,
\LNote{In al power.}{Satan,
\MNote{What kind of men ſhal follow Antichriſt.}
whoſe power to hurt is abridged by Chriſt, ſhal then be let looſe, &
ſhal aſſiſt Antichriſt in al manner of ſignes, wonders, and falſe
miracles, whereby many ſhal be ſeduced, not only Iewes: but al ſuch as
be deceiued & caried away by vulgar ſpeach only, of Heretikes that can
worke no miracles, much more ſhal follow this man of ſinne doing ſo
great wõders. And ſuch both now doe follow Heretikes, & then ſhal
receiue Antichriſt, that deſerue ſo to be forſakẽ of God, by their
forſaking of the vnitie & happie fellowſhip of
\Fix{SS.}{Saints}{abreviation, fixed in other}
in the Catholike Church, where only \Emph{is the Charitie of truth}, as
the Apoſtle here ſpeaketh.}
in al power, and lying ſignes and wonders, \V and in al ſeducing of
iniquitie to them that periſh, for that they haue not receiued the
charitie of the truth that they might be ſaued. \V Therfore
\SNote{\L{Deus mittet}
(ſaith S.~Auguſt.
\Cite{li.~20. de Ci. c.~19.)}
\L{quia Deus Diabolum facere iſta permittet.}
\Emph{God wil ſend, becauſe God wil permit the Diuel to doe theſe
things.} Whereby we may take a general rule that God's action or working
in ſuch things is his permiſſion. See
\XRef{Annot. Ro.~1,~24.}}
God wil ſend them the operation of errour, to beleeue
%%% o-2655
lying: \V that al may be iudged which haue not beleeued the truth, but
haue conſented to iniquitie.

\V But we ought to giue thankes to God alwaies for you, Brethren beloued
of God, that he hath choſen you firſt-fruits vnto ſaluation, in
ſanctification of ſpirit and faith of the truth: \V into the which alſo
he hath called you by our Ghoſpel, vnto the purchaſing of the glorie of
our Lord \Sc{Iesvs} Chriſt. \V Therfore, Brethren, ſtand; and hold
\TNote{\G{τὰς παραδόσεις}}
the
\LNote{Traditions.}{Not only the things written and ſet downe in the
holy Scriptures, but al other truths and points of religion vttered by
word of mouth and deliuered or giuen by the Apoſtles
\CNote{See S.~Denys
\Cite{Areop. Ec. Hier. c.~2.}}
to their ſcholers by tradition, be ſo here approued & els-where in the
Scripture it ſelf that
\MNote{Heretical tranſlation.}
the Heretikes purpoſely, guilefully, and of il conſcience (that belike
reprehendeth thẽ) refraine in their tranſlatiõs, from the Eccleſiaſtical
& moſt vſual word, \Emph{Tradition}, euer more when it is taken in good
part, though it expreſſe moſt exactly the ſignification of the 
\TNote{\G{παράδοσις}}
Greek word: but when it ſoundeth in their fond phantaſie againſt the
traditions of the Church (as indeed in true ſenſe it neuer doth) there
they vſe it moſt gladly. Here therfore and
\CNote{\XRef{1.~Cor.~11.}
\XRef{2.~Theſſ.~3.}}
in the like places, that the reader might not ſo eaſily like of
\MNote{Traditions vnwritten.}
Traditions vnwritten, here commended by the Apoſtle, they tranſlate it,
\Emph{Inſtructions, Conſtitutions, Ordinances}, and what they can inuent:
els, to hide the truth from the ſimple or vnwarie Reader, whoſe
tranſlations haue no other end but to beguile ſuch by art and
conueiance.

But
\MNote{Their authoritie and eſtimation, & examples of ſome peculiar
traditions out of the Fathers.}
\MNote{S.~Chryſoſtom.}
S.~Chryſoſtom
\Cite{(ho.~4. in 2.~Theſſ.~2.)}
and the other Greeke ſcholies or commentaries ſay hereupon, both written
and vnwritten precepts the Apoſtles gaue by traditiõ, and both be worthy
of obſeruatiõ.
\MNote{S.~Baſil.}
S.~Baſil
\Cite{(De Sp. Sancto c.~29. in principio)}
thus, \Emph{I account it Apoſtolike to cõtinue firmely euen in vnwrittẽ
traditiõs.} And to proue this, he alleageth this place of S.~Paul. In
the
\Cite{ſame booke c.~17.}
he ſaieth: \Emph{If we once goe about to reiect vnwritten cuſtoms as things
of no importance, we shal, ere we be aware, doe damage to the principal
parts of the faith, and bring the preaching of the Ghoſpel to a naked
name.} And for example of theſe neceſſarie traditiõs, he nameth the
ſigne of the Croſſe, praying towards the eaſt, the words ſpoken at the
eleuation or ſhewing of the holy Euchariſt, with diuerſe ceremonies vſed
before and after the cõſecration, the hallowing of the font, the
bleſſing of the oile, the anointing of the baptized with the ſame, the
three immerſions into the font, the words of abrenunciation and
exorciſmes of the partie that is to be baptiſed &c. \Emph{What
Scripture} (ſaith he) \Emph{taught theſe and ſuch like? none truly, al
comming of ſecret and ſilent tradition, wherwith our Fathers thought it
meet to couer ſuch myſteries.}

S.~Hierom
\MNote{S.~Hierom.}
\Cite{(Dialog. cont. Lucif. c.~4.}
\L{et}
\Cite{ep.~28. ad Licinium.)}
reckneth vp diuers the like traditiõs willing men to attribute to the
Apoſtles ſuch cuſtoms as the Church hath receiued in diuers chriſtian
countries.
\MNote{S.~Auguſtin.}
S.~Auguſtin eſteemeth the Apoſtolike traditiõs ſo much, that he plainely
affirmeth in ſundrie places, not only the obſeruatiõ of certaine
feſtiuities, faſts, ceremonies, and whatſoeuer other ſolemnities vſed in
the Catholike Church to be holy, profitable, and Apoſtolike, though they
be not written at al in the Scriptures: but he often alſo writeth, that
many of the articles of our religion and points of higheſt importance,
are not ſo much to be proued by ſcriptures, as by tradition. Namely
auouching that in no wiſe we could beleeue that children in their
infancie ſhould be baptized, \Emph{if it were not an Apoſtolical
tradition.}
\Cite{De Gen. ad lit. li.~10. c.~23.}
Tradition cauſed him to beleeue that the baptized of heretikes should
not be rebaptized, notwithſtanding S.~Cyprian's authoritie and the
manifold ſcriptures alleaged by him, though they ſeemed neuer ſo
pregnant.
\Cite{De bap. li.~2. c.~7.}
By tradition only, he and others condemned Heluidius the heretike for
denying the perpetual virginitie of our Lady. And without this, be the
Scriptures neuer ſo plaine, no Arian, no Macedonian, no Eutychian, no
Pelagian, no Zuinglian wil yeald.
\MNote{S.~Epiphanius.}
\Emph{we muſt vſe tradition} (ſaith S.~Epiphanius
\Cite{hær.~61. Apoſtolicorum)}
\Emph{For the Scripture hath not al things: and therfore the Apoſtles
deliuered certaine things in writing, certaine by tradition.} And for
that, he alleageth this place alſo of S.~Paul. And againe
\Cite{hær.~35. Melchiſed.}
\Emph{There be bounds ſet downe for the foundation and building vp of
our faith, the tradition of the Apoſtles, and holy Scriptures, and
ſucceſsion of doctrine, ſo that truth is euery way fenſed.}

S.~Irenæus
\MNote{S.~Irenæus.}
\Cite{(li.~3. c.~4.)}
hath one notable chapter, that in al queſtions we muſt haue recourſe to
the traditions of the Apoſtles: teaching vs withal, that the way to trie
an Apoſtolical tradition and to bring it to the fountaine, is by the
Apoſtolike ſucceſſion of Bishops, but ſpecially of the Apoſtolike See
of Rome: declaring in the ſame place that there be many barbarous
people, ſimple for learning, but for conſtancie in their faith moſt
wiſe, which neuer had Scriptures, but learned only by tradition.
\MNote{Tertullian.}
Tertullian
\Cite{(lib. de corona militis. nu.~3.)}
reckneth vp a great number of Chriſtian obſeruations or cuſtoms (as
\MNote{S.~Cyprian.}
S.~Cyprian in many places doth in a manner the ſame) wherof in fine he
concludeth: \Emph{Of ſuch and ſuch if thou require the rule of
Scriptures, thou shalt find none. Tradition shal be alleaged the
authour, cuſtom the confirmer, and faith the obſeruer.}
\MNote{Origen.}
Origen alſo of this matter writeth in plaine termes, that there be many
things done in the Church (which he here nameth) wherof there is no
eaſier reaſon to be giuen then tradition from Chriſt and the Apoſtles.
\Cite{ho.~5. in Numer.}
S.~Dionyſius Areopagita referreth the praying and oblation for the dead
in the Liturgie or Maſſe, to an Apoſtolical tradition.
\Cite{in fine Ec. Hierarch. c.~7. parte~3.}
So doth Tertullian
\Cite{De coron. Militis.}
S.~Augſutin
\Cite{De cura pro mortuis c.~3.}
S.~Chryſoſtom
\Cite{ho.~3. in ep. ad Philip. in Moral.}
S.~Damaſcene
\Cite{Ser. de defunctis in initio.}

We
\MNote{The Scriptures giuen vs by tradition, and the ſenſe thereof.}
might adde to al this, that the Scriptures themſelues, euen al the books
and parts of the holy Bible, be giuen vs by tradition: els we should not
nor could not take them (as they be indeed) for the infallible word of
God, no more then the workes of S.~Ignatius, S.~Clement, S.~Denys, and
the like. The true ſenſe alſo of the Scriptures (which Catholikes haue
and heretikes haue not) remaineth ſtil in the Church by tradition.
\MNote{The Creed an Apoſtolical tradition.}
The Creed is an Apoſtolike tradition.
\Cite{Ruſſin. in expo, Symb. in principio.}
\Cite{Hiero. ep.~61. c.~9.}
\Cite{Ambroſ. Serm.~38.}
\Cite{Aug. de Symb. ad Catechum. li.~3. c.~1.}
And what Scriptures haue they to proue that we muſt accept nothing not
expreſly written in Scriptures? We haue to the contrarie, plaine
Scriptures, al the Fathers, moſt euident reaſons, that we muſt either
beleeue traditions or nothing at al.
\MNote{An inuincible argument for the credit of Traditions.}
And they muſt be asked whether, if they were aſſured that ſuch things
and ſuch (which be not expreſſed in Scriptures) were taught & deliuered
by word of mouth from the Apoſtles, they would beleeue them or no? If
they ſay no, then they be impious that wil not truſt the Apoſtles
preaching: if they ſay they would, if they were aſſured that the
Apoſtles taught it: then to proue vnto them this point, we bring them
ſuch as liued in the Apoſtles daies, and the teſtimonies of ſo many
Fathers before named neer to thoſe daies, and the whole Churches
practiſe and aſſeueration deſcending downe from man to man to our
time. Which is a ſufficient proofe (at leaſt for a matter of fact) in al
reaſonable mens iudgement: Specially when it is knowen that S.~Ignatius
the Apoſtles equal in time, wrote a book of the Apoſtles traditions, as
Euſebius witneſſeth
\Cite{li.~3. Ec. hiſt. c.~30.}
And Tertullians book of preſcriptions againſt Heretikes, is to no other
effect but to proue that the Church hath this vantage aboue Heretikes,
that she can proue her truth by plaine Apoſtolical tradition, as none of
them can euer doe.}
traditions which you haue learned, whether it be by word, or by our
epiſtle. \V And our Lord \Sc{Iesvs} Chriſt himſelf and God and our
Father which hath loued vs, and hath giuen eternal conſolation, and good
hope in grace, \V 
\SNote{This word of exhorting implieth in it comfort and conſolation: as
\XRef{2.~Cor.~1. v.~4. &~6.}}
\TNote{\G{παρακαλέσαι}}
exhort your harts and confirme you in euery good worke and word.


\stopChapter


\stopcomponent


%%% Local Variables:
%%% mode: TeX
%%% eval: (long-s-mode)
%%% eval: (set-input-method "TeX")
%%% fill-column: 72
%%% eval: (auto-fill-mode)
%%% coding: utf-8-unix
%%% End:

