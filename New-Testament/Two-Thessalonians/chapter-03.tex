%%%%%%%%%%%%%%%%%%%%%%%%%%%%%%%%%%%%%%%%%%%%%%%%%%%%%%%%%%%%%%%%%
%%%%
%%%% The (original) Douay Rheims Bible 
%%%%
%%%% New Testament
%%%% Epistles
%%%% Two Thessalonians
%%%% Chapter 03
%%%%
%%%%%%%%%%%%%%%%%%%%%%%%%%%%%%%%%%%%%%%%%%%%%%%%%%%%%%%%%%%%%%%%%




\startcomponent chapter-03


\project douay-rheims


%%% 2800
%%% o-2660
\startChapter[
  title={Chapter 3}
  ]

\Summary{He deſireth their praiers, 4.~and inculcateth his precepts and
  traditions namely of working quietly for their owne liuing, commanding
  to excommunicate the diſobedient.}

For
\CNote{\XRef{Ep.~6,~18.}
\XRef{Col.~4,~3.}}
the reſt, Brethren, pray for vs, that the word of God may haue courſe
and be glorified, as alſo with you: \V and that we may be deliuered from
importunate and naughtie men. For al men haue not faith. \V But our Lord
is faithful, who wil confirme and keep you from euil. \V And we haue
confidence of you in our Lord, that the things which we command, both
you doe, and wil doe. \V And our Lord direct your harts in the charitie
of God, and patience of Chriſt.

\V And we denounce vnto you, Brethren, in the name of our Lord \Sc{Iesvs}
Chriſt, that you withdraw your ſelues from euery Brother walking
inordinately, and not according to the
\SNote{Here alſo (as is noted before
\XRef{2.~Theſſ.~2,~15.)}
the Aduerſaries in their tranſlatiõs auoid the word, \Emph{Tradition},
being plaine in
\Fix{thee}{the}{obvious typo, fixed in other}
Greek, leſt thẽſelues might ſeem to be noted as men walking
inordinately, and not according to Apoſtolical Tradition, as al
Schiſmatikes, Heretikes, and rebels to God's Church doe.}
\TNote{\G{παράδοσιν}}
tradition which they haue receiued of vs. \V For your ſelues know how
you ought to imitate vs: for we haue not been
%%% o-2661
vnquiet among you: \V
\CNote{\XRef{Act.~10.}
\XRef{1.~Cor.~4.}
\XRef{1.~Th.~2.}}
neither haue we eaten bread of any man gratis, but in labour & in toile
night and day working, leſt we ſhould burden any of you. \V
\CNote{\XRef{1.~Cor.~9,~6.}}
Not as though we had not authoritie: but that we might giue our ſelues a
paterne vnto you for to imitate vs. \V For alſo when we were with you,
this we denounced to you, that if any wil not worke,
\LNote{Neither let them eate.}{It
\MNote{The heretikes cauillation
\Fix{agaiſt}{againſt}{obvious typo, fixed in other}
Religious men that worke not, anſwered.}
is not a general precept or rule, that euery man ſhould liue by his
handy-worke, as the Anabaptiſts argue falſely againſt Gentlemen & the
Caluiniſts applie it peruerſely againſt the vacant life of the Clergie,
ſpecially of Monkes and other Religious men. But it is a natural
admonition only, giuen to ſuch as had not wherwith to liue of their
owne, or any right or good cauſe why to chalenge their finding of
others, and to ſuch as vnder the colour of Chriſtian libertie did paſſe
their time idly, curiouſly, vnprofitably, and ſcandalouſly, refuſing to
doe ſuch workes as were agreable to their former calling and bringing
vp. Such as theſe, were not tolerable, ſpecially there and then, when
the Apoſtle and others (that might lawfully haue liued of the altar and
their preaching) yet to diſburden their hearers, and for the better
aduancement of the Ghoſpel, wrought for their liuing:
\CNote{\XRef{1.~Cor.~9.}}
proteſting neuertheles continually, that they might haue done otherwiſe,
as wel as S.~Peter and the reſt did, who wrought not, but were found
otherwiſe iuſtly and lawfully, as al ſorts of the Clergie preaching or
ſeruing the Church and the altar, be, and ought to be,
\CNote{\Cite{See S.~Cypr. ep.~66.}}
by the law of God and nature.
\MNote{The ſpiritual trauailes of the Clergie.}
Whoſe ſpiritual labours farre paſſe al
\Fix{boldily}{bodily}{obvious typo, fixed in other}
trauailes, where the dueties and functions of that vocation be done
accordingly: as S.~Auguſtin affirmeth of his owne extraordinarie paines
incident to the Eccleſiaſtical affaires & regiment: inſteed of which, if
the vſe of the Church and his infirmitie would haue permitted it he
wiſheth he might haue laboured with his hands ſomme houres of the
day.
\MNote{Religious mens working with their hands.}
As ſome of the Clergie did euer voluntarily occupie themſelues in
teaching, writing, grauing, painting, planting, ſowing, embrodering, or
ſuch like ſeemely and innocent labours. See
\Cite{S.~Hierom ep.~114.}
\Cite{ſeu. præf. in Iob.}
and
\Cite{in vit. Hilario.}

And Monkes for the moſt part in the primitiue Church (few of them being
Prieſts, and many taken from ſeruile workes and handy-crafts, yea
often-times profeſſed of bond-men, made free by their maiſters to enter
into religion) were appointed by their ſuperiours to worke certaine
houres of the day, to ſupply the lackes of their Monaſteries: as yet the
Religious doe (women ſpecially) in many places, which ſtandeth wel with
their profeſſion. And S.~Auguſtin writeth a whole booke
\Cite{(de Opere Monacherum to.~3.)}
againſt the errour of certaine diſordered Monkes that abuſed theſe
words, (\L{Nolite eſſe ſoliciti}, \Emph{be not careful &c.} and
\L{Reſpicite volatilia cæli}, \Emph{behold the foules of the aire &c.})
to proue that they ſhould not labour at al, but pray only and commit
their finding to God: not only ſo excuſing their idlenes, but preferring
themſelues in holines aboue other their fellowes that did worke, and
erroneouſly expounding the ſaid Scriptures for their defence: as they
did other Scriptures, to proue
\MNote{Monkes were ſhauen in the primitiue Church, and Nonnes clipped of
their haire.}
they ſhould not be ſhauen after the manner of Monkes. Which letting
their heads to grow he much blameth alſo in them. See
\Cite{li.~2. Reſtract. c.~21.}
&
\Cite{de op. Monach. c.~31.}
and S.~Hierom
\Cite{ep.~48. c.~3.}
of Nonnes cutting their haire.

Where by the way you ſee that the Religious were ſhauen euen in
S.~Auguſtines time, who reprocheth them for their haire, calling them
\L{Crinitos} \Emph{Hairelings}, as the Heretikes now contrariewiſe
deride them by the word \L{Rafos}, \Emph{Shauelings}. So that there is a
great difference between the ancient Fathers and the new Proteſtants.
\MNote{S.~Auguſtines opinion concerning Religious mens working or not
working.}
And as for hand-labours, as
\CNote{\Cite{li. de op. Monach. c.~21.}}
S.~Auguſtin in the book alleadged would not haue Religious folke to
refuſe them, where neceſſitie, bodily ſtrength, and the order of the
Church or Monaſterie permit or require them; ſo he expreſly writeth,
that al can not nor are not bound to worke, and that whoſoeuer preacheth
or miniſtreth the Sacraments to the people or ſerueth the altar (as al
Religious men commonly now doe) may chalenge their liuing of them whom
they ſerue, and are not bound to worke, no nor ſuch neither as haue
been brought vp before in ſtate of Gentlemen, and haue giuen away their
lands or goods, and made themſelues poore for Chriſtes ſake. Which is to
be noted, becauſe the Heretikes affirme the ſaid Scripture and
S.~Auguſtin to condemne al ſuch for idle perſons.}
neither let him eate. \V For we haue heard of certaine among you that
walke vnquietly, working nothing, but curiouſly medling. \V And to them
that be ſuch we denounce, & beſeech them in our Lord \Sc{Iesvs} Chriſt,
that working with ſilence, they eate their owne bread.

\V But you, Brethren,
\CNote{\XRef{Gal.~6,~9.}}
faint not wel-doing. \V And if any
\LNote{Obey not.}{Our
\MNote{Eccleſiaſtical cenſures againſt the diſobedient.}
Paſtours muſt be obeied, and not only ſecular Princes. And ſuch as wil
not be obedient to their ſpiritual Gouernours, the Apoſtle (as
S.~Auguſtin ſaith) giueth order and commandment that they be corrected
by correption or admonition, \Emph{By degradation, excommunication, and
other lawful kinds of punishments}.
\Cite{Cont. Donatiſt, poſt. Collat. c.~4.~20.}
Read alſo this holy Fathers anſwer to ſuch as ſaid: \Emph{Let our
Prelates command vs only what we ought to doe, and pray for vs that we
may doe it: but let them not correct vs.} Where he proueth that Prelates
muſt not only command and pray, but puniſh alſo if that be not done
which is commanded.
\Cite{li. de correp. & grat. c.~3.}}
obey not our word,
\LNote{Note him.}{Diſobedient
\MNote{Not to communicate with excommunicate perſons but in certaine
caſes.}
perſons to be excommunicated, and the excommunicated to be
\Fix{ſepared}{ſeparated}{obvious typo, fixed in other}
from the companie of other Chriſtians, and the faithful not to keep any
companie or haue conuerſation with excõmunicated perſons, neither to
be partaker with them in the fault for which they are excommunicated,
nor in any other act of religion or office of life, except caſes of mere
neceſſitie and other preſcribed and permitted by the law: al this is
here inſinuated, and that al the Churches cenſures be grounded in
Scriptures and the examples of the Apoſtles.}
note him by an epiſtle: \V and doe not companie with him, that he may be
confounded: and doe not eſteem him as an enemie, but admoniſh him as a
Brother. \V And the Lord of peace himſelf giue you euerlaſting peace in
euery place. Our Lord be with you al. \V The ſalutation, with mine owne
hand, Paules: which is a ſigne in euery epiſtle. So I write. \V The
grace of our Lord \Sc{Iesvs} Chriſt be with you al. Amen.  


\stopChapter


\stopcomponent


%%% Local Variables:
%%% mode: TeX
%%% eval: (long-s-mode)
%%% eval: (set-input-method "TeX")
%%% fill-column: 72
%%% eval: (auto-fill-mode)
%%% coding: utf-8-unix
%%% End:

