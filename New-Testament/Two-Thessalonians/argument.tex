%%%%%%%%%%%%%%%%%%%%%%%%%%%%%%%%%%%%%%%%%%%%%%%%%%%%%%%%%%%%%%%%%
%%%%
%%%% The (original) Douay Rheims Bible 
%%%%
%%%% New Testament
%%%% Two Thessalonians
%%%% Argument
%%%%
%%%%%%%%%%%%%%%%%%%%%%%%%%%%%%%%%%%%%%%%%%%%%%%%%%%%%%%%%%%%%%%%%




\startcomponent argument


\project douay-rheims


%%% 2791
%%% o-2652
\startArgument[
  title={\Sc{The Argvment of the Second Epistle of S.~Pavl to the
  Thessalonians.}},
  marking={Argument of Two Thessalonians}
  ]

The ſecond to the Theſſalonians hath in the title as the
firſt: \Emph{Paul and Syluanus and Timothee}, &c. And therfore it ſeemeth
to haue been written in the ſame place, to wit, at Corinth, where they
remained
\CNote{\XRef{Act.~18. v.~11.}}
\Emph{a yeare and ſixe months}, & ſtraight vpon their anſwer to the
firſt epiſtle.

Firſt he thanketh God for their increaſe, and perſeuerance (comforting
them againe in thoſe perſecution) and praieth for their
accomplishment. Secondly he aſſureth them, that the day of Iudgement is
not at hand, putting them in
\Fix{remembance}{remembrance}{likely typo, fixed in other}
what he told them therof by
word of mouth, when he was preſent (as therfore he biddeth them
afterward
\CNote{\XRef{c.~2. v.~15.}}
to hold his Traditions vnwritten, no leſſe then the written) to wit,
that al thoſe perſecutions and hereſies, raiſed then, and afterward
againſt the Catholike Church, were but the myſterie of Antichriſt, &
not Antichriſt himſelf. But that there should come at length a plaine
Apoſtaſie, & then (the whole fore-running myſterie being once perfitly
wrought) should follow the reuelation of Antichriſt himſelf in perſon
(as after al the myſteries of the old Teſtament Chriſt \Sc{Iesvs} our
Lord came himſelf in the fulnes of time.) And then at length after al
this, the day of Iudgement and ſecond comming of Chriſt shal be at hand,
and not before, whatſoeuer pretenſe of viſion, or of ſome ſpeach of mine
(ſaith S.~Paul) any make to ſeduce you withal, or of my former epiſtle,
or any other. For which cauſe alſo, in the end of this epiſtle, he
biddeth them to know his hand, \Emph{which is a ſigne in euery epiſtle.}

Laſtly he requeſteth their praiers, and requireth them to keep his
commandements and Traditions: namely that the poore which are able, get
their owne liuing with working, as he alſo gaue them example, though he
were not bound thereto.


\stopArgument


\stopcomponent


%%% Local Variables:
%%% mode: TeX
%%% eval: (long-s-mode)
%%% eval: (set-input-method "TeX")
%%% fill-column: 72
%%% eval: (auto-fill-mode)
%%% coding: utf-8-unix
%%% End:
