%%%%%%%%%%%%%%%%%%%%%%%%%%%%%%%%%%%%%%%%%%%%%%%%%%%%%%%%%%%%%%%%%
%%%%
%%%% The (original) Douay Rheims Bible 
%%%%
%%%% New Testament
%%%% Epistles
%%%% Galatians
%%%% Chapter 02
%%%%
%%%%%%%%%%%%%%%%%%%%%%%%%%%%%%%%%%%%%%%%%%%%%%%%%%%%%%%%%%%%%%%%%




\startcomponent chapter-02


\project douay-rheims


%%% 2740
%%% o-2597
\startChapter[
  title={Chapter 2}
  ]

\Summary{He telleth forth the ſtorie begun in the laſt chapter, and how
  he reprehended Peter, 15.~and then ſpecially vrgeth the enſample of
  the Chriſtian Iewes, who ſought vnto Chriſt for iuſtification, and
  that by warrant alſo of their Law it-ſelf, as alſo becauſe otherwiſe
  Chriſt's death had been needles.}

%%% o-2598
Then after fourteen yeares I went vp againe to Hieruſalem with Barnabas,
taking Titus alſo with me. \V And I went vp according to reuelation: and
\LNote{Conferred with them.}{Though
\MNote{S.~Paul conferreth with S.~Peter and the reſt, for trail of his
doctrine.}
S.~Paul were taught his Ghoſpel of God and not of man, and had an
extraordinarie calling by Chriſt himſelf, yet by reuelation he was ſent
to Hieruſalem to conferre the ſaid Ghoſpel which he preached, with his
elders the ordinarie Apoſtles and Rulers of the Church, to put both his
vocation and doctrine to their trail and approbation, and to ioyne in
office, teaching, and ſocietie or communion with them. For there is no
extraordinarie or miraculous vocation, that can ſeuer or ſeparate the
perſon ſo called, in doctrine or fellowship of Chriſtian life and
religion, from the ordinarie knowen ſocietie of God's people and
Prieſts.
\MNote{The heretikes ſubmit their doctrine to no trail of Bishops or
Councel.}
Therfore whoſoeuer he be (vpon what pretence ſoeuer) that wil
not haue his calling and doctrine tried by the ordinarie Gouerners of
God's Church, or diſdaineth to goe vp to the principal place of
our religion, to conferre with Peter and other pillars of the Church, it is
euident that he is a falſe Teacher, a Schiſmatike, and an Heretike. By
which rule you may trie al your new Teachers of Luther's or Caluin's
ſchoole: who neuer did nor euer durſt put their preaching to ſuch
conference or trial of holy Councel or Bishops, as they ought to doe,
and would doe, if
it were of God, as S.~Paules was.}
conferred with them the Ghoſpel which I preach among the Gentils, but
apart with them that ſeemed to be ſome-thing, leſt perhaps
\LNote{In vaine.}{Though
\MNote{The approbation of S.~Paules doctrine by Peter and the reſt, was very
requiſit.}
S.~Paul doubted not of the truth of the Ghoſpel which he preached,
knowing it to be of the holy Ghoſt; yet becauſe other men could not, nor
would not acknowledge ſo much, til it were allowed by ſuch as were
without al exception knowen to be Apoſtles & to haue the ſpirit of
truth, to diſcerne whether the vocation, ſpirit, & Ghoſpel of Paul were
of God, he knew he should otherwiſe without conference with them, haue
loſt his labour, both for the time paſt and to come. \Emph{He had not
had} (ſaith S.~Hierom) \Emph{ſecuritie of preaching the Ghoſpel, if it
had not been approued by Peter's ſentence & the reſt that were with
him.}
\Cite{Hiero ep.~89. c.~2.}
See
\Cite{Tertul. li.~4. cont. Marc. nu.~3.}
Therfore by reuelation he went to conferre with the Apoſtles at
Hieruſalem, that by them hauing his Apoſtleship and Ghoſpel liked and
approued, he might preach with more fruit. Wherin we ſee, this holy
Apoſtle did not as the ſeditious proud Heretikes doe now a-daies, which
refuſing al man's atteſtation or approbation, wil be tried by Scriptures
only.
\MNote{No abſurditie that the Scriptures be approued by the Churches
teſtimonie.}
As alſo we may learne that it is no ſuch abſurditie as the Aduerſaries
would make it, to haue Scriptures approued by the Churches teſtimonie:
ſeeing the Ghoſpel which S.~Paul preached (being of as much certaintie
and of the ſame Holy Ghoſt that the Scriptures be) was to be put in
conference and examination of the Apoſtles, without al derogation to the
truth, dignitie, or certaintie of the ſame. And the cauilling of
Heretikes, that we make ſubiect God's Oracles to man's cenſure, and the
Scriptures to haue no more force then the Church is content to grant
vnto them, is vaine and falſe.
\MNote{The Church maketh not Canonical Scripture, but declareth that it
is ſo.}
For, to beare witnes or to giue euidence or atteſtation that the
preaching or writing of ſuch, is true and of the Holy Ghoſt, is not to
make it true: no more then the Gold-ſmith or touch-ſtone that trie and
diſcerne which is true gold, make it good gold; but they giue euidence
to man that ſo it is. And therfore that diſputation alſo, whether the
Scripture or the Church be of greater authoritie, is ſuperfluous: either
giuing teſtimonie to the other, and both aſſured by the Holy Ghoſt from
al errour:
\MNote{The Scripture & Church cõpared together for antiquitie,
authoritie, &c.}
the Church yet being before the Scriptures, the ſpouſe of
Chriſt, and proper dwelling, temple, or ſubiect of God, and his graces:
for the which Church the Scriptures were, and not the Church for the
Scriptures. In which Church there is iudicial authoritie by office and
iuriſdiction to determine of doubtful queſtions touching the ſenſe of
the Scriptures and other controuerſies in religion, & to punish
diſobedient perſons. Of which iudicial power the Scriptures be not
capable; as neither the truths and determinations of the ſame can be ſo
euident to men, nor ſo agreable and fit for euery particular reſolution,
as diuerſitie of times and perſons requireth. Certaine is the truth, and
great is the authoritie of both: but in ſuch diuers kinds, as they can
not be wel compared together. The controuerſie is much like as if a man
touching the ruling of a caſe in law or giuing ſentence in a matter of
queſtion, should aske, whether the iudge, or the euidence of the
parties, be of more authoritie or credit. Which were as friuolous a
diſpute, as it were a diſordered part for any mã to ſay, he would be
tried by no other iudge but by his owne writings or euidẽces. With ſuch
triflers and ſeditious perſons haue we to doe now a-daies in diuinitie,
as were intolerable in any prophane ſcience or facultie in the world.}
in vaine I should runne or had runne. \V But neither Titus which was
%%% 2741
with me, whereas he was a Gentil, was compelled to be circumciſed: \V
but becauſe of the falſe Brethren craftily brought in, which craftily
came in to eſpie our libertie that we haue in Chriſt \Sc{Iesvs}, that
they might bring vs into ſeruitude. \V To whom we yealded not ſubiection
no not for an houre, that the truth of the Ghoſpel may remaine with
you. \V But of thẽ that ſeemed to be ſome-thing (what they were
ſome-time, it is nothing to me.
\CNote{\XRef{Deu.~10,~17.}}
God accepteth not the perſon of man) for to me, they that ſeemed to be
ſomething,
\LNote{Added nothing.}{The
\MNote{The Scriptures alwaies true in themſelues, are ſo knowen to be by
the Church.}
Ghoſpel and preaching of S.~Paul was wholy of God, and therfore though
it were put to the Churches probatiõ, as gold is to the touch-ſtone; yet
being found in al points pure, nothing could be altered or amended
therin by the Apoſtles. Euẽ ſo the Scriptures which are indeed wholy of
the Holy Ghoſts enditing, being put to the Churches trial, are found,
proued, and teſtified vnto the world to be ſuch, & not made true,
altered, or amended by the ſame.
\Fix{Whithout}{Without}{likely typo, fixed in other}
which atteſtation of the Church, the holy Scriptures in themſelues were
alwaies true before: but not ſo knowen to be, to al Chriſtians, nor they
ſo bound to take them. And that is the meaning of the famous ſentence of
S.~Auguſtin
\Cite{Cont. ep. fund. c.~5.}
which troubleth the Heretikes ſo much: \Emph{I would not beleeue the
Ghoſpel} (ſaith he) \Emph{vnles the authoritie of the Church moued me.}}
added nothing. \V But contrariewiſe when they had ſeen, that to me was
committed the Ghoſpel of the
\SNote{See the
\XRef{marginal Annotation Rom.~2. v.~25.}}
prepuce, as
\LNote{To Peter of the circumciſion.}{We
\MNote{The Apoſtles commiſſion general through the world, & yet peculiar
to certaine Prouinces.}
may not thinke, as the Heretikes deceitfully teach, that the charge of
the Apoſtles was ſo diſtincted, that none could preach or exerciſe
iuriſdiction but in thoſe ſeueral places or towards thoſe peoples or
Prouinces only, wherunto by God's appointment or their owne lot or
election, they were ſpecially deſigned. For, euery Apoſtle might by
Chriſtes commiſſion
\XRef{(Mat.~28.}
\Emph{Goe, and teach al Nations}) vſe al ſpiritual function through the
whole world. Yet for the more particular regard and care of Prouinces,
and for peace and order ſake, ſome were appointed to one countrie, and
ſome to another: as, of the other Apoſtles we ſee in the Eccleſiaſtical
hiſtories,
\MNote{Iewes and Gentils ſpecially committed to the two principal
Apoſtles.}
and for S.~Peter and S.~Paul, it is plaine by this place &
other, that to thẽ as to the two cheefe & moſt renowmed Apoſtles, the
Church of al Nations was giuen, as deuided into two parts, that is,
Iewes and Gentils: the firſt and principal being S.~Peter's lot, that
herein alſo he might reſemble our Sauiour, who was ſent namely
\CNote{\XRef{Mat.~15.}}
\Emph{to the loſt sheepe of Iſrael}, and was properly
\CNote{\XRef{Ro.~15.}}
\Emph{the Miniſter of the Circumciſion}: the ſecond being S.~Paules,
whom Chriſt choſe ſpecially to preach to the Gentils:
\MNote{Neither Peter only of the Iewes, nor Paul Apoſtle of the Gentils
only.}
Not ſo for al that, that either he was limited to the Gentils only,
(whom the Actes of the Apoſtles report, in euery place, firſt to haue
entred into the Synagogues and preached Chriſt to the Iewes, as he wrote
alſo to the Hebrewes and euer had ſpecial regard and honour to them:) or
Peter ſo bound to the Iewes only, that he could not meddle with the
Gentils: ſeeing he was
\CNote{\XRef{Act.~10.}
&
\XRef{15. v.~7.}}
the man choſen of God, by whom the Gentils should firſt beleeue, who
firſt baptized them, and firſt gaue order concerning them.
\CNote{\Cite{Calu. li.~4. c.~6. nu.~15. Inſtit.}}
\MNote{Caluin's foolish reaſon that Peter was not B.~of Rome, & his
derogation from Peters Apoſtleship.}
Therfore the treacherie of Caluin is intolerable, that vpon this
diſtinction of the Apoſtles charge, would haue the ſimple ſuppoſe, that
S.~Peter could not be Bishop of Rome (ſo might he barre S.~Iohn from
Epheſus alſo) nor deale among the Gentils, as a thing againſt God's
ordinance and the appointment between him and S.~Paul: as though thereby
the one had bound himſelf to the other, not to preach or meddle within
his fellowes compaſſe. And which is further moſt ſeditious, he exhorteth
al men to keep faſt the foreſaid compact, and rather to haue reſpect to
S.~Paules Apoſtleship, then to S.~Peters: as though the preaching,
authoritie, and Apoſtleship of both were not a-like true, and al of one
holy Spirit, whether they preached to Iewes or Gentils, as both did
preach vnto both peoples, as is already proued, and at length, partly by
the daily decay of the Iewish ſtate and there incredulitie, and partly
for that in Chriſtianitie the diſtinction of Iew and Gentil ceaſed
after a ſeaſon,
\MNote{The Church founded at Rome by S.~Peter and S.~Paul.}
both went to the cheefe citie of the Gentils, and there founded the
Church common to the Hebrewes and al Nations, Peter firſt, and Paul
afterward. And therfore Tertul. ſaith,
\Cite{de Præſcript. nu.~14.}
\Emph{O happie Church, to which the Apoſtles powred out al doctrin with
their bloud! Where Peter ſuffereth like to our Lord's Paſsion, where
Paul is crowned with Iohn (Baptiſt's) death.}}
to Peter of the circumciſion (\V for he that wrought in Peter to the
Apoſtleſhip of circumciſion, wrought in me alſo among the Gentils) \V and
when they had knowen the grace that was giuen me, Iames and Cephas and
Iohn, which ſeemed to be pillars,
\LNote{Gaue the right hands of ſocietie.}{There
\MNote{Al Catholike Preachers and Paſtours muſt communicate with Peter
and his Succeſſours.}
is and alwaies ought to be, a common fellowship and fraternitie of al
Paſtours and Preachers of the Church. Into which ſocietie whoſoeuer
entreth not, but ſtandeth in Schiſme and ſeparation from Peter and the
cheefe Apoſtolike Paſtours, what pretence ſoeuer he hath, or whence
ſoeuer he chalengeth authoritie, he is a wolfe, and no true
Paſtour. Which vnion and communion together was ſo neceſſarie euen in
S.~Paules caſe, that, notwithſtanding his ſpecial calling of God, yet
the Holy Ghoſt cauſed him to goe vp to his elder Apoſtles to be receiued
into their fellowship or brotherhood. For it is to be noted, that
SS.~Peter, Iames, and Iohn were not ſent to S.~Paul, to ioyne with him
or to be tried for their doctrine and calling, by him: but cõtrariewiſe
he was ſent to thẽ as to the cheefe & knowen ordinarie Apoſtles. They
therfore gaue Paul their hands, that is to ſay, took him into their
ſocietie, and not he them. And S.~Hierom's rule concerning this, shal be
found true to the worlds end, ſpeaking of S.~Peter's Succeſſour:
\Emph{He that gathereth not with thee, ſcattereth.}
\Cite{Ep.~57.}
And in another place for the ſame cauſe he calleth Rome, \L{tutiſsimum
Communionis portum}, the moſt ſafe and ſure hauen of cõmunion or
ſocietie.
\Cite{Ep.~16. c.~4.}
\MNote{The Heretikes ridiculous argument againſt Peter's preeminence.}
And wheras the Heretikes by this alſo would proue that Peter had no
preeminence aboue Paul being his fellow Apoſtle, it is ridiculous. As
though al of one fellowship or brotherhood be alwaies equal; or as
though there were not order and gouernment,
\Fix{ſuperiorietie}{ſuperioritie}{possible typo, fixed in other}
and inferioritie, in euery ſocietie wel appointed. And they might
perceiue by this whole paſſage, that Peter was the ſpecial, and in more
ſingular ſort the Apoſtle of the Iewes, though Iames and Iohn were alſo:
as S.~Paul is alſo called in more ſingular ſort the Apoſtle and Doctour
of the Gentils then S.~Barnabas, and yet they were both a-like taken
here into this ſocietie, as they were both at once and a-like ſegregated
into this miniſterie, and ordered together.
\XRef{Act.~13.}
It is a poore reaſon then to ſay or thinke, S.~Peter not to be aboue
S.~Barnabas neither, becauſe of this ſocietie and fellowship vnto which
he was receiued together with S.~Paul.}
gaue to me and Barnabas the right hands of ſocietie: that we vnto the
Gentils, & they vnto the circumciſion: \V only that we ſhould be mindful
of the poore: the which ſame thing alſo I was careful to doe.

\V And when Cephas was come to Antioche,
\LNote{I reſiſted him.}{Wicked
\MNote{The Heretikes malitiouſly derogate from S.~Peter.}
Porphyrie (as S.~Hierom writeth) chargeth S.~Paul of enuie & malapert
boldnes, and S.~Peter of errour
\Cite{Præm. Comment. in Galat.}
Euen ſo the like impious ſonnes of Cham, for this, and for other things,
gladly charge S.~Peter, as though he had committed the greateſt crimes
in the world. For, it is the propertie of Heretikes and il men, to be
glad to ſee the Saints reprehended and their faults diſcouered, as we
may learne in the writings of S.~Auguſtin againſt Fauſtus the Manichee,
who gathered out al the acts of the holy Patriarches, that might ſeeme
to the People to be worthy blame.
\MNote{Paules reprehenſion of Peter teacheth vs the zeale of the one,
and humilitie of the other.}
Whom the ſaid holy Doctour defendeth at large againſt him: as both he,
and before him S.~Cyprian, find here vpon this Apoſtles reprehenſion,
much matter of praiſing both their vertues: S.~Paules great zeale, &
S.~Peters wonderful humilitie: that the one in the cauſe of God would
not ſpare his Superiour, and that the other, in that excellent dignitie,
would not take it in il part, nor by allegation of his Supremacie
diſdaine or refuſe to be controled by his Iunior. Which of the two they
count the greater grace and more to be imitated. \Emph{For neither
Peter} (ſaith S.~Cyprian) \Emph{whom our Lord choſe the firſt, and vpon
whom he built the Church, when Paul diſputed with him of circumciſion,
chalenged inſolently or arrogantly tooke any thing to himſelf, ſaying
that he had the Primacie, and therfore the later Diſciples ought rather
to obey him.}
\Cite{ep.~71. ad Quintum nu.~2.}
And S.~Auguſtin
\Cite{ep.~19. c.~2. in fine.}
\Emph{That} (ſaith he) \Emph{which was done of Paul profitably by the
libertie of charitie, the ſame Peter tooke in good part by holy and
benigne, Godlines of humilitie, and ſo he gaue vnto poſteritie a more
rare and holy example, if at any time perhaps they did amiſſe, to be
content to be corrected of their Iuniors, then Paul, to be bold and
confident: yea the inferiours to reſiſt their betters for defending the
truth of the Ghoſpel, brotherly charitie alwaies preſerued.}
\MNote{It proueth nothing againſt Peter's ſuperioritie, that he was
reprehended.}
By which notable ſpeaches of the Doctours we may alſo ſee how friuolouſly
the Heretikes argue hereupon, that S.~Peter could not be Superiour to
S.~Paul, being ſo reprehended of him: wheras the Fathers make it an
example to the Superiours, to beare with humilitie the correption or
controlement euen of their inferiours. Namely by this example
S.~Auguſtin
\Cite{(li.~2. de Bapt. c.~1.)}
excellently declareth, that the B.~Martyr S.~Cyprian, who walked awry
touching the rebaptizing of them that were chriſtned of Heretikes, could
not, nor would not haue been offended to be admonished & reformed in
that point by his fellowes or inferiours, much leſſe by a whole
Councel. \Emph{We haue learned}, ſaith he, \Emph{that Peter the Apoſtle,
in whom the Primacie of the Apoſtles by excellent grace is ſo
preeminent, when he did otherwiſe concerning circumciſion then the truth
required, was corrected of Paul the later Apoſtle. I thinke (without any
reproch vnto him) Cyprian the Bishop may be compared to Peter the
Apoſtle: howbeit I ought rather to feare leſt I be iniurious to
Peter. For who knoweth not that the principalitie of Apoſtleship is to be
preferred before any dignitie of Bishop whatſoeuer? But if the grace of
the Chaires or Sees differ, yet the glorie of the Martyrs is one.}
\MNote{The ſuperiour may be reprehended or admonished of the inferiour.}
And who is ſo dull that can not ſee, that the inferiour though not
by office and iuriſdiction, yet by the law of brotherly loue and
fraternal correption, may reprehend his ſuperiour? Did euer any man
wonder that a good Prieſt or any vertuous perſon should tell the Pope,
or any other great Prelate, or greateſt Prince in earth, their faults?
Popes may be reprehended, & are iuſtly admonished of their faults, &
ought to take in in good part, and ſo they doe & euer haue done, when it
commeth of zeale & loue, as of S.~Paul, Irenæus, Cyprian, Hierom,
Auguſtin, Bernard:
\MNote{Heretikes reprehenſion of Catholike Bishops is rather railing.}
But of Simon Magus, Nouatus, Iulian, Wiclife, Luther, Caluin, Beza, that
doe it of malice, & raile no leſſe at their vertues then their vices, of
ſuch (I ſay) God's Prelates muſt not be taught nor corrected, though they
muſt patiently take it, as our Sauiour did the like reproches of the
malitious Iewes; and as Dauid did the malediction of Semei.
\XRef{2.~Reg.~1.}}
I reſiſted him
\SNote{That is, \Emph{in preſence, before them al}, as Beza himſelf
expoundeth it. Yet the English Bezites to the more diſgracing of
S.~Peter, tranſlate, \Emph{to his face},
\Cite{No.~Teſtam.~1580.}}
\TNote{\G{κατὰ πρόσωπον}}
in face, becauſe he was
\LNote{Reprehenſible.}{The
\MNote{S.~Peter's errour was not in faith, but in conuerſation or
behauiour.}
Heretikes hereof againe inferre, that Peter thẽ did erre in faith, and
therfore the Popes may faile therin alſo. To which we anſwer, that how
ſoeuer other Popes may erre in their priuate teachings or writings,
wherof we haue treated before in
%%% !!! Messy --- CNote in middle of XRef
\XRef{the Annotation vpon theſe words, 
   \CNote{Luc.~22,~32.}
   \Emph{That thy faith faile not}:}
it is certaine that S.~Peter did not here faile in faith, nor erre in
   doctrine or knowledge. For it was \L{conuerſationis, non prædictionis
vitium}, as Tertullian ſaith
\Cite{de præſcript. nu.~7.}
It was a default in conuerſation, life, or regiment, which may be
committed of any man, be he neuer ſo holy, and not in
doctrine. S.~Auguſtin and whoſoeuer make moſt of it, thinke no otherwiſe
of it. But S.~Hierom and
\CNote{See \Cite{S.~Chryſoſt.}
\Cite{Theoph.}
&c.}
many other holy Fathers deeme it to haue been no fault at al, nor any
other thing then S.~Paul himſelf did vpon the like occaſion: and that
this whole combat was a ſet thing agreed vpon between them. It is a
ſchoole point much debated betwixt S.~Hierom and S.~Auguſtin.
\Cite{ep.~9.}
\Cite{11.}
\Cite{19. apud Auguſt.}}
reprehenſible. \V For before that certaine came from Iames, he did eate
with the Gentils: but when they were come, he withdrew and ſeparated
himſelf, fearing them that were of the circumciſion. \V And to his
ſimulation conſented the reſt of the Iewes, ſo that Barnabas alſo was
led of them into that ſimulation. \V But when I ſaw that they walked not
rightly to the veritie of the Ghoſpel, I ſaid to Cephas before them al:
If thou being a Iew, liueſt Gentil-like and not Iudaically, how doeſt
thou compel the Gentils to Iudaize?

\V We are by nature Iewes, and not of the Gentils, ſinners. \V But
knowing that
\CNote{\XRef{Ro.~3,~19.~20.}}
man is not iuſtified by the
%%% o-2599
workes
\SNote{By this & by the diſcourſe of this whole epiſtle, you may
perceiue, that when iuſtification is attributed to faith, the workes of
Charitie be not excluded, but the workes of Moyſes law: that is, the
ceremonies, Sacrifices, and Sacraments therof principally, and
conſequently al workes done merely by nature & free-wil, without the
faith, grace, ſpirit, and aid of Chriſt.}
of the Law, but by the faith of \Sc{Iesvs} Chriſt; we alſo beleeue in
Chriſt \Sc{Iesvs}, that we may be iuſtified by the faith of Chriſt, and
not by the workes of the Law: for the which cauſe, by the workes of the
Law no flesh ſhal be iuſtified. \V But if ſeeking to be iuſtified in
Chriſt, our ſelues alſo be found ſinners; is Chriſt them a Miniſter of
ſinne? God forbid. \V For if I build the ſame things againe which I haue
deſtroied, I make my ſelf a preuaricatour. \V For I by the Law, am dead
to the Law, that I may liue to God: with Chriſt I am nailed to the
croſſe. \V And I liue, now not I; but Chriſt liueth in me. And that that
I liue now in the fleſh, I liue in the faith of the Sonne of God, who
loued me, and deliuered himſelf for me. \V I caſt not away the grace of
God. For if iuſtice be by the Law, then Chriſt died in vaine.


\stopChapter


\stopcomponent


%%% Local Variables:
%%% mode: TeX
%%% eval: (long-s-mode)
%%% eval: (set-input-method "TeX")
%%% fill-column: 72
%%% eval: (auto-fill-mode)
%%% coding: utf-8-unix
%%% End:

