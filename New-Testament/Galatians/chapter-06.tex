%%%%%%%%%%%%%%%%%%%%%%%%%%%%%%%%%%%%%%%%%%%%%%%%%%%%%%%%%%%%%%%%%
%%%%
%%%% The (original) Douay Rheims Bible 
%%%%
%%%% New Testament
%%%% Galatians
%%%% Chapter 06
%%%%
%%%%%%%%%%%%%%%%%%%%%%%%%%%%%%%%%%%%%%%%%%%%%%%%%%%%%%%%%%%%%%%%%




\startcomponent chapter-06


\project douay-rheims


%%% 2753
%%% o-2610
\startChapter[
  title={Chapter 6}
  ]

\Summary{If any doe ſinne, the reſt that doe the workes of the Holy
  Ghoſt, muſt not therfore take pride in themſelues, but rather make
  humilitie of it, partly by fearing their owne fal, partly by looking
  ſtraitly to their owne workes. 6.~He exhorteth earneſtly to good
  workes, aſſuring them that they shal reap none other then heer they
  ſow. 11.~With his owne hand he writeth, telling them, the true cauſe
  why thoſe falſe Apoſtles preach Circumciſion, to be only to pleaſe the
  Iewes: 17.~and a plaine argument that he preacheth it not, to be this,
  that he is perſecuted of the Iewes.}

Brethren, and if a man be preoccupated in any fault, you that are
ſpiritual, inſtruct ſuch an one in the ſpirit of lenitie, conſidering
thine owne ſelf, leſt thou alſo be tempted. \V Beare ye one anothers
burdens: and ſo you ſhal fulfil the law of Chriſt. \V For if any man
eſteeme himſelf to be ſome-thing, wheras he is nothing, he ſeduceth
himſelf. \V But let euery one proue his owne worke, and ſo in himſelf
only ſhal he haue the glorie and not in another. \V For euery one ſhal
beare his owne burden. \V And let
\CNote{\XRef{1.~Cor.~9,~7.}}
him that is cathechized in the word,
%%% !!! unmarked in either
\LNote{Communicate.}{The
\MNote{Duety to our ſpiritual Teachers.}
great duety & reſpect that we ought to haue to ſuch as preach or teach
vs the Cath. faith: and not in regard only of their paines taken with
vs, and wel-deſeruing of vs by their doctrine; but that we may be
partakers of their merits, we ought ſpecially to doe good to ſuch, or
(as the Apoſtle ſpeaketh) communicate with them in al our temporal
goods, that we may be partakers of their ſpiritual. See 
\Cite{S.Auguſtin li.~2. Euang. quæſt. q.~8.}}
communicate to him that catechizeth
him, in al his goods. \V Be not deceiued, God is not mocked. \V For what
thĩgs a mã ſhal ſow, thoſe alſo shal he reap. For he that ſoweth in his
flesh, of the flesh alſo ſhal reap corruption. But he that ſoweth in the
ſpirit, of the ſpirit ſhal reap life euerlaſting. \V And
\CNote{\XRef{2.~Theſ.~3,~13.}}
doing good, let vs not faile. For in due time we ſhal
\SNote{The workes of mercie be the ſeed of life euerlaſting, and the
proper cauſe therof, and not faith only.}
reap
%%% o-2611
not failing. \V Therfore whiles we haue time, let vs worke good to al,
but
\LNote{Eſpecially.}{In
\MNote{In almes whom to preferre.}
giuing almes, though, we may doe wel in helping al that are in
neceſſitie, as farre as we can, yet we are more bound to ſuccour
Chriſtians, then Iewes or Infidels; and Catholikes, then Heretikes. See 
\Cite{S.~Hierom q.~1. ad Hedibiam.}}
eſpecially to the domeſticals of the faith.

\V See with what manner of letters I haue written to you with mine owne
hand. \V Whoſoeuer wil pleaſe in the flesh, they force you to be
circumcized, only that they may not ſuffer the perſecution of the croſſe
of Chriſt. \V For neither they that are circumciſed, doe keep the Law:
but they wil haue you to be circumciſed, that they may glorie in your
flesh. \V But
\SNote{Chriſt (ſaith S.~Auguſtin) choſe a kind of death, to hang on the
Croſſe, and to fixe or faſten the ſame croſſe in the foreheads of the
faithful; that the Chriſtian may ſay, God forbid that I should glorie
ſauing in the croſſe of our Lord \Sc{Iesvs Christ}.
\Cite{Expoſ. in Euang. Io. tract.~43.}}
God forbid that I ſhould glorie, ſauing in the croſſe of our Lord
\Sc{Iesvs} Chriſt; by whom the world is crucified to me, and I to the
world. \V For in Chriſt \Sc{Iesvs} neither Circumciſion auaileth ought,
nor Prepuce, but
\LNote{A new creature.}{Note
\MNote{Iuſtice an inherent qualitie in vs.}
wel that the Apoſtle calleth that here a new creature, which in the
\XRef{laſt chapter}
he termed, \Emph{faith working by charitie}, &
\XRef{(1.~Cor.~7,~19.)}
\Emph{the obſeruation of the commandements of God}. Wherby we may learne
that vnder the name of faith, is conteined the whole reformation of our
ſoules and our new creation in good workes: and alſo that Chriſtian
iuſtice is a very qualitie, condition, & ſtate of vertue and grace
reſident in vs, and not a phantaſtical apprehenſion of Chriſt's iuſtice
only imputed to vs.
\MNote{Faith with the other vertues is the formal cauſe of
iuſtification.}
Laſtly, that the faith which iuſtifieth, ioyned with the other vertues,
is properly the formal cauſe, & not the efficient or inſtrumental cauſe
of iuſtification: that is to ſay, theſe vertues put together, being the
effect of God's grace, be our new creature and our iuſtice in Chriſt.}
a new creature. \V And whoſoeuer ſhal follow this rule, peace vpon them,
and mercie, and vpon the Iſrael of God. \V From hence-forth let no man
be troubleſome to me. For I beare the markes of our Lord \Sc{Iesvs} in
my body. \V The grace of our Lord \Sc{Iesvs} Chriſt be with your ſpirit,
Brethren. Amen.


\stopChapter


\stopcomponent


%%% Local Variables:
%%% mode: TeX
%%% eval: (long-s-mode)
%%% eval: (set-input-method "TeX")
%%% fill-column: 72
%%% eval: (auto-fill-mode)
%%% coding: utf-8-unix
%%% End:

