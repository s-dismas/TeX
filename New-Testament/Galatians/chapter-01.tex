%%%%%%%%%%%%%%%%%%%%%%%%%%%%%%%%%%%%%%%%%%%%%%%%%%%%%%%%%%%%%%%%%
%%%%
%%%% The (original) Douay Rheims Bible 
%%%%
%%%% New Testament
%%%% Galatians
%%%% Chapter 01
%%%%
%%%%%%%%%%%%%%%%%%%%%%%%%%%%%%%%%%%%%%%%%%%%%%%%%%%%%%%%%%%%%%%%%




\startcomponent chapter-01


\project douay-rheims


%%% 2738
%%% o-2595
\startChapter[
  title={Chapter 1}
  ]

\Summary{After the foundation laid in the ſalutation, 6.~he exclaimeth
  againſt the Galatians, & their Falſe-Apoſtles, 11.~conſidering that
  the Ghoſpel which he preached to them, he had it immediately of Chriſt
  himſelf. 13.~Which to shew he beginneth to tel the ſtorie of his
  conuerſion and preaching ſince then, and that as he learned nothing of
  the other Apoſtles, ſo yet he had their approbation.}

Paul an Apoſtle not of men,
\LNote{Neither by man.}{Though
\MNote{S.~Paul ſent to preach by ordinarie impoſition of hands.}
he were not firſt by man's election, nomination, or aſſignment, but by
God's owne ſpecial appointment, choſen to be an Apoſtle; yet by the like
expreſſe ordinance of God he tooke orders or impoſition of hands of men,
as is plaine.
\XRef{Act.~18.}
Let vs beware then of ſuch falſe Apoſtles, as now a-daies intrude
themſelues to the office of Miniſterie and preaching, neither called of
God, nor rightly ordered of men.}
neither by man, but by \Sc{Iesvs} Chriſt, and God the Father that raiſed
him from the dead, \V and al the Brethren that are with me; to the
Churches of Galatia. \V Grace to you and peace from God the Father and
our Lord \Sc{Iesvs} Chriſt, \V who gaue himſelf for our ſinnes, that he
might deliuer vs from this preſent wicked world, according to the wil of
our God and Father: \V to whom is glorie for euer and euer. Amen.

\V I maruel that thus ſo ſoon you are transferred from him that called
you into the grace of Chriſt, vnto another Ghoſpel: \V which is not
another, vnles there be ſome that trouble you, and wil
\SNote{New Ghoſpellers that peruert, corrupt, or alter the one only true
and firſt deliuered Ghoſpel, are to be auoided. See
\Cite{S.~Auguſtin Cont. Fauſtum. li.~32. c.~27.}}
inuert the Ghoſpel of Chriſt. \V But although we,
\LNote{Or an Angel.}{Manie
\MNote{No shew of learning or vertue muſt moue vs from the faith.}
worthie obſeruations are made in the Fathers writings, of the earneſt
admonition of the Apoſtle, and much may we gather of the text
it-ſelf. Firſt, that the credit of any mã or Angel, for what learning,
eloquẽce, shew of grace or vertue ſoeuer, though he wrought miracles,
should not moue a Chriſtian man from that truth which he hath once
receiued in the Catholike Church: of which point Vicentius Lirinenſis
excellently 
\Fix{trateth}{treateth}{obvious typo, fixed in other}
\Cite{li. cont. profan. hæreſ. Nouitates.}
Whereby we may ſee that it is great pitie and shame, that ſo many follow
Luther & Caluin & ſuch other leud fellowes, into a new Ghoſpel, which
are ſo farre from Apoſtles and Angels, that they are not any whit
comparable with the old Heretikes in guifts of learning or eloquence,
much leſſe in good life.

Secondly
\MNote{Preaching cõtrarie to the faith receiued is forbiden, not other
preaching.}
S.~Auguſtin
\CNote{\Cite{Tract. 98. in Ioan.}}
noteth vpon the word, \Emph{Beſide}, that not al other teaching, or more
preaching then the firſt, is forbidden, but ſuch as is contrarie and
diſagreeing to the rule of faith. \Emph{The Apoſtle did not ſay}, ſaith
he, \Emph{If any man euangelize to you more then you haue receiued, but,
beſide that you receiued. For if he should ſay that, he should be
preiudicial to himſelf, who coueted to come to the Theſſalonians, that
he might ſupply that which was wanting to their faith. Now he that
ſupplieth, addeth that which was lacking, taketh not away that which
was, &c.} By which we ſee how friuolouſly and calumniouſly the Heretikes
charge the Church with addition to the Scriptures.

Thirdly,
\MNote{The Ghoſpel is not only in the written word of Scripture, but in
vnwrittẽ tradition alſo.}
as wel by the word \L{euangelizamus} (we euangelize) as the word
\L{accepiſtis} (you haue receiued) we may note that the firſt truth,
againſt which no ſecond Ghoſpelling or doctrine may be admitted, is not
that only which he wrote to the Galatians, or which is conteined either
in his or any other of the Apoſtles or Euangeliſtes writings, but that
which was by word of mouth alſo preached, taught, or deliuered them
firſt, before he wrote to them. Therfore the Aduerſaries of the Church
that meaſure the word of God or Ghoſpel by the Scriptures only, thinking
themſelues not to incurre S.~Paules curſe, except they teach directly
againſt the written word, are fouly beguiled. As therin alſo they
shamefully erre, when they charge the Catholikes with adding to the
Ghoſpel, when they teach any thing that is not in expreſſe words written
by the Apoſtles or Euangeliſts: not marking that the Apoſtle in this
Chapter, and els-where commonly calleth his & his fellowes whole
preaching, the Ghoſpel, be it written or vnwritten.

Fourthly,
\MNote{After-preaching & ouer-ſowing of nouelties, argueth falſe
doctrine.}
by the ſame words we ſee condemned al after-preaching, later doctrines,
new ſects and Authours of the ſame: that only being true, which was
firſt by the Apoſtles and Apoſtolike men as the lawful husband-men of
Chriſtes field, ſowed and planted in the Church: and that falſe, which
was later and as it were ouer-ſowen by the enemie. By which rule not
only Tertullian
\Cite{(de preſcrip. nu.~6. &~9.)}
but al other ancient Doctours, and ſpecially S.~Ireneus
\Cite{(li.~3. c.~2.~3.~4.)}
tried truth from falſehood, & condẽned old Heretikes, prouing Marcion,
Valentine, Cerdon, Menander, and ſuch like falſe Apoſtles, becauſe they
came in with their nouelties long after the Church was ſettled in former
truth.

Fifthly,
\MNote{The Apoſtles curſe vpon al that teach new doctrine, and draw men
from the Cath. Church.}
this curſe or execration pronounced by the Apoſtle, toucheth not only
the Galatians, or thoſe of the Apoſtles time, that preached otherwiſe
then they did, but it perteineth to al times, Preachers, and Teachers,
vnto the worlds end: and it concerneth them (as Vincentius Lirinenſis
ſaith) that preach a new faith, or change that old faith which they
receiued in the vnitie of the Catholike Church.
\CNote{\Cite{Li. cont. proph. hær. nouit.}}
\Emph{To preach any thing to Chriſtian men} (ſaith he) \Emph{beſides
that which they haue receiued, neuer was it lawful, neuer is it, nor
neuer shal it be lawful. To ſay anathema to ſuch, it hath been, & is,
and shal be alwayes behooful.} So S.~Auguſtin by this place holdeth al
accurſed, that draw a Chriſtian man from the ſocietie of the whole
Church, to make the ſeueral part of any one ſect: that cal to the hidden
conuenticles of heretikes, from the open & knowen Church of Chriſt: that
allure to the priuate, from the common: finally al that draw with
chatting curioſitie the children of the Catholike Church, by teaching
any thing beſides that they found in the Church.
\Cite{ep.~48. Pſal.~103. Con.~2.}
\CNote{\Cite{Aug. ep.~165.}}
mentioning alſo that a Donatiſt feined an Angel to haue admonished him to
cal his freind out of the Communion of the Catholike Church into his
ſect. And he ſaith, that if it had been an Angel indeed, yet should he
not haue heard him.
\MNote{Zeale againſt heretikes.}
Laſtly S.~Hierom vſeth this place, wherein the
Apoſtle giueth the curſe or anathema to al falſe teachers not once
but twiſe, to proue that the zeale of Catholike men ought to be ſo great
toward al Heretikes, and their doctrines, that they should giue them the
anathema, though they were neuer ſo deare vnto them. In which caſe,
ſaith this holy Doctour, I would not ſpare mine owne parents.
\Cite{Ad Pammach. c.~3. cont. Io. Hieroſ.}}
or an Angel from Heauen, euangelize to you beſide that which we haue
euangelized to you, be he anathema.  \V As we haue ſaid before, ſo now I
ſay againe: If any euangelize to you, beſide that which you haue
receiued, be he anathema. \V For doe I now vſe perſuaſion to men, or to
God? Or doe I ſeeke to pleaſe men? If I yet did pleaſe men, I ſhould not
be the ſeruant of Chriſt.

\V For I doe you to vnderſtand, Brethren, the Ghoſpel that
%%% o-2596
was euangelized of me, that it is not according to man. \V For neither
did I receiue it of man, nor learne it; but by the reuelation of
\Sc{Iesvs} Chriſt.

\V For you haue heard my conuerſation ſometime in Iudaiſme,
\CNote{\XRef{Act.~9,~1.}}
that aboue meaſure I perſecuted the Church of God, and expugned it, \V
and profited in Iudaiſme aboue many of mine equales in my Nation, being
%%% 2739
more abundantly an emulatour of the traditions of my Fathers. \V But
when it pleaſed him that ſeparated me from my mothers womb,
\CNote{\XRef{Act.~9,~3.}}
and called me by his grace, to reueale his Sonne in me, \V that I ſhould
euangelize him among the Gentils, incontinent I condeſcended not to
fleſh and bloud, \V neither came I to Hieruſalem to the Apoſtles my
Anteceſſours: but I went into Arabia, and againe I returned to
Damaſcus. \V Then, after three yeares I came to Hieruſalem
\LNote{To ſee Peter.}{In
\MNote{S.~Paul viſit S.~Peter of honour and reuerence toward him.}
what eſtimation S.~Peter was with this Apoſtle, it appeareth: ſeeing for
reſpect and honour of his perſon, and of duety as Tertullian
\Cite{de Præſcript.}
ſaith (notwithſtanding his great affaires Eccleſiaſtical) he went ſo
farre to ſee him: not in vulgar manner, but (as S.~Chryſoſtom noteth the
\TNote{ἱϛορῆσαι}
Greeke word to import) to behold him as men behold a thing or perſon of
name, excellencie, and maieſtie. For which cauſe, and to fill himſelf
with the perfect vew of his behauiour, he abode with him fifteen
daies. See  S.~Hierom
\Cite{ep.~101. ad Paulinum to.~3.}
who maketh alſo a myſterie of the number of daies that he taried with
S.~Peter. See S.~Ambroſe
\Cite{in Comment. huius loca},
and S.~Chryſoſtome
\Cite{vpon this place}
and
\Cite{ho.~87. in Ioan.}}
to ſee Peter: and taried with him fifteen daies. \V But other of the
Apoſtles ſaw I none; ſauing Iames
\SNote{S.~Iames was called our Lordes brother after the Hebrew phraſe of
the Iewes, by which neer kinſmen are called Brethrẽ: for they were not
Brethren indeed, but rather ſiſters children.}
the brother of our Lord. \V And the things that I write to you; behold
before God, that I lie not. \V After that I came into the parts of Syria
and Cilicia. \V And I was vnknowen by ſight to the Churches of Iewrie,
that were in Chriſt: \V but they had heard only, That he which perſecuted
vs ſometime, doth now euangelize the faith which ſometime he
expugned: \V and in me they glorified God.


\stopChapter


\stopcomponent


%%% Local Variables:
%%% mode: TeX
%%% eval: (long-s-mode)
%%% eval: (set-input-method "TeX")
%%% fill-column: 72
%%% eval: (auto-fill-mode)
%%% coding: utf-8-unix
%%% End:

