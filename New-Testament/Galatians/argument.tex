%%%%%%%%%%%%%%%%%%%%%%%%%%%%%%%%%%%%%%%%%%%%%%%%%%%%%%%%%%%%%%%%%
%%%%
%%%% The (original) Douay Rheims Bible 
%%%%
%%%% New Testament
%%%% Galatians
%%%% Argument
%%%%
%%%%%%%%%%%%%%%%%%%%%%%%%%%%%%%%%%%%%%%%%%%%%%%%%%%%%%%%%%%%%%%%%




\startcomponent argument


\project douay-rheims


%%% 2737
%%% o-2594
\startArgument[
  title={\Sc{The Argvment of Epistle of S.~Pavl to the Galatians.}},
  marking={Argument of Galatians}
  ]

That this Epiſtle may ſeeme to be the firſt that S.~Paul wrote, was
declared in the argument of the Epiſtle to the Romanes; notwithſtanding
that in the
\XRef{ſecond chapter}
it is euident to haue been written 14.~yeares at the leaſt after his
Conuerſion, and (as it is ſaid) from Epheſus, belike at that time of his
being there, which is mentioned
\XRef{Act.~18.}

The occaſion of it were ſuch falſe-apoſtles, as we read of,
\XRef{Act.~15.}
\L{Et quidam deſcendentes, &c.} \Emph{And certaine comming downe from
Iewrie, taught the Brethren} (that is the Chriſtian Gentils at Antioch)
\Emph{that vnles you be circumciſed according to the manner of Moyſes,
you can not be ſaued.} Such commers alſo to the Galatians (whom S.~Paul had
conuerted
\XRef{Act.~16.}
as himſelf mentioneth
\XRef{Gal.~1.}
and
\XRef{4.)}
did ſeduce them, ſaying, that al the other Apoſtles to whom they should
harken, then to Paul (who came they knew not from whence) did vſe
Circumciſion: yea and that Paul himſelf, when he came among them, durſt
doe none other. And to winne them more eaſily, they did not lay on them
the burden of the whole Law, but of Circumciſion only.

Againſt theſe deceiuers, S.~Paul declareth, that he receiued his
Apoſtleship and learned the Ghoſpel that he preacheth, of Chriſt himſelf
after his Reſurrection: and that the other Apoſtles (although he learned
nothing of them) receiued him into their ſocietie, and allowed wel of
his preaching to the Gentils, though themſelues being Iewes, and liuing
among the Iewes, had not yet left the ceremonies of the Law: howbeit
they did not put in them any hope of iuſtification, but in Chriſt alone
without them. He declareth moreouer that the ſaid Falſe-apoſtles belyed
him, in ſaying that he alſo preached Circumciſion ſometimes. Againe,
that they themſelues in preaching no more but Circumciſion, did againſt
the nature of Circumciſion, becauſe it is a profeſsion to obſerue the
whole Law: finally, whatſoeuer they pretended, that indeed they did it
only to pleaſe the Iewes, of whom otherwiſe they should be perſecuted.

So that in this Epiſtle he handleth the ſame matter, which in the
Epiſtle to the Romanes: but here leſſe exactly and more briefly, becauſe
the Galatians were very rude, and the Romanes contrariewiſe, \L{repleti
omni ſcientia}
\XRef{(Rom.~15.)}
\Emph{replenished with al knowledge}.


\stopArgument


\stopcomponent


%%% Local Variables:
%%% mode: TeX
%%% eval: (long-s-mode)
%%% eval: (set-input-method "TeX")
%%% fill-column: 72
%%% eval: (auto-fill-mode)
%%% coding: utf-8-unix
%%% End:
