%%%%%%%%%%%%%%%%%%%%%%%%%%%%%%%%%%%%%%%%%%%%%%%%%%%%%%%%%%%%%%%%%
%%%%
%%%% The (original) Douay Rheims Bible 
%%%%
%%%% New Testament
%%%% Galatians
%%%% Chapter 04
%%%%
%%%%%%%%%%%%%%%%%%%%%%%%%%%%%%%%%%%%%%%%%%%%%%%%%%%%%%%%%%%%%%%%%

%%% Latin checked by KK.



\startcomponent chapter-04


\project douay-rheims


%%% 2747
%%% o-2604
\startChapter[
  title={Chapter 4}
  ]

\Summary{That the Law was fit for the time of nonnage: but being now
  come to ful age, to deſire ſuch ſeruitude is abſurd, ſpecially for
  Gentils. 12.~And that he writeth this not of any diſpleaſure, but to
  tel them the truth, remembring how paſsingly they honoured him when he
  was preſent, and exhorting them therfore not to harken to the falſe
  Apoſtles in his abſence. 21.~By the allegorie alſo of Abraham's two
  ſonnes, shewing, that the children of the Iewes Synagogue shal not
  inherit, but we who are the children of the free-woman; that is of the
  Cath. Church of Chriſt.}

And I ſay, as long as the heire is a litle one, he differeth nothing
from a ſeruant, although he be Lord of al, \V but is vnder tutours and
gouernours vntil the time limited of the Father: \V ſo we alſo, when we
were litle ones, were
\LNote{Seruing.}{There
\MNote{External worship of God by vſe of creatures, neceſſarie: & how
the Heathen, Iewes, & Chriſtians differ in the ſame.}
can be no external worship of God nor aſſociation of men in religion,
either true or falſe, without the vſe of corporal things or
elements. The Heathen ſo vſed the creatures of elements that they ſerued
them as their Gods. The Iewes, of whom the Apoſtle here ſpeaketh, ſerued
not the creatures themſelues which they occupied in their ceremonies,
but they ſerued the only true God vnder the elements: that is to ſay,
being ſeruilely clogged, yoked, kept occupied & in awe, eith innumerable
fleshly, groſſe, & comberſom offices about creatures. The Chriſtians
neither ſerue elements, as the one, nor be kept in ſeruile thraldom
thereby as the other; but occupie only a few exceeding eaſie, ſweet,
ſeemely, and ſignificant, for an agreable exerciſe both of body and
mind. Wherof S.~Auguſtin ſaith thus,
\Cite{li.~3. c.~9. de doct. Chriſt.}
\MNote{The vſe of external elemẽts in the Sacraments.}
\Emph{Some few for many, moſt eaſie to be done, moſt honourable for
ſignification, and moſt cleane & pure for to be obſerued and kept, hath
our Lord himſelf and the Apoſtolical diſcipline deliuered.} And
\Cite{li. de ver. relig. c.~17.}
\Emph{Of the Wiſedom of God it-ſelf man's nature being taken, whereby we
were called into libertie, a few Sacraments moſt holſom were appointed
and inſtituted, which might conteine the ſocietie of Chriſtian people,
that is, of the free multitude vnder one God.}
And againe,
\Cite{cont. Fauſt. li.~19. c.~13.}
\Emph{The Sacraments are changed: they are made eaſier, fewer,
holſommer, happier.} The ſame he hath in the
\Cite{118.~epiſtle c.~1.}
and
\CNote{\Cite{ep.~118. c.~1.}
&
\Cite{in Pſ.~103. conc.~1.}}
many other places beſides. By which you may ſee, it is not al one to vſe
elements, viſible Sacraments or ceremonies, and to ſerue them as the
Pagans doe, or to ſerue vnder them as the Iewes did; wherewith the
Heretikes calumniouſly charge the Chriſtians.
\MNote{Our Sacramẽts few & eaſie, in reſpect of the Iewes.}
And as touching the ſmal number, facilitie, efficacie, and
ſignification, wherin the ſaid holy Father putteth the ſpecial
difference; who ſeeth not that for ſo many buſie Sacrifices, we haue but
one: for Sacraments wel-neer infinit, but ſeuen: al ſo eaſie, ſo ful of
grace, ſo ſignificant, as can be poſſible, as of euery one in their
ſeueral places is proued?

Here,
\MNote{S.~Auguſtin falſely alleaged of the Heretikes for two Sacraments
only.}
let the good Reader take heed of a double deceit vſed by the Aduerſaries
about S.~Auguſtines places alleaged. Firſt, in that they ſay he made but
two Sacraments, which is vntrue. For, although treating of the
difference between the Iewish Sacraments and ours, he namely giueth
example in Baptiſme and the Euchariſt (as ſometimes alſo for example he
nameth but one) yet he hath no word nor ſigne at al that there should be
no moe. But contrariewiſe in the foreſaid
\Cite{epiſtle~118.}
he inſinuateth, that beſides thoſe two, there be other of the ſame ſort
in the Scriptures.
\MNote{The other Sacraments proued out of S.~Auguſtin.}
Yea, with water and bread, which be the elements of the two foreſaid
Sacraments, he expreſly nameth oile alſo
\Cite{(li.~2. cont. lit. Petil. c.~104.)}
the element or matter of the Sacrament of Confirmation: which in the
ſame place he maketh to be a Sacrament as Baptiſme is. So doth he
affirme of the Sacrament of Orders
\Cite{li. de bapt. c.~1.}
and alſo of Matrimonie
\Cite{li. de bono. coniug. c.~14.}
of Penance likewiſe he ſpeaketh as of Baptiſme, which he calleth
Reconciliation,
\Cite{li.~1. de adult. coniug. c.~28.}
Laſtly by the booke
%%% !!! Things get confuſing here:
\Cite{de viſitatione infirmorum in S.~Auguſtin li.~2. c.~4.}
by
\Cite{Proſper de prædictionibus p.~2. c.~19.}
\Cite{S.~Innocentius ad Eugubinum to.~1. Conc. ep. ad Eugub. c.~8.}
\Cite{S.~Cyril, li.~2. in Leuiticum}
and
\Cite{S.~Chryſoſtom li.~3. de Sacerdotio},
\CNote{\Cite{Aug. ſer.~225. de temp.}
&
\Cite{de rectis. cath. conuerſ.}}
Extreme vnction is proued to be a Sacrament. It is falſe then that the
Heretikes affirme of S.~Auguſtin, by whoſe doctrine it is plaine that
though the elements or Sacraments of the new law be but few and very few
in compariſon of thoſe in the old law, yet there be no fewer then ſeuen
ſpecified by him. Which number of ſeuen the holy Councels of Florence
and Trent doe expreſly define to haue been inſtituted by Chriſt, againſt
theſe late Heretikes. See more of theſe Sacraments in their places,
\XRef{Act.~8.}
\XRef{1.~Tim.~4.}
\XRef{Io.~20.}
\XRef{Ia.~5.}
\XRef{Eph.~5.}

The
\MNote{S.~Auguſtin falſely alleaged againſt the ceremonies of the
Church.}
other forgerie of the Aduerſaries concerning the elements or ceremonies,
is, that S.~Auguſtin
\Cite{(ep.~119. c.~19.)}
should affirme, that the Church and Chriſtian people in his daies
(whervpon they inferre that it is ſo much more now) were ſo loaden with
obſeruation of vnprofitable ceremonies, that they were in as great
ſeruilitie and ſubiectiõ to ſuch things as the Iewes. He ſaith ſo indeed
of ſome particular preſumptions, inuentions, and vſages of certaine
perſons; as that ſome made it a heinous matter to touch the ground with
their bare feet within their own octaues, & ſuch like vanities. Wherby
ſome ſimple folkes might be infected, which this holy Doctour ſpecially
miſliked, & wisheth ſuch things (as they may, without ſcandal) to be
taken away. But that he wrote or meant ſo of any ceremonie that the
Church vſeth, either appointed by Scripture, or Councel, or cuſtom of
the Catholike Church, himſelf denieth it in expreſſe termes in the ſame
place, and in ſundrie other: where he alloweth al the holy ceremonies
done in the miniſtration of the Sacraments and els-where. Whereby it is
cleare, that the Churches moſt comely orders and ſignificant rites
pertaine not to the yoke of the old law, much leſſe to the ſuperſtition
of Gentilitie, as Heretikes affirme; but to the ſweet yoke of Chriſt and
light burden of his law, to order, & decencie, and inſtruction of the
faithful, in al libertie, loue, faith, grace, and Spirit.}
ſeruing vnder the
\SNote{That is, the rudiments of religion, wherin the carnal Iewes were
trained vp: or the corporal creatures, wherin their manifold Sacrifices,
Sacraments, & rites did conſiſt.}
elements of the world. \V But when the fulnes of time came, God ſent his
Sonne made of a woman, made vnder the Law: \V that he might redeem them
that were vnder the Law; that we might receiue the adoption of
ſonnes. \V And becauſe you are ſonnes,
\CNote{\XRef{Ro.~8,~15.}}
God hath ſent the Spirit of his ſonne into your harts, crying: Abba,
Father. \V Therfore now he is not a ſeruant, but a ſonne. And if a
ſonne, an heire alſo by God. \V But then indeed not knowing God, you
ſerued them that by nature are not Gods. \V But now when you haue knowen
God, or rather are knowen of God, how turne you againe to the
\LNote{Weake and poore.}{Whether he meane of the creatures which the
Gentils ſerued, (as it my ſeeme by the words before of ſeruing ſtrange
Gods) ſo the elements were moſt baſe and beggerly; or of the Iudaical
ceremonies and ſacraments (as moſt expound it) euen ſo their elements
were weake and poore in themſelues, not giuing life, ſaluation, and
remiſſion of ſinnes, nor being inſtruments or veſſels of grace, as the
7.~Sacraments of the new law be.}
weake & poore elements, which you wil ſerue againe? \V
\LNote{You obſerue daies.}{That
\MNote{The Heathenish and Iudaical obſeruatiõ of daies Heretically
compared with the Chriſtian obſeruation of feſtiuities and holy-daies,
&c.}
which S.~Paul ſpeaketh againſt the Idololatrical obſeruation of daies,
months, and times, dedicated by the Heathen to their falſe Gods, and to
wicked men or ſpirits, as to Iupiter, Mercurie, Ianus, Iuno, Diana, and ſuch
like, or againſt the ſuperſtitious differences of daies, fatal,
fortunate, or diſmol, and other obſeruations of times for good luck or
il luck in man's actions, gathered either by particular fanſie, or
popular obſeruation, or curious & vnlawful arts, or (laſtly) of the
Iudaical feſtiuities that were then ended & abrogated, vnto which
notwithſtanding certain Chriſtiã Iewes would haue reduced the
Galatians againſt the Apoſtles doctrine: al that (I ſay) doe the
Heretikes of our time falſely and deceitfully interpret againſt the
Chriſtian holy-daies, & the ſanctificatiõ & neceſſarie keeping of the
ſame. Which is not only cõtrarie to the Fathers expoſition, but againſt
the very Scriptures, and the practice of the Apoſtles and the whole
Church. 
\Cite{Aug. cont. Adimant. c.~16.}
\Cite{Ep.~118. c.~7.}
\Cite{Hiero. in hunc locum.}
\MNote{Sunday, Eaſter, Whitſuntide.}
In the 
\XRef{Apocalypſe c.~1.}
there is plaine mention of the Sunday, that is, our Lordes
day \L{(Dominicus dies)} into which the Iewes Sabboth was altered, their
Paſch into our Eaſter, their Pentecoſt into our Whitſontide: which were
ordained & obſerued of the Apoſtles themſelues.
%%% !!! Where do these go?
\CNote{\Cite{Orig. ho.~3. in diu.}
\Cite{Aug. ep.~28.}
&
\Cite{Ser. de Sanſtis. Fulgent Leo.}}
\MNote{The feſtiuities of Chriſt.}
And the antiquitie of the feaſts of Chriſtes Natiuitie, Epiphanie, &
Aſcenſion is ſuch, that they can be referred to no other origine but the
Apoſtles inſtitution: who (as S.~Clement teſtifieth
\Cite{li.~8. conſt. Apoſt. c.~39.)}
gaue order for celebrating their fellow Apoſtles, S.~Steuens & other
Martyrs daies after their death: and much more no doubt did they giue
order for Chriſtes feſtiuities.
\MNote{Other holy-dayes of Saints.}
According to which, the Church hath kept not only his, but S.~Steuens, &
the B.~Innocents, euen on the ſame daies they be now ſolemnely kept, &
his B.~Mothers, & other Saints, (as the Aduerſaries themſelues confeſſe)
aboue 1300.~yeares, as appeareth in the Barbarous combates betweene
Weſtphalus the Lutheran, & Caluin, & by the writings betwixt the
Puritans & Proteſtants.

For which purpoſe, ſee alſo how old the holy-day of S.~Polycarpe is in 
\Cite{Euſeb. li.~4. c.~14.}:
\MNote{Feſtiuities of our B.~Ladie.}
of the
\CNote{See the
\XRef{Annot. Act.~1. v.~14.}}
Aſſumption of our Ladie or her dormition in S.~Athanaſius, S.~Auguſtin,
S.~Hierom, S.~Damaſc, and both of that feaſt and of her Natiuitie in
S.~Bernard, who profeſſeth \Emph{he receiued them of the Church, & that
they ought to be moſt ſolemnely kept.}
\Cite{ep.~174.}
Wherin we can not but wonder at the new Church of England, that (though
againſt the pure Caluiniſtes wil and doctrine) keep other Saints and
Apoſtles daies of their death, and yet haue abolished this ſpecial feaſt
of our Ladies departure, which they might keep, though they beleeued not
her Aſſumption in body (wherof yet
\CNote{\Cite{ep. ad Timoth.}}
S.~Denys giueth ſo great teſtimonie) being aſſured she is departed at
the leaſt: except they either hate her, or thinke her worthy of leſſe
remembrance then any other Saint,
\CNote{Luc.~1. v.~48.}
herſelf prophecying the contrarie of al Catholike Generations, that they
should bleſſe her.
And indeed the Aſſumption is her proper day, as alſo
the feaſt of her Natiuitie:
%%% !!! Where does this go?
\MNote{See
\Cite{S.~Grego. li.~7. ep.~29.}
of Martyrs feaſts al the yeare, & Maſſes in the ſame.}
the other of the Purification and the
Annunciation, which they keep in England, being not ſo peculiar to her,
but belonging rather to Chriſtes Preſentation in the Temple, and his
Conception.
To conclude, we may ſee in
\Cite{S.~Cyprian. ep.~34.}
\Cite{Origen ho.~3. in diuerſ.}
\Cite{Tertulian de cor. mil.}
\Cite{S.~Gregorie Nazianzen de amoure pauperum},
\CNote{Conc. Gang. c.~20.}
\Cite{the Councel of Ganges},
yea and in the
\Cite{Councel of Nyce}
it-ſelf giuing ordeer for Eaſter and the certaine celebrating therof,
that Chriſtian Feſtiuities be holy, ancient, & to be obſerued on
preſcript daies and times, and that this is not Iudaical obſeruation of
daies as Aërius taught, for which he was condemned of Hereſie, as
S.~Epiphanius
\CNote{Epiph. Hær.~75.}
witneſſeth. But of holy-daies S.~Auguſtin sheweth both the reaſon and
his liking, in theſe memorable words. Firſt for the feaſts belonging to
our Lord, thus:
\CNote{Aug. de Ciuit. Dei li.~10. c.~16.}
\MNote{S.~Auguſtines words of Feſtiuities and holy-daies.}
\Emph{We dedicate and conſecrate the memorie of God's benefits with
ſolemnities, feaſts, and certaine appointed daies, leſt by tract of
times there might creep in ingrateful and vnkind obliuion.} Of the
feſtiuities of Martyrs thus: \Emph{Chriſtian people celebrate the
memories of Martyrs with religious ſolemnitie, both to moue themſelues
to imitation of them, and that they may be partakers of their merits,
and be holpen with their praiers.}
\Cite{Cont.~Fauſt. li.~20. c.~21.}
And of al Saints daies, thus: \Emph{Keep ye and celebrate with ſobrietie
the Natiuities of Saints, that we may imitate them which haue gone
before vs, and they may reioyce of vs which pray for vs.}
\Cite{In Pſ.~83. Conc.~2. in fine.}

And
\MNote{Preſcript faſting-daies.}
as is ſaid of preſcript daies of feaſts, ſo the like is to be ſaid
\CNote{\Cite{Hilar. prolog. in Pſal. explan.Epiph. hær..~75.}
&
\Cite{in fine li.~3. cõt. hær.}}
of faſts, which elſwhere we haue shewed to be of the Apoſtles
ordinance. And ſo alſo of the Eccleſiaſtical diuiſion of the yeare into
Aduent, Septuageſme, &c. the week into ſo many Feries,
\MNote{Canonical houres.}
the day into Houres of prayers, as the Prime, Third, the Sixth, the
None, &c. Wherof ſee
\CNote{\Cite{Cypri. de Orat. Do. nu.~15.}}
S.~Cyprian, who deriueth theſe things by the Scriptures from the
Apoſtles alſo, and counteth theſe things which the wicked Heretikes
reproue, to be ful of myſterie.
\MNote{Reading of the Scriptures according to the time of the yeare.}
Like vnto this alſo is it, that the holy Scriptures were ſo diſpoſed of,
and deuided, that certaine peeces (as is alwaies obſerued and practiſed
vntil this day) should be read at one time, & others at other times and
ſeaſons, throughout the yeare, according to the diuerſitie of our Lordes
actions and benefits, or the Saints ſtories then recorded. Which the
Puritane Caluiniſts alſo condemne of Superſtition, deſiring to bring in
hellish horrour and al diſorder. See
\Cite{Conc. Carthag.~3. c.~47.}
and
%%% !!! Fix me
\XRef{pag.~259 of this booke.}}
You obſerue daies, and months, and times, and yeares. \V I feare you,
%%% o-2605
leſt perhaps I haue laboured in vaine among you. \V Be ye as I, becauſe
I alſo am as you: Brethren, I beſeech you, you haue hurt me nothing. \V
And you know that by infirmitie of the fleſh I euangelized to you
heertofore: \V and your tentation in my fleſh you deſpiſed not, neither
reiected, but
\SNote{So ought al Catholike people receiue their Teachers in religion,
with al duetie, loue, and reuerence.}
as an Angel of God you receiued me, as Chriſt \Sc{Iesvs}. \V Where is
then your bleſſednes? for I giue you teſtimonie that if it could be
done, you would haue plucked out your eyes and haue giuen them to me. \V
Am I then become your enemie, telling you the truth? \V They emulate you
not wel: but they would exclude you, that you might emulate thẽ. \V But
doe you emulate the good in good alwaies: and not only when I am preſent
with you.

\V My litle children, whom I trauail withal againe, vntil Chriſt be
formed in you. \V And I would be with you now and change my voice:
becauſe I am confounded in you. \V Tel me, you that wil be vnder the
Law, haue you not read the Law? \V For it is written that
\CNote{\XRef{Gen.~16,~15.}
\XRef{21,~2.}}
Abraham had two ſonnes: one of the bond-woman, and one of the
free-woman. \V But he that of the bond-woman, was borne according to the
fleſh: and he that of the free-woman, by the promiſe. \V Which things
are ſaid
\LNote{By an allegorie.}{Here
\MNote{The Scriptures haue an allegorical ſenſe beſide the literal.}
we learne that the holy Scriptures haue beſide the literal ſenſe a
deeper ſpiritual and more principal meaning: which is not only to be
taken of the holy words, but of the very facts and Perſons reported:
both the ſpeaches and the actions being ſignificatiue ouer and aboue the
letter. Which pregnancie of manifold ſenſes if S.~Paul had not ſignified
himſelf in certaine places, the Heretikes had been leſſe wicked and
preſumptuous in condemning the holy Fathers allegorical expoſitions
almoſt wholy: who now shew themſelues to be mere brutish and carnal men,
hauing no ſenſe nor feeling of the profunditie of the Scriptures which
our holy Fathers the Doctours of God's Church ſaw.}
by an allegorie. For theſe are the two Teſtaments.
%%% 2748
The one from mount Sina, gendring vnto bondage; which is Agar, (\V for
Sina is a mountaine in Arabia,
\TNote{\G{συϛοιχεῖ}}
which hath affinitie to that which now is Hieruſalem) and ſerueth with
her children. \V But that Hieruſalem which is aboue, is free; which is
our mother. \V For it is written:
\CNote{\XRef{Eſ.~54,~1.}}
\Emph{Reioyce thou barren, that beareſt not: breake forth and crie, that
trauaileſt not: becauſe many are the children of the deſolate, more then
of her that hath a husband.} \V But
\CNote{\XRef{Ro.~9,~8.}}
we, Brethren, according to Iſaac, are the children of promiſe. \V But
\SNote{This mutual perſecution is a figure alſo of the Church iuſtly
perſecuting Heretikes, and contrariewiſe of Heretikes (which be the
children of the bond woman) vniuſtly perſecuting the Catholike Church.
\Cite{Aug. ep.~48.}}
as then he that was borne according to the fleſh, perſecuted him that
was after the ſpirit; ſo now alſo. \V But what ſaith the Scripture?
\CNote{\XRef{Gen.~21,~20.}}
\Emph{Caſt out the bond-woman and her ſonne. For the ſonne of the
bond-woman shal not be heire with the ſonne of the free-woman.} \V
Therfore, Brethren, we are not the children of the bond-woman, but of
the free: by the
\LNote{Freedom.}{He
\MNote{True Chriſtiã libertie.}
meaneth the libertie and diſcharge from the old ceremonies, Sacraments,
and the whole bondage of the Law, and from the ſeruitude of ſinne, and
the Diuel, to ſuch as obey him: but not libertie to doe what euery man
liſt, or to be vnder no obedience of ſpiritual or temporal lawes and
Gouerners: not a licence neuer to pray, faſt, keep holy-day, or
work-day, but when and how it ſeemeth beſt to euery man's
phantaſie. Such a diſſolute licentious ſtate is farre from the true
libertie which Chriſt purchaſed for vs.}
freedom wherwith Chriſt hath made vs free.


\stopChapter


\stopcomponent


%%% Local Variables:
%%% mode: TeX
%%% eval: (long-s-mode)
%%% eval: (set-input-method "TeX")
%%% fill-column: 72
%%% eval: (auto-fill-mode)
%%% coding: utf-8-unix
%%% End:

