%%%%%%%%%%%%%%%%%%%%%%%%%%%%%%%%%%%%%%%%%%%%%%%%%%%%%%%%%%%%%%%%%
%%%%
%%%% The (original) Douay Rheims Bible 
%%%%
%%%% New Testament
%%%% Galatians
%%%% Chapter 05
%%%%
%%%%%%%%%%%%%%%%%%%%%%%%%%%%%%%%%%%%%%%%%%%%%%%%%%%%%%%%%%%%%%%%%

%%% Latin checked by KK.



\startcomponent chapter-05


\project douay-rheims


%%% 2751
%%% o-2608
\startChapter[
  title={Chapter 5}
  ]

\Summary{Againſt the lie of the falſe Apoſtles, he proteſteth his mind
  of Circumciſion; 13.~and teſtifieth, that they are
  \Fix{calleth}{called}{obvious typo, fixed in other}
  to libertie. But yet leſt any miſconſter Chriſtian libertie, he
  telleth them that they shal not inherit the kingdom, vnles they
  abſtaine from the workes of the flesh, which are al mortal ſinnes; and
  doe the fruitful workes of the ſpirit, fulfilling al the commandements
  of the Law by Charitie.}

Stand, and be not holden in againe with the yoke of ſeruitude. \V Behold
I Paul tel you that if you be circumciſed, Chriſt ſhal profit you
nothing. \V And I teſtifie againe to euery man circumciding himſelf,
that he is a debter to doe the whole Law. \V You are euacuated from
Chriſt, that are iuſtified in the Law: you are fallen from grace. \V For
we in ſpirit, by faith, expect the hope of iuſtice. \V For in Chriſt
\Sc{Iesvs}
\CNote{\XRef{Gal.~6,~15.}}
neither circumciſion auaileth ought, nor prepuce: but
\LNote{Faith.}{This is the faith working by charitie, which S.~Paul
meaneth els-where, when he ſaith that faith doth iuſtifie. And note wel
that by theſe termes, Circumciſion and Prepuce not auailable to
iuſtification, it is plaine that in other places he meaneth the workes
of Circumciſion and Prepuce (that is, of the Iewes and Gentils) without
faith, which auaile not, but faith working by charitie: as who should
ſay, faith & good workes, not workes without faith.

Againe
\MNote{Iuſtification by faith only, diſproued by conference of
Scriptures.}
note here, that if the Proteſtants who pretend conference of places to
be the beſt or only way to explicate hard ſpeaches of the holy
Scriptures, had followed but their owne rule, this one text would haue
interpreted & cleared vnto them al other wherby iuſtice and
ſaluation might ſeeme to be attributed to faith alone: the Apoſtle here
ſo expreſly ſetting downe, the faith which he commendeth ſo much before,
not to be alone, but with charitie: not to be idle, but to be working by
Charitie; as S.~Auguſtin noteth.
\Cite{de fid. & op. c.~14.}
\MNote{How the Proteſtants admit charitie and good workes to
iuſtificatiõ.}
Further the good Reader
\Fix{moſt}{muſt}{obvious typo, fixed in other}
obſerue, that wheras the Proteſtants ſome of them confeſſe, that
Charitie and good workes be ioyned and requiſit alſo, and that they
exclude them not, but commend them highly, yet ſo that the ſaid Charitie
or good workes are no part of our iuſtice or any cauſe of iuſtification,
but as fruits and effects of faith only, which they ſay doth al, yea
though the other be preſent: this falſe gloſſe alſo is reproued
euidently by this place, which teacheth vs cleane contrarie: to wit,
that faith hath her whole actiuitie and operation toward iuſtice and
ſaluation, of charitie, and not contrariewiſe: without which it can not
haue any act meritorious or agreable to God for our ſaluation.
\MNote{Charitie is more principal then faith in iuſtification.}
For which cauſe S.~Auguſtin ſaith,
\Cite{li.~13. de Trin. c.~18.}
\L{Fidem non facit vtilem niſi charitas}, \Emph{nothing maketh faith
profitable  but charitie.} But the Heretikes anſwer, that where the
Apoſtle ſaith, \Emph{worketh by charitie}, he maketh charitie to be the
inſtrument only of faith in wel working, and therfore the inferiour
cauſe at the leaſt. But this alſo is eaſily refuted by the Apoſtles
plaine teſtimonie, affirming that charitie
\CNote{\XRef{1.~Cor.~13.}}
is the greater vertue, & that if a man had al faith & lacked charitie, he
were worth nothing. And againe,
\CNote{\XRef{Rom.~13.}
\XRef{1.~Tim.~1.}}
that Charitie is the perfection and accomplishment of the Law (as faith
is not) which can not agree to the inſtrumental or inferiour cauſe.
\MNote{How faith worketh by charitie.}
And therfore whẽ it is ſaid that faith worketh by charitie, it is not as
by an inſtrumẽt, but as the body worketh by the ſoul, the matter by the
forme, without which they haue no actiuitie. Wherupon the Schooles cal
Charitie, the forme or life of faith, that is to ſay, the force,
actiuitie, and operatiue qualitie therof, in reſpect of merit and
iuſtice. Which S.~Iames doth plainely inſinuate, when he maketh faith
without Charitie, to be as a dead corps without ſoul or life, and
therfore without profitable operation.
\XRef{c.~2. v.~26.}}
faith that worketh by charitie. \V You ranne wel, who hath hindred you
not to obey the truth? \V The perſuaſion is not of him that calleth you.
\V
\CNote{\XRef{1.~Co.~5,~6.}}
A litle leauẽ corrupted the whole paſte. \V I haue confidence in you in
our Lord, that you wil be of no other mind: but he that troubleth you,
ſhal beare the iudgement, whoſoeuer he be. \V And as for me, Brethren,
if as yet I preach circumciſion, why doe I yet ſuffer perſecution? then
is the ſcandal of the croſſe euacuated. \V I would they were alſo
cut-off that trouble you.

\V For you, Brethren, are called into libertie: only make not this
\LNote{Libertie an occaſion.}{They
\MNote{True libertie, not carnal and fleshly.}
abuſe the libertie of the Ghoſpel to the aduantage of their flesh, that
vnder pretenſe therof, shake of their obedience to the lawes of man, to
the decrees of the Church and Councels, that wil liue and beleeue as
they liſt, and not be taught by their Superiours, but fornicate with euery
Sect-maiſter that teacheth pleaſant & licentious things: and al this
vnder pretenſe of ſpirit, libertie, and freedom of the Ghoſpel. Such
muſt learne that al hereſies, ſchiſmes, and rebellions againſt the
Church & their lawful Prelates, be counted heer among the workes of the
flesh. See
\Cite{S.~Auguſtin de fid & op. c.~24,~29.}}
libertie an occaſion to the flesh, but by charitie ſerue
%%% o-2609
one another. \V For al the Law is fulfilled in one word:
\CNote{\XRef{Leu.~49,~18.}}
\Emph{Thou shalt loue thy neighbour as thy ſelf.} \V But if you bite and
eate one another, take heed you be not conſumed one of another. \V And I
ſay, walke in the ſpirit, and the luſts of the fleſh you ſhal not
accompliſh. \V For the fleſh luſteth againſt the ſpirit, and the ſpirit
againſt the fleſh: for theſe are aduerſaries one to another:
\SNote{Here men thinke (ſaith S.~Auguſtin) the Apoſtle denieth that we
haue free libertie of wil: not vnderſtãding that this is ſaid to thẽ, if
they wil not hold faſt the grace of faith cõceiued, by which only they
can walke in the ſpirit, & not accõplish the concupiſcences of the
flesh.
\Cite{in c.~5. Gal.}}
that not what things ſoeuer
%%% !!! Missing LNote, marked in both
%%% \LNote{}{}
you wil, theſe you doe. \V But if you be led by the ſpirit, you are not
vnder the Law.

\V And the workes of the fleſh be manifeſt, which are fornication,
vncleannes, impudicitie, lecherie, \V ſeruing of Idols, witch-crafts,
enmities, contentions, emulations, angers, brawles, diſſenſions,
ſects, \V enuies, murders, ebrietie, commeſſations, and ſuch like. Which I
fore-tel you, as I haue fore-told you, that they which doe
\SNote{S.~Auguſtin sheweth hereby that not only infidelitie is a
damnable ſinne.}
ſuch things, ſhal not obteine the Kingdom of God. \V But the fruit of
the Spirit is, Charitie, ioy, peace, patience, benignitie, goodnes,
longanimitie, \V mildnes, faith, modeſtie, continencie,
chaſtitie. Againſt ſuch there is no law. \V And they that be Chriſts,
haue crucified their fleſh with the vices and concupiſcences. \V If we
liue in the ſpirit, in the ſpirit alſo let vs walke. \V Let vs not be
made deſirous of vaine glorie, prouoking one another, enuying one
another.


\stopChapter


\stopcomponent


%%% Local Variables:
%%% mode: TeX
%%% eval: (long-s-mode)
%%% eval: (set-input-method "TeX")
%%% fill-column: 72
%%% eval: (auto-fill-mode)
%%% coding: utf-8-unix
%%% End:

