%%%%%%%%%%%%%%%%%%%%%%%%%%%%%%%%%%%%%%%%%%%%%%%%%%%%%%%%%%%%%%%%%
%%%%
%%%% The (original) Douay Rheims Bible 
%%%%
%%%% New Testament
%%%% John
%%%% Chapter 20
%%%%
%%%%%%%%%%%%%%%%%%%%%%%%%%%%%%%%%%%%%%%%%%%%%%%%%%%%%%%%%%%%%%%%%




\startcomponent chapter-20


\project douay-rheims


%%% 2532
%%% o-2373
\startChapter[
  title={Chapter 20}
  ]

\Summary{Vpon Eaſter day his body is miſſed in the Sepulcher, firſt by
  M.~Magdalene, 3.~ſecondly by Peter alſo & Iohn, the winding clothes
  yet remayning. 11.~Then to M.~Magdalene after she had ſeen two Angels,
\Sc{Iesvs} alſo himſelf appeareth. 18.~She hauing told to the Diſciples,
he 
\Fix{appereath}{appeareth}{obvious typo, fixed in other}
to them alſo the ſame day, and ſendeth them as himſelf was ſent, giuing
them the Holy Ghoſt to remit and to reteine ſinnes. 26.~Againe vpon low
Sunday he appeareth to them, letting Thomas ſee, that he might beleeue,
and commending ſuch as not ſeeing yet doe beleeue. 30.~The effect of
this booke.}

%%% o-2374
And
\CNote{\XRef{Mt.~28,~1.}
\XRef{Mr.~16,~1.}
\XRef{Lu.~24,~1.}}
\MNote{Eaſter day.}
the
\SNote{That is, the firſt day of the weeke, as ſome interpret it, takĩg
Sabboth (as ſometime it is) for a weeke. This is our Sunday, called
\L{Dies Dominica}, becauſe of our Lord's Reſurrectiõ. See the
\XRef{marg. annot. Luc.~24,~1.}}
firſt of the Sabboth, Marie Magdalene commeth early, when it was yet
darke, vnto the monument: and she ſaw the ſtone taken away from the
monument. \V She ranne therfore and commeth to Simon Peter, and to the
other Diſciple whom \Sc{Iesvs} loued, and ſaith to them: They haue taken
our Lord out of the monument, and we know not where they haue laid him.

\V Peter therfore went forth and that other Diſciple, and they came to
the monument. \V
\CNote{\XRef{Luc.~24,~12.}}
And both ranne together, and that other Diſciple did
out-runne Peter, and came firſt to the monument. \V And when he had
ſtouped downe, he ſaw the linned clothes lying: but yet he went not
in. \V Simon Peter therfore commeth, following him, and went into the
monument and ſaw the linnen clothes lying, \V and the napkin that had
been vpon his head, not lying with the linnen clothes, but apart,
wrapped vp into one place. \V Then therfore went in that other Diſciple
alſo which came firſt to the monument: and he ſaw, and beleeued. \V For
as yet they knew not the ſcripture, that he should riſe againe from the
dead. \V The Diſciples therfore departed againe to themſelues.

\V But
\CNote{\XRef{Mt.~28,~1.}
\XRef{Mr.~16,~5.}
\XRef{Luc.~24,~4.}}
Marie ſtood at the
\SNote{The Sepulchres of Martyrs (ſaith 
\Cite{S.~Hier. ep.~17.})
we doe honour euery-where, & putting
their holy ashes to our eyes, if we may, we touch it alſo with our
mouth: and be there ſome that thinke the monument wherein our Lord was
laid, is to be neglected; where the Diuel and his Angels, as often as
they are caſt out of the poſſeſſed before the ſaid monument, tremble and
roare as if they ſtood before the iudgement ſeate of Chriſt?}
monument without, weeping. Therfore as she was weeping, ſhe ſtouped
downe, & looked into the monument: \V and she ſaw two Angels in white,
ſitting, one at the head, and one at the feet, where the body of
\Sc{Iesvs} had been laid. \V They ſay to her: Woman, why weepeſt thou?
She ſaith to them: Becauſe they haue taken away my Lord, and I know not
where they haue put him. \V When she had ſaid thus, ſhe turned backward,
and ſaw \Sc{Iesvs} ſtanding; and ſhe knew not that it is \Sc{Iesvs}. \V
\Sc{Iesvs} ſaith to her:
%%% 2533
Woman, why weepeſt thou? whom ſeekeſt thou? She thinking that it was the
gardiner, ſaith to him: Sir, if thou haſt caried him away, tel me where
thou haſt laid him; and I wil take him away. \V \Sc{Iesvs} ſaith to her:
Marie. She turning ſaith to him: Rabboni (which is to ſay, Maiſter.) \V
\Sc{Iesvs} ſaid to her: Doe not touch me, for I am not yet aſcended to
my Father: but goe to my Brethren, and ſay to them, I aſcend to my
Father and your Father, my God and your God. \V Marie Magdalene
commeth and telleth 
%%% o-2375
the Diſciples, That I haue ſeen our Lord, and thus he ſaid vnto me.

\V Therfore when it was
\CNote{\XRef{Mr.~16,~14.}
\XRef{Lu.~24,~36.}
\XRef{1.~Cor.~15,~5.}}
late that day, the firſt of the Sabboths, and
\LNote{The doores were shut}{Such
\MNote{The being of Chriſts body in the B.~Sacrament without ſpace or
  quantitie correſpondent thereunto, is proued by other examples in
  Scriptures.}
Heretikes as deny Chriſts body to be, or that it can be in the
B.~Sacrament, for that it is in Heauen, & can not be in two places at
once, not without the natural manner of the quantitie, ſpace, or place
agreable to the condition of his humanitie, be inuincibly refuted by
Chriſts entring into the Diſciples, the doores shut: & by that his true
natural body whole & perfect in al his limmes, length, bredth, &
thicknes, diſtinct & diuers from the ſubſtance & corpulence of the wood,
was in the ſame proper place that the wood was in, & paſſed through the
ſame: as he alſo came out of his mothers wombe the clauſure not ſturred:
and paſſed through the ſtone, out of his Sepulcher. By al which the
Heretikes being plainely reproued, & conuinced of infidelitie, they
boldly deny the plaine Scriptures, or ſo fondly shift themſelues from
the euidence therof, that their impudencie is ſpecially to be marked in
this point.

Some
\MNote{Heretical shifts to auoid plaine Scripture.}
ſay, that he came in at the window: ſome, that the doore opened of
it-ſelf to let him in: ſome, that to come in, the doores being shut,
ſignifieth no more, but that he came in late in the euening, at what
time men vſe to shut their doores: and ſuch other flights to defend
falshood againſt expreſſe Scriptures, & againſt the Apoſtles teſtimonie,
who therfore tooke him to be a Spirit, becauſe they ſaw him ſtand
ſodenly in the middes of them, al the houſe being cloſe shut. And the
Fathers al confeſſe that he went in the doores being shut. See
\Cite{S.~Ambr. li.~10. in Lucam c.~24.}
\Cite{S.~Auguſtin ep.~3. ad Voluſian. & li.~22. de ciuit. c.~8.},
&
\Cite{S.~Cyril, in Io. li.~22, c.~13}
&
\Cite{S.~Hiero. li.~2. cont. Iounianum c.~21}
We know it is the natural courſe of God's ordinance, that euery body
should haue but one & his owne proper place fitted to the lineaments,
quantitie, termes & limites of the ſame: without which naturally the
bodies were no where, & conſequently not at al, as
\CNote{\Cite{Auguſt. ep.~57.}}
S.~Auguſtin ſaith \L{ad Dardanum}; 
\MNote{Chriſt can diſpoſe of his owne body & others aboue nature.}
but that God ſupernaturally & miraculouſly
can not by his omnipotencie diſpoſe otherwiſe of his owne body, then the
natural forme or quantitie or qualitie therof require, that is great
incredulitie: ſeing we muſt beleeue that he can doe ſo with any other
body of mere men or other creatures, the Scriptures being plaine that he
can make a
\CNote{\XRef{Mat.~19,~24.}}
camel paſſe through a needles eye, continuing in his natural
figure and quantitie ſtil: and
\CNote{\Cite{Aug. li.~22. c.~8. de ciu. Dei.}}
S.~Auguſtin telleth of a woman whoſe ring
fel from her girdle, both being faſt and whole: and Rupertus of a
Religious man, whoſe girdle faſt
\Fix{bulckled}{buckled}{obvious typo, fixed in other}
fel downe before him from his body.
\Cite{De off. Eccl.}

Therfore
\MNote{Vbiquetaries or Brentiani.}
it is too much vnfaithfulnes, by rules of place to embarre Chriſt of his
wil or wiſedom to be in the Sacrament how himſelf liſt, and on as many
Altars or places as he liketh. We deteſt for al that, the wicked hereſie
of certaine Proteſtants, holding quite contrarie to the Zuinglians, that
Chriſt according to his Humanitie is in euery place where the Diuinitie
is: which is both againſt faith, and the common rules of nature and
diuinitie.}
the doores were shut, where the Diſciples were gathered together for
feare of the Iewes, \Sc{Iesvs} came and ſtood in the middes, and ſaith
to them: Peace be to you. \V And when he had ſaid this, he ſhewed them
his handes and ſide. The Diſciples therfore were glad when they ſaw our
Lord. \V He ſaid therfore to them againe: 
\SNote{Though he gaue them his peace hard before, yet now entring to a
  new diuine action, to prepare their harts to grace and attention, he
  bleſſeth them againe.}
Peace be to you.
\LNote{As my Father}{As
\MNote{Chriſt sheweth his commiſſiõ, & ſo giueth the Apoſtles power to
  remit ſinnes.}
when he gaue them commiſſion to preach and baptize through the world, he
made mention of his owne power therein: ſo here before he inſtitute the
Sacrament of Penance, and giue them authoritie to remit ſinnes, leſt the
wicked should aske afterward, by what right they doe ſuch great
functions, he sheweth his Fathers commiſſion giuen to himſelf, and then
in plaine termes moſt amply imparteth the ſame to his Apoſtles: that
whoſoeuer deny the Apoſtles & their ſucceſſours, the Prieſts of Gods
Church, to haue right to remit ſinnes, should deny conſequently Chriſt
as man to haue authoritie to doe the ſame.}
As my Father hath ſent me, I alſo doe ſend you. \V When he had ſaid
this: 
\LNote{He breathed}{He
\MNote{The holy Ghoſt is here purpoſly giuen to the Apoſtles, to remit
  ſinnes.}
giueth the Holy Ghoſt in & by an external ſigne, to his Apoſtles, not
viſibly and to al ſuch purpoſes as afterward at whitſuntide, but for the
grace of the Sacrament of Orders, as
\CNote{\Cite{Aug. q. no. Teſt. q.~93. cont. Parmen. li.~2. c.~11.}}
S.~Auguſtin ſaith, and that none
make doubt of the Prieſts right in remiſſion of ſinnes, ſeeing the Holy
Ghoſt is purpoſly giuen them to doe this ſame. In which caſe if any be yet
contentious, he muſt deny the Holy Ghoſt to be God, & not to haue the
power to remit ſinnes.
\CNote{\Cite{Cyril. li.~12. c.~56. in Io.}}
\Emph{It is not abſurd} (ſaith S.~Cyril)
\Emph{that they forgiue ſinnes, which haue the Holy Ghoſt. For when they
remit of reteine, the Holy Ghoſt remitteth or reteineth in them; & that
they doe two waies, firſt in Baptiſme & then in Penance.} As 
S.~Amb.\ alſo
\Cite{(li.~3. c.~7. de pœnitentia)}
refelling the
Nouatians (a Sect of old Heretikes which pretending Gods glorie as our
new Sectaries doe, denied that Prieſts could remit ſinnes in the
Sacrament of Penance) asketh, why it should be more diſhonour to God, or
more impoſſible or inconuenient for men, to forgiue ſinnes by penance
then by Baptiſme, ſeeing it is the Holy Ghoſt that doeth it, by the
Prieſts office and miniſterie in both.}
he breathed vpon them; and he ſaid to them: Receiue ye the Holy Ghoſt:
\V
\LNote{Whoſe ſinnes}{Power
\MNote{The Sacramẽt of \Sc{Penance} inſtituted.}
to offer Sacrifice, which is the principal function and act of
Prieſthood, was giuen them at the inſtitution of the B.~Sacrament, the
ſecond & next ſpecial facultie of Prieſthood, conſiſting in remitting
ſinnes, is here beſtowed on them. And withal the holy Sacrament of
Penance, implying Contrition, Confeſſion, & Satisfaction in the
Penitent, and abſolution on the Prieſts part, is inſtituted. For in
that, that expreſſe power & cõmiſſiõ is giuen to Prieſts to remitte or
reteine al ſinnes: & in that, that Chriſt promiſeth, whoſe ſinnes ſoeuer
they forgiue, they be of God forgiuẽ alſo: & whoſe ſinnes they reteine,
they be reteined before God; it followeth necceſſarily, that we be boũd
to ſubmit our ſelues to their iudgement for releaſe of our ſinnes. For,
this wonderful power were giuen them in vaine, if none were bound to
ſeeke for abſolution at their hands.
\MNote{Men are bound to confeſſe al their mortal ſinnes, and that in
  particular.}
Neither can any rightly ſeeke for abſolution of them vnles they confeſſe
particularly at leaſt al their mortal offences, whether they be
committed in mind, hart, wil and
\CNote{\Cite{Cyp. de lapſ. nu.~11.}}
cogitation only, or in word and
worke. For God's Prieſts being in this Sacrament of Penance conſtituted
in Chriſts ſteed as iudges in cauſes of our conſcience, can not rightly
rule our caſes without ful & exact cognition & knowledge of al our
ſinnes, and the neceſſarie circumſtances &
\CNote{\Cite{Hiero. in 16.~Mat.}}
differences of the
ſame. Which can not otherwiſe be had of them being mortal men, then by
our ſimple, ſincere, & diſtinct vtterance to them of our ſinnes, with
humble contrite hart, ready to take & to doe penance according to their
iniunction. 
\MNote{To reteine ſinnes.}
For that authoritie to reteine ſinnes, conſiſteth ſpecially
in enioyning ſatisfaction & penitential workes of praying, faſting,
almes, & ſuch like. Al which God's ordinance whoſoeuer condemneth or
contemneth, as Heretikes doe, or neglecteth, as ſome careleſſe
Catholikes may perhaps doe, let them be aſſured they can not be ſaued. 
\MNote{The neceſſitie of this Sacrament.}
Neither muſt any ſuch Chriſtian man pretend or looke to haue his ſinnes
after Baptiſme, remitted by God only, without this Sacrament: (which was
the old Hereſie of the Nouations. 
\Cite{Ambro. li.~1. de pœnit. c.~2.}
\Cite{Socrat. li.~7. Ec. hiſt. c.~25.})
more then any may hope to be ſaued or haue his original or other ſinnes
before Baptiſme, forgiuen by God without the ſame Sacrament. Let no man
deceiue himſelf, this is the \Emph{ſecond table or borde after
  shipwracke}, as
\CNote{\Cite{Hiero. ad Demetriadem. c.~6. to.~1.}}
S.~Hierom calleth it, whoſoeuer take not hold of it,
shal perish without doubt, becauſe they contemne God's counſel & order
for their ſaluation; & therfore S.~Auguſtin
\Cite{(ep.~180.)}
ioyning both together, ſaith it is a pitiful caſe, when by the abſence
of God's Prieſts, men depart this life, \L{aut non regenerati, aut
  ligati}, that is, \Emph{either not regenerated} by Baptiſme, \Emph{or
  faſt bound}, and not abſolued by the Sacrament of penance and
reconciliation: becauſe they shal be excluded from eternal life, and
\Emph{deſtruction followeth them}. And S.~Victor 
\Cite{(li.~2. de perſecut. Vandalica)}
telleth the miſerable
lamentation of the people, when their Prieſts were banished by the Arian
Heretikes. \Emph{Who} (ſay they) \Emph{shal baptize theſe infants? who
shal miniſter penance vnto vs, & looſe vs from the bandes of ſinnes
&c?} And therfore S.~Cyprian very often (namely
\Cite{ep.~54})
calleth
it great crueltie, & ſuch as Prieſts shal anſwer for at the later day,
to ſuffer any man that is pœnitent of his ſinnes, to depart this life
without this reconciliation and abſolution: \Emph{becauſe} (ſaith he)
\Emph{the Law-maker himſelf} (Chriſt) \Emph{granted, that
\CNote{\XRef{Mt.~18.}}
things bound
  in earth, should alſo be bound in Heauen: and that thoſe things should
  there be looſed, which were looſed before here in the Church.} 
\MNote{The Heretikes Wraſtling againſt plaine Scripture.}
And it is a world to ſee, how the Heretikes wraſtle with this ſo plaine
a commiſſion of remitting ſinnes, referring it to preaching, to
denouncing God's threats vpon ſinners, and to we can not tel what els: 
\MNote{The English Miniſters heare confeſſions, and abſolue.}
though to our English Proteſtants this authoritie ſeemeth ſo cleer, that
in
\CNote{\Cite{See the Communion booke.}}
their order of viſiting the ſicke, their Miniſters acknowledge &
chalenge the ſame, viſing a formal abſolution according to the Churches
order, after the ſpecial confeſſion of the partie. But to conclude the
matter, let euery one that liſt to ſee the true meaning of Chriſts
words, and the Prieſts great power and dignitie giuen them by the ſame
words and other, marke wel theſe words of S.~Chryſoſtome: 
\MNote{Prieſts power to forgiue ſinnes, is aboue the power of Angels or
worldly Princes.}
\CNote{\Cite{Li.~3. de Sacred.}}
\Emph{For}, (ſaith he), \Emph{they that dwel on the earth, and conuerſe
  in it, to them is commiſsion giuen to diſpenſe thoſe things that are
  in Heauen: 
  to them it is giuen to haue the power which God would not to be giuen
  neither to Angels nor Archangels. For, neither to them was it ſaid:
  Whatſoeuer you shal bind in earth, shal be bound in Heauen: and
  whatſoeuer you shal looſe in earth, shal be looſed in Heauen. The
  earthly Princes indeed haue alſo power to bind, but the bodies only:
  but that bond of Prieſts which I ſpeake of, toucheth the very ſoule
  it-ſelf, and reacheth euen to the Heauens: in ſo much that whatſoeuer
  the Prieſts shal doe beneath, the ſelf-ſame God doth ratifie aboue,
  and the ſentence of the ſeruants the Lord doth confirme. For indeed
  what els is this, then that the power of al heauenly things is granted
  them of God? Whoſe ſinnes ſoeuer, ſaith he, you shal reteine, they are
  reteined. What power (I beſeech you) can be greater then this one? The
  Father gaue al power to the Sonne: but I ſee the ſame power altogether
  deliuered by the Sonne vnto them.} And as this concerneth the Prieſts
high authoritie to abſolue, ſo thereupon concerning confeſſion alſo to
be made vnto them, the ancient Fathers ſpeake in this ſort. 
\Cite{S.~Cypriã de Lapſis nu.~13.}
\Emph{They} (ſaith he)
\Emph{that haue greater faith and feare of God, though they did not fal
  in perſecution, yet becauſe they did only thinke it in their mind,
  this very cogitation they confeſſe to Gods Prieſts ſorrowfully and
  plainely, opening their conſcience, vttering and diſcharging the
  burden of their mind, and ſeeking holeſome medicine for their wounds
  though but ſmal and litle.} And a litle after: \Emph{Let euery one (my
  Brethren) I beſeech you, confeſſe his ſinne, whiles he is yet aliue,
  whiles his confeſsion may be admitted, whiles ſatisfaction and
  remiſsion made by the Prieſt is acceptable before God.} S.~Cyril (or
as ſome thinke, Origen)
\Cite{li.~2. in Leuit.},
calleth it a great part of penãce, whẽ a mã is ashamed, & yet openeth
his ſinnes to our Lords Prieſt. See alſo 
\Cite{Tertul. li. de Pœnit.}
\Cite{S.~Hier. in c.~10. Eccleſiaſtæ.}
\Cite{S.~Baſil. in Regulis breu. quæſt.~229.}
Who compare ſinners that refuſe to confeſſe, to them that haue ſome
diſeaſe in their ſecret partes, and are ashamed to shew it to the
Phyſicion or Surgeon, that might cure it. 
\MNote{Secret or auricular Confeſſion.}
Where they muſt needs meane
ſecret confeſſion to be made to them that may abſolue. And 
\Cite{S.~Leo ep.~80.}
moſt plainely (as before S.~Cyril) expreſly nameth
Prieſts. \Emph{That confeſsion is ſufficient which is made firſt to God
then to the Prieſts alſo.} And again: \Emph{It is ſufficient that the
guiltines of mens conſciences be vttered to the Prieſt only by the
ſecrecie of confeſsion.} S.~Hierome in
\XRef{16.~Mat.}
ſaith, that Prieſts looſe or bind, \L{audita peccatorum
varietate}, \Emph{hauing heard the
\Fix{varitie}{varietie}{obvious typo, fixed in other}
and differences of ſinnes.} S.~Paulinus writeth of S.~Ambroſe,
\CNote{\Cite{In vita D.~Am. prope finem.}}
\Emph{That as often as any confeſſed his ſinnes vnto him for to receiue
  penance, he ſo vvept for compaſsion, that thereby he cauſed the
  penitent to vveep alſo.} He addeth moreouer, that this holy Doctour
was ſo ſecret in this caſe, that no man knew the ſinnes confeſſed, but
God and himſelf. And S.~Auguſtin
\Cite{ho~49. de 50. homilijs to 10.}
ſaith thus: \Emph{Doe penance, ſuch as is done in the Church: let no man
ſay, I doe it ſecretly, I doe it to God: In vaine then vvas it ſaid:}
\CNote{\XRef{Mt.~18.}}
Whatſoeuer you shal looſe in earth, shal be looſed in Heauen. See
S.~Ambroſe
\Cite{de pœnitentia}
throughout, S.~Cyprian
\Cite{de Lapſis},
the booke
\Cite{de vera & falſa pœnit.}
in S.~Auguſtin: beſide al antiquitie which is ful of theſe ſpeaches
conſerning abſolution, and confeſſion.}
\Sc{Whoſe sinnes you shal forgive, they are forgiven them: and whose you
shal reteine, they are reteined.} \V But Thomas one of the Twelue, who is
called Didymus, was not with them when \Sc{Iesvs} came. \V The other
Diſciples therfore ſaid to him: We haue ſeen our Lord. But he ſaid to
them: Vnles I ſee in his handes the print of the nailes, and put my
finger into the place of the nailes, and put my hand into his ſide: I
wil not beleeue.

\V And after eight daies, againe his Diſciples were within; and Thomas
with them. \Sc{Iesvs} commeth
\SNote{See the annotation on the 19.~verſe of this Chap.}
the doores being shut, and ſtood in the middes, and ſaid: Peace be to
you. \V Then he ſaith to Thomas: Put in thy finger hither, and ſee my
handes, and bring hither thy hand, and put it into my ſide; & be not
incredulous but faithful. \V Thomas anſwered, & ſaid to him: My Lord, &
my God. \V \Sc{Iesvs} ſaith to him: Becauſe thou haſt ſeen me, Thomas,
thou haſt beleeued: 
\SNote{They are more happy that beleeue without ſenſible argument or
  ſight, then ſuch as be induced by ſenſe or reaſon to beleeue.}
Bleſſed are they that haue not ſeen & haue beleeued. \V
\CNote{\XRef{Io.~21,~25.}}
Many other
ſignes alſo did \Sc{Iesvs} in the ſight of his Diſciples, which are not
written in this Book. \V And theſe are written, that you may beleeue
that \Sc{Iesvs} is \Sc{Christ} the Sonne of God: and that beleeuing, you
may haue life in his name.

\stopChapter


\stopcomponent


%%% Local Variables:
%%% mode: TeX
%%% eval: (long-s-mode)
%%% eval: (set-input-method "TeX")
%%% fill-column: 72
%%% eval: (auto-fill-mode)
%%% coding: utf-8-unix
%%% End:
