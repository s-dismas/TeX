%%%%%%%%%%%%%%%%%%%%%%%%%%%%%%%%%%%%%%%%%%%%%%%%%%%%%%%%%%%%%%%%%
%%%%
%%%% The (original) Douay Rheims Bible 
%%%%
%%%% New Testament
%%%% John
%%%% Chapter 21
%%%%
%%%%%%%%%%%%%%%%%%%%%%%%%%%%%%%%%%%%%%%%%%%%%%%%%%%%%%%%%%%%%%%%%


%%% Latin checked by KK.



\startcomponent chapter-21


\project douay-rheims


%%% 2536
%%% o-2378
\startChapter[
  title={Chapter 21}
  ]

\Summary{Appearing againe in Galilee, where Peter was fishing with 
\Fix{this}{his}{obvious typo, fixed in other}
fellowes; and cauſing them after they had al night taken none, to catch
a great multitude, which Peter draweth to land, where he alſo dineth
them; 15.~he (expreſsing what this fishing ſignified) maketh Peter his
Vicar, committing vnto him the feeding of his lambs and sheep: 18.~and
reuealeth vnto him, that he alſo shal be crucified, to the glorie of
God, 20.~admonishing him to mind that rather then to be curious about
Iohns death.}

After \Sc{Iesvs} manifeſted himſelf againe to the Diſciples at the ſea
of Tiberias. And he manifeſted thus. \V There were together Simon Peter,
and Thomas who is called Didymus, and Nathanael which was of Cana in
Galilee, & the ſonnes of Zebedee, and two others of his Diſciples. \V
Simon Peter ſaid to them: I goe to fiſh. They ſay to him: We alſo come
with thee. And they went forth and got vp into the boat: and that night
they tooke nothing. \V But when morning was now come, \Sc{Iesvs} ſtood
on the ſhore: yet the Diſciples knew not that it was \Sc{Iesvs}. \V
\Sc{Iesvs} therfore ſaith to them: Children, haue you any meat? They
anſwered him, No. \V He ſaith to them: Caſt the net on the right ſide of the
boat; and you ſhal find. They therfore did caſt it: and now they were
not able to draw it for the multitude of fiſhes. \V That Diſciple
therfore whom \Sc{Iesvs} loued, ſaith to Peter: It is our Lord.
\SNote{See in
\Cite{S.~Auguſtin Tractat.~122. in Ioa.}
the great myſterie hereof concerning the \Sc{Chvrch}, and in 
\Cite{S.~Gregorie hom.~14. in Euang.}
and
\Cite{S.~Bernard li.~2. c.~3. de conſid.}
Peters \Sc{Primacie} here myſtically ſignified.}
Simon Peter when he had heard that it is our Lord, girded his coate vnto
him (for he was naked) & caſt himſelf into the ſea. \V But the other
Diſciples came in the boat (for they were not farre from the land, but as
it were two hundred cubits) drawing the net of fiſhes. \V Therfore after
they came downe to land, they ſaw hot coles lying, and fiſh laid
thereon, and bread. \V \Sc{Iesvs} ſaith to them: Bring hither of the
fiſhes that you tooke now. \V Simon Peter went vp, and drew the net to
the land, ful of great fiſhes, an hundred fiftie three. And although
%%% 2537
they were ſo many the net was not broken. \V \Sc{Iesvs} ſaith to thẽ:
Come, dine. And none of
\Var{them that ſate at meate,}{the diſciples,}
durſt aske him: Who art
thou? knowing that it is our Lord. \V And \Sc{Iesvs} cõmeth & taketh the
bread and giueth them, and the fiſh in like manner. \V This now the
\SNote{Not the third apparition, but the third day of his apparitiõs:
for he appeared in the very day of his Reſurrection often, againe vpõ
Low Sunday, then this third time. And S.~Marke ſaying, \Emph{laſt he
appeared}
\XRef{c.~16,~14.}
meaneth his laſt apparition the firſt day.}
third time \Sc{Iesvs} was manifeſted to his Diſciples, after he was
riſen frõ the dead.

\V Therfore when they had dined, \Sc{Iesvs} ſaith to Simon 
%%% o-2379
Peter: Simon
of Iohn, loueſt thou me more then theſe? He ſaith to him: Yea Lord, thou
knoweſt that I loue thee. He ſaith to him: \Sc{Feed my lambs}. \V He
ſaith to him againe: Simon of Iohn, loueſt thou me? He ſaith to him: Yea
Lord, thou knoweſt that I loue thee. He ſaith to him:
\TNote{\G{ποίμαινε} \Emph{feed & rule}}
\Sc{Feed my lambs}. \V He ſaith to him the third time: Simon of Iohn,
loueſt thou me? Peter was ſtroken ſad becauſe he ſaid vnto him the third
time, Loueſt thou me? And he ſaid to him: Lord thou knoweſt al things:
thou knoweſt that I loue thee. He ſaid to him: 
\LNote{Feed my sheep}{As
\MNote{Peter is here made the general Paſtour, & the Church is builded
  vpon him.}
it was promiſed him
\XRef{Mat.~16}
that the Church should be builded
vpon him, & that the keies of heauen should be giuen to him: ſo here it
is performed, & he is actually made the general Paſtour & Gouerner of al
Chriſts sheep. For though the other ten (as Matthias & Paul alſo
afterward) were Apoſtles, Bishops, Prieſts, & had authoritie to bind and
looſe, to remit & retaine, to preach, baptize, and ſuch like, as wel as
he: 
\MNote{The Proteſtãts otherwiſe denying this preeminence of Peter, yet
  to vphold their Archbishops, doe auouch & proue it againſt the Puritans.}
Yet in theſe things & al other Gouerment, Chriſt would haue him to
be their Head, and they to depend on him as Head of their Colledge, &
conſequently of the whole flocke of Chriſt: no Apoſtle, nor no Prince in
earth (if he acknowledge himſelf to be a sheep of Chriſt) exempted from
his charge. And that Chriſt maketh a difference betwixt Peter and the
reſt, and giueth him ſome greater preeminence and regiment then the
reſt, it is plaine by that he is asked whether he loue our Lord more
then the other Apoſtles doe: where, for equal charge no difference of
loue had been required.
\CNote{\Cite{Cypr. de vnit. Ec.}}
\Emph{To Peter} (ſaith S.~Cyprian) \Emph{our
  Lord after his Reſurrection ſaid: Feed my sheep, and builded his
  Church vpon him alone, & to him he giueth the charge of feeding his
  sheep. For although, after his Reſurrection he gaue his power alike to
  al, ſaying, As my Father ſent me, ſo I ſend you, take the Holy Ghoſt,
  if you remit to any their ſinnes, they shal be remitted &c. Yet to
  manifeſt vnitie, he conſtituted one Chaire, & ſo diſpoſed by his
  authoritie that vnitie should haue origine of one. The reſt of the
  Apoſtles were that Peter was, in equal fellowship of honour and power,
  but the beginning commeth of vnitie: the Primacie is giuen to Peter,
  that the Church of Chriſt may be shewed to be one, & one Chaire.}
S.~Chryſoſtom alſo ſaith thus:
\CNote{\Cite{Lib.~2. de Sacerd.}}
\MNote{Peters ſucceſſours ſucceede him in vniuerſal authoritie.}
\Emph{Why did our Lord sheed his bloud?
  truly to redeeme thoſe sheep, the cure of which he committed both to
  Peter and alſo his Succeſſours.} And a litle after. \Emph{Chriſt would
  haue Peter indowed with ſuch authoritie, and to be farre aboue al his
  other Apoſtles. For he ſaith: Peter, doſt thou loue me more then al
  theſe doe? Wherevpon our Maiſter might haue inferred, If thou loue me
  Peter, vſe much faſting, ſleep on the hard floure, watch much, be
  patrone to the oppreſſed, father to the orphans, and huſband to the
  widowes: but omitting al theſe things, he ſaith, Feed my sheep. For,
  al the foreſaid vertues certes may be done eaſily of many ſubiects,
  not only men but womẽ: but when it commeth to the gouernment of the
  Church and committing the charge of ſo many ſoules, al woman-kind muſt
  needes wholy giue place to the burden and greatnes thereof, and a
  great number of men alſo.} So writeth he.

And
\MNote{S.~Gregorie though he miſliked the title of \Emph{Vniuerſal
    Bishop}, yet is moſt plaine both in his writings & doings for the
  Popes Supremacie, as alſo S.~Leo the great.}
becauſe the Proteſtants would make the vnlearned thinke, that
S.~Gregorie deemed the Popes Supremacie to be wholy vnlawful and
Antichriſtian, for that he condemned Iohn of Conſtantinople for vſurping
the name of vniuerſal Bishop, reſembling his inſolence therein to the
pride of Antichriſt; note wel the wordes of this Holy Father in the very
ſame place and Epiſtle againſt the B.~of Conſtantinople, by which you
shal eaſily ſee that to deny him to be vniuerſal Bishop, is not to deny
Peter or the Pope to be Head of the Church, or ſupreme Gouerner of the
ſame, as our Aduerſaries fraudulently pretend.
\CNote{\Cite{Greg. li.~4. ep.~76.}}
\Emph{It is plaine to al men}, ſaith he, 
\Emph{that euer read the Ghoſpel, that by our Lordes mouth the charge of
  the whole Church was committed to S.~Peter Prince of the Apoſtles. For
  to him it was ſaid: Feed my sheep: for him was the prayer made that
  his faith should not faile: to him were the keies of Heauen giuen, and
  authoritie to bind and looſe: to him the cure of the Church and
  principalitie was deliuered: and yet he was not called the vniuerſal
  Apoſtle. This title indeed was offered for the honour of S.~Peter
  Prince of the Apoſtles, to the Pope of Rome by the holy Councel of
  Chalcedon: but none of that See did euer vſe it or conſent to take
  it.} Thus much S.~Gregorie.
\CNote{\Cite{See li.~1. ep.~73,~75.}
\Cite{li.~2. ep.~37,~45.}
\Cite{li.~4. ep.~95.}
\Cite{li.~7. ep.~63.}}
Who though he both practiced iuriſdiction
throughout al Chriſtendom, as other of that See haue euer done, and
alſo acknowledged the Principalitie and Soueraigntie to be in Peter and
his Succeſſours: yet would he not for iuſt cauſes vſe that title ſubiect
to vanitie & miſconſtruction. 
\MNote{The title of vniuerſal Bishop refuſed, but vniuerſal iuriſdiction
alwaies acknowledged and practiſed.}
But both he & al the Popes ſince haue
rather called thẽſelues, \L{Seruos ſeruorum Dei}, the Seruants of Gods
ſeruants. Though the word, \Emph{vniuerſal Bishop}, in that ſenſe
wherein the holy Councel of Chalcedon offered it to the See of Rome, was
true & Lawful. For that Coũcel would not haue giuen any Antichriſtian or
vniuſt title to any man. Only in the B.~of Conſtantinople and other,
which in no ſenſe had any right to it, and who vſurped it in a very falſe &
tyrannical meaning, it was inſolent, vniuſt, & Antichriſtian. See alſo
the Epiſtles of S.~Leo the Great concerning his practiſe of vniuerſal
iuriſdiction, though he refuſed the title of vniuerſal Bishop. And
\CNote{\Cite{Bernar. li.~2. c.~8. de cõſid.}}
S.~Bernard (that you may better perceiue that the general charge of
Chriſts sheep was not only giuen to Peters Perſon, but alſo to his
Succeſſours the Popes of Rome, as S.~Chryſoſtom alſo before alleaged
doth teſtifie) writeth thus to Eugenius: Thou art he to whom the keies
of Heauen are deliuered, & to whom the sheep are committed. There be
other Porters of Heauen, & other Paſtours of flockes: but thou haſt
inherited in more glorious & different ſort. For they haue euery one
their particular flocke, but to thee al vniuerſally, as one flocke to
one man, are credited, being not only the Paſtour of the sheep, 
\MNote{The Pope is Paſtour of al Paſtours.}
but the one Paſtour of al the Paſtours thẽſelues. But thou wilt aske me
how I proue that? Euen by our Lordes word. For to whõ of al, I ſay not
only Bishops, but Apoſtles, were the sheep to abſolutely & without
limitatiõ cõmitted? \Emph{If thou loue me Peter, feed my sheep} He ſaith
not, the people of this Kingdõ or that citie, but, \Emph{my sheep},
without al diſtinctiõ. So S.~Bernard. And hereunto may be added that the
ſecõd, 
\TNote{\G{ποίμαινε}}
\Emph{feed}, is in Greek a word that ſignifieth withal to gouerne & rule
as
\XRef{Ps.~2.}
\XRef{Mich.~3.}
\XRef{Mat.~2.}
\XRef{Apoc.~2.}
& therfore it is ſpokẽ of Dauid alſo & other
tẽporal Gouerners (as the Hebrew word anſwering thereunto) in the
\CNote{\XRef{2.~Reg. c.~5.}
\XRef{Pſ.~77.}}
Scriptures oftẽ & the Greek in profane writers alſo.}
\Sc{Feed my sheep}.\V Amen, amen I ſay to thee, when thou waſt yonger,
thou didſt gird thy 
ſelf, and didſt walke where thou wouldeſt. But when thou ſhalt be old
thou ſhalt ſtretch forth thy hands, and
\LNote{Another shal gird thee}{He
\MNote{Peter Crucified at Rome.}
prophecieth of Peters Martyrdõ, and of the kind of death which he should
ſuffer, that was, crucifying. Which
\CNote{\Cite{Beza. in hunc locum.}}
the Heretikes, fearing that it were a
ſtep to proue he was martyred in Rome, deny: whereas the Fathers
and ancient Writers are as plaine in this, as that he was at Rome.  
\Cite{Origen apud Euſeb. li.~3. c.~3.}
\Cite{Euſeb. li.~2. c.~24. Hiſt. Ec.}
\Cite{Tert. de præſcript. nu.~14.}
\Cite{Aug. tract.~123. in Ioan.}
\Cite{Chryſoſt. Beda in hunc locum.}}
another shal gird thee, and lead thee whither thou wilt not. \V And this
he ſaid, ſignifying by what death he ſhould glorifie God. And when he
had ſaid this, he ſaith to him: Follow me. \V Peter turning, ſaw that
Diſciple whom \Sc{Iesvs} loued, following,
\CNote{\XRef{Io.~13,~23.}}
who alſo leaned at the ſupper
vpon his breaſt, and ſaid, Lord who is he that ſhal betray thee? \V Him
therfore when Peter had ſeen, he ſaith to \Sc{Iesvs}: Lord and this man
what? \V \Sc{Iesvs} ſaith to him:
\SNote{So readeth 
\Cite{S.~Ambr. in Pſ.~45. & ſer.~20. in Ps.~118}
\Cite{S.~Aug. tra.~124 in Io.}
& moſt ancient copies and ſeruice bookes extãt in Latin. others
  read, \Emph{If I wil}: others, \Emph{If ſo I wil}, &c.}
So I wil haue him to remaine til I come, what to thee? follow thou
me. \V This ſaying therfore went abrode among the Brethren, that that
Diſciple dieth not. And \Sc{Iesvs} did not ſay to him, he dieth not;
but, So I wil haue him to remaine til I come, what to thee? \V This is
that Diſciple which giueth teſtimonie of theſe things, and hath written
theſe things: and we know that his teſtimonie is true.

\V But there are
\CNote{\XRef{Io.~20,~30.}}
many
\SNote{How few things are written of Chriſts actes & doctrine in
  compariſon of that which he did and ſpake: and yet the Heretikes wil
  needs haue al in Scripture, truſting not the Apoſtles owne preaching,
  or report of any thing that our Maiſter did or ſaid, if it be not
  written.}
other things alſo which \Sc{Iesvs} did: which if they were written in
particular, neither the world it-ſelf I thinke were able to conteine
thoſe books that ſhould be written.

\stopChapter


\stopcomponent


%%% Local Variables:
%%% mode: TeX
%%% eval: (long-s-mode)
%%% eval: (set-input-method "TeX")
%%% fill-column: 72
%%% eval: (auto-fill-mode)
%%% coding: utf-8-unix
%%% End:
