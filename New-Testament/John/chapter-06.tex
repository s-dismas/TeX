%%%%%%%%%%%%%%%%%%%%%%%%%%%%%%%%%%%%%%%%%%%%%%%%%%%%%%%%%%%%%%%%%
%%%%
%%%% The (original) Douay Rheims Bible 
%%%%
%%%% New Testament
%%%% John
%%%% Chapter 06
%%%%
%%%%%%%%%%%%%%%%%%%%%%%%%%%%%%%%%%%%%%%%%%%%%%%%%%%%%%%%%%%%%%%%%




\startcomponent chapter-06


\project douay-rheims


%%% 2497
%%% o-2323
\startChapter[
  title={Chapter 6}
  ]

\Summary{Hauing 
\MNote{The 3.~part. His Actes in Galilee, & in Iewrie, about the third
  Paſche and after.}
with fiue loaues fed fiue thouſand 16.~(walking alſo the night after
vpon the ſea) 22.~on the morow the people thereupon reſorting vnto him,
27.~he preacheth vnto them of the Bread which he wil giue: telling them
that he is come from Heauen, and therfore able to giue ſuch bread as can
quicken the world, euen his owne flesh: and that al his Elect shal
beleeue as much. 60.~Many notwithſtanding doe murmur at this doctrine
yea and become apoſtates, though he tel them that they shal ſee by his
Aſcenſion into Heauen, that he is deſcended from Heauen. But the Twelue
ſticke vnto him, Peter in al their names confeſſing that he is God
Omnipotent. 70.~Among which twelue yet (that no man be ſcandalized) he
ſignifieth that he foreknoweth which wil become a traitour: as among
the foreſaid, which would become apoſtates.}

After
\CNote{\XRef{Mt.~14,~13.}
\XRef{Mr.~6,~32.}
\XRef{Luc.~9,~10.}}
theſe things \Sc{Iesvs} went beyond the ſea of Galilee, which is
of Tiberias. \V and a great multitude followed him, becauſe they ſaw the
ſignes which he did vpon thoſe that were ſicke. \V \Sc{Iesvs} therfore
went vp into the mountaine, and there he ſate with his Diſciples. \V And
the Paſche was at hand, the feſtiual day of the Iewes. \V When
\Sc{Iesvs} therfore had lifted vp his eies, and ſaw that a very great
multitude commeth to him, he ſaith to Philippe: Whence ſhal we buie
bread, that theſe may eate? \V And this he ſaid, tempting him. For
himſelf knew what he would doe? \V Philippe anſwered him: Two hundred
penie worth of bread is not ſufficient for them, that euery man may take
a litle peece. \V One of his Diſciples, Andrew the brother of Simon
Peter, ſaith
%%% o-2333
to him: \V There is a boy here that hath fiue barley loaues, & two
fishes; but what are theſe among ſo many? \V \Sc{Iesvs} therfore ſaith:
Make the men to ſit downe. And there was much graſſe in the place. The
men therfore ſate downe, in number about fiue thouſand. \V \Sc{Iesvs}
therfore tooke the loaues; and when he had giuen thankes, he diſtributed
to them that ſate. In like manner alſo of the fishes as much as they
would. \V And after they were filled, he ſaid to his Diſciples: Gather
the fragments that are remaining, leſt they be loſt. \V They gathered
therfore, and filled twelue baſkets with fragments of the fiue barley
loaues, which remained to them that had eaten. \V Thoſe men therfore when
they had ſeen what a ſigne \Sc{Iesvs} had done, ſaid, That this is the
Prophet indeed that is to come into the world. \V \Sc{Iesvs} therfore
when he knew that they would come to take him, and make him King,
\CNote{\XRef{Mt.~14,~23.}
\XRef{Mr.~6,~46.}}
he fled into the mountaine himſelf alone.

\V And when euen was come, his Diſciples went downe to the ſea. \V And
when they were gone vp into the ship, they came beyond the ſea into
Capharnaum: And now it was darke and \Sc{Iesvs} was not come vnto
them. \V And the ſea aroſe, by reaſon of a great wind that blew. \V When
they had rowed therfore about fiue and twentie or thirtie
%%% 2498
furlongs, they ſee \Sc{Iesvs} walking vpon the ſea, and to draw nigh to
the ship, and they feared. \V But he ſaid to them: It is I, feare not \V
They would therfore haue taken him into the ship: and forthwith the ship
was at the land to which they went.

\V The next day, the multitude that ſtood beyond the ſea, ſaw that there
was no other boat there but one, and that \Sc{Iesvs} had not entred into
the boat with his Diſciples, but that his Diſciples only were
departed. \V But other boats came in from Tiberias beſide the place
where they had eaten the bread, our Lord
\SNote{Theſe wordes doe plainly import, that the giuing thankes was an
  effectual bleſſing of the bread and working the multiplication
  thereof.}
giuing thankes. \V When therfore the multitude ſaw that \Sc{Iesvs} was
not there, nor his Diſciples, they went vp into the boats, & came to
Capharnaum ſeeking \Sc{Iesvs}. \V And when they had found him beyond the
ſea, they ſaid to him: Rabbi, when cameſt thou hither? \V \Sc{Iesvs}
anſwered them, and ſaid: Amen, amen I ſay to you, you ſeeke me not
becauſe you haue ſeene ſignes, but becauſe you did eate of the loaues,
and were filled. \V 
\LNote{Worke not the meate}{By their greedy ſeeking after him for meate
  of the bodie, he taketh occaſion to draw them to the deſire of a more
  excellent food which he had to giue them, and ſo by litle and litle to
  open vnto them the great meate and myſterie of the B.~Sacrament:
  which (as he proueth) doth not only far paſſe their ordinarie bread
  or his maruelous multiplied loaues, but Manna it ſelf, which they
  thought came from Heauen, and ſo much wondered at it.}
Worke not the meate that periſheth, 
%%% o-2334
but that endureth vnto life euerlaſting, which the Sonne
of man wil giue you. For him the Father, God, hath ſigned. \V They ſaid
therfore vnto him: What shal we doe that we may worke the workes of God?
\V \Sc{Iesvs} anſwered, and ſaid to them: This is the worke of God, that
you beleeue in him whom he hath ſent. \V They ſaid therfore to him: What
ſigne therfore doeſt thou, that we may ſee, and may beleeue thee? what
workeſt thou?

\V Our
\CNote{\XRef{Exo.~16,~4.~14.}
\XRef{Pſ.~77,~24.}}
Fathers did eate Manna in the deſert as it is written,
\Emph{Bread from Heauen he gaue them to eate.} \V \Sc{Iesvs} therfore
ſaid to them: 
\SNote{Why we keepe the hebrew word, \Emph{Amen}, and tranſlate it not,
  See the 
\Fix{\XRef{Annot. c. vers.~34.}}{\XRef{Annot. c.~8. vers.~34.}}{obvious typo, fixed in other}}
Amen, amen I ſay to you, Moyſes gaue you not the bread from Heauen, but
my Father giueth you
\LNote{The true bread}{Though
\MNote{Why Chriſt is called bread, & beleeuing, eating.}
the Perſon of Chriſt incarnate, euen out of the Sacrament alſo, be meant
vnder the Metaphores of bread and drinke from Heauen: and our beleefe in
him, be ſignified by eating and feeding: yet the cauſes why they ſhould
be recommended vnto vs in ſuch termes, were, that he was to be eaten and
drunken indeed in the formes of bread & wine: for the which cauſe his
bodie on the croſſe is called
\CNote{\XRef{Ierem.~11,~19.}}
his bread: and his bloud ſhed on the croſſe,
\CNote{\XRef{Gen.~49,~11.}}
the bloud of the grape: no doubt becauſe the ſame bodie and
bloud were in the Holy Sacrament to be eaten and drunken. 
\MNote{What ſignifieth, The true bread.}
In which
ſpeaches, either of Chriſts Perſon generally, or peculiarly of the ſame
as in the B.~Sacrament, \Emph{the true bread} is not taken properly and
ſpecially for that ſubſtance which is of corne, and called with vs
bread; but generally for food or meate: and therfore it hath ioyned with
it lightly a terme ſignifying a more excellent ſort of ſubſtance: as,
the true bread, the bread of Heauen, the bread of life,
Superſubſtantial bread.
\MNote{The B.~Sacrament called bread.}
In which ſort the holy Sacrament which is Chriſts bodie, is both here,
and in
\CNote{\XRef{Lu.~24,~35.}}
S.~Luke and
\CNote{\XRef{Act.~2,~42.}
\XRef{20,~7.}
\XRef{1.~Cor.~10.}}
S.~Paul alſo, often called bread euen after
conſecration: not only for that it was made of bread, but becauſe it is
bread more truly, and by more excellent property and calling, then that
which ordinarily is named bread.}
the true breadfrom Heauen. \V For the bread of God it is that deſcendeth
from Heauen, 
and giueth life to the world. \V They ſaid therfore vnto him: Lord, giue
vs alwaies this bread. \V And \Sc{Iesvs} ſaid to them: I am the bread of
life, he that commeth to me, shal not hunger; and he that beleeueth in
me, shal neuer thirſt. \V But I ſaid to you that both you haue ſeen me and
you beleeue not. \V Al that the Father giueth me, shal come to me; and
him that commeth to me I wil not caſt forth. \V Becauſe I deſcended from
Heauen, not to doe mine owne wil, but the wil of him that ſent me. \V
For this is the wil of him that ſent me, the Father; that al that he
hath giuen me I leeſe not thereof, but raiſe it in the laſt day. \V And
this is the wil of my Father that ſent me; that euery one that ſeeth the
Sonne, and beleeueth in him, haue life euerlaſting, & I wil raiſe him in
the laſt day.

\V The Iewes therfore murmured at him, becauſe he had ſaid, I am the
bread which deſcended from Heauen; \V and they ſaid: Is not this
\Sc{Iesvs} the ſonne of Ioſeph, whoſe father and mother we know? How
then ſaith he, That I deſcended from Heauen? \V \Sc{Iesvs} therfore
anſwered and ſaid to them: Murmure not one to another: \V no man can
come to me, vnles the Father that ſent me,
\LNote{Draw him}{The
\MNote{God draweth vs with our free-wil.}
Father draweth vs and teacheth vs to come to his Sonne, and to beleeue
theſe high and hard myſteries of his incarnation and of feeding vs with
his owne ſubſtance in the Sacrament: not compelling or violently forcing
any againſt their wil or without any reſpect of their conſent, as
Heretikes pretend; but by the ſweet internal motions and perſuations of
his grace and ſpirit he wholy maketh vs of our owne wil and liking to
conſent to the ſame.}
draw him, and I wil raiſe him vp in the laſt day. \V It is written in the
Prophets:
\CNote{\XRef{Eſa.~54,~13.}}
\Emph{And al shal be docible of God.} Euery one that hath
heard of the Father, & hath learned, cõmeth to me. \V Not that any mã
hath ſeen the Father, but he which is of God; this hath ſeẽ the
Father. \V Amẽ, amen I ſay to you: He that beleeueth in me, hath life
euerlaſting. 
%%% 2499
\V I am the bread of life. \V Your fathers did eate
\LNote{Manna and died}{The
\MNote{The manifold preeminences of the 
  \Fix{B.~Sament}{B.~Sacrament}{Obvious typo, fixed in other.}
  aboue Manna.}
Heretikes holding the Fathers of the old Teſtament to haue eaten of the
ſame meate, and to haue had as good Sacraments as we, be here refuted:
Chriſt putting a plaine difference in the very ſubſtance thereof, and in
the graces and effects much more at large.
\CNote{\Cite{Aug. cõt. duas Ep. Pelag. li.~1. c.~19.}
&
\Cite{Ser.~2. de verb. Ap. c.~2.}}
Manna was only a figure of
the B.~Sacrament, though a very excellent figure thereof for many
cauſes. It came in a ſort from heauen, our Sacrament more: it was made
by God miraculouſly, our Sacrament more: it was to be eaten for the time
of their peregrination, our Sacrament more: it was to euery man what he
liked beſt, our Sacrament more: a litle thereof ſerued and ſufficed as
wel as much, our Sacrament more: it was reſerued for ſuch daies as it
could not be gathered, and our Sacrament much more: it was kept for a
memorial in the arke of the Teſtament, our Sacrament much more: the
diſcontented and incredulous murmured and gainſayed it, at our Sacrament
much more: it ſuſtained their bodies in the deſert, our Sacrament, both
bodie and ſoule much more.}
Manna in the deſert; and they died.\V This
%%% o-2335
is the bread that deſcendeth from Heauen: that if any man eate of it, he
die not. \V I am the liuing bread, that came downe from Heauen. If any
man eate of this bread, he ſhal liue for euer: and
\CNote{\XRef{Mt.~26,~26.}
\XRef{Mr.~14,~22.}
\XRef{Lu.~22,~19.}
\XRef{1.~Cor.~11,~24.}}
the bread which I wil giue, is my fleſh for the life of the world.

\V The Iewes therfore ſtroue among themſelues, ſaying:
\LNote{How can this man?}{\Emph{It}
\MNote{In the B.~Sacrament, \Emph{How} is a Iewish word.}
\Emph{came not to their mind that nothing was impoſſible to God, that
wickedly ſaid, How can this man giue vs his flesh? but we may make
great profit of their ſinne, beleeuing the Myſteries, and taking a
leſſon, neuer to ſay or once thinke, How? for it is a Iewish word and
worthy al punishment.} So ſaith,
\Cite{S.~Cyril. li.~4. c.~13. in Io.}
Neuertheles if one asked only for deſire to learne in
humilitie, as our Ladie did touching her hauing a child in her
virginitie, then he muſt take the Angels anſwer to her, That it is of
the Holy Ghoſt. So ſaith
\Cite{S.~Damaſcene li.~4. c.~14.}}
How can this man giue vs his fleſh to eate? \V \Sc{Iesvs} therfore ſaid
  to them: Amen, 
amen I ſay to you,
\LNote{Vnles you eate}{\Emph{Chriſt}
\MNote{The real preſence.}
\Emph{commending the Sacrament of the faithful vnto vs, ſaid, Except you
eate &c you cã not haue life in you So the life ſaith of life: and to
him that thinketh the life to be a lier, this meate shal be death & not
life to him.}
\Cite{Aug. Ser.~2. de verb. Ap. c.~1.}
And
\CNote{\Cite{Ser.~6. de ieiun.~7. menſ.}}
S.~Leo thus: \Emph{Becauſe our Lord ſaith, Except you eate &c let vs ſo
  communicate that we nothing doubt of the truth of Chriſts bodie and
  bloud: for that is receiued with mouth, which is beleeued in hart: and
they anſwer Amen in vaine, that diſpute againſt that which they receiue.}}
Vnles you eate the fleſh of the Sonne of man,
\LNote{And drinke}{This
\MNote{Receiuing in both kindes not neceſſarie.}
the Proteſtants alleage for the neceſsitie of receiuing in both kindes:
but in reſpect of themſelues (who lightly hold al this chapter to
pertaine nothing to the Sacramental receiuing, but to ſpiritual feeding
on Chriſt by faith only) it can make nothing for one kind or other. And
in reſpect of vs Catholikes, who beleeue Chriſts whole Perſon both
humanitie and Diuinitie, both flesh and bloud to be in either forme, and
to be wholy receiued no leſſe in the firſt, then in the ſecond or in
both, this place commandeth nothing for both the kindes.}
and drinke his bloud,
\LNote{You shal not haue life}{Though
\MNote{The Sacramental receiuing of Chriſts bodie, not alwaies
  neceſſarie to ſaluation.}
the Catholikes teach theſe wordes to be ſpoken of the Sacrament, yet
they meane not (no more then our Sauiour here doth) to exclude al from
ſaluation, that receiue not actually and Sacramentally vnder one or both
kindes. For then children that die after they be baptized and neuer
receiued Sacramentally, should perish: which to hold, were heretical.
\CNote{\Cite{Li.~1. de pot. merit. c.~10.}}
Neither did S.~Auguſtine meane, applying theſe words to infants alſo,
that they could not be ſaued without receiuing ſacramentally, as not
only the Heretikes, but Eraſmus did vnlearnedly miſtake him: 
\MNote{The true meaning of S.~Auguſtin's words touching infants
  receiuing of the B.~Sacrament.}
but his
ſenſe is that they were by the right of their Baptiſme ioyned to Chriſts
bodie Myſtical, & thereby ſpiritually partakers of the other Sacramẽt
alſo of Chriſts bodie & bloud. As al Catholike mẽ that be in priſon,
ioyning with the Church of God in hart & deſire to receiue & be
partakers with the Church of this Sacrament, and thoſe ſpecially that
deuoutly heare Maſſe & adore in preſence the bodie & bloud of Chriſt,
ioyning in hart with the Prieſt, al theſe receiue life & fruit of the
Sacramẽt, though at euery time they receiue not ſacramentally in one or
both kinds. And although in the Primitiue Church the Holy Sacrament in
the ſecond kind were often giuen euen to infants to ſanctifie them, yet
\CNote{\Cite{cõc. Tri. Seſ.~21. c.~4.}}
(as the holy Councel hath declared) it was neuer miniſtred vnto them
with opinion that they could not be ſaued without it. And therfore the
Heretikes doe vntruly charge the Church & Fathers with that errour.}
you ſhal not haue life in you. \V He that eateth my flesh, and drinketh
my bloud, hath life 
euerlaſting; and
\LNote{I wil raiſe him}{\Emph{As}
\MNote{The effects of the B.~Sacrament both in our bodie and ſoule.}
\Emph{the Sonne liueth by the Father, euen ſo doe we liue by his flesh},
ſaith
\Cite{S.~Hilarie. li.~8. de Trin.}
And S.~Cyril againe thus:
\CNote{\Cite{Cyril li.~4. c.~14.~15.}}
\Emph{Though by nature of our flesh we be corruptible, yet by
  participation of life we are reformed to the propertie of life. For
  not only our ſoules were to be lifted vp by the Holy Ghoſt to life
  euerlaſting, but this rude groſſe terreſtrial body of ours is to be
  reduced to immortalitie, by touching, taſting, & eating this agreable
  food of Chriſts body. And when Chriſt ſaith: I wil raiſe him vp, he
  meaneth that this body which he eateth, shal raiſe him. Our flesh}
(ſaith Tertullian)
\CNote{\Cite{Tertul. de reſur. car. nu.~7.}}
\Emph{eateth the body and bloud of Chriſt, that the
  ſoule may alſo be fatted. Therfore they shal both haue one reward at
  the Reſurrection.} And S.~Irenæus:
\CNote{\Cite{Li.~4. c.~34.}}
\Emph{How doe they affirme that our
bodies be not capable of life euerlaſting, which is nourished by the
body and bloud of our Lord? Either let them change their opinion, or els
ceaſe to offer the Euchariſt.} S.~Gregorie Nyſſene alſo ſaith:
\CNote{\Cite{Nyſſ. in orat. catech. magna.}}
\Emph{That liuely bodie entring into our bodie, changeth it and maketh
  it like and immortal.}}
I wil raiſe him vp in the laſt day. \V For my flesh, is
\LNote{Meate indeed}{Manna,
\MNote{The B.~Sacramẽt is the true Manna & water of the rock.}
was not the true meat: nor the water of the rocke, the drinke indeed:
for they did but driue away death or famine for a time and for this
life. \Emph{But the holy Bodie of Chriſt is the true food nourishing to life
  euerlaſting, and his bloud the true drinke that driueth death away
  vtterly, for they be not the bodie and bloud of a mere man, but of him
that being ioyned to life is made life and therfore are we the bodie and
members of Chriſt, becauſe by this benediction of the myſterie we
receiue the Sonne of God himſelf.} So ſaith
\Cite{S.~Cyril. li.~4. c.~16. in Io.}}
meate indeed: and my bloud is drinke indeed. \V He that eateth my flesh, and drinketh
my bloud, abideth in me, and I in him. \V As the liuing Father hath ſent
me, and I liue by the Father: and he that eateth me, the ſame alſo shal
liue by me. \V This is the bread that came downe from Heauen. Not as
your Fathers did eate Manna, and died. 
\LNote{He that eateth this bread}{By 
\MNote{The whole grace & effect therof in one kind; and therfore the
  people not defrauded.}
this place the
\CNote{\Cite{Conc. Trid. Seſ.~21. c.~1.}}
holy Councel proueth that for the grace & effect of the
Sacrament, which is the life of the ſoule there is no difference whether
a man receiue both kinds or one. Becauſe our Sauiour who before
attributed life to the eating & drinking of his bodie & bloud doth here
alſo affirme the ſame effect, which is life euerlaſting, to come of
eating only vnder one forme. Therfore the Heretikes be ſeditious
calumniatours that would make the people beleeue, the Catholike Church
and Prieſts to haue defrauded them of the grace & benefit of one of the
kinds in the Sacrament. Nay, it is they that haue defrauded the world,
by taking away both the real ſubſtance of Chriſt, and the grace from one
kind and both kinds, and from al other Sacraments. 
\MNote{Receiuing in one or both kinds, indifferent, according to the
  holy Churches appointment.}
The Church doth only (by the wiſedom of God's Spirit and by inſtruction
of Chriſt & his Apoſtles, according to time and place, for God's moſt
honour, the reuerẽce of the Sacrament, & the peoples moſt profit therby)
diſpoſe of the manner & order how the Prieſt, how the people shal
receiue, & al other Particular points,
\CNote{\Cite{Ep.~118. c.~6. ad Ianuarium.}}
\Emph{which himſelf} (ſaith
S.~Auguſtine) \Emph{did not take order for, that he might commit that to
the Apoſtles, by whom he was to diſpoſe his Churches affaires.} 
\MNote{Authoritie of Scriptures and the Primitiue Church for receiuing
  in one kind.}
Though both he and the Apoſtles and the Fathers of the primitiue Church
left vs example of receiuing vnder one kind. Chriſt
\CNote{\XRef{Lu.~24,~15.}}
\Emph{at Emmaus}, The Apoſtles
\XRef{Act.~2,~42.}
The primitiue Church in giuing the bloud only to children.
\Cite{Cypr. li.\de lapſis, nu.~10.}
In reſeruing moſt commonly the bodie only,
\Cite{Tertul li.~2. ad vxo. nu.~4.}
\Cite{Cypr. li. de lapſis. nu.~10.}
In houſeling the ſicke therwith,
\Cite{Euſeb. Ec. hiſt. li.~6. c.~36.}
In the holy Eremits alſo that receiued and reſerued it commonly
& not the bloud, in the wildernes,
\Cite{Baſil, ep. ad Cæſariam Patritiam},
and in diuers other caſes which were too long to rehearſe. 

Whereby
\MNote{The cauſes of the Churches practice & ordinance concerning one
  kind.}
the Church being warranted and in the ruling of ſuch things fully taught
by God's Spirit, as wel for the reprouing of certaine heretikes, that
Chriſt God and man was not whole and al in euery part of the Sacrament,
as ſpecially for that the Chriſtiã people being now enlarged, and the
communicants often ſo many at once, that neither ſo much wine could be
conueniently conſecrated, nor without manifold accidents of sheding or
abuſing be receiued (wherof the Proteſtants haue no regard, becauſe it
is but common wine which they occupie, but the Church knowing it to be
Chriſts owne bloud, muſt haue al dreadful regard) therfore I ſay she
hath decreed and for ſome hundreth yeares put in vſe that the 
\MNote{The Prieſts that ſay Maſſe, muſt receiue both kinds.}
Prieſt
ſaying Maſſe, should alwaies both conſecrate and alſo receiue both
kinds, becauſe he muſt expreſſe liuely the Paſsion of Chriſt, and the
ſeparation of his bloud from his bodie in the ſame, & for to imitate the
whole action & inſtitution as wel in ſacrificing as receiuing, as to
whom properly it was ſaid:
\CNote{\XRef{Lu.~22,~19.}
\XRef{1.~Cor.~11,~24.}}
\Emph{Doe this}; for that was ſpoken only to
ſuch as haue power therby to offer and cõſecrate: But the Lay men, & the
Clergie alſo when they doe not execute or ſay Maſſe themſelues should
receiue in one kind, being therby no leſſe partakers of Chriſts whole
Perſon and grace, then if they receiued both. For (as S.~Paul ſaith)
\CNote{\XRef{1.~Cor.~10,~18.}}
\Emph{He that eateth the hoſtes, is partaker of the Altar.} He that
eateth, ſaith he: for though there were drinke-offerings or libaments
ioyned lightly to euery Sacrifice, yet it was enough to eate only of one
kind, for to be partaker of the whole.}
He that eateth this bread, shal liue for euer. \V Theſe things he ſaid teaching in the Synagogue,
in Capharnaum.

\V Many therfore of his Diſciples hearing it, ſaid: This ſaying is hard,
and who can heare it? \V But \Sc{Iesvs} knowing with himſelf that his
Diſciples murmured at this, he ſaid to them: Doth this ſcandalize you?
\V 
\LNote{If you shal ſee}{Our
\MNote{Chriſt inſinuateth that faithles men shal not beleeue his
  preſence in the B.~Sacrament, becauſe he is aſcended.}
Sauiour ſeemeth to inſinuate, that ſuch as beleeue not his words
touching the Holy Sacrament, and thinke it impoſsible for him to giue
his Body to be eaten in ſo many places at once, being yet in earth,
should be much more ſcandalized & tẽpted after they ſaw or knew him to be
aſcended into Heauen. Which is proued true in the Capharnaites of this
time. Whoſe principal reaſon againſt Chriſts preſence in the Sacrament
is, that he is aſcended into Heauen: yea, who are ſo bold as to expound
this ſame ſentence for themſelues thus, It is not this body or flesh
which I wil giue you, for that I wil carie with me to Heauen. Whereby if
they meant only that the condition and qualities of his body in Heauen
should be other then in the Sacrament, it were tolerable: for
S.~Auguſtin ſpeaketh ſometime in that ſenſe. But to deny the ſubſtance
of the body to be the ſame, that is wicked.}
If then you shal ſee
\CNote{\XRef{Io.~3,~13.}}
the Sonne of man aſcend where he was before? \V It is the ſpirit that
quickeneth, 
\LNote{The flesh profiteth nothing}{If this ſpeach were ſpoken in the
  ſenſe of the Sacramentaries, it would take away Chriſts Incarnation,
  manhood, & death, no leſſe then his corporal preſence in the
  Sacrament. For if his flesh were not profitable, al theſe things
\Fix{ſ}{}{obvious typo, fixed in other}% to prevent paragraph break
were vaine. 
Therfore \Sc{Christ} denieth not his owne flesh to be
profitable, but that their groſſe and carnal conceiuing of his words, of his
flesh, & of the manner of eating the ſame, was vnprofitable: which is
plaine by the ſentence following where he warneth them, that his words
be ſpirit and life, of high Myſtical meaning, and not vulgarly & groſly
to be taken, as they tooke them. And it is the vſe of the Scripture to
cal mans natural ſenſe, reaſon, and carnal reſiſting or not reaching
ſupernatural truths, flesh or bloud, as, \Emph{Flesh and bloud reuealed
  not this to thee &c.}
\XRef{Mat.~16.}

This 
\MNote{The Capharnaites groſſe vnderſtanding of Chriſts flesh to be
  giuen or eaten. And, how his flesh doth profit, & not profit.}
carnalitie then of theirs, ſtood in two points ſpecially: firſt, that
they imagined that he would kil himſelf, & cut & mangle his flesh into
parts, & ſo giue it them raw or roſt to be eaten among them. Which could
not be meant, ſaith
\CNote{\Cite{Auguſt. Doct. Chr. li.~3. c.~13.}}
S.~Auguſtin: for that had conteined an heinous and
barbarous fact; and therfore they might & should haue been aſſured, that
he would command no ſuch thing: but ſome other ſweet ſenſe to be of his
hard, myſtical, or figuratiue words, & to be fulfilled in a Sacrament,
myſterie, and a maruelous diuine ſort, otherwiſe then they could
comprehend. 
\MNote{Chriſts flesh giueth life becauſe it is the flesh of God &
  man.}
Secondly, they did erre touching his flesh, in that they
tooke it to be flesh of a mere man, & of a dead man alſo, when it should
come to be eaten: of which kind of flesh Chriſt 
\Fix{her}{here}{obvious typo, fixed in other.}
pronounceth, that it profiteth nothing. Whereupon S.~Cyril ſaith:
\CNote{\Cite{Li.~4. c.~25. in Io.}}
\Emph{This body is not of Peter or Paul or any other like, but of Chriſt
\Sc{Iesvs} who is the life itſelf: and therfore this Body giueth life,
the very fulnes of the Diuinitie dwelling in it.} And the holy
\Cite{Councel of Epheſus in the 11. Anathematiſme}
expounded alſo by
the ſaid S.~Cyril: \Emph{The Euchariſt is not the body of any common
  perſon (for the flesh of a common man could not quicken) but of the
  \Sc{word} itſelf. But the Heretike Neſtorius diſſolueth the vertue of
  this myſterie, holding mans flesh only to be in the Euchariſt.} Thus
there. And
\CNote{\Cite{Ignatius apud Theodor. dial.~3.}}
S.~Ignatius cited of Theodorete, and many other Fathers haue
the like. Whereby we may ſee that it commeth of the Diuinitie & Spirit
(without which Chriſts flesh can not be) that this Sacrament giueth
life.}
the flesh profiteth nothing. The wordes that I haue ſpoken to you, be ſpirit and life. \V But there
be certaine of you
\LNote{That beleeue not}{It
\MNote{Iudas the chiefe of them that beleeue not the real preſence.}
is lacke of faith, you ſee here, that cauſeth men to ſpurne againſt this
high truth of the Sacrament: as alſo it may be learned here, that it is
the great & merciful guift of God that Catholike men doe againſt their
ſenſes & carnal reaſons, beleeue & ſubmit themſelues to the humble
acknowledging of this Myſterie: laſtly, that it may wel
\CNote{\XRef{verſ.~64.}}
by Chriſts
inſinuation of Iudas, be gathered, that he ſpecially ſpurned againſt our
Maiſters ſpeaches of the holy Sacrament.}
that beleeue not. For \Sc{Iesvs} knew from the beginning who they were that did not
beleeue, and who he was that would betray him. \V And he ſaid: Therfore
did I ſay to you, that no man can come to me, vnles it be giuen him of
my Father. \V After this many of his Diſciples
\LNote{Went back}{It
\MNote{Heretikes beleeue not the real preſence, becauſe they ſee bread &
wine: as the Iewes beleeued not his Godhead becauſe of the shape of a
poore man.}
can be no maruel to vs now that ſo many reuolt from the Church, by
offenſe or ſcandal vniuſtly taken at Chriſts body and bloud in the
Sacrament: ſeeing many of his Diſciples that ſaw his wonderful life,
doctrine, and miracles, forſooke Chriſt himſelf, vpon the ſpeach &
promiſe of the ſame Sacrament. For the myſterie of it is ſo ſupernatural
& diuine in itſelf, and withal ſo low and baſe for our ſakes, by the
shew of the formes of theſe terrene elements vnder which it is, and we
eate it; that the vnfaithful and infirme doe ſo ſtumble at Chriſt in the
Sacrament, as the Iewes & Gentils did at Chriſt in his humanitie. For,
the cauſes of contradictions of the incarnation & Tranſſubſtantiation be
like. And it may be verily deemed, that whoſoeuer now can not beleeue
the Sacrament to be Chriſt, becauſe it is vnder the formes of bread and
wine, and is eaten and drunken, would not then haue beleeued that Chriſt
had been God, becauſe he was in the shape of man, and crucified. 
\MNote{The diſciples reuolting at Chriſts words, proue that he ſpake not
metaphorically, as at other times.}
To
conclude, it was not a figure nor a myſterie of bare bread and wine, nor
any Metaphorical or Allegorical ſpeach, that could make ſuch a troup of
his Diſciples reuolt at once. When he ſaid he was a doore, a vine, a
way, a Paſtour, and ſuch like (vnto which kind of ſpeaches the
Proteſtants ridiculouſly reſemble the words of the holy Sacrament) who
was ſo mad to miſtake him, or to forſake him for the ſame? For the
Apoſtles at the leaſt would haue plucked them by the ſleeues, and ſaid:
Goe not away my Maſters, he ſpeaketh parables. The cauſe therfore
was their incredulitie, and the height of the Myſterie, for that they
neither knew the meanes how it might be preſent, nor would beleeue that
he was able to giue his flesh to be eaten in many places. And euen ſuch
is the vnbeleefe of the Heretikes about this matter at this day.}
went backe: and now they walked not with him.

\V \Sc{Iesvs} therfore ſaid to the Twelue: What, wil you alſo depart? \V
Simon 
\LNote{Peter anſwered}{Peter
\MNote{As S.~Peter beareth the perſon of al beleeuing Catholikes: ſo
  Iudas of al vnbeleeuing Heretikes. He being the firſt Arch-heretike;
  and this, againſt the B.~Sacrament, the firſt hereſie.}
anſwereth for the Twelue, not knowing that Iudas in hart was already
naught, and beleeued not Chriſts former words touching the B.~Sacrament,
but was to reuolt afterward as the other.
\CNote{\Cite{Cypr. ep.~55. nu.~3.}}
Wherein as Peter beareth the
perſon of the Church & al Catholike men, that for no difficulty of his
word, nor for any reuolt (be it neuer ſo general) of Schiſmatikes,
Heretikes, or Apoſtates, either for this Sacrament or any other Article,
wil euer forſake Chriſt: So Iudas was the chiefeſt ſuborner, maintayner,
& father of this hereſie againſt the real preſence of Chriſts bodie and
bloud in the B.~Sacrament, and of the reuolt from him for the ſame: as
S.~Auguſtin teacheth
\Cite{in enarratione Pſal.~34. ad ver.~22.}
&
\Cite{Pſal.~35. ad ver.~7.}
declaring withal that this was the
firſt hereſie againſt Chriſts doctrine, and worthily commending S.~Peter
for his humble obedience, in receiuing Chriſts ſpeach, and firmly
beleeuing his words to be true and good, which he did not yet vnderſtand. By
whoſe example therfore when company draweth vs to reuolt, let vs ſay
thus: Lord, whither or to whom shal we goe, when we haue forſaken thee?
to Caluin, Luther, or ſuch, and forſake thee and thy Church with the
vnfaithful multitude? No, thou haſt the words of life, and we beleeue
thee, and thy Church wil not nor can not beguile vs.
\CNote{\Cite{Tract.~27. in Euang. Io.}}
\Emph{Thou haſt}
(ſaith the ſame S.~Auguſtine) \Emph{life euerlaſting in the miniſtration
of thy body and bloud.} And a litle after: \Emph{Thou art life
  euerlaſting itſelf, and thou giueſt not in thy flesh and bloud but
  that which thy ſelf art.}}
Peter therfore anſwered him: Lord, to whom shal we goe? thou haſt the wordes of eternal life. \V
And we beleeue and haue knowen that thou art Chriſt the Sonne of God. \V
\Sc{Iesvs} anſwered them: Haue not I choſen you the Twelue; and of you
one is a Diuel? \V And he meant Iudas Iſcariot, Simons ſonne: for this
ſame was to betray him, whereas he was one of the Twelue.


\stopChapter


\stopcomponent


%%% Local Variables:
%%% mode: TeX
%%% eval: (long-s-mode)
%%% eval: (set-input-method "TeX")
%%% fill-column: 72
%%% eval: (auto-fill-mode)
%%% coding: utf-8-unix
%%% End:
