%%%%%%%%%%%%%%%%%%%%%%%%%%%%%%%%%%%%%%%%%%%%%%%%%%%%%%%%%%%%%%%%%
%%%%
%%%% The (original) Douay Rheims Bible 
%%%%
%%%% New Testament
%%%% John
%%%% Chapter 02
%%%%
%%%%%%%%%%%%%%%%%%%%%%%%%%%%%%%%%%%%%%%%%%%%%%%%%%%%%%%%%%%%%%%%%




\startcomponent chapter-02


\project douay-rheims


%%% 2486
%%% o-2320
\startChapter[
  title={Chapter 2}
  ]

\Summary{At the requeſt of his mother he worketh his firſt miracle
  turning water into wine at a mariage in Galilee, although the time of
  his manifeſtation be not yet come. 12.~Then in Hieruſalem at Paſche,
  being but one, and yet obſcure, he throweth out of the Temple moſt
  miraculouſly al the marchantes.  28.~And being yet of the blind Iewes
  aſked a ſigne, he ſignifieth ſo long before, that they should kill him,
  but he wil riſe againe the third day.  23.~Which alſo preſently they
  would doe, but that he knowing their falſe hartes (though many beleeue
  in him) wil not tarie among them.}

And the third day there was a mariage made in Cana of Galilee: and the
mother of \Sc{Iesvs} was there.  \V And 
\LNote{\Sc{Iesvs} alſo was called}{By
\MNote{Chriſt with his preſence honoureth and approueth Mariage.}
his vouchſafing to come with his to the Mariage, he approueth the
cuſtome of the faithful in meeting at honeſt feaſtes and recreations for
maintenance of loue, peace, and amitie among them ſelues: he reproueth
the hereſie of Tatian, Marcion, and ſuch like condemning wedlocke:
laſtly
\CNote{\Cite{Cyril. in 2.~Io. c.~22.}}
(as S.~Cyril ſaith) he ſanctifieth and bleſſeth the mariage of
the Faithful in the new Teſtament, making it a new creature in him, and
diſcharging it of the manifold maledictions and diſorders wherein it was
before.  By which benediction the often diuorces, remariages, and
pluralities of wiues, and the womens ſeruile ſubiection and imparitie in
that caſe, be redreſſed and reduced to the primitiue inſtitution, and ſo
Chriſtian mariage made a Sacrament.  See
\Cite{S.~Aug. de nupt. & concup. li.~1. c.~10. & 21.~li.~3 de adult. coniug. c.~8.}}
\Sc{Iesvs} alſo was called,and
his Diſciples to the mariage.  \V And the wine failing, the mother
of \Sc{Iesvs} ſaith to him: 
\LNote{They haue no wine}{Our
\MNote{Our Ladies interceſsion.}
Lady many waies vnderſtood that now the time approched of manifeſting him
ſelf to the world by miracles and preaching, and nothing doubted but
that he would now begin at her requeſt.  Whereby we learne that Chriſt
ordinarily giueth not his graces, but humbly asked and requeſted
thereunto; and that his mothers interceſsion is more then vulgarly
effectual, and that he denieth her nothing.}
They haue no wine.  \V And \Sc{Iesvs} ſaith
to her: 
\LNote{What is to me and thee?}{Becauſe
\MNote{Tranſlatours of holy Scriptures.}
this ſpeach is ſubiect to diuers ſenſes, we keepe the wordes of our text,
leſt by turning it into any English phraſe, we might ſtraiten the Holy
Ghoſts intention to ſome certaine ſenſe either not intended, or not only
intended, and ſo take away the choiſe and indifferencie from the Reader,
whereof (in holie Scripture ſpecially) al Tranſlatours muſt beware.
Chriſt then may meane here, what is that, woman, to me & thee being but
ſtrangers, that they want wine? as ſome interpret it.  Or (which is the
more proper vſe of that kind of ſpeach in holy writ) what haue I to doe
with thee? that is, why should I haue reſpect to thy deſire in this
caſe? in matters touching my charge & the commiſsion of my Father for
preaching, working miracles, and other graces, I muſt not be tied to
flesh and bloud.  Which was not a reprehenſion of our Lady, or
ſignification that he would not heare her in this or other things
pertaining to Gods glorie or the good of men, for the euent sheweth the
contrarie: But it was a leſſon to the companie that heard it, and namely
to his Diſciples, that reſpect of kinred should not draw them to doe
any thing againſt reaſon, or be the principal motion why they doe
their dueties, but Gods glorie.}
What is to me and thee woman? my houre commeth not yet. \V His
mother ſaith to the miniſters: 
\LNote{Whatſoeuer he shal ſay}{By
\MNote{Our Lady doubteth not but Chriſt wil grant her petition.}
this you ſee, our Lady by her diuine prudence and entire familiaritie
and acquaintance with al his manner of ſpeaches, knew it was no checke
to her, but a doctrine to others: & that ſhe had no repulſe, though he
ſeemed to ſay his time was not yet come to worke miracles: not doubting
but he would begin a litle before his ordinary time for her ſake,
as
\CNote{\Cite{li.~2. in Io. c.~23.}}
S.~Cyril thinketh he did: and therfore ſhe admonisheth the waiters to
marke wel, & to execute whatſoeuer Chriſt ſhould bid them doe.}
Whatſoeuer he ſhal ſay to you, doe ye.
\V And there were ſet there ſix water-pots of ſtone, according to the
purification of the Iewes, holding euery one two or three meaſures.  \V
\Sc{Iesvs} ſaith to them: Fil the water-pots with water.  And they
filled them vp to the top. \V And \Sc{Iesvs} ſaith to them: Draw now,
and carie to the cheefe ſteward.  And they caried it.  \V And after the
cheefe ſteward taſted the water made wine,
\SNote{He that ſeeth water turned into wine, needeth not diſpute or
  doubt how Chriſt changed bread into his body.}
and knew not whence it was,
but the miniſters knew that had drawne the water; the cheefe ſteward
calleth the bridegrome, \V and ſaith to him: Euery man firſt ſetteth the
good wine, and when they haue wel drunke, then that which is worſe.  But
thou haſt kept the good wine vntil now.  \V This beginning of miracles
did \Sc{Iesvs} in Cana of Galilee: and he manifeſted his glorie, and his
Diſciples beleeued in him.

\V After this he went downe to Capharnaum himſelf and his mother, and his
brethren, and his Diſciples; and there they remained not many daies. \V
And the Paſche of the Iewes was at hand, and \Sc{Iesvs} went vp to
Hieruſalem: \V and he found in the Temple them that ſold oxen and ſheep
and doues, and the bankers ſitting.  \V And when he had made as it were
a whip of litle cordes, he 
\LNote{Caſt them out}{By
\MNote{Prophaners of Gods Church are to be punished in ſoul & body
by the Spiritual power.}
this chaſtiſing corporally the defilers & abuſers of the Temple, he
doth not only ſhew his power, that being but one poore man he could by
force execute his pleaſure vpon ſo many ſturdy fellowes: but alſo his
ſoueraigne authoritie ouer al offenders; and that not vpon their ſoules
only, as by excommunication and ſpiritual penalties, but ſo farre as is
requiſite for the execution of ſpiritual iuriſdiction, vpon their bodies
and goods alſo.  That the Spiritualtie may learne, how farre and in what
caſes, for iuſt zeale of Chriſts Church, they may vſe and exerciſe both
ſpiritually and temporally their forces and faculties againſt offenders,
ſpecially againſt the prophaners of Gods Church, according to the
Apoſtles alluſion
\XRef{1.~Cor.~3.}
\Emph{If any defile the Temple of
  God him wil God deſtroy.}}
caſt them al out of the Temple, the ſheep
alſo and the oxen, and the money of the bankers he powred out, and the
tables he ouerthrew.  \V And to them that ſold doues, he ſaid: Take away
%%% o-2321
theſe things hence, and make not the houſe of my Father, a houſe of
marchandiſe.  \V And his Diſciples remembred that it is written:
\CNote{\XRef{Pſ.~68,~10.}}
\Emph{The zeale of thy houſe hath eaten me.} \V The Iewes therfore
anſwered and ſaid to him: What ſigne doeſt thou ſhew vs, that thou doeſt
theſe things  \V \Sc{Iesvs} anſwered and ſaid to them:
\CNote{\XRef{Mt.~26,~61.}
\XRef{27,~40.}}
Diſſolue this
temple, and in three daies I wil raiſe it. \V The Iewes therfore ſaid:
In fourtie and ſix yeares was this Tẽple built, & wilt thou raiſe it in
three daies? \V But he ſpake of the tẽple of his body
%%% 2487
\V Therfore when he was riſen againe from the dead, his Diſciples
remembred, that he ſaid this, and they beleeued the ſcripture and the
word that \Sc{Iesvs} did ſay.  \V And when he was at Hieruſalem in the
Paſche, vpon the feſtiual day, many beleeued in his name, ſeeing his
ſignes which he did.  \V But 
\LNote{\Sc{Iesvs} committed not himſelf}{S.~Auguſtine
\CNote{\Cite{Tract. in Io.~11.}}
\MNote{The B.~Sacrament is not to be giuen to nouices or yonglings
  in faith.}
applieth this their firſt faith and beleefe in Chriſt, ſodenly raiſed
vpon the admiration of his wonders, but yet not fully formed or
eſtablished in them, vnto the faith of Nouices or Catechumens in the
Church & Chriſt not committing his Perſon to them as yet, to the
Churches like warineſſe and wiſedom, in not opening nor giuing to them
our Lord in the B.~Sacrament, becauſe al were not to be truſted with
that high point without ful trial of their faith.}
\Sc{Iesvs} did not commit himſelf vnto
them, for that he knew al, \V and becauſe it was not needful for him
that any ſhould giue teſtimonie of man; for he knew what was in man.

\stopChapter


\stopcomponent


%%% Local Variables:
%%% mode: TeX
%%% eval: (long-s-mode)
%%% eval: (set-input-method "TeX")
%%% fill-column: 72
%%% eval: (auto-fill-mode)
%%% coding: utf-8-unix
%%% End:
