%%%%%%%%%%%%%%%%%%%%%%%%%%%%%%%%%%%%%%%%%%%%%%%%%%%%%%%%%%%%%%%%%
%%%%
%%%% The (original) Douay Rheims Bible 
%%%%
%%%% New Testament
%%%% John
%%%% Argument
%%%%
%%%%%%%%%%%%%%%%%%%%%%%%%%%%%%%%%%%%%%%%%%%%%%%%%%%%%%%%%%%%%%%%%




\startcomponent argument


\project douay-rheims


%%% 2480
%%% o-2314
\startArgument[
  title={\Sc{the argvment of s.~iohns ghospel.}},
  marking={Argument of S.~John's Gospel}
  ]

S.~Iohns Ghoſpel may be deuided into foure partes.

The firſt part is of the Actes of Chriſt before his ſolemne
manifeſtation of himſelfe, while Iohn Baptiſt was yet baptizing:
\XRef{Chap.~1.~2.~3.~4.}

The ſecond, of his Actes in Iurie (hauing now begunne his ſolemne
manifeſtation in Galilee, 
\XRef{Mat.~4,~12.})
the ſecond Eaſter or Paſche of his preaching:
\XRef{Chap.~5.}
For of the firſt Paſche, we had in the firſt part.
\XRef{chap.~2.~13}:
\Emph{And the Paſche
\footnote[j-arg-foot]{This ſpeach very cõmon in this Ghoſpel, as
  appeareth by the places here marked, declareth that he writeth to the
  Gentils.}
of the Iewes was at hãd.}  And that feaſt whereof we haue in this ſecond
part, 
\XRef{chap.~5,~1}:
\Emph{After this there was a feſtiual day
\note[j-arg-foot]
of the Iewes}, is thought of
\CNote{\Cite{Iren. li.~2. c.~39.}}
good Authors, to be the feaſt of Paſche.

The third part is of his Actes in Galilee, and in Iurie, about the third
Paſche, and after it:
\XRef{chap.~6, to the 12.}
For ſo we haue 
\XRef{chap.~6,~4}:
\Emph{And Paſche the feſtiual day
\note[j-arg-foot]
of the Iewes was at hand.}

The fourth part is of the fourth Paſche (which we haue in the end of
the 
\XRef{chap.~11,~55}:
\Emph{And the Paſche
\note[j-arg-foot]
of the Iewes was at hand}) that is to ſay, of the Holy weeke of his
Paſſion in Hieruſalem:
\XRef{chap.~12. vnto the end of the booke}.

By which diuiſion it is manifeſt, that the intent of this Euangeliſt
writing after the other three, was, to omit the Actes of Chriſt in
Galilee, becauſe the other three had written them at large: and to
report his Actes done in Iurie, which they had omitted.

And this he doth, becauſe Iurie with Hieruſalem and the Temple, being
the principal part of the Country, there abode the principal of the
Iewes, both for authoritie, and alſo for learning in the law or
knowledge of the Scriptures, and therfore that was the place, where our
Lord \Sc{Iesvs} finding in the Head it ſelfe and in the leaders of the
reſt, ſuch wilful obſtinacie and deſperate reſiſtance, as the Prophets
had foretold, did by this occaſion, much more plainely then in Galilee,
both  ſay and proue, at ſundry times, euen euery yeare of his preaching,
himſelfe to be the \Sc{Christ} that had beene ſo long promiſed vnto
them, and expected of them: and the ſame \Sc{Christ} to be not only a
man, as they imagined, but alſo the natural, conſubſtantial, and
coeternal Sonne of God the Father, who now had ſent him.  Therfore theſe
were the wordes and deedes that ſerued beſt the purpoſe of this
Euangeliſt, being to shew the glorie and excellencie of this Perſon
\Sc{Iesvs}: that thereby the Gentils might ſee how worthily Hieruſaleme
and the Iewes were reprobated who had refuſed yea & crucified ſuch an
one: and how wel & to their owne ſaluation themſelues might doe, to
receiue him and to beleeue in him.  For this to haue beene his purpoſe,
himſelfe declareth in the end, ſaying:
\CNote{\XRef{Io.~20,~31.}}
\Emph{Theſe are written, that you may beleeue that \Sc{Iesvs}
is \Sc{Christ} the Sonne of God: and that beleeuing, you may haue life
in his name.} 

And herevpon it is, that
\CNote{\Cite{Hier. in Catal.}}
S.~Hierom writeth thus in his life: \Emph{Iohn the Apoſtle
\CNote{\XRef{Io.~21,~20.}}
whom \Sc{Iesvs} loued very much, the
\CNote{\XRef{Mat.~4,~21.}}
ſonne of Zebedee, the brother of Iames the Apoſtle
\CNote{\XRef{Act.~12,~2.}}
whom Herod after our Lords 
%%% o-2315
Paſſion beheaded, laſt of al wrote the Ghoſpel, at the requeſt of the
Biſhops of Aſia, againſt Cerinthus, and other Heretikes, and ſpecially
againſt the
%%% 2481
aſſertion of the Ebionites then riſing, who ſay that Chriſt was not
before \Sc{Marie}.  Whereupon alſo he was compelled to vtter his Diuine
Natiuitie.}

Of his three Epiſtles, and of his Apocalypſe, shal be ſaid in their owne
places. 

It followeth in S.~Hierome, that \Emph{In the ſecond perſecution vnder
  Domitian, fourteene yeares after the perſecution of Nero he was exiled
  into the ile Patmos.  But after that Domitian was ſlaine, and his
  actes for his paſsing crueltie repealed by the Senate; vnder Nerua the
  Emperour he returned to Epheſus, and there continuing vnto the time of
  Traiane the Emperour, he founded and gouerned al the Churches of Aſia:
  and worne with old age, he died the threeſcore and eight yeare after
  the Paſsion of our Lord, and was buried beſides the ſame citie.}

\ArgHeading{Whoſe excellencie the ſame holy Doctour thus briefly
  deſcribeth.
\Cite{li.~1. Aduers. Iouinianum.}}

Iohn the Apoſtle, one of our Lords Diſciples, who was the yongeſt among
the Apoſtles, and whom the faith of Chriſt found a virgin, remained a
virgin, and therfore is
\CNote{\XRef{Io.~13,~23.~24.}}
more loued of our Lord, and lieth vpon the
breaſt of \Sc{Iesvs}: and that which Peter durſt not aske,
\CNote{\XRef{Io.~21,~20.}}
he deſireth
him to aske.  And after the reſurrection, when Marie Magdalen had
reported that our Lord was riſen againe, both of them ranne to the
Sepulchre,
\CNote{\XRef{Io.~20,~4.}}
but he came thither firſt: and when they were in the ship and
fished in the lake of Geneſareth, \Sc{Iesvs} ſtood on the shore, neither
did the Apoſtles know who they ſaw:
\CNote{\XRef{Io.~21,~7.}}
onely the virgin, knoweth the virgin
& ſaith to Peter: \Emph{It is our Lord.}  This Iohn was both an
Apoſtle, & Euãgeliſt, and Prophet.  An Apoſtle, becauſe he wrote to the
Churches as a Maiſter: an Euangeliſt, becauſe he compiled a booke of the
Ghoſpel, which (except Matthew) none other of the twelue Apoſtles did: a
Prophet, for he ſaw in the ile Patmos, where he was banniſhed by
Domitian the Emperour for the teſtimonie of our Lord, the Apocalipſe,
conteining infinite myſteries of things to come.  Tertullian alſo
reporteth, that at Rome being caſt into a barrel of hote boiling oile,
he came forth more pure and freſher or liuelier, then he went in.  Yea
and his Ghoſpel it ſelf much differeth from the reſt.  Matthew beginneth
to write as of a man: Marke of the prophecie of Malachie and Eſay.  Luke
of the Prieſt-hood of Zacharie: The firſt hath the face of a man,
becauſe of the genealogie: the ſecond the face of a lion, for the voice
of one crying in the deſert: the third the face of a calfe, becauſe of
the Prieſt-hood.  But Iohn as an Eagle flieth to the things on high, and
mounteth to the Father him ſelf, ſaying: \Emph{In the beginning was the
  \Sc{Word}, and the \Sc{Word} was with God, and God was the
  \Fix{\Sc{word}.}{\Sc{Word}.}{Likely typo, fixed in other}}
Thus farre S.~Hierome.

Vpon this Ghoſpel there are the famous commentaries of S.~Auguſtine
called
\Cite{Tractatus in Euang. Ioan. to.~9.}
and twelue bookes of S.~Cyrils commentaries.

\stopArgument


\stopcomponent


%%% Local Variables:
%%% mode: TeX
%%% eval: (long-s-mode)
%%% eval: (set-input-method "TeX")
%%% fill-column: 72
%%% eval: (auto-fill-mode)
%%% coding: utf-8-unix
%%% End:
